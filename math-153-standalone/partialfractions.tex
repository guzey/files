\documentclass[10pt]{amsart}
\usepackage{fullpage,hyperref,vipul}
\title{Partial fractions: an integrationist perspective}
\author{Math 153, Section 55 (Vipul Naik)}

\begin{document}
\maketitle

{\bf Corresponding material in the book}: Section 8.5.

{\bf What students should already know}: The integrals for $1/x$,
$1/(x^2 + 1)$, and $f'(x)/f(x)$, and the key trigonometric
substitution of $\arctan$.

{\bf What students should definitely get}: The statement about the
existence of a partial fraction decomposition. How such a
decomposition can be obtained, and how each of the pieces can be integrated.

\section*{Executive summary}

Words ...

\begin{enumerate}
\item For most practical purposes, we can study {\em monic}
  polynomials instead of arbitrary polynomials. A monic polynomial is
  a polynomial whose leading coefficient is $1$. The reason we can
  restrict attention to monic polynomials is that any nonzero
  polynomial can be expressed as a nonzero constant times a monic
  polynomial.
\item A nonconstant monic quadratic polynomial is irreducible (i.e.,
  cannot be expressed as a product of polynomials of smaller degree)
  if and only if it has negative discriminant.
\item Every nonconstant monic polynomial with real coefficients is a
  product of monic linear polynomials and irreducible monic
  quadratics, and this factorization is unique. Thus, all irreducible
  monic polynomials are either linear or quadratics with negative
  discriminant.
\item The partial fractions approach breaks up any rational function
  as the sum of a polynomial and rational functions of the form
  $R/Q^k$ where $Q$ is a monic irreducible factor of the original
  denominator and $R$ is a polynomial of degree strictly less than the
  degree of $Q$.
\item Each of these partial fraction pieces is easy to integrate. The
  case where $Q$ is linear, it is of the form $x - \alpha$, and the
  numerator is a constant, so this is a straightforward power
  integration. In the case where $Q$ is quadratic, we break $R$ as the
  sum of a constant and the derivative of $Q$. The constant part is
  handled by a trigonometric substitution, and the derivative of $Q$
  part is handled by the $u$-substitution $u = Q$.
\item The partial fractions approach shows that every rational
  function can be integrated, and we obtain an antiderivative that
  involves $\ln$ (evaluated at some linear function of $x$), $\arctan$
  (again, evaluated at some linear function of $x$), and other rational
  functions.
\item Using the partial fractions approach and the equivalence of
  repeated integrability with the integrability of $x$ times a
  function, we can show that any rational function can be {\em
  repeatedly} integrated, with the final answer in terms of $\arctan$,
  $\ln$, and rational functions.
\end{enumerate}

Actions ... Please go through the notes on partial fractions as well
as the discussion of these in the book. We here list only some salient
points:

\begin{enumerate}
\item Before beginning, make the denominator monic, and use the
  Euclidean algorithm to reduce to a problem where the degree of the
  numerator is less than the degree of the denominator.
\item The general approach is to first factorize the denominator and
  then break it up into partial fractions with unknown numerators. The
  coefficients of the numerator need to be determined. One way of
  doing this is to take a common denominator, multiply out, compare
  coefficients, and solve the resultant system of linear equations.
\item Instead of equating coefficients, we can also use a strategy of
  plugging in values. We plug in values so that a large number of the
  expressions that we are evaluating become zero.
\item In particular, if we want to write:

  $$\frac{r(x)}{(x - \alpha_1)(x - \alpha_2) \dots (x - \alpha_n)} = \frac{c_1}{x - \alpha_1} + \dots + \frac{c_n}{x - \alpha_n}$$

  where all the $\alpha_i$s are distinct and the degree of $r$ is less
  than $n$, then we get:

  $$c_i = \frac{r(\alpha_i)}{\text{product of $\alpha_i - \alpha_j$, all $j \ne i$}}$$

  We can use this to very rapidly write any fraction with denominator
  a product of distinct linear factors in terms of partial fractions,
  and then integrate it.
\item To handle:

  $$\frac{r(x) \, dx}{(q(x))^k}$$

  where $q$ is an irreducible quadratic, we do repeated division,
  taking quotients and remainders, and obtain the result in terms of
  partial fractions.
\item A thorough understanding of the partial fractions approach
  should allow you to predict, simply by looking at a rational
  function, whether the antiderivative expression for it will be (i) a
  rational function, (ii) something involving rational functions and
  $\arctan$, (iii) something involving rational functions and $\ln$,
  or (iv) something involving rational functions, $\arctan$, and
  $\ln$. For some practice of these, refer to the integration quiz.
\end{enumerate}

\section{The goal: integrate any rational function}

Today's goal is to obtain a general strategy that will allow us to
integrate any rational function. If we are able to accomplish this,
then by some remarks made earlier, it should also be possible to {\em
repeatedly} integrate any rational function.

What we are doing is based on many deep results in algebra about
polynomials, most of which we cannot explore in full detail because
you don't have the requisite background.

\subsection{A theorem about factorization of polynomials}

For convenience, we will deal here only with {\em monic}
polynomials. A monic polynomial is a polynomial whose leading
coefficient is $1$, i.e., the coefficient of the highest degree term
is $1$. Every nonzero polynomial can be expressed as a constant
multiple of a monic polynomial. Since constants can be pulled out of
integration and differentiation problems, do not affect the nature of
the zeros, and have a clear and easy-to-work-out effect on the graph
and behavior of a function, we do not really lose out on much by
restricting our analysis to monic polynomials.

To further simplify matters, we will deal with polynomials of degree
$1$ or more, i.e., with nonconstant polynomials. Thus, our focus is on
{\em nonconstant monic polynomials}.

We are now in a position to state the grand theorem valid {\em over
the real numbers}:

\begin{quote}
  Every nonconstant monic polynomial $p$ can be expressed as a product
  of nonconstant monic polynomials $p_1$, $p_2$, $\dots$, $p_n$, where
  each $p_i$ is either linear (i.e., degree $1$) or quadratic with
  negative discriminant. In other words, every nonconstant monic
  polynomial can be expressed as a product of nonconstant monic
  irreducible polynomials of degree $1$ or $2$. Moreover, the
  decomposition is unique up to ordering of the $p_i$s. The $p_i$s
  need not be distinct.
\end{quote}

This result essentially states that there is a uniqueness of
factorization for polynomials and all the irreducible polynomials are
of degree either $1$ or $2$.

\subsection{The two kinds of irreducible pieces}

It is worth examining a little more closely the nonconstant monic
polynomials that are irreducible, i.e., cannot be factorized
further. From the above theorem, the only such types are the linear
polynomials (polynomials of the form $x - a$) and the quadratic
polynomials with negative discriminant (polynomials of the form $x^2 +
Bx + C$ where $B^2 - 4C < 0$).

By the square completion technique, we can write:

$$x^2 + Bx + C = (x + (B/2))^2 - D/4$$

where $D = B^2 - 4C$ is the discriminant.

Since $D$ is negative, $-D/4$ is positive, and is the square of
something. Setting $\beta = -B/2$ and $\gamma = \sqrt{-D/4}$, we obtain:

$$x^2 + Bx + C = (x - \beta)^2 + \gamma^2$$

where $\gamma > 0$.

\subsection{Coming home}

We next look at the question: how do we integrate:

$$\int \frac{dx}{q(x)}$$

where $q$ is an {\em irreducible} nonconstant monic polynomial? We
already have the two cases above, so we consider both of them:

\begin{enumerate}
\item If $q(x) = x - a$, i.e., $q$ is linear, then the integral is
  $\ln |x - a| + C$. It's a nice coincidence that we have already tackled
  this integral.
\item If $q$ is quadratic with negative discriminant, we first write
  $q(x) = (x - \beta)^2 + \gamma^2$, with $\gamma > 0$. Then we use
  the integration formula to obtain that the integral is $(1/\gamma)
  \arctan((x - \beta)/\gamma) + C$. It's again a nice coincidence that
  we were talking about these integrals recently.
\end{enumerate}

Now, what about integrating something like:

$$\int \frac{a(x) \, dx}{q(x)}$$

where $q$ is an irreducible nonconstant monic polynomial, and $a$ is
any polynomial? We first use polynomial long division (yes!) to write:


$$a(x) = b(x)q(x) + r(x)$$

where $b$ is the quotient and $r$ is the remainder, with either $r =
0$ or the degree of $r$ less than the degree of $q$ (note that $r$
need not be monic). We thus obtain:

$$\frac{a(x)}{q(x)} = b(x) + \frac{r(x)}{q(x)}$$

To integrate the left side, we can integrate the right
side. Integrating the polynomial $b$ is straightforward, so what we are
left with is an integration of the form:

$$\int \frac{r(x) \, dx}{q(x)}$$

where the degree of $r$ is strictly smaller than the degree of $q$. We
make three cases:

\begin{enumerate}
\item $q$ has degree one: In this case, $r$ is a constant, and can be
  pulled out of the integration, and we are reduced to integrating
  $dx/q(x)$, which we discussed above.
\item $q$ has degree two but $r$ is still a constant: In this case, we
  pull $r$ out and are again reduced to integrating $dx/(q(x))$, which
  we discussed above.
\item $q$ has degree two and $r$ has degree one: This needs more
  thought, and we will now turn our attention to it.
\end{enumerate}

\subsection{Integating a linear over an irreducible quadratic}

Consider the integration problem, with $\gamma > 0$:

$$\int \frac{(Ax + B) \, dx}{(x - \beta)^2 + \gamma^2} $$

We use the following two known integrals:

$$\int \frac{1 \, dx}{(x - \beta)^2 + \gamma^2} = \frac{1}{\gamma} \arctan\left(\frac{x - \beta}{\gamma}\right)$$

and:

$$\int \frac{2(x - \beta) \, dx}{(x - \beta)^2 + \gamma^2} = \ln((x - \beta)^2 + \gamma^2)$$

The latter integral is based on the $f'/f$ formulation. Note that we
do not need an absolute value sign on the right because the expression
in parentheses is always positive.

We now try to write the integrand that we need to integrate in terms
of the two integrands that we know how to integrate. In other words,
we try to write:

$$Ax + B = c_1(1) + c_2(2(x - \beta))$$

where $c_1$ and $c_2$ are constants. We then try to solve for $c_1$
and $c_2$, and then split the integrand as a linear combination of the
known integrands, that we then integrate.

For instance:

$$\int \frac{x \, dx}{x^2 + x + 1}$$

Note that the denominator has negative discriminant, and we can use
square completion to write it as $(x + (1/2))^2 + (\sqrt{3}/2)^2$. We obtain:

$$\int \frac{x \, dx}{(x + (1/2))^2 + (\sqrt{3}/2)^2} $$

The derivative of the denominator is $2x + 1$. We now want to write:

$$x = c_1(1) + c_2(2x + 1)$$

Expanding and comparing coefficients, we obtain that $2c_2 = 1$ and
$c_1 + c_2 = 0$, which yields that $c_1 = -1/2$, $c_2 = 1/2$. We thus get:

$$\frac{-1}{2} \int \frac{dx}{(x + 1/2)^2 + (\sqrt{3}/2)^2}  + \frac{1}{2} \int \frac{2x + 1}{x^2 + x + 1} \, dx$$

This simplifies to:

$$\frac{-1}{\sqrt{3}} \arctan\left(\frac{x + (1/2)}{\sqrt{3}/2}\right) + \frac{1}{2} \ln(x^2 + x + 1)$$

With some practice, you should be able to see the splitting by
inspection, and hence will not need to set up a system of simultaneous
linear equations to determine the constants $c_1$ and $c_2$.

\subsection{Partial fractions}

Suppose we want to integrate a rational function of the form:

$$\frac{a(x)}{q(x)}$$

where $q$ is nonconstant and monic. Note that if $q$ is constant, this
is just polynomial integration. If $q$ is not monic, we can pull out a
constant to make it monic. Thus, we can, {\em without loss of
generality}, restrict to the case where $q$ is a nonconstant monic
polynomial.

We can first use long division to reduce this problem to integrating:

$$\frac{r(x)}{q(x)}$$

where the degree of $r$ is strictly less than the degree of $q$. The
claim is that the above can be written as a sum of expressions of
the form:

$$\frac{R(x)}{Q(x)^k}$$

where $Q$ is nonconstant monic irreducible (hence, degree either $1$
or $2$), the degree of $R$ is less than the degree of $Q$, $k$ is a
positive integer, and $Q^k$ divides $q$. We then devise strategies to:

\begin{enumerate}
\item find a way of taking a rational function and writing it as a sum
  of terms of the above form. This process is called {\em partial
    fraction decomposition}.
\item integrate each of the terms we obtain as a result of the
  decomposition.
\end{enumerate}

The unappealing idea behind (1) is setting up and solving a monstrous
system of linear equations (although there are shortcuts in some
cases). The unappealing idea behind (2) is to use trigonometric
substitutions in the rare cases where we cannot directly obtain a
result from one of the canned formulas.

\subsection{Denominator a product of two distinct linear terms}

Consider:

$$\int \frac{dx}{x^2 - 5x + 6}$$

The denominator can be factorized as $(x - 2)(x - 3)$, and we obtain:

$$\int \frac{dx}{(x - 2)(x - 3)}$$

An astute observation at this stage yields that:

$$\frac{1}{(x - 2)(x - 3)} = \frac{(x - 2) - (x - 3)}{(x - 2)(x - 3)} = \frac{1}{x - 3} - \frac{1}{x - 2}$$

Thus, the original integral expression becomes:

$$\int \frac{dx}{x - 3} - \int \frac{dx}{x - 2}$$

which simplifies to:

$$\ln|x - 3| - \ln |x - 2| + C$$

which can also be written as:

$$\ln\left|\frac{x - 3}{x - 2}\right| + C$$

In this case, the partial fraction decomposition that helped us was:

$$\frac{1}{(x - 2)(x - 3)} = \frac{1}{x - 3} - \frac{1}{x - 2}$$

In general, if we are given:

$$\frac{Ax + B}{(x - \alpha_1)(x - \alpha_2)}$$

We want to write it as:

$$\frac{Ax + B}{(x - \alpha_1)(x - \alpha_2)} = \frac{c_1}{x -
\alpha_1} + \frac{c_2}{x - \alpha_2}$$

We now want to determine $c_1$ and $c_2$ in terms of $A$ and $B$. We
do so by cross-multiplying and comparing coefficients. Specifically, we get:

$$Ax + B = c_1(x - \alpha_2) + c_2(x - \alpha_1)$$

This yields $A = c_1 + c_2$ and $B = -(c_1\alpha_2 + c_2\alpha_1)$. We
then proceed to solve for $c_1$ and $c_2$ in terms of $A$ and $B$.

In our case, $Ax + B$ was $1$, so $A = 0$ and $B = 1$, while $\alpha_1
= 2$, $\alpha_2 = 3$. Solving this system yielded $c_1 = -1$, $c_2 =
1$, which is what we used.

\subsection*{Aside: Speed to boast about}

If you want to get really quick at integrating such expressions, you
should solve these above systems in general (i.e., find expressions
for $c_1$ and $c_2$ in terms of $A$, $B$, $\alpha_1$, and $\alpha_2$,
and solve through to the end in the {\em general case}), and then
memorize the final expressions you get. You can then jump straight
from the statement of the question to the final answer. However, it is
recommended that before mastering such shortcuts, you should solve out
a few problems in full detail.

Of particular interest is the case where the numerator is $1$. In this
case, the formula is easy to master and fairly intuitive.

\subsection{Denominator a product of more than two distinct linear terms}

In this subsection, we explore the linear equations approach. In the
next subsection, we consider a slight variant of the approach that
works well for products of distinct linear terms, and is considerably
faster in many such cases.

Suppose we want to integrate:

$$\int \frac{dx}{x^3 - 2x}$$

We first note that the denominator is expressible as a product of
three linear terms, so we rewrite as:

$$\int \frac{dx}{x(x - \sqrt{2})(x + \sqrt{2})}$$

Our goal is to now find constants $c_1$, $c_2$, and $c_3$ such that:

$$\frac{1}{x^3 - 2x} = \frac{c_1}{x} + \frac{c_2}{x - \sqrt{2}} + \frac{c_3}{x + \sqrt{2}}$$

One way to do this is to take a common denominator on the right side
and simplify it. We get:

$$\frac{1}{x^3 - 2x} = \frac{c_1(x^2 - 2) + c_2x(x + \sqrt{2}) + c_3x(x - \sqrt{2})}{x^3 - 2x}$$

We now equate the numerators to get:

$$1 = c_1(x^2 - 2) + c_2x(x + \sqrt{2}) + c_3x(x - \sqrt{2})$$

Since these are equal as polynomials in $x$, we equate them
coefficient-wise, obtaining a system of simultaneous linear equations
in the variables $c_1$, $c_2$, and $c_3$, that we then dutifully
solve. Once we determine all the values, we write the answer, which is:

$$c_1 \ln|x| + c_2 \ln|x - \sqrt{2}| + c_3 \ln|x + \sqrt{2}| + C$$

Let's do the actual finding of $c_1$, $c_2$, and $c_3$. We get the
following system of simultaneous linear equations:

\begin{eqnarray*}
  c_1 + c_2 + c_3 & = & 0 \\
  \sqrt{2} c_2 - \sqrt{2} c_3 & = & 0\\
  -2c_1 & = & 1 
\end{eqnarray*}

Solving, we obtain that $c_1 = -1/2$, $c_2 = 1/4$, and $c_3 =
1/4$. The answer is thus:

$$\frac{-1}{2} \ln|x| + \frac{1}{4} \ln|x - \sqrt{2}| + \frac{1}{4} \ln|x + \sqrt{2}| + C$$

We can combine the second and third terms to get:

$$\frac{-1}{2} \ln|x| + \frac{1}{4} \ln|x^2 - 2| + C$$

We see how this can be generalized. If the denominator is a product of
$n$ distinct linear terms, say $(x - \alpha_1)(x - \alpha_2) \dots (x
- \alpha_n)$, we try to write the function as:

$$\frac{c_1}{x - \alpha_1} + \frac{c_2}{x - \alpha_2} + \dots + \frac{c_n}{x - \alpha_n}$$

We then take a common denominator, simplify the numerator, and equate
it as a polynomial to the numerator of the original integrand. We
equate coeffiicent-wise as polynomials in $x$, creating a system of
simultaneous linear equations with variables $c_i$, that we then
solve. The final answer is:

$$c_1 \ln|x - \alpha_1| + c_2 \ln|x - \alpha_2| + \dots + c_n\ln |x - \alpha_n|$$

\subsection{Another way of finding the coefficients for partial fractions}

Let us reconsider the problem of splitting into partial fractions:

$$\int \frac{r(x) \, dx}{(x - \alpha_1)(x - \alpha_2) \dots (x - \alpha_n)}$$

where all that $\alpha_i$s are distinct.

We want to write this as:

$$\frac{r(x)}{(x - \alpha_1)(x - \alpha_2) \dots (x - \alpha_n)} = \frac{c_1}{x - \alpha_1} + \frac{c_2}{x - \alpha_2} + \dots + \frac{c_n}{x - \alpha_n}$$

To do this, we take a common denominator and equate the numerators, getting:

$$r(x) = c_1(x - \alpha_2) \dots (x - \alpha_n) + c_2(x - \alpha_1)(x - \alpha_3) \dots (x - \alpha_n) + \dots + c_n(x - \alpha_1) \dots (x - \alpha_{n-1})$$

Earlier, we had proceeded from this point onward by expanding out,
comparing coefficients of each power of $x$, and equating these
coefficients. This gave us a system of linear equations in the
variables $c_1$, $c_2$, $\dots, c_n$. We then proceeded to solve this
system to find the $c_i$s.

For $n \le 2$, this procedure is feasible and quick. However, for $n
\ge 3$, the procedure becomes messy because we first need to do a lot
of tedious term multiplications to find coefficients, and then we need
to solve a tedious system of linear equations.

The approach we have used above is based on the key idea that {\em if
two polynomials are equal, they are equal coefficient-wise}.

The approach we will now use is based on the key idea that {\em if two
polynomials are equal, their values at every number are equal}.

Consider:

$$r(x) = c_1(x - \alpha_2) \dots (x - \alpha_n) + c_2(x - \alpha_1)(x - \alpha_3) \dots (x - \alpha_n) + \dots + c_n(x - \alpha_1) \dots (x - \alpha_{n-1})$$

This equality is an {\em identity} in $x$, in the sense that it is
true for all $x$. In particular, it is true if we set $x =
\alpha_1$. Doing this, we note that the only term on the right side
that survives is the first product, and we get:

$$r(\alpha_1) = c_1(\alpha_1 - \alpha_2)(\alpha_1 - \alpha_3) \dots (\alpha_1 - \alpha_n)$$

We thus get:

$$c_1 = \frac{r(\alpha_1)}{(\alpha_1 - \alpha_2)(\alpha_1 - \alpha_3) \dots (\alpha_1 - \alpha_n)}$$

We can obtain analogous expressions for the other $c_i$s.

With specific values, we can compute this easily. This method is
preferable (from a speed perspective) when $n \ge 3$. Please see
numerical examples in the book.

The idea is also important from a conceptual perspective. The key idea
is that when a given equation is an {\em identity} in one variable
(i.e., it is true for all possible values that that one variable can
take) we can plug in specific values of that variable to get specific
equations in the other variables.

\subsection{Denominator a higher power of a linear expression}

Consider the integration:

$$\int \frac{x^2 + 5x + 6}{(x - 1)^3} \, dx$$

Here, we can do a $u$-substitution with $u = x - 1$, so $x = u + 1$,
and the expression now becomes a sum of powers of $u$. Let's do
this. Note that $dx = du$, so we get:

$$\int \frac{(u + 1)^2 + 5(u + 1) + 6}{u^3} \, du$$

This simplifies to:

$$\int \frac{u^2 + 7u + 12}{u^3} \, du$$

Dividing each term by $u^3$ and splitting up, we get:

$$\int \frac{du}{u} + 7 \int \frac{du}{u^2} + 12 \int \frac{du}{u^3}$$

which becomes:

$$\ln|u| - \frac{7}{u} - \frac{6}{u^2}$$

Substituting back $u = x - 1$, we obtain:

$$\ln|x - 1| - \frac{7}{x - 1} - \frac{6}{(x - 1)^2} + C$$

The above approach is equivalent to the following superficially
different approach, where we try to rewrite the original function as:

$$\frac{\text{first constant}}{x - 1} + \frac{\text{second constant}}{(x - 1)^2} + \frac{\text{third constant}}{(x- 1)^3}$$

where each part is easy to integrate using the integration of power
functions. 

\subsection{Denominator a product of linear and quadratic irreducibles}

Integration is a thankless task. Accomplishments in integration invite
more integration problems rather than applause.

So far, we have tackled situations where the denominator is a single
irreducible, where it is a product of distinct linear terms, and where
it is a power of a linear term. We now deal with the case where the
denominator is a product involving distinct linear and quadratic
irreducibles. This is {\em almost} the end. The final straw will be
the case where the denominator is a product of possibly repeated
linear and quadratic terms.

We consider the integral:

$$\int \frac{x^3 + x + 1}{x^3 + x} \, dx$$

First, note that here, the degree of the numerator is {\em not}
strictly less than the degree of the denominator, which means that the
first step is long division. We do this, and obtain:

$$\int \left[1 + \frac{1}{x^3 + x}\right] \, dx$$

The $1$ integrates to $x$, so we need to integrate:

$$\int \frac{dx}{x^3 + x}$$

The denominator can be factorized as $x(x^2 + 1)$. It cannot be
factorized further because $x^2 + 1$ is an irreducible quadratic.

We try now to write it as:

$$\frac{1}{x^3 + x} = \frac{Ax + B}{x^2 + 1} + \frac{c}{x}$$

Again, we cross multiply and simplify the right side and equate
numerators, getting:

$$1 = x(Ax + B) + c(x^2 + 1)$$

Equating coefficients, we obtain:

\begin{eqnarray*}
  A + c & = & 0\\
  B & = & 0 \\
  c & = & 1 \\
\end{eqnarray*}

We thus get $A = -1$, $B = 0$, and $c = 1$. Plugging in, we obtain:

$$\frac{1}{x^3 + x} = \frac{-x}{x^2 + 1} + \frac{1}{x}$$

We now try integrating both pieces. Note that we can do each piece
separately, because it is of the form $r/q$ where $q$ is irreducible
and $r$ has smaller degree. In our case, the answer is:

$$\frac{-1}{2} \ln(x^2 + 1) + \ln|x|$$

Finally, we should remember that the original problem was different,
and we need to add a $x$ that comes from integrating the $1$. So the
answer to the original problem is:

$$x - \frac{1}{2}\ln(x^2 + 1) + \ln|x| + C$$

\subsection{Situation where the denominator is a power of an irreducible quadratic}

Suppose we want to carry out the integration:

$$\int \frac{Ax^3 + Bx^2 + Cx + D}{[(x - \beta)^2 + \gamma^2]^2} \, dx$$

The first step is to break up the integrand as:

$$\frac{\text{constant or linear}}{(x - \beta)^2 + \gamma^2} + \frac{\text{constant or linear}}{[(x - \beta)^2 + \gamma^2]^2} \text{Note: Correction made}$$

This can be done by long division. For instance, consider the problem:

$$\int \frac{x^3 + 2x^2 + 3x + 4}{(x^2 + 1)^2} \, dx$$

We use polynomial division to divide the numerator by $x^2 + 1$. We
obtain that:

$$x^3 + 2x^2 + 3x + 4 = (x^2 + 1)(x + 2) + (2x + 2)$$

Dividing both sides by $(x^2 + 1)^2$, we obtain:

$$\frac{x^3 + 2x^2 + 3x + 4}{(x^2 + 1)^2} = \frac{x + 2}{x^2 + 1} + \frac{2x + 2}{(x^2 + 1)^2}$$

Now, for each piece, we use the integration strategy we learned
earlier: write the numerator as a linear combination of $1$ and $2(x -
\beta)$. In this case, for instance, $2(x - \beta) = 2x$, so the first
fraction becomes:

$$\frac{1}{2}\frac{2x}{x^2 + 1} + 2 \cdot \frac{1}{x^2 + 1}$$

The second integrand becomes:

$$1 \cdot \frac{2x}{(x^2 + 1)^2} + 2 \cdot \frac{1}{(x^2 + 1)^2}$$

We have thus broken up the original integrand as a combination of four
terms, each of which has numerator either equal to $1$ or equal to the
derivative of the quadratic in the denominator. For the cases where
the numerator is $1$, we use the trigonometric substitution idea. For
the cases where the numerator is the derivative of the quadratic in
the denominator, we use a $u$-substitution where $u$ is the expression
in the denominator. In this case, our answer is:

$$\frac{1}{2} \ln(x^2 + 1) + 2 \arctan (x) - \frac{1}{x^2 + 1} + 2 \int \frac{dx}{(x^2 + 1)^2}$$

The fourth integral requires the trigonometric substitution $\theta =
\arctan x$, which reduces to integrating $\cos^2 \theta$, which we in
turn simplify and substitute back in terms of $x$.

\subsection{The overall strategy}

Here is a summary of the overall strategy (as before, we assume that
the denominator is nonconstant and monic):

\begin{enumerate}
\item First, factorize the denominator.
\item Collect all repeated factors together.
\item If the numerator and denominator have common factors, cancel.
\item If the degree of the numerator is greater than that of the
  denominator, use polynomial division to reduce to a problem where the
  degree of the numerator is less than that of the denominator.
\item Write the new rational function $r/q$ as a sum of rational
  functions of the form $R/Q^k$ where $Q$ is one of the irreducible
  factors, $k$ is at most equal to the multiplicity of $Q$ in $q$, and
  the degree of $R$ is less than the degree of $Q$. This may be
  accomplished by polynomial division along with some setting up of
  simultaneous linear equations.
\item For those cases where $Q$ is linear, integration is
  straightforward and gives an answer that is a rational function if
  $k > 1$ and a logarithmic function if $k = 1$.
\item For those cases where $Q$ is quadratic, write $R$ as a linear
  combination of $1$ and the derivative of $Q$. The $1$ part
  integrates using an arc tangent substitution, while the derivative
  of $Q$ part integrates by setting $u = Q$. For the derivative of $Q$
  part, the answer is a rational function if $k > 1$ and $\ln(Q)$ if
  $k = 1$. Note that we do not need the absolute value here because
  the quadratic is always positive.
\end{enumerate}

\section{Integrating between limits}

\subsection{Interval red alert}

One important thing to remember about rational functions is that they
need not be defined everywhere. Specifically, a rational function {\em
blows up} at the zeros of the denominator, i.e., the points where the
denominator becomes zero. Thus, it is meaningless to integrate a
rational function over an interval if the interval contains any of the
zeros of the denominator.

\subsection{Sporadic manna from heaven: symmetry}

When integrating rational functions, it is useful to be on the lookout
for symmetries (even/odd/mirror symmetry/half-turn symmetry) that may
result in a particular integral being zero without our having to
compute it as an indefinite integral. Here are some quick reminders:

\begin{enumerate}
\item A polynomial is an even function if and only if it is an {\em
  even polynomial}: all its terms with nonzero coefficients have even
  degree.
\item A polynomial is an odd function if and only if it is an {\em odd
  polynomial}: all its terms with nonzero coefficients have odd
  degree. Thus, $x^3$ is an odd polynomial, and so is $x^3 - 46 x$,
  but $x^3 + 1$ is not odd.
\item The quotient of two even polynomials is even, and the quotient
  of two odd polynomials is even. The quotient of an even polynomial
  by an odd polynomial, or the quotient of an odd polynomial by an
  even polynomial, is odd.
\end{enumerate}

In particular, suppose we are trying to compute:

$$\int_{-1}^1 \frac{x^5 - x}{(x^2 + 2)^3} \, dx$$

the answer is $0$, because the function is odd and the interval of
integration is symmetric about $0$.

This method may sometimes allow us to simplify one or more of the many
partial fractions that we obtain using the partial fraction
decomposition, allowing us to concentrate on the others.

\subsection{Computation as the last resort: boring but dependable}

When shortcuts fail, we simply compute the indefinite integral and
evaluate it between limits. At the end of this, we get an answer that
possibly involves $\ln$s and $\arctan$s. Often, these cannot be
further simplified for the particular limits of
integration. Sometimes, they can, but the answers are still messy. In
any case, we {\em have} reduced an integration problem to a problem
that involves looking up natural logarithm and arc tangent
tables. This is no small feat.

For instance, consider the problem:

$$\int_1^3 \frac{dx}{x^2 + x + 1}$$

An antiderivative that we obtain using our method is:

$$\frac{2}{\sqrt{3}} \arctan \left(\frac{x + (1/2)}{\sqrt{3}/2}\right)$$

Evaluating between limits, we obtain:

$$\frac{2}{\sqrt{3}} \left[\arctan(7/\sqrt{3}) - \arctan(\sqrt{3})\right]$$

The second $\arctan$ is precisely $\pi/3$, but the first one does not
correspond to any angle that we already know. Nonetheless, we {\em do}
know that it is between $\pi/3$ and $\pi/2$. The difference is thus
between $0$ and $\pi/6$, and the final answer is thus between $0$ and
$\pi/3\sqrt{3}$, which is less than $0.62$. If we had a better
understanding of $\arctan$ (something we will achieve by the end of
this course), we could probably estimate the integral even better.

\section{Summaries and miscellanea}

\subsection{Transcendental mess and repeated integration}

The term {\em transcendental} is used for functions that don't arise
from algebraic functions through the usual processes of combination,
composition, and takin inverses. It includes the exponential and
logarithmic functions, as well as the trigonometric and inverse
trigonometric functions.

As we have seen, integrating a rational function can give rise to a
transcendental function. However, only a very limited collection of
transcendental functions arise this way. In particular, only a small
number of ways of combining $\ln$ and $\arctan$ can arise in this way.

Further, recall that to {\em repeatedly integrate} a rational function
$f$, we use the fact that $f(x)$, $xf(x)$, $x^2f(x)$ and so on are all
rational functions and hence can all be integrated. All of these
rational functions have the same denominator (possibly a smaller one
due to cancellations).

Here are some observations:

\begin{itemize}
\item If the denominator can be completely factorized into {\em
  distinct} monic linear factors, then the antiderivative we get is a
  linear combination of $\ln$s (with constant coefficients) at these
  linear factors (absolute values), plus a polynomial. There is no
  appearance of $\arctan$.

  Repeated integration of such a rational function yields a
  combination of $\ln$s with polynomial coefficients, plus a
  polynomial.
\item If the denominator can be completely factorized into linear
  factors, possibly with repetition, then the antiderivative is a
  linear combination of $\ln$s, reciprocals of powers of these
  linear factors, plus a polynomial.

  Repeated integration of such a rational function yields a
  combination of $\ln$s with polynomial coefficients, plus a
  polynomial.
\end{itemize}

In particular, $\arctan$ starts appearing only once we introduce
irreducible {\em quadratic} factors. Some of you may be wondering why
$\arctan$ should appear at all, and why everything could not be done
with $\ln$. In fact, {\em if we were allowed to use the complex
numbers instead of the real numbers}, then we could break everything
down into linear factors, and everything could be effectively
accomplished using $\ln$. However, the complex numbers pose their own
problems. Back in the real world, we need both $\ln$ and $\arctan$,
which is a kind of imaginary-twisted version of $\ln$. This
relationship is based on the relationship between circular
trigonometric functions and hyperbolic trigonometric functions alluded
to in the past.

\subsection{A summary of techniques for various degrees of numerators and denominators}

Here is a summary for most basic cases -- we assume that the
denominator is monic:

\begin{enumerate}
\item The degree of the numerator is greater than the degree of the
  denominator: Use polynomial long division to convert to a problem
  where the degree of the numerator is less than the degree of the
  denominator.
\item The denominator is linear, and the numerator is constant: In this
  case, we use the formula:

  $$\int \frac{c}{x - \alpha} \, dx = c\ln|x - \alpha| + C$$
\item The denominator is quadratic {\em irreducible} and the numerator
  is constant or linear. We break the numerator into two pieces, one
  of which integrates to a logarithmic expression and the other one
  integrates to an arc tangent expression:

  $$\int \frac{Ax + B}{(x - \beta)^2 + \gamma^2} \, dx = \frac{A}{2} \ln\left[(x - \beta)^2 + \gamma^2\right] + \frac{B + A\beta}{\gamma} \arctan\left(\frac{x - \beta}{\gamma}\right) + C$$

\item The denominator is quadratic with two distinct factors $\alpha$
  and $\beta$:

  $$\int \frac{Ax + B}{(x - \alpha)(x - \beta)} \, dx= \frac{A\alpha + B}{\alpha - \beta} \ln|x - \alpha| + \frac{A\beta + B}{\beta - \alpha} \ln|x - \beta|$$

\item The denominator is a quadratic with a repeated linear factor:
  Here, we simply put $u$ as that linear factor and do a
  $u$-substitution. We can also work out an explicit formula:

  $$\int \frac{Ax + B}{(x - \alpha)^2} \, dx = A\ln|x - \alpha| - \frac{A\alpha + B}{x - \alpha} + C$$

\item The denominator is cubic with three distinct linear factors
  $\alpha_1$, $\alpha_2$, $\alpha_3$: We follow the partial fractions
  method and get a final answer as $c_1\ln|x - \alpha_1| + c_2\ln|x -
  \alpha_2| + c_3 \ln|x - \alpha_3|$. As discussed in an earlier
  section, if we numerator is the polynomial $r(x)$, then $c_1 =
  r(\alpha_1)/(\alpha_1 - \alpha_2)(\alpha_1 - \alpha_3)$, $c_2 =
  r(\alpha_2)/(\alpha_2 - \alpha_1)(\alpha_2 - \alpha_3)$, and $c_3 =
  r(\alpha_3)/(\alpha_3 - \alpha_1)(\alpha_3 - \alpha_2)$.

\item The denominator is cubic and splits as a product of a linear
  factor and an irreducible quadratic: We use partial fractions to try
  to write the fraction as:

  $$\frac{\text{constant}}{\text{the linear factor}} + \frac{\text{constant or linear}}{\text{the irreducible quadratic factor}}$$

  We know how to integrate each part. (One could write out a general
  formula along the lines above, but it is probably unilluminating).
\item The denominator is cubic and splits as a product of a linear
  factor repeated twice ($(x - \alpha)^2$) and an isolated linear
  factor $x - \beta$: In this case, we use partial fractions to write
  the fraction as:

  $$\frac{\text{constant or linear}}{(x - \alpha)^2} + \frac{\text{constant}}{x - \beta}$$

  We can then solve each part separately. Alternatively, we may choose
  to {\em directly} write the original as:

  $$\frac{\text{constant}}{(x - \alpha)^2} + \frac{\text{constant}}{x - \alpha} + \frac{\text{constant}}{x - \beta}$$

\item The denominator is the cube of $(x - \alpha)$: Then, set $u = x
  - \alpha$, substitute, and solve.
\item The denominator is of degree $4$ and is a product of two
  distinct irreducible quadratics: Obtain a partial fraction
  decomposition as:

  $$\frac{\text{constant or linear}}{\text{First irreducible quadratic}} + \frac{\text{constant or linear}}{\text{Second irreducible quadratic}}$$

  We know how to tackle each part.
\item The denominator is of degree $4$ and is the square of an
  irreducible quadratic: Obtain a partial fraction decomposition,
  using polynomial long division, as:

  $$\frac{\text{constant or linear}}{\text{irreducible quadratic}} + \frac{\text{constant or linear}}{\text{square of irreducible quadratic}}$$

  We know how to integrate the first fraction. For the second
  fraction, we break up the numerator as a linear combination of the
  derivative of the denominator and $1$. The derivative part is solved
  by taking $u$ equal to the denominator. The $1$ part is handled by a
  trigonometric substitution.
\item The denominator is of degree $4$ and is the product of an
  irreducible quadratic and two distinct irreducible linear factors:
  Use partial fractions to get:

  $$\frac{\text{constant or linear}}{\text{irreducible quadratic}} + \frac{\text{constant}}{\text{first linear factor}} + \frac{\text{constant}}{\text{second linear factor}}$$
\end{enumerate}

Try figuring out the remaining cases for a degree $4$ polynomial: (i)
product of an irreducible quadratic and the square of a linear factor,
(ii) product of four distinct linear factors , (iii) product of two
linear factors, each squared, (iv) product of one linear factor cubed
and another linear factor (v) product of a squared linear factor and
two other linear factors (vi) a single linear factor to the fourth
power.

If you're ambitious, you might want to work out all the cases for
polynomials of degree five.

\end{document}
