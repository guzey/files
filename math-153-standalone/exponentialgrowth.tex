\documentclass{amsart}
\usepackage{fullpage,hyperref,vipul,graphicx}
\title{Exponential growth and decay}
\author{Math 153, Section 55 (Vipul Naik)}

\begin{document}
\maketitle

{\bf Corresponding material in the book}: Section 7.6.

{\bf What students should definitely get}: The definition of
exponential growth and exponential decay, how this is used.

{\bf What students should hopefully get}: The intuition of exponential
growth and decay, how it is used in trend prediction, the discrete
versus continuous aspects of exponential growth and decay.

\section*{Note to newcomers}

The lecture notes cover the material that was intended for the
lecture, but may not correspond precisely with what is covered during
the lecture. In some cases, the lecture notes may have more details
about an example that was sketched quickly during lecture due to time
constraints.

Each lecture notes file begins with an ``executive summary'' which is
a useful way of reviewing the lecture {\em after} you have already
understood the material, and is not designed for people learning the
material the first time.

\section*{Executive summary}

\subsection{Basics of exponential growth and decay}

Words ...

\begin{enumerate}
\item A function $f$ is said to have exponential growth if $f'(t) =
  kf(t)$ for all $t$. Such a function must be of the form $f(t) :=
  Ce^{kt}$. Here, $C = f(0)$, and can be thought of as the initial
  value. $k$ is a parameter controlling the {\em rate of growth}. When
  $k > 0$, we have growth, and when $k < 0$, we have {\em decay}. When
  $k = 0$, there's no growth or decay.
\item For a function with exponential growth and growth rate $k$, the
  time taken for the function value to multiply by a number $q > 0$
  depends only on $q$ and $k$. Specifically, the time interval is
  $(\ln q)/k$. In particular, for growth, the doubling time is $(\ln
  2)/k$. Note that if $\ln q$ and $k$ have opposite signs, the time
  taken is negative -- which means that we need to go {\em back} in
  time to multiply by a factor of $q$.
\item If a function takes a time interval $t_{d1}$ to multiply by a
  factor of $q_1$ and a time interval $t_{d2}$ to multiply by a factor
  of $q_2$, we have the relation: $\ln(q_1)/t_{d1} =
  \ln(q_2)/t_{d2}$. Thus, given three of these quantities, we can
  calculate the fourth.
\item Exponential functions grow faster than all positive power
  functions (and hence all polynomial functions) while logarithmic
  functions grow slower than all positive power functions.
\item Exponential functions {\em decay} slower than linear
  functions. In other words, the time taken for the ``first'' $1/n$ of
  a material to decay is less than the time taken for the ``second''
  $1/n$ to decay, and so on.
\end{enumerate}

Actions ...

\begin{enumerate}
\item Suppose we know that $f$ is a function of the form $f(t) :=
  Ce^{kt}$, but we do not know the values of the constants $C$ and
  $k$. One way of determining these values is to determine the value
  of $f$ and $f'$ at some point $t_0$. We can then get $k$ as
  $f'(t_0)/f(t_0)$ and then solve to get $C = f(t_0)/e^{kt_0}$. This
  type of specification is termed an {\em initial value specification}
  and a problem with such a specification is termed an {\em initial
  value problem}.
\item Another way we can determine $C$ and $k$ is if we are given the
  value of $f$ at two points $t_1$ and $t_2$. In this case, we solve
  to obtain that $k = \frac{1}{t_2 - t_1} \ln[f(t_2)/f(t_1)]$, and we
  can plug back to get $C = f(t_1)/e^{kt_1}$. Note that this is a
  reformulation of the formula that the time taken to multiply by a
  factor of $q$ is $(\ln(q))/k$. This kind of specification is
  somewhat related to the notion of {\em boundary value
  specification}.
\item But in many real-world situations, we do not need to actually
  determine the constants $C$ and $k$. Rather, we use the fact that
  $\ln(q_1)/t_{d1} =\ln(q_2)/t_{d2}$ to compare the rates of growth in
  two intervals.
\item Even nicer, in many cases, we do not need to know the actual
  values of $\ln(q_1)$ and $\ln(q_2)$, because the only thing that
  matters is their {\em quotient} $\ln(q_2)/\ln(q_1)$, which can also
  be viewed as the relative logarithm $\log_{q_1}(q_2)$. Thus, for
  instance, if $q_2$ is a rational power of $q_1$, we know the
  quotient precisely even though we may not know $\ln(q_1)$ and
  $\ln(q_2)$. For instance, $\ln(8)/\ln(4) = 3/2$.
\end{enumerate}

\subsection{Compound interest}

\begin{enumerate}
\item Compound interest: This is written as $A(t) = A_0 e^{rt}$ where
  $r$ is the {\em continuously compounded interest rate}, $A_0$ is the
  initial principal or initial amount, and $A(t)$ is the amount at
  time $t$. The corresponding differential equation is $A'(t) =
  rA(t)$. Continuously compounded interest differs from simple interest,
  where $A(t) = A_0(1 + rt)$ and from discretely compounded interest,
  where the interest earned is added to the principal at periodic
  intervals.
\item The time taken for the amount to double under continuously
  compounded interest with rate $r$ is $(\ln 2)/r$, which is
  approximately $0.7/r$. When $r$ is expressed as a percentage, we
  need to divide $70$ by that percentage to get the doubling
  time. (Times here are typically measured in years). This is called
  the {\em rule of 70}. {\em Note}: The rule of 70 also applies to
  discretely compounded interest rates when $r$ is very small, but
  that is a topic for next quarter.
\end{enumerate}


\subsection{Radioactive decay}

\begin{enumerate}
\item A radioactive material undergoes {\em decay}, i.e., its quantity
  goes {\em down} with time. The constant $k$ in the expression
  $Ce^{kt}$ is thus a negative number.
\item The fraction that remains is $1$ minus the fraction that
  decays. Thus, if $1/3$ of the material decays, then the relevant $q$
  to plug into formulas is $q = 2/3$, {\em not} $1/3$.
\item The rate of decay of radioactive materials is typically measured
  by their half-life, which is the time taken for half the material to
  decay and half to remain. ($1/2$ is the only number that is equal to
  $1$ minus itself). We have the formula $k = (-\ln
  2)/\text{half-life}$.
\end{enumerate}

\section{Discrete exponential growth}

\subsection{Multiplication by division}

Suppose you know that a given unicellular organism divides into two
organisms precisely one hour after its ``birth''. This kind of
reproduction is called {\em binary fission}, and is observed, for
instance, in the amoeba. Suppose that, at time $0$, God (or Craig
Venter!) creates $N$ such unicellular organisms. Then, one hour down
the line, there will be $2N$ unicellular organisms (because each one
splits into two at that time). One more hour down the line, there will
be $4N$ organisms. More generally, $k$ hours from the time God/Venter
created the first bunch of organisms, the number of organisms will be
$2^kN$.

This is a kind of exponential growth, and it is an example of {\em
discrete} exponential growth. Incidentally, we see here that the base
of the exponent is $2$, because the discrete action is a division into
two organisms. Similarly, if instead we had {\em ternary fission},
where an organism splits into three distinct organisms, the natural
base of the exponent would be $3$.

As a more complicated example, suppose, in a society, every pair of
parents chooses to have three children and then dies. Then, with every
passage from one generation to the next, the population increases by a
factor of $3/2$. Thus, after $k$ generations, the new total population
is the original population times a factor of $(3/2)^k$. \footnote{We
are assuming here that in each generation, there is an exact balance
between males and females, that they all get paired into couples, and
that they are all able to fulfil their desired fertility, and that
there are no such complications as divorces, out-of-wedlock children, etc.}

The hallmark of such discrete processes, we see, is exponentiation,
where the base of exponentiation is a positive integer or, at worst, a
rational number (usually something simple such as $2$, $3$, or $3/2$),
and the exponent is also a positive integer.

\subsection{Approximating by continuous processes}

Even though a given unicellular organism may split in two after a
precise one hour, a huge collection of unicellular organisms may be
increasing continuously in population. Why? Because their births may
not all have been synchronized. Similarly, when dealing with human
populations, even though no single woman continuously gives birth to
children, births are happening across the world to women. There are
multiple issues:

\begin{enumerate}
\item Lack of synchronizations of starting times.
\item Tempo and quantity difference: Some amoeba may tend to split
  slightly faster than others. Some women may choose to have kids
  earlier than others, and some may choose to have more.
\item Other factors: Some amoeba may burst and die and hence not be
  able to undergo binary fission. Some women may die before giving
  birth to children, or may choose not to have children, or may be
  unable to have children.
\end{enumerate}

When all these factors are taken into account, it may turn out that
using discrete rates and natural numbers creates a model that is too
complicated for suitable analyses. A simple {\em continuous model} may
thus be more tractable. Note that {\em the use of continuous models to
track human or amoeba populations is only an approximation, and a
shaky one}. It works only when we are working with large and uniform
samples. But before discussing the drawbacks of continuous
approximations, let us appreciate their virtue.

\section{Continuous growth: an introduction}

\subsection{Proportional growth and exponential growth}

Recall that if $f(t) := Ce^{kt}$, then $f'(t) = Cke^{kt} = kf(t)$. In
other words, the derivative of an exponential function is the
coefficient in the exponent times the function itself. The fascinating
thing is that the {\em converse also holds}, i.e., the only solutions
to the functional equation $f'(t) = kf(t)$ are function of the form
$f(t) = Ce^{kt}$. For this lecture, we use the letter $t$ for the
domain variable because we are studying situations where the input
parameter is a time parameter.

This is an example of a {\em linear differential equation} (in fact,
it's a very special case). We will skip over the proof for now. It is
there in the book. In any case, we will return to this idea later in
greater generality when covering linear differential equations. 

The value $C$ can be determined as the value $f(0)$. In other words,
given the values $f(0)$ and the factor of proportionality $k$, the
function $f$ is uniquely determined. Specifying $f$ in such a way
corresponds to an {\em initial value problem} -- i.e., a problem where
a law governing the derivative is combined with the value of the
function at one point.

\subsection{Proportional growth laws}

We now explore the possible justifications for proportional growth
laws, i.e., laws where the growth rate is proportional to the current
quantity. One justification was encountered from the discrete
situation of reproduction rates: reproduction with a fixed number of
offspring after a fixed time period gives discrete exponential laws,
but various averaging effects may allow us to approximate them by
continuous proportional growth laws. In other words, it may be
reasonable to assume, at least in the short run, that if $P$ denotes
the population function, then $P$ is of the form:

$$P'(t) = kP(t)$$

which gives:

$$P(t) := P(0)e^{kt}$$

where $k$ is an exogenous parameter\footnote{``exogenous'' is jargon
for a parameter whose value is determined by things that are outside
of the current model} that measures the ratio of the instantaneous rate
of change in the population to the total population. Note that the
assumption here is that $k$ itself does not change with time. Is the
assumption reasonable?

For a colony of bacteria or amoebae growing in a large and sustaining
environment with no binding resource constraints, there may be so few
premature deaths and so little fighting for resources that the growth
rate remains constant even as the population grows. However, once the
population becomes very large, and the available resources in the
region start getting strained, the growth rate may fall -- or even
become negative. More generally, the growth rate may be constant or
near constant for a long time and then start a decline.

In the case of humans, the greater degree of human agency and the
existence of complex societal influences complicate matters. For
instance, the two main factors that affect world population growth
rates are {\em birth rates} and {\em death rates}. Within a subregion
of the world that is not closed to the outside world, {\em migration
rates} also play a role. Also, birth rates are determined by such
factors as {\em desired fertility for females}, {\em proportion of the
population comprising females of birth-giving ages}, while death rates
are also affected by population composition, access to nutrition,
sanitation, and health care, incidence of wars and tribal in-fighting,
and natural disasters. All the aforementioned factors exhibit not only
random fluctuations but systemic changes from one generation to the
next. To take an example, until the nineteenth century, European
nations lost significant fractions of their male populations in wars
-- and these fractions varied from generation to generation in each
country depending on the war/peace situation. So the idea of a
constant $k$ is nothing more than a polite fiction.

Nonetheless, one idea remains important: whatever role exogeneous
parameters play is played out through their effect on $k$, and the
effect is a proportional effect. If country $A$ and country $B$ are
similar in all relevant demographic characteristics (hence have the
same $k$) except that country $A$ has twice the population of country
$B$, the rate of population growth in country $A$ will be twice that
of country $B$.

Exponential growth is also sometimes called {\em geometric growth}
because if you measure population at discrete intervals and look at
the time series, you get a {\em geometric progression}. Geometric or
exponential growth is faster than {\em linear growth} or {\em
arithmetic growth}, where the rate of growth is constant and does not
depend on the current level. This is basically the fact that $e^x$
eventually grows faster and way way faster than any linear function.

Malthus was worried about the idea that population grows geometrically
while resource supplies will grow at a linear rate (what he called
{\em arithmetic growth}) ultimately leading to a severely
resource-constrained world with mass starvation and the end of human
civlization as we know it. Although Malthus's pessimistic predictions
did not come to bear in the human realm, the interplay between the
natural tendency for exponential growth and the finiteness of various
kinds of resource constraints is a theme that recurs in understanding
many ecological and biological phenomena. We may return to some of
these topics at a later stage.

\section{More arithmetic and computation with exponential growth}

\subsection{Using two observations to determine growth rates}

Suppose we have strong reason to believe that a particular growth is
exponential (based on our theoretical model for how such growth
occurs) but we do not have any theoretical way of determining the {\em
constant of proportionality} $k$. Then, what we do is to use {\em two
observation points}. Basically, we know that the function $f$ is of
the form:

$$f(t) := Ce^{kt}$$

where both $C$ and $k$ are unknown constants. Given the value of $f$
at two points, we can find both $C$ and $k$. Specifically, if $f(t_1)
= a_1$ and $f(t_2) = a_2$, then:

$$\frac{f(t_2)}{f(t_1)} = e^{k(t_2 - t_1)}$$

Taking logarithms both sides, we obtain:

$$k = \frac{1}{t_2 - t_1} \ln\left[\frac{f(t_2)}{f(t_1)}\right]$$

Once we know $k$ we can determine $C$.

Sometimes, there is no known reference point from which to measure
times. In this case, we can simply pick the first observation as the
time point $t = 0$ and measure times forward from there.

\subsection{Time taken to multiply by a fixed proportion}

For a growth function $f(t) := Ce^{kt}$, the time taken to multiply by
a factor of $q$ is given by:

$$\frac{\ln (q)}{k}$$

This is just a reformulation of the result derived previously.

In particular, the time taken to multiply by a fixed proportion is
independent of the original value. Thus, if the quantity doubles from
time $t = 1$ to time $t = 6$, it also doubles from time $t = 13$ to $t
= 18$. What matters is the {\em length of time interval}. 

\subsection{Exponential decay}

\includegraphics[width=3in]{exponentialdecay.png}

Mathematically, exponential decay is just like exponential growth. The
main difference is that in the case of growth, the constant of
proportionality $k$ is positive, so the function grows. With decay,
the constant of proportionality is negative. We now call $k$ the {\em
decay constant} and it is negative. A function undergoing exponential
decay has the $t$-axis (what we usually call the $x$-axis, but we're
dealing with functions of time) as a horizontal asymtptote, i.e., it
goes to zero as $t \to \infty$.

With exponential growth, the limit $\lim_{t \to \infty} f(t) =
\infty$. In a finite resource-constrained world, exponential growth
must stop at some stage simply because we run out of resources. With
exponential decay, the limit is zero. Thus, exponential decay will not
stop due to resource constraints.

\subsection{Taking logarithms}

A useful way of studying exponential growth with time is to plot the
graph of the {\em logarithm} of the function in terms of time. Some observations:

\begin{itemize}
\item The logarithm of an exponential function is a linear
  function. Specifically, the logarithm of $Ce^{kt}$ is $kt + \ln C$.
\item In particular, the {\em slope} of the linear function thus
  obtained is the growth rate $k$ and the {\em intercept} is the logarithm of the initial value.
\item If $k > 0$, we have growth, which corresponds to a positive
  slope or increasing linear function.
\item If $k < 0$, we have decay, which corresponds to a negative slope
  or linear decreasing function.
\item The fact that we can use two observations to find $C$ and $k$
  corresponds to the fact that after taking logarithms, knowing two
  points on a line determines the line.
\item In particular, given {\em more than two} observations, one way
  to {\em test} whether they do fit a pattern of exponential
  growth/decay is to take logarithms and test whether the observation
  points thus obtained are collinear.
\end{itemize}

Some of you may recall the concept of {\em logarithmic
differentiation}. The logarithmic derivative of a function $f$ is
defined as the derivative of $\ln|f|$, and is also defined as:

$$\frac{f'(x)}{f(x)}$$

In fact, exponential growth corresponds {\em precisely} to the
situation where the logarithmic derivative is a {\em constant
function}, or equivalently, the derivative is {\em proportional} to
the original function.

\subsection{Comparative analysis with different growth rates}

\begin{itemize}
\item If two quantities are growing exponentially with the {\em same
  exponential growth rate}, the quantity that starts out bigger stays
  bigger. In fact, the ratio of the two quantities remains constant as
  a function of time. Taking logarithms, we get a pair of parallel
  lines. Here's how the picture looks before taking logarithms:

  \includegraphics[width=3in]{samerateexpgrowth.png}
\item If two quantities are growing exponentially, the quantity with
  the faster growth rate, if {\em originally bigger, stays bigger},
  and if {\em originally smaller, overtakes the other quantity just
  once and after that stays bigger}. Below is a pictorial example
  where they start at the same point initially and the one with the
  faster growth rate becomes bigger:

  \includegraphics[width=3in]{samestartexpgrowth.png}
\end{itemize}

\section{Real world examples}

\subsection{Radioactive decay}

The standard example of exponential decay is radioactive decay. With
radioactive decay, the rate at which decay occurs is proportional to
the amount of undecayed stuff. Here, instead of doubling, we foresee
halving. A common measure of how quickly radioactive stuff decays is
given by its {\em half-life}, i.e., the amount of time taken for it to
become half of its original value. By the formula above, the half-life
is related to $k$ by the formula:

$$k = \frac{- \ln 2}{\text{Half-life}}$$

Again, the amount of time taken to decay by a factor of $q$ is
independent of the original mass.

A little further note on radioactive decay. At heart, radioactive
decay is a probabilistic process. The correct model to keep in mind
for radioactive decay is that any given nucleus has a fixed
probability of decaying per unit time. {\em This probability does not
depend on the number of other nuclei or how many nuclei have decayed
so far.} It is {\em not the case} that as less and less of the
substance is left, individual nuclei choose to decay slower and
slower. Rather, since the {\em number of nuclei left} is fewer, the
overall rate of decay declines.

Although radioactive decay is probabilistic, the {\em very large}
number of nuclei in a given sample makes the macroscopic measurements
qusai-deterministic. This is again a phenomenon that will appear
repeatedly: {\em randomness at the individual level appears
deterministic at the aggregate level.}

\subsection{Slowing down}

This is an important feature of exponential decay. With exponential
decay, the rate of decay is proportional to the amount that is
decaying. This exponential decay is {\em slower} than linear
decay. Thus, the time taken for $2/3$ of a material to decay is {\em
more than twice} the time taken for $1/3$ of the material taken to
decay. In other words, the time taken for the first $1/3$ of the
material to decay is less than the time taken for the next $1/3$ to
decay, because the next $1/3$ forms a larger fraction of the material
left over after $1/3$ has already decayed.

\subsection{Dating fossils}

Here, {\em dating} refers to determining how old a fossil is. The most
common tool for dating is carbon-14 dating. The idea is as follows:
the ratio of the unstable carbon-14 isotopes (6 protons, 8 neutrons)
to the stable carbon-12 isotope (6 protons, 6 neutrons) in most living
organisms is almost fixed. We assume that this ratio was the same in
the prehistoric time when the fossil lived. After death and
fossilization, there was no exchange of carbon with the surroundings,
so the carbon-12 remained the same, but carbon-14 underwent
radioactive decay, turning into the stable nitrogen-14. By determining
the current ratio in the fossil, it is possible to determine (under
all these assumptions) when the corresponding living organism lived.

The half-life of C-14 is about 5730 years, and carbon-14 dating has
been used to claim that some fossils are millions of years
old.\footnote{If the Young Earth Creationist claim that the earth is
only a few thousand years old is true, then at least one of the
assumptions/claims made in the preceding two paragraphs must be false.}

\subsection{Continuously compounded interest rate}

{\em Compound interest} is a form of interest earned where past
interest earned is added to the principal for future interest
computations. Compound interest as given by banks is usually done on a
discrete basis: the interest may be added to the principal every
month, in which case we say that the interest is {\em compounded
monthly}. Note that the same annual interest rate gives a higher {\em
effective} annual interest rate if the compounding is done more
frequently.

For instance, an annual interest rate of 100\% compounded every six
months effectively gives an annual interest rate of 125\%, because
after the first six months, the principal becomes $1.5$ times its
original value, and in the next six months, it becomes $1.5$ times
that value again, thus becoming $2.25$ times its original value.

{\em Continuously compounded interest} means that the interest is
compounded continuously. if the interest rate is $r$ (expressed as a
raw number, not a percentage), this means that the rate of change of
the principal is $r$ times the principal. We thus get exponential growth:

$$A = A_0e^{rt}$$

where $A_0$ is the initial principal and $A$ is the total amomunt
accumulated after time $t$.

Also, note that the doubling time for continuously compounded interest
is given by:

$$\frac{\ln 2}{r} \approx \frac{0.7}{r}$$

When the interest rate is expressed as a percentage, we get that the
doubling time is $70$ divided by the interest rate. This is the famous
rule of 70. Note that our derivation of this formula assumes {\em
continuously compounded interest}, which is not usually the way
interest rates are specified or calculated. To show that this formula
is also reasonably valid for interest rates that are compounded
annually, we need a further approximation result that we shall see
later in the course.

\subsection{Interest rates, present value, and discount rates}

[May not get time to cover this in class.]

When we say that ``the present value of such-and-such three years from
now is this much'' what we mean is, roughly, that in order to have
such-and-such three years from now, we need to invest this much. For
instance, at an interest rate of $10\%$ per annum (compounded
annually), the present value of 1100 dollars a year from now.

The concept of present value is used in more complicated contexts as
well. For instance, what is the present value of 1200 dollars next
year {\em plus} 1100 dollars two years from now? With an interest rate
of $10\%$ per annum compounded annually, the answer is 2000
dollars. If we start with 2000 dollars, this becomes 2200 dollars
after a year. Withdrawing 1200 of these dollars leaves 1000 dollars,
which in turn becomes 1100 dollars the year after that. This kind of
concept is useful if you are planning on drawing upon savings to pay
for your living after retirement.

Interest rates occur in a different guise as {\em discount rates}. For
instance, a discount rate of $1\%$ per annum means that we value the
same thing a year from now at only $99\%$ of how much we value it
right now. With a discount rate of $1\%$, you would be indifferent
between receiving a hundred dollars a year from now and $99$ dollars
today. A discount rate is related to an interest rate, because if your
money can earn $1\%$ interest in a year by being placed in the bank,
then that explains your $1\%$ discount rate.\footnote{There is
technical issue here, which is that discount rates are measured
negatively and interest rates are measured positively. For small
discount rates/interest rates, this is not a big issue in the discrete
situation. In the continuous version, it is a complete non-issue. This
subtlety will be discussed later.}

The concept of discounting is also used in policy analysis. When
comparing different ``costs'' and ``benefits'' to society that occur
over a long period of time, it is customary to specify a discount rate
used for making an overall judgment. A zero discount rate would mean
that a poicy that saves one life today is no more or less preferable
than an identical policy that saves one life five years from now. A
higher discount rate would mean that the former policy is preferable.

\subsection{Predicting the future: noise}

[May not get time to cover this in class.]

In situations such as radioactive decay, there are strong theoretical
grounds for the claim that the decay is, at the macroscopic level, an
exponential decay. The situation is less clear in other
cases. However, in many cases, there are reasonable grounds for a
claim that growth is approximately exponential.

However, the {\em secular exponential trend} needs to often be sorted
out from {\em periodic seasonal fluctuation} and {\em random
fluctuation}. In the context of US retail sales, if sales at a store
go up by a factor of $3$ from October to December, that does not mean
that they will triple every two months to reach $729$ times their
original volume by next October. Part of the explanation for the jump
may be seasonal patterns in consumer shopping (spurred on by discount
sales).

There are a number of ways to tease out the secular trends from the
seasonal trends and random fluctuations. We will not go into these in
detail, but here are some obvious things:

\begin{itemize}
\item Compare apples to apples: Choose the same day of the week and
  the same time of the year as far as possible. This helps overcome
  seasonal trends.
\item Use {\em moving averages} (for instance, average over the last
  364 days) rather than single data points. Take the base of the
  moving average as the ``period'' for any periodic trend, e.g., 364
  days works well because it is almost a year and is also a multiple
  of $7$. This handles both seasonal trend issues and random
  fluctuations.
\item When taking two data points to determine the rate of growth, do
  not take points that are spaced too closely, because even a small
  fluctuation can lead to a spurious high growth rate.
\end{itemize}

A deeper understanding of the methods used would require us to go into
statistical methods, which we are not equipped for.
\end{document}