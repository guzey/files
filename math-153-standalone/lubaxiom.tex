\documentclass{amsart}
\usepackage{fullpage,hyperref,vipul}
\title{The least upper bound axiom}
\author{Math 153, Section 55 (Vipul Naik)}

\begin{document}
\maketitle

{\bf Corresponding material in the book}: Section 11.1.

{\bf What students should already know}: The intuitive idea of the
real line, basic algebra and calculus as done in the course up to this
point for motivation/background.

{\bf What students should definitely get}: The definitions of least
upper bound and greatest lower bound, the definition of boundedness,
the statements about the existence of these in the real
numbers. Computation of these in special cases.

\section*{Executive summary}

Words ...

\begin{enumerate}
\item The real numbers satisfy the {\em least upper bound property}:
  any nonempty subset of the set of real numbers that is bounded from
  above has a least upper bound. This property does {\em not} hold if
  we replace the real numbers by the rational numbers.
\item The real numbers satisfy the {\em greatest lower bound
  property}: any nonempty subset of the set of real numbers that is
  bounded from below has a greatest lower bound. This property again
  does {\em not} hold if we replace the real numbers by the rational
  numbers.
\item We can prove the greatest lower bound property using the least
  upper bound property. There are two proofs of this. One of these
  proofs involve {\em reflection}: replacing a set by its set of
  negatives. The other proof, which is there in the book, is also
  worth going through. Please go through it. I'll go through it in
  review session. You will not be asked the proof in the test, but it
  may be helpful for multiple choice questions and other conceptually
  based problems.
\item The natural numbers satisfy a property that is somewhat similar
  to the greatest lower bound property for the reals, but stronger:
  any nonempty subset of the set of natural numbers has a least
  element. This is equivalent to the {\em principle of mathematical
  induction}.
\item If a nonempty subset of the real numbers has a {\em maximum}
  element, then that element is also the least upper bound of the
  set. Conversely, if the least upper bound of a set is in the set,
  then that is also the maximum element of the set.
\item If a nonempty subset of the real numbers has a {\em minimum}
  element, then that element is also the greatest lower bound of the
  set. Conversely, if the greatest lower bound of a set is in the set,
  then that is also the minimum element of the set.
\item A nonempty finite subset always has a maximum and a minimum
  element. Thus, its greatest lower bound and least upper bound are
  both in the set.
\item For an interval with lower endpoint $a$ and upper endpoint $b$,
  the least upper bound is $b$ and the greatest lower bound is
  $a$. Note that this holds for all the four possibilities for the
  interval: $[a,b]$, $(a,b)$, $[a,b)$, and $(a,b]$.
\item If $T$ is a nonempty subset of a nonempty bounded subset $S$ of
  $\R$, any lower bound for $S$ remains a lower bound for $T$ and any
  upper bound for $S$ remains an upper bound for $T$. However, we may
  have an upper bound for $T$ that is {\em not} an upper bound for
  $S$. Similarly, we may have a lower bound for $T$ that is {\em not}
  a lower bound for $S$. Thus, the least upper bound for $T$ is $\le$
  the least upper bound for $S$, and the greatest lower bound for $T$
  is $\ge$ the greatest lower bound for $S$.
\item A set does {\em not} have an upper bound if and only if it has
  arbitrarily large elements. Similarly, a set does {\em not} have a
  lower bound if and only if it has arbitrarily small elements (i.e.,
  negative elements of arbitrarily large magnitude).
\item If $M$ is the least upper bound of a nonempty subset $S$ of
  $\R$, then, for every $\epsilon > 0$, $S$ has a nonempty
  intersection with the interval $(M - \epsilon,M]$. In particular, if
  $M \notin S$, then $S$ has a nonempty intersection with the interval
  $(M - \epsilon, M)$. (See also the analogous theorem for greatest
  lower bounds, which is Theorem 11.1.4 in the book).
\end{enumerate}

Actions ...

\begin{enumerate}
\item To compute the greatest lower bound and least upper bound of a
  set, we first need to compute the set. Finding the set as a union of
  intervals is often useful.
\item Given a set $S$, we can construct corresponding sets such as $S
  + \lambda$ (translation), $-S$ (reflection about $0$), $f(S)$ (image
  of $S$ under a function $f$), and $\operatorname{abs}(S)$ (the set
  of absolute values of elements of $S$, i.e., folding about
  $0$). Please review the results that relate bounds for $S$ with
  bounds on these corresponding sets.
\end{enumerate}

\section{Ordering, boundedness, and related properties}

\subsection{A total ordering}

The real numbers, as we have come to understand them, are a {\em
totally ordered set}. This means that for any two real numbers $a$ and
$b$, one of three possibilities holds:

\begin{enumerate}
\item $a < b$: Equivalent formulations are $b > a$, $a - b < 0$, and
  $b - a > 0$. In words, we say that $a$ is {\em less than} or {\em
  smaller than} $b$.
\item $a = b$: Equivalent formulations are $b = a$, $a - b = 0$, and
  $b - a = 0$.
\item $a > b$: Equivalent formulations are $b < a$, $b - a < 0$, and
  $a - b > 0$. In words, we say that $a$ is {\em greater than} or {\em
  bigger than} $b$.
\end{enumerate}

We also know that if $a < b$ and $b < c$, then $a < c$.

Any two real numbers are {\em comparable}. We can say for sure that
one real number is bigger or smaller than another. In other words, the
real numbers can be arranged in increasing order, since they are
arranged on a line. This {\em total ordering} is something very
specific to the real numbers and subsets thereof. It breaks down for
more complicated mathematical structures, such as the complex numbers,
vector spaces over the real numbers, or polynomials. In many of those
cases, we can artificially impose an ordering, but it does not behave
in a nice and familiar way.

\subsection{The reals, the rationals, and the integers}

There are many subsets of the reals with qualitatively different
behavior, but we concentrate on four important subsets that, when
studied together, shed light on many of the issues of interest.

\begin{enumerate}
\item The {\em positive integers} or {\em natural numbers}, denoted
  $\N$. These start at $1$, and go like $1,2,3,\dots$. A very
  important feature of the positive integers is that {\em every
  nonempty subset of the positive integers has a smallest
  element}. This can be reformulated as the {\em principle of
  mathematical induction}: if a statement is true for $1$ and its
  truth for $k$ implies its truth for $k + 1$, then it is true for all
  positive integers.
\item The {\em integers}, denoted $\Z$. Unlike the positive integers,
  they do not satisfy the property that every nonempty subset has a
  least element. The integers stretch to infinity in both
  directions. However, just like the positive integers, they are {\em
  discrete} -- for every integer, there is a unique {\em successor} (a
  unique smallest integer that's the {\em next} integer) and a unique
  {\em predecessor} (a unique largest integer that's the {\em
  previous} integer).
\item The {\em rational numbers}, denoted $\Q$. These differ from the
  integers in the following important respect: they have a {\em
  density property}. Between any two rational numbers, there is yet
  another rational number. And this process can be repeated {\em ad
  infinitum}, so between any two rational numbers, there are {\em
  infinitely many} rational numbers. Although the rational numbers are
  {\em dense}, they are not {\em complete} -- there are holes. We can
  have a bunch of rational numbers that seem to be heading to some
  specific number that turns out to be irrational.
\item The {\em real numbers}, denoted $\R$. These include the rational
  numbers and more numbers that can be arrived at by taking limits of
  rational numbers. The set of real numbers is {\em complete} in the
  sense that it has no holes. If a sequence of real numbers is headed
  somewhere finite, it is headed towards a real number.
\end{enumerate}

We would like to make precise the notion of completeness of the real
numbers, i.e., the idea that the real numbers don't have holes. There
are many different (and equivalent) ways of doing this. The one we
will follow is the {\em least upper bound axiom}, which we discuss next.

\subsection*{Aside: Sequences and sizes of infinite sets}

A {\em sequence} of real numbers is defined as a function from $\N$ to
$\R$. The sequence can be written as a {\em list}:

$$f(1), f(2), \dots, f(n), \dots$$

Conversely, any list can be thought of as a function:

$$a_1, a_2, \dots, a_n, \dots$$

where the function is defined as $f(n) = a_n$.

An infinite subset of the reals is {\em countable} if there is a
sequence of real numbers that includes all elements in the subset. In
other words, we can count, or list, all elements of the subset. It
turns out that:

\begin{enumerate}
\item The set of natural numbers (positive integers) is countable. We
  can just use the sequence $1,2,3,\dots$, which corresponds to the
  identity function on $\N$.
\item The set of integers is countable. We can use the sequence
  $0,1,-1,2,-2,3,-3,\dots$. Note that this sequence involves {\em
  alternating} between the positive and negative numbers, so it does
  not preserve the usual ordering.
\item The set of rational numbers is countable. The way we list the
  rational numbers does not preserve the usual ordering, and involves
  a snake-like path.
\item The set of real numbers is {\em not} countable. In other words,
  there is no way to list {\em all} real numbers.
\end{enumerate}

Thus, in this sense, even though all the four sets $\N$, $\Z$, $\Q$,
and $\R$ are infinite, $\R$ is a bigger set than the other three. Note
that although $\N \subset \Z \subset \Q$, the ostensibly bigger and
much more dense set $\Q$ is {\em not} bigger in terms of number of
elements than the ostensibly smaller $\N$.

\subsection{Boundedness, upper bounds, and lower bounds}

Suppose $S$ is a nonempty subset of the real numbers. An element $a
\in \R$ is termed an {\em upper bound} for $S$ if $x \le a$ for all $x
\in S$. In particular, we allow the upper bound to be in the set. A
nonempty subset of $\R$ that has an upper bound is said to be {\em
bounded from above}.

An element $b \in \R$ is a {\em lower bound} for $S$ if $b \le x$ for
all $x \in S$. In particular, we allow the lower bound to be in the
set. A nonempty subset of $\R$ that has a lower bound is said to be
{\em bounded from below}.

A nonempty subset of $\R$ is said to be {\em bounded} if it is both
bounded from above and bounded from below. Some comments:

\begin{enumerate}
\item $a$ is an upper bound for $S$ if and only if $S \subseteq
  (-\infty,a]$.
\item $b$ is a lower bound for $S$ if and only if $S \subseteq [b,\infty)$.
\item $a$ is an upper bound for $S$ and $b$ is a lower bound for $S$
  if and only if $S \subseteq [b,a]$.
\end{enumerate}

\subsection{Least upper bound and greatest lower bound}

Suppose $S$ is a nonempty subset. An element $a \in \R$ is termed a
{\em least upper bound} for $S$ if $a$ is an upper bound for $S$ and,
for any upper bound $a'$ of $S$, $a \le a'$. In other words, $a$ is
the {\em least} possible element we can choose as an upper bound.

We can make two petty observations without much thought:

\begin{enumerate}
\item For a least upper bound to exist, the set must be bounded from
  above.
\item If $a$ and $a'$ are both least upper bounds for a nonempty
  subset $S$, then $a = a'$. This is because each one is less than or
  equal to the other.
\end{enumerate}

Now, whatever we have said so far applies inside the real numbers, but
it also applies inside the rational numbers, inside the integers, and
inside the positive integers. In fact, it applies more abstractly
inside any totally ordered set. However, what we are going to say {\em
now} is something very specific to the real numbers, in so far as it
captures a {\em completeness} property of the real numbers. This says that:

\begin{quote}
  Any nonempty subset of the real numbers that is bounded from above
  has a least upper bound {\em which is also a real number}.
\end{quote}

As already mentioned in the two points made above, the ``bounded from
above'' condition is clearly {\em necessary}, and also, the least
upper bound must be {\em unique}. Thus, for a subset $S$ of $\R$ that
is bounded from above, we denote the least upper bound of $S$ by
$\operatorname{lub}(S)$.

In a similar vein, we can define the notion of {\em greatest lower
bound}, and we have the following:

\begin{quote}
  Any nonempty subset of the real numbers that is bounded from below
  has a greatest lower bound.
\end{quote}

Similar to the previous case, the ``bounded from below'' condition is
{\em necessary}, and the greatest lower bound must be {\em unique}. We
denote this by $\operatorname{glb}(S)$.

\subsection{Pedestrian observations}

\begin{enumerate}
\item The following are equivalent for a nonempty subset $S$ of $\R$:
  \begin{enumerate}
  \item $S$ has a {\em maximum} element -- an element that is larger
    than every other element of $S$.
  \item The least upper bound of $S$ is in $S$.
  \end{enumerate}
  Moreover, if these equivalent conditions hold, the maximum element
  equals the least upper bound.
\item The following are equivalent for a nonempty subset $S$ of $\R$:
  \begin{enumerate}
  \item $S$ has a {\em minimum} element -- an element that is smaller
    than every other element of $S$.
  \item The greatest lower bound of $S$ is in $S$.
  \end{enumerate}
  Moreover, if these equivalent conditions hold, the minimum element
  equals the greatest lower bound.
\item A finite set has a maximum element and a minimum element. Thus,
  any finite set contains its least upper bound and greatest lower
  bound.
\item For an interval with lower endpoint $a$ and upper endpoint $b$,
  the least upper bound is $b$ and the greatest lower bound is
  $a$. Note that this holds for all the four possibilities for the
  interval: $[a,b]$, $(a,b)$, $[a,b)$, and $(a,b]$. The key
  distinction between the closed and open situation is not in the
  value of the upper/lower bound but in {\em whether that value is
  contained in the set we start with}.
\item If $T$ is a nonempty subset of a nonempty bounded subset $S$ of
  $\R$, any lower bound for $S$ remains a lower bound for $T$ and any
  upper bound for $S$ remains an upper bound for $T$. However, we may
  have an upper bound for $T$ that is {\em not} an upper bound for
  $S$. Similarly, we may have a lower bound for $T$ that is {\em not}
  a lower bound for $S$. Thus, the least upper bound for $T$ is $\le$
  the least upper bound for $S$, and the greatest lower bound for $T$
  is $\ge$ the greatest lower bound for $S$.
\item A (set does {\em not} have an upper bound) if and only if (it has
  arbitrarily large elements). Similarly, a set (does {\em not} have a
  lower bound) if and only if (it has arbitrarily small elements (i.e.,
  negative elements of arbitrarily large magnitude)).
\end{enumerate}

\subsection{Boundedness of a set and of related sets}

Given a nonempty set $S \subseteq \R$, we can construct the following sets:

\begin{enumerate}
\item {\em Translation}: $S + \lambda$, which is defined as $\{ s +
  \lambda : s \in S \}$: $a$ is an upper (respectively lower) bound
  for $S$ if and only if $a + \lambda$ is an upper (respectively
  lower) bound for $S + \lambda$. $S$ is bounded from above
  (respectively below) if and only if $S + \lambda$ is. The least
  upper bound and greatest lower bound both get translated to the
  right by $\lambda$. In other words, the least upper bound of $S +
  \lambda$ is $\lambda$ plus the least upper bound of $S$, and the
  greatest lower bound of $S + \lambda$ is $\lambda$ plus the greatest
  lower bound of $S$.
\item {\em Reflection about $0$}: $-S$, which is the set $\{ -s : s
  \in S \}$. $a$ is an upper (respectively lower) bound for $S$ if and
  only if $-a$ is a lower (respectively upper) bound for $S$. $S$ is
  bounded from above (respectively below) if and only if $-S$ is
  bounded from below (respectively above). The least upper bound for
  $-S$ is the negative of the greatest lower bound for $S$, and the
  greatest lower bound for $-S$ is the negative of the least upper
  bound for $S$ (these statements are subject to existence).
\item {Folding about $0$}: The set $\operatorname{abs}(S) = \{ |s| : s
  \in S \}$. Although this could be denoted $|S|$, that notation is
  often used for the {\em size} (or number of elements) of $S$, so we
  will refrain from that notation. $S$ is bounded (i.e., bounded from
  both above and below) if and only if $\operatorname{abs}(S)$ is
  bounded from above. Note that $\operatorname{abs}(S)$ is {\em
  always} bounded from below by $0$. Moreover, the least upper bound
  of $\operatorname{abs}(S)$ is the maximum of the absolute values of
  the greatest lower bound and least upper bound of $S$.
\item {\em An increasing function}: Suppose $f:\R \to \R$ is a
  continuous increasing function and $f(S)$ is the image of $S$ under
  $f$. Then, the least upper bound of the image $f(S)$ is the image
  under $f$ of the least upper bound of $S$, while the greatest lower
  bound of the image $f(S)$ is the image under $f$ of the greatest
  lower bound of $S$.
\item {\em A decreasing function}: Suppose $f: \R \to \R$ is a
  continuous decreasing function and $f(S)$ is the image of $S$ under
  $f$. Then, the greatest lower bound of the image $f(S)$ is the image
  under $f$ of the least upper bound of $S$, while the least upper
  bound of the image $f(S)$ is the image under $f$ of the greatest
  lower bound of $S$.
\item The least upper bound (respectively greatest lower bound) of a
  set of rational numbers may be rational or irrational. Similarly,
  the least upper bound (respectively greatest lower bound) of a set
  of irrational numbers may be rational or irrational.
\end{enumerate}

\section{Some important results}

\subsection{There are numbers arbitrarily close up to the least upper bound}

The statement of the theorem is:

\begin{quote}
  If $M$ is the least upper bound of a nonempty subset $S$ of $\R$,
  then, for every $\varepsilon > 0$, $S$ has a nonempty intersection with
  the interval $(M - \varepsilon,M]$. In particular, if $M \notin S$,
  then $S$ has a nonempty intersection with the interval $(M -
  \varepsilon, M)$.
\end{quote}

The proof is straightforward. Suppse there exists $\varepsilon > 0$ such
that $S$ has empty intersection with the interval. Then, we can easily
check that $M - \varepsilon$ is {\em also} an upper bound for $S$, which
contradicts the minimality of $M$ as an upper bound for $S$.

A similar statement holds for the greatest lower bound.

Basically, what these results are saying is that for something to
qualify as the least upper bound or greatest lower bound, it either
must be in the set or there must be stuff arbitrarily close to it that
is in the set. It should not be possible to isolate it away from the
set that it claims to be the best bound on.

\subsection{Deriving the least upper bound and greatest lower bound results from each other}

In the book's presentation, the statement about the existence of a
least upper bound is stated as an {\em axiom} while the statement
about the existence of a greatest lower bound is stated as a {\em
theorem}, proved using the least upper bound axiom.

In reality, there is an elaborate construction procedure for the real
numbers, and the details of that construction procedure guarantee both
the least upper bound and the greatest lower bound
properties. However, since we are not going through the construction
procedure, we take a shortcut by treating the least upper bound
existence statement as a {\em given}. It turns out, though, that the
statement about the greatest lower bound can be deduced from it. There
are two ways of deducing it:

\begin{enumerate}
\item Reflection about $0$: This interchanges the role of upper and
  lower bounds. We can use the least upper bound axiom on $-S$ to
  deduce the greatest lower bound property for $S$.
\item The method as given in the book: The crux of this method is the
  following observations: (a) the greatest lower bound of a set is the
  least upper bound of the set of lower bounds of the set, (b) the set
  of lower bounds on a nonempty set $S$ is bounded from above by an
  element of $S$, hence is bounded from above. We use the least upper
  bound axiom to argue that the set of lower bounds of $S$ has a least
  upper bound, and then use (a) to show that this least upper bound
  {\em of the set of lower bounds} of $S$ is also the greatest lower
  bound for $S$. For a formal proof, refer to the book.
\end{enumerate}

The second approach may seem a little dense (in the non-mathematical
sense) -- why go for this kind of twisted logic when we can simply use
the reflection about $0$ proof?  There is a deep reason.

The proof that involves the use of reflection about $0$ is drawing on
the additive structure of the real numbers and the behavior of the
negation operation. Thus, this proof {\em does not generalize} to
arbitrary totally ordered sets. On the other hand, the second proof
uses {\em nothing special} about the real numbers and is equally valid
for any totally ordered set. In other words, the second proof is valid
over any totally ordered set, and says that a totally ordered set
satisfies the least upper bound property if and only if it satisfies
the greatest lower bound property.
\end{document}