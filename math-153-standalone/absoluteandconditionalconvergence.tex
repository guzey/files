\documentclass[10pt]{amsart}
\usepackage{fullpage,hyperref,vipul}
\title{Absolute and conditional convergence}
\author{Math 153, Section 55 (Vipul Naik)}

\begin{document}
\maketitle

{\bf Corresponding material in the book}: Section 12.5.

{\bf What students should definitely get}: The distinction between
absolute and conditional convergence for series with both positive and
negative terms. The fact that for a conditionally convergent series,
different rearrangements can produce different sums. The statement
about convergence of series with terms of alternating signs whose
magnitudes monotonically go to zero.

\section*{Executive summary}

Words ...

\begin{enumerate}
\item {\em Not for discussion}: When discussing the convergence of a
  series, we can throw out all the terms that are zero.
\item {\em Not for discussion}: A series $\sum a_k$ is termed {\em
  absolutely convergent} if the series $\sum |a_k|$ converges. Note
  that for a series of nonnegative terms, being absolutely convergent
  is equivalent to being convergent.
\item Suppose a series $\sum a_k$ is absolutely convergent. Then, the
  positive terms converge (say, to $P$) and the negative terms
  converge (say, to $N$). Moreover, $\sum a_k$ is the sum $P + N$, and
  $\sum |a_k| = |P| + |N|$.
\item If a series is absolutely convergent, then it is convergent and
  {\em every rearrangement} of the series converges to the same sum.
\item If a series is not absolutely convergent but is convergent, it
  is termed {\em conditionally convergent}, and the positive terms add
  up to $+\infty$, the negative terms add up to $-\infty$, and both
  the positive terms and the negative terms go to $0$.
\item The Riemann series rearrangement theorem states that a series
  that is conditionally convergent but not absolutely convergent can
  be rearranged to give any real number as its sum. It can also be
  rearranged to give a sum of $+\infty$ and it can also be rearranged
  to give a sum of $-\infty$. It can also be rearranged so that the
  sequence of partial sums oscillates between any two fixed
  locations. [Recall that you heard some non-symbolic, purely didactic
  reasoning for this in class. Please review this reasoning from the
  class notes.]
\item The alternating series theorem states that a series whose terms
  have alternating signs, and have magnitudes {\em monotonically}
  decreasing to zero, must converge. Moreover, the point to which the
  series converges is the least upper bound of the decreasing sequence
  of even-numbered partial sums and the greatest lower bound of the
  increasing sequence of odd-numbered partial sums. [Recall the
  interpretation of this in terms of jumping along the number line.]
\item (Questions: What happens if the magnitudes go to zero but not
  monotonically? What happens if the series is not alternating? What
  happens if the magnitudes decrease monotonically but not to zero?)
\end{enumerate}

\section{Absolute convergence and its consequences}

\subsection{Absolutely convergent series}

A series $\sum a_k$ (where the terms may be positive, negative, or
zero) is termed {\em absolutely convergent} if the series $\sum |a_k|$
converges.

\subsection{The number line and triangle inequalities}

Flash back to when you learned about addition on the number line. Let
us use that procedure to understand how we sum up series and what
happens when the terms have mixed signs.

Consider a series:

$$\sum_{k=1}^\infty a_k$$

We start off at $0$. We now move a distance of $a_1$. More
specifically, if $a_1$ is positive, we move a distance of $a_1$ to the
right. If $a_1$ is negative, we move a distance of $|a_1| = -a_1$ to
the left. if $a_1$ is zero, we stay put.

We are now at $a_1$. We look at $a_2$. If $a_2$ is positive, we move a
distance of $a_2$ to the right. If $a_2$ is negative, we move a
distance of $-a_2$ to the left. We are now at $a_1 + a_2$. We keep
going this way. At the $k^{th}$ stage, we are at the partial sum of
the first $k$ terms, and we then traverse $a_{k+1}$ to reach the
partial sum of the first $k + 1$ terms.

We now see that:

\begin{enumerate}
\item If all the terms in the series are nonnegative, then the
  sequence of partial sums is monotonic increasing (i.e.,
  non-decreasing). If the series converges, it converges to the least
  upper bound of the sequence of partial sums. If the series diverges,
  this just means that the sequence of partial sums monotonically hops
  off to $\infty$.
\item If there are mixed signs among the terms in the series, then we
  keep going up and down (which, on the number line, means right and
  left). A special case is where the terms have alternating signs. In
  this case, the partial sum alternates between right moves and left
  moves. For the series to converge, the magnitude of these moves must
  keep getting smaller and smaller, so that we eventually get to some
  point in between. We can think of the partial sums as a swinging
  pendulum whose successive oscillations get smaller and smaller.
\item In general, a series with mixed signs can go forward a few steps
  and back a few steps. It could converge if, in all this back and
  forth, it is still zeroing in on some particular point. It could be
  oscillating between finite limits if its magnitude of oscillations
  never gets smaller than a certain amount. It could be oscillatorily
  diverging to $\pm \infty$ if the oscillations get bigger and bigger
  and extend farther and farther in both directions.
\end{enumerate}

\subsection{Remarkable results}

Here now are some of the results:

\begin{enumerate}
\item An absolutely convergent series is convergent.
\item More remarkably, {\em every rearrangement} of an absolutely
  convergent series is convergent, and they all converge to the same
  limit.
\item Suppose a series (eventually) has alternating terms and the
  magnitudes of the terms (eventually) {\em monotonically} decrease to
  $0$. Then, the series is convergent. {\em Recall that for any
  series, whether it has positive, zero, or negative terms, a
  necessary but not sufficient condition for convergence is that the
  terms must approach zero.}
\item A series that is convergent but not absolutely convergent can be
  rearranged (i.e., its terms permuted) to give {\em any specified
  real sum}, and can also be rearranged to go to $+\infty$ or to go to
  $-\infty$. It can also be rearranged to oscillate between {\em any
  two specified finite limits}. These results are called the {\em
  Riemann series rearrangement theorem}.
\end{enumerate}

In the next section, we explain each of these results.

\section{Justifications}

\subsection{Justification for (1) and (2)}

Absolutely convergent series are convergent and the sum doesn't depend
on how we arrange the terms. Why does this hold?

Let's take an absolutely convergent series and separate it into two
parts: the positive terms and the negative terms. The zero terms, if
they existed, can be discarded without affecting the sum. Since the
original series is absolutely convergent, the positive terms sum up to
a finite number and the negative terms sum up to a finite
number. Moreover, if we add the absolute values of these, we get the
sum of the absolute values of all terms. The whole series adds up to
the total of the sum of the positive sum and the negative
sum. Moreover, this is independent of the order in which we arrange
the terms.

For instance, if the positive terms all add up to $13$ and the
negative terms all add up to $-9$, the absolute values add up to $|13|
+ |-9| = 22$. The terms of the series add up to $13 + (-9) = 4$.

A real world analogy might help. Suppose you have a total revenue of
$13$ (across many items) and a total expenditure of $9$ (across many
items). You can choose the order in which you earn the various revenue
items and undertake the various expenditure items. The partial sum at
any time reflects your current balance. If you eventually exhaust all
the revenue and expenditure items, your final balance will be $4$,
regardless of the order in which you earn and spend.

\subsection{Justification for (3)}

This is seen using the hopping on the number line. The partial sums
are oscillating with smaller and smaller magnitudes. The lower ends of
the oscillation form an increasing sequence and the upper ends of the
oscillation form a decreasing sequence. The distance between these
sequences keeps getting smaller, and goes to zero, because the terms
are going to zero. Thus, the least upper bound of the increasing
sequence of lower ends of oscillation equals the greatest lower bound
of the decreasing sequence of upper ends of oscillation, and this also
turns out to be the series sum.

\subsection{Justification for (4)}

(4) is the most interesting, since it says that if a series is
convergent but not absolutely convergent, its sum is highly sensitive
to the ordering of the terms. In fact, a suitable rearrangement can
generate just about any sum conceivable. What's going on here? How
does the situation differ fundamentally from (1) and (2)?

Let's once again break up the series into positive and negative
parts. In the absolutely convergent case, both parts had finite
sums. If it is not absolutely convergent, at least one of the parts
must have an infinite sum. But if just one part has an infinite sum,
then the whole sum is also infinite. Since we are ending up with a
finite sum at the end, {\em both} sums must be infinite. Specifically,
the positive series sums up to $+\infty$ and the negative series sums
up to $-\infty$. Also, the magnitudes of the terms in both series must
still go to $0$.

Since both these parts go to infinities of opposite signs, the order
in which we pick the terms matters. Let us look once again at a
credit/debit analogy. Suppose the potential revenue you can earn from
all your revenue sources is infinite and so is the potential
expenditure you can incur from all your potential expenditure
sinks. You can adopt an {\em earn enough to binge strategy} -- you see
the biggest expenditure you can make, then earn enough of your biggest
revenue items to be able to make that expenditure, then make the
expenditure, then again earn enough to make the next expenditure on
the list, and so on. You can convince yourself that this way, your
balance will eventually get closer and closer to zero, because both
your biggest possible expenditures and your biggest possible revenue
items will get smaller and smaller as you proceed.

Or you can be a {\em prudent saver} -- earn a lot, and spend a
little. You may choose to earn $100$ units (using the biggest revenue
items) and then spend the biggest revenue items that fit within $10$
units (so you build savings of about $90$ units). Then, again, you
earn $100$ units, and spend $10$, and so on. This way, you build to an
eventual saving of $\infty$. Since the {\em total} amount that can be
earned and the {\em total} amount that can be spent is infinite, the
time line choices about when to earn and when to spend affect your
overall savings (or debt, as the case may be), even if you {\em
eventually} earn everything and spend everything.

Incidentally, Ponzi schemes are based on this sort of idea -- the idea
that with infinite revenue and infinite expenditures, you can make an
infinite profit, by suitably ordering the points in time when you
acquire revenue and expenditure. And they crash in the real world
because, of course, things are {\em not} infinite.

\end{document}