\documentclass[10pt]{amsart}
\usepackage{fullpage,hyperref,vipul}
\title{Limits at infinity and improper integrals}
\author{Math 153, Section 59 (Vipul Naik)}

\begin{document}
\maketitle

{\bf Corresponding material in the book}: Section 11.7.

{\bf What students should already know}: The intuitive definition of
limit, at least a dim memory of the $\varepsilon$-$\delta$ definition of
limit, graphing a function, the concepts of vertical tangents and
cusps, vertical asymptotes, and horizontal asymptotes. Basic rules for
limits at infinity.

{\bf What students should definitely get}: The few key ideas about
taking limits at infinity involving logarithmic, exponential,
trigonometric, and inverse trigonometric functions. The application of
these to improper integrals.

\section*{Executive summary}

Words ...

\begin{enumerate}
\item The integral $\int_a^\infty f(x) \, dx$ is defined as the limit
  $\lim_{L \to \infty} \int_a^L f(x) \, dx$. If $F$ is an
  antiderivative of $f$, this equals $\lim_{L \to \infty} F(L) - F(a)$.
\item The integral $\int_{-\infty}^a f(x) \, dx$ is defined as the
  limit $\lim_{L \to -\infty} \int_L^a f(x) \, dx$. If $F$ is an
  antiderivative of $f$, this equals $F(a) - \lim_{L \to -\infty} F(L)$.
\item The integral $\int_{-\infty}^\infty f(x) \, dx$ is defined as a
  double limit, where the upper limit of integration is limited to
  infinity, while the lower limit of integration is limited to
  negative infinity. If $F$ is an antiderivative of $f$, this equals
  $\lim_{L \to \infty} F(L) - \lim_{M \to -\infty} F(M)$.
\item Another kind of improper integral occurs where the function is
  integrated over an interval and is not defined at an endpoint of the
  interval. Here, we take the limit over intervals of integration
  where the interval gradually tends towards the trouble points. For
  instance, if a function $f$ is to be integrated over $[a,b]$ but $b$
  is a trouble point, we take $\lim_{c \to b} \int_a^c f(x) \, dx$. If
  $F$ is an antiderivative of $f$, then if $F$ extends continuously to
  $b$, this is just equal to $F(b) - F(a)$.
\item In general, if there are multiple trouble points, we first
  partition the interval of integration so that all the trouble points
  are at the partition boundaries. We then use the limiting procedure
  on each piece and add up across the pieces.
\end{enumerate}

Actions ...

\begin{enumerate}
\item The most straightforward way of computing an indefinite integral
  is to compute the corresponding antiderivative and take the
  difference between the upper and lower limits.
\item In some cases, this is either infeasible or terribly messy. In
  these cases, we may use the various other methods for computing
  definite integrals that bypass computing the antiderivative. These
  include the use of symmetry and a combination of $u$-substitution
  plus noticing that after the substitution, the upper and lower
  limits of integration become the same.
\item In yet other cases, taking the limit of the antiderivative may
  be hard, and we may need to use all the techniques discussed in
  preceding subsections for computing this antiderivative.
\end{enumerate}

\section{Graphing: a bird's eye review}

You have probably studied how to graph a function. I have taught
graphing in 151/152 in previous years following the outline in the
book (Section 4.6) with a bit of augmentation. The book breaks the
problem down into seven easy steps, and I augmented some additional
details to the steps. The overall idea was: {\em find the domain},
{\em find the intervals of interest}, {\em find the points and
  limiting behaviors of interest}, and {\em find the symmetries of
  interest}. Thus, we determined the intervals on which the function
was increasing/decreasing or concave up/down, we found the critical
points, inflection points, and local extreme values, as well as the
points of discontinuity, the vertical tangents and cusps, and the
vertical and horizontal asymptotes. We also checked for even functions
(more generaly, mirror symmetry) and odd functions (more generally,
half-turn symmetry). Then, we combined all this information to graph
the function.

\section{Limits at infinity}

Let us now explore the question of how we in general determined the
limit for a function at $\pm \infty$. Recall that, from the
perspective of the function's graph, this is equivalent to determining
the horizontal asymptotes. At the time we studied this problem, we
were limited by the small class of functions we were dealing
with. Now, we have expanded our collection of functions considerably,
and must accordingly deal with the much larger number of limit
questions.

(If these points do not all seem familiar to you, please review the
discussion of limits at infinity, which was a subtopic the previous
quarter, covered in the lecture on infinity, cusps, tangents, and
asymptotes).

\begin{enumerate}
\item There is a bunch of rules, such as $(\to \infty) \times (\to
  \infty) = \to \infty$.
\item For a nonconstant polynomial, the limit at $+\infty$ depends on
  the sign of its leading coefficient, while the limit at $-\infty$
  depends on the sign of its leading coefficient as well as the parity
  of its degree (i.e., whether the degree is even or
  odd). Specifically, a nonconstant polynomial of even degree and
  positive leading coefficient goes to $+\infty$ in both directions. A
  nonconstant polynomial of odd degree and positive leading
  coefficient goes to $+\infty$ in the positive direction and
  $-\infty$ in the negative direction. If the leading coefficient is
  negative, each limit becomes the negative of what it would have been
  if the leading coefficient were positive. The proof of this
  essentially follows from the limit statements for the function
  $x^n$.
\item For a rational function, if the degree of the numerator is {\em
  less than} the degree of the denominator, the limit as $x \to \pm
  \infty$ is zero. This can be proved by factoring out the largest
  power of $x$ from both numerator and denominator. If the degrees of
  the numerator and denominator are equal, the limit is the quotient
  of the leading coefficients. If the degree of the numerator is
  greater than the degree of the denominator, the limits are
  infinities, with the signs depending of the sign of the quotient of
  the leading coefficients and the parity of the difference of
  degrees. (That's a long-winded sentence, but you should already know
  what this means).
\end{enumerate}

We now look at some of the newer functions that we have added to our
discussion: trigonometric functions, exponential functions,
logarithmic functions, and inverse trigonometric functions. Here are
some general principles:

\begin{enumerate}
\item Remember that $(\to \infty)$ times anything that is bounded
  below by a positive number is still $\to \infty$. Similarly, $\to 0$
  times anything bounded from both above and below is still $\to
  0$. Thus, for instance as $x \to \infty$, $(\sin x + \cos x)/x^2$
  goes to zero because $\sin x + \cos x$ is bounded from both above
  and below, and $1/x^2 \to 0$. Similarly, $x^3(2 + \sin x)$ goes to
  $+\infty$ as $x \to \infty$ because $2 + \sin x \ge 1$ for all
  $x$. On the other hand, $x\sin x$ is wildly oscillating because as
  $x \to \infty$, $\sin x$ oscillates between $+1 $ and $-1$, while $x
  \to \infty$, so the product oscillates wildly with every-increasing
  magnitude.
\item As $x \to \infty$, $e^x \to \infty$ and as $x \to -\infty$, $e^x
  \to 0$. What is particularly important to remember (we will see a
  justification for this later in the course) is that $e^x$ {\em grows
  faster} than any polynomial. In particular, if $p$ is a polynomial
  with positive leading coefficient, then $\lim_{x \to \infty}
  e^x/p(x) = \infty$. Even though the $1/p(x)$ part goes to zero, the
  $e^x$ part goes to $\infty$ much faster. {\em Exponential growth} is
  much faster than {\em polynomial growth}. For similar reasons,
  $\lim_{x \to -\infty} p(x)e^x = 0$ for all polynomials $p$. As $x
  \to -\infty$, $e^x \to 0$ so quickly that no polynomial is able to
  salvage it.
\item As $x \to \infty$, $\ln(x) \to \infty$. What is notable is that
  $\ln(x)$ approaches $\infty$ slower than any polynomial or positive
  power function. Thus, $(\ln x)/x^r \to 0$ as $x \to \infty$ for any
  $r > 0$ and $(\ln x)/p(x) \to 0$ as $x \to \infty$ for any
  nonconstant polynomial $p$. Also, as $x \to 0$, $\ln x \to -\infty$,
  but it goes to $-\infty$ slower than $x^{-r}$ does for any
  $r$. Thus, $x^r \ln x \to 0$ as $x \to 0$ for any $r > 0$ and $p(x)
  \ln x \to 0$ as $x \to 0$ for any polynomial $p$ without constant
  term. Note that the existence of a constant term would send it to
  $\pm \infty$, where the sign of infinity depends on the sign of the
  constant term.
\end{enumerate}

One way of thinking of these functions and their behavior
at $\infty$ is in terms of a {\em hierarchy} where:

\begin{enumerate}
\item $\ln x$ and polynomials in $\ln x$ are at the bottom.
\item $x$ and polynomials in $x$ and power functions of $x$ are in between.
\item $\exp(x)$ is at the top.
\end{enumerate}

As $x \to \infty$, all of these tend to $\infty$, but logarithmic
functions are just no match for polynomial functions, which in turn
are no match for exponential functions. That is why, in colloquial
language, we use the term {\em exponential} for superfast, {\em
polynomial} for ordinarily fast, and {\em logarithmic} for a snail's
pace. Note that within each level, there is a considerably wide range
of incomparable functions. For polynomials, for instance, the larger
the degree, the faster it goes to $+\infty$. However, this rigidly
hierarchical society nonetheless looks monolithically weak and
primitive when compared to the infinitely superior exponentials, while
together they trample on the snail-like logarithms.

\subsection*{Aside: logarithm as something like $x^0$}

One way of thinking of the logarithm function is that it grows faster
than constant functions but slower than $x^r$ for any $r > 0$. Thus,
in mathematically imprecise language, the logarithm function grows like
$x^\varepsilon$ where $\varepsilon$ is an infinitesimal number -- greater
than $0$ but smaller than any positive number.

Another way of justifying this is to note that $x^r$ arises as the
antiderivative (up to constants) of $x^{r- 1}$, for $r \ne 0$. On the
other hand, the antiderivative of $x^{-1}$ is $\ln x$. Since $-1 + 1 =
0$, $\ln x$ is, in some sense, like $x^0$.

In a similar vein, $e^x$ can be thought of as something like
$x^\infty$ in the sense that it beats out $x^r$ for every finite $r$.

\section{The logarithmic transformation}

One of the more useful and less talked about transformations in the
context of understanding functions is to study the logarithm of a
function instead of the original function. Why do we do this? In some
cases, we do this because the expression for $\ln \circ f$ might be
nicer and more instructive than the expression for $f$. For instance,
the Gompertz function is described as:

$$e^{\frac{a - ke^{-abt}}{b}}$$

The logarithm of this looks nicer:

$$\frac{a - ke^{-abt}}{b}$$

There is also a more conceptual reason for often looking at $\ln \circ
f$. For some functions, it is the {\em multiplicative rate of growth}
that is more relevant than the {\em additive rate of growth}. This is
often true for population growths, which are sort-of exponential.

If we do {\em not} use the logarithmic transformation in these cases,
the problem that arises is that a fractional change from $10$ to $20$
looks a lot bigger than a fraction change from $0.1$ to $0.2$, even
though we want to accord the same visual weight to both fractional
changes.

Here are some important features of the logarithmic transformation:

\begin{enumerate}
\item The logarithmic transformation preserves where the function is
  increasing and where it is decreasing. 
\item As $f(x) \to \infty$, $\ln(f(x)) \to \infty$, and as $f(x) \to
  0$, $\ln(f(x)) \to -\infty$. Thus, under the logarithmic
  transformation, the upper half of the $y$-axis gets stretched to the
  entire $y$-axis.
\item The logarithmic transformation need {\em not} preserve the sense
  of concavity. We often say that a function $f$ is {\em
  logarithmically} concave up/down if $\ln \circ f$ is concave
  up/down. It is an easy exercise to work out how the sense of
  concavity of $\ln \circ f$ depends on $f$ and its derivatives.
\end{enumerate}

\section{Improper integrals}

So far, we have looked at definite integration as the following kind
of problem:

$$\int_a^b f(x) \, dx$$

where a continuous function $f$ is integrated over a (finite) closed
interval $[a,b]$. The fundamental theorem of calculus essentially
states that if $F$ if a function defined around $[a,b]$ such that $F'
= f$, then:

$$\int_a^b f(x) \, dx = F(b) - F(a)$$

We now look at some situations where we still want to integrate
between limits, but something fails.

\subsection{Integration where the interval of integration goes to infinity}

Consider for instance the integral:

$$\int_1^\infty \frac{dx}{x^2 + 1}$$

First, what does this {\em mean}? It can be interpreted as:

$$\lim_{L \to \infty} \int_1^L \frac{dx}{x^2 + 1}$$

This can now be evaluated. Specifically, it is:

$$\lim_{L \to \infty} [\arctan L - \frac{\pi}{4}]$$

As $L \to \infty$, $\arctan L \to \pi/2$, so we obtain:

$$\frac{\pi}{2} - \frac{\pi}{4} = \frac{\pi}{4}$$

The same general procedure can be followed for other functions. We
look at three cases:

\begin{enumerate}
\item $\int_a^\infty f(x) \, dx := \lim_{L \to \infty} \int_a^L f(x)
  \, dx$. In particular, if $F$ is an antiderivative for $f$, it is
  $\lim_{L \to \infty} [F(L) - F(a)]$. Note that since $F(a)$ is
  constant, this can be rewritten as $(\lim_{L \to \infty} F(L)) -
  F(a)$. The former limit is the horizontal asymptote value for $F$ as
  $L \to \infty$.
\item $\int_{-\infty}^a f(x) \, dx := \lim_{L \to -\infty} \int_L^a
  f(x) \, dx$. In particular, if $F$ is an antiderivative for $f$, it
  is $\lim_{L \to -\infty} [F(a) - F(L)]$. Since $F(a)$ is constant, this
  is $F(a) - \lim_{L \to -\infty} F(L)$. The latter limit is the
  horizontal asymptote value for $F$ as $L \to -\infty$.
\item $\int_{-\infty}^\infty f(x) \, dx = \lim_{L \to -\infty} \lim_{M
  \to \infty} \int_L^M f(x) \, dx$. Really making proper sense of this
  requires us to know what a {\em double limit} means -- and this is
  something that we will come to later. Intuitively, if $F$ is an
  antiderivative, this is the difference between the limiting values
  of $F$ at $+\infty$ and $-\infty$. Thus, for instance:

  $$\int_{-\infty}^\infty \frac{dx}{x^2 + 1} = \pi$$

  because the limiting value at $+\infty$ is $\pi/2$ and the limiting
  value at $-\infty$ is $-\pi/2$. Graphically, this is the width of
  the strip between the two horizontal asymptotes -- that at $+\infty$
  and that at $-\infty$.
\end{enumerate}

\subsection{Integration where the integrand goes to infinity at an endpoint}

Consider, for instance, the integral:

$$\int_0^1 \frac{dx}{\sqrt{1 - x^2}}$$

The integrand approaches $+\infty$ as $x \to 1$. To make sense of this
integral, we defined it as:

$$\lim_{L \to 1} \int_0^L \frac{dx}{\sqrt{1 - x^2}}$$

This simplifes to:

$$\lim_{L \to 1} [\arcsin L - \arcsin 0] = \frac{\pi}{2}$$

Note something important that is happening here. Although the
integrand itself goes off to $\infty$ at one of the endpoints of
integration, it has an antiderivative that has a finite limit at that
endpoint (where it has a one-sided vertical tangent). That is the
reason the definite integral is in fact well-defined and finite.

More generally:

\begin{enumerate}
\item If we have $\int_a^b f(x) \, dx$ with $a < b$ where $f$ is
  (possibly) undefined or not continuous at $b$ but is defined and
  continuous everywhere else, the integral is $\lim_{L \to b^-}
  \int_a^L f(x) \, dx$. If $F$ is an antiderivative for $f$, the
  integral is $\lim_{L \to b} F(L) - F(a)$. Note that, as the
  $\arcsin$ example shows, $f$ may go to $\infty$ while $F$ has a
  well-defined limit. In other cases, it happens that $f$ is
  oscillatory but $F$ has a well-defined limit.
\item If we have $\int_a^b f(x) \, dx$ with $a < b$ where $f$ is
  (possibly) undefined or not continuous at $b$ but is defined and
  continuous everywhere else, the integral is $\lim_{L \to a^+}
  \int_L^b f(x) \, dx$.
\item If we have problems at both endpoints, we need to do a double
  limit approaching both of them.
\item We could have a mixed situation where one of the limits is an
  actual infinity and the other is a trouble endpoint. The same
  limiting approach works.
\end{enumerate}

\subsection{If it's broken, partition first and then approach endpoints gently}

In addition to problems at the endpoints of the interval of
integration, we could have problems in the {\em interior} of the
interval of integration -- points inside the interval of integration
where the function is not continuous. The discontinuity may be a {\em
jump} discontinuity, an {\em infinite} discontinuity, or some other
kind of discontinuity. In such cases, we adopt a two-step procedure:

\begin{enumerate}
\item We first partition the original interval of integration into
  smaller intervals such that the trouble points are all at the ends
  of intervals of integration.
\item Next, for each of these smaller intervals, we write the integral
  as a limit, as described in the previous part.
\item Finally, we add up the integrals on each of the intervals. This
  is the grand piecing together.
\end{enumerate}

\section{Some subtleties and aspects of improper integration}

\subsection{What does it mean for the integral to infinity to be finite?}

Suppose $f$ is a continuous function defined on $[a,\infty)$ with the
property that $\int_a^\infty f(x) \, dx$ is finite. What can we
conclude about $f$.

Let $F$ be an antiderivative for $f$ on $[a,\infty)$ (with only a
one-sided derivative match at $a$). Then the fact that $\int_a^\infty
f(x) \, dx$ is finite is equivalent to the fact that $\lim_{x \to
\infty} F(x)$ is finite. Since $f = F'$, this means that $F$ has a
horizontal asymptote. What does this imply about $\lim_{x \to \infty}
f(x)$?

Think about it. If $F$ has a horizontal asymptote, what does its
derivative go to? You may be tempted to say that the derivative must
go to zero. This is {\em sort of} true, but there are
counterexamples. Specifically, what {\em is} true is the following:
{\em if} $\lim_{x \to \infty} f(x)$ exists, it must be zero. However,
it is possible that the limit does not exist at all. This can happen
if $f$ has very occasional spikes combined with longer lulls, i.e., if
the motion of $F$ is somewhat jerky.

\subsection{Integrating a function over all real numbers}

Suppose $f$ is a continuous function defined on all of $\R$. What does
it mean to say that $\int_{-\infty}^{\infty} f(x) \, dx$ is finite? We
saw earlier that this involves a two-step limiting procedure: we use a
limiting procedure for the upper end of integration going to
$+\infty$, and another limiting procedure for the lower end of the
integration going to $-\infty$.

{\em The idea we are going to mention now is extremely important}. The
question is: can we define the integral as follows?

$$\int_{-\infty}^\infty f(x) \, dx \stackrel{?}{=} \lim_{a \to \infty} \int_{-a}^a f(x) \, dx$$

In other words, is the integral on the whole of the real line simply
the limit of the integral values on intervals symmetric about the
origin, i.e., intervals of the form $[-a,a]$?

First, let's understand how this definition ostensibly differs from
the earlier definition offered. In the earlier definition, we had two
separate variables, one going to $+\infty$, and the other going to
$-\infty$, and the two variables bore no relation to another. We first
took a limit involving one variable, and then, having got an answer
that depends upon the other variable, we then took the limit in terms
of the other variable.

In the new version, we have a {\em single} variable that controls both
the approach to $+\infty$ and the approach to $-\infty$, and both
approaches are happening in a coordinated fashion -- rather than one
variable going off and reaching infinity first before we start moving
the other variable. The limit in the new version is thus a {\em much
more specific limit} than the limit in the general version.

Here's what is true: {\em if the limit in the general version exists
and is finite}, then it can be computed using the more specific
version. Basically, the general definition says that the limit would
exist however we did the approaches to the respective infinities. Our
new definition gives one such way of making the approach.

However, it is possible that the limit does not exist as per the
general definition and we still get an answer with the new
definition. That answer is wrong and (for the most part) meaningless.

Let's make this concrete with a discussion of odd functions.

\subsection{Odd functions integrated over the real line}

Consider an odd function $f$ defined over the entire real line
$\R$. By {\em odd}, we mean that $f(-x) = -f(x)$ for all $x \in
\R$. We know that integrating an odd function over any symmetric
interval of the form $[-a,a]$ gives the value $0$.

So, we get:

$$\lim_{a \to \infty} \int_{-a}^a f(x) \, dx = 0$$

So, can we conclude the following?

$$\int_{-\infty}^\infty f(x) \, dx \stackrel{?}{=} 0$$

{\em No}. What we can conclude is that {\em if} the integral on the
left side is a well defined finite number, {\em then} it must be zero.

What's happening? Think again to {\em why} the integral of an odd
function on a symmetric interval $[-a,a]$ is $0$. The reason is that
the integral on $[-a,0]$ cancels the integral on $[0,a]$. So, we have
these canceling numbers. As $a$ becomes larger, the numbers keep
canceling each other.

But here's the rub. What if the two numbers that are canceling each
other off are also becoming larger and larger in magnitude, or perhaps
staying roughly constant in magnitude? Then, it becomes very critical
that we are moving the two markers (the ones oing to $+\infty$ and
$-\infty$) in tandem -- if we displace one of them even a bit, then
the integral value will shift abruptly.

For instance, consider the function $f(x) := x$. This is an odd
function. We know that $\int_0^a f(x) \, dx = a^2/2$. And $\int_{-a}^0
f(x) \, dx = -a^2/2$, so the total integral on $[-a,a]$ is $0$. But
the point is that the two integrals individually are becoming larger
and larger in magnitude. The fact that they cancel out is a fortuitous
consequence of the way we're moving the markers. If we moved the two
markers at even slightly different rates, we would get a markedly
different limit. For instance:

$$\lim_{a \to \infty} \int_{-a}^{a + 1} f(x) \, dx = \lim_{a \to \infty} a + (1/2) = \infty$$

Also:

$$\lim_{a \to \infty} \int_{-a}^{a - 1} f(x) \, dx = \lim_{a \to \infty} -a + (1/2) = -\infty$$

Similarly, if we consider the function $f(x) := \sin x$, integrating
on symmetric intervals gives $0$, but if we displaced things slightly,
we would no longer get $0$.

\subsection{What's a good guarantee that stuff actually can be integrated?}

For {\em nonnegative} functions, the problems above do not arise. More
generally, if $\lim_{a \to \infty} \int_{-a}^a |f(x)| \, dx$ is
finite, then we do have:

$$\int_{-\infty}^\infty f(x) \, dx = \lim_{a \to \infty} \int_{-a}^a f(x) \, dx$$

We will explore these criteria, and explain the role of {\em absolute
values}, in a fascinating later discussion about debts, borrowing,
scams, and Ponzi-Madoff schemes.

An example of an odd function that we can immediately (not right now,
but after covering some more stuff in sequences and series) see is
integrable (and hence integrates to $0$) is $f(x) := x/(x^2 + 1)^2$.

\section{Integrability summary for power functions}

\subsection{Positive power functions}

Positive power functions are integrable on all closed and bounded
(finite) intervals, since they are continuous. They are not integrable
as $x \to \infty$ or as $x \to -\infty$.

\subsection{Negative power functions}

Here, we break into cases. For simplicity, we consider only
$(0,\infty)$, including improper integrals at the endpoints. In each
case, we compute an antiderivative and study its limiting behavior at
$0$ and at $\infty$.

\begin{enumerate}
\item For $r \in (-1,0)$, the proper integral is defined and finite on
  any finite interval $[a,b]$ with $0 < a < b < \infty$. Also, the
  improper integral with lower limit $0$ is defined and
  finite. However, an integral with upper limit $+\infty$ goes off to
  $\infty$. This is because the antiderivative starts with value $0$
  at $0$ and goes to $+\infty$ at $\infty$.
\item For $r = -1$, the proper integral is defined and finite on any
  finite interval $[a,b]$ with $0 < a < b < \infty$. The improper
  integral with lower limit $0$ is $+\infty$, and the improper
  integral with upper limit $+\infty$ is also $+\infty$. This is
  because the antiderivative, which is $\ln$, starts at $-\infty$ at
  $0$ and goes to $+\infty$ at $\infty$.
\item For $r < -1$, the proper integral is defined and finite on any
  finite interval $[a,b]$ with $0 < a < b < \infty$. The improper
  integral with lower limit $0$ is $+\infty$. The improper integral
  with positive lower limit and upper limit $+\infty$ is finite. This
  is because the antiderivative starts off at $-\infty$ at $0$ and
  goes to $0$ as $x \to \infty$.
\end{enumerate}

The ideas behind these come up repeatedly in other situations, as we
shall see.
\end{document}

