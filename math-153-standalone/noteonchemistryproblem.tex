\documentclass[10pt]{amsart}
\usepackage{fullpage,hyperref,vipul}
\title{Chemical reactions and differential equations}
\author{Math 15300, Section 21 (Vipul Naik)}
\begin{document}

\maketitle

\section{The question}

Here is the homework problem that you need to solve:

It is known that $m$ parts of chemical $A$ combine with $n$ parts of
chemical $B$ to produce a compound $C$. Suppose that the rate at which
$C$ is produced varies directly with the products of the amounts of
$A$ and $B$ present at that instant. Find the amount of $C$ produced
in $t$ minutes from an initial mixing of $A_0$ pounds of $A$ with
$B_0$ pounds of $B$, given that:

\begin{enumerate}
\item $n = m$, $A_0 = B_0$, and $A_0$ pounds of $C$ are produced in
  the first minute.
\item $n = m$, $A_0 = \frac{1}{2} B_0$, and $A_0$ pounds of $C$ are
  produced in the first minute.
\item $n \ne m$, $A_0 = B_0$, and $A_0$ pounds of $C$ are produced in
  the first minute.
\end{enumerate}

If you just want to solve the problem, you can directly follow the
hint in the book. However, it might be more helpful to understand the
background, which is what we explore in this short note.

\section{Forgetting time: the line of possible configurations}

\subsection{Conservation of mass and constant proportions}

For now, let us forget how the reaction proceeds with time, and
understand what exactly is happening in the reaction.

The reaction is of the form:

$$A + B \to C$$

We are also given that $m$ parts of $A$ combine with $n$ parts of $B$
to yield $C$. Although the question does not state this explicitly,
the {\em parts} here are by mass. This means, for instance, that $m$
pounds of $A$ combine with $n$ pounds of $B$. By the {\em law of
conservation of mass}, the output of $C$ is $m + n$ pounds.

For instance, consider the reaction:

$$2H_2 + O_2 \to 2H_2O$$

Here, the total weight of $2H_2$ is $4$ units and the total weight of
$O_2$ is $32$ units, so $4$ units (By mass) of hydrogen combine with
$32$ units (by mass) of oxygen to yield $36$ units (by mass) of
water. The proportions are thus $m:n = 4:32$, which can also be
written as $1:8$. Thus, $1$ part of hydrogen combines with $8$ parts
of oxygen to yield $9$ parts of water. Note that it is the ratio of
$m$ to $n$ that matters, not the values of $m$ and $n$ per se.

The fact that specific chemical reactions occur in specific
proportions was once considered so important as to actually be given a
name. This name is the {\em law of constant proportions} or the {\em
law of definite proportions}.

\subsection{Stoichiometric constraints}

Since a given chemical reaction occurs with constant proportions, this
indicates that the {\em change in mass} of each of the reactants and
products through the course of the reaction is in the same
proportion. For instance, in the hydrogen-oxygen reaction to produce
water, we know that $1$ pound of hydrogen would combine with $8$
pounds of oxygen to produce $9$ pounds of water. This means that if
$x$ pound of hydrogen are lost, then $8x$ pounds of oxygen are lost,
and $9x$ pounds of water are gained. Thus, if we know the initial
amounts of hydrogen, oxygen, and water, and we know the final amount
of hydrogen, we can measure the hydrogen lost, and use that to deduce
the oxygen lost (and hence the final amount of oxygen) and the amount
of water gained (and hence the final amount of water).

This can be viewed in terms of a three-dimensional configuration
space. Consider a three-dimensional space with the three axes marked
respectively by the current mass of hydrogen, the current mass of
oxygen, and the current mass of water.

What the stoichiometric constraint (or the law of constant
proportions) says is that, as the reaction proceeds, the configuration
moves along a straight line whose direction is determined by the
nature of the reaction. The straight line has the property that the
oxygen axis coordinate changes at eight times the rate of change of
the hydrogen axis coordinate, and the water axis coordinate changes in
the opposite direction at nine times the rate.

In the more general case, with:

$$A + B \to C$$

the coordinate change ratios are $m:n:-(m + n)$, and the path of the
reaction is along a straight line.

\subsection{Initial condition}

The path of the reaction is along a straight line whose direction is
determined by the nature of the reaction (i.e., by the ratio
$m:n$). However, there are many different straight lines with the same
direction. The {\em particular} straight line that the reaction is
confined to depends on the initial configuration.

Typically, we start out without any of the product (i.e., we only have
the reactants $A$ and $B$. In terms of the three-dimensional picture
with coordinates $A$, $B$, and $C$, the initial configuration is in
the $AB$-plane. {\em Where} it is in the $AB$-plane determines {\em
which} line the reaction moves against. (This can be thought of as a
three-dimensional analogue of the point-slope form).

\subsection{Keeping track of just one coordinate}

Suppose we want to know each coordinate at every point in time. We
claim that it is enough to know:

\begin{enumerate}
\item The initial value of each coordinate.
\item The value, at any given time $t$, of just one coordinate.
\end{enumerate}

In particular, if we know the initial amounts of $A$, $B$ and $C$
(which we assume to be zero), then knowing the quantity of $C$ at time
$t$ allows us to determine the quantities of $A$ and $B$ at time
$t$.

Here's how we can deduce what these quantities are. Let $C(t)$ denote
the quantity of $C$ at time $t$. We're assuming $C(0) = 0$, so $C(t)$
amount of $C$ was gained. By constant proportions, we note that the
amount of $A$ lost is $mC(t)/(m + n)$ and the amount of $B$ lost is
$nC(t)/(m + n)$. Thus, the amount of $A$ at time $t$ is:

$$A(t) = A(0) - \frac{m}{m + n}C(t)$$

Similarly, the amount of $B$ at time $t$ is:

$$B(t) = B(0) - \frac{n}{m + n}C(t)$$

In the problem setup, we are given that $A(0) = A_0$ and $B(0) =
B_0$.

\section{Bringing time and rates into the equation}

So far, we have largely focused on the path that the reaction
takes. The next relevant question is the {\em rate} at which the
reaction occurs. In the three-dimensional graphical representation,
this is basically determining how fast we're moving along the line.

When trying to measure the rate of a reaction, there is a little
ambiguity. Should we measure the rate at which $A$ is being lost, the
rate at which $B$ is being lost, or the rate at which $C$ is being
gained? By the law of constant proportions, these rates are all related:

$$\frac{-1}{m} \frac{dA}{dt} = \frac{-1}{n} \frac{dB}{dt} = \frac{1}{m + n} \frac{dC}{dt}$$

We can keep track of just {\em one} of the three, and the one we
choose to keep track of is $C$. Note that the proportionality in the
rates of change is the {\em differential version} of the expressions
$A(t) = A(0) - mC(t)/(m + n)$ and $B(t) = B(0) - nC(t)/(m + n)$.

Thus, when we say that the {\em rate of reaction} is proportional to
some quantity, that statement would apply to all the three rates of
change $dA/dt$ $dB/dt$, and $dC/dt$, with the constants of
proportionality themselves in the ratio $m:n:-(m + n)$.

\section{Formulating and solving the differential equation}

\subsection{Formulating the differential equation}

For convenience, we take pounds (the mass measure) as our unit of
measurement. Let $A_0$ and $B_0$ be the initial number of pounds of
$A$ and $B$ respectively.

If the number of pounds of $C$ at time $t$ is $C(t)$, then $A(t) = A_0
- mC(t)/(m + n)$ and $B(t) = B_0 - nC(t)/(m + n)$.

We know that the rate $dC/dt$ equals $kAB$, where $k$ is some unknown
constant. We thus have:

$$\frac{dC}{dt} = k\left[A_0 - \frac{m}{m + n}C(t) \right]\left[B_0 - \frac{n}{m + n}C(t) \right]$$

Note that this is an {\em autonomous} differential equation, which is
to be expected since it arises from a physical law that is time
translation invariant. (Note: The {\em law} is time translation
invariant, not the actual configuration).

Notice another feature of this problem. In general, when we solve a
differential equation, we introduce one free parameter. For this
differential equation, we {\em already} have a bunch of parameters,
and we will introduce {\em one more} parameter when we solve the
differential equation. However, the added parameter can easily be
determined from the initial condition $C(0) = 0$.
\subsection{Solving the differential equation}

The equation is autonomous and separable, and we can rearrange terms to get:

$$\int \frac{dC}{\left[A_0 - \frac{m}{m + n}C \right]\left[B_0 - \frac{n}{m + n}C \right]} = \int k \, dt$$

The right side integrates to $kt + c_1$, and the left side can be
integrated in one of the standard ways. Once we integrate the left
side, we can choose $c_1$ in such a way that $C(0) = 0$.

There are two natural cases for the left side:

\begin{enumerate}
\item The two linear factors in the denominator are equal, or
  proportional: In this case, the integrand is the reciprocal of the
  square of a linear function of $C$, and that is directly
  integrated. Note that this case occurs if $A_0/B_0 = m/n$, i.e., if
  the initial proportion of masses is the {\em stoichiometric
  proportion}. In the homework problem, part (a) is of this type.
\item The two linear factors in the denominator are not proportional:
  In this case, we need to use the partial fractions approach to
  integrate. In the homework problem, parts (b) and (c) are of this type.
\end{enumerate}

\subsection{An additional piece of information}

Suppose we are given $m/n$ as well as $A_0$, $B_0$. Then, the only
unknown in the original differential equation is $k$. Integration
yields another unknown parameter (arising as the additive constant
from integration) which can be determined using $C(0) = 0$. We thus
get a general expression for $C(t)$ that features $k$. However, since
$k$ is not given, we would like to determine it.

To determine $k$, we need some other piece of information. In the
question here, the additional piece of information is in the form of
the value $C(1)$. Specifically, we {\em plug in} the value $C(1)$ in
the general expression for $C(t)$ and {\em solve the corresponding
equation} to determine the value of $k$. Having obtained this value of
$k$, we {\em plug back} in the general expression for $C(t)$.

The use of a point-in-time measurement to determine $k$ is analogous
to many experimental approaches in physics and chemistry. In most of
these, a general physical or chemical law tells us that the expression
for a quantity is of some type, with one or more parameters
appearing. These parameters are physical constants but we do not have
the theoretical tools to determine them. So, we do an experiment that
measures certain things, from which we deduce the value of the parameter.

For instance, in physics, there are some laws of friction between dry
bodies, which explain how the magnitude of the friction force between
two bodies depends on a certain dimensionless constant called the
coefficient of friction, that depends on the surfaces in
contact. However, the coefficient of friction cannot be determined
theoretically. So, we do an experimental study that measures certain
quantities, from which we can deduce the coefficient of friction.
\end{document}