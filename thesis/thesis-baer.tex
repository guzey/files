\section{Baer correspondence: the basic setup}\label{sec:baer-correspondence-basics}

The {\em Baer correspondence} was introduced by Reinhold Baer in
\cite{Baer38}. We will review the correspondence in this section, and
study it in further detail in the next two sections (Sections
\ref{sec:baer-correspondence-more} and
\ref{sec:baer-correspondence-definition-relaxation}). This paves the
  way for the section after that (Section \ref{sec:bcuti}), which
  contains a novel contribution of this thesis, that we call the {\em
    Baer correspondence up to isoclinism}. This is relatively
  straightforward to prove and understand. It sets the stage for the
  next few sections, where we discuss the {\em Lazard correspondence}
  (introduced by Lazard in \cite{Lazardsoriginal}) and our variation
  of it, namely the {\em Lazard correspondence up to isoclinism}.

Many of the ideas described in this section are similar to, and build
upon, ideas in Section \ref{sec:abelian-lie-correspondence}, where we
describe the abelian Lie correspondence.

\subsection{Baer Lie groups and Baer Lie rings}\label{sec:baer-lie-definitions}

We begin with some definitions. These definitions are not standard,
and there may be somewhat different meanings in other sources that use
these words.

\begin{definer}[Baer Lie group]\label{def:baer-lie-group}
  A group $G$ is termed a {\em Baer Lie group} if $G$ is powered over
  the prime $2$ (per Definition \ref{def:poweredgroup}; explicitly,
  this means that every element of $G$ has a unique square root) {\em
    and} $G$ is nilpotent with nilpotency class at most $2$.
\end{definer}

\begin{definer}[Baer Lie ring]\label{def:baer-lie-ring}
  A Lie ring $L$ is termed a {\em Baer Lie ring} if $L$ is powered
  over the prime $2$ (i.e., its additive group is powered over $2$ per
  Definition \ref{def:poweredgroup}; explicitly, this means that every
  element has a unique half) {\em and} $L$ is nilpotent with
  nilpotency class at most $2$.
\end{definer}

The Baer correspondence is a correspondence:

\begin{center}
  Baer Lie groups $\leftrightarrow$ Baer Lie rings
\end{center}

A group and Lie ring that are in Baer correspondence have {\em the
  same underlying set}.

\subsection{Construction from group to Lie ring}\label{sec:baer-group-to-lie-ring}

Consider a Baer Lie group $G$ with multiplication denoted by
juxtaposition and the identity element denoted by $1$. We define a
Baer Lie ring, denoted $\log G$ or $\log(G)$, as follows. The
underlying set of $\log G$ is the same as the underlying set of $G$,
and the operations are as follows.

\begin{itemize}
\item The addition on $\log(G)$ is defined as $x + y :=
  \frac{xy}{\sqrt{[x,y]}}$ where $[x,y]$ denotes the group
  commutator. Note that because the group has class two, it does not
  matter whether we use the left action convention for the commutator
  $[x,y] = xyx^{-1}y^{-1}$ or the right action convention for the
  commutator $[x,y] = x^{-1}y^{-1}xy$: they both mean the same thing.

  Note that $[x,y] \in Z(G)$ (where $Z(G)$ denotes the center of $G$)
  because $G$ has class at most two. $G$ is $2$-powered, so by Lemma
  \ref{lemma:centerispoweringinvariant}, the center $Z(G)$ is also
  $2$-powered. Thus, $\sqrt{[x,y]}$ is also in $Z(G)$. Thus, we can
  ``divide'' $xy$ by this element unambiguously, without specifying
  whether the division is on the left or on the right.

  There are two alternative expressions for $x + y$ that are equal to
  the above and are sometimes more useful to use: $x + y =
  \sqrt{x}y\sqrt{x}$ and $x + y = \sqrt{xy^2x}$. For more on these
  expressions, see Section \ref{sec:twisted-multiplication}.
\item The zero element of $\log(G)$ is defined as equal to the
  identity element $1$ of $G$.
\item The additive inverse $-x$ is defined as $-x := x^{-1}$.
\item The Lie bracket $[x,y]_{\text{Lie}}$ is defined as the group
  commutator $[x,y]$. As remarked above, it does not matter whether we
  use the left action convention or the right action convention for
  the commutator.
\end{itemize}

\begin{lemma}
  $\log G$ (as defined above, with the same underlying set as $G$), is
  a Baer Lie ring.
\end{lemma}

\begin{proof}
  The proof requires showing the following:

  \begin{enumerate}
  \item Addition is associative
  \item Addition is commutative
  \item Identity and inverses work
  \item The Lie bracket is additive in the first variable
  \item The Lie bracket is additive in the second variable
  \item The Lie bracket is alternating
  \item The Lie bracket satisfies the Jacobi condition and gives a class two Lie ring
  \item The Lie ring is $2$-powered.
  \end{enumerate}
  
  {\em Proof of (1)}: We want to show that for every $x$, $y$, $z$ in
  $G$ (possibly equal, possibly distinct), we have:

  $$(x + y) + z = x + (y + z)$$

  We first consider the left side:

  $$(x + y) + z = \frac{\frac{xy}{\sqrt{[x,y]}} \cdot z}{\sqrt{\left[\frac{xy}{\sqrt{[x,y]}},z\right]}}$$

  We know that for $c$ central, $[a,bc] = [ac,b] = [a,b]$. Since the
  reciprocal of $\sqrt{[x,y]}$ is central, it can be dropped from
  inside commutator expressions, and we simplify to:

  $$(x + y) + z = \frac{xyz}{\sqrt{[x,y]}\sqrt{[xy,z]}}$$

  The square root operation is a homomorphism on the center, and we can thus rewrite it as:

  \begin{equation*}
    (x + y) + z = \frac{xyz}{\sqrt{[x,y][xy,z]}} \tag{$\dagger$}
  \end{equation*}

  Similarly, the right side of the associativity result we want to prove becomes:

  \begin{equation*}
    x + (y + z) = \frac{xyz}{\sqrt{[x,yz][y,z]}} \tag{$\dagger \dagger$} 
  \end{equation*}

  Thus, to prove associativity, it suffices to show that the right
  sides of $(\dagger)$ and $(\dagger\dagger)$ are equal, which in turn
  reduces to proving that:

  $$ [x,y][xy,z] = [x,yz][y,z]$$

  For a group of class two, the commutator map is a homomorphism in
  each of its coordinates (see Lemma
  \ref{lemma:iterated-commutator-is-multilinear} for a proof of this
  and related statements). Thus, both sides simplify to:

  $$[x,y][x,z][y,z]$$

  and hence both sides are equal, completing the proof.

  {\em Proof of (2)}: We want to show that for any $x$, $y$ in the
  group, $x + y = y + x$. The key ingredient to our proof is the
  observation that $[x,y]^{-1} = [y,x]$.

  We have, by definition:

  $$x + y = \frac{xy}{\sqrt{[x,y]}}$$

  and:

  $$y + x = \frac{yx}{\sqrt{[y,x]}}$$

  It thus suffices to prove that:

  $$\frac{xy}{\sqrt{[x,y]}} = \frac{yx}{\sqrt{[y,x]}}$$
  
  which is equivalent to proving that:

  $$\frac{xy}{\sqrt{[x,y]}} = yx\sqrt{[x,y]}$$

  which in turn is equivalent to proving that:

  $$xy = yx[x,y]$$

  which is true by the definition of $[x,y]$.

  {\em Proof of (3)}: We have:

  $$x + 1 = \frac{x1}{\sqrt{[x,1]}} = x$$

  and:

  $$1 + x = \frac{1x}{\sqrt{[1,x]}} = x$$

  Thus, $1$ is an identity element for the Lie ring addition.

  We also have:

  $$x + x^{-1} = \frac{xx^{-1}}{\sqrt{[x,x^{-1}]}} = 1$$

  and:

  $$x^{-1} + x = \frac{x^{-1}x}{\sqrt{[x^{-1},x]}} = 1$$

  {\em Proof of (4) and (5)}: These follow from the corresponding
  facts for the commutator map in a group of class two.

  {\em Proof of (6)}: This follows from the general fact that
  $[x,y]^{-1} = [y,x]$.

  {\em Proof of (7)}: In fact, {\em all} Lie products $[[x,y],z]$ are
  zero, so the Jacobi identity is trivially satisfied.

  {\em Proof of (8)}: We have:

  $$x + x = \frac{x^2}{\sqrt{[x,x]}} = x^2$$

  In other words, the double of an element in the Lie ring is the same
  as its square in the group. Since the group is 2-powered, the Lie
  ring is also 2-powered.
\end{proof}

\subsection{Construction from Lie ring to group}\label{sec:baer-lie-ring-to-group}

Suppose $L$ is a Baer Lie ring with addition $+$ and Lie bracket $[
  \ , \ ]$. We define a Baer Lie group $\exp(L)$ as follows. The
underlying set of $\exp(L)$ is the same as the underlying set of
$L$. The group operations are defined as follows:

\begin{itemize}
\item The group multiplication on $\exp(L)$ is defined as:

  $$xy = x + y + \frac{1}{2}[x,y]$$
\item The identity element for the group multiplication of $\exp(L)$ is
  defined as the element $0$ of $L$.
\item The inverse operation is defined as $x^{-1} := -x$.
\end{itemize}

\begin{lemma}\label{lemma:baer-correspondence-lie-ring-to-group}
  $\exp(L)$ (defined as above, with the same underlying set as $L$) is
  a Baer Lie group. Moreover, the group commutator in $\exp(L)$
  coincides with the Lie bracket in $L$.
\end{lemma}

\begin{proof}
  The proof requires showing the following:

  \begin{enumerate}
  \item Multiplication is associative.
  \item The identity element and inverses work.
  \item The commutator map in the group agrees with the Lie bracket.
  \item The group has nilpotency class at most two.
  \item The group is $2$-powered.
  \end{enumerate}

  {\em Proof of (1)}: Let $x$, $y$, $z$ be arbitrary (possibly equal,
  possibly distinct) elements of $L$.

  {\bf To prove}: $(xy)z = x(yz)$

  {\bf Proof}: We begin by simplifying the left side. We have:

  $$(xy)z = \left(x + y + \frac{1}{2}[x,y]\right) + z + \frac{1}{2}\left[x + y + \frac{1}{2}[x,y],z\right]$$

  We use the linearity of the Lie bracket, and also use that the
  Lie ring has nilpotency class two to simplify $[[x,y],z]$
  to zero. The expression simplifies to:

  $$(xy)z = x + y + z + \frac{1}{2}\left([x,y] + [x,z] + [y,z]\right)$$

  Similarly, we can show that:

  $$x(yz) = x + y + z + \frac{1}{2}\left([x,y] + [x,z] + [y,z]\right)$$

  Thus, associativity holds.

  {\em Proof of (2)}: From the expression for group multiplication, we
  obtain that if $[x,y] = 0$, then $xy = x + y$. In particular, this
  means that:

  $$x(0) = x + 0 = x$$

  and:

  $$(0)x = 0 + x = x$$

  Thus, $x(0) = (0)x = x$, so $0$ is an identity element for
  multiplication, and we obtain that the multiplication defined turns
  $L$ into a monoid.

  Further, we obtain that:

  $$x(-x) = x + (-x) = 0$$

  and:

  $$(-x)x = (-x) + x = 0$$

  Thus, $-x$ is a two-sided multiplicative inverse for $x$. $L$ is
  thus a monoid where every element has a two-sided multiplicative
  inverse, hence is a group. Further, $-x = x^{-1}$ in this group for
  all $x$.

  {\em Proof of (3)}: We want to compute an explicit expression for
  the group commutator $[x,y]_{\text{Group}} = xyx^{-1}y^{-1} = (xy)(yx)^{-1}$ in
  terms of the Lie ring operations.

  First, we use that $(yx)^{-1} = -(yx)$, so we get:

  $$[x,y]_{\text{Group}} = (xy)(-(yx)) = (xy) + (-(yx)) + \frac{1}{2}[xy,-(yx)]$$

  This simplifies to:

  $$[x,y]_{\text{Group}} = x + y + \frac{1}{2}[x,y] - (y + x + \frac{1}{2}[y,x]) + \frac{1}{2}[x + y + \frac{1}{2}[x,y],-(y + x + \frac{1}{2}[y,x])]$$

  Simplify using the fact that the Lie ring has class two, so that all
  Lie products of the form $[a,[b,c]]$ or $[[a,b],c]$ are zero. We get:

  $$[x,y]_{\text{Group}} = [x,y]$$

  Thus, the group commutator equals the Lie bracket. Note that the
  proof above is for the group commutator defined using the left
  action convention, but an analogous proof exists for the group
  commutator defined using the right action convention.

  {\em Proof of (4)}: This follows from (3) and the fact that the Lie
  ring has nilpotency class two.

  {\em Proof of (5)}: For any $x \in L$, $x^2$ with respect to the
  group structure is given by:

  $$x^2 = x + x + \frac{1}{2}[x,x] = 2x$$

  In other words, the squaring operation in the group coincides with
  the doubling operation in the Lie ring. The latter is bijective
  because the Lie ring is $2$-powered. Hence, the former operation is
  also bijective. Therefore, the group is $2$-powered.
\end{proof}

\subsection{Mutually inverse nature of the constructions}

We now show that $\exp$ and $\log$ are two-sided inverses of each
other.

\begin{lemma}\label{lemma:baer-mutual-inverses}
  The construction of the Baer Lie ring of a Baer Lie group and the
  construction of the Baer Lie group of a Baer Lie ring are two-sided
  inverses of each other. Explicitly:

  \begin{enumerate}
  \item Start with a Baer Lie group $G$. Then, $G = \exp(\log(G))$.
  \item Start with a Baer Lie ring $L$. Then, $L = \log(\exp(L))$.
  \end{enumerate}
\end{lemma}

\begin{proof}
  Note that the identity element and inverse map are the same for the
  group and Lie ring, and the Lie bracket for the Lie ring is the same
  as the commutator for the group, so the main thing to check is the
  interplay between the Lie ring addition and the group
  multiplication.

  {\em Proof of (1)}: We want to show that the ``new'' group
  multiplication coincides with the original group multiplication:

  $$(x + y) + \frac{1}{2}[x,y] = xy$$

  We begin by simplifying the left side. We begin by replacing $x + y$
  by its expression in terms of the group multiplication, and obtain:

  $$\frac{xy}{\sqrt{[x,y]}} + \frac{1}{2}[x,y]$$

  This further simplifies to:

  $$\frac{\frac{xy}{\sqrt{[x,y]}} \sqrt{[x,y]}}{\sqrt{\left[\frac{xy}{\sqrt{[x,y]}},\sqrt{[x,y]}\right]}}$$

  
  The lower denominator is the identity element because $\sqrt{[x,y]}$
  is central on account of the group having class two and being
  2-powered. The numerator simplifies to $xy$. This completes
  the proof.

  {\em Proof of (2)}: We want to show that the ``new'' Lie ring
  addition coincides with the original Lie ring addition. Explicitly,
  we want to show that:
  
  $$\frac{xy}{\sqrt{[x,y]}} = x + y$$

  We simplify the left side. Note that $\sqrt{[x,y]} =
  \frac{1}{2}[x,y]$ is central back in the Lie ring, so this becomes:
  
  $(x + y + \frac{1}{2}[x,y]) - \frac{1}{2}[x,y]$

  This simplifies to $x + y$, as desired.
\end{proof}

\subsection{Understanding the formulas in the Baer correspondence}\label{sec:mean-deviation}

Given two elements $x$ and $y$ in a Baer Lie group $G$, there are two
possible products we can consider: $xy$ and $yx$. The quotient of
these products $(xy)/(yx)$ is the commutator $[x,y]$. Note that we get
the same answer whether we use left or right quotients, because the
elements $xy$ and $yx$ commute on account of the class being two.

The ``average'' of the two products can be thought of as an element
$z$ that is midway between $xy$ and $yx$, i.e., we want $z$ to
satisfy:

$$\frac{xy}{z} = \frac{z}{yx}$$

If we rearrange and solve, we will get:

$$z = \frac{xy}{\sqrt{[x,y]}}$$

The Baer Lie ring construction sets $x + y$ to equal this ``average''
value and sets the Lie bracket $[x,y]$ to equal the ``distance''
between $xy$ and $yx$. This agrees with our earlier statement made in
Section \ref{sec:intro-malcev-and-lazard} about Lie-type
correspondences in general: ``The addition operation of the Lie ring
captures the abelian part of the group multiplication, whereas the Lie
bracket captures the non-abelian part of the group multiplication.''

\subsection{Twisted multiplication of a 2-powered group}\label{sec:twisted-multiplication}

Suppose $G$ is a 2-powered (not necessarily nilpotent) group. For any
$u \in G$, denote by $\sqrt{u}$ the unique element $v \in G$ such that
$v^2 = u$.\footnote{The typical case of interest is where $G$ is a
  finite group, in which case being 2-powered is equivalent to being
  an odd-order group. However, none of the statements here rely on
  finiteness.}

The {\em twisted multiplication} on $G$ has two somewhat different but
related definitions. The two definitions are denoted as $*_1$ and
$*_2$ below, and the reason for their equivalence is discussed below:

$$x *_1 y := \sqrt{x}y\sqrt{x}$$

$$x *_2 y := \sqrt{xy^2x} = \sqrt{(xy)(yx)}$$

The latter can be thought of as the ``mean'' between $xy$ and $yx$,
and more explicitly, it is the unique solution $z$ to:

$$z^{-1}xy = z(yx)^{-1}$$

For any $x,y \in G$, we have:

$$x^2 *_1 y^2  = (x *_2 y)^2 = xy^2x$$

Thus, the square map establishes an isomorphism between $(G,*_2)$ and
$(G,*_1)$.

As proved in \cite{Foguelinv}, $G$ acquires the
structure of a gyrocommutative gyrogroup (and in particular, the
structure of a loop) under either of these equivalent
operations. Explicitly:

\begin{itemize}
\item The identity element for $*_1$ equals the identity element for
  $*_2$, and both are equal to the identity element for $G$ as a group.
\item The inverse map for $*_1$ is the same as the inverse map for
  $*_2$, and both are equal to the inverse map for $G$ as a group.
\item In terms of $*_2$, the gyroautomorphism
  $\operatorname{gyr}([x,y])$ is defined to be conjugation in $G$ by
  $\sqrt{xy^2x}x^{-1}y^{-1} = (\sqrt{xy^2x})^{-1}xy$. Note that the
  conjugation element can be thought of as the ``mean deviation'',
  i.e., it is the distance between either of $xy$ and $yx$ and the
  ``mean'' between them, in a manner similar to that described in
  Section \ref{sec:mean-deviation}.
\end{itemize}

The connection of the twisted multiplication with the Baer
correspondence is as follows: given a Baer Lie group, the addition
operation in the corresponding Baer Lie ring coincides with the
twisted multiplication (in fact, it coincides with {\em both} $*_1$
and $*_2$). Explicitly, for a Baer Lie group, the Lie ring addition
has the following equivalent forms:

$$x + y = \frac{xy}{\sqrt{[x,y]}} = \sqrt{x}y\sqrt{x} = \sqrt{xy^2x}$$

\subsection{Preservation of homomorphisms and isomorphism of categories}\label{sec:baer-correspondence-homomorphism-preservation}

In Section
\ref{sec:abelian-lie-correspondence-homomorphism-preservation}, we
made a number of observations leading to the conclusion that the
category of abelian groups is isomorphic to the category of abelian
Lie rings, with the isomorphisms given explicitly by the $\log$ and
$\exp$ functors.

The analogous conclusion also holds for the Baer
correspondence. However, the reasoning behind these steps is somewhat
different from the abelian case. In the abelian case, the only
conceptual distinction between the group side and Lie ring side was
that the latter had a trivial Lie bracket operation. Other than that,
the operations were the same. In the Baer case, the group operations
and Lie ring operations are defined somewhat differently. The reason
that the correspondence works is that {\em any formula commutes with
  any homomorphism}. For instance, if $w(x,y)$ is the word describing
the Lie ring addition in terms of the group operations (including
powering operations), then for any group homomorphism $\varphi$:

$$\varphi(w(x,y)) = w(\varphi(x),\varphi(y))$$

In the case of the Baer correspondence, the formula in question is:

$$w(x,y) = \frac{xy}{\sqrt{[x,y]}}$$

and the assertion becomes:

$$\varphi\left(\frac{xy}{\sqrt{[x,y]}}\right) = \frac{\varphi(x)\varphi(y)}{\sqrt{[\varphi(x),\varphi(y)]}}$$

Since the Lie ring operations are defined in terms of the group
operations, a homomorphism of groups preserves the Lie ring
operations, and therefore gives a homomorphism of Lie rings. In the
opposite direction, since the group operations are defined in terms of
the Lie ring operations, a homomorphism of Lie rings preserves the
group operations, and therefore gives a homomorphism of
groups. Explicitly, in this case, if $\varphi$ is a Lie ring
homomorphism between Baer Lie rings, then the assertion is that:

$$\varphi\left(x + y + \frac{1}{2}[x,y]\right) = \varphi(x) + \varphi(y) + \frac{1}{2}[\varphi(x),\varphi(y)]$$

The explicit statements are below:

\begin{itemize}
\item $\log$ defines a functor from Baer Lie groups to Baer Lie rings:
  Suppose $G_1$ and $G_2$ are Baer Lie groups and $\varphi:G_1 \to
  G_2$ is a group homomorphism. Then, there exists a unique Lie ring
  homomorphism $\log(\varphi): \log(G_1) \to \log(G_2)$ that has the
  same underlying set map as $\varphi$. {\em Reason}: The Lie ring
  operations are defined as formal expressions in terms of the group
  operations and the square root operation, and this expression is
  preserved under homomorphisms.
\item $\exp$ defines a functor from Baer Lie rings to Baer Lie groups:
  Suppose $L_1$ and $L_2$ are Baer Lie rings and $\varphi:L_1 \to L_2$
  is a Lie ring homomorphism. Then, there exists a unique group
  homomorphism $\exp(\varphi): \exp(G_1) \to \exp(G_2)$ that has the
  same underlying set map as $\varphi$. {\em Reason}: The group
  operations are defined as formal expressions in terms of the Lie
  ring operations and the halving operation, and this expression is
  preserved under homomorphisms.
\item The $\log$ and $\exp$ functors are two-sided inverses of each
  other: This has four parts:
  \begin{itemize}
    \item For every Baer Lie group $G$, $G = \exp(\log(G))$. This is
      part of Lemma \ref{lemma:baer-mutual-inverses}.
    \item For every Baer Lie ring $L$, $L = \log(\exp(L))$. This is
      part of Lemma \ref{lemma:baer-mutual-inverses}.
    \item For every group homomorphism $\varphi:G_1 \to G_2$ of Baer
      Lie groups, $\exp(\log(\varphi)) = \varphi$. This follows
      immediately from the fact that both taking $\log$ and taking
      $\exp$ preserve the underlying set map.
    \item For every Lie ring homomorphism $\varphi:L_1 \to L_2$ of
      Baer Lie rings, $\log(\exp(\varphi)) = \varphi$. This follows
      immediately. This following immediately from the fact that both
      taking $\log$ and taking $\exp$ preserve the underlying set map.
  \end{itemize}
\end{itemize}

The upshot is that the Baer correspondence defines an isomorphism of
categories over the category of sets between the category of Baer Lie
groups and the category of Baer Lie rings. Here, by ``category of Baer
Lie groups'' we mean the full subcategory\footnote{Full subcategory
  means that all morphisms of the big category (in this case, the
  category of groups) between objects of the subcategory (in this
  case, Baer Lie groups) are morphisms in the subcategory.} of the
category of groups where the objects are the Baer Lie
groups. Similarly, by ``category of Baer Lie rings'' we mean the full
subcategory of the category of Lie rings where the objects are Baer
Lie rings.

Note that all steps of this reasoning can be repeated for the Lazard
correspondence. To avoid repetition, we will omit the details and
instead refer back to this section as needed.

\subsection{Consequences for the Baer correspondence of being an isomorphism of categories}\label{sec:baer-correspondence-isocat-consequences}

We have established above that the Baer correspondence defines an
isomorphism of categories. Mimicking the steps in Section
\ref{sec:abelian-lie-correspondence}, we obtain the following:

\begin{itemize}
\item The Baer correspondence preserves endomorphism
  monoids. Explicitly, if $G$ is a Baer Lie group and $L = \log(G)$,
  then $\operatorname{End}(G)$ and $\operatorname{End}(L)$ are
  isomorphic monoids, and define the same collection of set maps. The
  reasoning mimicks Section
  \ref{sec:abelian-lie-correspondence-aut-end}.
\item The Baer correspondence also preserves automorphism
  groups. Again, the reasoning mimicks Section
  \ref{sec:abelian-lie-correspondence-aut-end}.
\item We can define the Baer correspondence {\em up to isomorphism}
  (mimicking Section
  \ref{sec:abelian-lie-correspondence-up-to-isomorphism}). This is a
  correspondence:

  \begin{center}
    Isomorphism classes of Baer Lie groups $\leftrightarrow$
    Isomorphism classes of Baer Lie rings
  \end{center}
\end{itemize}

%\newpage

\section{Baer correspondence: additional remarks}\label{sec:baer-correspondence-more}

\subsection{Subgroups, quotients, and direct products}\label{sec:baer-correspondence-sub-quot-dp}

In Section \ref{sec:abelian-lie-correspondence-sub-quot-dp}, we noted
that the abelian Lie correspondence between an abelian group and an
abelian Lie ring gives rise to a correspondence between all subgroups
of the group and all subrings of the Lie ring.

A related result holds for the Baer correspondence, but there are more
caveats. The important caveat is that the 2-powered groups do {\em
  not} form a sub{\em variety} of the variety of groups. Therefore,
the Baer Lie groups do not form a subvariety of the variety of
groups. Let us examine subgroups, quotients, and direct products
separately:

\begin{itemize}
\item {\em Subgroups}: A subgroup of a Baer Lie group need not be a
  Baer Lie group. For instance, the group $\Q$ of rational numbers is
  a Baer Lie group, but its subgroup $\Z$ is not a Baer Lie group.
\item {\em Quotients}: A quotient group of a Baer Lie group need not
  be a Baer Lie group. For instance, the group $\Q$ of rational
  numbers is a Baer Lie group, but its quotient group $\Q/\Z$ is not a
  Baer Lie group.
\item {\em Direct products}: An arbitrary direct product of Baer Lie
  groups is a Baer Lie group.
\end{itemize}

Similar observations hold for Baer Lie rings:

\begin{itemize}
\item {\em Subrings}: A subring of a Baer Lie ring need not be a Baer
  Lie ring. For instance, the abelian Lie ring with additive group
  $\Q$ but its subring with additive group $\Z$ is not a Baer Lie ring.
\item {\em Quotient rings}: A quotient ring of a Baer Lie ring need
  not be a Baer Lie ring. For instance, the abelian Lie ring with
  additive group $\Q$ is a Baer Lie ring, but the quotient ring with
  additive group $\Q/\Z$ is not a Baer Lie ring.
\item {\em Direct products}: An arbitrary direct product of Baer Lie
  rings is a Baer Lie ring.
\end{itemize}

Due to the above considerations, we need to impose some restrictions
on the nature of the subgroup and nature of the quotient group in
order to use the Baer correspondence to obtain a correspondence
between subgroups and subrings, or between quotient groups and
quotient rings.

Call a subgroup $H$ of a Baer Lie group $G$ a {\em Baer Lie subgroup}
if $H$ is $2$-powered. In particular, this means that $H$ is a Baer
Lie group in its own right. Note also that for a normal subgroup $H$
of $G$, $H$ is a Baer Lie subgroup if and only if the quotient group
$G/H$ is a Baer Lie group (this follows from the two-out-of-three
theorem, Theorem \ref{thm:two-out-of-three}). In this case, we say
that $G/H$ is a {\em Baer Lie quotient group} of $G$.

Call a subring $M$ of a Baer Lie ring $L$ a {\em Baer Lie subring} if
$M$ is $2$-powered. In particular, this means that $M$ is a Baer Lie
ring in its own right.  Note also that for an ideal $I$ of $L$, $I$ is
a Baer Lie subring if and only if the quotient ring $L/I$ is a Baer
Lie ring (this follows from Theorem
\ref{lemma:two-out-of-three-lie}). In these equivalent cases, we will
say that $I$ is a {\em Baer Lie ideal} of $L$ and $L/I$ is a {\em Baer
  Lie quotient ring} of $L$.

\begin{itemize}
\item {\em Baer Lie subgroups correspond to Baer Lie subrings}:
  Suppose a Baer Lie ring $L$ is in Baer correspondence with a Baer
  Lie group $G$, i.e., $L = \log(G)$ and $G = \exp(L)$. Then, for
  every Baer Lie subgroup $H$ of $G$, $\log(H)$ is a subring of $L$,
  and the inclusion map of $\log(H)$ in $L$ is obtained by applying
  the $\log$ functor to the inclusion map of $H$ in $G$. In the
  opposite direction, for every Baer Lie subring $M$ of $L$, $\exp(M)$
  is a subgroup of $G$, and the inclusion map of $\exp(M)$ in $G$ is
  obtained by applying the $\exp$ functor to the inclusion map of $M$
  in $L$. The Baer correspondence thus gives rise to a correspondence:

  \begin{center}
    Baer Lie subgroups of $G$ $\leftrightarrow$ Baer Lie subrings of $L$
  \end{center}

\item {\em Quotient groups by normal Baer Lie subgroups correspond to
  quotient rings by Baer Lie ideals}: Suppose a Baer Lie ring $L$ is
  in Baer correspondence with a Baer Lie group $G$. Then, for every
  normal Baer Lie subgroup $H$ of $G$, $\log(G/H)$ is a quotient Lie
  ring of $L$, and the quotient map $L \to \log(G/H)$ is obtained by
  applying the $\log$ functor to the quotient map $G \to G/H$. In the
  opposite direction, for every Baer Lie ideal $I$ of $L$, $\exp(L/I)$
  is a quotient group of $G$, and the quotient map $G \to \exp(L/I)$
  is obtained by applying the $\exp$ functor to the quotient map $L
  \to L/I$. The Baer correspondence thus gives rise to
  correspondences:

  \begin{center}
    Normal Baer Lie subgroups of $G$ $\leftrightarrow$ Baer Lie ideals of $L$
  \end{center}
  
  \begin{center}
    Baer Lie quotient groups of $G$ $\leftrightarrow$ Baer Lie
    quotient rings of $L$
  \end{center}

\item {\em Direct products correspond to direct products}: Suppose $I$
  is an indexing set, and $G_i, i \in I$ is a collection of Baer Lie
  groups. For each $i \in I$, let $L_i = \log(G_i)$. Then, the
  external direct product $\prod_{i \in I} L_i$ is in Baer
  correspondence with the external direct product $\prod_{i \in I}
  G_i$. Moreover, the projection maps from the direct product to the
  individual direct factors are in Baer correspondence. Also, the
  inclusion maps of each direct factor in the direct product are in
  Baer correspondence.
\end{itemize}


\subsection{Twisted subgroups and subrings}

We introduce the notion of {\em twisted subgroup} of a group. Our
definition differs somewhat from the definition found in the
literature (see, for instance, \cite{Foguelinv}) in that our
definition requires twisted subgroups to be closed under taking
inverses. The condition is not stated in \cite{Foguelinv} because that
paper, and the other related literature, are interested primarily in
finite groups, where the condition is redundant.

\begin{definer}[Twisted subgroup]
  A subset $K$ of a group $G$ is termed a {\em twisted subgroup} if it
  satisfies the following three conditions:

  \begin{enumerate}
  \item For any $x,y \in K$ (possibly equal, possibly distinct), $xyx
    \in K$.
  \item The identity element of $G$ is in $K$.
  \item For any $x \in K$, $x^{-1} \in K$.
  \end{enumerate}

  Note that Conditions (1) and (2) imply closure under taking positive
  powers, so if every element of $K$ has finite order in $G$,
  condition (3) is redundant.
\end{definer}

We will call a twisted subgroup $K$ of a group $G$ {\em powered} over
a set of primes $\pi$ if the map $x \mapsto x^p$ is a bijection from
$K$ to itself for every $p \in \pi$.

We had earlier defined the addition in a Baer Lie group as follows:

$$x + y = \frac{xy}{\sqrt{[x,y]}}$$

As described in Section \ref{sec:twisted-multiplication}, we can
rewrite this as:

$$x + y = \sqrt{x} y \sqrt{x}$$

It follows that under the Baer correspondence, we have a correspondence:

\begin{center}
  $2$-powered twisted subgroups of the Baer Lie group $\leftrightarrow$ $2$-powered additive subgroups of the Baer Lie ring
\end{center}

\subsection{Characteristic subgroups and subrings}\label{sec:baer-correspondence-char-sub}

In Section \ref{sec:baer-correspondence-sub-quot-dp}, we noted that
the Baer correspondence between a Baer Lie group $G$ and a Baer Lie
ring $L$ induces a correspondence:

\begin{center}
  Baer Lie subgroups of $G$ $\leftrightarrow$ Baer Lie subrings in $L$
\end{center}

Recall that a Baer Lie subgroup of a Baer Lie group is simply a
$2$-powered subgroup. Similarly, a Baer Lie subring of a Baer Lie ring
is simply a $2$-powered subring.

In Section \ref{sec:baer-correspondence-isocat-consequences}, we noted
that the Baer correspondence preserves automorphisms groups, i.e.,
$\operatorname{Aut}(G)$ and $\operatorname{Aut}(L)$ define the same
collection of permutations on the underlying set. Thus, the above
correspondence preserves the property of being invariant under
automorphisms, and we obtain a correspondence:

\begin{center}
  Characteristic Baer Lie subgroups of $G$ $\leftrightarrow$
  Characteristic Baer Lie subrings of $L$
\end{center}

Note that, combined with the fact that normal Baer Lie subgroups of
$G$ correspond with Baer Lie ideals of $L$, this tells us that any
characteristic Baer Lie subring of $L$ (i.e., any $2$-powered
characteristic subring of $L$) is an ideal of $L$. Note that this is a
nontrivial statement about the structure of $L$, since there do exist
Lie rings that have characteristic subrings that are not ideals.

Conjecture \ref{conj:charpowering} (respectively, Conjecture
\ref{conj:charpowering-lie}) stated that for a prime set $\pi$, any
characteristic subgroup (respectively, characteristic Lie subring) in
a $\pi$-powered nilpotent group (respectively, $\pi$-powered nilpotent
Lie ring) must be $\pi$-powered. We can consider restricted versions
of Conjectures \ref{conj:charpowering} and \ref{conj:charpowering-lie}
to the case of Baer Lie groups and Baer Lie rings respectively. If
both restricted conjectures are true, then we have a correspondence:

\begin{center}
  Characteristic subgroups of $G$ $\leftrightarrow$ Characteristic
  subrings of $L$
\end{center}

Analogous remarks to the above remarks for characteristic subgroups
and subrings apply to the case of fully invariant subgroups and
subrings. Note that the conjecture that would be necessary for fully
invariant subgroups and subrings is implied by the conjecture for
characteristic subgroups.


\subsection{The $p$-group case}\label{sec:baer-p-group-case}

Let $p$ be an odd prime. We will use the term {\em $p$-group} for a
group (not necessarily finite) in which the order of every element is
a power of $p$. We will use the term {\em $p$-Lie ring} for a Lie ring
whose additive group is a $p$-group. The Baer correspondence restricts
to a correspondence:

\begin{center}
  $p$-groups of nilpotency class (at most) two $\leftrightarrow$ $p$-Lie rings of nilpotency class (at most) two
\end{center}

The square root operation in this case corresponds to a powering
operation. Explicitly, if $g$ is an element of a $p$-group $G$, the
order of $g$ is a prime power $p^k$. $\sqrt{g}$ equals $g^{(p^k +
  1)/2}$. In particular, it is a positive integral power of $g$.

Thus, {\em every} subset of a $p$-group that is closed under taking
positive powers is also closed under taking square roots. In
particular, {\em every} subgroup of a $p$-group is $2$-powered, and
{\em every} twisted subgroup of a $p$-group is $2$-powered.

Thus, in the case of $p$-groups for odd primes $p$, the
correspondences discussed in the preceding sections are particularly
easy. The correspondence between $2$-powered subgroups and $2$-powered
subrings becomes a correspondence:

\begin{center}
  Subgroups of $G$ $\leftrightarrow$ Subrings of $L$
\end{center}

We also obtain correspondences:

\begin{itemize}
\item Normal subgroups of $G$ $\leftrightarrow$ Ideals of $L$
\item Quotient groups of $G$ $\leftrightarrow$ Quotient rings of $L$
\item Characteristic subgroups of $G$ $\leftrightarrow$ Characteristic
  subrings of $L$
\item Fully invariant subgroups of $G$ $\leftrightarrow$ Fully
  invariant subrings of $L$
\end{itemize}



The correspondence between $2$-powered twisted subgroups and
$2$-powered additive subgroups of the Lie ring becomes a
correspondence:

\begin{center}
  Twisted subgroups of $G$ $\leftrightarrow$ Subgroups of the additive
  group of $L$
\end{center}

\subsection{Relation between the Baer correspondence and the abelian Lie correspondence}\label{sec:baer-correspondence-abelian-lie-correspondence-relation}

In an earlier section (Section
\ref{sec:abelian-lie-correspondence}), we introduced the abelian Lie
correspondence:

\begin{center}
  Abelian groups $\leftrightarrow$ Abelian Lie rings
\end{center}

In the preceding section (Section
\ref{sec:baer-correspondence-basics}), we introduced the Baer
correspondence:

\begin{center}
  Baer Lie groups $\leftrightarrow$ Baer Lie rings
\end{center}

We used the same symbols ($\log$ and $\exp$) to describe the functors
for both correspondences. This leads to a potential for ambiguity:
what happens if a group happens to be both an abelian group {\em and}
a Baer Lie group? It turns out that in this case, the two
correspondences agree. Explicitly:

\begin{itemize}
\item Suppose $G$ is a 2-powered abelian group, i.e., $G$ is {\em
  both} an abelian group and a Baer Lie group. Then, the two
  definitions of $\log G$ (based on the abelian Lie correspondence and
  Baer correspondence respectively) agree with each other.
\item Suppose $L$ is a 2-powered abelian Lie ring, i.e., $L$ is {\em
  both} an abelian Lie ring and a Baer Lie ring. Then, the two
  definitions of $\exp L$ (using the abelian Lie correspondence and
  Baer Lie correspondence) agree with each other.
\item Suppose $G_1$ and $G_2$ are 2-powered abelian groups and
  $\varphi:G_1 \to G_2$ is a homomorphism of groups. Then, the two
  definitions of $\log(\varphi)$ (based on the abelian Lie
  correspondence and Baer correspondence respectively) agree.
\item Suppose $L_1$ and $L_2$ are 2-powered abelian Lie rings and
  $\varphi:L_1 \to L_2$ is a homomorphism of Lie rings. Then, the two
  definitions of $\exp(\varphi)$ (based on the abelian Lie
  correspondence and Baer correspondence respectively) agree.
\end{itemize}

Every time we introduce a new correspondence between groups and Lie
rings, we will attempt to verify whether it is compatible with
existing correspondences. Checking compatibility will reduce to the
question of checking whether the formulas agree with each other in the
case of overlap. For illustrative purposes, consider the proof that
for a 2-powered abelian Lie ring $L$, the two definitions of $\exp L$
agree. The definitions of group multiplication in terms of the Lie
ring operations:

$$x + y \text{ for the abelian Lie correspondence}, \qquad x + y + \frac{1}{2}[x,y] \text{ for the Baer Lie correspondence}$$

These formulas agree for a 2-powered abelian group.

\subsection{Abelian subgroups, abelian quotient groups, center and derived subgroup}\label{sec:baer-correspondence-center-and-derived}

Suppose a group $G$ is in Baer correspondence with a Lie ring $L$. We
know that $G$ and $L$ have the same underlying set, and the commutator
map in $G$ coincides with the Lie bracket in $L$. From this, we can
deduce the following:

\begin{enumerate}
\item The abelian subgroups of $G$ are in abelian Lie correspondence
  with the abelian subrings of $L$, and this abelian Lie
  correspondence arises by restricting the Baer correspondence between
  $G$ and $L$.

  \begin{center}
    Abelian subgroups of $G$ $\leftrightarrow$ Abelian subrings of $L$
  \end{center}

  Note that for the $2$-powered abelian subgroups and $2$-powered
  abelian subrings, this coincides with the Baer correspondence.

  The correspondence also restricts to a correspondence:

  \begin{center}
    Abelian normal subgroups of $G$ $\leftrightarrow$ Abelian ideals of $L$
  \end{center}

  and to a subcorrespondence:

  \begin{center}
    Abelian characteristic subgroups of $G$ $\leftrightarrow$ Abelian
    characteristic subrings of $L$
  \end{center}

  Finally, the correspondence gives rise to a correspondence:

  \begin{center}
    Quotient groups of $G$ by abelian normal subgroups $\leftrightarrow$ Quotient rings of $L$ by abelian ideals
  \end{center}

  Note, however, that the instances of this last correspondence are
  {\em not} instances of the Baer correspondence. They do, however,
  form instances of the divided Baer correspondence described in
  Section \ref{sec:divided-baer-correspondence}.
\item For a subset $S$ of the common underlying set of $G$ and $L$,
  the subgroup generated by $S$ in $G$ is abelian if and only if the
  subring generated by $S$ in $L$ is abelian, and if so, the subgroup
  and subring are in abelian Lie correspondence.
\item The abelian quotient groups of $G$ are in abelian Lie
  correspondence with the abelian quotient Lie rings of $L$. Explicitly:

  \begin{center}
    Abelian quotient groups of $G$ $\leftrightarrow$ Abelian quotient Lie rings of $L$
  \end{center}

  We obtain a correspondence between the corresponding
  kernels. Explicitly, the Baer correspondence between $G$ and $L$
  restricts to a correspondence via identification of the underlying
  sets:

  \begin{center}
    Subgroups of $G$ containing $G'$ $\leftrightarrow$ Subrings of $L$ containing $L'$
  \end{center}

  Note, however, that each individual instance of this correspondence
  is {\em not} an instance of the Baer correspondence. Rather, it
  would be an instance of the generalization of the Baer
  correspondence described in Section
  \ref{sec:baer-correspondence-lucs-generalization}.
\item The center $Z(G)$ is in abelian Lie correspondence as well as in
  Baer correspondence with the center $Z(L)$. Note that the claim has
  two parts:

  \begin{enumerate}
  \item $Z(G)$ and $Z(L)$ have the same underlying set: This follows
    from the fact that the commutator map of $G$ coincides with the
    Lie bracket map of $L$, so the set of elements whose commutator
    with every element of $G$ is the identity element coincides with
    the set of elements whose Lie bracket with every element of $L$ is
    the zero element.
  \item Both $Z(G)$ and $Z(L)$ are $2$-powered, so that we can apply
    the Baer correspondence: This follows from Lemmas
    \ref{lemma:centerispoweringinvariant} and
    \ref{lemma:centerispoweringinvariant-lie}.
  \end{enumerate}

\item The inner automorphism group $G/Z(G)$ is in abelian Lie
  correspondence as well as in Baer correspondence with the inner
  derivation Lie ring $L/Z(L)$. This follows from the preceding point
  about $Z(G)$ being in correspondence with $Z(L)$ and the
  observations made regarding subgroups and quotients in Section
  \ref{sec:baer-correspondence-sub-quot-dp}.
\item The derived subgroup $G'$ is in abelian Lie correspondence as
  well as in Baer correspondence with the derived subring $L'$. Note
  that the claim has two parts:

  \begin{enumerate}
  \item $G'$ and $L'$ have the same underlying set: The commutator map
    in $G$ coincides with the Lie bracket map in $L$, so the set of
    commutators in $G$ coincides with the set of elements in $L$ that
    can be expressed as Lie brackets. We know that both $G'$ is an
    abelian group and $L'$ is an abelian Lie ring, because both $G$
    and $L$ have class two.  By point (2) above, $G'$ and $L'$ are in
    abelian Lie correspondence and have the same underlying set.
  \item $G'$ and $L'$ are both $2$-powered: This follows from Theorem
    \ref{thm:powering-lcs} and Lemma
    \ref{lemma:lie-ring-lcs-divisibility} respectively.
  \end{enumerate}
\item The abelianization $G^{\operatorname{ab}} = G/G'$ is in abelian
  Lie correspondence as well as in Baer correspondence with the
  abelianization $L^{\operatorname{ab}} = L/L'$. This follows from the
  preceding point and the observations regarding subgroups and
  quotients in Section \ref{sec:baer-correspondence-sub-quot-dp}.
\item The abelian subgroups of $G$ that contain $G'$ and are contained
  in $Z(G)$ are in abelian Lie correspondence with the abelian
  subgroups of $L$ that contain $L'$ and are contained in
  $Z(L)$. Moreover, the corresponding quotient group is in abelian Lie
  correspondence with the corresponding quotient ring.
\end{enumerate}

\subsection{Cyclic subgroups, preservation of element orders, and the parallels with one-parameter subgroups}\label{sec:baer-correspondence-cyclic-sub}

We noted in Section \ref{sec:baer-correspondence-center-and-derived}
that given a group $G$ and a Lie ring $L$ that are in Baer
correspondence, we obtain a correspondence, with each instance also an
instance of the abelian Lie correspondence:

\begin{center}
  Abelian subgroups of $G$ $\leftrightarrow$ Abelian subrings of $L$
\end{center}

In particular, we get a correspondence:

\begin{center}
  Cyclic subgroups of $G$ $\leftrightarrow$ Cyclic subrings of $L$
\end{center}

Another way of framing this is that the additive group of $L$ and the
group $G$ have the same cyclic subgroup structure. In other words, the
map $\exp: L \to G$ restricts to an isomorphism on cyclic subgroups of
the additive group of $L$. Some of the generalizations of the Baer
correspondence, including all those described in the next section,
continue to have this property, so that the remarks of the next
paragraph apply to these generalizations.

For any element of the common underlying set, its order as an element
of the additive group of $L$ is the same as its order as an element of
$G$. In the case that the common underlying set of $G$ and $L$ is
finite, this translates to the requirement that for every positive
integer $d$, the number of elements of $G$ of order $d$ equals the
number of elements of the additive group of $L$ as order $d$. In
particular, we obtain that the order {\em statistics} of $G$
(information about how many elements are there of any given order)
coincide with the order statistics of some abelian group. When
considering whether a group $G$ can participate in a potential
generalization of the Baer correspondence, this condition serves as a
potential filter.

The Baer correspondence for cyclic subgroups and subrings is closely
related to the correspondence between one-dimensional subspaces of the
real Lie algebra and one-parameter subgroups of the real Lie group
under the Lie correspondence described in Section
\ref{sec:lie-correspondence-intro} (we did not describe one-parameter
subgroups there).

\subsection{Inner automorphisms and inner derivations}\label{sec:baer-adjoint}

Supose $G$ is a Baer Lie group and $L$ is the corresponding Baer Lie
ring. Under the Baer correspondence, $Z(G)$ is in Baer correspondence
(and hence also in abelian Lie correspondence) with $Z(L)$, and the
quotient group $G/Z(G) \cong \operatorname{Inn}(G)$ is in Baer
correspondence (and hence also in abelian Lie correspondence) with the
quotient Lie ring $L/Z(L) \cong \operatorname{Inn}(L)$. The former
gives the inner automorphisms of the group $G$ (that are hence also
automorphisms of $L$). The latter gives the inner derivations of $L$.

Consider an element $x \in G$ with image $\overline{x}$ in the common
underlying set of $G/Z(G)$ and $L/Z(L)$. We would like to understand
the relationship between these two set maps from $L$ to $L$:

\begin{itemize}
\item The inner automorphism corresponding to $\overline{x}$, i.e.,
  the map $g \mapsto xgx^{-1}$, now viewed as a automorphism {\em of}
  $L$ as a Lie ring. As noted in Section
  \ref{sec:baer-correspondence-isocat-consequences}, the automorphisms
    of $L$ as a Lie ring coincide with the automorphisms of $G$ as a
    group. We will denote this map as $\operatorname{Ad}_x$.
\item The inner derivation corresponding to $\overline{x}$, i.e., the
  map $g \mapsto [x,g]$ where $[ \ , \ ]$ denotes the Lie bracket. We
  will denote this map as $\operatorname{ad}_x$. Derivations and inner
  derivations are discussed in more detail in the Appendix, Sections
  \ref{appsec:derivation} and \ref{appsec:derivation-lie}.
\end{itemize}

We first work out the set map $\operatorname{Ad}_x$ in terms of the
Lie ring operations.

$$\operatorname{Ad}_x(g) = xgx^{-1} = \left(x + g + \frac{1}{2}[x,g]\right)x^{-1}$$

This simplifies to:

$$\operatorname{Ad}_x(g) = x + g + \frac{1}{2}[x,g] + (-x) + \left[\left(x + g + \frac{1}{2}[x,g]\right),-x\right]$$

This simplifies to:

$$\operatorname{Ad}_x(g) = g + [x,g]$$

or equivalently:

$$\operatorname{Ad}_x(g) = g + \operatorname{ad}_x(g)$$

If we view both $\operatorname{Ad}_x$ and $\operatorname{ad}_x$ as
elements of the ring $\operatorname{End}_\Z(L)$ of endomorphisms of
the underlying {\em additive group} of $L$, then we can write the
above relationship as:

$$\operatorname{Ad}_x = 1 + \operatorname{ad}_x$$

The expression for $\operatorname{Ad}_x$ in terms of
$\operatorname{ad}_x$ is a truncated form of the power series for the
exponential function. It turns out that this is not a coincidence. We
will see in Section \ref{sec:lazard-adjoint} that for the Malcev and
  Lazard correspondences, $\operatorname{Ad}_x =
  \exp(\operatorname{ad}_x)$.

\section{Generalizations of the Baer correspondence: relaxing the definitions}\label{sec:baer-correspondence-definition-relaxation}

In the construction of the Baer Lie ring and Baer Lie group from one
another, we did not use the existence of unique square roots in the
whole group or Lie ring. Rather, we only made use of the fact that we
be able to make sense of the expression $\sqrt{[x,y]}$ when going from
groups to Lie rings and of the expression $\frac{1}{2}[x,y]$ when
going from Lie rings to groups, and further, that the outputs of these
expressions land in the respective centers.

\subsection{Generalization that allows for division within the lower central series}\label{sec:baer-correspondence-lcs-generalization}

This generalization is a correspondence:

Groups of nilpotency class (at most) two where the derived subgroup is
$2$-powered $\leftrightarrow$ Lie rings of nilpotency class (at most)
two where the derived subring is $2$-powered

Note that this correspondence has the advantage of including as
subcorrespondences {\em both} the abelian Lie correspondence {\em and}
the Baer correspondence.

This correspondence behaves nicely in a number of ways:

\begin{itemize}
\item {\em Isomorphism of categories}: We can define $\log$ and $\exp$
  functors and obtain an isomorphism of categories between the full
  subcategories of the category of groups and category of Lie rings as
  described above. We can then deduce consequences similar to those
  described for the Baer correspondence in Section
  \ref{sec:baer-correspondence-isocat-consequences}.
\item {\em Subgroups and subrings}: A subgroup of a group in this
  subcategory need not be in the subcategory. However, if we do
  restrict to subgroups that satisfy the condition for being in the
  category, we can deduce results analogous to the results stated for
  subgroups in Section \ref{sec:baer-correspondence-sub-quot-dp}. Note
  in particular that the correspondence between subgroups and subrings
  here includes two subcorrespondences:

  \begin{center}
    Abelian subgroups $\leftrightarrow$ Abelian subrings
  \end{center}

  \begin{center}
    Baer Lie subgroups $\leftrightarrow$ Baer Lie subrings
  \end{center}
\item {\em Direct products}: Direct products on the group side
  correspond with direct products on the Lie ring side. The statement
  is similar to that for the abelian Lie correspondence (as described
  in Section \ref{sec:abelian-lie-correspondence-sub-quot-dp}) and the
  Baer correspondence (as described in Section
  \ref{sec:baer-correspondence-sub-quot-dp}).
\end{itemize}

However, the categories in question are {\em not} well-behaved with
respect to the relation between subgroups and quotients. Explicitly,
it is possible to have a group in the category and a normal subgroup
that is also in the category, but such that the quotient group is not
in the category. For instance, consider the case that $G = UT(3,\Q)$
and $H$ is a copy of $\Z$ in the center. Then, both $G$ and $H$ are
objects of the category, and $H$ is normal in $G$, but the quotient
group $G/H$ is not an object of the category.

\subsection{Generalization that allows for division starting in the lower central series and ending in the upper central series}\label{sec:baer-correspondence-lucs-generalization}

This generalization is a correspondence:

Groups of nilpotency class (at most) two where every element of the
derived subgroup has a unique square root in the center
$\leftrightarrow$ Lie rings of nilpotency class (at most) two where
every element of the derived subring has a unique half in the center.

Note that the ``unique square root in the center'' clause could be
interpreted in two ways: one could understand it to mean that there is
a unique square root in the whole group that happens to be in the
center, or one could understand it to mean that there is a unique
square root among the elements in the center. However, for nilpotent
groups, these are equivalent. If every element of the derived subgroup
has a unique square root among elements of the center, then in
particular the identity element has a unique square root among
elements in the center, so the center is $2$-torsion-free. Thus. by
Theorem \ref{thm:torsion-free-equivalence-theorem}, the whole group is
$2$-torsion-free, so the squaring map is injective on the whole group.

Similarly, both ways of interpreting ``unique half in the center'' on
the Lie ring side are equivalent to each other.

Note also that by Lemma \ref{lemma:centerispoweringinvariant}, the
condition on the group side can be reformulated as ``group of
nilpotency class (at most) two where every element of the derived
subgroup has a unique square root in the whole group.'' A similar
reformulation is possible on the Lie ring side by Lemma
\ref{lemma:centerispoweringinvariant-lie}.

The correspondence is nice in a number of ways:

\begin{itemize}
\item {\em Isomorphism of categories}: We can define $\log$ and $\exp$
  functors and obtain an isomorphism of categories between the full
  subcategories of the category of groups and category of Lie rings as
  described above. We can then deduce consequences similar to those
  described for the Baer correspondence in Section
  \ref{sec:baer-correspondence-isocat-consequences}.
\item {\em Subgroups and subrings}: A subgroup of a group in this
  subcategory need not be in the subcategory. However, if we do
  restrict to subgroups that satisfy the condition for being in the
  category, we can deduce results analogous to the results stated for
  subgroups in Section \ref{sec:baer-correspondence-sub-quot-dp}.
\item {\em Direct products}: Direct products on the group side
  correspond with direct products on the Lie ring side. The statement
  is similar to that for the abelian Lie correspondence (as described
  in Section \ref{sec:abelian-lie-correspondence-sub-quot-dp}) and the
  Baer correspondence (as described in Section
  \ref{sec:baer-correspondence-sub-quot-dp}).
\end{itemize}

However, the categories in question are {\em not} well-behaved with
respect to the relation between subgroups and quotients. Explicitly,
it is possible to have a group in the category and a normal subgroup
that is also in the category, but such that the quotient group is not
in the category. For instance, consider the case that $G = UT(3,\Q)$
and $H$ is a copy of $\Z$ in the center. Then, both $G$ and $H$ are
objects of the category, and $H$ is normal in $G$, but the quotient
group $G/H$ is not an object of the category.

\subsection{Incomparability of the generalizations}

Neither of the two preceding generalizations contains the other. Explicitly:

\begin{itemize}
\item Any abelian group with $2$-torsion would be covered under the
  first generalization (described in Section
  \ref{sec:baer-correspondence-lcs-generalization}) but not the second
  (described in Section
  \ref{sec:baer-correspondence-lucs-generalization}).
\item An example of a group that would be covered under the second
  generalization but not the first is the group $UT(3,\mathbb{Z})
  *_{\mathbb{Z}} \mathbb{Q}$ where $*$ denotes the central product where
  we identify the center of $UT(3,\mathbb{Z})$ with a copy of
  $\mathbb{Z}$ in $\mathbb{Q}$.
\end{itemize}

There are generalizations that are strictly more general than both the
above generalizations. We consider one such generalization below.

\subsection{The divided Baer correspondence: a generalization that uses additional structure}\label{sec:divided-baer-correspondence}

This generalization of the Baer correspondence involves specifying
additional structure on the Lie ring and on the group. {\em We use new
  terminology in this section. However, the results of this section
  are not necessary for our main results, and the terminology used
  here is not required for our main results.}

\begin{definer}[Baer-divided Lie group]
  A {\em Baer-divided Lie group} is a group $G$ of nilpotency class at
  most two equipped with an alternating $\Z$-bilinear map $\{ , \}:G
  \times G \to G$ (i.e., the map is a homomorphism in each coordinate
  holding the other coordinate fixed, and $\{ x,x \} = 1$ for all $x
  \in G$) such that the following hold:

  \begin{enumerate}
  \item $\{ x,y \}^2 = [x,y]$ for all $x,y \in G$, where $[x,y]$
    denotes the group commutator.
  \item $\{ \{x,y \} , z \}$ is the identity element of $G$ for all
    $x,y,z \in G$.
  \end{enumerate}
\end{definer}

\begin{definer}[Baer-divided Lie ring]
  A {\em Baer-divided Lie ring} is a Lie ring $L$ (with Lie bracket
  denoted $[ \ , \ ]$) equipped with an alternating $\Z$-bilinear map
  $\{ , \}: L \times L \to L$ such that the following hold:

  \begin{enumerate}
  \item $2\{ x, y \} = [x,y]$ for all $x,y \in L$.
  \item $\{ , \}$ also defines a Lie bracket on the additive group of
    $L$, and the corresponding Lie ring has nilpotency class two.
  \end{enumerate}

  In particular, the original Lie ring $L$ with Lie bracket $[ \ ,
    \ ]$ is also a Lie ring of nilpotency class at most two.
\end{definer}


We can now define the {\em divided Baer correspondence} as a
correspondence:

\begin{center}
  Baer-divided Lie groups $\leftrightarrow$ Baer-divided Lie rings
\end{center}

In the direction from the group to the Lie ring, the correspondence
uses the formula:

$$x + y := \frac{xy}{\{ x,y \}}$$

and:

$$[x,y] = [x,y]_{\text{Group}}$$

In the direction from the Lie ring to the group, the correspondence
uses the formula:

$$xy := x + y + \{ x, y \}$$

In other words, the roles of $\frac{1}{2}[x,y]$ and $\sqrt{[x,y]}$ are
taken over by the operation $\{x, y\}$ (of the group or the Lie
ring). The key difference is that this operation is an {\em additional
  structure specified} rather than being purely dependent on the group
or Lie ring as an abstract structure. 

The Baer correspondence and the generalizations of it described
earlier can be reframed in terms of the divided Baer correspondence as
follows:

\begin{enumerate}
\item In the case of the Baer correspondence, as well as in the
  generalization described in Section
  \ref{sec:baer-correspondence-lucs-generalization}, there is a {\em
    unique} possibility for $\{ x, y \}$. In fact, in the Baer
  correspondence as well as the generalization, the uniqueness is at
  the level of {\em elements}: every element of the form $[x,y]$ has a
  unique square root or half. Note that there could be situations that
  fall outside this generalization where there is a unique possibility
  for $\{ x,y \}$, even though individual elements in the derived
  subgroup (respectively, derived subring) have non-unique halves, as
  discussed in the example of $UT(3,\Z[1/2]) \times \Z/2\Z$ below.
\item In the case of the generalization described in Section
  \ref{sec:baer-correspondence-lcs-generalization}, the possibility
  for $\{ x, y\}$ need not be unique. Nonetheless, there is a
  particular {\em canonical} choice we can make for $\{ x, y\}$,
  namely the unique option available {\em within} the deried subgroup
  or derived subring.
\end{enumerate}

Below, we describe some examples of the divided Baer correspondence.

\begin{enumerate}
\item The case $G = UT(3,\Q)/\Z$ and $L = NT(3,\Q)/\Z$: $G$ is the
  group described as Example (3) in the counterexample list in Section
  \ref{sec:group-ctex} and $L$ is the Lie ring described in Example
  (3) in the counterexample list in Section \ref{sec:lie-ring-ctex}.

  In this case, $G$ is not a Baer Lie group and $L$ is not a Baer Lie
  ring, because neither $G$ nor $L$ is 2-powered. However, we can give
  $G$ the structure of a Baer-divided Lie group as follows. Note that
  $Z(G) = G'$ is isomorphic to the subgroup $\Q/\Z$ in $G$, and
  $G/Z(G) = G/G'$ is isomorphic to $\Q \times \Q$. The commutator map
  $G/Z(G) \times G/Z(G) \to G'$ can be described as the composite of
  the maps:

  $$(\Q \times \Q) \times (\Q \times \Q) \to \Q \to \Q/\Z$$

  where the map on the left is given by:

  $$((a_1,b_1),(a_2,b_2)) \mapsto a_1b_2 - a_2b_1$$

  and the map $\Q \to \Q/\Z$ is the quotient map.

  There is a canonical choice of half for this map, namely, the composite:

  $$(\Q \times \Q) \times (\Q \times \Q) \to \Q \to \Q/\Z$$

  where the first map is:

  $$((a_1,b_1),(a_2,b_2)) \mapsto \frac{1}{2}(a_1b_2 - a_2b_1)$$

  Note that we circumvent the problem of non-unique halves in the
  derived subgroup or subring by performing the halving in an
  intermediate group (namely $\Q$) through which we factor the map.

  Similar observations hold on the Lie ring side.

\item The case $G = UT(3,\Z[1/2]) \times \Z/2\Z$ and $L =
  NT(3,\Z[1/2]) \times \Z/2\Z$: Note that this case actually falls
  under the earlier generalization described in Section
  \ref{sec:baer-correspondence-lcs-generalization}. One interesting
  feature of this example is that although in general the choice of
  half is non-unique when the center has 2-torsion, in this case,
  there is a unique choice of divided Baer structure. Explicitly, each
  element of the derived subgroup has two halves: a half inside the
  first direct factor, and a half that has a nontrivial second
  coordinate. However, a linear choice of $\{ , \}$ requires that we
  choose each half inside the first direct factor. Similar
  observations hold on the Lie ring side.

\item The case where $G$ and $L$ are as follows:

  $$G = \langle a,b,c \mid a^4 = b^4 = c^4 = 1, ab = ba, ac = ca, [b,c] = a^2 \rangle$$

  $$L = \langle a,b,c \mid 4a = 4a = 4c = 0, [a,b] = 0, [a,c] = 0, [b,c] = 2a \rangle$$

  In this case, $Z(G) = \langle a \rangle$ and $G' = \langle a^2
  \rangle$. $Z(G)$ is isomorphic to $\Z/4\Z$ and $G'$ is the unique
  subgroup of order two. We can define $\{ , \}$ as $\{ b,c \} = a$
  with the rest of the definition following from that. Note that this
  is not the unique choice of $\{ \ , \ \}$, because we could choose
  $\{ b,c \} = a^{-1}$ as well. However, it is a choice that works.

  A similar construction works for $L$, and the divided Baer
  correspondence between $G$ and $L$ works as expected.
\end{enumerate}

\subsection{Category-theoretic perspective on the divided Baer correspondence}

The Baer-divided Lie groups form a category as follows:

\begin{itemize}
\item The objects of the category are Baer-divided Lie groups.
\item For Baer-divided Lie groups $G_1$ and $G_2$, the morphisms from
  $G_1$ to $G_2$ are group homomorphisms $\varphi:G_1 \to G_2$
  satisfying the additional condition $\varphi(\{x,y \}) =
  \{\varphi(x),\varphi(y) \}$ for all $x,y \in G_1$. Note that in the
  case that $\{ \ , \ \}$ is defined canonically in terms of the group
  operations, the additional condition is satisfied for {\em all}
  group homomorphisms.
\end{itemize}

We can similarly define a category of Baer-divided Lie rings. The
divided Baer correspondence defines an isomorphism of categories
between the category of Baer-divided Lie groups and the category of
Baer-divided Lie rings.

There is a natural forgetful functor from the category of Baer-divided
Lie groups to the category of groups. This functor forgets the $\{ \ ,
\ \}$-structure and simply stores the underlying group structure. This
forgetful functor is faithful but not full: there may well be
homomorphisms between groups that do not preserve the $\{ \ ,
\ \}$-structure. The functor is not injective or surjective on
objects. It is not injective because there exist groups with multiple
possibilities for $\{ \ , \ \}$. It is not surjective, even to the
subcategory of groups of nilpotency class two, because there exist
such groups for which there is no possible $\{ \ , \ \}$-structure.

The following are true.

\begin{itemize}
\item If $G$ is a Baer Lie group, or more generally, a group that fits
  the generalization described in Section
  \ref{sec:baer-correspondence-lucs-generalization}, then the functor
  is injective to $G$, i.e., there is a unique Baer-divided Lie group
  structure on $G$. Further, if $G_1$ and $G_2$ are two such groups,
  then all homomorphisms between them are realized as Baer-divided Lie
  group homomorphisms, so the functor behaves as a full functor if we
  restrict to such groups.
\item If $G$ fits the generalization described in Section
  \ref{sec:baer-correspondence-lcs-generalization}, then the functor
  need not be injective to $G$, but there does exist a canonical
  choice of Baer-divided Lie group structure on $G$. In other words,
  if we restrict attention to the subcategory of groups described in
  Section \ref{sec:baer-correspondence-lcs-generalization}, we can
  obtain a functor from this subcategory to the category of
  Baer-divided Lie groups that is a one-sided inverse to the forgetful
  functor.
\end{itemize}

Analogous observations hold on the Lie ring side.

\subsection{The case of finite $p$-groups}

Note that for odd primes $p$, $p$-groups of class two and $p$-Lie
rings of class two fall in the domain of the Lazard correspondence,
and we do not need to rely on any of the generalizations. The case $p
= 2$ is interesting. In this case, the following are true:

\begin{itemize}
\item The generalization described in Section
  \ref{sec:baer-correspondence-lcs-generalization} applies only to
  abelian $2$-groups and abelian $2$-Lie rings, and not to any others.
\item The generalization described in Section
  \ref{sec:baer-correspondence-lucs-generalization} does not apply to
  any nontrivial $2$-groups or nontrivial $2$-Lie rings, because the
  center is nontrivial and hence is not $2$-powered.
\item The divided Baer correspondence generalization applies to some
  but not all $2$-groups of class two. Explicitly, a necessary but not
  sufficient condition for a group $G$ to be the underlying group of a
  Baer-divided Lie group is that $G' \subseteq \mho^1(Z(G))$, where
  $\mho^1(Z(G))$ denotes the set of squares of elements in $Z(G)$. A
  similar necessary but not sufficient condition exists on the Lie
  ring side.
\end{itemize}

%\newpage

\section{Baer correspondence up to isoclinism}\label{sec:bcuti}

The concept of ``Baer correspondence up to isoclinism'' is a novel
contribution of this thesis. Many of the results in this section are
based on joint work with John Wiltshire-Gordon.

\subsection{Motivation}

In Section \ref{sec:baer-correspondence-center-and-derived}, we noted
that if $G$ and $L$ are in Baer correspondence, the subgroups of $G$
that contain $G'$ and are contained in $Z(G)$ are in abelian Lie
correspondence with the subrings of $L$ that contain $L'$ and are
contained in $Z(L)$. Further, for each such subgroup and subring in
correspondence, the associated quotient group and quotient ring are
also in abelian Lie correspondence. In other words, we can build the
group $G$ as a central extension of groups, and the Lie ring $L$ as a
central extension of Lie rings, where the central subgroup of $G$ is
in abelian Lie correspondence with the central subring of $L$, and the
quotient group of $G$ is in abelian Lie correspondence with the
quotient ring of $L$.

We describe two extreme cases below.

\begin{enumerate}
\item The case where the subgroup is $Z(G)$ and the subring is
  $Z(L)$. We can depict this case as follows:

  $$\begin{array}{lllll}
    0 \to & Z(G) \to & G \to & G/Z(G) \to & 1 \\
    & \downarrow^{\log} & \downarrow^{\log} & \downarrow^{\log}&  \\
    0  \to & Z(L) \to & L \to & L/Z(L) \to & 0\\
  \end{array}$$

  Note that the $\log$ connecting $G$ and $L$ arises from the Baer
  correspondence. The $\log$ connecting $Z(G)$ with $Z(L)$, and the
  $\log$ connecting $G/Z(G)$ with $L/Z(L)$, can be viewed as arising
  both from the Baer correspondence and from the abelian Lie
  correspondence, as described in Section
  \ref{sec:baer-correspondence-center-and-derived}.
\item The case where the subgroup is $G'$ and the subring is $L'$. We
  can depict this case as follows:

  $$\begin{array}{lllllllll}
    0 & \to & G' & \to & G & \to & G/G' \to & 1 \\
    && \downarrow^{\log} && \downarrow^{\log} && \downarrow^{\log}&  \\
    0  & \to & L' & \to & L & \to & L/L' \to & 0\\
  \end{array}$$
 
  An analogous observation to the note in point (1) above applies
  here, with $Z(G)$, $Z(L)$, $G/Z(G)$, and $L/Z(L)$ replaced by $G'$,
  $L'$, $G/G'$, and $L/L'$ respectively.
\end{enumerate}

Our aim is to generalize the Baer correspondence to a situation where
the middle downward arrow (connecting $G$ with $L$) is missing, but we
still have an abelian Lie correspondence between the central subgroup
and central subring, an abelian Lie correspondence between the
quotient group and quotient ring, and a condition to the effect that
the commutator map on the group side looks the same as the Lie bracket
map on the Lie ring side.

\subsection{Definition of the correspondence}\label{sec:bcuti-def}

The Baer correspondence up to isoclinism is a correspondence that we
will define between the following two sets:

Equivalence classes up to isoclinism of groups of nilpotency class at
most two $\leftrightarrow$ Equivalence classes up to isoclinism of Lie rings of nilpotency class at most two

Suppose $G$ is a group of nilpotency class at most two and $L$ is a
Lie ring of nilpotency class at most two. A {\em Baer correspondence
  up to isoclinism} between $G$ and $L$ is a pair $(\zeta,\varphi)$
where:

\begin{itemize}
\item $\zeta$ is an isomorphism from the abelian group
  $\operatorname{Inn}(G)$ to the abelian group that is the additive
  group $\exp(\operatorname{Inn}(L))$ of $\operatorname{Inn}(L)$, and
\item $\varphi$ is an isomorphism from the abelian group $G'$ to the
  abelian group that is the additive group $\exp(L')$ of $L'$, 
\end{itemize}

such that the following diagram commutes:

$$\begin{array}{ccc}
  \operatorname{Inn}(G) \times \operatorname{Inn}(G) & \stackrel{\zeta \times \zeta}{\to} & \exp(\operatorname{Inn}(L)) \times \exp(\operatorname{Inn}(L)) \\
  \downarrow^{\omega_{G}}  & & \downarrow^{\exp(\omega_{L})}\\
  G' & \stackrel{\varphi}{\to} & \exp(L')\\
\end{array}$$

where $\omega_G$ is the map $\operatorname{Inn}(G) \times
\operatorname{Inn}(G) \to G'$ obtained from the commutator map on $G$,
and $\omega_L$ is the map $\operatorname{Inn}(L) \times
\operatorname{Inn}(L) \to L'$ obtained from the Lie bracket on $L$. We
had introduced the notation for the maps $\omega_G$ and $\omega_L$ in
Sections \ref{sec:isoclinism-and-homoclinism} and
\ref{sec:isoclinism-and-homoclinism-lie}.

We say that $G$ and $L$ are {\em in Baer correspondence up to
  isoclinism} if there exists a Baer correspondence up to isoclinism
between $G$ and $L$.

The following are easy to verify. All the groups and Lie rings
referred to below are of nilpotency class at most two. The proofs of
all the assertions below rely on a similar commutative diagram setup to
the setup used in Section \ref{sec:homoclinism-composition} to prove
that a composite of homoclinisms is a homoclinism.

\begin{itemize}
\item If $G_1$ and $G_2$ are isoclinic groups, and $G_1$ and $L$ are
  in Baer correspondence up to isoclinism, then $G_2$ and $L$ are also
  in Baer correspondence up to isoclinism.
\item If $L_1$ and $L_2$ are isoclinic Lie rings, and $G$ and $L_1$
  are in Baer correspondence up to isoclinism, then $G$ and $L_2$ are
  also in Baer correspondence up to isoclinism.
\item If $G_1$ and $G_2$ are groups and $L$ is a Lie ring such that
  $G_1$ is in Baer correspondence up to isoclinism with $L$ and $G_2$
  is also in Baer correspondence up to isoclinism with $L$, then $G_1$
  and $G_2$ are isoclinic.
\item If $L_1$ and $L_2$ are Lie rings and $G$ is a group such that
  $G$ is in Baer correspondence up to isoclinism with $L_1$ and $G$ is
  in Baer correspondence up to isoclinism with $L_2$, then $L_1$ and
  $L_2$ are isoclinic Lie rings.
\end{itemize}


In other words, the definition we gave above establishes a
correspondence between {\em some} equivalences classes up to
isoclinism of groups and {\em some} equivalence classes up to
isoclinism of Lie rings. However, it is not yet clear that the
correspondence applies to {\em every} equivalence class up to
isoclinism of groups and to {\em every} equivalence class up to
isoclinism of Lie rings. Essentially, we need to show two things:

\begin{enumerate}
\item For every group $G$ of nilpotency class at most two, there
  exists a Lie ring $L$ of nilpotency class at most two such that $G$
  is in Baer correspondence up to isoclinism with $L$.
\item For every Lie ring $L$ of nilpotency class at most two, there
  exists a group $G$ of nilpotency class at most two such that $G$ is
  in Baer correspondence up to isoclinism with $L$.
\end{enumerate}

(1) is relatively easy to show: we can take the associated graded Lie
ring of a group. (2) is harder to show. We will now discuss some ideas
related to group extension theory that will help us establish both (1)
and (2) in a better way.

\subsection{Exterior square for an abelian group}\label{sec:exteriorsquare-abelian-group-proofs}

Suppose $G$ is an abelian group. Then, the following are canonically
isomorphic:

\begin{enumerate}
\item The exterior square $G \wedge_\Z G$ of $G$ as an abelian group.
\item The exterior square $G \wedge G$ of $G$ as a group (as defined
  in section \ref{sec:exteriorsquare}). The map in the forward
  direction $G \wedge_\Z G \to G \wedge G$ is as follows. Note that
  the commutator map in a class two group is $\Z$-bilinear, so that
  the map $G \times G \to G \wedge G$ is $\Z$-bilinear. Thus, it gives
  rise to a homomorphism $G \wedge_\Z G \to G \wedge G$.
\item The Schur multiplier $M(G)$ of $G$ as a group.
\item The exterior square $G \wedge G$ of $G$ as an abelian Lie
  ring. The exterior square is itself an abelian Lie ring with the
  same additive group structure. We have already described this map in
  Section \ref{sec:exteriorsquare-abelian-lie}.
\item The Schur multiplier $M(G)$ of $G$ as an abelian Lie ring.
\end{enumerate}

We first note the equivalence of (2) with (3) and also the equivalence
of (4) with (5). For the equivalence of (2) and (3), note the
canonical short exact sequence, introduced in Section
\ref{sec:exteriorsquare}:

$$0 \to M(G) \to G \wedge G \to [G,G] \to 1$$

Since $G$ is abelian, $[G,G]$ is trivial and we get an isomorphism
$M(G) \cong G \wedge G$.

Similarly, the equivalence of (4) with (5) follows from the analogous
canonical short exact sequence for Lie rings introduced in Section
\ref{sec:exteriorsquare-lie}.

It remains to show that the canonical map from (1) to (2) is an
isomorphism and the canonical map from (1) to (4) is an isomorphism.

To see this, we can rely on the explicit presentations for the
exterior square provided in Section \ref{sec:exteriorsquare-explicit}
(for groups) and \ref{sec:exteriorsquare-explicit-lie} (for Lie rings)
to confirm that in the case that the group (respectively Lie ring) is
abelian, its exterior square as a group (respectively Lie ring) agrees
with its exterior square as an abelian group.

In order to better keep track of whether we are thinking of $G$ as an
abelian group or as an abelian Lie ring, it may sometimes help to use
the $\log$ and $\exp$ functors of the abelian Lie correspondence (as
described in Section \ref{sec:abelian-lie-correspondence}) to go back
and forth between the descriptions. With this language, we can rewrite
the above results in the form:

\begin{itemize}
\item For any abelian group $G$, $\log(G \wedge G)$ is canonically
  isomorphic to $\log G \wedge \log G$, where the $\wedge$ on the left
  represents the exterior square as a group and the $\wedge$ on the
  right represents the exterior square as a Lie ring.
\item For any abelian Lie ring $L$, $\exp(L \wedge L)$ is canonically
  isomorphic to $\exp L \wedge \exp L$ where the $\wedge$ on the left
  represents the exterior square as a Lie ring and the $\wedge$ on the
  right represents the exterior square as a group.
\end{itemize}

%% The second approach is to explicitly construct extensions that realize
%% the exterior square. We already did this for the Lie ring case,
%% establishing the equivalence of (1) and (4), with Lemma
%% \ref{lemma:exteriorsquare-of-abelian-group}, using the free class two
%% Lie ring.

%% The equivalence of (1) and (2) is harder because there is no group
%% theory analogue of the ``free class two Lie ring'' for an arbitrary
%% abelian group. Instead, we proceed via the free group route. The idea
%% is to proceed in the following steps:

%% \begin{enumerate}
%% \item Suppose $F$ is the free abelian group on a set $S$ such that $G
%%   \cong F/R$ for some normal subgroup $R$ of $F$.
%% \item Suppose $K$ is the free class two group on $S$. Then, $F =
%%   K^{\operatorname{ab}}$.
%% \item We can use an explicit normal form for the elements of $K$ (the
%%   idea being to use a Malcev basis) to verify that the commutator map
%%   homomorphism $F \wedge_\Z F \to [K,K]$ is an isomorphism. The
%%   details are deferred to the Appendix, {\em TONOTDO: Insert section}.
%% \item By Hopf's formula (Section
%%   \ref{sec:hopf-formula-class-one-more}), $G \wedge G$ is canonically
%%   isomorphic to $[K,K]/[K,T]$ where $T$ is the inverse image in $K$ of
%%   the subgroup $R$ of $F = K^{\operatorname{ab}}$. Thus, $G \wedge G$
%%   is canonically isomorphic to $(F \wedge_\Z F)/(F \wedge_\Z R)$,
%%   which is $G \wedge_\Z G$. If we trace what the maps are, they turn
%%   out to be the same as the map $G \wedge_\Z G \to G \wedge G$.
%% \end{enumerate}

\subsection{Description of central extensions for an abelian group}\label{sec:ses-uct-abelian}

Suppose $A$ and $G$ are abelian groups. Recall from Section
\ref{sec:second-cohomology-group-classify-extensions} that $H^2(G;A)$,
termed the {\em second cohomology group for trivial group action} of
$G$ on $A$, is a group whose elements correspond with the central
extensions with central subgroup $A$ and quotient group $G$.

We are now assuming that $G$ is abelian. Hence, all the extension
groups have nilpotency class at most two. Further,
$G^{\operatorname{ab}} = G$ and $M(G) = G \wedge G$ as
discussed. Thus, the short exact sequence described in
Section \ref{sec:ses-uct} simplifies to:

\begin{equation}\label{eq:ses-uct-abelian}
  0 \to \operatorname{Ext}^1_{\mathbb{Z}}(G;A) \to H^2(G;A) \to \operatorname{Hom}(G \wedge G,A) \to 0
\end{equation}

The short exact sequence splits (but not necessarily canonically) and we get:

$$H^2(G;A) \cong \operatorname{Ext}^1_{\mathbb{Z}}(G;A) \oplus \operatorname{Hom}(G \wedge G,A)$$

We will now proceed to explain the meaning of the short exact sequence
in this context.

\subsubsection{Description of the left map of the sequence}\label{sec:ses-uct-abelian-left-map}

The map:

$$\operatorname{Ext}^1_{\mathbb{Z}}(G;A) \to H^2(G;A)$$

can be interpreted as follows. The underlying set of the group on the
left is canonically identified with the set of all {\em abelian} group
extensions with subgroup $A$ and quotient group $G$. The group on the
right is the group whose elements are all the {\em central} extensions
with central subgroup $A$ and quotient group $G$. Every abelian group
extension is a central extension, and there is therefore a canonical
injective set map from $\operatorname{Ext}^1_{\mathbb{Z}}(G;A)$ to
$H^2(G;A)$. This set map turns out to be a {\em group homomorphism}
based on the way the group structures on
$\operatorname{Ext}^1_{\mathbb{Z}}(G;A)$ and $H^2(G;A)$ are
defined. Delving into the group structure on
$\operatorname{Ext}^1_{\mathbb{Z}}(G;A)$ will be too much of a
diversion from our goal here, so we skip it.

The image of the map is described as precisely the set of those
cohomology classes whose representative 2-cocycles are {\em
  symmetric}, i.e., any 2-cocycle $f$ in that cohomology class
satisfies the property that $f(x,y) = f(y,x)$ for all $x,y \in
G$. This can be easily deduced from the discussion in Section
\ref{sec:explicit-cocycle-description-of-extension}.

\subsubsection{Description of the right map of the sequence}\label{sec:ses-uct-abelian-right-map}

The map:

$$H^2(G;A) \to \operatorname{Hom}(G \wedge G,A)$$

can be described as follows. For any group extension $E$, the
commutator map $E \times E \to E$ descends to a set map:

$$\omega_{E,G}: G \times G \to A$$

Our earlier definition of $\omega_{E,G}$ defined it as a map to
$[E,E]$, but $[E,E]$ lies in the image of $A$ (under the inclusion of
$A$ in $E$), so it can be viewed as a map to $A$.

Note that the image of the map is in $A$ {\em because} $G$ is
abelian. Further, $\omega_{E,G}$ is bilinear, because the image of the map
is central. It thus defines a group homomorphism $G \wedge G \to A$.

The homomorphism above can also be described in terms of the how it
operates at the level of $2$-cocycles (this description requires
understanding the explicit description of the second cohomology group
using the bar resolution, as given in Section
\ref{sec:cohomology-explicit}). Explicitly, the map:

$$H^2(G;A) \to \operatorname{Hom}(G \wedge G,A)$$

arises from a homomorphism:

$$Z^2(G;A) \to \operatorname{Hom}(G \wedge G,A)$$

given by:

$$f \mapsto \operatorname{Skew}(f)$$

where $\operatorname{Skew}(f)$ is the map $(x,y) \mapsto f(x,y) -
f(y,x)$.

Intuitively, this is because the commutator of two elements represents
the distance between their products in both possible orders, i.e.,
$[x,y]$ is the quotient $(xy)/(yx)$. Whether we use left or right
quotients does not matter because the group has class two.

Based on the discussion in Section \ref{sec:extensionsuptoisoclinism},
the homomorphism:

$$H^2(G;A) \to \operatorname{Hom}(G \wedge G,A)$$

classifies extensions {\em up to isoclinism of group extensions}. In
other words, the fibers for this map are precisely the equivalence
classes up to isoclinism of group extensions.

\subsection{Description of central extensions for an abelian Lie ring}\label{sec:ses-uct-lie-abelian}

Suppose $A$ and $L$ are abelian Lie rings. Recall that
$H^2_{\text{Lie}}(L;A)$, called the {\em second cohomology group for
  trivial Lie ring action}, is a group whose elements correspond to
the congruence classes of central extensions with central subring $A$
and quotient Lie ring $L$.

We are now assuming that $L$ is abelian. Hence, $L^{\operatorname{ab}}
= L$ and $M(L) = L \wedge L$. The short exact sequence of Section
\ref{sec:ses-uct-lie} simplifies to:

\begin{equation}\label{eq:ses-uct-lie-abelian}
0 \to \operatorname{Ext}^1_{\mathbb{Z}}(L;A) \to H^2_{\text{Lie}}(L;A) \to \operatorname{Hom}(L \wedge L, A) \to 0
\end{equation}

The short exact sequence splits {\em canonically}, and we get a
canonical isomorphism:

$$H^2_{\text{Lie}}(L;A) \cong \operatorname{Ext}^1_{\mathbb{Z}}(L;A) \oplus \operatorname{Hom}(L \wedge L,A)$$

Note that $L$ being abelian is crucial to the splitting being
canonical. We will understand the splitting in more detail, but first we need to explain what the maps are.

\subsubsection{Description of the left map of the sequence}\label{sec:ses-uct-lie-abelian-left-map}

The map:

$$\operatorname{Ext}^1_{\mathbb{Z}}(L;A) \to H^2_{\text{Lie}}(L;A)$$

can be described as follows. The group on the left is canonically
identified with the {\em abelian} Lie rings arising as extensions with
subring $A$ and quotient ring $L$. The group on the right is
canonically identified with the {\em central} extensions with subring
$A$ and quotient ring $L$. Every abelian Lie ring extension is a
central extension, and there is therefore a canonical injective set
map from $\operatorname{Ext}^1_{\mathbb{Z}}(L;A)$ to $H^2_{\text{Lie}}(L;A)$. This
set map turns out to be a {\em group homomorphism} based on the way
the group structures on $\operatorname{Ext}^1_{\mathbb{Z}}(L;A)$ and
$H^2_{\text{Lie}}(L;A)$ are defined. Delving into the group structure on
$\operatorname{Ext}^1_{\mathbb{Z}}(L;A)$ will be too much of a
diversion from our goal here, so we skip it.

\subsubsection{Description of the right map of the sequence}\label{sec:ses-uct-lie-abelian-right-map}

The map:

$$H^2_{\text{Lie}}(L;A) \to \operatorname{Hom}(L \wedge L,A)$$

is defined as follows. For any extension Lie ring $M$, consider the
Lie bracket $M \times M \to M$. This descends to a
$\mathbb{Z}$-bilinear map:

$$\omega_M: L \times L \to A$$

Note that the image is in $A$ because $L$ is abelian. The map can be
viewed as a homomorphism from $L \wedge L$ to $A$, and hence as an
element of $\operatorname{Hom}(L \wedge L,A)$.

Note also that, per the discussion in Section
\ref{sec:extensionsuptoisoclinism-lie}, this homomorphism classifies
the extension {\em up to isoclinism of Lie ring extensions}. In other
words, the fibers of this map are precisely the equivalence classes up
to isoclinism of Lie ring extensions.
\subsubsection{Canonical splitting}\label{sec:ses-uct-lie-abelian-canonical-splitting}

We will now describe how the short exact sequence below splits:

\begin{equation*}
0 \to \operatorname{Ext}^1_{\mathbb{Z}}(L;A) \to H^2_{\text{Lie}}(L;A) \to \operatorname{Hom}(L \wedge L, A) \to 0
\end{equation*}

We can describe the splitting either by specifying the projection
$H^2_{\text{Lie}}(L;A) \to \operatorname{Ext}^1_{\mathbb{Z}}(L;A)$ or by specifying
the inclusion $\operatorname{Hom}(L \wedge L,A) \to H^2_{\text{Lie}}(L;A)$. We do both.

The projection:

$$H^2_{\text{Lie}}(L;A) \to \operatorname{Ext}^1_{\mathbb{Z}}(L;A)$$

is defined as follows. For any extension Lie ring $M$, map it to the
extension Lie ring that is {\em abelian} as a Lie ring and has the
same additive group as $M$. In other words, keep the additive
structure intact, but ``forget'' the Lie bracket.

The inclusion:

$$\operatorname{Hom}(L \wedge L,A) \to H^2_{\text{Lie}}(L;A)$$

is defined as follows. Given a bilinear map $b:L \times L \to A$,
define the extension Lie ring as a Lie ring $M$ whose additive group
is $L \oplus A$, and where the Lie bracket is:

$$[(x_1,y_1),(x_2,y_2)] = [0,b(x_1,x_2)]$$

In other words, we use the direct sum for the additive structure, and
use the bilinear map to define the Lie bracket.

In light of this, we can think of the direct sum decomposition as follows:

$$H^2_{\text{Lie}}(L;A) \cong \operatorname{Ext}^1_{\mathbb{Z}}(L;A) \oplus \operatorname{Hom}(L \wedge L,A)$$

The projection onto the first component stores the additive structure
of the Lie ring, while destroying, or forgetting, the Lie bracket. The
projection onto the second component preserves the Lie bracket while
replacing the additive structure with a direct sum of $L$ and
$A$. Note also that the latter projection is equivalent to passing to
the associated graded Lie ring. The associated graded Lie ring is
discussed in more detail in the Appendix, Sections
\ref{appsec:associated-graded-lie},
\ref{appsec:associated-graded-group}, and
\ref{appsec:associated-graded-functor}.

\subsection{The Baer correspondence up to isoclinism for extensions}\label{sec:bcuti-extensions}

Suppose $A$ and $G$ are abelian groups. Denote by $L$ the abelian Lie
ring whose additive group is $G$. In other words, $L = \log G$ under
the abelian Lie correspondence described in Section
\ref{sec:abelian-lie-correspondence}.

We abuse notation regarding $A$, using the same letter $A$ to denote
the abelian group and the abelian Lie ring, which might more properly
be written as $\log A$ when viewed as a Lie ring. We engage in this
abuse because, throughout this document, we deal with central
extensions, so that the base of the extention is always abelian. We do
not abuse notation when dealing with $G$ and $L$, because the
distinction will be helpful when we describe our later generalization,
the Lazard correspondence up to isoclinism, in Section
\ref{sec:lcuti}.

We have discussed above two short exact sequences:

\begin{equation*}
  0 \to \operatorname{Ext}^1_{\mathbb{Z}}(G;A) \to H^2(G;A) \to \operatorname{Hom}(G \wedge G,A) \to 0
\end{equation*}

and

\begin{equation*}
0 \to \operatorname{Ext}^1_{\mathbb{Z}}(L;A) \to H^2_{\text{Lie}}(L;A) \to \operatorname{Hom}(L \wedge L, A) \to 0
\end{equation*}

We have canonical isomorphisms between the left terms of the exact
sequences and between the right terms of the exact sequences:

$$\begin{array}{ccccccccc}
  0 &\to &\operatorname{Ext}^1_{\mathbb{Z}}(G;A) &\to &H^2(G;A) &\to &\operatorname{Hom}(G \wedge G,A) &\to &0\\
  & & \downarrow & & & & \downarrow & & \\
  0 &\to &\operatorname{Ext}^1_{\mathbb{Z}}(L;A) & \to & H^2_{\text{Lie}}(L;A) & \to & \operatorname{Hom}(L \wedge L, A) & \to & 0\\
\end{array}$$

We now try to understand both component isomorphisms in greater
detail.

\subsubsection{The isomorphism of the $\operatorname{Ext}^1$ groups on the left side, and its relation to the abelian Lie correspondence}

We have a canonical isomorphism of groups:

$$\operatorname{Ext}^1_{\mathbb{Z}}(G;A) \cong \operatorname{Ext}^1_{\mathbb{Z}}(L;A)$$

This is because, although we use different symbols for $G$ and $L$,
they both have the same underlying additive group, and the
$\operatorname{Ext}^1$ computation uses only the underlying additive
group.

The elements of $\operatorname{Ext}^1_{\mathbb{Z}}(G;A)$ correspond to
the extension groups with subgroup $A$ and quotient group $G$ {\em
  where the extension group is abelian}. The elements of
$\operatorname{Ext}^1_{\mathbb{Z}}(L;A)$ correspond to the extension
Lie rings with subring $A$ and quotient Lie ring $L$ {\em where the
  extension Lie ring is abelian}. The isomorphism above therefore
gives a correspondence:

\begin{center}
Group extensions with subgroup $A$ and quotient group $G$ such that
the extensions are abelian groups $\leftrightarrow$ Lie ring
extensions with subring $A$ and quotient ring $L$ such that the
extensions are abelian Lie rings
\end{center}

For each group extension and Lie ring extension that are in bijection
(in other words, each pair of elements in the two isomorphic groups
that are in bijection with each other), the corresponding extension group is in abelian
Lie correspondence with the corresponding extension Lie ring.

\subsubsection{The isomorphism of the $\operatorname{Hom}$ groups on the right side, and its relation to the Baer correspondence up to isoclinism}\label{sec:bcuti-extensions-main}

We have a canonical isomorphism:

$$\operatorname{Hom}(G \wedge G,A) \cong \operatorname{Hom}(L \wedge L, A)$$

We reviewed the meanings of the two groups in Sections
\ref{sec:ses-uct-abelian-right-map} and \ref{sec:ses-uct-lie-abelian-right-map}. The group
$\operatorname{Hom}(G \wedge G,A)$ classifies the central extensions
with central subgroup $A$ and quotient group $G$ up to isoclinism of
extension. The group $\operatorname{Hom}(L \wedge L,A)$ classifies
the central extensions with central subring $A$ and quotient Lie ring
$L$ up to isoclinism of the extension.

The two $\operatorname{Hom}$ groups are isomorphic because, since $L =
\log G$, Section \ref{sec:exteriorsquare-abelian-group-proofs} tells
us that the additive group of $L \wedge L$ is isomorphic to $G \wedge
G$ as an abelian group. Thus, the homomorphism groups can also be
identified with one another.

The isomorphism gives a correspondence:

\begin{center}
  Equivalence classes up to isoclinism of Lie ring extensions with
  central subring $A$ and quotient Lie ring $L$ $\leftrightarrow$
  Equivalence classes up to isoclinism of group extensions with central
  subgroup $A$ and quotient group $G$
\end{center}

Any particular instance of this bijection (i.e., an equivalence class
of Lie ring extensions and an equivalence class of group extensions
that are in bijection with each other) is termed a {\em Baer
  correspondence up to isoclinism for extensions}.

We now state an important lemma that relates the Baer correspondence
up to isoclinism for extensions with the Baer correspondence up to
isoclinism.

\begin{lemma}\label{lemma:bcuti-extensions-implies-bcuti}
  Supose $A$ and $G$ are abelian groups and $L = \log G$ is the
  corresponding abelian Lie ring. Suppose $E$ is a group extension
  with central subgroup $A$ and quotient group $G$. Suppose $N$ is a
  Lie ring extension with central subring $\log A$ (which we denote as
  $A$ via abuse of notation) and quotient Lie ring $L$. Suppose
  further than the equivalence class up to isoclinism of the group
  extension $E$ corresponds, via the above bijection, to the
  equivalence class of the Lie ring extension $N$. Then, the group $E$
  is in Baer correspondence up to isoclinism with the Lie ring $N$.
\end{lemma}

\begin{proof}
  We have the following map induced by the commutator map in $E$:

  $$\omega_{E,G}: G \times G \to A$$

  Similarly, we have the following map induced by the Lie bracket map
  in $N$:

  $$\omega_{N,L}: L \times L \to A$$

  The extensions being in correspondence up to isoclinism means that
  $\omega_{N,L} = \log(\omega_{E,G})$, in the sense that the underlying set
  map of $\omega_{N,L}$ coincides with the underlying set map of
  $\omega_{E,G}$. Note also that the images of these maps need not be all
  of $A$. The image of the first map generates a subgroup that can be
  identified with $E'$, whereas the image of the second map generates
  a subgroup that can be identified with the additive group of
  $N'$. Both maps coincide, so $E'$ and $N'$ are in abelian Lie
  corespondence.

  The image of $Z(E)$ in $G$ coincides with the normal subgroup $\{ x
  \in G \mid \omega_{E,G}(x,y) = 0 \ \forall \ y \in G\}$. Similarly,
  the image of $Z(N)$ in $L$ coincides with the ideal $\{ x \in L \mid
  \omega_{N,L}(x,y) = 0 \ \forall \ y \in L \}$. The underlying sets
  coincide, so the normal subgroup and ideal are in abelian Lie
  correspondence. Thus, the quotient group of $G$ by the image of
  $Z(E)$ in $G$ is in abelian Lie correspondence with the quotient
  group of $L$ by the image of $Z(N)$ in $L$. Thus, $E/Z(E)$ is in
  abelian Lie correspondence with $N/Z(N)$. Thus, the descended maps:

  $$\omega_E: E/Z(E) \times E/Z(E) \to E'$$

  and

  $$\omega_N: N/Z(N) \times N/Z(N) \to N'$$

  are in correspondence.
\end{proof}

\subsubsection{Relation between the middle groups}\label{sec:bcuti-extensions-splitting}

We have demonstrated the existence of canonical isomorphisms between
the left groups and between the right groups in the two short exact
sequences:

$$\begin{array}{ccccccccc}
  0 &\to &\operatorname{Ext}^1_{\mathbb{Z}}(G;A) &\to &H^2(G;A) &\to &\operatorname{Hom}(G \wedge G,A) &\to &0\\
  & & \downarrow & & & & \downarrow & & \\
  0 &\to &\operatorname{Ext}^1_{\mathbb{Z}}(L;A) & \to & H^2_{\text{Lie}}(L;A) & \to & \operatorname{Hom}(L \wedge L, A) & \to & 0\\
\end{array}$$

As described in Sections \ref{sec:ses-uct-abelian} and
\ref{sec:ses-uct-lie-abelian}, both short exact sequences
split. Therefore, it is possible to find an isomorphism $H^2(G;A) \to
H^2_{\text{Lie}}(L;A)$ that establishes an isomorphism of the short
exact sequences:

$$\begin{array}{ccccccccc}
  0 &\to &\operatorname{Ext}^1_{\mathbb{Z}}(G;A) &\to &H^2(G;A) &\to &\operatorname{Hom}(G \wedge G,A) &\to &0\\
  & & \downarrow & & \downarrow & & \downarrow & & \\
  0 &\to &\operatorname{Ext}^1_{\mathbb{Z}}(L;A) & \to & H^2_{\text{Lie}}(L;A) & \to & \operatorname{Hom}(L \wedge L, A) & \to & 0\\
\end{array}$$

Note, however, that the middle isomorphism is not canonical. In fact,
choosing a middle isomorphism is equivalent to choosing a splitting of
the top sequence. This is because the bottom sequence splits
canonically, as described in Section
\ref{sec:ses-uct-lie-abelian-canonical-splitting}.  We will return to
a discussion of this in Sections \ref{sec:bc-and-bcuti} and
\ref{sec:baer-correspondence-cocycle-level}.

\subsection{The Baer correspondence up to isoclinism for groups: filling the details}

We are now in a position to flesh out the remaining details of the
Baer correspondence up to isoclinism, which we defined in Section
\ref{sec:bcuti-def}:

\begin{center}
  Equivalence classes up to isoclinism of groups of nilpotency class at
  most two $\leftrightarrow$ Equivalence classes up to isoclinism of Lie
  rings of nilpotency class at most two
\end{center}
There are two pending facts we need to establish:

\begin{enumerate}
\item For every group $G$ of nilpotency class at most two, there
  exists a Lie ring $L$ of nilpotency class at most two such that $G$
  is in Baer correspondence up to isoclinism with $L$.
\item For every Lie ring $L$ of nilpotency class at most two, there
  exists a group $G$ of nilpotency class at most two such that $G$ is
  in Baer correspondence up to isoclinism with $L$.
\end{enumerate}

We had already noted in Section \ref{sec:bcuti-def} that (1) can be
achieved by using the associated graded Lie ring for the group. We
will now provide a better way to think about both (1) and (2). We will
begin with (1).

\subsubsection{Explicit construction from the group to the Lie ring}

We are given a group $G$ of nilpotency class at most two, and we need
to find a Lie ring $L$ of nilpotency class at most two such that $L$
and $G$ are in Baer correspondence up to isoclinism.

\begin{enumerate}[(a)]
\item Consider $G$ as a central extension:

  $$0 \to Z(G) \to G \to G/Z(G) \to 1$$

  Consider the equivalence class up to isoclinism of this extension.

\item Based on the discussion in Section
  \ref{sec:bcuti-extensions-main}, this equivalence class corresponds
  to an equivalence class up to isoclinism of Lie ring extensions with
  central subring $\log(Z(G))$ and quotient Lie ring $\log(G/Z(G))$. Let
  $L$ be any extension Lie ring in this equivalence class.

\item By Lemma \ref{lemma:bcuti-extensions-implies-bcuti}, $L$ and $G$
  are in Baer correspondence up to isoclinism.
\end{enumerate}

Note that in this direction, one {\em can} make a canonical choice of
$L$ based on $G$, namely, one can take the associated graded ring for
the central series $0 \le Z(G) \le G$. Note that this choice of $L$
will be the same for all $G$ in the equivalence class up to
isoclinism. The ability to make a canonical choice here is related to
the canonical splitting of the short exact sequence for Lie ring
extensions with abelian quotient group, as discussed in Section
\ref{sec:ses-uct-lie-abelian-canonical-splitting}.

\subsubsection{Explicit construction from the Lie ring to the group}

We are given a Lie ring $L$ of nilpotency class at most two, and we
need to find a group $G$ of nilpotency class at most two such that $L$
and $G$ are in Baer correspondence up to isoclinism.

\begin{enumerate}[(a)]
\item Consider $L$ as a central extension:

  $$0 \to Z(L) \to L \to L/Z(L) \to 0$$

  Consider the equivalence class up to isoclinism of this extension.

\item Based on the discussion in Section
  \ref{sec:bcuti-extensions-main}, this equivalence class corresponds
  to an equivalence class up to isoclinism of group extensions with
  central subgroup $\exp(Z(L))$ and quotient group $\exp(L/Z(L))$. Let
  $G$ be any extension group in this equivalence class.

\item By Lemma \ref{lemma:bcuti-extensions-implies-bcuti}, $L$ and $G$
  are in Baer correspondence up to isoclinism.
\end{enumerate}

\subsubsection{Preservation of order}\label{sec:bcuti-preserves-order}

In both directions, the constructions preserve the orders. In other
words, if we start with a finite group and use the construction in the
direction from groups to Lie rings, the Lie ring that we obtain has
the same order as the group that we started with. Similarly, if we
start with a finite Lie ring and use the construction in the direction
from Lie groups to groups, the group that we obtain has the same order
as the Lie ring that we started with.

This does not imply that {\em every} group and every Lie ring that are
in Baer correspondence up to isoclinism must have the same
order. Rather, we are saying that the answer to the existence question
continues to be affirmative even after we impose the condition that
the orders have to be equal.

In particular, given a a finite $2$-group of nilpotency class $2$, we
can find a finite $2$-Lie ring (i.e., a Lie ring whose additive group
is a finite $2$-group) of nilpotency class $2$ such that the group and
Lie ring are in Baer correspondence up to
isoclinism. Similarly, given a finite $2$-Lie ring of nilpotency class
$2$, we can find a finite $2$-group of nilpotency class $2$ such that
the group and Lie ring are in Baer correspondence up to isoclinism.


\subsection{Relating the Baer correspondence and the Baer correspondence up to isoclinism}\label{sec:bc-and-bcuti}

If a Baer Lie group $G$ is in Baer correspondence with a Baer Lie ring
$L$, then $G$ and $L$ are in Baer correspondence up to isoclinism (as
defined in Section \ref{sec:bcuti-def}). Explicitly, the Baer
correspondence between $G$ and $L$ can be used to define an
isomorphism $\zeta$ between $\operatorname{Inn}(G)$ and the additive
group of $\operatorname{Inn}(L)$, and also an isomorphism $\varphi$
between $G'$ and the additive group of $L'$, satisfying the
compatibility condition for being a Baer correspondence up to isoclinism.

Another way of framing this is that if we restrict attention to Baer
Lie groups and Baer Lie rings, then the Baer correspondence up to
isoclinism can be {\em refined} to a correspondence that works up to
isomorphism, namely the usual Baer correspondence up to
isomorphism. In fact, it can be refined even further to obtain the
strict Baer correspondence between individual groups and Lie rings, as
has been done in the previous two sections (Sections
\ref{sec:baer-correspondence-basics} and
\ref{sec:baer-correspondence-more}).

We now turn to how the Baer correspondence relates to the Baer
correspondence up to isoclinism for group extensions. Let $G$ and $A$
be abelian groups, and let $L = \log G$ be the abelian Lie ring with
additive group $G$. In Section \ref{sec:bcuti-extensions}, we worked
out the following relation between the universal coefficient theorem
short exact sequences, where the downward maps are isomorphisms:

$$\begin{array}{ccccccccc}
  0 &\to &\operatorname{Ext}^1_{\mathbb{Z}}(G;A) &\to &H^2(G;A) &\to &\operatorname{Hom}(G \wedge G,A) &\to &0\\
  & & \downarrow & & & & \downarrow & & \\
  0 &\to &\operatorname{Ext}^1_{\mathbb{Z}}(L;A) & \to & H^2_{\text{Lie}}(L;A) & \to & \operatorname{Hom}(L \wedge L, A) & \to & 0\\
\end{array}$$
 
We had also noted in Section
\ref{sec:ses-uct-lie-abelian-canonical-splitting} that the second
short exact sequence splits canonically, i.e.,:

$$H^2_{\text{Lie}}(L;A) \cong \operatorname{Ext}^1_{\mathbb{Z}}(L;A) \oplus \operatorname{Hom}(L \wedge L, A)$$

Thus, as observed in Section \ref{sec:bcuti-extensions-splitting},
specifying an isomorphism between the middle groups such that the
diagram commutes is equivalent to specifying a splitting of the first
short exact sequence.

Now, consider the case that $G$ and $A$ are both $2$-powered abelian
groups. In that case, by Lemma \ref{lemma:powering-extension-group},
all the extensions with central subgroup $A$ and quotient group $G$
are themselves $2$-powered, and are therefore Baer Lie
groups. Similarly, all the extension Lie rings with central subring
$A$ and quotient Lie ring $L$ are Baer Lie rings. Further, the Baer
correspondence establishes a bijection between the group extensions
and Lie ring extensions in such a manner that the diagram
commutes. Equivalently, it provides a canonical splitting of the top
row. Explicitly, we now obtain a commutative diagram with the middle
isomorphism filled in:

$$\begin{array}{ccccccccc}
  0 &\to &\operatorname{Ext}^1_{\mathbb{Z}}(G;A) &\to &H^2(G;A) &\to &\operatorname{Hom}(G \wedge G,A) &\to &0\\
  & & \downarrow & & \downarrow & & \downarrow & & \\
  0 &\to &\operatorname{Ext}^1_{\mathbb{Z}}(L;A) & \to & H^2_{\text{Lie}}(L;A) & \to & \operatorname{Hom}(L \wedge L, A) & \to & 0\\
\end{array}$$

Equivalently, we have an explicit splitting:

$$H^2(G;A) \cong \operatorname{Ext}^1_{\mathbb{Z}}(G;A) \oplus \operatorname{Hom}(G \wedge G,A)$$

We will now discuss explicitly how this splitting works.

\subsection{Cocycle-level description of the Baer correspondence}\label{sec:baer-correspondence-cocycle-level}

Suppose $G$ and $A$ are abelian groups (we will soon restrict to the
case that one or both of $G$ and $A$ is $2$-powered). Consider the
following two short exact sequences. The first is the short exact
sequence relating the coboundary, cocycle and cohomology groups,
originally described in Section
\ref{sec:ses-coboundary-cocycle-cohomology}:

$$0 \to B^2(G;A) \to Z^2(G;A) \to H^2(G;A) \to 0$$

The second is the universal coefficient theorem short exact sequence,
originally described in Section \ref{sec:ses-uct} and described
specifically for abelian $G$ in Section \ref{sec:ses-uct-abelian}:

$$0 \to \operatorname{Ext}^1_{\mathbb{Z}}(G;A) \to H^2(G;A) \to \operatorname{Hom}(G \wedge G,A) \to 0$$

The first short exact sequence need not split. An example where it
does not split was discussed in Section
\ref{sec:ses-coboundary-cocycle-cohomology}. The second short exact
sequence does always split but the splitting need not be canonical (see
Section \ref{sec:ses-uct-non-canonical-splitting}).

The right parts of these short exact sequences give surjective
homomorphisms, which we can compose:

$$Z^2(G;A) \to H^2(G;A) \to \operatorname{Hom}(G \wedge G,A)$$

As we discussed in Section \ref{sec:ses-uct-abelian-right-map}, the
composite of these maps is the skew map. Explicitly, the composite is
the map $f \mapsto \operatorname{Skew}(f)$, that sends a function $f$
to the function:

$$\operatorname{Skew}(f) = (x,y) \mapsto f(x,y) - f(y,x)$$

Note that the function $\operatorname{Skew}(f)$ is a
$\mathbb{Z}$-bilinear map $G \times G$ to $A$, which can be
interpreted as a homomorphism $G \wedge G \to A$.

Now, suppose that $G$ and $A$ are both $2$-powered abelian groups. In
that case, there is a canonical splitting of the composite map, given
as follows:

$$f \mapsto \frac{1}{2}f$$

In other words, a $\mathbb{Z}$-bilinear map $f: G \times G \to A$ is
sent to $\frac{1}{2}f:G \times G \to A$. Note that any
$\mathbb{Z}$-bilinear map is a $2$-cocycle (in general, any $n$-linear
map is a $n$-cocycle) so this works.

In particular, {\em both} the short exact sequences split, and we get
canonical direct sum decompositions:

$$Z^2(G;A) \cong B^2(G;A) \oplus H^2(G;A), \text{ splitting is } H^2(G;A) \to Z^2(G;A)$$

$$H^2(G;A) \cong \operatorname{Ext}^1_{\mathbb{Z}}(G;A) \oplus \operatorname{Hom}(G \wedge G,A), \text{ splitting is } \operatorname{Hom}(G \wedge G,A) \to H^2(G;A)$$

Note that the first short exact sequence need not split for all $G$
and $A$ (see the discussion in Section
\ref{sec:ses-coboundary-cocycle-cohomology}) and the existence of a
splitting is itself a piece of information. The second short exact
sequence does split for all $G$ and $A$, but the splitting is not in
general canonical, as discussed in Section
\ref{sec:ses-uct-non-canonical-splitting}, so the case where $G$ and
$A$ are both $2$-powered is special in that we obtain a {\em
  canonical} splitting.

The splitting map $\operatorname{Hom}(G\wedge G,A) \to H^2(G;A)$ is
the same as the one arising from the Baer correspondence. Explicitly,
as noted in Section \ref{sec:bcuti-extensions-splitting}, specifying the
splitting map $\operatorname{Hom}(G \wedge G, A) \to H^2(G;A)$ is
equivalent to specifying an isomorphism of $H^2(G;A)$ and
$H^2_{\text{Lie}}(L;A)$ such that the diagram below commutes:

$$\begin{array}{ccccccccc}
  0 &\to &\operatorname{Ext}^1_{\mathbb{Z}}(G;A) &\to &H^2(G;A) &\to &\operatorname{Hom}(G \wedge G,A) &\to &0\\
  & & \downarrow & & \downarrow & & \downarrow & & \\
  0 &\to &\operatorname{Ext}^1_{\mathbb{Z}}(L;A) & \to & H^2_{\text{Lie}}(L;A) & \to & \operatorname{Hom}(L \wedge L, A) & \to & 0\\
\end{array}$$

This isomorphism can be described in an alternative way. Let $E$ be an
extension group corresponding to an element of $H^2(G;A)$. Let $N =
\log(E)$ via the Baer correspondence. We can relate two short exact
sequences via a $\log$ functor.

$$\begin{array}{lllll}
    0 \to & A \to & E \to & G \to & 1 \\
    & \downarrow^{\log} & \downarrow^{\log} & \downarrow^{\log}&  \\
    0  \to & A \to & N \to & L \to & 0\\
\end{array}$$

Note that we abuse notation again, using the same letter $A$ for $A$
as a group and as a Lie ring.

Then, the element of $H^2_{\text{Lie}}(L;A)$ that corresponds to the
second row is the same as the image of the element of $H^2(G;A)$ under
the isomorphism described earlier.

\subsection{Relaxation of the $2$-powered assumption}

We consider what can be said when the assumption that both $G$ and $A$
are $2$-powered is relaxed.

$$0 \to \operatorname{Ext}^1_{\mathbb{Z}}(G;A) \to H^2(G;A) \to \operatorname{Hom}(G \wedge G,A) \to 0$$

Note that if the group $\operatorname{Hom}(G \wedge G,A)$ itself is
$2$-powered, then the preceding construction can still be carried out,
and we can obtain a splitting of the short exact sequence. The
correspondence between the extension group and the extension Lie ring
need no longer be an instance of the Baer correspondence. However, it
will continue to be an instance of the {\em divided} Baer
correspondence described in Section
\ref{sec:divided-baer-correspondence}. Note in particular that this
includes the case where the group $A$ alone is $2$-powered. It will
also include the case where the group $G$ alone is $2$-powered.

In the case that $\operatorname{Hom}(G \wedge G,A)$ is not
$2$-powered, the preceding method for obtaining a splitting will not
work. However, we know that the short exact sequence must still
split. For some choices of $G$ and $A$, it is possible to obtain an
automorphism-invariant splitting, even though such a splitting does
not arise from the Baer correspondence or any of its generalizations
described here.

\subsection{Inner automorphisms and inner derivations in the context of the correspondence up to isoclinism}\label{sec:bcuti-adjoint}

Many aspects of the relationship between inner automorphisms and inner
derivations described in Section \ref{sec:baer-adjoint} continue to be
valid, with suitable modification, for the Baer correspondence up to
isoclinism.

Suppose $G$ is a group of nilpotency class two and $L$ is a Lie ring
of nilpotency class two such that $G$ and $L$ are in Baer
correspondence up to isoclinism. In particular, this means that the
groups $G/Z(G) \cong \operatorname{Inn}(G)$ and $L/Z(L) \cong
\operatorname{Inn}(L)$ are in abelian Lie correspondence up to
isomorphism.

The adjoint action of $G$ on $L$ is defined as follows:

$$\operatorname{Ad}: G \to \operatorname{Aut}(L)$$

For any $u \in G$, define $\operatorname{Ad}_u$ as follows. Denote by
$\overline{u}$ the image of $u$ in $G/Z(G)$. Denote by $x$ an element
of $L$ such that the image of $x$ in $L/Z(L)$ corresponds to the
element $\overline{u}$ under the abelian Lie correspondence between
$L/Z(L)$ and $G/Z(G)$. We define $\operatorname{Ad}_u$ as the
following automorphism of $L$:

$$\operatorname{Ad}_u(g) = g + [x,g]$$

It can easily be verified that $\operatorname{Ad}_u$ is an
automorphism of $L$. It can also be verified that
$\operatorname{Ad}_{uv} = \operatorname{Ad}_u\operatorname{Ad}_v$,
making $\operatorname{Ad}$ a homomorphism. 

More conceptually, we can write the description as follows:

$$\operatorname{Ad}_u = 1 + \operatorname{ad}_x$$

where the images of $u$ and $x$ modulo the respective centers are in
abelian Lie correspondence.

\section{Examples of the Baer correspondence up to isoclinism}\label{sec:bcuti-ex}

In the case that $G$ and $A$ are odd-order abelian groups, the {\em
  original} Baer correspondence works. To obtain finite examples where
the Baer correspondence works only up to isoclinism, we need to look
at $2$-groups. Further, our examples must be cases where the quotient
$\operatorname{Hom}(G \wedge G,A)$ is nontrivial, so that there is at
least some non-abelian extension.\footnote{The abelian extensions can
  be put in correspondence based on the correspondence between abelian
  groups and abelian Lie rings, which, although not strictly part of
  the Baer correspondence as have defined it, falls under the
  generalization described in Section
  \ref{sec:baer-correspondence-lcs-generalization}}

\subsection{Extensions with quotient the Klein four-group and center cyclic of order two}

The smallest sized example is: $A = \mathbb{Z}_2$ is the cyclic group
of order $2$ and $G = V_4$ is the Klein four-group, isomorphic to
$\mathbb{Z}_2 \times \mathbb{Z}_2$.

The short exact sequences discussed in Sections
\ref{sec:ses-uct-abelian} and \ref{sec:ses-uct-lie-abelian}, along
with the canonical isomorphisms discussed in Section
\ref{sec:bcuti-extensions}, give the following:

$$\begin{array}{ccccccccc}
0 & \to & \operatorname{Ext}^1(V_4;\mathbb{Z}_2) & \to & H^2(V_4;\mathbb{Z}_2) & \to & \operatorname{Hom}(V_4 \wedge V_4,\mathbb{Z}_2) & \to & 0 \\
& & \downarrow & & & & \downarrow & & \\
0 & \to & \operatorname{Ext}^1(V_4;\mathbb{Z}_2) & \to & H^2_{\text{Lie}}(V_4;\mathbb{Z}_2) & \to & \operatorname{Hom}(V_4 \wedge V_4,\mathbb{Z}_2) & \to & 0\\
\end{array}$$

Recall that both short exact sequences split, and, as per the
discussion in Section \ref{sec:ses-uct-lie-abelian}, the Lie ring
short exact sequence splits canonically (with the splitting separating
out the addition and Lie bracket parts).

It turns out that:

\begin{itemize}
\item $\operatorname{Ext}^1(V_4;\mathbb{Z}_2)$ is itself isomorphic to $V_4$, the Klein four-group.
\item $\operatorname(V_4 \wedge V_4)$ is isomorphic to $\mathbb{Z}_2$,
  and thus, $\operatorname{Hom}(H_2(V_4;\mathbb{Z}),\mathbb{Z}_2)$ is
  isomorphic to $\mathbb{Z}_2$.
\item Thus, both of the second cohomology groups (the group and Lie
  ring side) are isomorphic to the elementary abelian group of order
  eight.
\end{itemize}

On the group side, we have the following eight extensions (eight being
the order of the cohomology group):

\begin{enumerate}[(a)]
\item Elementary abelian group of order eight (1 time).
\item $\mathbb{Z}_4 \oplus \mathbb{Z}_2$ (3 times).
\item $D_8$ (3 times).
\item $Q_8$ (1 time).
\end{enumerate}

(a) and (b) together form the image of $\operatorname{Ext}^1$ (total
size 4) while (c) and (d) form the non-identity coset of that
image.

On the Lie ring side, the eight extensions (eight being the order of the cohomology group) are:

\begin{enumerate}[(a)]
\item Abelian Lie ring, additive group elementary abelian of order eight (1 time)
\item Abelian Lie ring, additive group direct product of
  $\mathbb{Z}_4$ and $\mathbb{Z}_2$ (3 times).
\item The niltriangular matrix Lie ring ($3 \times 3$ strictly upper
  triangular matrices) over the field of two elements. (1 time)
\item The semidirect product of $\mathbb{Z}_4$ and $\mathbb{Z}_2$ as
  Lie rings. (3 times).
\end{enumerate}

(a) and (b) together form the image of $\operatorname{Ext}^1$ (total
size 4) while (c) and (d) form the non-identity coset of that
image.

Note that there is no canonical bijection between the set of eight
group extensions and the set of eight Lie ring extensions, but we can
naturally correspond the images of $\operatorname{Ext}^1$ in both. The
problem arises when attempting an element-to-element identification of the
non-identity cosets in the two cases. In other words, we have a
correspondence at a coset level:

$$\{ D_8, D_8, D_8, Q_8 \} \leftrightarrow \{ \text{The four non-abelian Lie ring extensions} \} $$

But there is no clear-cut way of making sense of {\em which} Lie ring
extension to correspond to {\em which} group. This is an example of a
situation where the Baer correspondence up to isoclinism does not seem
to have any natural refinement to a correspondence up to isomorphism.

Note that in this case, it so happens that we can use an
automorphism-invariance criterion and get a unique
automorphism-invariant bijection. This would map the niltriangular
matrix Lie ring to the quaternion group and the semidirect product of
$\Z_4$ and $\Z_2$ to the dihedral group. However, this does not give a
meaningful bijection at the level of elements. For instance, as
described in Section \ref{sec:baer-correspondence-cyclic-sub}, one
feature that holds in all generalizations described so far for the
Baer correspondence is that the correspondence restricts to
isomorphism between cyclic subgroups and cyclic subrings. In
particular, the rder statistics of the group (i.e., the multiset of
the orders of the elements) in the group must match the order
statistics of the additive group of the Lie ring. However, the order
statistics of $D_8$ do not match the order statistics of any abelian
group of order $8$. The same is true for $Q_8$.


\subsection{Extensions with quotient the Klein four-group and center cyclic of order four}

We consider a slight variation of the preceding example. We set $G =
V_4$ as before, but now set $A = \mathbb{Z}_4$, so $A$ is the cyclic
group of order four. The extension groups and extension Lie rings are
all of order 16.

It turns out that:

\begin{itemize}
\item $\operatorname{Ext}^1(V_4;\mathbb{Z}_4)$ is itself isomorphic to $V_4$, the Klein four-group.
\item $\operatorname(V_4 \wedge V_4)$ is isomorphic to $\mathbb{Z}_2$,
  and thus, $\operatorname{Hom}(H_2(V_4;\mathbb{Z}),\mathbb{Z}_4)$ is
  isomorphic to $\mathbb{Z}_2$.
\item Thus, both of the second cohomology groups (the group and Lie
  ring side) are isomorphic to the elementary abelian group of order
  eight.
\end{itemize}

On the group side, we have the following eight extensions (eight being
the order of the cohomology group):

\begin{enumerate}[(a)]
\item $\mathbb{Z}_4 \oplus V_4$ (1 time).
\item $\mathbb{Z}_8 \oplus \mathbb{Z}_2$ (3 times).
\item The group $M_{16} = M_4(2)$, given by the presentation $\langle
  a,x \mid a^8 = x^2 = 1, xax^{-1} = a^5 \rangle$ (3
  times). This group has ID (16,6) in the SmallGroups library
    used in GAP and Magma.
\item The group $D_8 *_{\Z_2} \Z_4 = Q_8 *_{\Z_2} \Z_4$ (1 time).
  This group has ID (16,13) in the SmallGroups library used in GAP and
  Magma.
\end{enumerate}

(a) and (b) together form the image of $\operatorname{Ext}^1$ (total
size 4) while (c) and (d) form the non-identity coset of that
image.

On the Lie ring side, we have the following eight extensions (eight
being the order of the cohomology group):

\begin{enumerate}[(a)]
\item Abelian Lie ring with additive group $\mathbb{Z}_4 \oplus V_4$
  (1 time).
\item Abelian Lie ring with additive group $\mathbb{Z}_8 \oplus
  \mathbb{Z}_2$ (3 times).
\item Lie ring with presentation $\langle
  a,x \mid 8a = 2x = 0, [a,x] = 4a \rangle$ (3
  times).
\item Lie ring with presentation $\langle a,x,y \mid 4a = 2x = 4y = 0,
  2a = 2y, [a,x] = 2a, [a,y] = [x,y] = 0 \rangle$ (1 time).
\end{enumerate}

(a) and (b) together form the image of $\operatorname{Ext}^1$ (total
size 4) while (c) and (d) form the non-identity coset of that
image.

We can naturally correspond the images of $\operatorname{Ext}^1$ in
both. We also have a correspondence at a coset level:

$$\text{The four non-abelian group extensions} \leftrightarrow \text{The four non-abelian Lie ring extensions}$$

In general, we cannot refine this to a canonical bijection at the
level of individual extensions. However, in this case, there does
exist an automorphism-invariant splitting, under which the type (c)
for groups corresponds to the type (c) for Lie rings, and the type (d)
for groups corresponds to the type (d) for Lie rings. The splitting
here is somewhat nicer than the splitting in the preceding example
because for each group and Lie ring in correspondence, we can obtain a
bijection between the group and the Lie ring that preserves the cyclic
subgroup structure, similar to the description in Section
\ref{sec:baer-correspondence-cyclic-sub}.

%% There are also larger examples where there is no
%% automorphism-invariant splitting of the short exact sequence for group
%% extensions, and thus no hope even of obtaining an
%% automorphism-invariant bijection. {\em TONOTDO: Either insert example (or
%%   Appendix link) or delete this para.}

%% {\em TONOTDO: I could insert more background and generalizations worked
%%   on earlier, perhaps in the Appendix; however, this is not necessary.}

