\documentclass[a4paper]{amsart}

%Packages in use
\usepackage{fullpage, hyperref, vipul, diagrams}


%Title details
\title{Exponentials and the log category}
\author{Vipul Naik}
\thanks{\copyright Vipul Naik, Ph.D. student, University of Chicago}

%List of new commands
\newcommand{\logcategory}[1]{\textsc{Log-Category}\left(#1\right)}
\newcommand{\enrichedlogcategory}[2]{\textsc{Enriched-Log-Category}\left(#1 \, ; \, #2\right)}
\newcommand{\characteristiclogcategory}[1]{\textsc{Characteristic-Log-Category}\left(#1 \right)}
\newcommand{\aut}[1]{\text{Aut}\left(#1\right)}
\newcommand{\inn}[1]{\text{Inn}\left(#1\right)}
\makeindex

\begin{document}
\maketitle
%\tableofcontents

\begin{abstract}
  In this write-up, I jot down some ideas I have about a ``log
  category'', a category such that the exponential maps we usually
  encounter in the theory of Lie algebras actually become
  homomorphisms between objects of the category. 
\end{abstract}

\section{Groups with perp-structure}

\subsection{Definition of  $\perp$-structure}

The ``log-category'' that I will describe in later sections can be
viewed as a kind of faithful representations on a category of groups
with additional structure. We shall first describe these groups with
additional structure, which may be of independent interest.

\begin{definer}[Group with perp-structure]
  A \definedind{perp-structure}, or \definedind{$\perp$-structure} on
  a group $G$ is a symmetric binary relation on $G$ such that for any
  $a \in G$, the set of $b$ such that $a \perp b$, is a subgroup of $G$.
\end{definer}

Given two structures $\perp$ and $\perp'$,we say that $\perp$ is coarser
than $\perp'$ if $a \perp b \implies a \perp' b$.

\begin{definer}[$\perp$-morphism]
  Let $G$ and $H$ be groups with $\perp$-structures. A
  \definedind{perp-morphism} from $G$ to $H$, also called a
  $\perp$-morphism from $G$ to $H$, is a map $f: G \to H$ such that
  whenever $a \perp b$, then $f(a) \perp f(b)$ and $f(ab) = f(a)f(b)$.
\end{definer}

We can form a category whose objects are groups with
$\perp$-structures, and whose morphisms are
$\perp$-morphisms. $\perp$-morphisms are weaker than the naive notion
of homomorphism as a map that preserves both the group operation and
the $\perp$ relation. Thus, in the category where morphisms are
$\perp$-morphisms, two non-isomorphic groups may be isomorphic
in the sense of $\perp$-morphisms.

For convenience, we shall call morphisms which preserve both
structures honest morphisms:

\begin{definer}[honest morphism]
  Let $G$ and $H$ be groups with $\perp$-structures. A
  \definedind{honest morphism} from $G$ to $H$ is a map $f:G \to H$
  such that $f(ab) = f(a)f(b)$ for all $a,b \in G$, and $a \perp b$
  implies $f(a) \perp f(b)$.
\end{definer}

\subsection{Basic types of $\perp$-structures}

Given an abstract group $G$, there are the following natural
$\perp$-structures on $G$:

\begin{definer}[Natural $\perp$-structures on a group]
  The following are some natural $\perp$-structures on a group $G$:

  \begin{itemize}

  \item The \definedind{trivial structure} is the finest possible
    $\perp$-structure: $a \perp b$ for all $a,b \in G$.

  \item The \definedind{discrete structure} is the coarsest possible
    $\perp$-structure: $a \perp b$ iff one of $a$ and $b$ is the
    identity element of $G$.

  \item The \definedind{reflexivity structure} is the coarsest
    $\perp$-structure such that $a \perp a $ for all $a \in G$.  Under
    the reflexivity structure, whenever $a$ and $b$ generate a cyclic
    subgroup, $a \perp b$. However, it is not in general true that $a
    \perp b$ implies that $a$ and $b$ generate a cyclic subgroup. To
    see why this is not true, observe that it may happen that $\langle
    a , b \rangle$ and $\langle a, c\rangle$ may be cyclic but
    $\langle a, bc \rangle$ is not cyclic.

  \item The \definedind{commutation structure} is defined as follows:
    $a \perp b \iff ab = ba$. Under the commutation structure, $a
    \perp b$ iff the subgroup generated by $a$ and $b$ is Abelian.
    The commutation structure is in general finer than the reflexivity
    structure.

  \item The \definedind{nilpotence class-$c$ structure} is defined as
    follows: $a \perp b$ iff $\langle a,b \rangle$ is a group of
    nilpotence class $c$. To show that this is actually a
    $\perp$-structure, we need to verify that $a \perp b$ and $a \perp
    c$ implies $a \perp bc$ -- a routine exercise involving
    commutators.

  \item The \definedind{solvable length-$l$ structure} is defined as
    follows: $a \perp b$ iff $\langle a,b \rangle$ is a group of
    solvable length $l$. To show that this is a $\perp$-structure is
    again a routine exercise involving commutators.

  \end{itemize}
\end{definer}

A $\perp$-structure is termed
\adefinedproperty{perp-structure}{reflexive} if it is finer than the
reflexivity structure; in other words $a \perp a$ for any $a$. Many of
the $\perp$-structures we shall encounter are naturally reflexive.

Some comments:

\begin{itemize}

\item There is a faithful and full functor from the category of groups to the
  category of groups with $\perp$-structure, which sends a group $G$
  to $G$ with the trivial $\perp$-structure. This is because for
  groups with trivial $\perp$-structure, a morphism is a usual group
  homomorphism.

\item Consider the category of groups with $1$-homomorphisms: a
  $1$-homomorphism is a map of groups which is a homomorphism when
  restricted to any cyclic subgroup. There is a faithful and full functor from
  this category to the category of groups with $\perp$-structure:
  namely, it sends a group to the group with the reflexivity
  structure.

\item Consider the category of groups with quasihomomorphisms: a
  quasihomomorphism $f:G \to H$ is a map $f$ such that $f(ab) =
  f(a)f(b)$ whenever $a$ and $b$ commute. There is a faithful and full
  functor from this category to the category of groups with
  $\perp$-structures, which sends a group $G$ to $G$ with the
  commutation structure.

\end{itemize}

We thus see that the category of groups with $\perp$-structures
``contains'' the category of groups in the usual sense, and also some
categories with modified notions of homomorphisms. We shall now see
that this category can also contain objects like rings and Lie
algebras.

\subsection{Perp-structures of ring type}

\begin{definer}[Perp-structure functor on category of unital rings]
  The perp-structure functor on the category of rings is defined as
  follows. Start with a unital ring $R$ (possibly without a
  multiplicative identity). Treat $R$ as an Abelian group under
  addition, and define $a \perp b$ in $R$ if $a$ and $b$ commute
  multiplicatively. This gives $R$ the structure of a group with
  $\perp$ structure. Further, any ring homomorphism $f:R \to S$
  induces an honest morphism of the associated groups with $\perp$-structure.
\end{definer}

We do not really need to have a multiplicative unit; in fact we do not
even need the ring to be associative. Thus, we can define
perp-structure functors on rings without unity, on Lie rings, and on
other non-associative algebras. For Lie rings, there is an alternative
definition of the $\perp$-structure functor, which is equivalent to
the previous definition when two times a nonzero element is nonzero:

\begin{definer}[Perp-structure functor on Lie rings]
  The perp-structure functor on the category of Lie rings is defined
  as follows: Start with a Lie ring $L$. Treat $L$ as an Abelian group
  under addition, and define $a \perp b$ in $R$ if $[a,b]=0$ (where
  $[\ ,\ ]$ is the Lie bracket).
\end{definer}

Given an {\em associative} ring, the group with perp-structure for the
ring is the same as the group with perp-structure for the associated
Lie ring, as per the above definition.

\subsection{Perp-structure of biadditive type}

There is an even more general notion of a perp-structure of biadditive
type. First, a definition:

\begin{definer}[Bihomomorphic structure]
  Let $A$ be an group. A bihomomorphic structure on $A$ is a map $b:A
  \times A \to B$ where $B$ is a group, such that for any $x$ the maps
  $y \mapsto b(x,y)$ and $y \mapsto b(y,x)$ are group
  homomorphisms. The bihomomorphic structure is said to be weakly
  symmetric if $b(x,y) = e \iff b(y,x) = e$. It is termed
  alternating if $b(x,y) = b(y,x)^{-1}$.
\end{definer}

A weakly symmetric bihomomorphic structure on a group gives a
$\perp$-structure; namely, $x \perp y \iff b(x,y) = e$. Weak symmetry
is necessary to ensure symmetry of the bihomomorphic structure. An
alternating bihomomorphic structure gives a reflexive
$\perp$-structure.

When the groups $A$ and $B$ are Abelian, then a bihomomorphic
structure is termed a {\em biadditive} structure.

Some remarks:

\begin{itemize}

\item The $\perp$-structure associated to a ring $R$ can be viewed as
  coming from the biadditive structure $b: R \times R \to R$ given by
  $b(x,y) = xy - yx$. $b$ is alternating, hence the $\perp$-structure
  is reflexive.

\item The $\perp$-structure associated to a Lie ring $L$ can be viewed
  as coming from the biadditive structure $[ \ , \ ]$. The Lie bracket
  is alternating, hence the $\perp$-strucure is reflexive.

\item For a group of nilpotence class two, the map $(x,y) \mapsto
  [x,y]$ is a bihomomorphic structure on the group, and the
  commutation relation comes from this bihomomorphic structure (this
  generalizes in different ways).

\item Consider a vector space $V$ over a field $\field$, and let $b: V
  \times V \to \field$ be a symmetric bilinear form. Then, $b$ is a
  biadditive structure on $V$ as an Abelian group, and the associated
  $\perp$-structure is the usual notion of orthogonality of
  vectors. This is {\em not} reflexive.

\item Consider a vector space $V$ over a field $\field$, and let $b:V
  \times V \to field$ be an alternating bilinear form. Then, $b$ is an
  alternating biadditive structure on $V$ as an Abelian group, and the
  associated $\perp$-structure is the notion of orthogonality with
  respect to the alternating bilinear form. This is reflexive.

\end{itemize}

\section{Some more notions for $\perp$-structures}

We would like to argue that groups with $\perp$-structures are a
correct level of generality in which to study many phenomena. In order
to do this, we need to get a handle on what it means for two groups
with $\perp$-structures to be isomorphic. We shall see that this
notion of isomorphism encompasses intuitive ideas like the statement
that a ``Lie group is the same as its Lie algebra'': a statement that
can only be justified in a context where both are objects in the same
category, a feat which is difficult to accomplish without
$\perp$-structures. We will do this in several steps. 

\subsection{Some definition}

We give some definitions:

\begin{definer}[Perp-center]
  The $\perp$-center of a group $(A,\perp)$ is defined as the set of
  $a \in A$ such that $a \perp b$ for all $b \in A$. The
  $\perp$-structure is trivial iff the $\perp$-center is the whole
  group. A group is termed \adefinedproperty{group}{perp-centerless}
  if its $\perp$-center is trivial.
\end{definer}

\begin{definer}[Morphism properties]\label{morphprops}
  A $\perp$-morphism $f:A \to B$ of groups with $\perp$-structure:

  \begin{itemize}

  \item \adefinedproperty{morphism}{faithful} if $f(x) \perp f(y) \iff
    x \perp y$

  \item \adefinedproperty{morphism}{injective} if the associated
    set-theoretic map is injective.

  \item \adefinedproperty{morphism}{central} if the inverse image of
    the identity element of $B$ contains only elements in the center
    of $A$.

  \item \adefinedproperty{morphism}{almost full} if the image of $A$
    under $f$ generates $B$ as a group, and if the induced maps on the
    $G$-center is surjective.

  \item \adefinedproperty{morphism}{full} if it is surjective.
  \end{itemize}

\end{definer}


\subsection{Action of a group on a group with $\perp$-structure}

By action of a group on a group with perp-structure, we mean an action
as {\em honest} automorphisms: automorphisms that preserve both the
group structure and the $\perp$-relation. Formally:

\begin{definer}[Action of a group on a group with perp-structure]\label{groupactonperp}
  Let $G$ be a group, and $A$ be a group with $\perp$-structure. Then
  a $G$-action on $(A,\perp)$ is a homomorphism $\phi:G \to \aut{A}$
  (as a group) such that for any $g \in G$, $a \perp b \implies
  \phi(g)a \perp \phi(g)b$. We usually omit the $\phi$ and write $ga$
  for $\phi(g)a$.
\end{definer}

We can now define a perp-representation category:

\begin{definer}[Perp-representation category of a group]
  Let $G$ be a group. The \definedind{perp-representation category} of
  $G$ is the category whose objects are actions of $G$ on groups with
  perp-structures, and whose morphisms are $\perp$-morphisms $f:A \to
  B$ such that for all $a \in A$:

  $$g(f(a)) = f(ga)$$
\end{definer}

\subsection{Representations of a group with $\perp$-structure}

On the other hand, we may be interested in starting with a group with
perp-structure, say $(G,\perp)$, and seeing how this can act. We make
a definition:

\begin{definer}[Linear representation of a group with $\perp$-structure]
  Suppose $(G,\perp)$ is a group with $\perp$-structure. A linear
  representation of this is a $\perp$-morphism from $(G,\perp)$ to
  $GL(V)$ with the trivial $\perp$-structure, where $V$ is a vector space
  over a field $\field$.
\end{definer}

Here, when $\perp$ is the trivial sturcture on $G$, we get the usual
notion of a linear representation.

Both these notions shall turn out useful, but it's now time to
introduce the idea of the log category.

\section{Log category}

\subsection{Log category of a group}

\begin{definer}[Log category]
  Let $G$ be a group. The \definedind{log category} of $G$, denoted
  $\logcategory{G}$, is the category whose objects comprise the
  following data:

  \begin{itemize}

  \item A group $A$ with a $\perp$-structure

  \item An action of $\inn{G}$ on $A$ (in the sense of definition \ref{groupactonperp}). We shall view this as an action of $G$ on $A$.

  \end{itemize}

  Satisfying the following two conditions:

  \begin{itemize}

  \item The $\perp$-center is the same as the setof $G$-fixed points.

  \item The map from $\inn{G}$ to $\aut{A}$ is faithful. In other
    words, every inner automorphism of $G$ acts nontrivially on $A$.

  \end{itemize}

  The morphisms in the category are $\perp$-morphisms $f: A \to B$
  such that for any $a \in A$ and $g \in G$:

  $$f(ga) = gf(a)$$

\end{definer}

The log category of a group is a subcategory of its
perp-representation category, comprising representations that are
``faithful'' and ``essential'' in some sense. 

We shall now study objects in the log category of a group that are of
special types: particularly of ring-type, of Lie ring-type, of
biadditive type, and so on.

We also introduce the notion of the enriched log category, and
characteristic log category:

\begin{definer}[Enriched log category]
  Let $G$ be a group, and $K$ a subgroup of $\aut{G}$ containing
  $\inn{G}$. Then the objects of $\enrichedlogcategory{G}{K}$ are
  defined as the following data:

  \begin{itemize}

  \item A group $A$ with a $\perp$-structure

  \item An action of $K$ on $A$ (in the sense of definition
    \ref{groupactonperp}). This also gives a $G$-action on $A$, via
    the map $G \to \inn{G}$.

  \end{itemize}

  such that the following conditions are satisfied:

  \begin{itemize}

  \item The $\perp$-center is the same as the setof $G$-fixed points.

  \item The map from $K$ to $\aut{A}$ is faithful. In other
    words, every element of $K$ acts nontrivially on $A$.

  \end{itemize}

  The morphisms in the category are $\perp$-morphisms $f: A \to B$
  such that for any $a \in A$ and $k \in K$:

  $$f(ka) = kf(a)$$
\end{definer}

Finally, the notion of characteristic log category:

\begin{definer}
  The characteristic log category of $G$, denoted
  $\characteristiclogcategory{G}$, is the category
  $\enrichedlogcategory{G}{\aut{G}}$.
\end{definer}

Every group, with the $\perp$-structure being the commutation
structure, is naturally an object in its own characteristic log
category.

\subsection{Log categories of Abelian groups}

The log category of an Abelian group is the usual category of groups
with group homomorphisms. This is because the inner automorphism group
of an Abelian group is trivial, hence the set of fixed points under
its action is the whole group. By the effectiveness condition of the
definition, this forces the $\perp$-structure on any object in the log
category to be trivial, and thus forces homomorphisms to be usual
group homomorphisms.

\subsection{Objects of ring type in the log category}

We know what it means for a group with $\perp$-structure to be of ring
type: the additive structure should come from the ring's addition, and
the $\perp$-structure should come from multiplicatively commuting in
the ring. An object of ring type in the log category of $G$ is a group
with $\perp$-structure which arises from a ring, and such that all
elements of $\inn{G}$ (or $K$, or $\aut{G}$) give {\em ring}
automorphisms. In other words, the action of the group should preserve
not just the $\perp$-structure, but the underlying ring structure.

We give some typical examples:

\begin{enumerate}

\item Suppose $R$ is a commutative ring. Then consider the group ring
  $R[G]$ of a group. $R[G]$ can be viewed as an object in the
  $\characteristiclogcategory{G}$, where an automorphism of $G$
  acts in the corresponding way on the freely generating set
  corresponding to $G$.

\item More generally, suppose $\rho:G \to GL(V)$ is a linear
  representation. Let $K$ be the group of automorphisms $\sigma$ of
  $G$ such that $\rho \circ \sigma \simeq \rho$. Then the image of
  $\field[G]$ via the action $\rho$, is an object of
  $\enrichedlogcategory{G}{K}$. In particular, for a finite group $G$,
  and if the characteristic of $K$ does not divide the order of $G$,
  then any linear representation of $G$ gives rise to an object in
  $\enrichedlogcategory{G}{K}$ where $K$ is the group of class
  automorphisms of $G$.
  
\item More generally, suppose $A$ is a ring, and $G$ is a
  multiplicative subset of $A$ such that $A$ is generated by $G$ as a
  $Z(A)$-module. Then $A$ has the structure of an object in
  $\logcategory{G}$, where the elements of $G$ act by inner
  automorphisms in the ring $A$. The condition of being generated over
  the center by $G$ is sufficient to ensure the faithfulness
  conditions. For instance $M(n,\R)$ is in the log category of
  $GL(n,\R)$.

  In this general case, $A$ may not be viewable as an object in
  $\characteristiclogcategory{G}$.

\end{enumerate}

\subsection{Nilpotent algebras}

Another interesting example of the log category is a nilpotent algebra
and its algebra group. By a nilpotent algebra $N$ I mean a ring
without identity $N$, such that there is a $n$ forwhich $N^n = 0$. To
any nilpotent algebra, one can associate a group called the algebra
group. For each $x \in N$, associate the formal element $1 + x$, and
define $(1 + x)(1 + y) = 1 + (x + y + xy)$. The identity element is
$1$, and existence of inverses is guaranteed by the formula:

$$(1 + x)(1 - x + x^2 - x^3 + \ldots) = 1$$

If we denote the algebra group by $G$, then $N$ is an object in
$\logcategory{G}$, as follows. The relation $\perp$ is that of commuting
in $N$, and the action is:

$$(1 + x).y = (1 + x)y(1 + x)^{-1}$$

where the multiplication can be viewed as happening inside the
unitization of $N$.

Note that the set-theoretic identification of $N$ with $G$ via the map
$x \mapsto 1 + x$ preserves $\perp$, but it is {\em not} a
$\perp$-morphism because $(1 + x)^2 \ne 1 + x^2$.

\subsection{Objects of Lie ring type}

We know of a number of classical situations where, to a group, we can
associate a corresponding Lie algebra. We shall now try to describe
how these Lie algebras can be viewed as being in the log category of
the group.

First, by an object of Lie ring type in $\enrichedlogcategory{G}{K}$,
we mean a Lie ring $L$ which gives an object in
$\enrichedlogcategory{G}{K}$ for the associated $\perp$-structure, and
such that all elements of $K$ give {\em Lie ring automorphisms} of
$L$.

Here are some facts:

\begin{itemize}

\item For a real or complex Lie group, its Lie algebra is an object of
  Lie ring-type in its log category. It is enriched under all
  topological (or smooth) automorphisms of the real or complex Lie
  group.

\item Under the Lazard correspondence, the Lazard Lie ring associated
  with a $p$-group is an object of Lie ring-type in its log category.
  In fact, it is an object of Lie ring-type in the characteristic log
  category of $G$.

\end{itemize}

\subsection{Associated Lie ring}

If $S$ is a $p$-group, there is a standard construction of an
associated Lie ring to $S$. This is a group with $\perp$-structure;
there is also an action of $\aut{G}$ on the associated Lie
ring. However, the associated Lie ring is {\em not} in the log
category of $G$. The reason it fails to be so is that the action of
$\inn{G}$ on the associated Lie ring is trivial; this is in sharp
contrast to the hypotheses needed for an object in the log category.

\section{The notion of a logarithm}

\subsection{Definition of logarithm}

In this section I use the set of definitions \ref{morphprops}.

\begin{definer}[Logarithm of a group]
  A \definedind{logarithm} of a group $G$ is an object $A \in
  \logcategory{G}$ together with an almost full morphism $\exp:A \to
  G$ in the log category. By {\em almost full}, we mean that the image
  of $A$ generates $G$.

  If $K$ is a subgroup of $\aut{G}$ containing $\inn{G}$, an
  \definedind{enriched logarithm} over $K$ is defined as an object $A
  \in \enrichedlogcategory{G}{K}$ together with an almost full
  morphism $\exp:A \to G$ in $\enrichedlogcategory{G}{K}$. When $K =
  \aut{G}$, we say that $A$ is a \definedind{characteristic logarithm}
  for $G$.

  A (enriched/characteristic) logarithm is termed:

  \begin{itemize}

  \item \adefinedproperty{logarithm}{full} if the associated morphism
    is full i.e. surjective.
  \item \adefinedproperty{logarithm}{faithful} if the associated
    morphism is faithful.
  \item \adefinedproperty{logarithm}{injective} if the associated
    morphism is injective.
  \item \adefinedproperty{logarithm}{central} if the associated
    morphism is central. (Any faithful or injective logarithm is central).
  \item \adefinedproperty{logarithm}{reversible} if it is full,
    faithful and injective. Equivalently, the associated morphism has
    a two-sided inverse morphism.
  \end{itemize}

  A reversible logarithm for a group is isomorphic to the group as an
  object in the log category. Similarly, a reversible characteristic
  logarithm for a group is isomorphic to the group as an object in the
  characteristic log category.
\end{definer}

\subsection{Direct sum theorem}

I now present a basic result, special cases of which shall keep
popping up:

\begin{theorem}[Direct sum theorem]
  Suppose $S$ is a group, and $\S = A_1, A_2, \ldots, A_r$ is a
  collection of subgroups of $S$, with the property that the
  collection is {\em self-conjugate}, in the sense that $gA_ig^{-1}
  \in \S$ for any $A_i \in \S$. Further, suppose that $G$ is generated
  by the set-theoretic union of the $A_i$s. Consider:

  $$A = \bigoplus_{i=1}^r A_i$$

  And the map:

  $$\exp(a_1,a_2,\ldots,a_r) = a_1a_2\ldots a_r$$

  is an almost full $\perp$-morphism from $B$ to $S$. Further:

  \begin{itemize}

  \item If every element of $A_i$ commutes with every element of $A_j$
    for $i \ne j$, then the map from $A$ to $S$ is a group homomorphism.

  \item $A$ naturally has a $\perp$-structure, and an
    $\inn{S}$-action, and the map $\exp:A \to S$ is a $\perp$-morphism
  \item If $S = A_1A_2\ldots A_r$, which will happen if the $A_i$ are
    all normal or if their set-theoretic union is $S$, then the
    $\perp$-morphism given above is full.

  \item If the $A_i$ are all normal subgroups of $S$, then the
    $\inn{S}$-action along with the $\perp$-structure makes $A$ an
    object of $\logcategory{S}$, and the $\perp$-morphism is a full
    logarithm.

  \end{itemize}
\end{theorem}

\begin{proof}[Structure on $A$ as an object in $\logcategory{S}$]
  We first describe the $\perp$-structure on $B$. Pick tuples:

  $$a = (a_1, a_2, \ldots, a_r), a' = (a_1', a_2', \ldots,a_r')$$

  Then $a \perp a'$ if $a_i \perp a_j'$ for every $i,j \in
  \oneton{r}$.

  We now describe the action of $S$ on $A$. For $g \in S$ and $a \in
  A$, do the following. Pick $\sigma(i)$ as the index of the group
  $gA_ig^{-1}$.  Now send $a = (a_1,a_2,\ldots,a_r)$ to the element
  $a'$ such that $a'_{\sigma(i)}$ is $gag^{-1}$. Thi action factors
  through the quotient map to $\inn{S}$.

  For any tuple $a = (a_1,a_2,\ldots,a_r)$, the set of elements $\perp$
  to it is a subgroup of $A$; in fact it is:

  $$\bigoplus \left( A_i \cap C_S(a_1) \cap C_S(a_2) \cap \ldots \cap C_S(a_r)\right)$$

  The $\perp$-center comprises precisely those elements which, in
  every coordinate, commute with every element in every $A_i$. Since
  the $A_i$s generate $S$ as a group, this subgroup is:

  $$\bigoplus A_i \cap Z(S)$$

  We now consider the case where all the $A_i$s are normal subgroups
  of $S$. In this case, $S$ acts coordinate-wise on the $A_i$s, and
  the set of fixed points is same as the above.

  Finally the kernel of the $S$-action on $A$ is precisely $Z(S)$;
  this again follows from the fact that $A_i$ generate $S$ as a group.
\end{proof}

\begin{proof}[Proof that the map gives an almost full logarithm]
  Consider the map:

  $$\exp:A \to S$$

  given by:

  $$\exp(a_1,a_2,\ldots,a_r) = a_1a_2\ldots a_r$$

  We want to check various things. First, if $a \perp a'$ we want to
  show that $\exp(a) \perp \exp(a')$ and $\exp(aa') =
  \exp(a)\exp(a')$. Both of these are routine checks which use the
  fact that we can move the $a_j'$s pas the $a_i$s by
  commutation. Hence we have a $\perp$-morphism.

  The fact that the morphism is almost full follows from the fact
  that the $A_i$s generate $G$.

  If every element in $A_i$ commutes with every element in $A_j$, then
  clearly the map is a group homomorphism.

  If $S = A_1A_2 \ldots A_r$, then the map is surjective, and hence is
  a full $\perp$-morphism.

  Finally, if all the $A_i$ are normal, then for $a \in A$, $g.a$ is
  described as:

  $$(g.a_1, g.a_2, \ldots, g.a_r)$$

  and thus:

  $$\exp(g.a) = g.(a_1a_2 \ldots a_r) = g.\exp(a)$$

  We are using the fact that since $A_i$ is normal, the action of $S$
  does not permute the $A_i$s.
\end{proof}

There is another version in the case of normal subgroups, which is
sometimes more useful:

\begin{theorem}[Logarithm in terms of reversible characteristic logarithms for subgroups]
  Suppose $\S = \{ N_1 ,N_2, \ldots, N_r \}$ is a collection of normal
  subgroups of a group $S$, which generate $S$. Suppose $\exp_i:M_i \to N_i$
  is a reversible logarithm enriched by the $S$-action, for each $i$.
  Then the group:

  $$M = \bigoplus_{i=1}^r M_i$$

  naturally acquires the structure of a full logarithm for $S$.
\end{theorem}

The proof is essentially the same as before, except that we now work
with $M_i$s instead of the $N_i$s. In the case where $M_i = N_i$, we
get a special case of the previous theorem.

\subsection{A special $\perp$-morphism for any group}

Every group can be expressed as the union of its cyclic subgroups,
because every element is in the cyclic group it generates. From the
direct sum theorem, we see that this gives a $\perp$-morphism from the
direct sum of all cyclic subgroups, to the group
itself. Unfortunately, although the direct sum of all cyclic subgroups
has a $\perp$-structure with a $\perp$-morphism to $G$, this is not a
morphism in the log category unless the cyclic subgroups are all
normal. All cyclic subgroups being normal is a very rare situation: it
essentially happens only for groups constructed using the quaternion
group and Abelian groups.

\section{Existence of Abelian logarithms}

\subsection{Existence of a reversible Abelian logarithm}

\begin{theorem}[Reversible Abelian logarithm implies nilpotent]
  If $G$ is a finite group and has a reversible logarithm which is an
  Abelian group, then $G$ is nilpotent.
\end{theorem}

\begin{proof}
  First note that any reversible logarithm must also be reflexive. Let
  $A$ be a reversible logarithm for $G$. Then, we want to show that
  $G$ is nilpotent. It suffices to show that $G$ is the direct product
  of its Sylow subgroups, for which it suffices to show that if $g,h
  \in G$ are elements of order $p^\alpha$, $q^\beta$, with $p \ne q$
  primes, then $g$ and $h$ commute. 

  By the map $\exp:A \to G$, pick $a = \exp^{-1}(g)$ and $b =
  \exp^{-1}(h)$. Then the order of $a$ is $p^\alpha$ and the order of
  $b$ is $q^\beta$ (a reversible logarithm preserves orders of
  elements). Consider the element $a + b$. Since $p^\alpha a$ is the
  identity, $p^\alpha (a + b) = p^\alpha b$, and similarly, $q^\beta(a
  + b) = q^\beta a$. Thus the order of $a + b$ divides $p^\alpha
  q^\beta$. Moreover, the order cannot be of the form $p^\gamma
  q^\beta$ for $\gamma < \alpha$, and similarly it cannot be of the
  form $p^\alpha q^\delta$ for $\delta < \beta$. Thus the order is
  exactly $p^\alpha q^\beta$.

  We now want to show that both $a$ and $b$ are expressible as
  multiples of $a + b$. To do this, let $r$ be such that $rq^\beta = 1
  \mod p^\alpha$. then $rq^\beta(a + b) = a$. Similarly, let $s$ be
  such that $sp^\alpha = 1 \mod q^\beta$. Then $sp^\alpha(a + b) = b$.

  The upshot is that $a$ and $b$ are both in the cyclic subgroup
  generated by $a + b$. Since $(a + b) \perp (a + b)$, we get $a \perp
  b$, and hence $g \perp h$, yielding that $g$ and $h$ commute.
\end{proof}

The assumptions of reversibility can be somewhat weakened; for
instance, we can show that there is no full, central and reflexive
Abelian logarithm for a non-nilpotent group. I do not know how
necessary the various assumptions are.

There do exist natural situations where non-nilpotent groups have full
reflexive logarithms; the most natural being $M_n(\C)$ as a full
logarithm of ring-type for the group $GL(n,\C)$.

\subsection{The Lazard Lie ring}

We are now in a position to state the Lazard's correspondence:

\begin{theorem}[Lazard's correspondence]
  Suppose $S$ is a finite $p$-group such that for any three elements
  of $S$, the subgroup generated by those three elements has
  nilpotence class at most $p-1$. Then, $S$ admits a characteristic
  reversible logarithm of Lie ring type, called the Lazard Lie ring of
  $S$.
\end{theorem}

Since the logarithm is of Lie ring type, it is in particular a
reversible Abelian logarithm.

\subsection{Groups without Lazard Lie rings}

We now investigate the situation for groups which do not have Lazard
Lie rings, but are nilpotent. None of the examples we shall see have
reversible Abelian logarithms (and I do not know of any reversible
Abelian logarithm that does not arise from a Lazard
correspondence). However, many of these groups do have corresponding
Lie rings, which live in the log category of the group, without a map
either way. Further, we can find a bigger object in the log category
with maps both to the group and to the Lie ring.

The general picture looks like this:

\begin{diagram}
  & & \text{Big Abelian} & \text{ group} & \\
  & \ldTo & & \rdTo & \\
  \text{Original} & \text {group} & & & \text{Lie algebra}
\end{diagram}

The theorem that I hope to establish, based on Theorem 2 of Professor
Glauberman's note, is the following:

\begin{theorem}[Group generated by normal subgroups with Lazard Lie rings]
  Suppose $S$ is generated by a family $\S = \{ N_1, N_2, \ldots, N_r
  \}$ of normal subgroups, with Lazard Lie rings $L_1, L_2, \ldots,
  L_r$. Then, there exists a Lie ring $L$, such that:

  \begin{itemize}

  \item $L_i$ is a subring of $L$ for each $i$, and the $L_i$s
    generate $L$.

  \item $L$ has the structure of an object of Lie ring-type in
    $\logcategory{S}$, where the action of $g \in S$ on $L_i \subset
    E$, is the pullback of the action of $g$ on $N_i$ by conjugation.

  \item Let $M = \bigoplus_{i=1}^r L_i$ as an Abelian group. Then if
    we give $L$ the structure of an object in $\logcategory{S}$ by the
    prescription of the direct sum theorem, the natural map $M \to L$
    is a morphism in $\logcategory{S}$.

  \end{itemize}

  The morphism picture is thus:

  \begin{diagram}
    & & M & & \\
    & \ldTo & & \rdTo & \\
    S & & & L
  \end{diagram}
\end{theorem}

I believe that this should follow by a re-interpretation of the
theorems of Professor Glauberman's paper.

\subsection{The way ahead}

The following are some natural next questions:

\begin{enumerate}

\item What are the abstract properties that Lazard Lie rings have,
  over and above being reversible characteristic logarithms of Lie ring-type?

\item What property must a $p$-group have, in order to possess a
  reversible characteristic logarithm of Lie ring-type? Must any such
  reversible logarithm be a ``Lazard Lie ring''?

\item What property must a $p$-group have, in order to possess a
  reversible logarithm (not necessarily characteristic) of Lie
  ring-type? Must any such reversible logarithm be characteristic? 
  
\item What property must a $p$-group have, in order to have a
  reversible Abelian logarithm? Must any such reversible Abelian
  logarithm be of Lie ring-type?

\item What property must a $p$-group have, in order to have a full
  logarithm of Lie ring-type (we're no longer assuming the logarithm
  to be faithful)?

\item What property must a $p$-group have, in order to have a full
  Abelian logarithm? (The direct sum theorems tell us that it is
  sufficient that it be generated by normal subgroups each with a
  reversible Abelian logarithm).

\end{enumerate}

On a parallel track are questions like:

\begin{enumerate}

\item Given a $p$-group, what is the minimum possible nilpotence class
  of a reversible logarithm for the $p$-group?

\item Given a $p$-group, what is the minimum possible nilpotence class
  of a full (reflexive) logarithm for the $p$-group?

\item What can we say about non-Abelian $p$-groups which do not have reversible
  logarithms of smaller nilpotence class? Such groups do exist; for
  instance, the quaternion group, and the dihedral group.

\item What can we say about non-Abelian $p$-groups which do not have
  full (reflexive) logarithms of smaller nilpotence class?

\item Can we say anything about the existence of full logarithms for
  non-nilpotent finite groups? (Note: Reversible logarithms do not
  exist).

\end{enumerate}



\printindex
\end{document}