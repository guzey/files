\documentclass[10pt]{amsart}

%Packages in use
\usepackage{fullpage, hyperref, vipul, amssymb, setspace, enumerate, pdfpages}

%Title details
\title{Lazard correspondence: talk}
\author{Vipul Naik}

%List of new commands
\newcommand{\Skew}{\operatorname{Skew}}
\newcommand{\ad}{\operatorname{ad}}
\begin{document}
\maketitle
\onehalfspacing

Overall breakdown:

\begin{itemize}
\item Oral review of pre-written material: 5-7 minutes
\item Outline of main proof parts: 5-10 minutes
\item Baer correspondence up to isoclinism (in-depth discussion): 20-25 minutes, {\em and/or}
\item Global Lazard correspondence (more details): 20-25 minutes
\end{itemize}

\section{Pre-written material}

Total time on this section: about 5-7 minutes


\subsection{Background board}

Background material (zero time):

\begin{itemize}
\item Nilpotent groups
\item Lie rings ($\Z$-Lie algebras)
\item Basic homological algebra (short exact sequences, etc.)
\item Basic language from category theory and universal algebra
\item $\Z[\pi^{-1}]$ for prime sets $\pi$
\end{itemize}

\section{Loose intro board (write beforehand)}

{\em Say time}: 1-2 minutes

\begin{itemize}
\item Groups are hard because the noncommutativity is entangled with
  the group operations.

  {\em Say (don't write)}: Rings are better because, even though they
  have noncomutative operations, there is an underlying additive group
  that is abelian and we can achieve a lot by using that.

\item Lie rings have similar axioms to groups, and therefore similar
  behavior.
\item Lie rings are nicer because the non-abelianness (stored via the
  Lie bracket) is cleanly separated from the group operation.

  {\em Say (don't write)}: If we can somehow reversibly transform
  between a group and a Lie ring, we can use this fact to study the
  group.
\end{itemize}

\subsection{Nilpotent groups (write beforehand)}

{\em Say time}: 1 minute

A group $G$ is termed nilpotent of nilpotency class $c$ if:

$$[[ \dots [x_1,x_2],\dots,x_{c+1}] = 1 \ \forall \ x_1,x_2,\dots,x_{c+1} \in G$$

Here, $[x,y] = xyx^{-1}y^{-1}$ is the group commutator. 

{\em Say (don't write)}: We're using the left action convention. Note
that a group is abelian if and only if the commutator of any two
elements is the identity element. For nilpotent, we require the
``higher-order'' versions of commutators, i.e., iterated commutators,
to be trivial.

There's a similar definition for nilpotent Lie ring, with commutator
replaced by Lie bracket.

\subsection{Lazard correspondence}

{\em Say time}: 2 minutes

The global Lazard correspondence is a correspondence:

Some groups (leave space, later write $\pi_c$-powered groups of class
$c$) $\leftrightarrow$ Some Lie rings (leave space, write
$\pi_c$-powered Lie rings of class $c$)


{\em Put arrow-marked $\exp$ and $\log$}

The group and Lie ring have the same underlying set, and there are
formulas for the group operations in terms of the Lie ring operations
(and vice versa) that are inverses of each other.

\begin{itemize}
\item Baker-Campbell-Hausdorff formula: group multiplication in terms
  of Lie ring operations
\item Inverse Baker-Campbell-Hausdorff formulas: Lie ring addition and
  Lie bracket in terms of group multiplication
\end{itemize}

Baker-Campbell-Hausdorff formula:

$$xy + x + y + \frac{1}{2}[x,y] + \frac{1}{12}([x,[x,y]] - [y,[x,y]]) + \dots$$

{\em Say, don't write}: Note that the formula involves division by
some numbers. The weight $c$ term in the formula involves division by
a number all of whose prime divisors are less than or equal to $c$. In
other words, the larger the class we have, the larger the primes we
need to divide by (loosely speaking).

{\em Write}: $\pi_c$ = the set of primes less than or equal to $c$.

$\pi_c$-powered group = group where every element has a unique
$p^{th}$ root for all $p \in \pi_c$

\subsection{Isoclinism}

{\em Say time}: 2 minutes

For any group $G$, the commutator map in $G$ descends to a map of sets:

$$\omega_G: \operatorname{Inn}(G) \times \operatorname{Inn}(G) \to G'$$

An {\em isoclinism} from $G_1$ to $G_2$ is a pair of isomorphisms
$(\zeta,\varphi)$ where $\zeta$ is an isomorphism from
$\operatorname{Inn}(G_1)$ to $\operatorname{Inn}(G_2)$ and $\varphi$ is an
isomorphism from $G_1'$ to $G_2'$, satisfying the condition that:

\begin{equation}\label{eq:homoclinism-pointed}
  \varphi(\omega_{G_1}(x,y)) = \omega_{G_2}(\zeta(x),\zeta(y))
\end{equation}

Pictorially:

$$\begin{array}{ccc}
  \operatorname{Inn}(G_1) \times \operatorname{Inn}(G_1) & \stackrel{\zeta \times \zeta}{\to} & \operatorname{Inn}(G_2) \times \operatorname{Inn}(G_2) \\
  \downarrow^{\omega_{G_1}}  & & \downarrow^{\omega_{G_2}}\\
  G_1' & \stackrel{\varphi}{\to} & G_2'\\
\end{array}$$

Abelian $\iff$ Isoclinic to the trivial group

{\em Say, don't write}: Note that both the inner automorphism group
and the derived subgroup are quantitative measurements of the
``non-abelianness'' of the group. The notion of isoclinism can thus
properly be thought of as saying ``equivalent modulo the subvariety of
abelian groups.'' In particular, a group is abelian if and only if it
is isoclinic to the trivial group.

\section{Erase first few intro boards, then continue}

Total time with this section: 5-10 minutes

\subsection{Global Lazard correspondence up to isoclinism}

{\em Write + say together time}: 3 minutes

Global Lazard correspondence up to isoclinism from a Lie ring $L$ to a
group $G$ is a pair of isomorphisms $(\zeta,\varphi)$ where $\zeta$ is
an isomorphism from $\operatorname{Inn}(L)$ to
$\log(\operatorname{Inn}(G))$ and $\varphi$ is an isomorphism from
$L'$ to $\log(G')$ such that:

$$\varphi(\omega_L(x,y)) = \omega^{\text{Lie}}_G(\zeta(x),\zeta(y))$$

This is equivalent to the requirement that:

$$\varphi(\omega^{\text{Group}}_L(x,y)) = \omega_G(\zeta(x),\zeta(y))$$

(make the diagrams, discuss)

It's a correspondence:

Equivalence classes up to isoclinism of $\pi_c$-powered groups of class $c + 1$ $\leftrightarrow$ Equivalence classes  up to isoclinism of $\pi_c$-powered Lie rings of class $c + 1$

\subsection{Key difficulty: showing existence}

{\em Write + say time}: 2 minutes

We'll apply ideas from group extension theory and Lie ring extension
theory to:

$$0 \to Z(L) \to L \to L/Z(L) \to 0$$

and

$$0 \to Z(G) \to G \to G/Z(G) \to 1$$

{\em Key idea (say, don't write, rest of this section)}: We begin by
viewing $L$ as an extension with central subring $Z(L)$ and quotient
ring $L/Z(L) \cong \operatorname{Inn}(L)$. We obtain the corresponding
Lie bracket map $\operatorname{Inn}(L) \times \operatorname{Inn}(L)
\to L'$. We then obtain a desired commutator map
$\exp(\operatorname{Inn}(L)) \times \exp(\operatorname{Inn}(L)) \to
\exp(L')$ by using the formula describing the commutator map in terms
of the Lie bracket map. Finally, we demonstrate the existence of a
group $G$ that realizes this commutator map.

We will show that equivalence classes of groups up to isoclinism can
be described by storing the commutator structure in an abstract
fashion, without reference to an actual group in that equivalence
class.

This will be useful to the final step of our proof of existence
established above: instead of directly trying to construct the groups
in the equivalence class up to isoclinism, we construct the commutator
structure. In the notation above, we construct the desired commutator
map $\exp(\operatorname{Inn}(L)) \times \exp(\operatorname{Inn}(L))
\to \exp(L')$.

Below, we provide a few more details about how we store the commutator
structure abstractly. This discussion may be accessible only to people
familiar either with group cohomology or with some other type of
cohomology theory that is structurally similar. Note also that the
group $G$ that we use here is not the same as the group $G$ used above.

\subsection{Universal coefficient theorem: technical details}

{\em Say + write time}: 2 minutes

Central extensions with central subgroup $A$ and quotient group $G$:

\begin{equation*}
  0 \to \operatorname{Ext}^1_{\mathbb{Z}}(G^{\operatorname{ab}},A) \to H^2(G;A) \to \operatorname{Hom}(M(G),A) \to 0
\end{equation*}

It splits, but not canonically:

$$H^2(G;A) \cong \operatorname{Ext}^1_{\mathbb{Z}}(G^{\operatorname{ab}},A) \oplus \operatorname{Hom}(M(G),A)$$

The map:

$$H^2(G;A) \to \operatorname{Hom}(M(G),A)$$

classifies extensions up to ``isoclinism of extensions.''

Surjectivity is key.

{\em Say, don't write}: Surjectivity tells us: every homomorphism from
$M(G)$ to $A$ can be ``realized'' via an equivalence class up to
isoclinism of extensions.

We can do something similar on the Lie ring side.

\subsection{Boils down to isomorphism of Schur multipliers}

{\em Say + write time}: 1 minute

Schur multiplier $M(G)$ ``classifies'' equivalence classes up to
isoclinism of extensions of $G$.

Schur multiplier $M(L)$ ``classifies'' equivalence classes up to
isoclinism of extensions of $L$.

We show that if $G$ and $L$ are in global Lazard correspondence, then
$M(G) \cong M(L)$ canonically.

This allows us to prove results for groups that are extensions of
groups in Lazard correspondence, i.e., $\pi_c$ 

\subsection{Option bifurcation}

{\em Say + write time}: 2 minutes

We could either:

\begin{itemize}
\item go over the Baer correspondence up to isoclinism (the case $c =
  1$ and $c + 1 = 2$) in detail. The advantage here is that we could
  delve deeply into examples and understand everything explicitly, or
\item go over the general case, but we will have to hand-wave quite a
  bit and will not be able to cover examples.
\end{itemize}

\section{First option: the Baer correspondence up to isoclinism}

\subsection{The case $c = 1$: the Baer correspondence up to isoclinism}

For now, $G$ and $A$ are abelian groups ($A$ also viewed as an abelian
Lie ring by abuse of notation), and $L$ is an abelian Lie ring.

The short exact sequence classifying extensions of abelian groups is:

\begin{equation}\label{eq:ses-uct-abelian}
  0 \to \operatorname{Ext}^1_{\mathbb{Z}}(G;A) \to H^2(G;A) \to \operatorname{Hom}(G \wedge G,A) \to 0
\end{equation}

(explain in detail each of the maps and what it means)

{\em Say some version of this}: The map:

$$\operatorname{Ext}^1_{\mathbb{Z}}(G;A) \to H^2(G;A)$$

can be interpreted as follows. The underlying set of the group on the
left is canonically identified with the set of all {\em abelian} group
extensions with subgroup $A$ and quotient group $G$. The group on the
right is the group whose elements are all the {\em central} extensions
with central subgroup $A$ and quotient group $G$. Every abelian group
extension is a central extension, and there is therefore a canonical
injective set map from $\operatorname{Ext}^1_{\mathbb{Z}}(G;A)$ to
$H^2(G;A)$.

The map:

$$H^2(G;A) \to \operatorname{Hom}(G \wedge G,A)$$

can be described as follows. For any group extension $E$, the
commutator map $E \times E \to E$ descends to a set map:

$$\omega_{E,G}: G \times G \to A$$

Our earlier definition of $\omega_{E,G}$ defined it as a map to
$[E,E]$, but $[E,E]$ lies in the image of $A$ (under the inclusion of
$A$ in $E$), so it can be viewed as a map to $A$.

Note that the image of the map is in $A$ {\em because} $G$ is
abelian. Further, $\omega_{E,G}$ is bilinear, because the image of the map
is central. It thus defines a group homomorphism $G \wedge G \to A$.

The homomorphism above can also be described in terms of the how it
operates at the level of $2$-cocycles (this description requires
understanding the explicit description of the second cohomology group
using the bar resolution, as given in Section
\ref{sec:cohomology-explicit}). Explicitly, the map:

$$H^2(G;A) \to \operatorname{Hom}(G \wedge G,A)$$

arises from a homomorphism:

$$Z^2(G;A) \to \operatorname{Hom}(G \wedge G,A)$$

given by:

$$f \mapsto \operatorname{Skew}(f)$$

where $\operatorname{Skew}(f)$ is the map $(x,y) \mapsto f(x,y) -
f(y,x)$.

Intuitively, this is because the commutator of two elements represents
the distance between their products in both possible orders, i.e.,
$[x,y]$ is the quotient $(xy)/(yx)$. Whether we use left or right
quotients does not matter because the group has class two.

Based on the discussion in Section \ref{sec:extensionsuptoisoclinism},
the homomorphism:

$$H^2(G;A) \to \operatorname{Hom}(G \wedge G,A)$$

{\em Write}: Similarly for Lie rings:

\begin{equation}\label{eq:ses-uct-lie-abelian}
0 \to \operatorname{Ext}^1_{\mathbb{Z}}(L;A) \to H^2_{\text{Lie}}(L;A) \to \operatorname{Hom}(L \wedge L, A) \to 0
\end{equation}

The short exact sequence splits {\em canonically}, and we get a
canonical isomorphism:

$$H^2_{\text{Lie}}(L;A) \cong \operatorname{Ext}^1_{\mathbb{Z}}(L;A) \oplus \operatorname{Hom}(L \wedge L,A)$$

{\em Say}: We can describe the splitting either by specifying the projection
$H^2_{\text{Lie}}(L;A) \to \operatorname{Ext}^1_{\mathbb{Z}}(L;A)$ or by specifying
the inclusion $\operatorname{Hom}(L \wedge L,A) \to H^2_{\text{Lie}}(L;A)$. We do both.

The projection:

$$H^2_{\text{Lie}}(L;A) \to \operatorname{Ext}^1_{\mathbb{Z}}(L;A)$$

is defined as follows. For any extension Lie ring $M$, map it to the
extension Lie ring that is {\em abelian} as a Lie ring and has the
same additive group as $M$. In other words, keep the additive
structure intact, but ``forget'' the Lie bracket.

The inclusion:

$$\operatorname{Hom}(L \wedge L,A) \to H^2_{\text{Lie}}(L;A)$$

is defined as follows. Given a bilinear map $b:L \times L \to A$,
define the extension Lie ring as a Lie ring $M$ whose additive group
is $L \oplus A$, and where the Lie bracket is:

$$[(x_1,y_1),(x_2,y_2)] = [0,b(x_1,x_2)]$$

In other words, we use the direct sum for the additive structure, and
use the bilinear map to define the Lie bracket.

In light of this, we can think of the direct sum decomposition as follows:

$$H^2_{\text{Lie}}(L;A) \cong \operatorname{Ext}^1_{\mathbb{Z}}(L;A) \oplus \operatorname{Hom}(L \wedge L,A)$$

The projection onto the first component stores the additive structure
of the Lie ring, while destroying, or forgetting, the Lie bracket. The
projection onto the second component preserves the Lie bracket while
replacing the additive structure with a direct sum of $L$ and
$A$. Note also that the latter projection is equivalent to passing to
the associated graded Lie ring.%%TODO: Check statement: (we will
                               %%return to this topic later).

\subsection{Baer correspondence up to isoclinism for extensions}

{\em Write}: We have demonstrated the existence of canonical
isomorphisms between the left groups and between the right groups in
the two short exact sequences:

$$\begin{array}{ccccccccc}
  0 &\to &\operatorname{Ext}^1_{\mathbb{Z}}(G;A) &\to &H^2(G;A) &\to &\operatorname{Hom}(G \wedge G,A) &\to &0\\
  & & \downarrow & & & & \downarrow & & \\
  0 &\to &\operatorname{Ext}^1_{\mathbb{Z}}(L;A) & \to & H^2_{\text{Lie}}(L;A) & \to & \operatorname{Hom}(L \wedge L, A) & \to & 0\\
\end{array}$$

{\em Say}: As described in Sections \ref{sec:ses-uct-abelian} and
\ref{sec:ses-uct-lie-abelian}, both short exact sequences
split. Therefore, it is possible to find an isomorphism $H^2(G;A) \to
H^2_{\text{Lie}}(L;A)$ that establishes an isomorphism of the short
exact sequences.

{\em Do}: Mark canonical, non-canonical isomorphisms.

{\em Say}: Note, however, that the middle isomorphism is not
canonical. In fact, choosing a middle isomorphism is equivalent to
choosing a splitting of the top sequence. This is because the bottom
sequence splits canonically, as we just described.

When we have the actual Baer correspondence, that gives us a canonical
middle isomorphism, or equivalently, a canonical splitting of the
universal coefficient theorem short exact sequence.

(can also explain this in the context of the BCH formula if necessary).

\subsection{Cocycle-level description of the Baer correspondence}\label{sec:baer-correspondence-cocycle-level}

Suppose $G$ and $A$ are abelian groups (we will soon restrict to the
case that one or both of $G$ and $A$ is $2$-powered). Consider the
following two short exact sequences. The first is the short exact
sequence relating the coboundary, cocycle and cohomology groups,
originally described in Section
\ref{sec:ses-coboundary-cocycle-cohomology}:

$$0 \to B^2(G;A) \to Z^2(G;A) \to H^2(G;A) \to 0$$

The second is the universal coefficient theorem short exact sequence,
originally described in Section \ref{sec:ses-uct} and described
specifically for abelian $G$ in Section \ref{sec:ses-uct-abelian}:

$$0 \to \operatorname{Ext}^1_{\mathbb{Z}}(G;A) \to H^2(G;A) \to \operatorname{Hom}(G \wedge G,A) \to 0$$

The first short exact sequence need not split. An example where it
does not split was discussed in Section
\ref{sec:ses-coboundary-cocycle-cohomology}. The second short exact
sequence does always split but the splitting need not be canonical (see
Section \ref{sec:ses-uct-non-canonical-splitting}).

The right parts of these short exact sequences give surjective
homomorphisms, which we can compose:

$$Z^2(G;A) \to H^2(G;A) \to \operatorname{Hom}(G \wedge G,A)$$

As we discussed in Section \ref{sec:ses-uct-abelian-right-map}, the
composite of these maps is the skew map. Explicitly, the composite is
the map $f \mapsto \operatorname{Skew}(f)$, that sends a function $f$
to the function:

$$\operatorname{Skew}(f) = (x,y) \mapsto f(x,y) - f(y,x)$$

Note that the function $\operatorname{Skew}(f)$ is a
$\mathbb{Z}$-bilinear map $G \times G$ to $A$, which can be
interpreted as a homomorphism $G \wedge G \to A$.

Now, suppose that $G$ and $A$ are both $2$-powered abelian
groups.\footnote{The assumption can be modified to requiring that any
  one of $G$ and $A$ be $2$-powered, in which case we will need to use
  one of the generalizations of the Baer correspondence described in
  Section \ref{sec:baer-correspondence-definition-relaxation}, but we
  do not describe it here since it is not necessary for our purpose.}

In that case, there is a canonical splitting of the composite map,
given as follows:

$$f \mapsto \frac{1}{2}f$$

In other words, a $\mathbb{Z}$-bilinear map $f: G \times G \to A$ is
sent to $\frac{1}{2}f:G \times G \to A$. Note that any
$\mathbb{Z}$-bilinear map is a $2$-cocycle (in general, any $n$-linear
map is a $n$-cocycle) so this works.

In particular, {\em both} the short exact sequences split, and we get
canonical direct sum decompositions:

$$Z^2(G;A) \cong B^2(G;A) \oplus H^2(G;A), \text{ splitting is } H^2(G;A) \to Z^2(G;A)$$

$$H^2(G;A) \cong \operatorname{Ext}^1_{\mathbb{Z}}(G;A) \oplus \operatorname{Hom}(G \wedge G,A), \text{ splitting is } \operatorname{Hom}(G \wedge G,A) \to H^2(G;A)$$

Note that the first short exact sequence need not split for all $G$
and $A$ (see the discussion in Section
\ref{sec:ses-coboundary-cocycle-cohomology}) and the existence of a
splitting is itself a piece of information. The second short exact
sequence does split for all $G$ and $A$, but the splitting is not in
general canonical, as discussed in Section
\ref{sec:ses-uct-non-canonical-splitting}, so the case where $G$ and
$A$ are both $2$-powered is special in that we obtain a {\em
  canonical} splitting.

The splitting map $\operatorname{Hom}(G\wedge G,A) \to H^2(G;A)$ is
the same as the one arising from the Baer correspondence. Explicitly,
as noted in Section \ref{sec:bcuti-extensions-splitting}, specifying the
splitting map $\operatorname{Hom}(G \wedge G, A) \to H^2(G;A)$ is
equivalent to specifying an isomorphism of $H^2(G;A)$ and
$H^2_{\text{Lie}}(L;A)$ such that the diagram below commutes:

$$\begin{array}{ccccccccc}
  0 &\to &\operatorname{Ext}^1_{\mathbb{Z}}(G;A) &\to &H^2(G;A) &\to &\operatorname{Hom}(G \wedge G,A) &\to &0\\
  & & \downarrow & & \downarrow & & \downarrow & & \\
  0 &\to &\operatorname{Ext}^1_{\mathbb{Z}}(L;A) & \to & H^2_{\text{Lie}}(L;A) & \to & \operatorname{Hom}(L \wedge L, A) & \to & 0\\
\end{array}$$

This isomorphism can be described in an alternative way. Let $E$ be an
extension group corresponding to an element of $H^2(G;A)$. Let $N =
\log(E)$ via the Baer correspondence. We can relate two short exact
sequences via a $\log$ functor.

$$\begin{array}{lllll}
    0 \to & A \to & E \to & G \to & 1 \\
    & \downarrow^{\log} & \downarrow^{\log} & \downarrow^{\log}&  \\
    0  \to & A \to & N \to & L \to & 0\\
\end{array}$$

Note that we abuse notation again, using the same letter $A$ for $A$
as a group and as a Lie ring.

Then, the element of $H^2_{\text{Lie}}(L;A)$ that corresponds to the
second row is the same as the image of the element of $H^2(G;A)$ under
the isomorphism described earlier.

\subsection{A setting where the Baer correspondence works only up to isoclinism}\label{sec:bcuti-ex}

In the case that $G$ and $A$ are odd-order abelian groups, the {\em
  original} Baer correspondence works. To obtain finite examples where
the Baer correspondence works only up to isoclinism, we need to look
at $2$-groups. Further, our examples must be cases where the quotient
$\operatorname{Hom}(G \wedge G,A)$ is nontrivial, so that there is at
least some non-abelian extension.\footnote{The abelian extensions can
  be put in correspondence based on the correspondence between abelian
  groups and abelian Lie rings, which, although not strictly part of
  the Baer correspondence as have defined it, falls under the
  generalization (1) of it described in Section
  \ref{sec:baer-correspondence-definition-relaxation}}

The smallest sized example is: $A = \mathbb{Z}_2$ is the cyclic group
of order $2$ and $G = V_4$ is the Klein four-group, isomorphic to
$\mathbb{Z}_2 \times \mathbb{Z}_2$.

The short exact sequences discussed in Sections
\ref{sec:ses-uct-abelian} and \ref{sec:ses-uct-lie-abelian}, along
with the canonical isomorphisms discussed in Section
\ref{sec:bcuti-extensions}, give the following:

$$\begin{array}{ccccccccc}
0 & \to & \operatorname{Ext}^1(V_4;\mathbb{Z}_2) & \to & H^2(V_4;\mathbb{Z}_2) & \to & \operatorname{Hom}(V_4 \wedge V_4,\mathbb{Z}_2) & \to & 0 \\
& & \downarrow & & & & \downarrow & & \\
0 & \to & \operatorname{Ext}^1(V_4;\mathbb{Z}_2) & \to & H^2_{\text{Lie}}(V_4;\mathbb{Z}_2) & \to & \operatorname{Hom}(V_4 \wedge V_4,\mathbb{Z}_2) & \to & 0\\
\end{array}$$

Recall that both short exact sequences split, and, as per the
discussion in Section \ref{sec:ses-uct-lie-abelian}, the Lie ring
short exact sequence splits canonically (with the splitting separating
out the addition and Lie bracket parts).

It turns out that:

\begin{itemize}
\item $\operatorname{Ext}^1(V_4;\mathbb{Z}_2)$ is itself isomorphic to $V_4$, the Klein four-group.
\item $\operatorname(V_4 \wedge V_4)$ is isomorphic to $\mathbb{Z}_2$,
  and thus, $\operatorname{Hom}(H_2(V_4;\mathbb{Z}),\mathbb{Z}_2)$ is
  isomorphic to $\mathbb{Z}_2$.
\item Thus, both of the second cohomology groups (the group and Lie
  ring side) are isomorphic to the elementary abelian group of order
  eight.
\end{itemize}

On the group side, we have the following eight extensions (eight being
the order of the cohomology group):

\begin{enumerate}[(a)]
\item Elementary abelian group of order eight (1 time).
\item $\mathbb{Z}_4 \oplus \mathbb{Z}_2$ (3 times).
\item $D_8$ (3 times).
\item $Q_8$ (1 time).
\end{enumerate}

(a) and (b) together form the image of $\operatorname{Ext}^1$ (total
size 4) while (c) and (d) form the non-identity coset of that
image.

On the Lie ring side, the eight extensions (eight being the order of the cohomology group) are:

\begin{enumerate}[(a)]
\item Abelian Lie ring, additive group elementary abelian of order eight (1 time)
\item Abelian Lie ring, additive group direct product of
  $\mathbb{Z}_4$ and $\mathbb{Z}_2$ (3 times).
\item The niltriangular matrix Lie ring ($3 \times 3$ strictly upper
  triangular matrices) over the field of two elements. (1 time)
\item The semidirect product of $\mathbb{Z}_4$ and $\mathbb{Z}_2$ as
  Lie rings. (3 times).
\end{enumerate}

(a) and (b) together form the image of $\operatorname{Ext}^1$ (total
size 4) while (c) and (d) form the non-identity coset of that
image.

Note that there is no canonical bijection between the set of eight
group extensions and the set of eight Lie ring extensions, but we can
naturally correspond the images of $\operatorname{Ext}^1$ in both. The
problem arises when attempting an element-to-element identification of the
non-identity cosets in the two cases. In other words, we have a
correspondence at a coset level:

$$\{ \text{The four non-abelian Lie ring extensions} \} \leftrightarrow \{ D_8, D_8, D_8, Q_8 \}$$

But there is no clear-cut way of making sense of {\em which} Lie ring
extension to correspond to {\em which} group. This is an example of a
situation where the Baer correspondence up to isoclinism does not seem
to have any natural refinement to a correspondence up to isomorphism.

Note that in this case, it so happens that we can use an
automorphism-invariance criterion and get a unique
automorphism-invariant bijection. This would map the niltriangular
matrix Lie ring to the quaternion group and the semidirect product of
$\Z_4$ and $\Z_2$ to the dihedral group. However, this does not give a
meaningful bijection at the level of elements. For instance, one
feature that holds in all generalizations described so far for the
Baer correspondence is that the correspondence restricts to
isomorphism between cyclic subgroups and cyclic subrings. In
particular, the multiset of the orders of the elements in the group
must match the multiset of the orders of the elements in the additive
group of the Lie ring. However, the multiset of the orders of the
elements of $D_8$ does not match the multiset of the orders of
elements in any abelian group of order $8$. The same is true of $Q_8$.


\section{Second option: going over the general case, but in a handwavy fashion}

\subsection{Optional extra: explain about the exterior square and Schur multiplier}

{\em Write header}: Exterior square

{\em Say, don't write}: Recall how we define exterior square of a {\em
  vector space}. We have a concept of alternating bilinear maps from
the vector space. The exterior square is an object such that
homomorphisms from it correspond to alternating bilinear maps from the
vector space.

{\em Write}: Exterior square of $G$ is the ``freest'' group $G \wedge G$
generated by formal symbols $x \wedge y$, $x,y \in G$, such that for
any central extension $E$ of $G$, if we consider the map:

$$\omega_{E,G}: G \times G \to [E,E]$$ (slight variant of the
$\omega_E$ described above)

this extends to a group homomorphism:

$$\Omega_{E,G}: G \wedge G \to [E,E]$$

{\em If there's time}: Originally introduced by Miller (1952), later
described by Brown and Loday (1987) and Graham Ellis (1987).

There is an explicit presentation with generators and relations.

{\em Say (don't write)}: There is a similar definition of the exterior
square of a Lie ring. Note that the exterior square of a Lie ring as a
Lie ring is not the same as its exterior square as an abelian
group. It is a quotient of that, however.

{\em Write}: There's a canonical short exact sequence (central extension):

$$0 \to M(G) \to G \wedge G \to [G,G] \to 1$$

Right map, on a generating set: $$x \wedge y \mapsto [x,y]$$

$M(G)$ = Schur multiplier of $G$

{\em Say (don't write)}: We can think of $M(G)$ as the formal products
of expressions of the form $x \wedge y$ that, when we evaluate in $G$,
become trivial. This has the flavor of looking at central extensions. Indeed:

{\em Write}: 

$$\begin{array}{ccccccccc}
0 & \to & M(G) & \to & G \wedge G & \to & [G,G] & \to & 1\\
 &&               && \downarrow     &&\downarrow&& \\
0 & \to & A & \to & E & \to & G & \to & 1\\
\end{array}$$

{\em Do}: make dashed arrow $M(G) \to A$.  This is the map $\beta:M(G)
\to A$, same as the element of $\operatorname{Hom}(M(G),A)$ that
appears in the universal coefficient theorem short exact sequence.

{\em Write}: Let $B$ be the image of $\beta$ and $\beta': M(G) \to B$
be the restriction. We have a map:

$$\begin{array}{ccccccccc}
0 & \to & M(G) & \to & G \wedge G & \to & [G,G] & \to & 1\\
 &&   \downarrow^{\beta'}  &&  \downarrow^{\Omega_{E,G}}     && \downarrow^{\text{id}} && \\
0 & \to & B &\to & [E,E] & \to & [G,G] & \to & 1\\
\end{array}$$

{\em Say or write}: Knowing the map $M(G) \to B$ determines the map
$\Omega_{E,G}$. Latter classifies extensions up to isoclinism, hence
so does the former.

{\em Say}: Analogous results hold for the Lie ring. The key aspect we
need to prove is that the Schur multiplier, which classifies
extensions up to isoclinism, is the same for the group and the Lie
ring, if the group and the Lie ring are in Lazard correspondence. This
will allow us to draw conclusions about the central extension groups
and Lie rings, that has class one more than the original group and Lie
ring. We will in fact show that the short exact sequences:

{\em Write}:

$$0 \to M(L) \to L \wedge L \to [L,L] \to 0$$

and

$$0 \to M(G) \to G \wedge G \to [G,G] \to 1$$

are in Lazard correspondence up to (canonical) isomorphism.

{\em Say}: {\em Why this is plausible}: Note that if $G$ has class
$c$, $G \wedge G$ has class at most $\lfloor (c + 1)/2 \rfloor$, so if
$G$ is in the domain of the Lazard correspondence, so is $G \wedge G$.

{\em Why this is nontrivial}: The intermediate formulas that we would
use to calculate $G \wedge G$ and $L \wedge L$ are inside groups and
Lie rings that are not in Lazard correspondence (e.g., if we use the
Hopf formula). We need to show that despite this, the subgroups that
end up getting used in the formula are in Lazard correspondence.

\subsection{The finite case}

Global Lazard correspondence:

finite $p$-groups of class $\le p - 1$ $\leftrightarrow$ finite
$p$-Lie rings of class $\le p - 1$ (write below: additive group is a
finite $p$-group)

Global Lazard correspondence up to isoclinism:

equivalence classes up to isoclinism of finite $p$-group of class $\le
p$ $\leftrightarrow$ equivalence classes up to isoclinism of finite
$p$-Lie rings of class $\le p$

\end{document}
