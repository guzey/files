\documentclass[10pt]{amsart}

%Packages in use
\usepackage{fullpage, hyperref, vipul}

%Title details
\title{Lie cring: key results}
\author{Vipul Naik}

%List of new commands
\newcommand{\Skew}{\operatorname{Skew}}

\begin{document}
\maketitle

\section*{Generic facts with precise statements and wide-ranging consequences}

Below is a list of nontrivial results for which we have precise
statements of and (almost) have proofs of. Most of these are
formulated in the language of crings, which is a nice structure; many
of them have alternative phrasings that don't involve any cring
terminology.

\begin{enumerate}

\item If we start with a cring and take its skew, we get something
  distributive (easy to prove, not impressive per se)

\item Two near-Lie crings are homologous if and only if they have the same
  skew (can be formulated without using the "cring" terminology) (tricky
  to prove, has deeper implications about "uniqueness" of division by
  two)

\item For the BCH formula, the denominators grow slowly enough for us
  to make "smooth" versions of these that give us potential formulas
  for crings. (Not our result; this is the one proved by Alekseev,
  though it was probably shown originally by Lazard.)

\item The 2-smooth version of the BCH formula, if applied to a
  nilpotent near-Lie cring (satisfying 3-additivity and 3-skew
  symmetry), gives a group. In other words, exponentiation of a
  nilpotent near-Lie cring works and gives a group: We have a schema for
  the proof, but need to figure out exactly how to write it. We have
  explicit proofs for class two and class three.

\item If we start with a nilpotent near-Lie cring (satisfying
  3-additivity and 3-skew symmetry) of class $n$, we can skew it to
  get a Lie ring, and use the 2-smooth BCH formula to get a group. The
  Lie ring and the group have exactly the same nilpotency class, and
  this is $\le n$, possibly smaller. So, we could have a class $k$
  group with a class $k$ Lie ring but the Lie cring we need to go via
  has class three: We don't have a general proof, but some
  computations that suggest how a general proof would proceed.

\end{enumerate}

\section*{Generic facts we don't have precise statements for yet}

These are all contingent on a definition of prings: A ring along with
a piece of information that plays the role of ``division by $p$''. For
$p = 2$, a pring should just turn out to be a cring. 

\begin{enumerate}
\item Analogue to (4) above: The $p$-smooth version of the BCH formula,
  if applied to a (near-?)Lie pring, gives a group. We have some idea
  of what this might mean and why it might be true, but need to
  formulate definitions more clearly.
\item Analogue to (5) above.
\end{enumerate}

\section*{Interpretation in finite groups, $2$-groups, etc.}

The key for all points below is that the existence of a cring/pring
gives a corresponding Lie ring, and a $1$-isomorphism between the
additive group of the Lie ring (same as that of the cring/pring) and
the group. Thus, it provides {\em an explanation of $1$-isomorphisms}.
\begin{enumerate}
\item Existence of class two Lie crings and near-Lie crings
  interpreted in terms of extension theory, cohomology, and splitting.
\item Necessary and sufficient condition for class two $2$-group to
  have class two Lie cring: $P' \le \mho^1(Z(P))$. Necessity pretty
  much clear with the definitional setup, sufficiency requires some
  cool basic linear algebra manipulation to construct cocycles.
\item Some sufficient conditions, and other necessary conditions, for
  class two $2$-group to have class two near-Lie crings. There is
  quite a gap here between the necessary and the sufficient.
\item All $1$-isomorphisms between abelian and non-abelian groups
  ``explained'' via crings for orders $16$ and $32$. Probably all for
  order $64$ too? Still under investigation. True in general? What
  about $1$-isomorphisms between non-abelian pairs? What about
  $1$-isomorphisms for odd $p$ that are outside the scope of the
  Lazard correspondence?
\item Also gives an explanation for the absence of $1$-isomorphisms
  and ``Lie rings'' for many groups, in terms of cohomology theory,
  and in terms of the existence of crings that are not nilpotent.
\item Lie trings (prings for $p = 3$) explain {\em some}
  $1$-isomorphisms for order $3^5$, but probably not all. Need notion
  of {\em near}-Lie pring/tring?
\end{enumerate}

\end{document}