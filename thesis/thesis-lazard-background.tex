\section{Adjoint groups and the exponential and logarithm maps}\label{sec:adjoint-exp-log}

\subsection{Remarks on the approach followed from this point onward}

Many of the identities that we will obtain in this section and the
subsequent sections are {\em formal} identities. They make sense in a
wide array of situations, if appropriately interpreted. Below, we
outline our typical logic flow.

\begin{itemize}
\item Some of our identities start off as identities involving
  infinite series that make sense over the reals, or over real Lie
  groups. In those contexts, the identities may have specific
  interpretations related to differential equations, although those
  interpretations do not concern us directly.
\item Our identities are valid {\em formally} in a noncommutative (but
  associative) power series algebra over $\Q$. Note that we need to
  use power series algebras because the identities involve infinite
  series.
\item As a result, our identities are valid in free nilpotent
  associative $\Q$-algebras of nilpotency class $c$, where the
  identities get truncated to expressions of finite length. We
  therefore get identities involving the truncated expressions of
  finite length with the assumption about nilpotency class.
\item We notice that the truncated expressions, for which the identity
  is valid, have only a finite number of coefficients, that in turn
  have only a finite number of prime divisors of their
  denominators. Typically, the truncated expression to class $c$ uses
  only the primes that are less than or equal to $c$. Denote
  this prime set by $\pi_c$.
\item We then notice that our identities (in truncated form) are valid
  in free nilpotent (associative) $\Z[\pi_c^{-1}]$-algebras of
  nilpotency class $c$.
\item We conclude that our identities (in truncated form) are valid in
  all nilpotent (associative) $\Z[\pi_c^{-1}]$-algebras of nilpotency
  class at most $c$, on account of our being able to express such
  algebras as quotients of the free nilpotent $\Z[\pi_c^{-1}]$-algebras
  of nilpotency class $c$.
\end{itemize}

\subsection{Some background on adjoint groups and algebra groups}\label{sec:adjoint-group}

An associative (not necessarily unital) ring $N$ is termed a {\em
  radical ring} if for every $x \in N$ there exists $y \in N$ such
that $x + y + xy = 0$.

For any radical ring $N$, we define the {\em adjoint group}
corresponding to $N$ as the set $1 + N$, i.e., the set of formal symbols:

$$\{ 1 + x \mid x \in N \}$$

equipped with the multiplication:

$$(1 + x)(1 + y) = 1 + (x + y + xy)$$

The identity element for this adjoint group is $1 + 0$ (simply denoted
as $1$). For any $x \in N$, the inverse of $1 + x$ is the element $1 +
y$ where $y$ is an element satisfying $x + y + xy = 0$. Such an
element exists by the assumption that $N$ is radical. The uniqueness
and two-sidedness of inverses follows from group
theory.\footnote{Specifically, if every element of a monoid has a
  right inverse, then every element has a two-sided inverse and the
  two-sided inverse is unique.}

An {\em algebra group} over a field $F$ is defined as a group arising
as the adjoint group corresponding to an associative algebra $N$ over
$F$ that is also a radical ring.

Suppose $G$ is an algebra group over a field $F$ corresponding to a
radical ring $N$ that is an associative algebra over $F$. A subgroup
$H$ of $G$ is termed an {\em algebra subgroup} if $H = 1 + M$ for a
subalgebra $M$ of $N$ (note that $M$ must also be a radical ring for
$H$ to be a subgroup).

The following facts about algebra groups can be easily checked.

\begin{enumerate}
\item An associative ring in which every element is nilpotent is a
  radical ring. In particular, an associative ring that is itself
  nilpotent is a radical ring.

  The proof of this assertion relies on the observation that if $x^n =
  0$, then the element $y = \sum_{i=1}^{n-1} (-1)^ix^i = -x + x^2 -
  x^3 + \dots + (-1)^{n-1}x^{n-1}$ satisfies $x + y + xy =
  0$. Secretly, the expression above relies on expanding $(1 -
  (-x)^n)(1 - (-x))$ as a power series in $x$.
\item Suppose $K$ is a field extension of a field $F$. Then, any
  $K$-algebra group naturally acquires the structure of a $F$-algebra
  group. In particular, any algebra group over a field of
  characteristic zero is a $\mathbb{Q}$-algebra group. Similarly any
  algebra group over a field of characteristic $p$ is a
  $\mathbb{F}_p$-algebra group.
\item It is possible to have two non-isomorphic
  $\mathbb{F}_p$-algebras $N_1$ and $N_2$ such that the algebra group
  corresponding to $N_1$ is isomorphic (as an abstract group) to the algebra group
  corresponding to $N_2$. In fact, we can construct examples of
  non-isomorphic associative algebras over $\mathbb{F}_2$ whose
  corresponding algebra groups are both isomorphic to
  $\mathbb{Z}/4\mathbb{Z} \oplus \mathbb{Z}/2\mathbb{Z}$.
\item Suppose $N$ is a radical $\mathbb{F}_p$-algebra where $p$ is a
  prime number. Then, we have $(1 + x)^p = 1 + x^p$ for all $x \in
  N$. In other words, the $p^{th}$ power map in the algebra and the
  algebra group correspond to each other.
\item A finite $\mathbb{F}_p$-algebra $N$ is radical if and only if
  every element of $N$ is nilpotent. One direction was already
  established in (1). For the reverse direction, note that if $1 + x$ has
  finite order $p^k$, then $1 + x^{p^k} = (1 + x)^{p^k} = 1$, so
  $x^{p^k} = 0$, and thus, $x$ is nilpotent.
\item For any field $F$ and a positive integer $n$, consider the group
  $UT(n,F)$ of upper triangular unipotent matrices over $F$. $UT(n,F)$
  is the algebra group corresponding to $NT(n,F)$, the strictly upper
  triangular matrices over $F$, where we view $NT(n,F)$ as an {\em
    associative} $F$-algebra with the usual addition and
  multiplication of matrices.
\item Any $\mathbb{F}_q$-algebra group $G = 1 + N$ of order $q^m$ is
  isomorphic to an algebra subgroup of the algebra group $UT(m+1,q) =
  UT(m+1,\mathbb{F}_q)$. The proof idea is to consider $G$ as a
  multiplicative subgroup of the ring $N + \mathbb{F}_q$ (the {\em
    unitization} of $N$) and then consider the action of $G$ on the
  underlying vector space of $N + \mathbb{F}_q$ by
  multiplication. This action is faithful, and defines an injective
  homomorphism $G \to GL(m+1,\mathbb{F}_q)$. By Sylow's theorem, $G$
  can be conjugated to a subgroup inside any $p$-Sylow subgroup of
  $GL(m+1,q)$ (where $p$ is the underlying prime of $q$). $UT(m+1,q)$
  is one such $p$-Sylow subgroup.
\end{enumerate}

\subsection{Exponential map inside an associative ring: the torsion-free case}\label{sec:exp-and-log-unital}

Suppose $R$ is an associative unital ring. For now, assume that the
additive group of $R$ is torsion-free (we will later relax the
assumption). An element $x \in R$ is termed {\em exponentiable} if the
following sum makes sense, and if so the sum is termed the {\em
  exponential} of $x$:

$$\exp(x) = e^x = \sum_{m=0}^\infty \frac{x^m}{m!} = 1 + x + \frac{x^2}{2!} + \frac{x^3}{3!} + \dots$$

The notations $\exp(x)$ and $e^x$ are both used. The $\exp$ notation
is more helpful when the argument to the function is complicated and
cumbersome to write in a superscript.

Here, $x^0 = 1$, and $x^m/m!$ is the unique element $y \in R$ such
that $m!y = x^m$. Note that uniqueness follows from our assumption
that $R$ is torsion-free.

For the sum to make sense, we need two conditions:

\begin{itemize}
\item $x$ is nilpotent, i.e., there exists a natural number $n$ such
  that $x^n = 0$. The smallest such $n$ is the {\em nilpotency} of $x$.
\item For all positive integers $m < n$, $m!$ divides $x^m$, i.e.,
  there exists an element $y \in R$ (unique by the torsion-free
  assumption) such that $m!y = x^m$.
\end{itemize}

If both the above conditions hold, then we can rewrite:

$$e^x = \sum_{m=0}^{n - 1} \frac{x^m}{m!}$$

For an element $x \in R$, we say that $x$ is {\em logarithmable} if
there exists a positive integer $n$ such that $(x - 1)^n = 0$, {\em and} the following can be computed:

$$\log x := (x - 1) - \frac{(x - 1)^2}{2} + \frac{(x - 1)^3}{3} - \dots + \frac{(-1)^n(x - 1)^{n-1}}{n}$$

Explicitly, $x$ is logarithmable if the following two conditions hold:

\begin{itemize}
\item $x$ is {\em unipotent}, or equivalently, $x - 1$ is nilpotent,
  i.e., there exists a natural number $n$ such that $(x - 1)^n =
  0$. The smallest such $n$ is termed the {\em unipotency} of $x$.
\item For all positive integers $m < n$, $m$ divides $(x - 1)^m$,
  i.e., there exists an element $y \in R$ such that $my = (x - 1)^m$.
\end{itemize}

The following can be deduced from formal manipulation:

\begin{itemize}
\item If $x \in R$ is exponentiable and $e^x$ is logarithmable, then
  $\log(e^x) = x$.
  
  %% It can also be shown ({\em TONOTDO: Can it? Do calculations in
  %%   appendix}) that if $x \in R$ is exponentiable then $e^x$ is
  %%   logarithmable, but this is harder and we will not need it for our
  %%   results.
\item If $x \in R$ is logarithmable and $\log x$ is exponentiable, then
  $e^{\log x} = x$.

  %% It can also be shown ({\em TONOTDO: Can it? Do calculations in
  %%   appendix}) that if $x \in R$ is logarithmable then $\log x$ is
  %% exponentiable, but this is harder and we will not need it for our
  %% results.
\end{itemize}

This follows from the fact that the identities hold formally on
account of these being the usual Taylor series for the exponential and
logarithm functions that are inverses of each other.
    
\subsection{Exponential to and logarithm from the adjoint group}\label{sec:exp-and-adjoint}

Suppose $N$ is an associative (but not necessarily commutative and
generally {\em not} unital) ring whose additive group is torsion-free
and $x$ is a nilpotent element of $N$ satisfying the following two
conditions:

\begin{itemize}
\item $x$ is nilpotent, i.e., there exists a natural number $n$ such
  that $x^n = 0$. The smallest such $n$ is the {\em nilpotency} of $x$.
\item For all positive integers $m < n$, $m!$ divides $x^m$, i.e.,
  there exists an element $y \in R$ (unique by the torsion-free
  assumption) such that $m!y = x^m$.
\end{itemize}

Then, we can make sense of the element $e^x$ as an element of the
adjoint group $1 + N$. Explicitly:

$$e^x := 1 + \left(x + \frac{x^2}{2!} + \frac{x^3}{3!} + \dots \frac{x^{n-1}}{(n - 1)!}\right)$$

Similarly, for an element $1 + x, x \in N$, we say that $1 + x$ is
{\em logarithmable} if the following two conditions hold:

\begin{itemize}
\item $x$ is nilpotent, i.e., there exists a natural number $n$ such
  that $x^n = 0$. The smallest such $n$ is the {\em nilpotency} of $x$.
\item For all positive integers $m < n$, $m$ divides $x^m$, i.e.,
  there exists an element $y \in R$ (unique by the torsion-free
  assumption) such that $my = x^m$.
\end{itemize}

We then define $\log(1 + x)$ as an element of $N$:

$$\log(1 + x) = x - \frac{x^2}{2} + \frac{x^3}{3} - \dots + \frac{(-1)^nx^{n-1}}{n - 1}$$

We have observations similar to those in Section \ref{sec:exp-and-log-unital}:

\begin{itemize}
\item If $x \in N$ is exponentiable and $e^x$ is logarithmable, then
  $\log(e^x) = x$.
\item If $x \in N$ is logarithmable and $\log x$ is exponentiable, then
  $e^{\log x} = x$.
\end{itemize}

Note that this notion of exponential differs slightly from the
preceding definition in that the exponential of an element is now no
longer in the ring but rather in its adjoint group. However, we can
embed $N$ inside its unitization $R = N + \mathbb{Z}$ and both the
definitions would then agree.\footnote{Note that the definition of
  ``unitization'' depends on what commutative unital ring we are
  considering $N$ as an algebra over. The default assumption is to
  treat it as an algebra over $\Z$, in which case the unitization is
  $N + \Z$. If, howevever, we are viewing $N$ as an algebra over a
  field $F$, the unitization is $N + F$.} In fact, via this method, we
  can deduce all results here from the results of Section
  \ref{sec:exp-and-log-unital} without redoing any of the work.

\subsection{The exponential and logarithm as global maps}\label{sec:exp-log-global}

Suppose $N$ is an associative ring whose additive group is
torsion-free and in which {\em every} element is exponentiable (to the
adjoint group) and {\em every} element of the adjoint group is
logarithmable. Then, the exponential can be defined as a {\em global}
set map:

$$\exp:N \to 1 + N$$

In this case, the logarithm is also a global map:

$$\log:1 + N \to N$$

and further, the two maps are inverses of each other.

The following are some cases where these hypotheses are satisfied:

\begin{itemize}
\item The case that $N$ is a nilpotent $\mathbb{Q}$-algebra. This will
  be the case that interests us the most in the beginning.
\item The case that $N$ is a torsion-free $\Z[\pi_c^{-1}]$-algebra of
  nilpotency class at most $c$ where $\pi_c$ is the set of all primes
  less than or equal to $c$. Many of our generalizations will apply to
  this case. (See the next subsection regarding relaxation to the
  non-torsion-free case).
\end{itemize}

\subsection{Truncated exponentials and the case of torsion}\label{sec:truncated-exponentials}

So far, we have considered the definition of exponential and logarithm
in the context of torsion-free additive groups. The torsion-free
assumption is significant because it guarantees that all the summands
of the form $x^n/n!$ or $x^n/n$ are {\em uniquely defined} if they
exist. We now consider whether this torsion-free assumption can be
relaxed somewhat.

In the case that an element $x$ satisfies $x^n = 0$ for some natural
number $n$, we can truncate the exponential series and get:

$$e^x = 1 + \left(x + \frac{x^2}{2!} + \dots + \frac{x^{n-1}}{(n - 1)!}\right)$$

Similarly, we can truncate the logarithm series and get:

$$\log(1 + x) = x - \frac{x^2}{2} + \frac{x^3}{3} - \dots + \frac{(-1)^nx^{n-1}}{n - 1}$$

Thus, we can relax the {\em torsion-free} assumption to the assumption
that the ring is $\pi$-torsion-free where $\pi$ is the set of all
primes strictly less than the nilpotency. Alternatively, if $N$ is a
nilpotent associative ring of nilpotency class $c$ (explicitly, this
means that $x_1x_2 \dots x_cx_{c+1} = 0$ for all $x_1$, $x_2$,
$\dots$, $x_c$, $x_{c+1}$ $\in N$) that is
$\pi_c$-torsion-free where $\pi_c$ is the set of primes less than or
equal to $c$, then we can make unique sense of the terms $x^m/m!$ and
$x^m/m$ used in the definitions of the exponential and logarithm maps.

In particular, if $N$ is a $\Z[\pi_c^{-1}]$-algebra of nilpotency class
at most $c$, then the exponential and logarithm maps make sense globally.

There is, however, a small caveat. Namely, the process of truncating
the exponential map involves a choice, even though the choice we have
made is a canonical choice. Consider again the infinite series for the
exponential:

$$e^x = 1 + \left(x + \frac{x^2}{2!} + \frac{x^3}{3!} + \dots\right)$$

The $m^{th}$ term of the summation is $\frac{x^m}{m!}$. In the case
that $m \ge n$, the numerator is the zero element of $N$. We are
therefore trying to make sense of the computation:

$$\frac{0}{m!}$$

A canonical candidate for the answer is $0$. However, if $N$ has
$p$-torsion for some prime $p$ less than or equal to $m$, this is not
the {\em unique} candidate for the answer. Our decision to truncate
the exponential implicitly involves making the canonical choice of the
answer of $0$ for all terms $x^m/m!$, even though this choice is not
uniquely fixed.

Despite the non-uniqueness of these choices, all the formal
manipulations involving exponentials and logarithms continue to be
valid. A quick explanation for this is as follows: all the proofs for
these manipulations involve using the existing (truncated) expressions
and then applying the operations of addition, subtraction,
multiplication, and composition. None of these operations is capable
of introducing new primes into the denominators. Thus, we do not ever
need to confront the non-uniqueness of division by larger primes when
mimicking the proofs that work over the rational numbers.

%%A slightly different version of the same argument follows.

\subsection{Exponential and logarithm maps preserve abelian and cyclic subgroup structures}

The following lemmas follow from manipulation similar to the formal
manipulation used when dealing with power series over the reals. We
therefore omit the proofs.

\begin{lemma}\label{lemma:exp-and-commutative}
  Suppose $c$ is a positive integer and $\pi_c$ is the set of primes
  less than or equal to $c$. Suppose $N$ is an associative
  $\Z[\pi_c^{-1}]$-algebra of nilpotency class less than or equal to
  $c$. Suppose $x,y \in N$ are elements such that $xy = yx$. Then, the
  exponential map $\exp: N \to 1 + N$ satisfies the condition that:

  $$\exp(x + y) = \exp(x) \exp (y)$$
\end{lemma}

\begin{lemma}\label{lemma:exp-and-powering}
  Suppose $c$ is a positive integer and $\pi_c$ is the set of primes
  less than or equal to $c$. Suppose $N$ is an associative
  $\Z[\pi_c^{-1}]$-algebra of nilpotency class less than or equal to
  $c$. Suppose $x \in N$ and $n \in \Z$. Then, the exponential map
  $\exp: N \to 1 + N$ satisfies the condition that:

  $$\exp(nx) = (\exp(x))^n$$
\end{lemma}

This follows from the preceding lemma, combined with a proof by
induction.

%\newpage

\section{Free nilpotent groups and the exponential and logarithm}\label{sec:free-nilpotent-exp-log}

\subsection{Free associative algebra and free nilpotent associative algebra}\label{sec:free-associative-algebra}

The notation and results here follow Khukhro's book \cite{Khukhro},
Chapters 9 and 10. For brevity, we omit some proofs and provide
citations to Khukhro.

For this and the next few subsections, $c$ is a fixed but arbitrary
positive integer. The algebraic structures $F$, $A$, and $L$ are all
dependent on $c$. However, to keep the notation as uncluttered as
possible, we will not use $c$ as an explicit parameter to these.

Denote by $\mathcal{A}$ the free associative $\mathbb{Q}$-algebra on a
generating set $S = \{ x_1,x_2,\dots \}$ The generating set may have
any cardinality. By the well-ordering principle, we will index the
generating set by a well-ordered set. 

The algebra $\mathcal{A}$ is naturally a {\em graded} associative
algebra. Explicitly, the $i^{th}$ graded component of $\mathcal{A}$ is
the $\mathbb{Q}$-subspace generated by all products of length $i$ of
elements from the generating set. Formally, $\mathcal{A}$ has the
following direct sum decomposition as a vector space:

$$\mathcal{A} = \bigoplus_{i=1}^\infty \mathcal{A}_i$$

and further:

$$\mathcal{A}_i\mathcal{A}_j \subseteq \mathcal{A}_{i+j}$$

For any positive integer $i$, we can define an ideal:

$$\mathcal{A}^i = \bigoplus_{j=i}^\infty \mathcal{A}_j$$

For any positive integer $c$, define the associative algebra:

$$A = \mathcal{A}/\mathcal{A}^{c+1}$$

$A$ can be described as the free nilpotent associative algebra of
class $c$ on the same generating set $S$. Explicitly, this means that
all products in $A$ of length more than $c$ become zero.

Based on the discussion in Sections \ref{sec:exp-and-adjoint}
and \ref{sec:exp-log-global}, the exponential and logarithm map are
globally defined, i.e., we have global maps:

$$\exp:A \to 1 + A, \log: 1 + A \to A$$

that are inverses of each other. Explicitly:

\begin{eqnarray*}
  \exp(x) & = & 1 + \left(x + \frac{x^2}{2!} + \dots + \frac{x^c}{c!}\right)\\
  \log(1 + x) & = & x - \frac{x^2}{2} + \dots + \frac{(-1)^{c-1}x^c}{c}\\
\end{eqnarray*}


\subsection{Free Lie algebra and free nilpotent Lie algebra}

Denote by $\mathcal{L}$ the Lie subring of $\mathcal{A}$ generated by
the free generating set $S$. Note that $\mathcal{L}$ is {\em only} a
Lie ring, {\em not} a $\mathbb{Q}$-Lie
algebra. $\mathbb{Q}\mathcal{L}$ is the $\mathbb{Q}$-Lie algebra
generated by $S$ in $\mathcal{A}$.

By \cite{Khukhro}, Theorem 5.39, $\mathcal{L}$ is the free Lie ring on
$S$, and $\mathbb{Q}\mathcal{L}$ is the free $\Q$-Lie algebra on
$S$. Further, for every prime set $\pi$,
$\mathbb{Z}[\pi^{-1}]\mathcal{L}$ is the free $\pi$-powered Lie ring
on $S$.

We can define $L$ in the following equivalent ways:

\begin{itemize}
\item $L = \mathcal{L}/\gamma_{c+1}(\mathcal{L}) =
  \mathcal{L}/(\mathcal{L} \cap \mathcal{A}^{c+1})$.
\item $L$ is the Lie subring generated by the image of $S$ inside $A$,
  i.e., it is the Lie subring generated by the freely generating set
  inside $A$.
\end{itemize}

Denote by $1 + A$ the adjoint group to $A$.

Denote by $F$ the subgroup of $1 + A$ generated by the elements
$e^{x_i}$, $x_i \in S$.

\begin{lemma}
  $1 + A$ is a rationally powered group.
\end{lemma}

\begin{proof}
  The maps $\log:1 + A \to A$ and $\exp:A \to 1 + A$ are inverses of
  each other. We know that $A$ is a $\Q$-algebra, and Lemma
  \ref{lemma:exp-and-powering} tells us that the map $\exp:A \to 1 +
  A$ preserves the cyclic subgroup structure (i.e., $\exp(nx) =
  (\exp(x))^n$ for any $n \in \Z$). Thus, $1 + A$ is also rationally
  powered.
\end{proof}
  
\begin{theorem}\label{thm:free-nilpotent}
  \begin{enumerate}
  \item For each $x_i$ in the generating set $S$, choose an element $y_i
    \in A$ that has no homogeneous degree one component. Then, the
    subgroup of $1 + A$ generated by the elements $1 + x_i + y_i$ is a
    free nilpotent group of class $c$ on the generating set $\{ 1 +
    x_i + y_i \mid i \in I \}$.
  \item The subgroup $F$ of $1 + A$ generated by the elements
    $e^{x_i}$, $x_i \in S$ is a free nilpotent group of class $c$ on
    the generating set $e^{x_i}, x_i \in S$. Thus, it is canonically
    isomorphic to the free nilpotent group of class $c$ on $S$.
  \item For any prime set $\sigma$, the subgroup $\sqrt[\sigma]{F}$ of
    $1 + A$ (where $F$ is defined as in part (2)) is a free
    $\sigma$-powered nilpotent group of class $c$ on the generating
    set $e^{x_i}, x_i \in S$.
  \end{enumerate}
\end{theorem}

\begin{proof}
  {\em Proof of (1)}: See \cite{Khukhro}, Theorem 9.2.

  {\em Proof of (2)}: This follows from (1), setting

  $$y_i = \sum_{j=2}^c \frac{x_i^j}{j!}$$

  {\em Proof of (3)}: By (2), we obtain that $\hat{F}^\sigma$ is the
  free $\sigma$-powered free class $c$ nilpotent group on the set
  $e^{x_i}, x_i \in S$. We also know that the group $1 + A$ is
  rationally powered, hence in particular it is torsion-free and
  $\sigma$-powered. Thus, $\hat{F}^\sigma$ is canonically isomorphic
  to $\sqrt[\sigma]{F}$ inside $1 + A$. This proves the result.
\end{proof}

%\newpage

\section{Baker-Campbell-Hausdorff formula}\label{sec:bch}

We use the same notation as in the preceding section (Section
\ref{sec:free-nilpotent-exp-log}).

\subsection{Introduction}

Consider the case of a generating set $\{ x_1,x_2 \}$ of size two. $A$
is the free associative algebra of class $c$ generated by
$\{ x_1,x_2 \}$. $L$ is the Lie subring of $A$ generated by the elements
$x_1$ and $x_2$. Note that $A$ is powered over all primes on account
of being a $\mathbb{Q}$-algebra, and further, every element of $A$ is
nilpotent, in fact, $u^{c+1} = 0$ for all $u \in A$. Thus, for all $u \in
A$, it makes sense to consider the element $e^u \in 1 + A$ defined as
follows:

$$e^u = 1 + \left(u + \frac{u^2}{2!} + \frac{u^3}{3!} + \dots + \frac{u^c}{c!}\right)$$

The Baker-Campbell-Hausdoroff formula for class $c$ is a formal
expression $H_c(x_1,x_2)$ with the property that:

$$e^{H_c(x_1,x_2)} = e^{x_1}e^{x_2}$$

It is not {\em a priori} obvious that $H_c(x_1,x_2)$ exists, but
\cite{Khukhro}, Section 9.2 demonstrates that $H_c(x_1,x_2)$ exists, {\em and
moreover, that $H_c(x_1,x_2) \in \mathbb{Q}L$}, i.e., it is in the
$\mathbb{Q}$-Lie subalgebra generated by $x_1$ and $x_2$.

It is easy to see that in the Baker-Campbell-Hausdorff formula for
class $c$, truncating to products of length $c - 1$ and lower gives
the Baker-Campbell-Hausdorff formula for class $c - 1$. Equivalently,
the Baker-Campbell-Hausdorff formula for class $c$ can be obtained by
adding a degree $c$ term to the Baker-Campbell-Hausdorff formula for
class $c - 1$.

We can thus define an infinite Baker-Campbell-Hausdorff formula as follows:

$$H(x_1,x_2) = t_1(x_1,x_2) + t_2(x_1,x_2) + \dots$$

where each $t_i(x_1,x_2)$ is in the $i^{th}$ homogeneous component of
$\mathbb{Q}\mathcal{L}$, with the further property that if we truncate
the summation to:

$$H_c(x_1,x_2) = t_1(x_1,x_2) + t_2(x_1,x_2) + \dots + t_c(x_1,x_2)$$

then for $A = \mathcal{A}/\mathcal{A}^{c+1}$, we have:

$$e^{H_c(x_1,x_2)} = e^{x_1}e^{x_2}$$

\subsection{Computational procedure for the Baker-Campbell-Hausdorff formula}\label{sec:bch-computation}

The following procedure can be used to compute the
Baker-Campbell-Hausdorff formula. The procedure as described here is
incomplete, because it only gives the formula inside the associative
algebra, but does not express it in terms of Lie products. There are
closed-form expressions using Lie products, but these are extremely
messy to work with, so we provide only the conceptual outline for
obtaining $H(x_1,x_2)$ as an expression in terms of $x_1$ and $x_2$ in
the associative algebra. See \cite{Khukhro}, Sections 5.3 and 9.9 for
more more details, and see \cite{Lazardeffective} for the most
efficient known computational procedure.

First, we begin by considering the product:

$$e^{x_1}e^{x_2} = \sum_{k=0}^\infty \sum_{l=0}^\infty \frac{x_1^k}{k!}\frac{x_2^l}{l!}$$

Subtract $1$ and obtain:

$$w = e^{x_1}e^{x_2} - 1 = \sum_{k,l \ge 0, 0 < k + l} \frac{x_1^kx_2^l}{k!l!}$$

We have a formal power series:

$$\log(1 + w) = w - \frac{w^2}{2} + \frac{w^3}{3} - \dots $$

It can also be formally verified that:

$$e^{\log(1 + w)} = 1 + w = 1 + (e^{x_1}e^{x_2} - 1) = e^{x_1}e^{x_2}$$

Thus, $H(x_1,x_2) = \log(1 + w)$. Formally:

$$H(x_1,x_2) = \sum_{k,l \ge 0, 0 < k + l} \frac{x_1^kx_2^l}{k!l!} - \frac{1}{2}\left(\sum_{k,l \ge 0, 0 < k + l} \frac{x_1^kx_2^l}{k!l!}\right)^2 + \frac{1}{3}\left(\sum_{k,l \ge 0, 0 < k + l} \frac{x_1^kx_2^l}{k!l!}\right)^3 - \dots$$

The above calculation gives the full Baker-Campbell-Hausdorff
formula. If we are interested in the class $c$
Baker-Campbell-Hausdorff formula, we can use truncated versions of
both the exponential and the logarithm power series.

\subsection{Homogeneous terms of the Baker-Campbell-Hausdorff formula}\label{sec:bch-homogeneous-terms}

The infinite Baker-Campbell-Hausdorff formula has the form:

$$H(x_1,x_2) = t_1(x_1,x_2) + t_2(x_1,x_2) + \dots$$

where $t_i(x_1,x_2)$ is the homogeneous component of degree $i$. The
first few homogeneous components are given below:

\begin{small}
\begin{table}[htbp]
\caption{Truncations of the Baker-Campbell-Hausdorff formula}\label{T4}
\begin{tabular}{|l|l|l|}
  \hline
  $i$ & $t_i(x_1,x_2)$ & $H_i(x_1,x_2)$\\\hline
  $1$ & $x_1 + x_2$ & $x_1 + x_2$ \\\hline
  $2$ & $\frac{1}{2}[x_1,x_2]$ & $x_1 + x_2 + \frac{1}{2}[x_1,x_2]$\\\hline
  $3$ & $\frac{1}{12}([x_1,[x_1,x_2]] - [x_2,[x_1,x_2]])$ & $x_1 + x_2 + \frac{1}{2}[x_1,x_2] + \frac{1}{12}([x_1,[x_1,x_2]] - [x_2,[x_1,x_2]])$ \\\hline
  $4$ & $- \frac{1}{24}[x_2,[x_1,[x_2,x_2]]]$ & $H_3(x_1,x_2) - \frac{1}{24}[x_2,[x_1,[x_2,x_2]]]$\\\hline
\end{tabular}
\end{table}
\end{small}

The case $i = 1$ is obvious. The case $i = 2$ is derived in the
Appendix, Section \ref{appsec:bch-class-two}. The case $i = 3$ is derived
in the Appendix, Section \ref{appsec:bch-class-three}.

For explicit descriptions of higher degree terms of the
Baker-Campbell-Hausdorff formula, see \cite{Lazardeffective}.

\subsection{Universal validity of the Baker-Campbell-Hausdorff formula}\label{sec:bch-formula-universal-validity}

The Baker-Campbell-Hausdorff formula is valid wherever it makes
sense. Explicitly, the following holds. Note that Lemma
\ref{lemma:exp-and-commutative} can be viewed as a special case of
this theorem (or rather, of Theorem
\ref{thm:bch-universal-validity-pi-powered}, the generalization to
$\Z[\pi_c^{-1}]$-algebras).

\begin{theorem}\label{thm:bch-universal-validity-rationals}
  Suppose $R$ is a nilpotent $\mathbb{Q}$-algebra. Then, the
  Baker-Campbell-Hausdorff formula is valid for any $x,y \in R$:

  $$e^{H(x,y)} = e^xe^y$$

  where we make sense of the exponentials as expressions in the
  adjoint group $1 + R$, which can be viewed as a multiplicative
  subgroup inside the unitization $\mathbb{Q} \oplus R$.
\end{theorem}

\begin{proof}
  Denote by $c$ the nilpotency class of $R$. Let $A$ be the free
  nilpotent $\mathbb{Q}$-algebra of class $c$ on two generators $x_1$
  and $x_2$. There is a unique $\Q$-algebra homomorphism $\theta:A
  \to R$ that sends $x_1$ to $x$ and $x_2$ to
  $y$. The existence of this homomorphism is guaranteed by $A$ being
  the {\em free} class $c$ nilpotent $\mathbb{Q}$-algebra on two
  generators.

  It is useful to extend $\theta$ to a homomorphism between the unitizations:

  $$\varphi: \Q \oplus A \to \Q \oplus R$$

  where $\varphi$ acts as the identity map on the first coordinate and
  acts as $\theta$ on the second coordinate. $\varphi$ is a
  $\Q$-algebra homomorphism satisfying $\varphi(x_1) = x$ and
  $\varphi(x_2) = y$.
  
  We now use the fact that the corresponding identity holds in $A$,
  and the fact that homomorphisms preserve all formulas, to obtain the
  identity for $x$ and $y$. Explicitly, we know that:

  $$e^{H_c(x_1,x_2)} = e^{x_1}e^{x_2}$$

  Apply $\varphi$ to both sides:
  
  $$\varphi(e^{H_c(x_1,x_2)}) = \varphi(e^{x_1}e^{x_2})$$

  $\Q$-algebra homomorphisms commute with exponentiation and with
  products, so this becomes:
  
  $$e^{\varphi(H_c(x_1,x_2))} = e^{\varphi(x_1)}e^{\varphi(x_2)}$$
  
  $\Q$-algebra homomorphism also commute with $H_c$, and we get:
  
  $$e^{H_c(\varphi(x_1),\varphi(x_2))} = e^{\varphi(x_1)}e^{\varphi(x_2)}$$
  
  We had set $\varphi(x_1) = x$ and $\varphi(x_2) = y$, so we get:
  
  $$e^{H_c(x,y)} = e^xe^y$$

  as desired.
\end{proof}

\subsection{Formal properties of the Baker-Campbell-Hausdorff formula}\label{sec:bch-formal-properties}

Below are some important properties of the Baker-Campbell-Hausdorff
formula:

\begin{enumerate}
\item The $i^{th}$ homogeneous component $t_i(x_1,x_2)$ of the
  Baker-Campbell-Hausdorff formula is symmetric if $i$ is odd and
  skew-symmetric if $i$ is even. Explicitly:

  $$t_i(x_1,x_2) = (-1)^{i-1}t_i(x_2,x_1)$$

\item The Baker-Campbell-Hausdorff formula is associative, i.e., we
  have the following formal identity:

  $$H(H(x_1,x_2),x_3) = H(x_1,H(x_2,x_3))$$

  Equivalently, for every positive integer $c$, the following identity
  holds in class $c$:

  $$H_c(H_c(x_1,x_2),x_3) = H_c(x_1,H_c(x_2,x_3))$$

  Note that by ``holds in class $c$'' we mean that we {\em truncate}
  the formula to all Lie products of degree at most $c$, and set all
  higher degree Lie products to be zero.
\item The Baker-Campbell-Hausdorff formula satisfies:

  $$H(x,0) = H(0,x) = x$$

  Equivalently, for every positive integer $c$, we have:

  $$H_c(x,0) = H_c(0,x) = x$$

  Note that in this case, explicit truncation to class $c$ is not
  necessary.

\item The Baker-Campbell-Hausdorff formula satisfies:

  $$H(x,-x) = H(-x,x) = 0$$

  Equivalently, for every positive integer $c$, we have:

  $$H_c(x,-x) = H_c(-x,x) = 0$$

  Note that in this case, explicit truncation to class $c$ is not
  necessary.
\end{enumerate}

\subsection{Universal validity of the formal properties of the Baker-Campbell-Hausdorff formula}\label{sec:bch-formula-universal-lie-validity}

Note that the universal validity being alluded to here differs in
spirit from the universal validity that was alluded to in Section
\ref{sec:bch-formula-universal-validity}. The universal validity
alluded to earlier was the universal validity of the
Baker-Campbell-Hausdorff formula in relation with the exponential map
in an {\em associative} algebra. The universal validity alluded to
here is in {\em Lie} algebras.

\begin{theorem}\label{thm:bch-group-axiom-universal-validity}
  Suppose $N$ is a nilpotent $\mathbb{Q}$-Lie algebra of
  nilpotency class $c$ for some positive integer $c$. Then, the
  following hold for all $x,y,z \in N$:
    
  \begin{eqnarray*}
    H_c(H_c(x,y),z) & = & H_c(x,H_c(y,z)) \\
    H_c(x,0) & = & x \\
    H_c(0,x) & = & x \\
    H_c(x,-x) & = & 0 \\
    H_c(-x,x) & = & 0 \\
  \end{eqnarray*}
\end{theorem}

\begin{proof}
  All identities except the first are obvious from the expressions. We
  therefore concentrate on the first identity.

  Denote by $L$ the free Lie ring on the set $\{x_1,x_2,x_3\}$. Then,
  $\Q L$ is the free $\Q$-Lie algebra on $\{x_1,x_2,x_3\}$. Consider
  the unique $\Q$-algebra homomorphism $\varphi: \mathbb{Q}L \to N$
  defined as follows: $\varphi(x_1) = x$, $\varphi(x_2) = y$, and
  $\varphi(x_3) = z$. 

  We know that $H_c(H_c(x_1,x_2),x_3) = H_c(x_1,H_c(x_2,x_3))$ by the
  formal associativity of the Baker-Campbell-Hausdorff formula. The
  homomorphism $\varphi$ is a homomorphism of $\mathbb{Q}$-Lie
  algebras, hence it preserves the identity, and we obtain that:

  $$H_c(H_c(x,y),z) = H_c(x,H_c(y,z))$$

\end{proof}

\subsection{Primes in the denominator for the Baker-Campbell-Hausdorff formula}\label{sec:bch-primes-in-denominators}

The lemma below allows us to restrict the Baker-Campbell-Hausdorff
formula to $\mathbb{Z}[\pi_c^{-1}]L$ where $\pi_c$ is the set of primes
less than or equal to $c$.

\begin{lemma}\label{lemma:bch-primes-in-denominators}
  It is possible to express the Baker-Campbell-Hausdorff formula
  $H_c(x,y)$ in a manner where all primes that appear as divisors of
  denominators of the coefficients are less than or equal
  to $c$.
\end{lemma}

\begin{proof}
  The Baker-Campbell-Hausdorff formula is obtained from the class $c$
  exponential and logarithm formulas using the operations of addition,
  subtraction, multiplication, and composition. The class $c$
  exponential and logarithm formulas use only the primes less than or
  equal to $c$, and the operations of addition, subtraction,
  multiplication and composition cannot introduce new prime divisors
  into the denominators, so the Baker-Campbell-Hausdorff formula does
  not have any other primes in its denominator.

  Note that the above only demonstrates the result for the {\em
    associative} expression for the Baker-Campbell-Hausdorff
  formula. However, \cite{Khukhro}, Theorem 5.39\footnote{It would be
    helpful to read the surrounding discussion in Section 5.3}
  demonstrates that we can rewrite the expression as a sum of basic
  Lie products in a manner that does not use any new prime divisors
  in the denominator.
\end{proof}

A somewhat stronger result is true: Suppose $p$ is a prime. Then, the
largest $k$ such that $p^k$ divides one or more of the denominators in
the coefficients for $H_c(x_1,x_2)$ is at most $\left \lfloor \frac{c
  - 1}{p - 1}\right \rfloor$. This is derived in the Appendix, Section
\ref{appsec:bch-prime-power-divisor-bound}. Note that this would
immediately imply the preceding lemma, but it has additional
significance.

\subsection{Universal validity assuming powering over the required primes}

Below, we present results analogous to those described in Sections
\ref{sec:bch-formula-universal-validity} and
\ref{sec:bch-formula-universal-lie-validity}, but with the base ring
taken to be $\Z[\pi_c^{-1}]$ instead of $\Q$, where $\pi_c$ is the set of
primes less than or equal to $c$.

\begin{theorem}\label{thm:bch-universal-validity-pi-powered}
  Suppose $c$ is a positive integer, $\pi_c$ is the set of all primes
  less than or equal to $c$, and $R$ is an associative
  $\mathbb{Z}[\pi^{-1}]$-algebra that is nilpotent of nilpotency class
  at most $c$. Then, the Baker-Campbell-Hausdorff formula is valid
  for any $x,y \in R$:
    
  $$e^{H_c(x,y)} = e^xe^y$$
    
  where we make sense of the exponentials as expressions in the
  adjoint group $1 + R$, which can be viewed as a multiplicative
  subgroup inside the unitization $\mathbb{Z}[\pi^{-1}] \oplus R$.
\end{theorem}

\begin{proof}
  Let $A$ be the free nilpotent associative $\Q$-algebra of class $c$
  on the two generators $x_1$ and $x_2$ and let $B$ be the
  $\mathbb{Z}[\pi^{-1}]$-subalgebra generated by $x_1$ and
  $x_2$. Clearly, $B$ is the free $\Z[\pi^{-1}]$-algebra on $x_1$ and
  $x_2$. There is a natural homomorphism $\theta: B \to R$ of
  $\mathbb{Z}[\pi^{-1}]$-algebras that sends $x_1$ to $x$ and $x_2$ to
  $y$. The existence of this homomorphism is guaranteed by $B$ being
  the {\em free} class $c$ nilpotent $\mathbb{Z}[\pi^{-1}]$-algebra on
  two generators. Further, $\theta$ extends uniquely to a homomorphism
  $\varphi: \Z[\pi^{-1}] \oplus B \to \Z[\pi^{-1}] \oplus R$ between
  the unitizations of $B$ and $R$ as $\Z[\pi^{-1}]$-algebras.

  Note that the Baker-Campbell-Hausdorff formula is valid for the
  elements $x_1,x_2 \in A$, i.e., we have:

  $$e^{H_c(x_1,x_2)} = e^{x_1}e^{x_2}$$

  Both sides of the identity and all intermediate calculations happen
  inside $B$, because the exponential map as well as the
  Baker-Campbell-Hausdorff formula all involve division only by the
  primes in $\pi_c$ (by Lemma
  \ref{lemma:bch-primes-in-denominators}). Thus, the above identity
  holds in $B$.

  We now apply $\varphi$ to both sides and obtain the conclusion. The
  details are analogous to those of Theorem
  \ref{thm:bch-universal-validity-rationals}. The main difference is
  that the homomorphism $\varphi$ is now a $\Z[\pi^{-1}]$-algebra
  homomorphism rather than a $\Q$-algebra homomorphism. The reason it
  commutes with the formulas is that all the formulas involved make
  sense over $\Z[\pi^{-1}]$.
\end{proof}

\begin{theorem}\label{thm:bch-group-axiom-universal-validity-pi-powered}
  Suppose $c$ is a positive integer and $\pi_c$ is the set of all primes
  less than or equal to $c$. Suppose $N$ is a nilpotent
  $\mathbb{Z}[\pi^{-1}]$-Lie algebra of nilpotency class $c$. Then, the
  following hold for all $x,y,z \in N$:
    
  \begin{eqnarray*}
    H_c(H_c(x,y),z) & = & H_c(x,H_c(y,z)) \\
    H_c(x,0) & = & x \\
    H_c((0,x) & = & x\\
    H_c(x,-x) & = & 0 \\
    H_c(-x,x) & = & 0 \\
  \end{eqnarray*}

\end{theorem}

\begin{proof}
  All identities except the first one are obvious from the
  expressions. We therefore concentrate on proving the first identity.

  Denote by $L$ the free Lie ring on the set $\{
  x_1,x_2,x_3 \}$. Then, $\Z[\pi^{-1}]L$ is the free
  $\Z[\pi^{-1}]$-Lie algebra on the set $\{ x_1,x_2,x_3 \}$. Consider
  the unique $\Z[\pi^{-1}]$-algebra homomorphism $\varphi:
  \mathbb{Z}[\pi^{-1}]L \to N$ defined by the conditions $\varphi(x_1)
  = x$, $\varphi(x_2) = y$, $\varphi(x_3) = z$.

  We know that $H_c(H_c(x_1,x_2),x_3) = H_c(x_1,H_c(x_2,x_3))$ by the
  formal associativity of the Baker-Campbell-Hausdorff formula. Lemma
  \ref{lemma:bch-primes-in-denominators} tells us that this identity
  holds inside $\Z[\pi^{-1}]L$. The homomorphism $\varphi:
  \Z[\pi^{-1}]L \to N$ is a homomorphism of $\Z[\pi^{-1}]$-algebras,
  hence it preserves the identity, and we obtain that:

  $$H_c(H_c(x,y),z) = H_c(x,H_c(y,z))$$
\end{proof}


%\newpage

\section{The inverse Baker-Campbell-Hausdorff formulas}\label{sec:inverse-bch}

\subsection{Brief description of the formulas}

There are two inverse Baker-Campbell-Hausdorff formulas. The first
inverse Baker-Campbell-Hausdorff formula is a formula to compute:

$$h_1(x,y) = \exp(\log x + \log y)$$

The second inverse Baker-Campbell-Hausdorff formula is a formula to
compute:

$$h_2(x,y) =\exp([\log x, \log y])$$

We will now provide a few important details regarding the formulas
that will help understand the correspondence. However, we do not
attempt to be comprehensive here. More information about the inverse
Baker-Campbell-Hausdorff formula is in \cite{Khukhro}, Section
10.1. Lemma 10.7 in particular establishes the key nature of the
formula. For an explicit description of the first few terms for both
$h_1$ and $h_2$, as well as an efficient computation strategy, see
\cite{Lazardeffective}.


\subsection{Origin of the formulas}\label{sec:inverse-bch-origins}

We follow the notation of Section
\ref{sec:free-nilpotent-exp-log}. The highlights of the notation
follow: $\mathcal{A}$ is the free associative $\Q$-algebra on a
generating set $S = \{ x_1,x_2,\dots \}.$ $\mathcal{L}$ is the Lie
subring (not the $\Q$-Lie subalgebra, but just the $\Z$-Lie
subalgebra) of $\mathcal{A}$ generated by $S$. $c$ is a fixed positive
integer. We denote by $\mathcal{A}_c$ the $c^{th}$ graded component in
the natural gradation of $\mathcal{A}$, and we denote by
$\mathcal{A}^c$ the sum of all graded components at and beyond
$c$. Define $A = \mathcal{A}/\mathcal{A}^{c+1}$ and define $L$ to be
the Lie subring of $A$ generated by $S$.\footnote{Note that the $S$
  viewed as a subset inside $A$ is the image of the $S$ viewed as a
  subset of $\mathcal{A}$.} By Theorem \ref{thm:free-nilpotent}, the
set $\{ e^{x_i} \mid x_i \in S \}$ generates a free nilpotent group of
class $c$ and is a freely generating set for it. We denote this group
as $F$.

Corollary 9.22 in Khukhro's book (\cite{Khukhro}) states that
$\sqrt{F}$ is a free rationally powered nilpotent group of nilpotency
class $c$. This is a consequence of Theorem
\ref{thm:pi-powered-envelope}. Further, Theorem 10.4 of \cite{Khukhro}
states that $\sqrt{F} = e^{\Q L}$. Thus, every element of $e^{\Q L}$
can be expressed in terms of the elements $e^{x_i}$ using group
operations as well as taking roots.

The first inverse Baker-Campbell-Hausdorff formula expresses $e^{x_1 +
  x_2}$ as a word in terms of $e^{x_1}$ and $e^{x_2}$ using the group
operations and taking roots. Explicitly, the first inverse
Baker-Campbell-Hausdorff formula in class $c$ is a formula $h_{1,c}$
such that:

$$e^{x_1 + x_2} = h_{1,c}(e^{x_1},e^{x_2})$$

Note importantly that the element $e^{x_1+x_2}$ need not lie inside
$F$, but it does lie inside $\sqrt{F}$.

Similarly, the second inverse Baker-Campbell-Hausdorff formula
expresses $e^{[x_1,x_2]}$ as a word in terms of $e^{x_1}$ and
$e^{x_2}$ using the group operations and taking roots. The second
inverse Baker-Campbell-Hausdorff formula in class $c$ is a formula
$h_{2,c}$ such that:

$$e^{[x_1,x_2]} = h_{2,c}(e^{x_1},e^{x_2})$$

Here, $[x_1,x_2]$ denotes the Lie bracket of $x_1$ and $x_2$ inside
$L$. Inside $A$, this can be viewed as the expression $x_1x_2 -
x_2x_1$. However, the latter does not make sense as an expression
inside $L$.

\subsection{Formal properties of the inverse Baker-Campbell-Hausdorff formulas}\label{sec:inverse-bch-formal-properties}

\begin{itemize}
\item As with the original Baker-Campbell-Hausdorff formula, both the
  inverse \\ Baker-Campbell-Hausdorff formulas can be truncated to class
  $c$ for any positive integer $c$. We will denote the truncations as
  $h_{1,c}$ and $h_{2,c}$ respectively.
\item The class $c + 1$ inverse Baker-Campbell-Hausdorff formula is
  obtained by taking the class $c$ inverse Baker-Campbell-Hausdorff
  formula and multiplying by a suitable product of iterated
  commutators, each of length $c + 1$. Note that the precise nature of
  the terms depends on whether we multiply on the left or the
  right.\footnote{Technically, the new term that needs to be inserted
    is the same whether on the left or on the right, because it is
    central and therefore commutes with the rest of the
    expression. However, the choice of whether previous terms were
    inserted on the left or the right affects the precise choice of
    the term being inserted at a given stage, so in that sense, it
    does matter whether the higher degree terms are being inserted on
    the left or the right.}
\item For $h_1(x_1,x_2)$, the degree one term is $x_1x_2$ and the
  degree two term is $[x_1,x_2]^{-1/2}$, the reciprocal of the square
  root of the group commutator.
\item For $h_2(x_1,x_2)$, there is no degree one term, and the degree
  two term is the group commutator $[x_1,x_2]$.
\end{itemize}

\subsection{Universal validity of Lie ring axioms for inverse Baker-Campbell-Hausdorff formulas}

We begin by establishing the universal validity over rationally
powered nilpotent groups.

\begin{theorem}\label{thm:inverse-bch-lie-ring-axiom-universal-validity}
  Suppose $G$ is a rationally powered nilpotent group of nilpotency
  class $c$. Then, the following are true for all $x,y,z \in G$:

  \begin{small}
  \begin{eqnarray*}
    h_{1,c}(h_{1,c}(x,y),z) & = & h_{1,c}(x,h_{1,c}(y,z)) \\
    h_{1,c}(x,1) & = & x \\
    h_{1,c}(1,x) & = & x \\
    h_{1,c}(x,x^{-1}) & = & 1 \\
    h_{1,c}(x^{-1},x) & = & 1 \\
    h_{1,c}(x,y) & = & h_{1,c}(y,x)\\
    h_{2,c}(x,h_{1,c}(y,z)) & = & h_{1,c}(h_{2,c}(x,y),h_{2,c}(x,z))\\
    h_{2,c}(h_{1,c}(x,y),z) & = & h_{1,c}(h_{2,c}(x,z),h_{2,c}(y,z))\\
    h_{2,c}(x,x) & = & 1 \\
    h_{1,c}(h_{1,c}(h_{2,c}(h_{2,c}(x,y),z),h_{2,c}(h_{2,c}(y,z),x)),h_{2,c}(h_{2,c}(z,x),y)) & = & 1 \\
  \end{eqnarray*}
  \end{small}
\end{theorem}

\begin{proof}
  Let $S = \{x_1,x_2,x_3\}$ and use the setup of Section
  \ref{sec:inverse-bch-origins}. $F$ is therefore a free nilpotent
  group of class $c$ with freely generating set comprising $e^{x_1}$,
  $e^{x_2}$, $e^{x_3}$. There is therefore a unique group homomorphism
  from $F$ to $G$ sending $e^{x_1}$ to $x$, $e^{x_2}$ to $y$, and
  $e^{x_3}$ to $z$. By Theorem \ref{thm:pi-powered-envelope} and the
  fact that $G$ is rationally powered, this extends to a unique group
  homomorphism from $\sqrt{F}$ to $G$. All the above identites hold
  for $\sqrt{F}$ because of the Lie ring structure on $L$. The
  identities are preserved under homomorphisms, therefore they also
  hold in $G$ (the proof details are similar to those for Theorems
  \ref{thm:bch-universal-validity-rationals} and
  \ref{thm:bch-group-axiom-universal-validity}).
\end{proof}

\subsection{Universal validity of Lie ring axioms: the $\pi_c$-powered case}\label{sec:inverse-bch-pi-powered}

Denote by $\pi_c$ the set of all primes less than or equal to
$c$. Theorem 10.22 of Khukhro's book \cite{Khukhro} states that for
any prime set $\sigma \supseteq \pi_c$, we have $\sqrt[\sigma]{F} =
e^{\Z[\sigma^{-1}]L}$. In particular, this means that $\sqrt[\pi_c]{F} =
e^{\Z[\pi_c^{-1}]L}$. Therefore, both the formulas $h_{1,c}$ and
$h_{2,c}$ (which describe elements of $e^L$ and hence elements of
$e^{\Z[\pi_c^{-1}]L}$) can be written using $p^{th}$ roots only for the
primes $p$ that are in $\pi_c$.

We will later see, in Lemma \ref{lemma:lie-bracket-denominators}, that
a slightly stricter bound applies to $h_{2,c}$, though the bound here
is tight for $h_{1,c}$. However, we do not need the stricter bound
for our present purpose.

Based on this, we can formulate a $\pi_c$-powered version of the preceding theorem.

\begin{theorem}\label{thm:inverse-bch-lie-ring-axiom-universal-validity-pi-powered}
  Suppose $G$ is a $\pi_c$-powered nilpotent group of nilpotency class
  $c$. Then, the following are true for all $x,y,z \in G$:
    
  \begin{eqnarray*}
    h_{1,c}(h_{1,c}(x,y),z) & = & h_{1,c}(x,h_{1,c}(y,z)) \\
    h_{1,c}(x,1) & = & x \\
    h_{1,c}(1,x) & = & x \\
    h_{1,c}(x,x^{-1}) & = & 1 \\
    h_{1,c}(x^{-1},x) & = & 1 \\
    h_{1,c}(x,y) & = & h_{1,c}(y,x)\\
    h_{2,c}(x,h_{1,c}(y,z)) & = & h_{1,c}(h_{2,c}(x,y),h_{2,c}(x,z))\\
    h_{2,c}(h_{1,c}(x,y),z) & = & h_{1,c}(h_{2,c}(x,z),h_{2,c}(y,z))\\
    h_{2,c}(x,x) & = & 1 \\
    h_{1,c}(h_{1,c}(h_{2,c}(h_{2,c}(x,y),z),h_{2,c}(h_{2,c}(y,z),x)),h_{2,c}(h_{2,c}(z,x),y)) & = & 1 \\
  \end{eqnarray*}
\end{theorem}

\begin{proof}
  Let $S = \{x_1,x_2,x_3\}$ and use the setup of Section
  \ref{sec:inverse-bch-origins}. $F$ is therefore a free nilpotent
  group of class $c$ with freely generating set comprising $e^{x_1}$,
  $e^{x_2}$, $e^{x_3}$. There is therefore a unique group homomorphism
  from $F$ to $G$ sending $e^{x_1}$ to $x$, $e^{x_2}$ to $y$, and
  $e^{x_3}$ to $z$. By Theorem \ref{thm:pi-powered-envelope} and the
  fact that $G$ is $\pi_c$-powered, this extends to a unique group
  homomorphism from $\sqrt[\pi_c]{F}$ to $G$. All the above identities
  hold inside $\sqrt[\pi_c]{F}$, therefore, they also hold in $G$.
\end{proof}

%\newpage

\section{The Malcev correspondence}\label{sec:malcev-correspondence}

\subsection{The class $c$ Malcev correspondence}

The class $c$ Malcev correspondence is a correspondence:

Rationally powered nilpotent groups of nilpotency class at most $c$
$\leftrightarrow$ Rationally powered nilpotent Lie rings of nilpotency
class at most $c$ (i.e., nilpotent $\mathbb{Q}$-Lie algebras)

The class $c$ Malcev correspondence has a number of features similar
to the abelian Lie correspondence (described in Section
\ref{sec:abelian-lie-correspondence}) and the Baer correspondence
(described in Sections \ref{sec:baer-correspondence-basics} and
\ref{sec:baer-correspondence-more}). To avoid repetition, we refer to
relevant sections for the earlier correspondences where helpful.

Given a rationally powered nilpotent Lie ring $L$ of nilpotency class
at most $c$, the {\em Malcev Lie group} for $L$, denoted $\exp(L)$, is
a rationally powered nilpotent group of nilpotency class at most $c$
defined as follows:

\begin{itemize}
\item The underlying set of $\exp(L)$ is the same as the underlying set of
  $L$.
\item The group operation of $\exp(L)$ is defined as follows: $xy =
  H_c(x,y)$ where $H_c$ is the class $c$ Baker-Campbell-Hausdorff
  formula, and the operations in the formula are interpreted over $L$.
\item The identity element $1 \in \exp(L)$ of the group is the same as the
  zero element $0 \in L$.
\item The inverse map is defined as $x^{-1} := -x$.
\end{itemize}

Conversely, given a rationally powered nilpotent group $G$ of
nilpotency class at most $c$, the {\em Malcev Lie ring} for $G$,
denoted $\log(G)$, is a rationally powered nilpotent Lie ring of
nilpotency class at most $c$ defined as follows:

\begin{itemize}
\item The underlying set of $\log(G)$ is the same as the underlying
  set of $G$.
\item The Lie ring addition is defined using the first inverse
  Baker-Campbell-Hausdorff formula in terms of the group
  operations. Explicitly, we define $x + y := h_{1,c}(x,y)$, where
  $h_{1,c}$ is the first inverse Baker-Campbell-Hausdorff formula for
  class $c$, and the operations in the formula are interpreted over
  $G$.
\item The Lie bracket is defined using the second
  inverse-Baker-Campbell-Hausdorff formula in terms of the group
  operations. Explicitly, we define $[x,y] := h_{2,c}(x,y)$, where
  $h_{2,c}$ is the inverse Baker-Campbell-Hausdorff formula for class
  $c$, and the operations in the formula are interpreted over $G$.
\item The zero element $0 \in \log(G)$ is defined to be the identity
  element $1 \in G$.
\item The negation map in $\log(G)$ is defined to be the same as the
  inverse map in the group, i.e., $-x := x^{-1}$.
\end{itemize}

\subsection{Why the class $c$ Malcev correspondence works}

The class $c$ Malcev correspondence works in the direction from Lie
rings to groups due to Theorem
\ref{thm:bch-group-axiom-universal-validity}. It works in the reverse
direction due to Theorem
\ref{thm:inverse-bch-lie-ring-axiom-universal-validity}. The fact that
the two directions are inverses of each other (i.e., applying the
correspondence in one direction and then in the other direction
returns to the original object) also follows from the original setup
of the formulas.

\subsection{Description of the Malcev correspondence (combining all possibilities for nilpotency class)}

The class $c$ Malcev correspondences for different values of $c$ are
related in the following manner. Suppose $c_1 \le c_2$. Then, the
class $c_1$ Malcev correspondence is a subcorrespondence of the class
$c_2$ Malcev correspondence. Explicitly, this means that the
subcategory (on the group side and the Lie ring side respectively) for
the class $c_1$ Malcev correspondence is a full subcategory of the
subcategory (on the group and the Lie ring side respectively) for the
class $c_2$ Malcev correspondence. Further, the $\exp$ and $\log$
functors for the class $c_1$ Malcev correspondence are obtained by
restricting to that subcategory the $\exp$ and $\log$ functors for the
class $c_2$ Malcev correspondence.

Thus, as we increase the value of $c$, both sides of the
correspondence become larger. We can thus consider a combined
correspondence for all classes. This is the {\em Malcev
  correspondence}. The Malcev correspondence is a correspondence:

\begin{center}
  Rationally powered nilpotent groups $\leftrightarrow$ Rationally
  powered nilpotent Lie rings
\end{center}
\subsection{The Malcev correspondence as an isomorphism of categories over Set}

We can use reasoning analogous to that used for the Baer
correspondence in Section
\ref{sec:baer-correspondence-homomorphism-preservation} to deduce that
the Malcev correspondence defines an isomorphism of categories over
the category of sets between the following two categories: the
category of rationally powered nilpotent groups and the category of
rationally powered nilpotent Lie rings. We can also deduce some
immediate consequences similar to those deduced for the Baer
correspondence in Section
\ref{sec:baer-correspondence-isocat-consequences}.

%\newpage

\section{The global Lazard correspondence}\label{sec:global-lazard-correspondence}

\subsection*{Remark about the ``global'' and ``3-local'' terminology}

Although the terms ``Lazard Lie group'' and ``Lazard Lie ring'' are
reasonably standard, the prefix adjectives ``global'' and ``3-local''
are not standard. Some sources, such as Khukhro's book \cite{Khukhro},
use the term ``Lazard correspondence'' only for the global Lazard
correspondence, and do not describe the $3$-local Lazard
correspondence. Others, including Lazard's original paper
\cite{Lazardsoriginal}, describe the $3$-local Lazard correspondence.

We will separate the two ideas because the statements and proofs are
easier to understand in the global class case, and due to
considerations of space and complexity, we provide complete proofs
only for the global class case. We begin by understanding the global
class case of the Lazard correspondence.

\subsection{Definitions of global class Lazard Lie group and global class Lazard Lie ring}

\begin{definer}[Global class $c$ Lazard Lie group]
  Suppose $G$ is a nilpotent group and $c$ is a positive
  integer. Denote by $\pi_c$ the set of all primes less than or equal
  to $c$. We say that $G$ is a {\em global class $c$ Lazard Lie group}
  if the following two conditions are satisfied:

  \begin{itemize}
  \item The nilpotency class of $G$ is at most equal to $c$.
  \item $G$ is $\pi_c$-powered, i.e., $G$ is powered over all primes
    less than or equal to $c$.
  \end{itemize}

  We say that $G$ is a global Lazard Lie group if $G$ is a global
  class $c$ Lazard Lie group for some positive integer
  $c$. Equivalently, $G$ is a global Lazard Lie group if it is powered
  over all primes less than or equal to its nilpotency class.
\end{definer}

\begin{definer}[Global class $c$ Lazard Lie ring]
  Suppose $L$ is a nilpotent Lie ring and $c$ is a positive
  integer. Denote by $\pi_c$ the set of all primes less than or equal
  to $c$. We say that $L$ is a {\em global class $c$ Lazard Lie ring}
  if the following two conditions are satisfied:

  \begin{enumerate}
  \item The nilpotency class of $L$ is at most equal to $c$.
  \item $L$ is $\pi_c$-powered, i.e., $L$ is powered over all primes
    less than or equal to $c$.
  \end{enumerate}

  We say that $L$ is a global Lazard Lie ring if $L$ is a global class
  $c$ Lazard Lie ring for some positive integer $c$. Equivalently, $L$
  is a global Lazard Lie ring if $L$ is powered over all primes less
  than or equal to its nilpotency class.
\end{definer}

\subsection{Possibilities for the set of values of $c$ for a global class $c$ Lazard Lie group}\label{sec:global-class-c-multiple-c-values}

We re-examine the two conditions for a group $G$ to be a ``global
class $c$ Lazard Lie group'':

\begin{enumerate}
\item The nilpotency class of $G$ is at most equal to $c$.
\item $G$ is powered over all primes less than or equal to $c$.
\end{enumerate}

Condition (1) becomes weaker as we increase $c$. Condition (2), on the
other hand, becomes {\em stronger} as we increase $c$. Overall,
therefore, being a global class $c$ Lazard Lie group is neither
stronger nor weaker than being a global class $(c + 1)$ Lazard Lie
group.

For instance, an abelian group with $2$-torsion is a global class $1$
Lazard Lie group, but not a global class $2$ Lazard Lie group. On the
other hand, a finite non-abelian $p$-group of class two for odd $p$
(such as the unitriangular group $UT(3,p)$, which is a non-abelian
group of order $p^3$ and exponent $p$) is a global class $2$ Lazard
Lie group, but not a global class $1$ Lazard Lie group. Thus, neither
condition implies the other.

We now turn to what we {\em can} say about the set of possible values
$c$ for which a given group is a global class $c$ Lazard Lie group.

Suppose $G$ is a nilpotent group. Denote by $c_0$ the nilpotency class
of $G$. Let $p_0$ be the smallest prime such that $G$ is {\em not}
powered over $p_0$, if there exists such a prime ($G$ is rationally
powered if and only if no such $p_0$ exists). There are four
possibilities:

\begin{itemize}
\item $c_0 \ge p_0$: In this case, $G$ is {\em not} a global class $c$
  Lazard Lie group for any value of $c$.
\item $c_0 = p_0 - 1$: In this case, $G$ is a global class $c_0$
  Lazard Lie group, but is not a global class $c$ Lazard Lie group for
  any other value of $c$.
\item $c_0 < p_0 - 1$: In this case, $G$ is a global class $c$ Lazard
  Lie group for all $c$ satisfying $c_0 \le c \le p_0 - 1$.
\item $G$ is rationally powered: In this case, $G$ is a global class
  $c$ Lazard Lie group for all $c$ satisfying $c_0 \le c$.
\end{itemize}

The upshot is that the set of values $c$ for which $G$ is a global
class $c$ Lazard Lie group is either empty or a single value or a
contiguous (possibly finite and possibly infinite) subsegment of the
nonnegative integers.

Analogous remarks apply for the case of nilpotent Lie rings.

\subsection{The global class $c$ Lazard correspondence}

The global class $c$ Lazard correspondence is a correspondence:

\begin{center}
  Global class $c$ Lazard Lie groups $\leftrightarrow$ Global class $c$
  Lazard Lie rings
\end{center}
The global class $c$ Lazard correspondence operates in a manner quite
similar to the class $c$ Malcev correspondence. We describe it
explicitly below.

Given a global class $c$ Lazard Lie ring $L$, the corresponding global
class $c$ Lazard Lie ring $\exp(L)$ is defined as follows:

\begin{itemize}
\item The underlying set of $\exp(L)$ is the same as the underlying set of
  $L$.
\item The group operation of $\exp(L)$ is defined as follows: $xy =
  H_c(x,y)$ where $H_c$ is the class $c$ Baker-Campbell-Hausdorff
  formula, and the Lie brackets in the formula are interpreted as Lie
  brackets in $L$. The reason this makes sense is that we have assumed
  that $L$ powered over all primes less than or equal to $c$, and by
  Lemma \ref{lemma:bch-primes-in-denominators}, these are the only
  primes that appear as divisors of the denominators in $H_c(x,y)$.
\item The identity element $1 \in \exp(L)$ of the group is the same as the
  zero element $0 \in L$.
\item The inverse map is defined as $x^{-1} := -x$.
\end{itemize}

In the reverse direction, given a global class $c$ Lazard Lie group
$G$ of nilpotency class $c$, the {\em Lazard Lie ring} for $G$ is a
rationally powered nilpotent Lie ring $\log(G)$ of nilpotency class $c$
defined as follows:

\begin{itemize}
\item The underlying set of $\log(G)$ is the same as the underlying set of
  $G$.
\item The Lie ring addition is defined using the first inverse
  Baker-Campbell-Hausdorff formula $h_{1,c}$ in terms of the group
  operations. Note that, as explained in Section
  \ref{sec:inverse-bch-pi-powered}, the formula for $h_{1,c}$ makes
  sense in a $\pi_c$-powered nilpotent group of nilpotency class at most
  $c$.
\item The Lie bracket is defined using the second
  inverse-Baker-Campbell-Hausdorff formula in terms of the group
  operations. Note that, as explained in Section
  \ref{sec:inverse-bch-pi-powered}, the formula for $h_{2,c}$ makes
  sense in a $\pi_c$-powered nilpotent group of nilpotency class at most
  $c$.
\item The zero element of $\log(G)$ is defined to be the identity
  element of $G$.
\item The negation map in the Lie ring is defined to be the same as
  the inverse map in the group, i.e., $-x := x^{-1}$.
\end{itemize}

The forward direction (from Lie rings to groups) works because of
Theorem \ref{thm:bch-group-axiom-universal-validity-pi-powered}. The
reverse direction works due to Theorem
\ref{thm:inverse-bch-lie-ring-axiom-universal-validity-pi-powered}. The
directions are reverses of each other due to the formal properties of
the formulas.

\subsection{The global class one Lazard correspondence is the abelian Lie correspondence}

The global class one Lazard correspondence is the abelian Lie
correspondence described in Section
\ref{sec:abelian-lie-correspondence}:

\begin{center}
  Abelian groups $\leftrightarrow$ Abelian Lie rings
\end{center}
\subsection{The global class two Lazard correspondence is the Baer correspondence}

Recall the definition of Baer Lie group from Section
\ref{sec:baer-lie-definitions}: a $2$-powered group of nilpotency
class at most two. This agrees with the definition of a global class
$2$ Lazard Lie group. Similarly, the definition of Baer Lie ring in
Section \ref{sec:baer-lie-definitions} agrees with the definition of a
global class $2$ Lazard Lie ring.

The class two Baker-Campbell-Hausdorff formula, stated in Section
\ref{sec:bch-homogeneous-terms} and worked out in the Appendix,
Section \ref{appsec:bch-class-two}, is:

$$H_2(x,y) = x + y + \frac{1}{2}[x,y]$$

This is precisely the same as the formula in the direction from Lie
rings to groups in the Baer correspondence. Similarly, the class two
inverse Baker-Campbell-Hausdorff formulas are:

$$x + y = \frac{xy}{\sqrt{[x,y]}}$$

$$[x,y]_{\text{Lie}} = [x,y]_{\text{Group}}$$

These are precisely the same as the formulas in the direction from
groups to Lie rings in the Baer correspondence.

Thus, the Baer correspondence is the same as the global class two
Lazard correspondence.

\subsection{Interaction of global Lazard correspondences for different classes}

In Section \ref{sec:global-class-c-multiple-c-values}, we noted that a
given global Lazard Lie group may be a global class $c$ Lazard Lie
group for multiple values of $c$. The analogous observation applies to
Lie rings. This leads to potential for ambiguity regarding which
global class $c$ Lazard correspondence we are referring to.

It turns out that the global class $c$ Lazard correspondences for
different values of $c$ agree with each other wherever both are
applicable. Explicitly, the following are true:

\begin{itemize}
\item Suppose $G$ is a group of nilpotency class exactly $c_0$, and
  $p_0$ is the smallest prime for which $G$ is not powered (see below
  for the case that $G$ is rationally powered). Suppose that $c_0 \le
  p_0 - 1$. Then, for each $c$ satisfying $c_0 \le c \le p_0 - 1$, $G$
  is a global class $c$ Lazard Lie group. Therefore, for each such
  $c$, we can consider a definition of $\log(G)$ based on the global
  class $c$ Lazard correspondence. All these definitions coincide.

  In the case that $G$ is rationally powered and has class exactly
  $c_0$, we can consider a definition of $\log(G)$ based on the global
  class $c$ Lazard correspondence for all $c \ge c_0$. All the
  definitions coincide.
\item Suppose $L$ is a Lie ring of nilpotency class exactly $c_0$, and
  $p_0$ is the smallest prime for which $L$ is not powered (see below
  for the case that $L$ is rationally powered). Suppose that $c_0 \le
  p_0 - 1$. Then, for each $c$ satisfying $c_0 \le c \le p_0 - 1$, $L$
  is a global class $c$ Lazard Lie ring. Therefore, for each such $c$,
  we can consider a definition of $\exp(L)$ based on the global class
  $c$ Lazard correspondence. All these definitions coincide.

  In the case that $L$ is rationally powered and has class exactly
  $c_0$, we can consider a definition of $\exp(L)$ based on the global
  class $c$ Lazard correspondence for all $c \ge c_0$. All the
  definitions coincide.
\end{itemize}

Thus, we can define the global Lazard correspondence as the union of
all the global class $c$ Lazard correspondences. We now make a brief
philosophical remark regarding why this behaves a little worse than
the correspondences discussed earlier.

An observation people make early on in their study of algebraic
structures is that whereas intersections behave nicely with respect to
presrving closure and important structural attributes, unions rarely
do. For instance, an intersection of subgroups is a subgroup, but a
union of two subgroups is not a subgroup unless one of them contains
the other. In general, unions are not guaranteed to preserve closure
except in situations where they are unions of ascending chains or of
directed sets.

We encounter a similar type of problem dealing with the global Lazard
correspondence, albeit the problem is now at a higher level of
abstraction. Each of the individual global class $c$ Lazard
correspondences behaves very nicely, just as the Baer correspondence
and the abelian Lie correspondence do. However, when we combine the
correspondences, we are dealing with heterogeneous types of objects,
and we need to be more careful. In particular, the combined
correspondence does not behave well with respect to direct
products. Also, it does not behave well with respect to the relation
between subgroups and quotient groups: the normal subgroups that are
in the category are not the same as the normal subgroups for which the
quotient groups are in the category. We will return to these in more
detail in Section \ref{sec:global-lazard-correspondence-sub-quot-dp}.

\subsection{Isomorphism of categories}

We follow here the general template outlined in Section
\ref{sec:isocat-template}.

Each global class $c$ Lazard correspondence defines an isomorphism of
categories over the category of sets\footnote{This means that the
  functors in both directions preserve the underlying set.} between the
full subcategory of the category of groups comprising the global
class $c$ Lazard Lie groups and the full subcategory of the category
of Lie rings comprising the global class $c$ Lazard Lie rings. The
functor from Lie rings to groups is the $\exp$ functor and the
functor from groups to Lie rings is the $\log$ functor.

The global Lazard correspondence can be viewed as an isomorphism of
categories over the category of sets between the full subcategory of
the category of groups comprising all the global Lazard Lie groups,
and the full subcategory of the category of Lie rings comprising all
the global Lazard Lie rings.

The set of objects involved in the global Lazard correspondence is the
union of the sets of objects involved in each of the global class $c$
Lazard correspondences. The set of morphisms, however, is strictly
larger, because the new correspondence includes morphisms between
global class $c$ Lazard Lie groups for different values of $c$. For
instance, an inclusion of $\Z$ in $UT(3,\Q)$ is a morphism in the
category on the group side, because $\Z$ is a global class $1$ Lazard
Lie group and $UT(3,\Q)$ is a global class $2$ Lazard Lie group.

We can deduce a number of immediate consequences of the isomorphism of
categories similar to those in Section
\ref{sec:baer-correspondence-isocat-consequences}.

\subsection{Subgroups, quotients, and direct products}\label{sec:global-lazard-correspondence-sub-quot-dp}

We describe how subgroups, quotient groups, and direct products
interact with the global class $c$ Lazard correspondence.

\subsubsection{Subgroups}\label{sec:global-lazard-correspondence-subgroups}

Suppose $G$ is a global class $c$ Lazard Lie group and $L = \log(G)$
is the corresponding global class $c$ Lazard Lie ring. Then, the
global class $c$ Lazard correspondence gives a bijective
correspondence:

\begin{center}
  Global class $c$ Lazard Lie subgroups of $G$ $\leftrightarrow$
  Global class $c$ Lazard Lie subrings of $L$
\end{center}

The global Lazard correspondence gives a bijective correspondence:

\begin{center}
  Global Lazard Lie subgroups of $G$ $\leftrightarrow$
  Global Lazard Lie subrings of $L$
\end{center}

The latter correspondence is somewhat more general than the former,
because it includes subgroups (respectively, subrings) that are global
Lazard Lie groups (respectively, global Lazard Lie rings) for smaller
nilpotency class values. In particular, it includes all abelian
subgroups and abelian subrings, as well as all Baer Lie subgroups and
Baer Lie subrings. These may not qualify for the earlier
correspondence because they may not be powered over {\em all} the
primes less than or equal to $c$. For instance, any copy of $\Z$
inside $UT(3,\Q)$ is part of the latter correspondence but not the
former.

\subsubsection{Quotient groups}\label{sec:global-lazard-correspondence-quot}

Suppose $G$ is a global class $c$ Lazard Lie group and $L = \log(G)$
is the corresponding global class $c$ Lazard Lie ring. We have a
correspondence:

\begin{center}
  Global class $c$ Lazard Lie groups that are quotient groups of $G$
  $\leftrightarrow$ Global class $c$ Lazard Lie rings that are
  quotient Lie rings of $L$
\end{center}

We also obtain correspondences between the corresponding kernels:

\begin{center}
  Global class $c$ Lazard Lie groups that are normal subgroups of $G$
  $\leftrightarrow$ Global class $c$ Lazard Lie rings that are ideals
  of $L$
\end{center}

Note that the equivalence relies on Theorem \ref{thm:two-out-of-three}
and Lemma \ref{lemma:two-out-of-three-lie}. These show that in a
$\pi_c$-powered nilpotent group (respectively, $\pi_c$-powered
nilpotent Lie ring) a normal subgroup (respectively, ideal) is
$\pi_c$-powered if and only if the quotient group (respectively,
quotient Lie ring) is $\pi_c$-powered. Here, we take $\pi_c$ to be the
set of all primes less than or equal to $c$.

We also have a correspondence:

\begin{center}
  Global Lazard Lie groups that are quotient groups of $G$
  $\leftrightarrow$ Global Lazard Lie rings that are quotient Lie
  rings of $L$
\end{center}

This gives rise to another correspondence:

\begin{center}
  Normal subgroups of $G$ for which the quotient group is a global
  Lazard Lie group $\leftrightarrow$ Ideals of $L$ for which the
  quotient Lie ring is a global Lazard Lie ring
\end{center}

Note, however, that these normal subgroups are {\em not necessarily
  the same as} the normal subgroups that are also global Lazard Lie
groups. For instance, consider the global class two Lazard Lie group
(i.e., the Baer Lie group) $G = UT(3,\Q)$. A subgroup $H$ isomorphic
to $\Z$ inside the center of $G$ is a normal subgroup that is a global
class one Lazard Lie group. However, $G/H$ is not a global Lazard Lie
group, because it has class two but has $2$-torsion in the center. On
the other hand, consider the subgroup $K$ of $G$ such that $K/Z(G)$ is
isomorphic to $\Z \times \Z$ inside $G/Z(G) \cong \Q \times \Q$. $K$
is a normal subgroup that is not a global Lazard Lie group, but the
quotient group $G/K$ is a global Lazard Lie group.

\subsubsection{Direct products}

Suppose $G_i, i \in I$, are all global class $c$ Lazard Lie groups for
the same value of $c$. Then, the external direct product $\prod_{i \in
  I} G_i$ is also a global class $c$ Lazard Lie group. Moreover,
$\log(\prod_{i \in I} G_i) = \prod_{i \in I} \log(G_i)$.

On the other hand, it may happen that $G_1$ is a global class $c_1$
Lazard Lie group, $G_2$ is a global class $c_2$ Lazard Lie group, and
the direct product $G_1 \times G_2$ is not a global class $c$ Lazard
Lie group for any value of $c$. For instance, this happens if $G_1 =
\Z/2\Z$ (with $c_1 = 1$) and $G_2 = UT(3,\Q)$ (with $c_2 = 2$).

\subsection{Inner derivations and inner automorphisms}\label{sec:lazard-adjoint}

Suppose $G$ is a global class $c$ Lazard Lie group and $L$ is the
corresponding global class $c$ Lazard Lie ring. The group $G/Z(G)
\cong \operatorname{Inn}(G)$ is in global class $c - 1$ Lazard
correspondence with the Lie ring $L/Z(L) \cong
\operatorname{Inn}(L)$. In particular, $G/Z(G)$ and $L/Z(L)$ have the
same underlying set.

For any $x$ in the common underlying set of $G$ and $L$, denote by
$\overline{x}$ its image in the common underlying set of $G/Z(G)$ and
$L/Z(L)$. We have two set maps of interest from $L$ to itself induced
by $x$:

\begin{itemize}
\item The automorphism of $L$ arising from the inner automorphism of
  $G$ given by conjugation by $x$, i.e., the map $g \mapsto
  xgx^{-1}$. We denote this as $\operatorname{Ad}_x$.
\item The inner derivation of $L$ given as $g \mapsto [x,g]$. We
  denote this as $\operatorname{ad}_x$.
\end{itemize}

We then have the relationship:

$$\operatorname{Ad}_x = \exp(\operatorname{ad}_x)$$

where the exponentiation occurs inside the ring
$\operatorname{End}_\Z(L)$ of the additive group of $L$.

Deducing the result would require us to return to the free group
scenario, obtain the result in that scenario where the group and Lie
ring are both subsets in an associative ring, and then apply
homomorphisms. We are not using this result for our main proofs, so we
skip the proof. Interested readers may look at Lemma 3.3 of
George Glauberman's paper \cite{Partialextensions}.

This relationship continues to be valid in the general case (the
$3$-local case) that we discuss in the next section.

\subsection{The case of adjoint groups}\label{sec:adjoint-lazard}

In Section \ref{sec:adjoint-exp-log}, we introduced the general idea
of the adjoint group of a radical ring. Suppose $N$ is a nilpotent
associative ring of nilpotency class $c$ (this means that any product
of length $c + 1$ or more is zero) and $1 + N$ is the corresponding
adjoint group. We can view $N$ as a Lie ring by using the same
additive structure and defining:

$$[x,y] := xy - yx$$

$N$ as a Lie ring also has nilpotency class at most $c$ (although the
nilpotency class as a Lie ring could be smaller). The following turn
out to be true.

\begin{itemize}
\item The adjoint group $1 + N$ is a nilpotent group of nilpotency
  class at most $c$. Note that this is true regardless of whether $N$
  is powered over any primes.
\item Suppose the additive group of $N$ is powered over the set
  $\pi_c$ of all primes less than or equal to $c$. Then, the Lie ring
  $N$ and the group $1 + N$ are in global class $c$ Lazard
  correspondence {\em up to isomorphism}. Moreover, the isomorphism is
  given by the set maps $\exp:N \to 1 + N$ and $\log: 1 + N \to N$
  described in Section \ref{sec:adjoint-exp-log} (specifically, see
  Section \ref{sec:truncated-exponentials}). This follows quite
  directly from the analytical framework described in Section
  \ref{sec:adjoint-exp-log}.
\end{itemize}

\subsection{The case of the niltriangular matrix Lie ring and the unitriangular matrix group}\label{sec:unitriangular-lazard}
%%TONOTDO: Have a section before this describing the adjoint group in general
The material discussed here builds on Section
\ref{sec:unitriangular-lie-correspondence} in the introduction.

Suppose $R$ is a commutative associative unital ring and $n$ is a
positive integer. We define $UT(n,R)$ as the group of $n \times n$
upper triangular matrices with $1$s on the diagonal, with the group
operation being the usual matrix multiplication. we define $NT(n,R)$
as the Lie ring of $n \times n$ strictly upper triangular matrices,
with the Lie bracket defined as $[x,y] = xy - yx$ where the
multiplication here is matrix multiplication. We had considered groups
of the form $UT(n,R)$ and $NT(n,R)$ (in the case $n = 3$) when
describing counterexamples to naively plausible statements about
powering and divisibility in Sections \ref{sec:group-ctex} and
\ref{sec:lie-ring-ctex}.

The following turn out to be true for any positive integer $c$. Denote
by $\pi_c$ the set of all primes less than or equal to $c$.

\begin{enumerate}
\item The group $UT(c+1,R)$ is a group of nilpotency class $c$ and the
  Lie ring $NT(c+1,R)$ is a Lie ring of nilpotency class $c$.
\item The group $UT(c+1,R)$ is the adjoint group to the radical ring
  $NT(c+1,R)$, based on the definitions in Section
  \ref{sec:adjoint-exp-log}.
\item If the additive group of $R$ is $\pi_c$-powered, the additive
  group of $NT(c+1,R)$ is also $\pi_c$-powered, and the group
  $UT(c+1,R)$ is also $\pi_c$-powered. This follows from (2) and the
  discussion in Section \ref{sec:adjoint-lazard}.
\item If the equivalent conditions for (3) hold, then the group
  $UT(c+1,R)$ is in global class $c$ Lazard correspondence {\em up to
  isomorphism} with $NT(c+1,R)$. The logarithm and exponential maps
  used to describe the bijection of sets are the usual matrix
  logarithm and exponential maps. This follows by combining (2) and
  the discussion in Section \ref{sec:adjoint-lazard}.
\end{enumerate}

Particular cases of interest for the above scenario are:

\begin{itemize}
\item $R$ is a field of characteristic zero. Note that in this case,
  we get an instance of the class $c$ Malcev correspondence.
\item $R$ is a field of characteristic $p$, where $p > c$.
\item $R$ is a local ring of characteristic $p^k$, where $p$ is the
  underlying prime and $p > c$. For instance, the case $c = 2$ and $R
  = \Z/9\Z$.
\end{itemize}

%\newpage

\section{The general definition of the Lazard correspondence: $3$-local case}\label{sec:lazard-correspondence}

\subsection{Definition of local nilpotency class}\label{sec:local-nilpotency-class-def}

This is the general version of the Lazard correspondence.

\begin{definer}[$k$-local nilpotency class]
  Suppose $G$ is a group and $k$ is a positive integer (we are
  generally interested in $k \ge 2$). The {\em $k$-local nilpotency
    class} of $G$ is the supremum over all subgroups $H$ of $G$ with
  generating set of size at most $k$ of the nilpotency class of
  $H$. Note that if any subgroup of $G$ with generating set of size at
  most $k$ is non-nilpotent, {\em or} if there is no common finite
  upper bound on the nilpotency classes of all such subgroups, then
  the $k$-local nilpotency class is $\infty$.

  We say that $G$ is a group of $k$-local nilpotency class (at most)
  $c$ if the $k$-local nilpotency class of $G$ is finite and less than
  or equal to $c$.
\end{definer}

\begin{definer}[$k$-local nilpotency class of a Lie ring]
  Suppose $L$ is a Lie ring and $k$ is a positive integer (we are
  generally interested in $k \ge 2$). The {\em $k$-local nilpotency
    class} of $L$ is the supremum over all Lie subrings $M$ of $L$ with
  generating set of size at most $k$ of the nilpotency class of
  $M$. Note that if any Lie subring of $L$ with generating set of size at
  most $k$ is non-nilpotent, {\em or} if there is no common finite
  upper bound on the nilpotency classes of all such subgroups, then
  the $k$-local nilpotency class is $\infty$.

  We say that $L$ is a Lie ring of $k$-local nilpotency class (at most)
  $c$ if the $k$-local nilpotency class of $L$ is finite and less than
  or equal to $c$.
\end{definer}

\subsection{Definition of Lazard Lie group and Lazard Lie ring, and the correspondence}

We can now define Lazard Lie group and Lazard Lie ring.

\begin{definer}[Lazard Lie group]
  Suppose $G$ is a group and $c$ is a positive integer. $G$ is termed
  a {\em class $c$ Lazard Lie group} if {\em both} the following
  conditions are satisfied:

  \begin{enumerate}
  \item The $3$-local nilpotency class of $G$ is at most $c$.
  \item $G$ is powered over all primes less than or equal to $c$.
  \end{enumerate}
\end{definer}

\begin{definer}[Lazard Lie ring]
  Suppose $L$ is a Lie ring and $c$ is a positive integer. $L$ is
  termed a {\em class $c$ Lazard Lie ring} if {\em both} the
  following conditions are satisfied:

  \begin{enumerate}
  \item The $3$-local nilpotency class of $L$ is at most $c$.
  \item $L$ is powered over all primes less than or equal to $c$.
  \end{enumerate}
\end{definer}

The ($3$-local) class $c$ Lazard correspondence is a correspondence:

\begin{center}
  ($3$-local) class $c$ Lazard Lie groups $\leftrightarrow$
  ($3$-local) class $c$ Lazard Lie rings
\end{center}

Note that the ``($3$-local)'' qualifier may be specified on occasions
where there is potential for confusion with the global class $c$
Lazard correspondence. However, by default, in this section and later,
if we just say {\em class $c$ Lazard} (followed by any of the terms
{\em correspondence}, {\em Lie group}, and {\em Lie ring}), we are
  referring to the $3$-local case.

This correspondence works in a manner analogous to the global class $c$ Lazard
correspondence, and the remarks made in the preceding section (Section
\ref{sec:global-lazard-correspondence}) about the global Lazard
correspondence apply to the ($3$-local) Lazard correspondence. For
brevity, we will not repeat the observations.

Note that the {\em formulas} use only $2$ elements at a time. So one
might naively expect that a ``$2$-local'' condition would
suffice. However, the associativity condition for groups, and
correspondingly, many of the Lie ring identities (associativity of
addition, distributivity, and the Jacobi identity), reference three
arbitrary elements at a time. This is why we need to impose the
``$3$-local'' condition. With this caveat, the reasoning is similar to
that for the global class $c$ Lazard correspondence. All the other
remarks made in the previous section also apply. We had touched on the
significance of the numbers $2$ and $3$ in an introductory section
(Section \ref{sec:globally-and-locally-nilpotent}) and we are now
making use of the observations made at the time.

\subsection{Divergence between the $3$-local and global class cases}

For $c = 1,2,3$, the global class $c$ Lazard correspondence coincides
with the class $c$ Lazard correspondence. The cases $c = 1$ and $c =
2$ are obvious: if every $3$-generated subgroup (respectively Lie
subring) is abelian, the whole group (respectively, Lie ring) is
abelian. Similarly, if every $3$-generated subgroup (respectively Lie
subring) has class at most two, the whole group (respectively, Lie
ring) has class at most two. This is because the class two condition
involves checking the triviality of $[[x,y],z]$, and this expression
uses only three elements. The case $c = 3$ is more interesting,
because it is not immediate. However, providing details of this case
would distract us from our main goal, so we omit it. The case of $c =
3$ for Lie rings is discussed in the Appendix,
Section \ref{appsec:3-local-class-three-implies-global-class-three}.

The global class $c$ Lazard correspondence becomes strictly weaker
than the class $c$ Lazard correspondence for $c \ge 4$, as
demonstrated by Lazard in \cite{Lazardsoriginal}. For more about the
relationship between local and global nilpotency, see the literature
on Engel conditions and local nilpotency. In particular, the paper
\cite{Pilgrim} by Pilgrim and the paper \cite{Plotkin} by Plotkin
provide a summary of the known results relating local and global
nilpotency.

\subsection{The $p$-group case of the Lazard correspondence}

Recall that a $p$-group is a group in which the order of every element is
a power of $p$. We use the term {\em $p$-Lie ring} for a Lie ring
whose additive group is a $p$-group.

The {\em global} Lazard correspondence is a correspondence:

$p$-groups of nilpotency class at most $p - 1$ $\leftrightarrow$
$p$-Lie rings of nilpotency class at most $p - 1$

The Lazard correspondence (including the $3$-local case) is a
correspondence:

$p$-groups of $3$-local nilpotency class at most $p - 1$
$\leftrightarrow$ $p$-Lie rings of $3$-local nilpotency class at most
$p - 1$.
