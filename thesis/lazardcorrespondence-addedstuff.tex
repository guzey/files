\documentclass[10pt]{amsart}

%Packages in use
\usepackage{fullpage, hyperref, vipul, amssymb}

%Title details
\title{Lazard correspondence: stuff to be added}
\author{Vipul Naik}

%List of new commands
\newcommand{\Skew}{\operatorname{Skew}}
\newcommand{\ad}{\operatorname{ad}}
\begin{document}
\maketitle

The current version of the Lazard correspondence document is very
messy, with some portions and references confused. So I have prepared
a summary document below of the key changes that I had planned to make
and will hopefully complete making in a few days.

\begin{itemize}

\item Add the result that if $G$ is a nilpotent group such that $G/G'$
  is powered over a prime $p$, then $G$ is also powered over $p$ (this
  is only a slight modification of the existing proof that the derived
  subgroup is closed under divisibility relations by primes).

\item Introduce (with references if available?) the notion of a free
  $\pi$-powered class $c$ nilpotent group with $d$ generators. Note
  that these are ``generators'' in the $\pi$-powered group sense, and
  they will not usually generate $G$ as an abstract group. This is
  analogous to the free class $c$ nilpotent group with $d$
  generators. In the free class $c$ nilpotent group with $d$
  generators, we start off with an abelianization that is a free
  abelian group of rank $d$. In the free $\pi$-powered case, the
  abelianization is a free module of rank $d$ over the ring
  $\mathbb{Z}[\pi^{-1}]$, i.e., $\mathbb{Z}$ localized at the set of
  all $\pi$-numbers.

  We can think of the process of going from the free nilpotent class
  $c$ group with $d$ generators to the corresponding $\pi$-powered
  group as tensoring the ``base ring'' with $\mathbb{Z}[\pi^{-1}]$. If
  $\pi$ is the set of all primes, we have tensored the base ring with
  $\mathbb{Q}$.

\item Completely analogous observations hold for Lie rings. Note that
  in the Lie ring case, the tensoring of the base ring agrees with our
  usual concept of tensoring base rings.

\item Suppose $G$ is a $\pi$-powered nilpotent group of class $c$ and
  has a generating set of size $d$ as a $\pi$-powered group. We will
  show we can write $G$ as $F/R$ where $F$ is a free $\pi$-powered
  nilpotent group of class $c + 1$ on $d$ generators, and $R$ is a
  $\pi$-powered normal subgroup of $F$. We will then show the following:

  \begin{itemize}
  \item $G \wedge G$ can be canonically identified with the quotient group
  $[F,F]/[F,R]$.
  \item $[G,G]$ can be canonically identified with the quotient group
    $[F,F]/(R \cap [F,F])$.
  \item The Schur multiplier of $G$, which is defined as the kernel of
  the natural map $G \wedge G \to [G,G]$, can be identified with the
  quotient group $(R \cap [F,F])/[F,R]$.
  \end{itemize}

  {\em How this stands relative to the existing literature}: Similar
  facts to the above are well known when $F$ is actually a free group,
  rather than being free relative to a nilpotency class and
  powering. The result for the Schur multiplier is called the ``Hopf
  formula'' in the literature. Our statement easily follows from that,
  once you note that we can transition to the free group of class $c +
  1$ and with $\pi$-powering without changing any of the quotients of
  interest.

  A little note as to why we use class $c + 1$ instead of class
  $c$. We're starting with an actual free group $F$, and looking at
  $[F,F]/[F,R]$. We want to replace $F$ by $F/\gamma_{c+2}(F)$, i.e.,
  we declare trivial anything in $\gamma_{c+2}(F)$. We know that $G
  \cong F/R$ has class $c$, so $\gamma_{c+1}(F) \le R$. Thus,
  $\gamma_{c + 2}(F) \le [F,R]$, hence is contained in the denominator
  group of the $[F,F]/[F,R]$ quotient. Thus, quotienting out by this
  does not lead to any loss of information.

  There is also a direct interpretation in terms of extension theory
  (basically, the central extensions of a class $c$ group are class at
  most $c + 1$, so we do not need to look deeper when calculating
  these groups that classify the extensions).

\item We will demonstrate a similar result for Lie rings: if $L$ is a
  $\pi$-powered nilpotent Lie ring of class $c$ that has a generating
  set of size $d$ as a $\pi$-powered Lie ring, there exists a free
  $\pi$-powered nilpotent Lie ring $U$ of class $c + 1$ on $d$
  generators and a $\pi$-powered ideal $I$ in $U$, such that $L \cong
  U/I$. We will then show that the exterior square $L \wedge L$ can be
  identified with the quotient $[U,U]/[U,I]$.

\item The Lazard correspondence need not work on class $c + 1$ even if
  it worked on class $c$ (this is the edge case where $c + 1$ is a
  prime not in $\pi$).

  If the Lazard correspondence did work, we could do a Lazard
  correspondence to correspond $F \leftrightarrow U$, $R
  \leftrightarrow I$, and thus get isomorphisms at all levels.

  Since the Lazard correspondence need not work, we need to work
  around this difficulty. Imagine that we were working with
  $\mathbb{Q}$. Then, the problem would not exist. We need to show
  that under the relevant formulas as they happen over $\mathbb{Q}$,
  the free $\pi$-powered Lie ring and the free $\pi$-powered group get
  mapped within each other.

\end{itemize}

\end{document}