\documentclass[10pt]{amsart}

%Packages in use
\usepackage{fullpage, hyperref, vipul, amssymb}

%Title details
\title{Extending the Baer correspondence to 2-groups: result summary}
\author{Vipul Naik}

%List of new commands
\newcommand{\Skew}{\operatorname{Skew}}
\newcommand{\Aut}{\operatorname{Aut}}
\newcommand{\Ext}{\operatorname{Ext}}
\makeindex
\begin{document}
\maketitle

Here are the main results presented in a form that stands independent
of the more detailed document. In our setup, $E$ is a (finite) group
of nilpotency class two. Without loss of generality, we can assume $E$
to be a $2$-group of class two. $A$ is a central subgroup of $E$ and
$G = E/A$ is abelian. In other words, $[E,E] \le A \le Z(E)$.

We use the following two terms:

\begin{enumerate}
\item For $E$ a group with central subgroup $A$ and abelian quotient
  group $G$, a {\em generalized Baer Lie ring} for the extension is a
  Lie ring $L$ also viewed as a central extension with base $A$ and
  quotient $G$, along with a bijection from $E$ to $L$ that (i) is a
  $1$-isomorphism (ii) induces identity maps on the $A$ and $G$
  parts. Usually,we identify $E$ with $L$ as a set.
\item For $E$ a group of nilpotency class two, a {\em generalized Baer
  Lie ring} is a Lie ring $L$ along with a bijection between $L$ and
  $E$ that arises as a generalized Baer Lie ring for some way of
  viewing $E$ as a central extension with abelian quotient group.
\end{enumerate}

Now, the results.

\begin{theorem}[Uniqueness theorem]\label{uniqueness}
  Assume $A$ and $G$ are both finite abelian groups with $G$ acting
  trivially on $A$. Then, for any extension group $E$ with base $A$
  (central) and quotient group $G$, there is at most one generalized
  Baer Lie ring (up to equivalence of extensions) for that extension
  group.
\end{theorem}

Here's a conjecture that I've almost (but not completely) proved:

\begin{conjecture}
  If a generalized Baer Lie ring exists for $E$, then the generalized
  Baer Lie ring is independent of the way of viewing $E$ as a central
  extension (i.e., it is independent of the choice of $A$ and $G$).
\end{conjecture}

Next, we consider various conditions for a generalized Baer Lie ring to exist:

\begin{theorem}
  Suppose $E$ is a class two $2$-group such that $E/Z(E)$ is the
  direct product of a cyclic group and an elementary abelian
  group. Then, $E$ has a generalized Baer Lie ring if and only if
  $[E,E] \subseteq \mho^1(Z(E))$.
\end{theorem}

Here's a theorem that follows from the above using the construction of
a central product.

\begin{theorem}\label{embedding}
  Suppose $P$ is a class two $2$-group such that $P/Z(P)$ is the
  direct product of a cyclic group and an elementary abelian
  group. Then, $P$ can be embedded in a class two $2$-group $E$ such
  that $E$ has a generalized Baer Lie ring.
\end{theorem}

Here's my conjecture.

\begin{conjecture}
  Suppose $E$ is a class two $2$-group. Let $r$ be the number such
  that $2^r$ is the second largest of the elementary divisors for
  $E/Z(E)$. Then, $E$ has a generalized Baer Lie ring if and only if
  $[E,E] \le \mho^r(Z(E))$.
\end{conjecture}

If this conjecture were true, we would also be able to prove the
following using the central product construction:

\begin{conjecture}
  Every class two $2$-group $P$ can be embedded in some class two
  $2$-group $E$ such that $E$ has a generalized Baer Lie ring.
\end{conjecture}

Note that this last conjecture could well be true even if the
conjecture before it were false -- because the specifics of that
conjecture are not that important to proving this one.

A quick numerical summary:

\begin{itemize}
\item For order $16$, there are two cases of a non-abelian group of
  class two having a generalized Baer Lie ring. These explain all the
  non-abelian groups of class two that are $1$-isomorphic to abelian
  groups.
\item For order $32$, there are six cases of a non-abelian group of
  class two having a generalized Baer Lie ring. These explain six of
  the eight non-abelian groups of class two that are $1$-isomorphic to
  abelian groups.
\end{itemize}
Areas that I am exploring:

\begin{itemize}
\item Proving all the conjectures stated above: the generalized Baer
  Lie ring being independent of the central extension perspective
  (this should be easy), and the conjecture for the case where
  $E/Z(E)$ is more complicated than just a direct product of a cyclic
  group an dan elementary abelian group.
\item In principle, there is a way of extending this to class three
  for the primes $2$ and $3$ (but no further) -- however, I have not
  connected these principled possibilities to some of the actual
  examples that arise for the primes $2$ and $3$.
\item There are {\em some} examples (two such of order $32$) where a
  non-abelian group of class two is $1$-isomorphic to an abelian group
  but this $1$-isomorphism does {\em not} seem to arise from a
  generalized Baer Lie ring. I am trying to figure out whether the
  $1$-isomorphism arises purely as an accident or whether there is
  some alternative deeper explanation for it.
\end{itemize}
\end{document}
