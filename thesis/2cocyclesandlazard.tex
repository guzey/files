\documentclass[10pt]{amsart}

%Packages in use
\usepackage{fullpage, hyperref, vipul, amssymb}

%Title details
\title{2-cocycles and a Lazard-like correspondence for some 2-groups}
\author{Vipul Naik}

%List of new commands
\newcommand{\Skew}{\operatorname{Skew}}
\newcommand{\Aut}{\operatorname{Aut}}
\newcommand{\Ext}{\operatorname{Ext}}
\makeindex
\begin{document}
\maketitle

In this document, we outline a basic theoretical framework that allows
us to generalize a particular case of the Lazard correspondence -- the
{\em Baer correspondence}, to some situations beyond the original
formulation of the correspondence. To keep the length of the document
reasonable, proofs are omitted though outlines are provided. All the
proofs are either standard results or have been written up in detail
elsewhere (or both).

\section{Class two groups: 2-cocycles with trivial group action}

\subsection{Definitions of cocycle, coboundary, and cohomology class}

The following setup forms the basis of much of the discussion in this
document. Consider abelian groups $A$ and $G$. Although both $A$ and
$G$ are abelian, we will use additive notation for $A$ (denoting its
operation by $+$, inverse by $-$, and identity element by $0$) and
multiplicative notation for $G$ (denoting its multiplication by
concatenation, inverse by ${}^{-1}$, and identity element by $1$), since
some of the ideas we discuss here will generalize to the case of
non-abelian $G$.

Denote by $C^2(G,A)$ the group of {\em $2$-cocycles} for the trivial
action of $G$ on $A$. Denote by $B^2(G,A)$ the group of {\em
$2$-coboundaries} for the trivial action of $G$ on $A$. In other
words, $C^2(G,A)$ is the group of functions $f:G \times G \to A$
satisfying:

$$f(g_2,g_3) - f(g_1g_2,g_3) + f(g_1,g_2g_3) - f(g_1,g_2) = 0 \ \forall \ g_1,g_2,g_3 \in G$$

$B^2(G,A)$ is the subgroup comprising those $f$ for which there exists
a function $\varphi:G \to A$ such that:

$$f(g_1,g_2) = \varphi(g_1g_2) - \varphi(g_1) - \varphi(g_2) \ \forall \ g_1, g_2 \in G$$

The quotient group $C^2(G,A)/B^2(G,A)$ is defined as the {\em
$2$-cohomology group} (or second cohomology group) $H^2(G,A)$ for the
trivial group action of $G$ on $A$. The elements of $H^2(G,A)$ are
thus cosets of $B^2(G,A)$ in $C^2(G,A)$ and these elements are termed
{\em 2-cohomology classes}.

\subsection{Relationship with extensions: standard facts}

Here are some standard facts that we will use repeatedly. Note that
while some of these facts have analogues when $G$ is non-abelian
and/or the action is nontrivial, others do no:

\begin{enumerate}
\item Every element of $C^2(G,A)$ can be used to define a group $E$
  with $A$ embedded as a central subgroup of $E$ and a specified
  isomorphism $E/A \to G$. As a set, we take $E = A \times G$, and the
  multiplication is defined as:

  $$(a_1,g_1)(a_2,g_2) = (a_1 + a_2 + f(g_1,g_2), g_1g_2)$$

  The corresponding embedding of $A$ in $E$ is given by:

  $$a \mapsto (a - f(1,1), 1)$$

  where $1$ denotes the identity element of $G$. The corresponding
  projection map from $E$ to $G$ is:

  $$(a,g) \mapsto g$$

  In other words, every element of $C^2(G,A)$ gives rise to a central
  extension with base $A$ and quotient group $G$.
\item Given an extension, we can obtain a $2$-cocycle that yields that
  extension by choosing a section, i.e., a set theoretic map $s:G \to
  E$ that is a one-sided inverse to the projection map from $E$ to
  $G$. For a section $s$, the corresponding $2$-cocycle $f$ is defined
  by:

  $$f(g_1,g_2) = s(g_1)s(g_2)\{s(g_1g_2)\}^{-1}$$

\item Two elements of $C^2(G,A)$ give rise to congruent extensions if
  and only if they are in the same coset of $B^2(G,A)$ in
  $C^2(G,A)$. This is roughly because shifting by a $2$-coboundary
  roughly corresponds with choosing a different set theoretic section
  of the extension. Specifically, shifting a section by a function
  $\varphi:G \to A$ corresponds to shifting the corresponding
  $2$-cocycle by the boundary of $\varphi$.

  Thus, the set of congruence classes of extensions is identified with
  the $2$-cohomology group $H^2(G,A)$. Note that for all these
  extensions, the big group is a {\em group of nilpotency class at
  most two}. We will often use the term {\em nilpotency class two} to
  mean nilpotency class at most two.

\item For every extension, we can choose a {\em normalized
  $2$-cocycle} representing it. A normalized $2$-cocycle is a
  $2$-cocycle with the property that $f(g,1) = 0$ for all $g \in G$
  (note that this is equivalent to assuming that $f(1,1) =
  0$). Specifically, if the section $s:G \to E$ that we pick chooses
  the identity element of $E$ as the image of the identity element of
  $G$, we get a normalized $2$-cocycle.

  In terms of the group of $2$-cocycles, this says that the normalized
  $2$-cocycles span the $2$-cocycles modulo the $2$-coboundaries. In
  other words, every $2$-cocycle is the sum of a normalized
  $2$-cocycle and a $2$-coboundary. For most practical purposes, we
  can thus restrict attention to normalized $2$-cocycles. With
  normalized $2$-coycles, the embedding of $A$ in $E$ becomes more
  straightforward:

  $$a \mapsto (a,1)$$

\item Both $\Aut(G)$ and $\Aut(A)$ act via composition on the group
  $C^2(G,A)$ and the action takes the subgroup $B^2(G,A)$ to itself --
  hence there is an action on the quotient group
  $H^2(G,A)$.\footnote{Note: This uses the fact that the action of $G$
  on $A$ is trivial. If the action were nontrivial, we would still
  have some automorphisms acting -- however, these automorphisms would
  have to lie in suitable centralizers.} The actions commute (because
  composition is associative, and the two actions by composition are
  on opposite sides) so we get an action of $\Aut(G) \times \Aut(A)$
  on $H^2(G,A)$. Any two elements of $H^2(G,A)$ in the same orbit
  under this action correspond to extensions that are essentially the
  same up to relabeling of $G$ and $A$ -- in particular, the
  isomorphism class of the big group $E$ is the same. The quotient
  space (or the set or orbits), however, does not have a group
  structure for obvious reasons.

  We define $C^2_{Fix}(G,A)$ as the group of $2$-cocycles fixed under
  the action and $H^2_{Fix}(G,A)$ as the group of $2$-cohomology
  classes fixed under the action. Note that $H^2_{Fix}(G,A)$ could, in
  principle, be bigger than the image in cohomology of
  $C^2_{Fix}(G,A)$.

\item For a function $f:G \times G \to A$, define $\Skew f$ as
  $(g_1,g_2) \mapsto f(g_1,g_2) - f(g_2,g_1)$. Call a $2$-cocycle $f$
  {\em symmetric} if $\Skew f = 0$. We then have the following:

  \begin{enumerate}
  \item $\Skew$ is a linear map on the set of all functions $f:G
    \times G \to A$ under pointwise addition. Further, it maps
    $2$-cocycles to $2$-cocycles (it does a lot more, as we shall see
    later).
  \item Every $2$-coboundary is a symmetric $2$-cocycle. Thus, the
    property of being a symmetric $2$-cocycle is well defined up to
    cohomology.
  \item The group of symmetric $2$-cocycles forms a subgroup of
    $C^2(G,A)$ (that we denote by $C^2_{sym}(G,A)$), and its image in
    $H^2(G,A)$ forms a subgroup of $H^2(G,A)$. This subgroup
    corresponds to those extensions where the big group $E$ is also an
    abelian group. We denote this subgroup by $H^2_{Sym}(G,A)$. Hence,
    this subgroup is isomorphic to $\Ext^1(G,A)$.
  \item The subgroup $H^2_{sym}(G,A)$ is invariant (not necessarily
    pointwise) under the action of $\Aut(G) \times \Aut(A)$.
  \item $\Skew f$ represents the commutator map in the big group $E$
    as follows. For $x,y \in E$, denote by $\overline{x},
    \overline{y}$ the corresponding elements in $G$. Then, $[x,y] =
    f(\overline{x},\overline{y})$ where the element of $A$ in the
    right side is interpreted naturally as an element of $E$ via the
    embedding of $A$ in $E$.
  \item In particular, each coset of $H^2_{sym}(G,A)$ in $H^2(G,A)$
    corresponds to all the congruence classes of extensions that give
    rise to one particular commutator map.
  \item The image of $\Skew$ is in the group of alternating biadditive
    maps from $G$ to $A$. This can be seen both algebraically by
    manipulating equations involving $2$-cocycles and in terms of the
    known properties of the commutator map for groups of class
    two. More on this later.
  \end{enumerate}
\end{enumerate}

\section{Twisting into abelianness}

\subsection{The idea of twisting}
Given a group $E$ constructed as one of these extensions, we can twist
or shift $E$ by a $2$-cocycle $f$ by defining a new multiplication
$*$:

$$x * y := f(\overline{x},\overline{y})xy$$

Here, $f(\overline{x},\overline{y})$ is the element in the central
subgroup $A$ that we obtain by applying the $2$-cocycle $f$ to the
images of $x$ and $y$ in the quotient group $G$. The new group
obtained can naturally be viewed as a central extension with base $A$
and quotient group $G$, corresponding to the $2$-cocycle $c + f$. In
other words, the original $2$-cocycle has been modified by $f$.

Shifting by $-f$, or untwisting by $f$, would mean using the formula:

$$x * y := \frac{xy}{f(\overline{x},\overline{y})}$$


\subsection{Specific kinds of $2$-cocycles and twistings}

\begin{enumerate}
\item (Recall): A $2$-cocycle is a {\em normalized $2$-cocycle} if,
  whenever either of the inputs is the identity element, so is the
  output.

  Every cohomology class contains at least one normalized
  $2$-cocycle. Thus, we will always deal with normalized $2$-cocycles
  unless otherwise specified.

  Twisting a given group extension by a normalized 2-cocycle preserves
  the identity element.

\item A $2$-cocycle is an {\em IIP 2-cocycle} if it is normalized and,
  whenever the two inputs are inverses of each other, so is the
  output. IIP 2-cocycles form a subgroup of $C^2(G,A)$ that we denote
  by $C^2_{IIP}(G,A)$.

  If there are no elements of order $2$, then every cohomology class
  contains an IIP 2-cocycle. However, in the presence of elements of
  order $2$ in $G$, it is possible to have $2$-cohomology classes
  without any representative IIP 2-cocycle. We define $H^2_{IIP}(G,A)$
  as the subgroup of $H^2(G,A)$ comprising those $2$-cohomology
  classes that can be represented by an IIP $2$-cocycle. As noted
  here, if $G$ has no elements of order $2$, then $H^2_{IIP}(G,A) =
  H^2(G,A)$.

  Twisting a given group extension by an IIP 2-cocycle preserves the
  identity element {\em and} preserves inverses.
\item A $2$-cocycle is a {\em cyclicity-preserving $2$-cocycle} if it
  is normalized and, whenever both inputs lie inside a common cyclic
  subgroup, it takes the value $0$. Cyclicity-preserving $2$-cocycles
  form a subgroup of $C^2(G,A)$ that we denote by
  $C^2_{CP}(G,A)$. $C^2_{CP}(G,A)$ is a subgroup of $C^2_{IIP}(G,A)$.
   
  Many cohomology classes do not contain cyclicity-preserving
  $2$-cocycles, so having a representative that is a
  cyclicity-preserving $2$-cocycle puts a nontrivial constraint on a
  $2$-cohomology class. We define $H^2_{CP}(G,A)$ as the subgroup of
  $H^2(G,A)$ comprising those $2$-cohomology classes that can be
  represented by a cyclicity-preserving $2$-cocycle.

  Twisting a given group extension by a cyclicity-preserving
  $2$-cocycle preserves the cyclic subgroup structure, i.e., it is a
  $1$-isomorphism. Here, we define a {\em 1-isomorphism} as a
  bijection between groups whose restriction to any cyclic subgroup on
  either side is an isomorphism.

  In particular, it preserves the directed power graph, undirected
  power graph, order statistics, and other such information about the
  group.
\end{enumerate}

\subsection{What we want to do}

We are given a group $E$ with a central subgroup $A$ and abelian
quotient group $G$. (In most cases, $A$ is taken as precisely the
center of $E$; however, it may not remain the precise center after
twisting, hence we do not include this as a condition in the general
discussion). Let $c$ be a $2$-cocycle describing the extension $E$. We
want to write:

$$c = f + s$$

where $f$ is a cyclicity-preserving $2$-cocycle and $s$ is a symmetric
$2$-cocycle. If we have such a decomposition, then twisting $E$ by
$-f$ (or the untwisting of $E$ by $f$) gives us an abelian group
described by the $2$-cocycle $s$.

The group of $2$-cocycles that we can write in this form is the sum
$C^2_{CP}(G,A) + C^2_{sym}(G,A)$ inside $C^2(G,A)$. The corresponding
group of $2$-cohomology classes covered is $H^2_{CP}(G,A) +
H^2_{sym}(G,A)$.

\subsection{Uniqueness theorem}

The following uniqueness result is important:

\begin{theorem}\label{uniqueness}
  Assume $A$ and $G$ are both finite abelian groups with $G$ acting
  trivially on $A$. The following (easily seen to be equivalent) are
  true:
  \begin{enumerate}
  \item $C^2_{sym}(G,A) \cap C^2_{CP}(G,A) \subseteq B^2(G,A)$.
  \item $H^2_{sym}(G,A) \cap H^2_{CP}(G,A)$ is trivial.
  \item For any extension group $E$ with base $A$ (central) and
    quotient group $G$, there is at most one abelian group extension
    (up to congruence) that can be obtained from it by twisting using
    a cyclicity-preserving $2$-cocycle. All the cyclicity-preserving
    $2$-cocycles that can be used are in the same cohomology class.
  \end{enumerate}
\end{theorem}

The proof is related to the idea that two finite abelian group that
are $1$-ismorphic are in fact isomorphic.

The key idea behind the proof is to exploit the fact that cocycle
groups are groups: namely, we can translate stuff to the identity
element of a group. Proving the specific statement {\em at} the
identity element (i.e., for the split extension) turns out to be a lot
easier than proving it elsewhere. However, because of the nature of
groups, the proof applies everywhere.

\begin{proof}
  (1) can be reinterpreted as follows: Let $E$ be the extension $A
  \times G$. Let $f$ be a symmetric cyclicity-preserving
  $2$-cocycle. Let $F$ be the extension obtained by twisting $E$ by
  $f$. Then, $F$ is congruent to $E$.

  We prove this as follows. First, note that since $f$ is
  cyclicity-preserving, it preserves orders of elements. If we denote
  the multiplication in $F$ by $*$ and the multiplication in $E$ by
  $+$, then:

  $$x * y := x + y + f(x,y)$$

  Since $f$ is cyclicity-preserving, $*$ and $+$ agree on cyclic
  subgroups and in particular the cyclic subgroups are the same.

  Moreover, the identification of $E$ with $F$ induces identity maps
  on the corresponding subgroups $A$ and the corresponding quotients
  $G$. Since $E$ has the property that for every $g \in G$, there
  exists $x \in E$ mapping to $G$ and having the same order as $g$,
  $F$ also has this property.

  Since $f$ is symmetric, $F$ is abelian. $F$ is thus an abelian
  extension with base $A$, quotient $G$, and the property that for
  every $g \in G$, there exists $x \in E$ mapping to $G$ and having
  the same order as $g$.

  Suppose now that we write $G$ as an internal direct product of
  cyclic subgroups $T_1, T_2, \dots, T_n$ generated by elements $t_1,
  t_2, \dots, t_n$ respectively. For each $t_i$, let $u_i$ be an
  element of $F$ in its coset and having the same order. Let $U_i$ be
  the subgroup generated by $u_i$. Then, each $U_i$ is isomorphic to
  the corresponding $T_i$. Moreover, $U_1 \times U_2 \times \dots
  \times U_n$ (in $F$) is a complement to $A$ in $F$ with the natural
  projection to $G$ being an isomorphism, and hence $F$ is congruent
  to $A \times G$. (Note: I've skipped some details here to avoid
  messiness, but have worked them out elsewhere).
\end{proof}

\subsection{Stating what we need to do}

We are confronted with the question: given a $2$-coycle $c$, how do we
go about finding a nice (i.e., cyclicity-preserving) $2$-cocycle $f$
such that $c - f$ is symmetric? Note that the symmetric $2$-cocycles
are the kernel of the skew map, so this is equivalent to requiring
that $\Skew c = \Skew f$, or equivalently, that $\Skew f$ is the
commutator map. In other words, we need to find a cyclicity-preserving
$2$-cocycle $f$ whose skew is precisely the commutator, so that when
we untwist by $f$, we land in the abelian case.

Let $\lambda = \Skew c$ be the commutator map.

\begin{lemma}
  $\lambda$ is alternating and biadditive. In particular, it is skew
  symmetric.  
\end{lemma}

This follows from the known properties of the commutator map for class
two groups. However, it can also be deduced without reference to
groups at all, using algebraic manipulations of $2$-cocycles. In other
words, the following is true and can be proved purely by cocycle
equation manipulation:

\begin{lemma}
  Let $c$ be a $2$-cocycle for a trivial group action of an abelian
  group on an abelian group. Then, $\Skew c$ is alternating and
  biadditive.
\end{lemma}

Either way, $\lambda$ is alternating and biadditive. We also note the
following easy to prove but perhaps not very well known fact:

\begin{lemma}
  Any biadditive function is a $2$-cocycle. More generally, any
  function $G^n \to A$ that is additive in each coordinate is a
  $n$-cocycle.
\end{lemma}

\section{Baer's solution to the twisting problem}


\subsection{Implications and interpretation}

\begin{lemma}
  \begin{enumerate}
  \item The group of alternating biadditive maps from $G$ to $A$ is
    isomorphic to $\operatorname{Hom}(\bigwedge^2G,A)$.
  \item The $\Skew$ map establishes an isomorphism between
    $H^2(G,A)/H^2_{sym}(G,A)$ and the subgroup of
    $\operatorname{Hom}(\bigwedge^2G,A)$ identified with the image of
    the $\Skew$ map on $H^2(G,A)$.
  \item If every alternating biadditive map arises as the skew of some
    $2$-cocycle, we have:

    $$\operatorname{Hom}(\bigwedge^2G, A) \cong H^2(G,A)/H^2_{sym}(G,A)$$
  \end{enumerate}
\end{lemma}
One question that I have not resolved (but should probably be easy to
resolve) is:

\begin{quote}
  Does every alternating biadditive map occur as the skew of some (not
  necessarily IIP) $2$-cocycle?
\end{quote}

If so, that would make the isomorphism in (3) unconditional.
\subsection{Baer's solution}


Note that $\lambda$ itself is cyclicity-preserving since it is
alternating and biadditive. Moreover, it is skew symmetric, so $\Skew
\lambda = 2 \lambda$. This is off by a factor of $2$: we want $f$ such
that $\Skew f = \lambda$. Thus, if we could halve the $\lambda$, we
would have the desired $f$ whereby we could twist by $-f$.

This is the crux of Baer's idea: if $A$ is $(1/2)$-powered (in the
sense that there is a halving operation that commutes with the group
operations) then we can define $f(x,y) = \frac{1}{2}[x,y]$ in $A$. In
the big group $E$, this halving would be represented as a square root,
and we would get the formula $f(x,y) = \sqrt{[x,y]}$, giving Baer's
formula:

$$x + y := \frac{xy}{\sqrt{[x,y]}}$$

\subsection{Something special about Baer's approach}

Baer's approach has a number of nice features that will not be
replicated in all the other approaches that we'll see:

\begin{enumerate}
\item Baer actually manages to select a single canonical $2$-cocycle
  to twist by. Note that this canonical $2$-cocycle is not the only
  cyclicity-preserving $2$-cocycle to twist by, but it is still
  uniquely determined. Other $2$-cocycles can be used but are not as
  good.

  The uniqueness theorem that we have says that if we can find a nice
  $2$-cocycle to twist by, it is unique up to
  $2$-coboundaries. However, it is not unique in an absolute sense. In
  some of the generalizations that we consider, there are two mirror
  symmetric choices of $2$-cocycle that are interchanged via
  conjugation by one of the acting automorphisms.

\item Baer's $2$-cocycle is more than just a cyclicity-preserving
  $2$-cocycle: it is alternating, biadditive, skew symmetric,
  commutativity-preserving, subgroup-preserving, and much more. We
  will find that skew symmetry and the alternating property are often
  replicated in the $2$-cocycles we manage to find as well, but the
  other properties (biadditivity, commutativity-preserving,
  subgroup-preserving) often are not.
\end{enumerate}


\section{The situation with $2$-groups}

In this section, we explore in detail the various cases where $A$ and
$G$ are both $2$-groups.

\subsection{The case of $G$ elementary abelian}

When $G$ is elementary abelian, every cyclicity-preserving $2$-cocycle
is skew symmetric and takes values in the $4$-torsion of
$A$. Moreover, for $f$ a cyclicity-preserving $2$-cocycle, if $x$ and
$y$ are distinct, then $f(x,y)$ determines the value of $f$ for all
pairs of elements in $\langle x, y \rangle$, which are either equal to
$f(x,y)$ or equal to $-f(x,y)$. We state this formally:

\begin{lemma}\label{CP2cocycleonelab}
  Suppose $G$ is an elementary abelian $2$-group and $A$ is an abelian
  group with trivial action of $G$ on $A$. Then, the following are
  equivalent for a $2$-cocycle $f$:

  \begin{enumerate}
  \item $f$ is an IIP $2$-cocycle.
  \item $f$ is a cyclicity-preserving $2$-cocycle.
  \item $f$ is a skew-symmetric IIP $2$-cocycle.
  \item $f$ is a skew-symmetric cyclicity-preserving $2$-cocycle.
  \end{enumerate}

  Further, any $f$ satisfying these equivalent conditions also satisfies:

  \begin{enumerate}
  \item $$4f(x,y) = 0 \ \forall \ x,y \in G$$
  \item $$f(x,y) = f(y,xy) = f(xy,x) = -f(y,x) = -f(xy,y) = -f(x,xy) \ \forall \ x,y \in G$$
  \item $2f$ is alternating and biadditive.
  \end{enumerate}
\end{lemma}

An easy consequence of this is the following lemma:

\begin{lemma}
  Suppose $G$ is an elementary abelian $2$-group and $A$ is an abelian
  group with trivial action of $G$ on $A$. Then, the following are
  equivalent for a $2$-cocycle $f$:

  \begin{enumerate}
  \item $f$ is biadditive and IIP.
  \item $f$ is biadditive and cyclicity-preserving.
  \item $f$ is alternating and biadditive.
  \end{enumerate}

  Also, under these equivalent classes, $f$ is a symmetric
  cyclicity-preserving $2$-cocycle and hence (by the uniqueness theorem
  or otherwise) a $2$-coboundary.
\end{lemma}

We now discuss how the results obtained here affect our original problem. 

\begin{quote}
  For an alternating biadditive map $\lambda: G \times G \to A$, find
  a cyclicity-preserving $2$-cocycle $f$ such that $\Skew f =
  \lambda$?
\end{quote}

Since, by lemma \ref{CP2cocycleonelab} in the previous section, $f$
must be skew symmetric, we get that $\Skew f = 2f$. Thus, in the
context of elementary abelian $2$-groups, the question becomes:

\begin{quote}
  For an alternating biadditive map $\lambda:G \times G \to A$, does
  there exist a skew-symmetric cyclicity-preserving $2$-cocycle $f$
  such that $2f = \lambda$?
\end{quote}

\subsection{The Klein four-group}

We now restrict attention to the case where $G$ is a Klein
four-group. Before proceeding further, we note that in order to be
able to tackle the general case of elementary abelian $2$-groups, it
is important to first tackle the case of the Klein four-group because
any elementary abelian $2$-group contains a bunch of Klein
four-groups.

First, what are the possible alternating biadditive maps on the Klein
four-group? Alternating biadditive maps on $G$ are equivalent to
homomorphisms from $\bigwedge^2(G)$. When $G$ is the Klein four-group,
$\bigwedge^2(G)$ is cyclic of order $2$. The only possible
homomorphisms out from this are the trivial homomorphism and a mapping
that sends the non-identity element to an element of order $2$.

In the latter case, successfully {\em halving} the $2$-cocycle
requires the image element of order $2$ to itself be double of
something. It turns out that this is sufficient. We state this with
two back-to-back lemmas.

\begin{lemma}\label{KleinfourCPcocycleclassification}
  Suppose $G$ is a Klein four-group generated by $x$ and $y$. Let $A$
  be any abelian group. Let $F$ be the subgroup of $A$ comprising
  those elements whose order divides $4$. Then consider the map $\chi$
  from $F$ to the set of functions $G \times G \to A$ where $\chi(t)$
  is the $2$-cocycle $f$ given by:

  $$f(x,y) = f(y,xy) = f(xy,x) = t$$

  and:

  $$f(y,x) = f(xy,y) = f(x,xy) = -t$$

  Finally, we define $f$ to be $0$ whenever both its inputs are
  identical or either of its inputs is $0$.

  Then, the following are true:

  \begin{enumerate}
  \item $\chi$ is an isomorphism from $F$ to $C^2_{CP}(G,A)$. In
    particular, $\chi(t)$ is a cyclicity-preserving $2$-cocycle for
    every $t$ and $\chi(t + u) = \chi(t) + \chi(u)$.
  \item $\chi(t)$ is the {\em unique} cyclicity-preserving $2$-coycle
    $f$ satisfying $f(x,y) = t$.
  \item Let $K$ be the subgroup of $F$ comprising the elements whose
    order divides $2$. Then, for any $t \in K$, $\chi(t)$ is a
    symmetric cyclicity-preserving $2$-cocycle and hence a
    $2$-coboundary.
  \item The image of $K$ coincides with the group of alternating
    biadditive maps from $G$ to $A$.
  \item $K$ is {\em precisely} the subgroup of $F$ whose image under
    $\chi$ is in the subgroup of $2$-coboundaries. Thus, $\chi$
    induces an isomorphism from $F/K$ to $H^2_{CP}(G,A)$.
  \end{enumerate}

  In the $p$-group jargon, if $A$ is a $2$-group, we would denote $F$
  as $\Omega_2(A)$ and $K$ as $\Omega_1(A)$. Thus, in the $p$-group
  jargon, $C^2_{CP}(G,A) \cong \Omega_2(A)$ and $H^2_{CP}(G,A) \cong
  \Omega_2(G,A)/\Omega_1(G,A)$.
\end{lemma}

An easy corollary of this is:

\begin{lemma}
  Suppose $G$ is a Klein four-group generated by distinct elements $x$
  and $y$. Suppose $\lambda$ is an alternating biadditive map from $G
  \times G$ to $A$. Suppose $\lambda(x,y) = a$ for some element $a \in
  A$. Then, the mapping $\chi$ of the previous lemma establishes a
  bijection between the sets:

  $$\{ b \in A | 2b = a \} \leftrightarrow \{ f \in C^2_{CP}(G,A) \mid \Skew f = \lambda \}$$

  where, by lemma \ref{CP2cocycleonelab}, all elements in
  $C^2_{CP}(G,A)$ are $2$-cocycles and hence $\Skew f = 2f$.
\end{lemma}

Finally, we can summarize the previous lemmas plus more in a
theorem:

\begin{theorem}\label{Kleinfoursummary}
  Suppose $G$ is a Klein four-group generated by distinct elements $x$
  and $y$. Let $F$ be the subgroup of $G$ comprising elements whose
  order divides $4$ (thus, $F = \Omega_2(A)$ when $A$ is a $2$-group),
  $K$ be the subgroup of $A$ comprising elements whose order divides
  $2$ (thus, $K = \Omega_1(A)$ when $A$ is a $2$-group) and $L$ be the
  subgroup $2F$ (thus, $L = \mho^1(\Omega_2(A))$ when $A$ is a
  $2$-group. Let $\chi$ be the mapping from $F$ to $C^2_{CP}(G,A)$
  described in lemma \ref{KleinfourCPcocycleclassification}. Then:

  \begin{enumerate}
  \item $$F \cong^{\chi} C^2_{CP}(G,A)$$

    In $2$-group jargon:

    $$\Omega_2(A) \cong^{\chi} C^2_{CP}(G,A)$$

  \item $$F/L \cong K \cong^{\chi} B^2_{CP}(G,A) \cong
    \operatorname{Hom}(\bigwedge^2G,A) \cong^{\Skew} H^2(G,A)/H^2_{sym}(G,A)$$

    In $2$-group jargon, $K$ is $\Omega_1(A)$.

  \item $$F/K \cong L \cong H^2_{CP}(G,A) \cong (H^2_{CP}(G,A) + H^2_{sym}(G,A))/H^2_{sym}(G,A)$$

    In $2$-group jargon, $F/K$ is $\Omega_2(A)/\Omega_1(A)$ and $L$ is
    $\mho^1(\Omega_2(A))$.
  \item $$K/L \cong H^2(G,A)/(H^2_{CP}(G,A) + H^2_{sym}(G,A))$$

    In $2$-group jargon, $K/L$ is $\Omega_1(A)/(\mho^1(\Omega_2(A)))$.
  \item In particular, $H^2(G,A) = H^2_{CP}(G,A) + H^2_{sym}(G,A)$ if
    and only if $K = L$, or, every element of order $2$ has a half. In
    other words, every central extension with base $A$ and quotient
    group $G$ can be untwisted to an abelian extension if and only if
    every element of $A$ having order $2$ is twice of something.
  \end{enumerate}
\end{theorem}

\subsection{Some examples with the Klein four-group and various $A$s}

Theorem \ref{Kleinfoursummary} completes the discussion for the case
where $G$ is the Klein four-group, but some concrete examples may
help. In these concrete examples, we actually look at the extension
groups. We also note the role of the $\operatorname{Aut}(G) \times
\operatorname{Aut}(A)$ group in permuting the possibilities.

In specific examples, we restrict attention to $A$ a $2$-group,
because the $2'$-part splits off and does not interact in any
interesting way with $G$.

In the case that $A$ is a cyclic $2$-group, we get the following:

\begin{enumerate}
\item $H^2(G,A)$ is an elementary abelian group of order $8$.
\item $H^2_{sym}(G,A)$ is a Klein four-group inside this elementary
  abelian group. Its identity element is $A \times G$. Its other
  three elements all have big group isomorphic to $\Z_{2^{m+1}}
  \times \Z_2$. These three non-identity elements form one orbit
  under the action of $\Aut(G)$ and hence also of $\Aut(G) \times
  \Aut(A)$.
\item $H^2(G,A)/H^2_{sym}(G,A)$ is cyclic of order $2$. The
  non-identity coset has $4$ elements. One of these elements has big
  group isomorphic to the central product $Q_8 *_{\Z_2} \Z_{2^m}$,
  which, for $m > 1$, is also $D_8 *_{\Z_2} \Z_{2^m}$. The three other
  elements are $D_8$ when $m = 1$ and $M_{2^{m+2}}$ when $m > 1$,
  where $M_{2^n}$ is the class two $2$-group with cyclic maximal
  subgroup. These three other elements are permuted by the action of
  $\Aut(G)$, and form a single orbit under $\Aut(G) \times \Aut(A)$.
\item In the case $m = 1$, $H^2_{CP}(G,A)$ is trivial, which is
  basically because $A$ does not have elements of order $4$.
\item For $m > 1$, $H^2_{CP}(G,A)$ is cyclic of order $2$, and its
  non-identity element corresponds to the group $Q_8 *_{\Z_2}
  \Z_{2^m}$. Thus, for $m > 1$, $H^2(G,A) = H^2_{CP}(G,A) +
  H^2_{sym}(G,A)$ so every group can be untwisted to an abelian
  group. We thus get two group correspondences: 
    
  $$Q_8 *_{\Z_2} \Z_{2^m} \to \Z_{2^m} \times \Z_2 \times \Z_2$$
  
  and
  
  $$M_{2^{m + 2}} \to \Z_{2^{m+1}} \times \Z_2$$
\item In all cases, $H^2_{CP}(G,A) \subseteq H^2_{Fix}(G,A)$. Equality
  holds for $m > 1$. Does this hold in general? Probably not. We
  consider this question later.
\end{enumerate}


\end{document}
