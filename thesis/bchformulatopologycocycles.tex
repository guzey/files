\documentclass[10pt]{amsart}

%Packages in use
\usepackage{fullpage, hyperref, vipul}

%Title details
\title{Very sketchy ideas on Baker-Campbell-Hausdorff, topology, cocycles}
\author{Vipul Naik}

%List of new commands
\newcommand{\Skew}{\operatorname{Skew}}

\begin{document}
\maketitle

The goal here is to consider the Baker-Campbell-Hausdorff formula, its
significance, and the way formulas of this sort operate.

It is not intended to provide theorems or results but rather to provide a roadmap for how to think.

Not for public circulation!

\subsection*{The Baker-Campbell-Hausdorff formula format}

The Baker-Campbell-Hausdorff formula is a formula of the form:

$$x * y := t_1(x,y) + t_2(x,y) + t_3(x,y) + \dots + t_n(x,y) + \dots$$

Here, we have:

\begin{eqnarray*}
  t_1(x,y) & = & x + y \\
  t_2(x,y) & = & [x,y] \\
  t_3(x,y) & = & \frac{1}{12}[x,[x,y]] - \frac{1}{12}[y,[x,y]]
\end{eqnarray*}

etc. where $[ , ]$ denotes the Lie bracket.

If the Lie algebra structure arises as the commutator map of an
associative algebra, then $[x,y] = xy - yx$.

We note three interesting perspectives:

\begin{enumerate}
\item The {\em continuous norm} perspective that makes sense for real
  and complex structures: This states something roughly along the
  lines of there existing a norm structure on the Lie algebra that is
  sub-additive and sub-multiplicative, i.e., $\| x + y \| \le \| x \|
  + \| y \|$ and $\| [x,y] \| \le \| x \| \| y \|$. Note that a norm
  for an associative operation can be tweaked to get a norm for the
  corresponding Lie algebra operation.

  For instance, the matrix algebra of $m \times m$ matrices over the
  real numbers has a norm defined as $m$ times the {\em maximum} of
  the matrix entries. This is sub-additive and sub-multiplicative. For
  the corresponding Lie algebra, we define the norm as $2m$ times the
  maximum of the matrix entries. This takes care of the fact that we
  are adding $xy$ and $-yx$.
  
  In the norm perspective, $t_n$ is small because it is a sum with
  constant coefficients of $n$ times iterated Lie brackets, which in
  turn are small because of the sub-multiplicativity of the norm. If
  $x,y$ are $O(\epsilon)$, then $t_n(x,y)$ is $O(\epsilon^n)$, where
  the coefficient on the $O$ depends on $n$ but is independent of
  $\epsilon$.

  Note that in this perspective, the nilpotency does not have much to
  do with the terms becoming smaller. Rather, the terms are becoming
  smaller because of the products whose commutator is the Lie bracket.

  Note also that this notion of small is a {\em relative} notion: it
  says that if $x$ and $y$ are small, then $t_n(x,y)$ is
  correspondingly smaller. The $t_n$s do not get small in an
  absolute or uniform sense from a norm perspective.

\item The {\em nilpotency} perspective that makes sense across the
  board: Here, the $t_n$ are {\em small} in the sense that they lie in
  the $n^{th}$ member of the lower central series.

  This is both an absolute and relative notion. As an absolute notion,
  it says that the entire image of $t_n$ is ``small'' in the sense of
  being contained in the $n^{th}$ member of the lower central series.

  In relative terms, it says that $t_n(x,y)$ is ``small'' in the sense
  of being contained in the $n^{th}$ member of the lower central
  series of the subalgebra generated by $x$ and $y$.

  The notion is useful as long as the lower central series is strictly
  descending.

\item The {\em $p$-adic norm} perspective: Unlike the real or complex
  norm, $p$-adic norm notions are inherently discrete and not
  continuous. What we hope is that the $t_n(x,y)$ get small in the
  sense that they have large powers of $p$ dividing them.

  Unfortunately, this is not the case. In fact, the opposite may be
  the case, since we end up dividing by powers of $p$ and thus
  depleting the $p$-adic norm.
\end{itemize}

\end{document}
