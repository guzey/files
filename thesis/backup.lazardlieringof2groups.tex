\documentclass[10pt]{amsart}

%Packages in use
\usepackage{fullpage, hyperref, vipul, amssymb}

%Title details
\title{Lazard Lie rings of 2-groups}
\author{Vipul Naik}

%List of new commands

\makeindex

\begin{document}
\maketitle
%\tableofcontents

\begin{abstract}
  The theory of Lazard Lie rings of $p$-groups makes some assumptions
  about the nilpotency class of the group (or of important subgroups)
  being smaller than $p$. Here, we consider some techniques that can
  be used to extend this theory to more $p$-groups. In particular, we
  consider the case of $2$-groups of class two.
\end{abstract}

\section{Lazard Lie rings for odd-order class two groups}

\subsection{The construction}

In this note, when we say that a group (or Lie ring) has nilpotency
class $c$, we mean that the nilpotency class of the group is at most
$c$. This convention saves space while allowing us to claim that the
property of having class $c$ is closed under taking subgroups and
quotients.

Suppose $p$ is an odd prime and $P$ is a finite $p$-group of
nilpotency class two. The Lazard Lie ring of $P$ is defined, as a set,
as equal to $P$, with the addition defined as:

$$x + y := \frac{xy}{\sqrt{[x,y]}}$$

and the Lie bracket $[x,y]$ coincides with the commutator $[x,y]$
(hence, we use the same notation for both). Conversely, given a Lie
ring $L$ of class two, we can define a corresponding Lazard Lie group
with the same underlying set as $L$:

$$xy := x + y + \frac{1}{2}[x,y]$$

The expressions are more complicated for class greater than $2$. In
particular, for class greater than $2$, the Lie bracket of the Lie
ring does not coincide with the commutator in the group. In this
document, we concentrate on class two.

\subsection{Proof for going from groups to Lazard Lie rings}

We state the result as a theorem. (It's already been proved earlier,
but the proof is explicitly given here to make it easier to see how
later proofs arise by modifying it).

Note that for groups of nilpotency class two, it does not matter
whether we use the left convention $[x,y] = xyx^{-1}y^{-1}$ or the
right convention $[x,y] = x^{-1}y^{-1}xy$, because they both take the
same value.

\begin{theorem}
  Suppose $p$ is an odd prime and $P$ is a finite $p$-group of
  nilpotency class two. Then, $P$ has the structure of a Lie ring of
  nilpotency class two with the following operations:

  \begin{enumerate}
  \item The $0$ element is the multiplicative identity of $P$.
  \item The additive inverse operation is the same as the
    multiplicative inverse operation.
  \item The addition is defined as:

    $$x + y := \frac{xy}{\sqrt{[x,y]}}$$

    where $\sqrt{[x,y]}$ is the unique element whose square is
    $[x,y]$, and is also given as $[x,y]^{(p^k + 1)/2}$ where $p^k$ is
    the exponent of the group.
  \item The Lie bracket $[x,y]$ is defined as the commutator $[x,y]$.
  \end{enumerate}
\end{theorem}

\begin{proof}
  First, note that since $P$ has nilpotency class two, $[x,y]$ is
  central and hence $\sqrt{[x,y]} = [x,y]^{(p^k+1)/2}$ is also
  central. Thus, the fraction notation makes sense.

  We show that the operation $+$ as defined in (3) is associative. We have:

  $$(x + y) + z = \frac{\frac{xy}{\sqrt{[x,y]}} \cdot z}{\sqrt{\left[\frac{xy}{\sqrt{[x,y]}},z\right]}}$$

  Noting that central elements can be eliminated from inside
  commutators without affecting the value, we simplify to:

  $$(x + y) + z = \frac{xyz}{\sqrt{[x,y]}\sqrt{[xy,z]}}$$

  Since the $\sqrt{}$ operation is just a raising to the power of $(p^k
  + 1)/2$ operation, it is a homomorphism on the center and we can thus write:

  $$(x + y) + z = \frac{xyz}{\sqrt{[x,y][xy,z]}}$$

  In a similar manner, we can deduce that:

  $$x + (y + z) = \frac{xyz}{\sqrt{[x,yz][y,z]}}$$

  Thus, it suffices to prove that:

  $$[x,y][xy,z] = [x,yz][y,z]$$

  But since the commutator map is a bihomomorphism for groups of class
  two, both sides become $[x,y][x,z][y,z]$, so the equality holds.

  This proves that the operation defined in (3) is associative.

  It is also clear that the operation agrees with the group
  multiplication whenever $x$ and $y$ commute in the group. In
  particular, the identity element acts as zero and multiplicative
  inverses act as additive inverses.

  We next show that $+$ is commutative. For this, note that:

  $$x + y = \frac{xy}{\sqrt{[x,y]}}$$

  and

  $$y + x = \frac{yx}{\sqrt{[y,x]}}$$

  It thus suffices to show that:

  $$\frac{xy}{\sqrt{[x,y]}} = \frac{yx}{\sqrt{[y,x]}}$$

  which is equivalent to:

  $$\frac{xy}{yx} = \frac{\sqrt{[x,y]}}{\sqrt{[y,x]}}$$

  The left side is $[x,y]$. The right side is also $[x,y]$, which
  follows immediately from the definition and the fact that $[y,x] =
  [x,y]^{-1}$.

  We have thus shown that $P$ is an abelian group under $+$ and have
  demonstrated (1)-(3). Next, we need to argue that the commutator map
  defines a Lie bracket on $P$.
  Note that:

  $$[x+y,z] = [\frac{xy}{\sqrt{[x,y]}},z] = [xy,z] = [x,z][y,z]$$
  
  Also:

  $$[x,z] + [y,z] = \frac{[x,z][y,z]}{\sqrt{[[x,z],[y,z]]}} = [x,z][y,z]$$

  This proves that the map is linear in the first variable. A similar
  proof shows linearity in the second variable. Alternation follows
  from the fact that $[x,x]$ is the identity element back in the
  group.

  The Jacobi identity holds vacuously since any commutator
  $[[x,y],z]$ is the identity element. We thus need to check
  bilinearity and alternation.

  Also, since $[[x,y],z]$ is the identity, the Lie ring has nilpotency
  class two.
\end{proof}

\subsection{Proof for going from Lie rings to groups}

\begin{theorem}
  Suppose $p$ is an odd prime and $L$ is a Lie ring of order a power
  of $p$ and nilpotency class two. Then, $L$ has the structure of a
  group of nilpotency class two with the following operations:

  \begin{enumerate}
  \item The identity element is the $0$ element of $L$.
  \item The multiplicative inverse operation is the same as the
    additive inverse operation.
  \item The multiplication is defined by:

    $$xy := x + y + \frac{1}{2}[x,y]$$

  \item The commutator $[x,y]$ is defined as the Lie bracket $[x,y]$.
  \end{enumerate}
\end{theorem}

\begin{proof}
  The two substantive parts of the proof are showing associativity and
  showing that the commutator agrees with the Lie bracket.

  For associativity:

  $$(xy)z = \left(x + y + \frac{1}{2}[x,y]\right) + z + \frac{1}{2}\left[x + y + \frac{1}{2}[x,y],z\right]$$

  This simplifies to:

  $$(xy)z = x + y + z + \frac{1}{2}\left([x,y] + [x,z] + [y,z]\right)$$

  Similarly, we can show that:

  $$x(yz) = x + y + z + \frac{1}{2}\left([x,y] + [x,z] + [y,z]\right)$$

  Thus, associativity holds.

  The stuff about additive inverses being multiplicative inverses and
  identity being $0$ is easily verified.

  We next want to show that the commutator with this definition of
  multiplication is the same as the Lie bracket. The multiplicative
  commutator is $[x,y] = (xy)(yx)^{-1}$ which is $(xy)$ times $(-yx)$,
  which is:

  $$xy + (-yx) + \frac{1}{2}[xy,-yx]$$

  which simplifies to:

  $$x + y + \frac{1}{2}[x,y] - (y + x + \frac{1}{2}[y,x]) + \frac{1}{2}[x + y + \frac{1}{2}[x,y],-(y + x + \frac{1}{2}[y,x])]$$

  Simplifying, we are left with the Lie bracket $[x,y]$.
\end{proof}
\subsection{Important things to note}

Some important things to note about the constructions described above:

\begin{enumerate}
\item The $\sqrt{[x,y]}$ symbol is justified because $p$ is odd,
  because in a finite $p$-group, there is a unique square root, and is
  the $(p^k+1)/2^{th}$ power of $[x,y]$ where $p^k$ is the exponent of
  the group.
\item The fraction notation has been used for simplification and it is
  justified because the element $[x,y]$ (and hence its square root) is
  central.
\end{enumerate}

\subsection{Properties of the Lazard correspondence}

The Lazard correspondence has the following properties (proofs not
here, but are to be found in any standard reference on the subject
(give refs)):

\begin{enumerate}
\item Any automorphism of a group is an automorphism of its Lazard Lie
  ring as a Lie ring, and vice versa.
\item Any endomorphism of a group is an endomorphism of its Lazard Lie
  ring as a Lie ring, and vice versa.
\item Subgroups of the group correspond to Lie subrings of the Lazard Lie
  ring, and vice versa.
\item Normal subgroups of the group correspond to Lie ideals of the
  Lazard Lie ring, and vice versa.
\item Abelian subgroups of the group correspond to abelian subrings
  (i.e., trivial Lie bracket) of the Lazard Lie ring, and vice
  versa. Moreover, the additive structure on the abelian subring is
  exactly the same as the group operation on the abelian subgroup.
\item In particular, cyclic subgroups of the group correspond to
  cyclic subrings of the Lie ring. Thus, in particular, the order
  statistics for a group are the same as those of the additive group
  of its Lazard Lie ring.
\end{enumerate}

\section{Moving to $2$-groups}

\subsection{The basic goal}

The goal is to determine the conditions under which the above
construction can be extended to the case of $2$-groups of class
two. Let us look once again at the formula:

$$x + y := \frac{xy}{\sqrt{[x,y]}}$$

There are two problems with this:

\begin{enumerate}
\item It is not necessary that we have a square root of $[x,y]$ in the
  center of $P$.
\item Even if square roots exist for all elements of $[P,P]$, the
  non-uniqueness of the square root creates a problem of choice.
\end{enumerate}

\subsection{Existence of square roots and ensuring associativity}

A mathematical formulation for the existence of square roots in the
{\em center} is that $[P,P] \le \mho^1(Z(P))$. This is a somewhat
stronger condition than being of class two, and it eliminates many
groups of class two. For instance, it eliminates the special and
extraspecial groups and many others.

However, it is not obvious that this condition is not sufficient. We
want our choice of square root to be such that the $+$ operation
defined using it is associative.

Let us denote by $f(x,y)$ this {\em chosen} square root. Thus, $f$ is a
function:

$$f:P \times P \to Z(P)$$

with the property that $(f(x,y))^2 =[x,y]$. Switching to additive
notation in $Z(P)$, we get $2f(x,y) = [x,y]$. Also, $f$ should be
constant on the cosets of $Z(P)$, so it descends to a map:

$$\overline{f}:P/Z(P) \times P/Z(P) \to Z(P)$$

Working out the associativity condition, for $+$, we obtain that:

$$f(x,y) + f(xy,z) = f(x,yz) + f(y,z)$$

Thus, $\overline{f}$ is a 2-cocycle from $P/Z(P)$ to $Z(P)$ for the
trivial action, with the additional property that $2f$ is the
commutator cocycle form $P/Z(P)$ to $Z(P)$.

In particular, the set of possible values of $f$, if nonempty, is a
coset of the set of elements of order two in the group of 2-cocycles.

\subsection{The conditions: groups to Lie rings}

We list here many important conditions on $f: P \times P \to Z(P)$
such that $f$ is constant on the cosets of $Z(P)$ and hence descends
to a map $\overline{f}:P/Z(P) \times P/Z(P) \to Z(P)$:

\begin{enumerate}
\item The {\em cocycle condition}: This is the condition that
  $\overline{f}$ be a cocycle for the trivial action of $P/Z(P)$ on
  $Z(P)$ for the trivial action.
\item The {\em skew equals commutator condition}: This is the
  condition that $f(x,y)(f(y,x))^{-1} = [x,y]$ for all $x$ and $y$.
\item The {\em identity-preservation condition}: This is the condition
  that if either of the inputs to $f$ is the identity element, the
  output of $f$ is the identity element.
\item The {\em inverse-preservation condition}: This is the condition
  that if the two inputs to $f$ are inverses of each other, the output
  of $f$ is the identity element.
\item The {\em half-commutator condition}: This is the condition that
  $2f$ be the commutator map.
\item The {\em skew-symmetry condition}: This states that $f(y,x) =
  -f(x,y)$.
\item The {\em cyclicity-preserving condition}: This is the condition
  that $f(x,y) = 0$ if $x$ and $y$ are both elements of a cyclic subgroup.
\item The {\em commutativity-preserving condition}: This is the
  condition that if $[x,y] = 0$, then $f(x,y) = 0$. Conditional to
  (2), this states that if $2f(x,y) = 0$, we also have $f(x,y) = 0$.
\item The {\em bilinearity condition}: This states that $f$ is
  bilinear.
\item The {\em strong congruence condition}: This states that if $x$
  and $y$ commute, then $f(x,z) = f(x,yz)$ for all $z$.
\item The {\em subgroup-closure condition}: This states that
  restricting $f$ to any subgroup of $P$ gives a map to within that
  subgroup. (Here, we think of $f$ as a map on elements rather than
  cosets, which we can do).
\item The {\em quotient-closure condition}: This states that $f$
  descends to any quotient.
\item The {\em subquotient-closure condition}: This states that $f$
  restricts-cum-descends to any subquotient.
\end{enumerate}

Note that {\em all} these conditions (or rather, their analogues) are
satisfied in the case when $p$ is an odd prime, $P$ is a $p$-group of
class two, and $f$ is the square root of the commutator.

Also worth noting are the following: cyclicity-preservation (condition
(7)) implies both identity-preservation (condition (3)) and
inverse-preservation (condition (4)). Also, conditions (5) and (6)
together imply condition (2)

On the other hand, all these conditions {\em cannot} be simultaneously
true in the case of non-abelian 2-groups. The problem arises from the
incompatibility of (5), (8), and (9). (5) and (8) imply that the image
of $f$ does not contain any non-identity elements of order $2$. If we
also assume condition (9), then the image of $f$ cannot contain any
non-identity elements, because if $f(x,y)$ is a non-identity element
of order $2^k$, we have that $f(x^{2^{k-1}},y)$ is an element of order
$2$.

We will encounter situations where we sacrifice (9) and have all the
other conditions.

Note also the implications: condition (10) implies condition (8)
implies condition (7), and condition (13) implies conditions (12) and
(11).

\subsection{The theorems: groups to Lie rings}

\begin{theorem}\label{lazardgrouptoring1234}
  Suppose $P$ is a finite $p$-group (including possibly $p = 2$) and
  $f$ is a $2$-cocycle for the trivial action of $P/Z(P)$ on $Z(P)$
  satisfying conditions (1)-(4) above (i.e., $f$ is a $2$-cocycle
  whose skew is the commutator, it is identity-preserving, and it is
  inverse-preserving). Then, we can define a Lie ring structure on $P$
  as follows, giving a Lie ring of nilpotency class two:

  \begin{enumerate}
  \item The zero element is the identity element of $P$.
  \item The additive inverse is the same as the multiplicative inverse
    in $P$.
  \item The addition is defined as:

    $$x + y := \frac{xy}{f(x,y)}$$
  \item The Lie bracket is defined as the commutator in $P$.
  \end{enumerate}
\end{theorem}

Before we go into the proof, note that the Lie ring structure here
depends on the choice of $f$. This means that the statement of the
theorem allows for different Lie ring structures to emerge from
different choices of $f$. It may turn out later that some of these Lie
rings are isomorphic. The possibility of multiple Lie rings of a group
is something we shall explore later.

\begin{proof}
  Here is a broad sketch of the proof:
  \begin{enumerate}
  \item The cocycle condition on $f$ yields the associativity of $+$.
  \item The identity-preserving and inverse-preserving conditions
  shows (1) and (2).
  \item The skew is commutator condition yields the commutativity of
    $+$.
  \item The statements about the Lie bracket follow directly form the
    properties of class two.
  \end{enumerate}

  Let's execute these steps one by one.

  {\em Associativity of $+$}: Consider:

  $$(x + y) + z = \frac{\frac{xy}{f(x,y)} \cdot z}{f\left(\frac{xy}{f(x,y)},z\right)}$$

  Note that since $f(x,y) \in Z(P)$ and $f$ depends only on the cosets
  with respect to $Z(P)$ of its elements, $f(\frac{xy}{f(x,y)},z) =
  f(xy,z)$. Thus, the above expression simplifies to:

  $$(x + y) + z = \frac{xyz}{f(x,y)f(xy,z)}$$

  Similarly, we obtain:

  $$x + (y + z) = \frac{xyz}{f(x,yz)f(y,z)}$$

  The equality of the two expressions depends on the equality:

  $$f(x,y)f(xy,z) = f(x,yz)f(y,z)$$

  which, in the additive notation, is precisely the 2-cocycle
  condition.

  {\em The agreement of identity and inverses}: $f$ is
  identity-preserving. Thus, if either of the inputs to $f$ is the
  identity, $f$ takes the identity. This shows that $x + id = id + x =
  x$. Also, since $f(x,x^{-1})$ is the identity, $x + x^{-1} =
  xx^{-1}/id = xx^{-1} = id$.

  We now move to commutativity:

  {\em Commutativity of $+$}: We have:

  $$x + y := \frac{xy}{f(x,y)}$$

  We also have:

  $$y + x := \frac{yx}{f(y,x)}$$

  Equality of these requires that:

  $$\frac{f(x,y)}{f(y,x)} = [x,y]$$

  Or, in additive notation:

  $$f(x,y) - f(y,x) = [x,y]$$

  This follows from the skew-symmetry condition on $f$ and the fact
  that $2f$ is the commutator map.

  We finally turn to statements about the Lie bracket. We want to
  argue that $[x,y]$ is a well-defined Lie bracket. In particular, we
  want to show that $[x + y,z] = [x,z] + [y,z]$. We begin with the left side.

  $$[x + y,z] = [\frac{xy}{f(x,y)},z] = [xy,z] = [x,z][y,z]$$

  Here, we use the fact that since $f(x,y) \in Z(P)$, it can be
  removed from inside the commutator. Later, we use nilpotency class two.

  Also:

  $$[x,z] + [y,z] = \frac{[x,z][y,z]}{f([x,z],[y,z])} = [x,z][y,z]$$

  The denominator is the identity because both $[x,z]$ and $[y,z]$ are
  in the center.

  A similar proof works for right linearity. The alternation property
  follows from the fact that $[x,x]$ is the identity element when the
  commutator is viewed in the group.

  Finally the Jacobi identity holds vacuously since class two forces
  any commutator $[[x,y],z] = 0$. This also forces the Lie ring to
  have nilpotency class two.
\end{proof}

\subsection{Groups to Lie rings: notation and results}

We will denote, for a group $P$ and a 2-cocycle $f$ satisfying the
conditions of the preceding theorem, the corresponding Lie ring as:

$$\operatorname{Lie}(P,f)$$

For practical purposes, we try to find a function $f$ that is somewhat
more specific than the $f$ outlined in the above theorem. We require
$f$ to satisfy condition (1) and conditions (5)--(7), i.e., it is a
half-commutator, it is skew-symmetric, and it is cyclicity-preserving.

For small groups, it turns out that these conditions are no more
stringent. For instance, if $P/Z(P)$ is elementary abelian,
identity-preserving and inverse-preserving together are equivalent to
being cyclicity-preserving. Further, in many cases, a cocycle $f$
satisfying conditions (1)--(4) can be modified somewhat to also
satisfy (5)--(7).

To avoid having to repeatedly list conditions, we abbreviate (1)--(4)
as the $B$-level system and conditions (1),(5)--(7) as the $C$-level
system. A cocycle $f$ satisfying (1)--(4) will be termed an $B$-level
cocycle and the corresponding Lie ring will be termed an $B$-level Lie
ring. Similarly for $C$-level. Note that any $C$-level cocycle is
automatically $B$-level. (This is just temporary notation. I hope to
come up with something better once the conceptual understanding improves).

Here now is a theorem listing some important fallouts.

\begin{theorem}\label{lazardgrouptoringmore}
  Suppose $P$ is a finite $p$-group (including possibly, and of most
  interest, the case $p = 2$) and $f$ is a $B$-level cocycle for
  $P$. Use the Lie ring construction of the previous theorem.

  \begin{enumerate}
  \item Assume condition (7). Then, $P$ and the additive group of its
    Lazard Lie ring are $1$-isomorphic: the bijection between them
    restricts to an isomorphism on cyclic subgroups.
  \item Assume condition (8) (commutativity-preservation). Then, two
    elements commute in the multiplicative group of $P$ if and only if
    their Lie bracket is zero, which in turn happens if and only if
    their sum in the Lie ring equals their product in the group.
  \item The subgroup-closure condition (numbered (8)): the statement
    that $f$ is closed on subgroups guarantees that every subgroup of
    $P$ is a Lie subring. Moreover, this Lie subring is precisely the
    Lie ring for the subgroup using the restriction of $f$.
  \item The subgroup-closure condition {\em or} the quotient-closure
    condition also guarantees that every normal subgroup of $P$ is an
    ideal in the corresponding Lie ring structure, and further, that
    the normalizer of any subgroup of $P$ equals its idealizer as a
    Lie subring.
  \end{enumerate}
\end{theorem}

\begin{proof}
  All these statements are directly proved from the definitions. I'll
  fill in details later.
\end{proof}

\subsection{The conditions: Lie rings to groups}

Let $L$ be a Lie ring and $f:L \times L \to L$ be a map that desends
to a map $\overline{f}: L/Z(L) \times L/Z(L) \to Z(L)$. We list here
many important conditions on $f$:

\begin{enumerate}
\item The {\em cocycle condition}: This is the condition that $f$
  descends to a cocycle for the trivial action of $L/Z(L)$ on $Z(L)$.
\item The {\em skew equals commutator condition}: This is the
  condition that $f(x,y)(f(y,x))^{-1} = [x,y]$ for all $x$ and $y$.
\item The {\em identity-preservation condition}: This is the condition
  that if either of the inputs to $f$ is the identity element, the
  output of $f$ is the identity element.
\item The {\em inverse-preservation condition}: This is the condition
  that if the two inputs to $f$ are inverses of each other, the output
  of $f$ is the identity element.
\item The {\em half-Lie bracket condition}: This is the condition that
  $2f$ be the Lie bracket.
\item The {\em skew-symmetry condition}: This states that $f(y,x) =
  -f(x,y)$.
\item The {\em cyclicity-preserving condition}: This is the condition
  that $f(x,y) = 0$ if $x$ and $y$ are both elements of a common
  cyclic subgroup.
\item The {\em commutativity-preserving condition}: This is the
  condition that if $[x,y] = 0$, then $f(x,y) = 0$. Combined with (2),
  this states that if $2f(x,y) = 0$, then $f(x,y) = 0$.
\item The {\em bilinearity condition}: This states that $f$ is
  bilinear.
\item The {\em strong congruence condition}: This states that if
  $[x,y] = 0$, then $f(x,z) = f(x,yz)$ for all $z$.
\item The {\em subring-closure condition}: This states that
  restricting $f$ to any subgroup of $P$ gives a map to within that
  subgroup. (Here, we think of $f$ as a map on elements rather than
  cosets, which we can do).
\item The {\em quotient-closure condition}: This states that $f$
  descends to any quotient.
\item The {\em subquotient-closure condition}: This states that $f$
  restricts-cum-descends to any subquotient.
\end{enumerate}

The observations are the same as before.

\subsection{The theorems: Lie rings to groups}

\begin{theorem}\label{lazardringtogroup1234}
  $L$ is a Lie ring of order a power of $2$ and nilpotency class
  two. Suppose $f$ is a $2$-cocycle from $L/Z(L)$ to $Z(L)$ satisfying
  conditions (1)-(4). Then, $L$ has the structure of a group of
  nilpotency class two with the following operations:

  \begin{enumerate}
  \item The identity element is the $0$ element of $L$.
  \item The multiplicative inverse operation is the same as the
    additive inverse operation.
  \item The multiplication is defined by:

    $$xy := x + y + f(x,y)$$

  \item The commutator $[x,y]$ is defined as the Lie bracket $[x,y]$.
  \end{enumerate}
\end{theorem}

\begin{proof}
  Here is a broad sketch of the proof:
  \begin{enumerate}
  \item The cocycle condition on $f$ yields the associativity of
    multiplication.
  \item The identity-preserving and inverse-preserving conditions shows
    (1) and (2).
  \item The skew is Lie bracket condition and the inverse-preserving
    condition shows that the commutator agrees with the Lie bracket.
  \item The statements about the Lie bracket follow direcly form the
    properties of class two.
  \end{enumerate}

  Let's execute these steps:

  {\em Associativity of group multiplication}:

  We have:

  $$(xy)z = x + y + f(x,y) + z + f(x + y + f(x,y),z)$$

  Since $f$ descends to the cosets of $Z(L)$, $f(x + y + f(x,y),z) =
  f(x + y,z)$, and we obtain:

  $$(xy)z = x + y + z + f(x,y) + f(x + y,z)$$

  Similarly, we obtain:

  $$x(yz) = x + y + z + f(x,y + z) + f(y,z)$$

  The cocycle condition tells us that:

  $$f(x,y) + f(x + y,z) = f(x,y + z) + f(y,z)$$

  This yields associativity.

  {\em Agreement of identity and inverses}: From
  identity-preservation, we see that $f(x,0) = f(0,x) = 0$ and
  $f(x,-x) = 0$, so $x\cdot 0 = x + 0 + f(x,0) = x$, and similarly, $0
  \cdot x = 0$. Also, by inverse-preservation, we have $x \cdot (-x) =
  0$.

  {\em Agreement of commutator and Lie bracket}: The commutator is
  $(xy)(yx)^{-1}$ which is $(xy)$ times $(-yx)$. We simplify and obtain:

  $$(x + y + f(x,y)) - (y + x + f(y,x)) + f(x + y + f(x,y),-(y + x + f(y,x)))$$

  We simplify to:

  $$f(x,y) - f(y,x) + f(x + y + f(x,y),-(y + x + f(y,x))$$

  Going to cosets of the center, we note that $f(x + y + f(x,y),-(y +
  x + f(y,x)) = f(x + y, -(y + x)) = 0$ by cyclicity-preservation. We
  thus get $f(x,y) - f(y,x)$, which, by the skew is Lie bracket
  condition, is $[x,y]$.

  The statements about class two now follow directly.

\end{proof}

\section{$M_{16}$: a case study}

\subsection{The Lie ring and the correspondence}

In order to get a good hands-on feel of what the Lazard Lie rings look
like for $2$-groups, we consider in detail the group $M_{16}$, given
by the presentation (I use $e$ to denote the identity element):

$$M_{16} := \langle a, x \mid a^8 = x^2 = e, xax = a^5 \rangle$$

In commutator notation, this is:

$$M_{16} := \langle a,x \mid a^8 = x^2 = e, [a,x] = a^4 \rangle$$

Let $P = M_{16}$. $P$ is a group of nilpotency class two with the
property that every proper subgroup and every proper quotient is
abelian. (hence it is a minimal non-abelian group). The derived
subgroup $\langle a^4 \rangle$ is cyclic of order two. The center
$\langle a^2 \rangle$ is cyclic of order four. The quotient by the
center is elementary abelian and is generated by the images of $a$ and
$x$.

The commutator map, viewed as a map from $P/Z(P) \times P/Z(P) \to
Z(P)$, is:

$$[a,a] = e, [x,x] = [ax,ax] = e, [a,x] = a^4, [x,a] = [ax,x] = [x,ax] = [ax,a] = [a,ax] = a^4$$

The map $f$ must double to this map. The cyclicity-preserving
condition on $f$ forces $f(a,a) = f(x,x) = f(ax,ax) = e$. Further, we have
$f(x,a) = f(a,x)^{-1}$ and two similar relations. We also have, by the
cocycle condition, that:

$$f(a,x)f(ax,x) = f(a,x^2)f(x,x)$$

Both terms on the right side becomes zero, so we obtain that:

$$f(ax,x) = f(a,x)^{-1}$$

Similarly, we obtain that:

$$f(a,a)f(a^2,x) = f(a,ax)f(a,x)$$

Thus yielding that $f(a,ax) = f(a,x)^{-1}$. Thus, the only choice we
have is for $f(a,x)$. Note that this general procedure can be mimicked
whenever $P/Z(P)$ is a Klein four-group.

We choose $f(a,x) = a^2$. Note that the other choice -- $a^6$ -- in
this case gives an isomorphic Lie ring, something we shall return to
later.

We thus obtain:

$$f(a,x) = f(ax,a) = f(x,ax) = a^2, \qquad f(x,a) = f(a,ax) = f(ax,x) = a^6$$

We can use this to work out the general addition rules:

\begin{eqnarray*}
  a^k + a^l & = & a^{k+l}\\
  a^{2k} + a^lx & = & a^{2k+l}x\\
  a^{2k+1} + a^{2l}x & = & a^{2k + 2l-1}x\\
  a^{2k+1} + a^{2l+1}x & = & a^{2k + 2l + 4}x\\
  a^{2k}x + a^{2l}x & = & a^{2k + 2l}\\
  a^{2k+1}x + a^{2l}x & = & a^{2k+2l+3}\\
  a^{2k+1}x + a^{2l+1}x & = & a^{2k+2l+6}
\end{eqnarray*}

Since these multiplication rules seem a little unintuitive, here is a
different way of thinking of the Lie ring that is computationally
easier. 

We define the Lie ring additively as $\Z_8 \oplus \Z_2$, with the Lie
bracket given by $[(1,0),(0,1)] = (4,0)$. The group elements are
identified with the Lie ring as follows: $a^k = (k,0)$, $a^{2k}x =
(2k,1)$, and $a^{2k+1}x = (2k+3,1)$.

Alternatively, the Lie ring presentation is:

$$\langle a,x \mid 8a = 2x = 0, [a,x] = 4a \rangle$$

\subsection{The mirror Lie ring}

Notice that we made a choice in setting $f(a,x) = a^2$ and we could
have chosen $f(a,x) = a^6$. This alternative choice would have yielded
an alternative construction. The Lie ring would turn out to be
isomorphic to the original one, but the correspondence would be:

$$a^k = (k,0), a^{2k}x = (2k,1), a^{2k+1}x = (2k-1,1)$$

In other words, the twisting happens the other way. We can verify
that, by conjugating by suitable automorphisms of the groups, these
two correspondences are equivalent in all relevant respects. The
important point, though, is that as a set map, the Lazard-like
correspondence is {\em not} uniquely determined even in this simple
case.

\subsection{Investigating the conditions on $f$}

By construction $f$ is a $C$-level cocycle (which is why we were able
to construct a Lie ring in the first place). We now turn attention to
properties (8)-(13).

$f$ does satisfy property (8) (commutativity-preservation) for the
simple reason that there isn't enough space for anything else -- if
two elements commute, they are in the same cyclic group modulo the
center. This is true whenever $P/Z(P)$ is a Klein four-group.

Thus, $f$ obviously cannot satisfy property (9) (because (9) is
incompatible with (5) + (8)).

$f$ does not satisfy property (10), the strong congruence
condition. For instance, $f(a,x)$ is not the same as $f(a,ax)$ -- they
are $a^2$ and $a^6$ respectively.

$f$ satisfies property (11) because any proper subgroup is abelian and
by property (8), as already mentioned, $f$ is zero on any abelian
subgroup.

$f$ satisfies property (12). However, something interesting
happens. When we look at the quotient of the group $M_{16}$ by the
derived subgroup, we get $\Z_4 \times \Z_2$, an abelian group. When we
take the quotient for Lie rings, we get the abelian Lie ring $\Z_4
\times \Z_2$. Naively, we might hope that the bijection between them
is the usual identity map.

However, the bijection comes with a little twist, and the reason for
this is that, roughly speaking, when we descend to a quotient, the
property of being cyclicity-preserving is not preserved. Down in the
quotient, we have cases where $f$ takes as its values elements of
order $2$. This spoils the Lie ring proof.

More surprises of this sort routinely occur.

\subsection{Investigating the Lazard correspondence further in this case}

How far does the Lazard correspondence between $M_{16}$ and its Lazard
Lie group extend? Let's recall some general features of the Lazard
correspondence and see whether they apply in this case:

\begin{enumerate}
\item {\em Automorphisms}: Indeed, the automorphisms of the group
  match those of its Lie ring.
\item {\em Endomorphisms}: ? (probably yes)
\item {\em Subgroups and subrings}: Yes, by theorem
\item {\em Normal subgroups}: Yes, by theorem
\item {\em Abelian subgroups}: Yes, by theorem
\item {\em Cyclic subgroups and order statistics}
\end{enumerate}

\subsection{A small lemma and another group of order $16$}

Here is a little lemma that essentially reworks all the detailed steps
we traced through for $M_{16}$.

\begin{lemma}
  If $P$ is a group of order $16$ such that the center $Z(P)$ is a
  cyclic group of order four and the derived subgroup is cyclic of
  order two, then we can find a cocycle $f$ satisfying conditions
  (1)-(4). Moreover, there are precisely two possibilities for $f$ and
  they both give mirror Lie rings. Also, both these choices of $f$ are
  commutativity-preserving.
\end{lemma}

\begin{proof}
  Since $Z(P)$ has order $4$, $P/Z(P)$ is abelian of order four. It
  cannot be cyclic, so it must be a Klein four-group. Further, the
  derived subgroup must be contained in the center, hence it must be
  the unique subgroup of order two inside the center. Let $a,x$ be
  elements of $P$ whose images generate $P/Z(P)$ and let $y$ be a
  generator of $Z(P)$. Then, we must have $[a,x] = [x,a] = [ax,x] =
  [x,ax] = [ax,a] = [a,ax] = y^2$. We can now set $f(a,x) = y$, and
  this forces $f(x,a) = y^{-1}$, $f(a,ax) = y^{-1}$, $f(ax,a) = y$,
  $f(x,ax) = y$, $f(ax,x) = y^{-1}$. Alternatively, setting $f(a,x) =
  y^{-1}$ inverts all the other values of $f$. We can also see that
  either choice of $f$ is commutativity-preserving.
\end{proof}

There are precisely two isomorphism classes of groups of order $16$
with centerof order four. One of these is $M_{16}$, which has been
discussed above. The other group, $P$, can be described as a central
product of $D_8$ and $\Z_4$ over a common central subgroup of order
two. Explicitly:

$$P := \langle a,x,y \mid a^4 = x^2 = e, a^2 = y^2, xax = a^{-1}, ay = ya, xy = yx \rangle$$

This group $P$ has a corresponding Lie ring, whose additive group is
$\Z_4 \oplus \Z_2 \oplus \Z_2$. Note that the letters $a$, $x$, $y$
exactly match up with the letters in the lemma above, so it is easy to
work out the cocycles.

Note that unlike the case of $M_{16}$, $f$ does {\em not} satisfy the
subgroup-closure condition on $P$. In particular, subgroups of $P$
need not always be subrings of the corresponding Lie ring. This is
unsurprising since $P$ contains such renegade subgroups as the
dihedral group and the quaternion group, which are well-known for not
having anything resembling a Lie ring. While this seems problematic
from the perspective of $P$, it is actually a great way of
disciplining groups such as $D_8$ and $Q_8$ with Lie structures.

\section{General considerations}

\subsection{1-isomorphisms, quasihomomorphisms, and order statistics}

How far can we take the Lazard correspondence for $2$-groups and in
general? While the considerations given so far are {\em expansive},
describing how we can expand the Lazard correspondence to new
situations, we now give certain {\em restrictive} conditions. These
restrictive conditions help narrow the field of empirical
investigation (for instance, when looking at groups of order $16$,
$32$, and $64$) and allows us to zero in on important cases.

We make the following temporary definitions (these definitions are not
standard but are similar to some standard terms used in related areas,
hence the names are plausible):

\begin{enumerate}
\item We say that two finite groups have the same {\em order
  statistics}, or are {\em order statistics-equivalent}, if, for every
  positive integer $d$, they have the same number of elements of order $d$.
\item A set map $f:G \to H$ between groups is termed a {\em
  1-isomorphism} if $f$ is bijective and it restricts to an
  isomorphism (both ways) on cyclic subgroups. We say that two groups
  are {\em 1-isomorphic} if there is a 1-isomorphism between
  them. Clearly, 1-isomorphic finite groups have the same order statistics.
\item A set map $f:G \to H$ is termed a {\em quasihomomorphism} if its
  restriction to any abelian subgroup of $G$ is a homomorphism.
\end{enumerate}

In particular, we are interested in finite $p$-groups that:

\begin{enumerate}
\item are order statistics-equivalent to abelian $p$-groups (weakest
  condition)
\item are 1-isomorphic to abelian $p$-groups (intermediate condition)
\item admit bijective quasihomomorphisms to abelian $p$-groups
  (strongest condition)
\end{enumerate}

Lazard Lie groups as we usually understand them satisfy all three
conditions, with the corresponding abelian group being the Lazard Lie
ring. Note that this holds for class bigger than two as well.

What about $2$-groups of class two? The version of the Lazard
correspondence in Theorem \ref{lazardgrouptoring1234} does not
guarantee anything. If, however, we assume that $f$ is
cyclicity-preserving (condition (7)), the group is 1-isomorphic to an
abelian $p$-group, namely the additive group of its Lazard Lie
ring. If we assume condition (8)
(commutativity-preservation), the identification with the Lazard Lie
ring is a bijective quasihomomorphism (theorem
\ref{lazardgrouptoringmore}, part (1)). $M_{16}$ is thus a special
case.

This suggests that, in order to determine examples for theorem
\ref{lazardgrouptoring1234}, we can use the following empirical sieve:
restrict attention to finite $2$-groups $P$ such that $[P,P] \le
\mho^1(Z(P))$ (which is necessary for a cocycle of the required type
to exist) and such that the group has the same order statistics as an
abelian $2$-group. The advantage of these two conditions is that they
can be used very quickly to filter out most groups given a computer or
a classification by hand. The study of the few remaining cases for
order $16$ and order $32$ are discussed in the next few sections.


\subsection{Order $16$: groups with the same order statistics as abelian groups}

We discussed earlier that there are exactly two isomorphism classes of
groups of order $16$ whose center is cyclic of order four, and that we
can construct Lie rings for these. These two groups are $M_{16}$ and
the central product of $D_8$ and $\Z_4$. We now show that no other
non-abelian groups of order $16$ come even close to having Lazard Lie
rings -- because no other non-abelian group is 1-isomorphic to an
abelian group of order $16$. More details are given below.

(MOre )A link to more detailed A A
There are the following five non-abelian groups of order $16$ that
have the same order statistics as an abelian group:

\begin{enumerate}
\item $M_{16}$, which has been discussed above. It has $\Z_8 \times
  \Z_2$ as the additive group of its Lazard Lie ring. Thus, it admits
  a bijective quasihomomorphism to $\Z_8 \times \Z_2$ and is in
  particular 1-isomorphic to that group.
\item The group that can be described as the central product of $D_8$
  and $\Z_4$ with common center cyclic of order two. It can also be
  described as the central product of the quaternion group and $\Z_4$
  with common center cyclic of order two. It has $\Z_4 \oplus \Z_2
  \oplus \Z_2$ as the additive group of its Lazard Lie ring, and thus
  admits a bijective quasihomomorphism to that abelian group.
\item A group with presentation $\langle a,b,c \mid a^4 = b^2 =
  c^2 = e, ab = ba, bc = cb, cac^{-1} = ab \rangle$. This has the same
  order statistics as $\Z_4 \times \Z_2 \times \Z_2$ but is not
  1-isomorphic to it because it has two non-identity elements that are
  squares, unlike $\Z_4 \times \Z_2 \times \Z_2$.
\item The group that can be described as the nontrivial semidirect
  product of $\Z_4$ and $\Z_4$. This has the same order statistics as
  $\Z_4 \times \Z_4$ but is not 1-isomorphic to it because it has two
  non-identity elements that are squares, unlike $\Z_4 \times \Z_4$
  which has three such elements.
\item The group that can be described as the direct product $Q_8
  \times \Z_2$. This has the same order statistics as $\Z_4 \times
  \Z_4$ but is not 1-isomorphic to it because it has only one
  non-identity element that is a square, unlike $\Z_4 \times \Z_4$
  which has three such elements.
\end{enumerate}

Of the remaining groups $9$ groups of order $16$, three are abelian,
one is $D_8 \times \Z_2$, and the other three are the maximal class
groups ($D_{16}$, $SD_{16}$, and $Q_{16}$) -- clearly, none of the
non-abelian groups among these has the same order statistics as an
abelian group.

\subsection{Caught in a Lie ring at last: $D_8$ and $Q_8$}

Groups that are not order statistics-equivalent to abelian $p$-groups
cannot have Lie rings of their own in the sense that we have been
describing. However, this can be remedied if these groups can be
realized as {\em subgroups} inside bigger groups that have Lazard Lie
rings in the weak sense described by \ref{lazardgrouptoring1234}, but
without the subgroup-closure condition.

Consider, for instance, the group $P$ described a little earlier:

$$P := \langle a,x,y \mid a^4 = x^2 = e, a^2 = y^2, xax^{-1} = a^{-1}, xy = yx, ay = ya\rangle$$

$P$ contains a subgroup $\langle a,x\rangle$ isomorphic to $D_8$ (in
fact, it contains three such subgroups related by outer
automorphisms). $P$ also contains a subgroup $\langle a, xy \rangle$
which is the unique subgroup isomorphic to a quaternion group. Its
non-central elements are $a,a^3, xy, a^2xy, axy, a^3xy$.

Since we have $f(a,x) = y$, neither the dihedral nor the quaternionic
subgroups are closed under $f$. Thus, they are not Lie
subrings. However, the addition is {\em partially defined}. It is
likely that this approach to a patchy version of a Lie ring structure
can be related to other such approaches in the past. For instance,
Glauberman's approach (give refs) uses abelian normal subgroups and
more generally, normal subgroups of small nilpotency class that
generate the whole group. The approach here uses {\em supergroups}
rather than {\em subgroups} to give a patchy Lie structure, but it
might agree on some specifics.

I intend to work out the example of this group $P$ in more detail,
since it probably holds the key to many other similar examples for
larger groups.

\section{Questions raised}

\subsection{Specific questions}

These are very specific questions that will either have a {\em yes}
answer with a proof or a {\em no} answer with an explicit
counterexample.

\begin{enumerate}
\item Suppose $P$ is a finite $2$-group satisfying $[P,P] \le
  \mho^1(Z(P))$. Is it necessary that $P$ has a Lazard Lie ring in the
  sense of theorem \ref{lazardgrouptoring1234}? Specifically, does $P$
  have the necessary cocycle by which we can carry out the
  construction? 

  My guess is {\em no}, but the counterexamples would probably be of
  order $64$ or higher. The idea would be to have a lot of commutators
  clashing to the same value in a manner that taking square roots gives
  a collection of mutually inconsistent constraints.
\item Suppose $L$ is a finite Lie ring of order a power of $2$ such
  that $[L,L] \le \mho^1(Z(L))$. Is it necessary that $L$ has a Lazard
  Lie group in the sense of theorem
  \ref{lazardringtogroup1234}. Specifically, does $L$ have the
  necessary cocycle by which we can carry out the construction?

  My guess is {\em no}, for reasons similar to those in the previous
  part.
\item Suppose we have two Lazard Lie rings arising for a group from
  different choices of cocycle in theorem \ref{lazardgrouptoring1234}?
  Must they be isomorphic? (This is equivalent to asking whether two
  different Lie rings can give rise to isomorphic Lazard Lie groups).

  My guess is {\em no}.
\item Suppose we have two Lazard Lie groups arising from a Lie ring
  for different choices of cocycle in theorem
  \ref{lazardgrouptoring1234}. Must they be isomorphic? (This is
  equivalent to asking whether two different group can give rise to
  isomorphic Lazard Lie rings).

  My guess is {\em no}.
\item If a finite $2$-group of class two is 1-isomorphic to an abelian
  $2$-group, must it admit a Lazard Lie ring?
\item If a finite $2$-group has a bijective quasihomomorphism to an
  abelian group, must this homomorphism correspond to a Lazard Lie
  ring construction?
\item Is every finite $2$-group of class two a subgroup of a finite
  $2$-group that admits a Lazard Lie ring in the sense of theorem
  \ref{lazardgrouptoring1234}?
\end{enumerate}

\subsection{More open-ended questions}

Here are some:

\begin{enumerate}
\item How does the construction here generalize to nilpotency class
  $3$ or higher for $2$-groups? Does it involve more cocycles?
\item How does the construction generalize to nilpotency class $p$ or
  higher for $p$-groups?
\item We see that the correspondence fails to perfectly match
  subgroups to subrings, because the subgroup-closure condition is
  violated. Can we better characterize these patchy sort-of subrings
  in the Lie ring that correspond to subgroups of the Lazard Lie group?
\item One of the problems with descending to quotients is that the
  cyclicity-preserving condition is not preserved when we pass to
  quotients. What is the best way to think about this and/or to
  rectify this? Is there a procedure for rectifying a cocycle that is
  not cyclicity-preserving to get one that is?
\end{enumerate}

\section{Groups of order $32$}

\subsection{A metacyclic construction}

The construction of the Lazard Lie ring for $M_{16}$ can be
generalized to any group $P$ of the form:

$$P := \langle a,x \mid a^{2^m} = x^{2^n} = e, xax^{-1} = a^{2^l + 1} \rangle$$

where $m$, $n$, $l$ are positive integers with $2l + 1 \le n$.

The corresponding Lie ring $L$ is $\Z_{2^m} \oplus \Z_{2^n}$ with the
Lie bracket of the two generators being $2^l$ times the first
generator. (I need to insert more details on what $f$ is).

We now move to the groups of order $32$. We list here the various
obvious possibilities for non-abelian groups of class two admitting
Lazard Lie rings in the sense of theorem \ref{lazardgrouptoring1234}.

\begin{enumerate}
\item The direct product of $M_{16}$ and $\Z_2$ admits a Lazard Lie
  ring which is the direct product of the Lazard Lie ring for $M_{16}$
  and $\Z_2$. The additive group of this is $\Z_8 \times \Z_2 \times
  \Z_2$. (for GAP reference: ID 36 for the abelian group and 37 for
  $M_{16} \times \Z_2$).
\item The direct product of the group $D_8 *_{\Z_2} \Z_4$ (the other
  group of order $16$ admitting a Lazard Lie ring) with $\Z_2$ also
  admits a Lazard Lie ring. The additive group is $\Z_4 \times
  E_8$. (for GAP reference: ID 45 for the abelian group and 48 for the
  non-abelian group).
\item The group $M_{32}$ admits a Lazard Lie ring (using the general
  metacyclic construction). The additive group of this is $\Z_{16}
  \times \Z_2$. (for GAP reference: ID 16 for the abelian group and 17
  for $M_{32}$).
\item The semidirect product $\Z_8 \rtimes \Z_4$ where the generator
  of $\Z_4$ acts by the multiplication by $5$ map, admits a Lazard Lie
  ring (again by the metacyclic construction) whose additive group is
  isomorphic to $\Z_8 \times \Z_4$. (for GAP reference: ID 3 and ID 4).
\end{enumerate}

Do these ``obvious'' examples of Lazard Lie groups exhaust all the
Lazard Lie groups of order $32$? 

As before, we first filter out the groups $P$ that satisfy the
property that $P' \le \mho^1(Z(P))$. It turns out that all these
groups are okay with respect to order statistics. There are only two
non-abelian groups among these that are not covered in the above
list. These two non-abelian groups are described by presentations
below:

$$\langle a,b,c \mid a^4 = b^4 = c^2 = e, ab = ba, ac = ca, cbc^{-1} = a^2b \rangle$$

(has GAP ID 24 and has the same order statistics as $\Z_4 \times \Z_4
\times \Z_2$)

and

$$\langle a,b,c \mid a^8 = b^2 = c^2 = e, ab = ba, cbc^{-1} = a^4b, cac^{-1} = a^5 \rangle$$

(has GAP ID 38 and has the same order statistics as $\Z_8 \times \Z_2
\times \Z_2$). I haven't yet been able to figure out more about the
first group, so I turn now to the second group.

\subsection{More information on groups of order $32$}

Here's what I found via GAP for non-abelian groups:

\begin{itemize}
\item The derived subgroup in agemo subgroup of center ($P' \le
  \mho^1(Z(P))$) gives precisely the six groups above: GAP IDs 4, 17, 24,
  37, 38, 48.
\item The condition that order statistics match those of an abelian
  group gives 18 groups, including the above six and ten more, with
  GAP IDs 2, 5, 8, 12, 22, 23, 28, 29, 33, 47. It turns out that the
  group with GAP ID 8 has nilpotency class three, while all the others
  have nilpotency class two.
\item The condition that the order root statistics match those of an
  abelian group gives 8 groups, including the six in point (1) and the
  groups with GAP IDs 2 and 33.
\end{itemize}
\subsection{The group that is sort of like $M_{16} \times \Z_2$}

Define:

$$P := \langle a,b,c \mid a^8 = b^2 = c^2 = e, ab = ba, cbc^{-1} = a^4b, cac^{-1} = a^5 \rangle$$

$P$ looks a lot like $M_{16} \times \Z_2$ (GAP ID: 37). In fact, if
the relation $cbc^{-1} = a^4b$ were replcaed by the relation $cbc^{-1}
= b$, then we would get $M_{16} \times \Z_2$.

There is probably an alternative presentation of $P$ that casts more
insight. This is something I will return to later.

\section{Groups of order $64$}

These are dumps from a GAP investigation.

Here are the GAP IDs of all the non-abelian groups with the same order
statistics as an abelian group (there are 88 of them):

3, 4, 5, 15, 16, 17, 23, 24, 25, 27, 28, 29, 30, 31, 44, 45, 51, 56,
57, 58, 59, 60, 61, 62, 63, 64, 65, 66, 68, 69, 70, 72, 73, 74, 76,
77, 78, 79, 80, 81, 82, 84, 85, 86, 87, 88, 89, 91, 93, 94, 103,
104, 105, 112, 113, 114, 152, 154, 184, 185, 193, 194, 195, 197,
198, 200, 201, 203, 204, 208, 209, 210, 212, 214, 215, 216, 217,
220, 222, 223, 224, 225, 247, 248, 249, 262, 263, 266

Of these, 73 have class two, and these have group IDs:

3, 17, 27, 28, 29, 44, 51, 56, 57, 58, 59, 60, 61, 62, 63, 64, 65, 66,
68, 69, 70, 72, 73, 74, 76, 77, 78, 79, 80, 81, 82, 84, 85, 86, 87,
88, 89, 103, 104, 105, 112, 113, 114, 184, 185, 193, 194, 195, 197,
198, 200, 201, 203, 204, 208, 209, 210, 212, 214, 215, 216, 217, 220,
222, 223, 224, 225, 247, 248, 249, 262, 263, 266

The remaining 15 have nilpotency class exactly three, and these have group IDs:

4, 5, 15, 16, 23, 24, 25, 30, 31, 45, 91, 93, 94, 152, 154

Here are the GAP IDs of all the non-abelian groups with the same order
root statistics as an abelian group (there are 33 of them):

3, 17, 25, 27, 28, 51, 56, 57, 61, 64, 77, 82, 84, 85, 86, 112, 113,
114, 184, 185, 195, 198, 201, 209, 210, 217, 220, 223, 247, 248, 249,
263, 266

Of these, 32 have nilpotency class two, and there is one group of
nilpotency class three. This has group ID 25.

Finally, there are 17 groups that satisfy the property that the
derived subgroup is contained in the first agemo subgroup of the
center, with group IDs:

3, 27, 51, 57, 84, 85, 86, 112, 184, 185, 195, 198, 247, 248, 249, 263, 266

and all of them have the same order root statistics as some abelian group.

\section{Groups of order $128$}

There are $50$ non-abelian groups of order $128$ whose derived
subgroup is contained in the first agemo subgroup of the center. Here
are the IDs:

43, 44, 129, 130, 160, 180, 181, 182, 183, 184, 185, 186, 187, 457,
458, 476, 479, 838, 839, 840, 842, 894, 989, 990, 999, 1002, 1602,
1603, 1604, 1605, 1606, 1607, 1649, 1652, 1654, 1696, 1697, 2137,
2138, 2139, 2153, 2156, 2160, 2176, 2302, 2303, 2304, 2305, 2322, 2325

It is also true that each of these have the same order root statistics
as an abelian group. This means that we cannot rule out the
possibility that it actually arises as a Lazard Lie ring.

\section{Bracketed class two groups}

\subsection{Bracketed class two groups}

We define a {\em bracketed class two group} as a group $G$ with a binary
operation $\{ \ , \ \}$ as follows:

\begin{enumerate}
\item {\bf Identity as nil}: $\{ g,e \} = e$.
\item {\bf Alternating property}: $\{ g,g \} = e$ and $\{ g,h \} = \{
  h,g \}^{-1}$ for all $g,h \in G$.
\item {\bf Linearity}: $\{g,hk \} = \{ g,h \} \{ g,k\}$
  and $\{ gh,k \} = \{ g,k \} \{ h,k \}$.
\item {\bf Class two conditions}: $\{ \{ g,h \}, k \} = e$, $[[g,h],k]=e$, $[\{g,h\},k] = e$,
  $\{[g,h],k\} = e$.
\end{enumerate}

We define the following:

\begin{enumerate}
\item The {\em bracket center} of a bracketed class two group $G$ is
  the set of all elements $g \in G$ such that $[g,h] = \{ g, h \} = e$
  for all $h \i nG$.
\item The {\em bracket derived subgroup} of a bracketed class two
  group $G$ is the group generated by all elements of the form $[g,h]$
  and $\{ g,h \}$. Note that it is normal and invariant under the
  bracket and the quotient is an abelian group with trivial bracket.
\end{enumerate}

Here are some easy observations:

\begin{itemize}
\item A bracketed class two group where the bracket coincides with the
  commutator map is the same thing as a group of nilpotency class two.
\item A bracketed class two group where the group operation is abelian
  is the same thing as a Lie ring of nilpotency class two.
\end{itemize}

Bracketed class two groups can thus be thought of as simultaneously
generalizing class two groups viewed with the usual commutator and
class two Lie rings.

\subsection{Spraying bracketed class two groups}

In theorem \ref{lazardgrouptoring1234}, we started with a class two
group and chose a cocycle satisfying three additional conditions (skew
is commutator, identity-preserving, inverse-preserving). We then
twisted the group multiplication by this $B$-level cocycle to obtain
an abelian group.

Revisiting the proof of theorem \ref{lazardgrouptoring1234}, we note
that the skew is commutator condition on the cocycle was used to prove
commutativity, and the remaining conditions were used to prove the
remaining properties. This suggests a natural question: what happens
if we drop the skew is commutator condition? We still get a
group. Further, on this new group, if we choose as our bracket the
commutator arising from the old group structure, we get a bracketed
class two group. This result is stated as a theorem below.

\begin{theorem}
  Let $P$ be a class two $p$-group and let $f$ be a map $P \times P
  \to Z(P)$ that is constant on the cosets of $Z(P)$ in each
  coordinate and, further, that satisfies the cocycle condition and
  the identity-preserving and inverse-preserving conditions. Consider
  $P$ with the following bracketed group structure:

  \begin{enumerate}
  \item The multiplication is given by $x * y :=
  \frac{xy}{f(x,y)}$
  \item The identity element is the same as that for $P$ originally.
  \item The inverses are the same as for $P$ originally.
  \item $\{ x,y \}$ is the commutator $[x,y]$ in the original group
    $P$.
  \end{enumerate}
\end{theorem}

\begin{proof}
  The proof that $P$ gets a group structure with the operations
  defined in parts (1)-(3) is the same as parts of the proof given in
  theorem \ref{lazardgrouptoring1234}. Note that these parts of the
  proof require only the cocycle condition and the
  identity-preserving and inverse-preserving conditions.

  {\em If $x \in Z(P)$, then $x * y = y * x = xy$ for all $y$}: This
  follows from the properties of $f$.

  {\em The center with respect to the original multiplication lies
  inside the center with respect to the new multiplication}: This
  follows from the above.

  {\em Any commutator of $P$ with respect to $*$ lies in the center of
  $P$ with respect to the original multiplication}: out the
  expressions yields that the commutator with respect to $*$ is:

  $$[x,y]_{*} = \frac{f(y,x)}{f(x,y)}[x,y]$$

  This is a product of elements in $Z(P)$ with respect to the original
  multiplication, hence is in $Z(P)$.

  {\em The commutator in the original group gives a bracket structure
  of class two}: Because the original group has class $2$, we have
  $[[x,y],z]$ is the identity element by definition. We check the
  various properties for being a bracket structure.

  \begin{enumerate}
  \item {\bf Identity as nil}: Obviously true, because it is a
    commutator in the original group structure and the identity element is the same.
  \item {\bf Alternating property}: Obviously true, because it holds
    in the original group structure and the notion of identity and
    inverses is the same.
  \item {\bf Linearity}: Note that $[g,hk] = [g,h][g,k]$ in the
    original group. Since all the elements are central in the original
    group, the multiplication works the same way in both groups.
  \item {\bf Class two conditions}: We already proved that $[g,h]_{*}$
    is in the center with respect to the original group. Also, $\{ g,h
    \} = [g,h]$ in the original group so it is in the center of the
    original group. Hence, both these commutators lie in the centers
    with respect to both structures, giving the class two conditions.
  \end{enumerate}
\end{proof}

For simplicity, we abbreviate cocycles which are both
identity-preserving and inverse-preserving as IIP cocycles.

\subsection{A review of the Lazard Lie ring from this perspective}

Here is a way of thinking about what's happening. We start with a
class two group. Consider the set of IIP cocycles on this group. This
set is in fact a subgroup of the group of all cocycles. Every element
in this set corresponds to a bracketed class two group. For a class
two $p$-group $P$, we will call each such bracketed class two group an
IIP bracketed class two group for $P$.

Now, it is possible that some of these IIP cocycles are actually
$B$-level cocycles. The set of $B$-level cocycles, if nonempty, is a
coset, because the skew is commutator condition, if nonempty, gives a
coset and the intersection of a coset and a subgroup, if nonempty, is
a coset. The elements in this coset correspond to $B$-level Lie rings
for the group.

If we are a little more choosy, we can restrict attention to $C$-level
cocycles. This gives a possibly smaller (and possibly empty) coset
inside the coset of $B$-level cocycles. Again, each of these gives a
$C$-level Lie ring for the group. As discussed earlier, the additive
group of a $C$-level Lie ring is 1-isomorphic to the original group,
so $C$-level Lie rings are more tightly constrained.

\subsection{IIP and CP cocycles}

A somewhat stronger condition that being IIP is being
cyclicity-preserving, which we abbreviate as CP. A CP cocycle is thus
a cyclicity-preserving cocycle. The set of CP cocycles for a given
class two group $P$ is a subgroup of the group of IIP cocycles. Every
element of this subgroup gives a twisting of the multiplication of
$P$, and hence gives rise to a bracketed class two group. However,
since we're now demanding CP and not just IIP, we have the additional
information that the new group is 1-isomorphic to $P$. This puts
plenty of restrictions.


\end{document}