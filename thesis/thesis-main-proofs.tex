\section{The Lie bracket and group commutator in terms of each other: prime bounds}\label{sec:group-commutator-and-lie-bracket-ito-each-other}

\subsection{Group commutator formula in terms of the Lie bracket}\label{sec:group-commutator-ito-lie-bracket}

For reasons that will become clear later, we will work with the class
$(c + 1)$ Baker-Campbell-Hausdorff formula instead of the class $c$
Baker-Campbell-Hausdorff formula, where $c$ is a positive
integer. This is purely a matter of notation, and its main advantage
is that later results that use the results here can do so directly
without needing to increase or decrease the values by one.

In the usual Lazard correspondence, there is an explicit formula for
the group commutator of two elements in terms of Lie brackets. In
fact, there is an infinite series, which can be obtained from the
Baker-Campbell-Hausdorff formula series, whose truncations give the
formula for the group commutator. We will now describe how this
formula is obtained.

{\em Note that the explicit expressions here are sensitive to whether
  we use the left or right action convention for computing the group
  commutator. We will use the left action convention.}

$$[x,y]_{\text{Group}} = (xy)(yx)^{-1}$$

Inverses in the group correspond to taking the negative in the Lie
ring, so this becomes:

$$[x,y]_{\text{Group}} = (xy)(-(yx))$$

We now need to apply the Baker-Campbell-Hausdorff formula to expand
each piece.

We have the following in the class $(c + 1)$ case:

\begin{eqnarray*}
  xy = x + y + t_2(x,y) + t_3(x,y) + \dots + t_{c+1}(x,y)\\
  yx = y + x + t_2(y,x) + t_3(y,x) + \dots + t_{c+1}(y,x)\\
\end{eqnarray*}

We thus get the following expression for $[x,y]_{\text{Group}}$.

\begin{small}
$$(x + y + t_2(x,y) + t_3(x,y) + \dots + t_{c+1}(x,y))$$
$$- (y + x + t_2(y,x) + t_3(y,x) + \dots + t_{c+1}(y,x))$$
$$+ t_2(xy,-(yx)) + t_3(xy,-(yx)) + \dots + t_{c+1}(xy,-(yx))$$
\end{small}

Based on the symmetry and skew symmetry properties deduced in the
preceding section, we obtain the following, where $c'$ is the largest
even number less than or equal to $c + 1$:

\begin{small}
$$[x,y]_{\text{Group}} = 2(t_2(x,y) + t_4(x,y) + \dots + t_{c'}(x,y)) + t_2(xy,-(yx)) + t_3(xy,-(yx)) + \dots + t_{c+1}(xy,-(yx))$$
\end{small}

It is also the case that $t_{c+1}(xy,-(yx)) = 0$. This is because when we
expand these out, all the degree $c$ terms in the product are iterated
products with each piece equal to $(x + y)$ or $-(x + y)$, and all
higher degree terms are anyway zero. Thus, the group commutator
simplifies to:

\begin{small}
$$[x,y]_{\text{Group}} = 2(t_2(x,y) + t_4(x,y) + \dots + t_{c'}(x,y)) + t_2(xy,-(yx)) + t_3(xy,-(yx)) + \dots + t_c(xy,-(yx))$$
\end{small}

Denote this formula of $x$ and $y$ by $M_{c+1}(x,y)$. In other words:

$$[x,y]_{\text{Group}} = M_{c+1}(x,y)$$

Note that in the case that we are using the Lazard correspondence up
to isomorphism, such as the case where we are using the exponential
and logarithm maps inside an associative ring of class $c + 1$,
$M_{c+1}$ can be interpreted as follows:

$$e^{M_{c+1}(x,y)}= [e^x,e^y]_{\text{Group}}$$

The distinction between this and the earlier interpretation is that
with the earlier intepretation, we identified $x$ and $e^x$ as the
same element, whereas now, we are treating them as different
elements. The latter interpretation makes sense when we are describing
the exponential and logarithm maps inside an associative ring.

Note that the explicit formula $M_{c+1}$ is sensitive to whether we
are using the left action convention or the right action convention,
but the existence of a formula of the sort is not, and the general
properties of the prime divisors of denominators for the formula are
not. The corresponding formula with the right action convention would
expand the product $(yx)^{-1}(xy)$ instead of $(xy)(yx)^{-1}$, and the
steps would be fairly similar. The results from this point onward do
not depend on the choice of action convention (left versus right).

We are now in a position to prove a lemma.

\begin{lemma}\label{lemma:commutator-denominators}
  In the formula $M_{c+1}(x,y)$ for the group commutator
  $[x,y]_{\text{Group}}$ in terms of Lie ring operations (addition and
  Lie bracket) in class $c + 1$, all prime divisors of the
  denominators are less than or equal to $c$.
\end{lemma}

\begin{proof}
  We divide into cases:

  \begin{itemize}
  \item $c = 1$, so that $c + 1 = 2$: In this case, we can work the
    formula out and we get that $[x,y]_{\text{Group}} = [x,y]$. This
    satisfies the condition, since there are no prime factors of the
    denominator.
  \item $c$ is even, so that $c + 1$ is odd: In this case, the
    formula above shows that we use terms only up to $t_c$, and do not
    use $t_{c+1}$. Thus, we can only get the primes less than or equal
    to $c$.
  \item $c$ is odd and at least $3$, so that $c + 1$ is even and
    greater than $2$: In this case, $c + 1$ is composite. We know from the
    formula that we only use primes less than or equal to $c + 1$. Since
    $c + 1$ is composite, this means we only use primes less than or equal
    to $c$.
  \end{itemize}
\end{proof}

It is also possible to construct an infinite series expression whose
truncations give the group commutator formulas for various choices of
$3$-local nilpotency class.

The importance of these observations is as follows. We can make sense
of the formula for $M_{c+1}(x,y)$ for certain Lie rings that are {\em
  not} Lazard Lie rings. Specifically, we can make sense of this
formula for Lie rings of $3$-local nilpotency class $(c + 1)$ which
are uniquely divisible by primes strictly {\em less} than $c + 1$, but
not by $c + 1$ itself. This case is of interest (i.e., it conveys
nontrivial information) when $c + 1$ itself is a prime number. We will
build on the results here in Sections
\ref{sec:lazard-correspondence-derived} and \ref{sec:lcuti}.

The procedure outlined above has been used to compute $M_{c+1}$ for
small values of $c$ in the Appendix. The case $c = 2$ is described in
Section \ref{appsec:M3formula} and the case $c = 3$ is described in
Section \ref{appsec:M4formula}.

\subsection{Inverse Baker-Campbell-Hausdorff formula: bound on denominators}\label{sec:lie-bracket-ito-group-commutator}

We want to prove a similar result for the bounds on prime powers of
denominators for the formula $h_2$ that describes the Lie bracket of
the Lie ring in terms of group commutators. The result follows from
Theorem 6.4 in \cite{Lazardeffective}, but we provide a minimalistic
proof below that builds on Lemma \ref{lemma:commutator-denominators}.

\begin{lemma}\label{lemma:lie-bracket-denominators}
  In the formula $h_{2,c + 1}(x,y)$ for the Lie bracket
  $[x,y]_{\text{Lie}}$ in terms of group commutators, all prime
  divisors of the denominators in the exponents are less than or equal
  to $c$.
\end{lemma}

\begin{proof}
  Recall that $M_{c+1}(x,y)$ expresses the group commutator in terms
  of the Lie bracket and its iterations, whereas $h_{2,c + 1}(x,y)$
  expresses the Lie bracket in terms of the group commutator and its
  iterations. These formulas are therefore ``inverses'' of each other
  in the following sense. Denote by $(h_{2,c + 1} \circ M_{c +
    1})(x,y)$ the formula that substitutes, in each of the commutators
  appearing in the expression for $h_{2,c + 1}$, the formula for
  $M_{c+1}$, and then expands group products of these commutators as
  Lie ring expressions using the class $(c + 1)$
  Baker-Campbell-Hausdorff formula. Then:

  $$(h_{2,c+1} \circ M_{c+1})(x,y) \equiv [x,y]_{\text{Lie}},$$

  where the equality is considered modulo $\mathcal{A}^{c+2}$ where
  $\mathcal{A}$ is the free associative $\Q$-algebra with generators
  $x$ and $y$. Replacing $c + 1$ by $c$ above, we obtain:

  $$(h_{2,c} \circ M_c)(x,y) \equiv [x,y]_{\text{Lie}} \pmod{\mathcal{A}^{c+1}}$$

  Thus, we get that:

  $$(h_{2,c} \circ M_{c})(x,y) \equiv [x,y]_{\text{Lie}} + \chi_{c+1}(x,y) \pmod{\mathcal{A}^{c+2}}$$
 
 where $\chi_{c+1}$ is a $\mathbb{Q}$-linear combination of weight $c
 + 1$ Lie products. Note that since $\chi_{c+1}$ is obtained by
 composing and manipulating terms from $h_{2,c}$, $M_{c}$, and the
 class $c$ Baker-Campbell-Hausdorff formula, all the prime divisors
 appearing in its denominators are less than or equal to $c$.

 We also have that:

 $$M_{c+1}(x,y) = M_c(x,y) + \xi_{c+1}(x,y)$$

 where, by Lemma \ref{lemma:commutator-denominators}, all the prime
 divisors of denominators appearing in the expression for $\xi_{c+1}$ are
 less than or equal to $c$. It follows that:

 $$(h_{2,c} \circ M_{c+1})(x,y) \equiv [x,y]_{\text{Lie}} + \chi_{c+1}(x,y) + \xi_{c+1}(x,y)$$

 It now follows that:

 $$h_{2,c+1}(x,y) = h_{2,c}(x,y)(\tilde{\chi}_{c+1}(x,y)\tilde{\xi}_{c+1}(x,y))^{-1}$$

 where $\tilde{\chi}$ and $\tilde{\xi}$ denote the same expressions as
 those used for $\chi$ and $\xi$ but interpreting the Lie brackets as
 commutators. This works because the $(c+1)$-fold iterated commutator
 coincides with the $(c + 1)$-fold iterated Lie bracket in class $c +
 1$. In other words, the set map:

$$(x_1,x_2,\dots,x_{c+1}) \mapsto [[ \dots [x_1,x_2],x_3],\dots,x_c],x_{c+1}]$$

is the same whether we interpret the brackets on the right as group
commutators or as Lie brackets: this follows immediately from the
formula $M_{c+1}$.

 In particular, note that the only prime divisors that appear in the
 denominators of the exponents on the right are primes less than or
 equal to $c$, completing the proof.
\end{proof}

The case $c = 2$ (so that $c + 1 = 3$) has been worked out in the
Appendix, Section
\ref{appsec:lie-bracket-denominators-illustration}. Readers interested
in understanding the proof via a concrete illustration are advised to
read this.
%\newpage

\section{Lazard correspondence, commutativity relation, and central series}\label{sec:lazard-correspondence-commutativity-relation-central-series}

\subsection{The commutativity relation and the upper central series}

We begin with an important lemma.

\begin{lemma}\label{lemma:commutativity-relation-same-group-lie-ring}
  Suppose $G$ is a ($3$-local) class $c$ Lazard Lie group and $L =
  \log(G)$ is its Lazard Lie ring. Then, for elements $x,y \in G$, the
  commutator $[x,y]$ is the identity element of $G$ if and only if the
  Lie bracket $[x,y]$ is zero. In other words, $x$ and $y$ commute as
  group elements if and only if they commute as Lie ring elements.
\end{lemma}

\begin{proof}
  In Section \ref{sec:group-commutator-ito-lie-bracket}, we described
  a formula $M_{c+1}$ for the group commutator in terms of the Lie
  bracket. The formula makes it clear that if the Lie bracket is zero,
  the group commutator is the identity element.

  In Section \ref{sec:inverse-bch}, we described a formula $h_{2,c+1}$
  for the Lie bracket in terms of the group commutator. The formula
  makes it clear that if the group commutator is the identity element,
  then the Lie bracket is zero.
\end{proof}

It is possible to prove one direction directly from the
Baker-Campbell-Hausdorff formula rather than referencing the above
sections. The idea behind a direct proof would be to show that if the
Lie bracket $[x,y]$ is $0$, then $xy = x + y = yx$, so that the group
commutator is zero.

We now show that the Lazard correspondence relates the upper central
series of the Lie ring with the upper central series of the group.

\begin{theorem}
  Suppose $G$ is a ($3$-local) class $c$ Lazard Lie group and $L =
  \log(G)$ is its Lazard Lie ring. Then the following are true:

  \begin{enumerate}
  \item $\log(Z(G)) = Z(L)$ (this is an instance of the $3$-local
    class $c$ Lazard correspondence)
  \item $\log(G/Z(G)) = L/Z(L)$ (this is an instance of the $3$-local
    class $c$ Lazard correspondence)
  \item For all positive integers $i$, we have $\log(Z^i(G)) = Z^i(L)$
    and $\log(G/Z^i(G)) = L/Z^i(L)$.  where $Z^i$ denotes the $i^{th}$
    member of the upper central series (for $G$ or $L$ respectively).
  \item For all positive integers $i > j$, $\log(Z^i(G)/Z^j(G)) =
    Z^i(L)/Z^j(L)$ where $Z^i,Z^j$ denote the $i^{th}$ and $j^{th}$
    members of the lower central series (for $G$ or $L$ respectively).
  \end{enumerate}
\end{theorem}

Note that since we are in general using the $3$-local class $c$ Lazard
correspondence, the global class of $G$ and $L$ may be greater than
$c$. Thus, the result is potentially of interest even for $i \ge c$.

\begin{proof}
  {\em Proof of (1)}: The fact that $\log(Z(G)) = Z(L)$ follows
  immediately from the preceding lemma (Lemma
  \ref{lemma:commutativity-relation-same-group-lie-ring}). By Lemmas
  \ref{lemma:centerispoweringinvariant} and
  \ref{lemma:centerispoweringinvariant-lie}, $Z(G)$ and $Z(L)$ are
  both $\pi_c$-powered, so this is an instance of the $3$-local class
  $c$ Lazard correspondence.

  {\em Proof of (2)}: This follows from (1) and from facts for the
  $3$-local class $c$ Lazard correspondence similar to those for the
  global Lazard correspondence described in Section
  \ref{sec:global-lazard-correspondence-quot}.  Thus, $\log(G/Z(G)) =
  L/Z(L)$.

  {\em Proof of (3)}: We obtain this by induction using (1) and (2).

  {\em Proof of (4)}: This follows from (3), using reasoning similar
  to that in (2).
\end{proof}

\subsection{The commutator map and the lower central series}

We begin with a lemma.

\begin{lemma}\label{lemma:derived-subgroup-same-group-lie-ring}
  Suppose $G$ is a $3$-local class $c$ Lazard Lie group and $L =
  \log(G)$ is its Lazard Lie ring. Suppose $H$ is a $\pi_c$-powered
  normal subgroup of $G$ and $I = \log(H)$ is the corresponding
  $\pi_c$-powered ideal of $L$. Then, $[G,H]$ is a $\pi_c$-powered
  normal subgroup of $G$, $[L,I]$ is a $\pi_c$-powered ideal of $L$,
  and $[L,I] = \log([G,H])$.
\end{lemma}

\begin{proof}
  $[G,H]$ is normal by basic group theory, and it is $\pi_c$-powered
  by Lemma \ref{lemma:lcs-subgroup-divisibility}. $[L,I]$ is a
  $\pi_c$-powered ideal by basic Lie ring theory. Thus, $\log([G,H])$
  is a $3$-local class $c$ Lazard Lie subring of $L$, and
  $\exp([L,I])$ is a $3$-local class $c$ Lazard Lie subgroup of
  $G$. It therefore suffices to show that a generating set for $[G,H]$
  is inside $[L,I]$ and that a generating set for $[L,I]$ is inside
  $[G,H]$.

  {\em Proof that $\log([G,H]) \subseteq [L,I]$}: It suffices to show
  that every element of the form $[g,h]$, for $g \in G, h \in H$, is in
  $[L,I]$. This follows from the formula $[g,h] = M_{c+1}(g,h)$
  described in Section \ref{sec:group-commutator-ito-lie-bracket}. In
  particular, we use Lemma \ref{lemma:commutator-denominators} to
  argue that the expression is in $[L,I]$.

  {\em Proof that $[L,I] \subseteq \log([G,H])$}: It suffices to show
  that every element of the form $[x,y]$, for $x \in L$, $y \in I$, is
  in $[G,H]$. This follows from the formula $[x,y] = h_{2,c+1}(x,y)$
  described as the second inverse Baker-Campbell-Hausdorff formula in
  Section \ref{sec:inverse-bch}. In particular, we use Lemma
  \ref{lemma:lie-bracket-denominators} to argue that the expression is
  in $[G,H]$.
\end{proof}

We now show that the Lazard correspondence relates the lower central
series of the Lie ring with the lower central series of the group.

\begin{theorem}\label{thm:lazard-correspondence-lcs}
  Suppose $G$ is a $3$-local class $c$ Lazard Lie group and $L =
  \log(G)$ is its Lazard Lie ring. Then, the following hold.

  \begin{enumerate}
  \item For all positive integers $i$, $\log(\gamma_i(G)) =
    \gamma_i(L)$ where $\gamma_i$ denotes the $i^{th}$ member of the
    lower central series (for $G$ or $L$ respectively).
  \item For all positive integers $i < j$,
    $\log(\gamma_i(G)/\gamma_j(G)) = \gamma_i(L)/\gamma_j(L)$. In
    particular, setting $i = 1$, $\log(G/\gamma_j(G)) =
    L/\gamma_j(L)$.
  \end{enumerate}
\end{theorem}

\begin{proof}
  {\em Proof of (1)}: This follows by inducting on Lemma
  \ref{lemma:derived-subgroup-same-group-lie-ring}. The base case $i =
  1$ is given. For the inductive step from $i$ to $i + 1$, set $H =
  \gamma_i(G)$ and $I = \gamma_i(L)$.

  {\em Proof of (2)}: This follows from (1).
\end{proof}

%\newpage

\section{The Malcev and Lazard correspondence in the free case}\label{sec:malcev-lazard-free}

\subsection{Setup}\label{sec:malcev-lazard-free-setup}

We use the same notation as in Section
\ref{sec:free-associative-algebra}. The notation is reviewed below for
convenience. Denote by $\mathcal{A}$ the free associative
$\mathbb{Q}$-algebra on a generating set $S = \{ x_1,x_2,\dots
\}$. The generating set may have any cardinality. By the well-ordering
principle, we will index the generating set by a well-ordered set. For
a positive integer $c$, define $A :=
\mathcal{A}/\mathcal{A}^{c+1}$. $A$ can be described as the free
nilpotent associative algebra of class $c$ on the same generating set
$S$. Explicitly, this means that all products in $A$ of length more
than $c$ become zero.

Denote by $\mathcal{L}$ the Lie subring of $\mathcal{A}$ generated by
the free generating set $S$. Note that $\mathcal{L}$ is {\em only} a
Lie ring, {\em not} a $\mathbb{Q}$-Lie
algebra. $\mathbb{Q}\mathcal{L}$ is the $\mathbb{Q}$-Lie algebra
generated by $S$ in $\mathcal{A}$.

We can define $L$ in the following equivalent ways:

\begin{itemize}
\item $L = \mathcal{L}/\gamma_{c+1}(\mathcal{L}) =
  \mathcal{L}/(\mathcal{L} \cap \mathcal{A}^{c+1})$.
\item $L$ is the Lie subring generated by the image of $S$ inside $A$,
  i.e., it is the Lie subring generated by the freely generating set
  inside $A$.
\end{itemize}

Denote by $F$ the group generated by the elements $e^{x_i}$, $x_i$
varying over the generating set $S$ of $L$. As per Theorem
\ref{thm:free-nilpotent}, $F$ is a free nilpotent group with $e^{x_i}$
forming a freely generating set.

We can define an exponential map

$$\exp:L \to 1 + A$$

by:

$$\exp(x) = 1 + x + \frac{x^2}{2!} + \dots + \frac{x^c}{c!}$$

We can also define a logarithm map:

$$\log:F \to A$$

by:

$$\log x = (x - 1) - \frac{(x - 1)^2}{2} + \frac{(x - 1)^3}{3} - \dots + \frac{(-1)^c(x - 1)^c}{c}$$

We might naively hope that $\exp(L) = F$ and $\log(F) = L$. This,
however, is not the case. A somewhat weaker version of the statement
{\em is} true, as described in Section
\ref{sec:inverse-bch-pi-powered} (the original proof is in Khukhro's
book \cite{Khukhro}, Theorem 10.22), and we recall that below, because
we will be using it extensively in some of our proofs. Denote by
$\pi_c$ the set of all primes less than or equal to $c$. Then, for any
prime set $\sigma \supseteq \pi_c$, $\sqrt[\sigma]{F} =
\exp(\Z[\sigma^{-1}]L)$. In particular, both of these hold:

\begin{itemize}
\item $\sqrt[\pi_c]{F} = \exp(\Z[\pi_c^{-1}]L)$
\item $\sqrt{F} = \exp(\Q L)$ 
\end{itemize}

\subsection{Interpretation as correspondences up to isomorphism}\label{sec:malcev-lazard-correspondence-interpretation}

In Section \ref{sec:abelian-lie-correspondence-up-to-isomorphism}, we
had described the abelian Lie correspondence {\em up to
  isomorphism}. This differs from the abelian Lie correspondence in
the important sense that the group and the Lie ring need not have the
same underlying set. Instead, we are given a set map (denoted $\exp$)
from the Lie ring to the group and a set map (denoted $\log$) from the
group to the Lie ring that play the role of identifying the sets with
each other. We noted in Section \ref{sec:isocat-template} that this
notion of relaxing {\em up to isomorphism} applies to all the
similarly defined correspondences in this document. In Section
\ref{sec:baer-correspondence-isocat-consequences}, we noted that the
notion applies to the Baer correspondence.

The notion of a correspondence up to isomorphism also applies to the
Malcev correspondence and the Lazard correspondence. In both cases, we
need to specify a set map from the Lie ring to the group (customarily
denoted $\exp$) and a set map from the group to the Lie ring
(customarily denoted $\log$) that play the role of identifying the sets.

Using the same notation as in Section
\ref{sec:malcev-lazard-free-setup}, we obtain the following
correspondences up to isomorphism:

\begin{itemize}
\item The Lie ring $\Q L$ and the group $\sqrt{F}$ are in class $c$
  Malcev correspondence up to isomorphism. The $\exp$ and $\log$ maps
  defining the correspondence coincide with the $\exp$ and $\log$ maps
  defined in Section \ref{sec:malcev-lazard-free-setup}. This is not a
  coincidence: the terminology $\exp$ and $\log$ that we introduced in
  the abstract setting was specifically {\em chosen} based on this
  setup.
\item For any prime set $\sigma$ containing $\pi_c$, the Lie ring
  $\Z[\sigma^{-1}]L$ and the group $\sqrt[\sigma]{F}$ are in global
  class $c$ Lazard correspondence up to isomorphism. The $\exp$ and
  $\log$ maps defining the correspondence coincide with the $\exp$ and
  $\log$ maps defined in Section \ref{sec:malcev-lazard-free-setup}.
\item In particular, the Lie ring $\Z[\pi_c^{-1}]L$ and the group
  $\sqrt[\pi_c]{F}$ are in global class $c$ Lazard correspondence up
  to isomorphism, with the same $\exp$ and $\log$ maps.
\end{itemize}

\subsection{Restriction of the Malcev correspondence to groups and Lie rings that lack adequate powering}

We continue using the same notation as above: $L$ is a free nilpotent
Lie ring of class $c$ on the generating set $S$, and $F$ is a free
nilpotent group of class $c$ on the generating set $e^S$, which is in
canonical bijection with $S$. We had already noticed that $L$ and $F$
are not in Lazard correspondence with one another. We will now take a
closer look at the statement.

Any element of $L$ can be written as a sum involving the $x_i$s and
Lie products involving the $x_i$s. Consider, for instance, the case
that $|S| = 2$ and the generating set is $\{ x_1, x_2 \}$. Consider
the element $x_1 + x_2$ of $L$. Then, to find $e^{x_1 + x_2}$, we can
use the inverse Baker-Campbell-Haudorff formula, and obtain:

$$e^{x_1 + x_2} = h_{1,c}(e^{x_1},e^{x_2})$$

The formula $h_{1,c}$ involves taking $p^{th}$ roots for primes $p \in
\pi_c$. Therefore, even though the elements $e^{x_1}$ and $e^{x_2}$ are in
$F$, the element $e^{x_1 + x_2}$ is not in $F$ if $c \ge 2$.

Consider a more complicated element of $L$.

$$z = x_1 + x_2 + 3[x_1,x_2]$$

We can compute $e^z$ by splitting $z$ as a sum $(x_1 + x_2) +
3[x_1,x_2]$, then splitting $x_1 + x_2$ as a sum again, and also
$h_{2,c}$ to compute $e^{[x_1,x_2]}$. The final formula is:

$$e^z = h_{1,c}(h_{1,c}(e^{x_1},e^{x_2}),(h_{2,c}(e^{x_1},e^{x_2}))^3)$$

Once again, we are guaranteed that $e^z \in \sqrt[\pi_c]{F}$. However,
it will not in general be in $F$ itself.

There do exist some elements of $L$ whose exponential is in $F$. For
instance, all the elements of the generating set $S$ have exponentials
in $F$. Other examples also exist. For instance, in the case $c = 2$,
the exponential of $[x_1,x_2]$ is $[e^{x_1},e^{x_2}]$ and is inside
$F$. The subset of $L$ comprising those elements whose exponential is
in $F$ is a generating set for $L$ as a Lie ring (because in
particular it contains $S$) and therefore in particular it is {\em
  not} a subring of $L$.

Suppose now that we consider the Lie ring $\Z[\sigma^{-1}]L$ and the
group $\sqrt[\sigma]{F}$, where $\sigma$ is a subset of $\pi_c$ that
does not include all the primes in $\pi_c$. Note that the whole Lie
ring $\Z[\sigma^{-1}]L$ is not in Lazard correspondence with the whole
group $\sqrt[\sigma]{F}$. In particular, $\exp(\Z[\sigma^{-1}]L)$ is
not contained in $\sqrt[\sigma]{F}$, and $\log(\sqrt[\sigma]{F})$ is
not contained in $\Z[\sigma^{-1}]L$.

It might still be the case, however, that there are subrings of
$\Z[\sigma^{-1}]L$ that are in Lazard correspondence with certain
subgroups of $\sqrt[\sigma]{F}$. The naive intuition is that, as long
as the formulas involved require division only by the primes in
$\sigma$ and not by any of the other primes in $\pi_c$, the
correspondence should work.

We will now establish a few important results of this sort.

\subsection{Lazard correspondence between derived subgroup and derived subring}\label{sec:lazard-correspondence-derived}

The following result is an extremely important first step both in
defining and in establishing important aspects of the Lazard
correspondence up to isoclinism.

We will perform our calculuations in class $c + 1$. This will make it
easier to use our results directly in later sections. However, it will
be a slight departure from the preceding discussion. For clarity, we
use subscripts to remind ourselves of the class we are operating in.

Let $c$ be a positive integer and $\pi_c$ be the set of all primes
less than or equal to $c$. Let $S$ be a set. The algebra $A =
A_{c+1}$, the Lie ring $L = L_{c+1}$, and the group $F = F_{c+1}$ are
defined the same way as in Sections \ref{sec:malcev-lazard-free-setup}
and \ref{sec:malcev-lazard-correspondence-interpretation}, but
replacing $c$ by $c + 1$. In particular, $A$ is a free
$\Q$-algebra on $S$ of nilpotency class $c + 1$, $L$ is the Lie
subring of $A_{c+1}$ generated by $S$, and $D$ is the group
generated by $e^{x_i}, x_i \in S$.

Let $K$ be the group $\sqrt[\pi_c]{F}$ and $N$ be the group
$\mathbb{Z}[\pi_c^{-1}](L)$. As shown in Theorem
\ref{thm:free-nilpotent}, $K$ is a free $\pi_c$-powered class $(c +
1)$ group on $e^S$ (and therefore canonically isomorphic to the free
$\pi_c$-powered class $(c + 1)$ group on $S$), and $N$ is a free
$\pi_c$-powered class $(c + 1)$ Lie ring on $S$.

Before proceeding, we note a few basic facts that we will repeatedly
use in the course of our proofs.

\begin{itemize}
\item $\sqrt{F} = \sqrt{K}$ is the free $\Q$-powered group on $e^S$.
\item $\Q N = \Q L$ is the free $\Q$-powered Lie ring on $S$.
\end{itemize}

Our eventual goal will be to show that the exponential and logarithm
map establish a Lazard corespondence between $[N,N]$ and $[K,K]$. We
will begin by establishing some preliminaries.

\begin{lemma}\label{lemma:lazard-correspondence-derived-setup}
  With notation as above, the following are true:

  \begin{enumerate}
  \item $[N,N]$ is a global class $c$ Lazard Lie ring, i.e., it is a
    $\pi_c$-powered Lie ring of nilpotency class at most $c$.
  \item $[K,K]$ is a global class $c$ Lazard Lie group, i.e., it is a
    $\pi_c$-powered group of nilpotency class at most $c$.
  \item Under the exponential map from $A$ to $1 + A$, $\exp([N,N])$
    is a $\pi_c$-powered subgroup of $1 + A$ (in fact, it is a
    subgroup of $\sqrt{F}$). Moreover, the exponential and logarithm
    maps establish a Lazard correspondence up to isomorphism between
    $[N,N]$ and $\exp([N,N])$.
  \item Under the logarithm map from $1 + A$ to $A$, $\log([K,K])$ is
    a $\pi_c$-powered Lie subring of $A$ (in fact, it is a Lie subring
    of $\Q L$). Moreover, the exponential and logarithm maps
    establish a Lazard correspondence up to isomorphism between
    $\log([K,K])$ and $[K,K]$.
  \end{enumerate}
\end{lemma}

\begin{proof}
  We prove the parts one by one.

  \begin{enumerate}
  \item $[N,N]$ is a global class $c$ Lazard Lie ring: 
    \begin{itemize}
    \item $[N,N]$ is $\pi_c$-powered: This follows from the fact that
      $N$ is $\pi_c$-powered and Lemma
      \ref{lemma:lie-ring-lcs-divisibility}.
    \item $[N,N]$ has nilpotency class at most $c$: In fact, since $N$
      has nilpotency class $c + 1$, $[N,N]$ has nilpotency class at
      most $\lfloor (c + 1)/2 \rfloor$, which is less than or equal to
      $c$.
    \end{itemize}
  \item $[K,K]$ is a global class $c$ Lazard Lie group:
    
    \begin{itemize}
    \item $[K,K]$ is $\pi_c$-powered: The follows from the fact that
      $K$ is $\pi_c$-powered and Theorem \ref{thm:powering-lcs}.
    \item $[K,K]$ has nilpotency class at most $c$: In fact, since $K$
      has nilpotency class $c + 1$, $[K,K]$ has nilpotency class at
      most $\lfloor (c + 1)/2 \rfloor$, which is less than or equal to
      $c$.
    \end{itemize}

  \item This follows directly from (1) and applying the global Lazard
    correspondence between {\em subrings and subgroups} described in
    Section \ref{sec:global-lazard-correspondence-subgroups}, to the
    Lazard correspondence between $\Q L$ and $\sqrt{F}$.
  \item This follows directly from (2) and applying the global Lazard
    correspondence between {\em subrings and subgroups} described in
    Section \ref{sec:global-lazard-correspondence-subgroups}, to the
    Lazard correspondence between $\Q L$ and $\sqrt{F}$.
  \end{enumerate}
\end{proof}

\begin{lemma}\label{lemma:exp-log-error-term}
  With notation as above, the following are true (recall that
  $Z(\sqrt{K})$ is the center of $\sqrt{K} = \sqrt{F}$):

  \begin{enumerate}
  \item Under the exponential map $\exp: \Q L = \Q N \to \sqrt{F} =
    \sqrt{K}$ (which in turn is obtained from the exponential map
    $\exp:A \to 1 + A$), we have $\exp(N) \subseteq
    KZ(\sqrt{K})$. 
  \item Under the logarithm map $\log: \sqrt{F} = \sqrt{K} \to \Q L =
    \Q N$ (which in turn is obtained from the logarithm map $\log:1 +
    A \to A$), we have $\log(K) \subseteq N + Z(\Q N)$.
  \item Under the exponential map $\exp: \Q L = \Q N \to \sqrt{F} =
    \sqrt{K}$ (which in turn is obtained from the exponential map
    $\exp:A \to 1 + A$), we have $K \subseteq
    \exp(N)Z(\sqrt{K})$. 
  \item Under the logarithm map $\log: \sqrt{F} = \sqrt{K} \to \Q L =
    \Q N$ (which in turn is obtained from the logarithm map $\log:1 +
    A \to A$), we have $N \subseteq \log(K) + Z(\Q N)$.

  \end{enumerate}
\end{lemma}

\begin{proof}
  Consider the quotient map $A \to A/A^{c+1}$. This map factors out
  the products of length $c + 1$ and the quotient is therefore the
  free nilpotent associative algebra of nilpotency class $c$ on
  $S$. Based on the setup described in Section
  \ref{sec:free-nilpotent-exp-log}, the image of $F$ under the
  quotient map is $F_c$ (the class $c$ version of $F$) and the image
  of $L$ under the quotient map is $L_c$ (the class $c$ version of
  $L$). Denote by $K_c$ and $N_c$ the images of $K$ and $N$ under the
  quotient map. Then, $K_c = \sqrt[\pi_c]{F_c}$ and $N_c =
  \Z[\pi_c^{-1}]L_c$, so that $K_c$ and $N_c$ are in global class $c$
  Lazard correspondence up to isomorphism.

  {\em Proof of (1)}: For $x \in N$, denote by $\overline{x}$ the
  image of $x$ in $N_c$. Then, $\exp(\overline{x}) =
  \overline{\exp(x)} \in K_c$, so $\exp(x)$ can be written in the form
  $g + a$ where $g \in K$ and $a$ is in the $(c+1)^{th}$ graded
  component of $A$. This can be rewritten as $g(1 + g^{-1}a)$. The
  element $1 + g^{-1}a$ is an element of $\sqrt{F} = \sqrt{K}$, and
  $g^{-1}a$ is in the $(c+1)^{th}$ graded component of $A$, so that $1
  + g^{-1}a \in Z(\sqrt{F}) = Z(\sqrt{K})$.

  {\em Proof of (2)}: For $g \in K$, denote by $\overline{g}$ the
  image of $g$ in $K_c$. Then, $\log(\overline{x}) =
  \overline{\log(x)} \in N_c$, so $\log x$ can be written in the form
  $x + a$ where $x \in N$ and $a$ is in the $(c+1)^{th}$ graded
  component of $A$, so that $a \in Z(\Q L) = Z(\Q N)$.

  {\em Proof of (3)}: For $g \in K$, part (2) says that we can write
  $\log g = x +a$ where $a$ is central. Exponentiating both sides, we
  obtain that $g = e^xe^a$, with $e^a \in Z(\sqrt{K})$.

 {\em Proof (4)}: For $x \in N$, part (1) says that we can write $e^x
 = gu$ where $g \in K$ and $u \in Z(\sqrt{K})$. Taking logarithms both
 sides, and using the centrality of $u$, we obtain that $x = \log g +
 \log u$, with $\log u \in Z(\Q N)$.
\end{proof} 

\begin{theorem}[Lazard correspondence between derived subgroup and derived subring]\label{thm:lazard-correspondence-derived}
  With notation as above, the derived subgroup $[K,K]$ and the derived
  subring $[N,N]$ are in Lazard correspondence up to
  isomorphism. Moreover, this Lazard correspondence arises as a
  restriction of the Malcev correspondence between $\sqrt{F} =
  \hat{K}$ and $\Q N = \Q L$ described in Section
  \ref{sec:malcev-lazard-correspondence-interpretation}.
\end{theorem}

Before we begin the proof, we make a few remarks. First, note that $K$
and $N$ are very close to being in Lazard correspondence
themselves. If we had inverted all primes in $\pi_{c+1}$, then we
would have obtained the Lazard correspondence. This also means that if
$c + 1$ is composite, then $K$ and $N$ {\em are} in Lazard
correspondence. In that case, the result follows from Theorem
\ref{thm:lazard-correspondence-lcs}.

The case of interest is where $c + 1$ is prime, so that $K$ and $N$
are not in Lazard correspondence themselves. In other words, when we
compute $e^u$ for some $u \in N$, we do obtain an element of
$\sqrt{K}$ but not necessarily an element of $K$. We want to show that
if the element of $N$ that we start with is in $[N,N]$, then computing
the exponential gives an element in $[K,K]$, and that every element of
$[K,K]$ can be obtained in this fashion.

We are now in a position to begin the proof.

\begin{proof}
  By Lemma \ref{lemma:lazard-correspondence-derived-setup}, it
  suffices to show that $\exp([N,N]) = [K,K]$, or equivalently, that
  $\log([K,K]) = [N,N]$. We can show this in two steps: showing that
  $[K,K] \subseteq \exp([N,N])$ and showing that $[N,N] \subseteq
  \log([K,K])$.

  \begin{enumerate}
  \item Proof that $[K,K] \subseteq \exp([N,N])$: Lemma
    \ref{lemma:lazard-correspondence-derived-setup} established that
    $\exp([N,N])$ is a group. Thus, it suffices to show that a
    generating set of $[K,K]$ is in $\exp([N,N])$. Specifically, it
    suffices to show that for all $g, h \in K$, $[g,h] \in
    \exp([N,N])$.

    By Lemma \ref{lemma:exp-log-error-term} (part (3)), $g = e^xu$ and
    $h = e^yv$ where $x,y \in N$ and $u,v \in Z(\sqrt{K})$. Thus,
    $[g,h] = [e^x,e^y] = e^{M_{c+1}(x,y)}$. By Lemma
    \ref{lemma:commutator-denominators}, $M_{c+1}$ uses only division
    by primes in $\pi_c$, so that $[g,h] \in \exp([N,N])$.

  \item Proof that $[N,N] \subseteq \log([K,K])$: We showed in Lemma
    \ref{lemma:lazard-correspondence-derived-setup} that $\log([K,K])$
    is a Lie ring. Thus, it suffices to show that a generating set of
    $[N,N]$ is in $\log([K,K])$.  Specifically, it suffices to show
    that for $x,y \in N$, $[x,y] \in \log([K,K])$.

    By Lemma \ref{lemma:exp-log-error-term} (part (4)), $x = \log g +
    s$ and $y = \log k + t$ where $g, k \in K$ and $s,t \in Z(\Q
    N)$. Thus, $[x,y] = [\log g, \log k] = \log(h_{2,c+1}(g,k))$. By
    Lemma \ref{lemma:lie-bracket-denominators}, $h_{2,c+1}(g,k) \in
    [K,K]$, so $[x,y] \in \log([K,K])$.
  \end{enumerate}
\end{proof}

In later applications of the result, we will use the result in its
{\em abstract} form, i.e., instead of treating $K$ and $N$ as subsets
inside $\Q + A$ as we did above, we will treat them as the abstract
free $\pi_c$-powered group and Lie ring respectively of class $c + 1$.

\begin{lemma}\label{lemma:exp-log-error-term-sub}
  Continuing notation from the preceding theorem (Theorem
  \ref{thm:lazard-correspondence-derived}), let $K_1 =
  K/\gamma_c(K)$ and $N_1 = N/\gamma_c(N)$. Let $R$ be the
  $\pi_c$-powered normal subgroup of $K$ containing $\gamma_c(K)$
  while $J$ is a $\pi_c$-powered ideal of $N$ containing
  $\gamma_c(N)$. Let $R_1 = R/\gamma_c(K)$ and $J_1 =
  J/\gamma_c(N)$. Further, suppose that $R_1 = \exp(J_1)$ with respect
  to the Lazard correspondence given by the usual logarithm and
  exponential map between $N_1$ and $K_1$. Then, the following are true:

  \begin{enumerate}
  \item Under the exponential map $\exp: \Q L = \Q N \to \sqrt{F} =
    \sqrt{K}$ (which in turn is obtained from the exponential map
    $\exp:A \to 1 + A$), we have $\exp(J) \subseteq
    RZ(\sqrt{K})$. 
  \item Under the logarithm map $\log: \sqrt{F} = \sqrt{K} \to \Q L =
    \Q N$ (which in turn is obtained from the logarithm map $\log:1 +
    A \to A$), we have $\log(R) \subseteq J + Z(\Q N)$.
  \item Under the exponential map $\exp: \Q L = \Q N \to \sqrt{F} =
    \sqrt{K}$ (which in turn is obtained from the exponential map
    $\exp:A \to 1 + A$), we have $R \subseteq
    \exp(J)Z(\sqrt{K})$. 
  \item Under the logarithm map $\log: \sqrt{F} = \sqrt{K} \to \Q L =
    \Q N$ (which in turn is obtained from the logarithm map $\log:1 +
    A \to A$), we have $J \subseteq \log(R) + Z(\Q N)$.

  \end{enumerate}
\end{lemma}

\begin{proof}
  The proof is analogous to the proof of Lemma
  \ref{lemma:exp-log-error-term}.
\end{proof}

\begin{theorem}\label{thm:lazard-correspondence-intersect-kernel}
  Continuing notation from Lemma \ref{lemma:exp-log-error-term-sub},
  the Lazard correspondence between $[K,K]$ and $[N,N]$ restricts to
  a Lazard correspondence between $R \cap [K,K]$ and $J \cap [N,N]$.
\end{theorem}

Before we begin the proof, note that $R$ and $J$ are not themselves
necessarily in Lazard correspondence. They {\em would be} in Lazard
correspondence if $c + 1$ were composite, because in that case, $\pi_c
= \pi_{c+1}$. We want to show here that even though $R$ and $J$ may
themselves fail to be in Lazard correspondence, their respective
intersections with $[K,K]$ and $[N,N]$ are in Lazard correspondence.

\begin{proof}
  %% Without loss of generality (for the proof, not for the statement),
  %% we can assume here that $R$ is contained in $[K,K]$ and $J$ is
  %% contained in $[N,N]$ (if not, replace $R$ by $R \cap [K,K]$ and $J$
  %% by $J \cap [N,N]$, modifying the definitions of $R_1$ and $N_1$
  %% accordingly).
  We prove the result in several steps:

  \begin{enumerate}
  \item $R \cap [K,K]$ is a global class $c$ Lazard Lie group: By
    Lemma \ref{lemma:lazard-correspondence-derived-setup}, $[K,K]$ is
    a global class $c$ Lazard Lie group. In particular, it is
    $\pi_c$-powered and has nilpotency class at most $c$. $R$ is given
    to be $\pi_c$-powered. Thus, $R \cap [K,K]$ is $\pi_c$-powered,
    and its nilpotency class is at most $c$. Note that the only bound
    we have on the nilpotency class of $R$ is $c + 1$, but bounding the
    class of $[K,K]$ is sufficient to bound the class of $R \cap
    [K,K]$.
  \item $J \cap [N,N]$ is a global class $c$ Lazard Lie ring: By Lemma
    \ref{lemma:lazard-correspondence-derived-setup}, $[N,N]$ is a
    global class $c$ Lazard Lie ring. In particular, it is
    $\pi_c$-powered and has nilpotency class at most $c$. $J$ is
    $\pi_c$-powered. Thus, $J \cap [N,N]$ is $\pi_c$-powered, and its
    nilpotency class is at most $c$. Note that the only bound we have
    on the nilpotency class of $J$ is $c + 1$, but bounding the class
    of $[N,N]$ is sufficient to bound the class of $J \cap [N,N]$.
  \item $\exp(J \cap [N,N]) \subseteq R \cap [K,K]$: We know that
    $\exp([N,N]) = [K,K]$, so $\exp(J \cap [N,N]) \subseteq [K,K]$. In
    addition, $\exp(J \cap [N,N]) \subseteq \exp(J)$. By Lemma
    \ref{lemma:exp-log-error-term-sub} (part (1)), we obtain that
    $\exp(J \cap [N,N]) \subseteq RZ(\sqrt{K})$, and hence $\exp(J
    \cap [N,N]) \subseteq RZ(\sqrt{K}) \cap [K,K]$. We will show that
    $RZ(\sqrt{K}) \cap [K,K] = R \cap [K,K]$, completing the proof.

    Consider an element $g \in RZ(\sqrt{K}) \cap [K,K]$. Then, $g =
    uv$ with $u \in R$ and $v \in Z(\sqrt{K})$. We obtain that $v =
    u^{-1}g$. Since both $R$ and $[K,K]$ are subgroups of $K$, we
    obtain that $v \in K$, so $v \in Z(\sqrt{K}) \cap K$, so that $v
    \in Z(K)$. But $Z(K) = \gamma_c(K)$ (both are precisely the
    elements in $K$ that have the form $1 + a$ with $a$ a homogeneous
    degree $c + 1$ element of $A$), so $v \in \gamma_c(K) \subseteq
    R$. Thus, $g = uv \in R$, so that $g \in R \cap [K,K]$ as desired.
  \item $\log(R \cap [K,K]) \subseteq J \cap [N,N]$: We know that
    $\log([K,K])$ $= [N,N]$, so $\log(R \cap [K,K]) \subseteq
    [N,N]$). In addition, $\log(R \cap [K,K]) \subseteq \log(R)$. By
    Lemma \ref{lemma:exp-log-error-term-sub} (part (2)), we obtain
    that $\log(R \cap [K,K]) \subseteq J + Z(\Q N)$, and hence $\log(R
    \cap [K,K]) \subseteq (J + Z(\Q N)) \cap [N,N]$. We will now show
    that $(J + Z(\Q N)) \cap [N,N] = J \cap [N,N]$, completing the
    proof.

    Consider an element $x \in (J + Z(\Q N)) \cap [N,N]$. Then, $x= y
    + z$ with $y \in J$ and $z \in Z(\Q N)$. We obtain that $z = x -
    y$. Since both $J$ and $[N,N]$ are subrings of $N$, we obtain that
    $z \in N$, so $z \in Z(\Q N) \cap N$, so $z \in Z(N)$. We have
    $Z(N) = \gamma_c(N)$, so $z \in \gamma_c(N) \subseteq J$. Thus, $x
    = y + z \in J$, so $x \in J \cap [N,N]$ as desired.
  \end{enumerate}
\end{proof}

\begin{theorem}\label{thm:lazard-correspondence-commutator-lie}
  Continuing notation from the preceding theorem (Theorem
  \ref{thm:lazard-correspondence-derived}), the Lazard correspondence
  between $[K,K]$ and $[N,N]$ restricts to a Lazard correspondence
  between $[K,R]$ and $[N,J]$.
\end{theorem}

\begin{proof}
  \begin{enumerate}
  \item $[K,R]$ is a global class $c$ Lazard Lie group and $[N,J]$ is a
    global class $c$ Lazard Lie ring: This requires verifying that
    both $[K,R]$ and $[N,J]$ are $\pi_c$-powered, and both have class
    at most $c$. We break this down further:

    \begin{itemize}
    \item $[K,R]$ is a $\pi_c$-powered group: This follows from Lemma
      \ref{lemma:lcs-subgroup-divisibility}.
    \item $[N,J]$ is a $\pi_c$-powered Lie ring: This follows from $N$
      and $J$ both being $\pi_c$-powered, and the additivity of the
      Lie bracket.
    \item $[K,R]$ has nilpotency class at most $c$: $[K,R]$ is a
      subgroup of $[K,K]$, which has nilpotency class at most $c$ as
      established in Theorem \ref{thm:lazard-correspondence-derived}.
    \item $[N,J]$ has nilpotency class at most $c$: $[N,J]$ is a
      subring of $[N,N]$, which has nilpotency class at most $c$ as
      established in Theorem \ref{thm:lazard-correspondence-derived}.
    \end{itemize}

  \item $\exp([N,J])$ is a group with which $[N,J]$ is in global
    class $c$ Lazard correspondence up to isomorphism via $\exp$, and
    $\log([K,R])$ is a Lie ring that is in global class $c$ Lazard
    correspondence with it via $\exp$: These follow from Step (1) and
    the fact that the exponential and logarithm maps establish
    correspondences between global class $c$ Lazard Lie subgroups and
    global class $c$ Lazard Lie subrings.

  \item $[K,R] \subseteq \exp([N,J])$ and $[N,J] \subseteq
    \log([K,R])$: We show both parts:

    \begin{itemize}
    \item $[K,R] \subseteq \exp([N,J])$: In Step (2), we established
      that $\exp([N,J])$ is a group. Thus, it suffices to show that
      every commutator between an element of $K$ and an element of $R$
      is in $\exp([N,J])$. Explicitly, given $g \in K$ and $h \in R$,
      we need to show that $[g,k] \in \exp([N,J])$. 

      By Lemma \ref{lemma:exp-log-error-term} (part (3)) and Lemma
      \ref{lemma:exp-log-error-term-sub} (part (3)), $g = e^xu$ and $h
      = e^yv$ where $x \in N$, $y \in J$ and $u,v \in
      Z(\sqrt{K})$. Thus, $[g,h] = [e^x,e^y] = e^{M_{c+1}(x,y)}$. By
      Lemma \ref{lemma:commutator-denominators}, $M_{c+1}$ uses only
      division by primes in $\pi_c$, so that $[g,h] \in \exp([N,J])$.

    \item $[N,J] \subseteq \log([K,R])$: In Step (2), we established
      that $\log([K,R])$ is a Lie ring. Thus, it suffices to show that
      a generating set of $[N,J]$ is in $\log([K,R])$.  Specifically,
      it suffices to show that for $x \in N$ and $y \in J$, $[x,y] \in
      \log([K,R])$.

      By Lemma \ref{lemma:exp-log-error-term} (part (4)) and Lemma
      \ref{lemma:exp-log-error-term-sub} (part (4)), $x = \log g + s$
      and $y = \log k + t$ where $g \in K$, $k \in R$, and $s,t \in
      Z(\Q N)$. Thus, $[x,y] = [\log g, \log k] =
      \log(h_{2,c+1}(g,k))$. By Lemma
      \ref{lemma:lie-bracket-denominators}, $h_{2,c+1}(g,k) \in
          [K,R]$, so $[x,y] \in \log([K,R])$.
    \end{itemize}
  \end{enumerate}

  Combining these, we obtain the result.
\end{proof}


%\newpage

\section{Homology of powered nilpotent groups}\label{sec:homology-of-powered-nilpotent-groups}

\subsection{Hopf's formula variant for Schur multiplier of powered nilpotent group}\label{sec:hopf-formula-pi-powered}

So far, we have studied the extension theory and homology theory of
groups {\em qua} groups. We now consider the extension theory and
homology theory in the context of $\pi$-powered groups. For the
results presented throughout this section, $\pi$ is {\em any} set of
primes. Later, we will apply the results to the case where $\pi =
\pi_c$ is the set of all primes less than or equal to $c$, but the
results for this section are not restricted to such sets of primes.

It turns out that, in the nilpotent case, the Schur multiplier, Schur
covering group, exterior square, and all related constructions {\em
  qua} group are the same as the corresponding constructions {\em qua}
$\pi$-powered group. This is specific to the nilpotent case.

\begin{lemma}\label{lemma:homology-pi-powered}
  Suppose $G$ is a $\pi$-powered nilpotent group. Then, the following
  hold:

  \begin{enumerate}
    \item All the homology groups $H_i(G;\mathbb{Z})$, $i > 0$, are
      $\pi$-powered abelian groups.
    \item The Schur multiplier $M(G)$, which is $H_2(G;\mathbb{Z})$, is
      a $\pi$-powered abelian group.
  \end{enumerate}
\end{lemma}

\begin{proof}
  {\em Proof of (1)}: This follows from May and Ponto's text,
  \cite{ConciseII}, Theorem 6.1.1, implication (iii) $\implies$ (iv),
  setting $Z = K(G,1)$ and $T$ as the complement of $\pi$ in the set
  of primes. The result appeared earlier in
  \cite{HiltonMislinRoitberg}, Theorem 2.9. However, the notation and
  surrounding explanation in May and Ponto's text are easier to
  follow. 

  {\em Proof of (2)}: This follows immediately from (1).
\end{proof}

\begin{lemma}\label{lemma:schur-cover-pi-powered}
  Suppose $G$ is a $\pi$-powered nilpotent group. Every Schur covering
  group of $G$ is $\pi$-powered. Since Schur covering groups exist and
  are by definition initial objects in the category of central
  extensions of $G$ with homoclinisms, there exist initial objects in
  the category of extensions of $G$ with homoclinisms that are
  $\pi$-powered.
\end{lemma}

\begin{proof}
  By the preceding lemma (Lemma \ref{lemma:homology-pi-powered}),
  $M(G) = H_2(G;\mathbb{Z})$ is $\pi$-powered. Consider a Schur
  covering group $E$ of $G$. $E$ is a central extension of the form:

  $$0 \to M(G) \to E \to G \to 1$$

  By Lemma \ref{lemma:powering-extension-group}, $E$ is also
  $\pi$-powered. By definition, any Schur covering group gives an
  inital object in the category of central extensions of $G$ with
  homoclinisms, so the last part of the statement follows.

  Schur covering groups exist by Theorem
  \ref{thm:schur-covering-groups-exist}, so the final sentence
  follows.
\end{proof}

We are now in a position to prove results that establish analogues of
Hopf's formula (discussed in Section \ref{sec:hopf-formula-proofs}):

\begin{theorem}
  Suppose $\pi$ is a set of primes and $G$ is a $\pi$-powered
  nilpotent group. Express $G$ in the form $G \cong F/R$ where $F$ is
  a free $\pi$-powered group. Then, the following are true.

  \begin{enumerate}
  \item The group $E_1 = F/[F,R]$, with the natural quotient map $E_1
    \to G$, is an initial object in the category of central extensions
    of $G$ with homoclinisms.
  \item The exterior square $G \wedge G$ is canonically isomorphic to
    the quotient group $[F,F]/[F,R]$.
  \item The Schur multiplier $M(G)$ is canonically isomorphic to the
    quotient group $(R \cap [F,F])/[F,R]$:

    \begin{equation}\label{eq:pi-powered-hopf-formula}
      M(G) \cong (R \cap [F,F])/[F,R]
    \end{equation}
  \end{enumerate}
\end{theorem}

\begin{proof}
  {\em Proof of (1)}: Let $E_2$ be a Schur covering group of $G$. By
  Lemma \ref{lemma:schur-cover-pi-powered}, $E_2$ is $\pi$-powered.

  Consider the commutator map $\omega_1:G \times G \to E_1$ and denote
  by $\Omega_1$ the corresponding group homomorphism:

  $$\Omega_1: G \wedge G \to [E_1,E_1]$$

  Consider also the extension:

  $$0 \to A \to E_2 \stackrel{\mu}{\to} G \to 1$$
  
  with the natural commutator map $\omega_2:G \times G \to E_2$ and the
  corresponding commutator map homomorphism:

  $$\Omega_2:G \wedge G \to [E_2,E_2]$$
  
  Our goal is to show that there there exists a unique homomorphism
  $\varphi:[E_1,E_1] \to [E_2,E_2]$ such that $\varphi \circ
  \omega_1 = \omega_2$, or equivalently, $\varphi \circ \Omega_1 =
  \Omega_2$.

  The map $\nu:F \to G$ lifts to a map $\psi:F \to E_2$ on account of
  $F$ being a free $\pi$-powered group and $E_2$ being a $\pi$-powered
  group (note that the lift is not necessarily unique). Explicitly,
  this means that $\mu \circ \psi = \nu$.

  We know that $\nu(R)$ is trivial, so $\mu(\psi(R))$ is trivial. Thus,
  $\psi(R)$ lands inside the kernel of $\mu$, which is the image of $A$
  in $E_2$. Thus, $\psi(R)$ is a central subgroup of $E_2$. Therefore,
  $\psi([F,R]) = [\psi(F),\psi(R)]$ is trivial.
  
  Thus, $\psi$ descends to a homomorphism $\theta:E_1 \to E_2$, where
  $E_1 = F/[F,R]$ as defined above, with the property that $\mu \circ
  \theta = \overline{\nu}$. The condition $\mu \circ \theta =
  \overline{\nu}$ can be interpreted as saying that $\theta$ is a
  homomorphism between the central extensions $(E_1,\overline{\nu})$ and
  $(E_2,\mu)$. Denote by $\varphi:E_1' \to E_2'$ the restriction of
  $\theta$ to $E_1'$. Thus, by Lemma
  \ref{lemma:homomorphism-restriction-homoclinism}, $\varphi$ defines a
  homoclinism of the central extensions. 

  Now, we know that $(E_2,\mu)$ is an initial object in the category,
  so there is a homoclinism from it to any other object in the
  category. Composing with the above gives a homoclinism from
  $(E_1,\overline{\nu})$ to any central extension of $G$. By Lemma
  \ref{lemma:uniqueness-of-homoclinism}, this homoclinism is
  unique. Finally, Lemma \ref{lemma:initobj} establishes from this that
  $(E_1,\overline{\nu})$ defines an initial object in the category of central
  extensions of $G$ with homoclinisms.

  {\em Proof of (2)}: By definition, $G \wedge G$ is isomorphic to the
  derived subgroup $[E_1,E_1]$ on account of $E_1$ being an initial
  object in the category of central extensions with
  homoclinisms. $[E_1,E_1]$ is canonically isomorphic to
  $[F,F]/[F,R]$.

  {\em Proof of (3)}: The Schur multiplier is the kernel of the map
  $[E_1,E_1] \to [G,G]$, or equivalently, it is the kernel of the map
  $[F,F]/[F,R] \to [F,F]/(R \cap [F,F])$. The kernel works out to $(R
  \cap [F,F])/[F,R]$.
\end{proof}


\subsection{Hopf's formula variant for Schur multiplier of powered nilpotent group: class one more version}\label{sec:hopf-formula-pi-powered-class-one-more}

Suppose $\pi$ is a set of primes and $G$ is a $\pi$-powered nilpotent
group of nilpotency class $c$. Suppose $G$ can be expressed in the
form $F/R$ where $F$ is a free $\pi$-powered nilpotent group of class
$c + 1$ and $R$ is a normal subgroup of $F$. Then:

\begin{equation}\label{eq:pi-powered-hopf-formula-class-one-more}
  M(G) \cong (R \cap [F,F])/([F,R])
\end{equation}

Also:

\begin{equation}\label{eq:pi-powered-exteriorsquare-hopf-formula-class-one-more}
  G \wedge G \cong [F,F]/[F,R]
\end{equation}

Our application of these results, in Section \ref{sec:lcuti}, will
combine these results with the results of Section
\ref{sec:malcev-lazard-free}. In that section, we use the letter $K$
for the free group, and $R$ for the normal subgroup that is being
factored out. Therefore, in our application of the above theorem, the
letter $K$ will appear instead of the letter $F$.

\subsection{Remark on powered Schur multipliers and powered Schur covering groups}

It is possible to define concepts of Schur multiplier, Schur covering
group, and exterior square, all {\em within the variety of
  $\pi$-powered groups}. In other words, we can mimic the steps of
Section \ref{sec:exteriorsquare-and-homoclinism}, replacing ``group''
by $\pi$-powered group'', and obtain corresponding notions of
$\pi$-powered exterior square and $\pi$-powered Schur multiplier. We
can also define $\pi$-powered Schur covering group.

The results of Sections \ref{sec:hopf-formula-pi-powered} and
\ref{sec:hopf-formula-pi-powered-class-one-more} tell us that for
nilpotent groups, the $\pi$-powered Schur multiplier coincides with
the usual Schur multiplier and the $\pi$-powered exterior square
coincides with the usual exterior square.

\subsection{Results about isoclinisms for $\pi$-powered central extensions}\label{sec:isoclinisms-pi-powered-central-extensions}

The results here follow from the results of Section
\ref{sec:pi-powered-isoclinism-results} using reasoning similar to the
way the results in Section
\ref{sec:homoclinisms-words-central-extensions} follow from the
results in Section \ref{sec:homoclinism-misc-results}. Note that we
restrict our statements to $\pi$-powered class $c$ nilpotent groups
and $\pi$-powered class $c$ words. Note that although the result is
stated for $\pi$-powered class $c$ words, it also applies to
$\pi$-powered words in general, because a $\pi$-powered class $c$ word
can be viewed as an equivalence class of $\pi$-powered words that are
equal in any $\pi$-powered class $c$ group.

\begin{lemma}\label{lemma:iterated-commutator-descends-extension-version-pi-powered}
  Suppose $c \ge 1$ and $\pi$ is a set of primes. Suppose $G$ is a
  $\pi$-powered group and $w(g_1,g_2,\dots,g_n)$ is a $\pi$-powered
  class $c$ word in $n$ letters with the property that $w$ evaluates
  to the identity element in every $\pi$-powered abelian group. The
  following are true.

  \begin{enumerate}
  \item For every $\pi$-powered central extension $E$ of $G$ such that
    $E$ has class at most $c$, $w$ can be used to define a set map
    $\chi_{w,E}: G^n \to [E,E]$.
  \item For any homoclinism between $\pi$-powered central extensions
    $E_1$ and $E_2$, with the central extension specified via a
    homomorphism $\varphi:[E_1,E_1] \to [E_2,E_2]$, we have that:

    $$\varphi \circ \chi_{w,E_1} = \chi_{w,E_2}$$
  \end{enumerate}
\end{lemma}

%%TONOTDO: Possibly insert more detail here

\begin{proof}
  {\em Proof of (1)}: This is similar to the proof of Theorem
  \ref{thm:iterated-commutator-descends-to-inn-pi-powered}. Alternatively, we can
  deduce it from the {\em result} of Theorem
  \ref{thm:iterated-commutator-descends-to-inn-pi-powered} by noting that the
  map factors as follows:

  $$G^n \to (E/Z(E))^n \to [E,E]$$

  {\em Proof of (2)}: This is similar to the proof of Theorem
  \ref{thm:iterated-commutator-commutes-homoclinisms-pi-powered}. Alternatively,
  we can deduce it from the {\em result} of Theorem
  \ref{thm:iterated-commutator-commutes-homoclinisms-pi-powered} by factoring
  through $E/Z(E)$.
\end{proof}

We can now prove the theorem.

\begin{theorem}\label{thm:iterated-commutator-map-to-exteriorsquare-pi-powered}
  Suppose $\pi$ is a set of primes. Suppose $G$ is a $\pi$-powered
  nilpotent group and $w(g_1,g_2,\dots,g_n)$ is a $\pi$-powered word
  in $n$ letters with the property that $w$ evaluates to the identity
  element in every $\pi$-powered abelian group. Then, there exists a
  set map $X_w:G^n \to G \wedge G$ with the property that for any
  $\pi$-powered central extension $E$ of $G$, $\Omega_{E,G} \circ X_w
  = \chi_{w,E}$.
\end{theorem}

\begin{proof}
  Set $c$ to be one more than the nilpotency class of $G$, and treat
  $w$ as a $\pi$-powered class $c$ word. Apply Part (1) of Lemma
  \ref{lemma:iterated-commutator-descends-extension-version-pi-powered}
  to the case where the extension $E_0$ is an initial object in the
  category of central extensions of $G$, {\em and} $E_0$ is
  $\pi$-powered (this is possible by Lemma
  \ref{lemma:schur-cover-pi-powered}). The rest of the proof is
  analogous to Theorem
  \ref{thm:iterated-commutator-map-to-exteriorsquare}.
\end{proof}

%\newpage

\section{Homology of powered Lie rings}\label{sec:homology-of-powered-nilpotent-lie-rings}

This section develops the Lie ring analogues of the results in Section
\ref{sec:homology-of-powered-nilpotent-groups}. One key difference is
that whereas the results of the previous section were restricted to
nilpotent groups, some of the results in this section apply in full
generality to all Lie rings.

\subsection{Hopf's formula for powered Lie rings}\label{sec:hopf-formula-pi-powered-lie}

We now establish the Lie ring analogues of the results describe in
Section \ref{sec:hopf-formula-pi-powered}.

\begin{lemma}\label{lemma:homology-pi-powered-lie}
  Suppose $L$ is a $\pi$-powered Lie ring. Then, the following hold:

  \begin{enumerate}
    \item All the homology groups $H_i(L;\mathbb{Z})$, $i > 0$, are
      $\pi$-powered abelian groups.
    \item The Schur multiplier $M(L)$, which is $H_2(L;\mathbb{Z})$, is
      a $\pi$-powered abelian group.
\end{enumerate}
\end{lemma}

\begin{proof}
  {\em Proof of (1)}: This is a direct consequence of how we define
  cohomology, and follows from the fact that the universal enveloping
  algebra is the same except in degree zero whether taken over
  $\mathbb{Z}$ or $\mathbb{Z}[\pi^{-1}]$, and the homology groups are
  defined in terms of the universal enveloping algebra. %% TONOTDO: I
  %% intend to provide more background regarding Lie ring cohomology
  %% in the (pending) Appendix section.

  {\em Proof of (2)}: This follows immediately from (1).
\end{proof}

\begin{lemma}\label{lemma:schur-cover-pi-powered-lie}
  Suppose $L$ is a $\pi$-powered  Lie ring. Every Schur covering
  Lie ring of $L$ is $\pi$-powered. Since Schur covering Lie rings exist and
  are by definition initial objects in the category of extensions of
  $L$ with homoclinisms, there exist initial objects in the category
  of extensions of $L$ with homoclinisms that are $\pi$-powered.
\end{lemma}

\begin{proof}
  By the preceding lemma (Lemma \ref{lemma:homology-pi-powered-lie}),
  $M(L)$ is $\pi$-powered. Consider a Schur covering Lie ring $N$ of
  $L$. $N$ is a central extension of the form:

  $$0 \to M(L) \to N \to L \to 1$$

  By Lemma \ref{lemma:two-out-of-three-lie}, $N$ is also
  $\pi$-powered. By definition, any Schur covering Lie ring gives an
  inital object in the category of central extensions of $L$ with
  homoclinisms, so the last part of the statement follows.

  Schur covering Lie rings exist by Theorem
  \ref{thm:schur-covering-lie-rings-exist}, so the final sentence
  follows.
\end{proof}

We are now in a position to prove results that establish analogues of
Hopf's formula (discussed in Section \ref{sec:hopf-formula-proofs}):

\begin{theorem}
  Suppose $\pi$ is a set of primes and $L$ is a $\pi$-powered
   Lie ring. Express $L$ in the form $L \cong F/R$ where $F$ is
  a free $\pi$-powered Lie ring. Then, the following are true.

  \begin{enumerate}
  \item The Lie ring $N_1 = F/[F,R]$, with the natural quotient map $N_1
    \to L$, is an initial object in the category of central extensions
    of $L$ with homoclinisms.
  \item The exterior square $L \wedge L$ is canonically isomorphic to
    the quotient Lie ring $[F,F]/[F,R]$.
  \item The Schur multiplier $M(L)$ is canonically isomorphic to the
    quotient Lie ring $(R \cap [F,F])/[F,R]$:

    \begin{equation}\label{eq:pi-powered-hopf-formula-lie}
      M(L) \cong (R \cap [F,F])/[F,R]
    \end{equation}
  \end{enumerate}
\end{theorem}

\begin{proof}
  {\em Proof of (1)}: Let $N_2$ be a Schur covering Lie ring of $L$. By
  Lemma \ref{lemma:schur-cover-pi-powered-lie}, $N_2$ is $\pi$-powered.

  Consider the commutator map $\omega_1:L \times L \to N_1$ and denote
  by $\Omega_1$ the corresponding Lie ring homomorphism:

  $$\Omega_1: L \wedge L \to [N_1,N_1]$$

  Consider also the extension:

  $$0 \to A \to N_2 \stackrel{\mu}{\to} L \to 1$$
  
  with the natural commutator map $\omega_2:L \times L \to N_2$ and the
  corresponding commutator map homomorphism:

  $$\Omega_2:L \wedge L \to [N_2,N_2]$$
  
  Our goal is to show that there there exists a unique homomorphism
  $\varphi:[N_1,N_1] \to [N_2,N_2]$ such that $\varphi \circ
  \omega_1 = \omega_2$, or equivalently, $\varphi \circ \Omega_1 =
  \Omega_2$.

  The map $\nu:F \to L$ lifts to a map $\psi:F \to N_2$ on account of
  $F$ being a free $pi$-powered Lie ring and $N_2$ being a $\pi$-powered
  Lie ring (note that the lift is not necessarily unique). Explicitly,
  this means that $\mu \circ \psi = \nu$.

  We know that $\nu(R)$ is trivial, so $\mu(\psi(R))$ is trivial. Thus,
  $\psi(R)$ lands inside the kernel of $\mu$, which is the image of $A$
  in $N_2$. Thus, $\psi(R)$ is a central Lie subring of $N_2$. Therefore,
  $\psi([F,R]) = [\psi(F),\psi(R)]$ is trivial.
  
  Thus, $\psi$ descends to a homomorphism $\theta:N_1 \to N_2$, where
  $N_1 = F/[F,R]$ as defined above, with the property that $\mu \circ
  \theta = \overline{\nu}$. The condition $\mu \circ \theta =
  \overline{\nu}$ can be interpreted as saying that $\theta$ is a
  homomorphism between the central extensions $(N_1,\overline{\nu})$ and
  $(N_2,\mu)$. Denote by $\varphi:N_1' \to N_2'$ the restriction of
  $\theta$ to $N_1'$. Thus, by Lemma
  \ref{lemma:homomorphism-restriction-homoclinism-lie}, $\varphi$ defines a
  homoclinism of the central extensions. 

  Now, we know that $(N_2,\mu)$ is an initial object in the category,
  so there is a homoclinism from it to any other object in the
  category. Composing with the above gives a homoclinism from
  $(N_1,\overline{\nu})$ to any central extension of $L$. By Lemma
  \ref{lemma:uniqueness-of-homoclinism-lie}, this homoclinism is
  unique. Finally, Lemma \ref{lemma:initobj-lie} establishes from this that
  $(N_1,\overline{\nu})$ defines an initial object in the category of central
  extensions of $L$ with homoclinisms.

  {\em Proof of (2)}: By definition, $L \wedge L$ is isomorphic to the
  derived subring $[N_1,N_1]$ on account of $N_1$ being an initial
  object in the category of central extensions with
  homoclinisms. $[N_1,N_1]$ is canonically isomorphic to
  $[F,F]/[F,R]$.

  {\em Proof of (3)}: The Schur multiplier is the kernel of the map
  $[N_1,N_1] \to [L,L]$, or equivalently, it is the kernel of the map
  $[F,F]/[F,R] \to [F,F]/(R \cap [F,F])$. The kernel works out to $(R
  \cap [F,F])/[F,R]$.
\end{proof}

\subsection{Hopf's formula variant for Schur multiplier of powered nilpotent Lie ring: class one more version}\label{sec:hopf-formula-pi-powered-class-one-more-lie}

Suppose $\pi$ is a set of primes and $L$ is a $\pi$-powered nilpotent
Lie ring of nilpotency class $c$. Suppose $L$ can be expressed in the
form $F/R$ where $F$ is a free $\pi$-powered nilpotent Lie ring of class
$c + 1$ and $R$ is an ideal of $F$. Then:

\begin{equation}\label{eq:pi-powered-hopf-formula-class-one-more-lie}
  M(L) \cong (R \cap [F,F])/([F,R])
\end{equation}

Also:

\begin{equation}\label{eq:pi-powered-exteriorsquare-hopf-formula-class-one-more-lie}
  L \wedge L \cong [F,F]/[F,R]
\end{equation}

Our application of these results, in Section \ref{sec:lcuti}, will
combine these results with the results of Section
\ref{sec:malcev-lazard-free}. In that section, we used the letter $N$
for the free Lie ring, and $J$ for the ideal is being factored
out. Therefore, in our application of the above theorem, the letter
$N$ will appear instead of the letter $F$, and the letter $J$ will
appear instead of the letter $R$.

\subsection{Sidenote on powered Schur multipliers and powered Schur covering Lie rings}

It is possible to define concepts of Schur multiplier, Schur covering
group, and exterior square, all {\em within the variety of
  $\pi$-powered Lie rings}. In other words, we can mimic the steps of
Section \ref{sec:exteriorsquare-and-homoclinism-lie}, replacing ``Lie
ring'' by $\pi$-powered Lie ring'', and obtain corresponding notions
of $\pi$-powered exterior square and $\pi$-powered Schur
multiplier. We can also define $\pi$-powered Schur covering Lie ring.

The results of Sections \ref{sec:hopf-formula-pi-powered-lie} and
\ref{sec:hopf-formula-pi-powered-class-one-more-lie} tell us that the
$\pi$-powered Schur multiplier coincides with the usual Schur
multiplier and the $\pi$-powered exterior square coincides with the
usual exterior square.

\subsection{Results about isoclinisms for $\pi$-powered central extensions of Lie rings}\label{sec:isoclinisms-pi-powered-central-extensions-lie}

The results here follow from the results of Section
\ref{sec:pi-powered-isoclinism-results} using reasoning similar to the
way the results in Section
\ref{sec:homoclinisms-words-central-extensions} follow from the
results in Section \ref{sec:homoclinism-misc-results}. Unlike the
corresponding section for groups (Section
\ref{sec:isoclinisms-pi-powered-central-extensions}), we do not
restrict ouselves to the nilpotent case. However, we {\em can}
formulate all our results for the nilpotent setting. The proofs remain
similar.

\begin{lemma}\label{lemma:iterated-bracket-descends-extension-version-pi-powered}
  Suppose $\pi$ is a set of primes. Suppose $L$ is a $\pi$-powered Lie
  ring and $w(g_1,g_2,\dots,g_n)$ is a $\pi$-powered Lie
  ring word in $n$ letters with the property that $w$ evaluates to the
  zero element in every abelian Lie ring. The following are true.

  \begin{enumerate}
  \item For every central extension $N$ of $L$, $w$ can be used to
    define a set map $\chi_{w,N}: L^n \to [N,N]$.
  \item For any homoclinism between central extensions $N_1$ and
    $N_2$, with the central extension specified via a homomorphism
    $\varphi:[N_1,N_1] \to [N_2,N_2]$, we have that:

    $$\varphi \circ \chi_{w,N_1} = \chi_{w,N_2}$$
  \end{enumerate}
\end{lemma}

%%TONOTDO: Possibly insert more detail here

\begin{proof}
  {\em Proof of (1)}: This is similar to the proof of Theorem
  \ref{thm:iterated-bracket-descends-to-inn-pi-powered}. Alternatively, we can
  deduce it from the {\em result} of Theorem
  \ref{thm:iterated-bracket-descends-to-inn-pi-powered} by noting that the
  map factors as follows:

  $$L^n \to (N/Z(N))^n \to [N,N]$$

  {\em Proof of (2)}: This is similar to the proof of Theorem
  \ref{thm:iterated-bracket-commutes-homoclinisms-pi-powered}. Alternatively,
  we can deduce it from the {\em result} of Theorem
  \ref{thm:iterated-bracket-commutes-homoclinisms-pi-powered} by factoring
  through $N/Z(N)$.
\end{proof}

We can now prove the theorem.

\begin{theorem}\label{thm:iterated-bracket-map-to-exteriorsquare-pi-powered}
  Suppose $\pi$ is a set of primes. Suppose $L$ is a $\pi$-powered Lie
  ring and $w(g_1,g_2,\dots,g_n)$ is a $\pi$-powered Lie ring word in
  $n$ letters with the property that $w$ evaluates to the identity
  element in every abelian Lie ring. Then, there exists a set map
  $X_w:L^n \to L \wedge L$ with the property that for any central
  extension $N$ of $L$, $\Omega_{N,L} \circ X_w = \chi_{w,N}$.
\end{theorem}

\begin{proof}
  Apply Part (1) of Lemma
  \ref{lemma:iterated-bracket-descends-extension-version-pi-powered} to
  the case where the extension $N_0$ is an initial object in the
  category of central extensions of $L$ and such that $N_0$ is
  $\pi$-powered. Such an extension exists by Lemma
  \ref{lemma:schur-cover-pi-powered-lie}. The rest of the proof is
  analogous to the proof of Theorem \ref{thm:iterated-bracket-map-to-exteriorsquare}.
\end{proof}


%\newpage

\section{Commutator-like and Lie bracket-like maps}\label{sec:commutator-and-lie-bracket-like-maps}

This section describes maps that can be used to compute a
``commutator-like'' expression for Lie rings whose nilpotency class is
one more than desired (or the $3$-local variant thereof), and
correspondingly, a ``Lie bracket-like'' expression for groups whose
nilpotency class is one more than desired (or the $3$-local variant
thereof). We use the formula $M_{c+1}$ (introduced in Section
\ref{sec:group-commutator-ito-lie-bracket}) to express the group
commutator as a $\pi_c$-powered word that makes use of the Lie
bracket. We use the formula $h_{2,c+1}$ (the second inverse
Baker-Campbell-Hausdorff formula) to express the Lie bracket as a
$\pi_c$-powered word that makes use of the group commutator.

For simplicity, we will restrict our definitions to situations where
the groups and Lie rings involved are {\em globally} nilpotent, but
without any restriction on their global nilpotency class. This will
allow us to apply the theorems stated about nilpotent groups and Lie
rings. It is possible to relax these assumptions to some extent, but
the ensuing greater generality will not be worth the additional cost
in the complexity of proofs.

\subsection{Commutator-like map for Lie ring whose class is one more}\label{sec:commutator-like-map}

Suppose $c$ is a natural number and $\pi_c$ is the set of primes less
than or equal to $c$.

Suppose $L$ is a $\pi_c$-powered nilpotent Lie ring such that the
inner derivation Lie ring $L/Z(L)$ is a $3$-local class $c$ Lazard Lie
ring. We can define a set map:

$$\omega_L^{\text{Group}}: \operatorname{Inn}(L) \times \operatorname{Inn}(L) \to [L,L]$$

The set map is defined as follows:

$$\omega^{\text{Group}}_L(x,y) := M_{c+1}(\tilde{x},\tilde{y})$$

where $M_{c+1}$ is the formula for the group commutator in terms of
the Lie bracket described in Section
\ref{sec:group-commutator-ito-lie-bracket}, and $\tilde{x}$,
$\tilde{y}$ are elements of $L$ that map to $x$ and $y$ respectively
in $L/Z(L)$. Note that
this is defined because of the following:

\begin{itemize}
\item The expression $M_{c+1}$ makes sense because $L$ is $\pi_c$-powered,
  and Lemma \ref{lemma:commutator-denominators} shows that all prime
  divisors of the denominators of coefficients for $M_{c+1}$ are in $\pi_c$.
\item The output of the expression is independent of the choice of
  lifts, because the Lie bracket map descends to a map $L/Z(L) \times
  L/Z(L) \to [L,L]$, and $M_{c+1}$ is obtained by using the Lie
  bracket and its iterations. See Theorem
  \ref{thm:iterated-bracket-descends-to-inn-pi-powered} for an explanation.
\end{itemize}

Further, in the case that $L$ itself is a $3$-local class $c$ Lazard
Lie ring, $\omega_L^{\text{Group}}(x,y)$ agrees with the
commutator-induced map $\omega_G: \operatorname{Inn}(G) \times
\operatorname{Inn}(G) \to G'$ where $G$ is the Lazard Lie group for
$L$. Recall that $\omega_G$ was originally defined in Section
\ref{sec:isoclinism-definition} along with the definition of
isoclinism.

This again follows from the definition of $M_{c+1}$ as introduced in
Section \ref{sec:group-commutator-ito-lie-bracket}.

We now show that the map $\omega^{\text{Group}}$ is an isoclinism-invariant.

\begin{lemma}\label{lemma:omega-group-isoclinism-invariant}
  Suppose $c$ is a natural number and $\pi_c$ is the set of primes less
  than or equal to $c$. Suppose $L_1$ and $L_2$ are isoclinic
  $\pi_c$-powered Lie rings, such that $\operatorname{Inn}(L_1) \cong
  \operatorname{Inn}(L_2)$ is a $3$-local class $c$ Lazard Lie ring.

  Suppose $(\zeta,\varphi)$ is an isoclinism from $L_1$ to $L_2$:
  $\zeta$ is the isomorphism $\operatorname{Inn}(L_1) \to
  \operatorname{Inn}(L_2)$ and $\varphi$ is the isomorphism $L_1' \to
  L_2'$. Then:

  $$\varphi(\omega^{\text{Group}}_{L_1}(x,y)) = \omega^{\text{Group}}_{L_2}(\zeta(x),\zeta(y))$$
\end{lemma}

\begin{proof}
  Apply Theorem
  \ref{thm:iterated-bracket-commutes-homoclinisms-pi-powered} to the
  word $\omega^{\text{Group}}$ and obtain the result.
\end{proof}

%% We now state a lemma that is similar to Lemma
%% \ref{lemma:commutativity-relation-same-group-lie-ring}.

%% \begin{lemma}
%%   Suppose $c$ is a natural number and $\pi_c$ is the set of primes
%%   less than or equal to $c$. Suppose $L$ is a $\pi_c$-powered Lie ring
%%   such that $\operatorname{Inn}(L)$ is a $3$-local class $c$ Lazard
%%   Lie ring. Consider the set maps $\omega,
%%   \omega^{\text{Group}}:\operatorname{Inn}(L) \times
%%   \operatorname{Inn}(L) \to [L,L}$. For $x,y \in
%%   \operatorname{Inn}(L)$, $\omega(x,y) = 0$ if and only if $\omega^{\tet{
\subsection{Lie bracket-like map for group whose class is one more}\label{sec:lie-bracket-like-map}

A special case of the map described here appeared in Theorem 5.1 of
the 2008 paper \cite{GG08} by George Glauberman. Glauberman's paper
proves certain properties of the map that we do not prove in this
section, but that follow from stronger results we will prove later in
the course of establishing the global Lazard correspondence up to
isoclinism. We discuss the relationship in more detail in Section
\ref{sec:rel-gg08}.

Suppose $c$ is a natural number and $\pi_c$ is the set of primes less
than or equal to $c$. 

Suppose $G$ is a $\pi_c$-powered nilpotent group such that the inner automorphism
group $G/Z(G)$ is a $3$-local class $c$ Lazard Lie group. We can
define a set map:

$$\omega_G^{\text{Lie}}: G/Z(G) \times G/Z(G) \to [G,G]$$

The set map is defined as follows:

$$(x,y) \mapsto h_{2,c+1}(\tilde{x},\tilde{y})$$

where $h_{2,c+1}$ is the formula for the Lie bracket in terms of the
group commutator and its iterations. It is the second of the two
inverse Baker-Campbell-Hausdorff formulas described in Section
\ref{sec:inverse-bch}. Note that this is defined because:

\begin{itemize}
\item The expression $h_{2,c+1}$ makes sense because $G$ is $\pi_c$-powered,
  and Lemma \ref{lemma:lie-bracket-denominators} shows that all prime
  divisors of the denominators of coefficients for $h_{2,c+1}$ are in $\pi_c$.
\item The output of the expression is independent of the choice of
  lifts, because the commutator map descends to a map $G/Z(G) \times
  G/Z(G) \to [G,G]$, and $h_{2,c+1}$ is obtained by using the
  commutator map and its iterations. See Theorem
  \ref{thm:iterated-commutator-descends-to-inn-pi-powered} for an
  explanation.
\end{itemize}

Further, in the case that $G$ itself is a $3$-local class $c$ Lazard
Lie group, $\omega_G^{\text{Lie}}(x,y)$ agrees with the definition of
the Lie bracket-induced map $\omega_L: \operatorname{Inn}(L) \times
\operatorname{Inn}(L) \to L'$ where $L$ is the Lazard Lie ring for
$G$. Recall that $\omega_L$ was defined in Section
\ref{sec:isoclinism-definition-lie} where we introduced the definition
of isoclinism and homoclinism for Lie rings. 

This again follows from the definition of $h_{2,c+1}$. 

We now show that $\omega_G^{\text{Lie}}$ is an isoclinism-invariant.

\begin{lemma}\label{lemma:omega-lie-isoclinism-invariant}
  Suppose $c$ is a natural number and $\pi_c$ is the set of primes less
  than or equal to $c$. Suppose $G_1$ and $G_2$ are isoclinic
  $\pi_c$-powered groups such that $\operatorname{Inn}(G_1) \cong
  \operatorname{Inn}(G_2)$ is a $3$-local class $c$ Lazard Lie group.

  Suppose $(\zeta,\varphi)$ is an isoclinism from $G_1$ to $G_2$:
  $\zeta$ is the isomorphism $\operatorname{Inn}(G_1) \to
  \operatorname{Inn}(G_2)$ and $\varphi$ is the isomorphism $G_1' \to
  G_2'$. Then:

  $$\varphi(\omega^{\text{Lie}}_{G_1}(x,y)) = \omega^{\text{Lie}}_{G_2}(\zeta(x),\zeta(y))$$
\end{lemma}

\begin{proof}
  Apply Theorem
  \ref{thm:iterated-commutator-commutes-homoclinisms-pi-powered} to
  the word $\omega^{\text{Lie}}$ and obtain the result. Note that the
  $c$ used in that theorem refers to a bound on the global class of
  $G_1$ and $G_2$, and may be a different value from the value of $c$
  used here. The fact of significance is that $G_1$ and $G_2$ are
  nilpotent, so the theorem can be applied.
\end{proof}

%% {\em TONOTDO: It should be possible to remove the assumption that $G$
%%   itself is $\pi_c$-powered and simply impose the condition on $G'$
%%   (along with on $\operatorname{Inn}(G)$); this will require proving
%%   additional results in Sections \ref{sec:group-powering} and
%%   \ref{sec:lie-ring-powering}. This will allow us to consider examples
%%   where we take central products with arbitrary abelian groups.}

\subsection{Commutator-like map for Lie ring extensions}\label{sec:commutator-like-map-for-extensions}

In Section \ref{sec:commutator-like-map}, we described a
commutator-like map that can be defined for any $\pi_c$-powered Lie
ring whose inner derivation Lie ring is nilpotent and of $3$-local
nilpotency class at most $c$. We will now consider the corresponding
notion for Lie ring {\em extensions}. Consider a central extension of
Lie rings, with the following central extension short exact sequence:

$$0 \to A \to N \to L \to 0$$

The Lie bracket map in $N$ descends to a $\Z$-bilinear map:

$$\omega_{N,L}: L \times L \to [N,N]$$

Suppose now that $N$ is $\pi_c$-powered and that $L$ has $3$-local
nilpotency class at most $c$. Then, we can define a commutator-like
map:

$$\omega_{N,L}^{\text{Group}}: L \times L \to [N,N]$$

The map $\omega_{N,L}^{\text{Group}}$ can be defined as the map
$\chi_{w,N}$ described in Lemma
\ref{lemma:iterated-bracket-descends-extension-version-pi-powered}
where $w$ is the word $M_{c+1}$.

$\omega_{N,L}^{\text{Group}}$ can be obtained by composing the
quotient map $L \times L \to N/Z(N) \times N/Z(N)$ with the map
$\omega_N^{\text{Group}}$ described in Section
\ref{sec:commutator-like-map} (the Lie ring is now $N$ instead of
$L$).

\subsection{Lie bracket-like map for group extensions}\label{sec:lie-bracket-like-map-for-extensions}

In Section \ref{sec:lie-bracket-like-map}, we described a Lie
bracket-like map that can be defined for any $\pi_c$-powered nilpotent
group whose inner automorphism group is of $3$-local nilpotency class
at most $c$. We will now consider the corresponding notion for Lie
ring extensions. Consider a central extension of groups, with the
following central extension short exact sequence:

$$0 \to A \to E \to G \to 1$$

The commutator map in $E$ descends to a set map:

$$\omega_{E,G}: G \times G \to [E,E]$$

Suppose now that $E$ is $\pi_c$-powered and that $G$ has $3$-local
nilpotency class at most $c$. Then, we can define a Lie bracket-like
map:

$$\omega_{E,G}^{\text{Lie}}: G \times G \to [E,E]$$

The map $\omega_{E,G}^{\text{Lie}}$ can be defined as the map
$\chi_{w,N}$ described in Lemma
\ref{lemma:iterated-commutator-descends-extension-version-pi-powered}
where $w$ is the word $M_{c+1}$.

This map $\omega_{E,G}^{\text{Lie}}$ can be obtained by composing the
quotient map $G \times G \to E/Z(E) \times E/Z(E)$ with the map
$\omega_E^{\text{Lie}}$ described in Section
\ref{sec:lie-bracket-like-map} (the group is now $E$ instead of $G$).

\subsection{The commutator in terms of the Lie bracket: the central extension and exterior square version}\label{sec:group-commutator-ito-lie-bracket-exteriorsquare-version}

Suppose $c$ is a positive integer and $L$ is a $\pi_c$-powered
nilpotent Lie ring of $3$-local nilpotency class at most $c$. Our goal
is to define a set map:

$$\tilde{M}_{c+1}(x,y): L \times L \to L \wedge L$$

where $L \wedge L$ is the exterior square of $L$ as defined in Section
\ref{sec:exteriorsquare-lie} (and later explicitly in Section
\ref{sec:exteriorsquare-explicit-lie}).

The definition of $\tilde{M}_{c+1}$ is as follows: it is the map $X_w$
of Theorem \ref{thm:iterated-bracket-map-to-exteriorsquare-pi-powered}
where $\pi = \pi_c$ and $w$ is the word $M_{c+1}$. The word $M_{c+1}$
satisfies the hypotheses of the theorem, so the theorem applies.

Using Theorem
\ref{thm:iterated-bracket-map-to-exteriorsquare-pi-powered} again, the
map $\omega_{N,L}^{\text{Group}}$ defined in Section
\ref{sec:commutator-like-map-for-extensions} is related to the map
$\tilde{M}_{c+1}$ defined in Section as follows:

$$\omega_{N,L}^{\text{Group}}(x,y) = \Omega_{N,L}(\tilde{M}_{c+1}(x,y))$$

where the maps are $\omega_{N,L}^{\text{Group}}: L \times L \to [N,N]$,
$\Omega_{N,L}: L \wedge L \to [N,N]$, and $\tilde{M}_{c+1}:L \times L
\to L \wedge L$.

\subsection{The Lie bracket in terms of the commutator: the central extension and exterior square version}\label{sec:lie-bracket-ito-group-commutator-exteriorsquare-version}

Suppose $c$ is a positive integer and $G$ is a $\pi_c$-powered nilpotent group
of $3$-local nilpotency class at most $c$. Our goal is to define a set
map:

$$\tilde{h}_{2,c+1}(x,y): G \times G \to G \wedge G$$

where $G \wedge G$ is the exterior square of $G$ as defined in Section
\ref{sec:exteriorsquare} (and later explicitly in Section
\ref{sec:exteriorsquare-explicit}). 

The definition of $\tilde{h}_{2,c+1}$ is as follows: it is the map
$X_w$ of Theorem
\ref{thm:iterated-commutator-map-to-exteriorsquare-pi-powered} where
$\pi = \pi_c$, $w$ is the word $h_{2,c+1}$, and the value $c$ of the
theorem is taken to be $c + 1$. The word $h_{2,c+1}$ satisfies the
hypotheses of the theorem, so the result applies.

Using Theorem
\ref{thm:iterated-commutator-map-to-exteriorsquare-pi-powered} again,
the map $\omega_{E,G}^{\text{Lie}}$ defined in Section
\ref{sec:lie-bracket-like-map-for-extensions} is related to the map
$\tilde{h}_{2,c+1}$ as follows:

$$\omega_{E,G}^{\text{Lie}}(x,y) = \Omega_{E,G}(\tilde{h}_{2,c+1}(x,y))$$

where the maps are $\omega_{E,G}^{\text{Lie}}: G \times G \to [E,E]$,
$\Omega_{E,G}: G \wedge G \to [E,E]$, and $\tilde{h}_{2,c+1}:G \times G
\to G \wedge G$.

%\newpage

\section{Lazard correspondence up to isoclinism}\label{sec:lcuti}

\subsection{Definition of Lazard correspondence up to isoclinism}\label{sec:lcuti-def}

The Lazard correspondence up to isoclinism combines the idea of
isoclinism with the idea of the Lazard correspondence.

\begin{definer}[$3$-local class $(c + 1)$ Lazard correspondence up to isoclinism]
  Suppose $c$ is a positive integer and $\pi_c$ is the set of all
  prime numbers less than or equal to $c$. Suppose $G$ is a
  $\pi_c$-powered nilpotent group whose inner automorphism group and
  derived subgroup are both $3$-local class $c$ Lazard Lie groups, and
  $L$ is a $\pi_c$-powered nilpotent Lie ring whose inner derivation
  Lie ring and derived subring are both $3$-local class $c$ Lazard Lie
  rings. A Lazard correspondence up to isoclinism from $L$ to $G$ is a
  pair of isomorphisms $(\zeta,\varphi)$ where $\zeta$ is an
  isomorphism from $\operatorname{Inn}(L)$ to
  $\log(\operatorname{Inn}(G))$ and $\varphi$ is an isomorphism from
  $L'$ to $\log(G')$ such that:

  $$\varphi(\omega_L(x,y)) = \omega^{\text{Lie}}_G(\zeta(x),\zeta(y))$$

  This is equivalent to the requirement that:

  $$\varphi(\omega^{\text{Group}}_L(x,y)) = \omega_G(\zeta(x),\zeta(y))$$
\end{definer}

The following are easy to verify for the $3$-local class $(c + 1)$ Lazard
correspondence up to isoclinism. All the groups mentioned below are
$\pi_c$-powered groups that have the property that their inner
automorphism group and derived subgroup are both $3$-local class $c$
Lazard Lie groups, and all the Lie rings mentioned below are
$\pi_c$-powered Lie rings that have the property that their inner
derivation Lie ring and derived subring are both $3$-local class $c$
Lazard Lie rings. %% {\em TONOTDO: Once the $\pi_c$-powering assumption is
  %% removed from the preceding subsection, remove it from here}.

\begin{itemize}
\item If $G_1$ and $G_2$ are isoclinic groups, and $G_1$ and $L$ are
  in $3$-local class $(c + 1)$ Lazard correspondence up to isoclinism,
  then $G_2$ and $L$ are also in $3$-local class $(c + 1)$ Lazard
  correspondence up to isoclinism. This follows from Lemma
  \ref{lemma:omega-lie-isoclinism-invariant}.
\item If $L_1$ and $L_2$ are isoclinic Lie rings, and $G$ and $L_1$
  are in $3$-local class $(c + 1)$ Lazard correspondence up to
  isoclinism, then $G$ and $L_2$ are in $3$-local class $(c + 1)$
  Lazard correspondence up to isoclinism. This follows from Lemma
  \ref{lemma:omega-group-isoclinism-invariant}.
\item If $G_1$ and $G_2$ are groups and $L$ is a Lie ring such that
  $G_1$ is in $3$-local class $(c + 1)$ Lazard correspondence up to
  isoclinism with $L$ and $G_2$ is also in $3$-local class $(c + 1)$ Lazard
  correspondence up to isoclinism with $L$, then $G_1$ and $G_2$ are
  isoclinic.
\item If $L_1$ and $L_2$ are Lie rings and $G$ is a group such that
  $G$ is in $3$-local class $(c + 1)$ Lazard correspondence up to isoclinism
  with $L_1$ and $G$ is in $3$-local class $(c + 1)$ Lazard correspondence
  up to isoclinism with $L_2$, then $L_1$ and $L_2$ are isoclinic Lie
  rings.
\end{itemize}

In other words, the definition we gave above establishes a
correspondence between {\em some} equivalences classes up to
isoclinism of groups and {\em some} equivalence classes up to
isoclinism of Lie rings. However, it is not yet clear that the
correspondence applies to {\em every} equivalence class up to
isoclinism of groups of the specified type and to {\em every}
equivalence class up to isoclinism of Lie rings of the specified
type. Ideally, we would like to demonstrate the following two facts:

\begin{enumerate}
\item For every $\pi_c$-powered group $G$ whose inner automorphism group
  and derived subgroup are both $3$-local class $c$ Lazard Lie groups,
  there exists a $\pi_c$-powered Lie ring $L$ whose inner derivation Lie
  ring and derived subring are both $3$-local class $c$ Lazard Lie
  rings such that $G$ is in $3$-local class $(c + 1)$ Lazard
  correspondence up to isoclinism with $L$.
\item For every $\pi_c$-powered Lie ring $L$ whose inner derivation Lie
  ring and derived subring are both $3$-local class $c$ Lazard Lie
  rings, there exists a $\pi_c$-powered group $G$ whose inner
  automorphism group and derived subgroup are both $3$-local class $c$
  Lazard Lie groups, such that $G$ is in $3$-local class $(c + 1)$
  Lazard correspondence up to isoclinism with $L$.
\end{enumerate}

Unfortunately, the full proofs of these results would require us to
develop further machinery and notation that would take too much
work. Therefore, we restrict attention to the global class case. The
proofs in the $3$-local case are analogous, but would require us to
deal with a $3$-local version of the free constructions that we used
earlier, which would considerably complicate the presentation. For
this reason, we restrict our proof to the global case.

\subsection{Global Lazard correspondence up to isoclinism}\label{sec:glcuti-def}

We begin with a couple of lemmas about the structures between which we
aim to establish the correspondence.

\begin{lemma}\label{lemma:global-class-one-more}
  Suppose $c$ is a positive integer and $\pi_c$ is the set of all
  primes less than or equal to $c$. Suppose $G$ is a $\pi_c$-powered
  nilpotent group of nilpotency class at most $c + 1$. Then, both
  $\operatorname{Inn}(G)$ and $G'$ are global class $c$ Lazard Lie
  groups, i.e., both $\operatorname{Inn}(G)$ and $G'$ are
  $\pi_c$-powered groups of nilpotency class at most $c$.
\end{lemma}

\begin{proof}
  We need to establish four facts:

  \begin{enumerate}
  \item $\operatorname{Inn}(G)$ has nilpotency class at most $c$: This
    is obvious from the condition on the nilpotency class of $G$.
  \item $\operatorname{Inn}(G)$ is $\pi_c$-powered: This follows from
    Lemma \ref{lemma:inn-aut-is-powering-invariant}.
  \item $G'$ has nilpotency class at most $c$: In fact, $G'$ has
    nilpotency class at most $\lfloor (c + 1)/2 \rfloor$, which in
    turn is at most $c$, but the naive bound of $c$ can be established
    in a straightforward manner by looking at the lower central series
    of $G$ and comparing with that of $G'$. More explicitly, this
    follows from the fact that the lower central series is a strongly
    central series. See the Appendix, Section
    \ref{appsec:strongly-central} for more details.
  \item $G'$ is $\pi_c$-powered: This follows from Theorem
    \ref{thm:powering-lcs}.
  \end{enumerate}
\end{proof}

\begin{lemma}
  Suppose $c$ is a positive integer and $\pi_c$ is the set of all primes
  less than or equal to $c$. Suppose $L$ is a $\pi_c$-powered Lie ring of
  nilpotency class at most $c + 1$. Then, both $\operatorname{Inn}(L)$
  and $L'$ are global class $c$ Lazard Lie rings, i.e., both
  $\operatorname{Inn}(L)$ and $L'$ are $\pi_c$-powered groups of
  nilpotency class at most $c$.
\end{lemma}

\begin{proof}
  We need to establish four facts:

  \begin{enumerate}
  \item $\operatorname{Inn}(L)$ has nilpotency class at most $c$: This
    is obvious from the condition on the nilpotency class of $L$.
  \item $\operatorname{Inn}(L)$ is $\pi_c$-powered: This follows from
    Theorem \ref{thm:ucsqpi-lie}.
  \item $L'$ has nilpotency class at most $c$: In fact, $L'$ has
    nilpotency class at most $\lfloor (c + 1)/2 \rfloor$, which in
    turn is at most $c$, but the naive bound of $c$ can be established
    very straightforwardly by looking at the lower central series of
    $L$ and comparing with that of $L'$. %% {\em TONOTDO: Link to appendix
      %% section on lower central series being strongly central for Lie
      %% rings}.
  \item $L'$ is $\pi_c$-powered: This follows from Lemma
    \ref{lemma:lie-ring-lcs-divisibility}.
  \end{enumerate}
\end{proof}

We can now define the correspondence:

\begin{definer}[Global class $(c + 1)$ Lazard correspondence up to isoclinism]
  Suppose $c$ is a positive integer and $\pi_c$ is the set of (all)
  prime numbers less than or equal to $c$. Suppose $G$ is a
  $\pi_c$-powered group of nilpotency class at most $c + 1$, and $L$ is
  a $\pi_c$-powered Lie ring of nilpotency class at most $c + 1$. A
  global class $(c + 1)$ Lazard correspondence up to isoclinism from $L$ to
  $G$ is a pair of isomorphisms $(\zeta,\varphi)$ where $\zeta$ is an
  isomorphism from $\operatorname{Inn}(L)$ to
  $\log(\operatorname{Inn}(G))$ (note that this $\log$ uses the global
  Lazard correspondence) and $\varphi$ is an isomorphism from $L'$ to
  $\log(G')$ such that:

  $$\varphi(\omega_L(x,y)) = \omega^{\text{Lie}}_G(\zeta(x),\zeta(y))$$

  This is equivalent to the requirement that:

  $$\varphi(\omega^{\text{Group}}_L(x,y)) = \omega_G(\zeta(x),\zeta(y))$$
\end{definer}

The observations from Section \ref{sec:lcuti-def} about the $3$-local
class $(c + 1)$ Lazard correspondence up to isoclinism continue to apply
here: the correspondence establishes a correspondence between {\em
  some} $\pi_c$-powered groups of class at most $c + 1$ and {\em some}
$\pi_c$-powered Lie rings of class at most $c + 1$. However, it is not yet clear that the
correspondence applies to {\em every} equivalence class up to
isoclinism of groups of the specified type and to {\em every}
equivalence class up to isoclinism of Lie rings of the specified
type. Essentially, we need to establish two facts:

\begin{enumerate}
\item For every $\pi_c$-powered group $G$ of nilpotency class at most $c
  + 1$, there exists a $\pi_c$-powered Lie ring $L$ of nilpotency class
  at most $c + 1$ such that $L$ and $G$ are in global class $(c + 1)$ Lazard
  correspondence up to isoclinism.
\item For every $\pi_c$-powered Lie ring $L$ of nilpotency class at most
  $c + 1$, there exists a $\pi_c$-powered group $G$ of nilpotency class
  at most $c + 1$ such that $L$ and $G$ are in global class $(c + 1)$ Lazard
  correspondence up to isoclinism.
\end{enumerate}

\subsection{The Baer correspondence up to isoclinism is the case $c = 1$}

If we set $c = 1$, then the global class $(c + 1)$ Lazard
correspondence up to isoclinism reduces to the Baer correspondence up
to isoclinism, as described in Section \ref{sec:bcuti}. The general
proof that we will now give follows steps very similar to the proof of
the statement for the Baer correspondence. The key difference is that
the group whose Schur multiplier and exterior square we are computing
is no longer an abelian group. Therefore, we have to explicitly use
the Lazard correspondence rather than the abelian Lie correspondence
to move back and forth between the group and the Lie ring.


\subsection{Global Lazard correspondence preserves Schur multipliers}\label{sec:global-lazard-correspondence-preserves-schur-multipliers}

For both groups and Lie rings, the Schur multiplier is the object that
controls the equivalence class of the extension up to isoclinism. The
first step in establishing the global Lazard correspondence up to
isoclinism is therefore to establish that Schur multipliers are
preserved under the global Lazard correspondence.

A particular case of this statement appeared as a conjecture in the
paper \cite{SchurmultiplierandLazard} by Eick, Horn, and Zandi in
September 2012, stated informally after Theorem 2 of the
paper. Specifically, the authors conjectured that for a finite
$p$-groups of nilpotency class at most $p - 1$, the Schur multiplier
of the group coincides with the Schur multiplier of its Lazard Lie
ring. The authors proved the corresponding statement for $p$-groups of
nilpotency class at most $p - 2$ by noting that the central extensions
of the group are in Lazard correspondence with the central extensions
of the Lie ring.\footnote{The authors write: ``Based on various
  example computations, see also [7], we believe that Theorems 1 and 2
  also hold for finite $p$-groups of class $p - 1$. However, our proofs
  do not extend to this case.''  The reference [7] alluded to by the
  authors has not yet been published or made available online.}
\begin{theorem}\label{thm:global-lazard-correspondence-preserves-schur-multipliers}
  Suppose $c$ is a positive integer and $\pi_c$ is the set of all primes
  less than or equal to $c$. Suppose $G$ is a $\pi_c$-powered group of
  nilpotency class at most $c$ and $L$ is its Lazard Lie ring under
  the global Lazard correspondence. Then:

  \begin{enumerate}
  \item The short exact sequences:

    $$0 \to M(L) \to L \wedge L \to [L,L] \to 0$$

    and

    $$0 \to M(G) \to G \wedge G \to [G,G] \to 1$$

    are canonically in Lazard correspondence.

  \item $L \wedge L$ and $G \wedge G$ are in Lazard correspondence up
    to canonical isomorphism. Moreover, if we denote the isomorphism
    as a set map $\exp: L \wedge L \to G \wedge G$, we have $\exp(x
    \wedge y) = \tilde{h}_{2,c+1}(x,y)$ where $\tilde{h}_{2,c+1}$ is
    the adaptation of $h_{2,c+1}$ described in Section
    \ref{sec:lie-bracket-ito-group-commutator-exteriorsquare-version}. Similarly,
    under the inverse set map $\log: G \wedge G \to L \wedge L$, we
    have $\log(x \wedge y) = \tilde{M}_{c+1}(x,y)$, where
    $\tilde{M}_{c+1}$ is the adaptation of $M_{c+1}$ described in
    Section
    \ref{sec:group-commutator-ito-lie-bracket-exteriorsquare-version}.
  \item $M(L)$ and $M(G)$ are canonically isomorphic as abelian groups.
  \end{enumerate}
\end{theorem}

Note that these theorems are framed in terms of the Lazard
correspondence {\em up to isomorphism} rather than the strict Lazard
correspondence (in the sense of equality of sets). It does not make
sense to do the latter here because a strict Lazard correspondence
would require us to keep track of the strict definition of the
sets. However, although our results are {\em up to isomorphism}, they
are {\em up to canonical isomorphism}, which means that they commute
with the isomorphisms induced by transitioning from a group to an
isomorphic group.

\begin{proof}
  {\em Proof of (1)}: Denote by $K_1$ the free $\pi_c$-powered group
  of class $c$ on the underlying set of $G$. Denote by $R_1$ the
  kernel of the natural homomorphism $K_1 \to G$.

  Denote by $K$ the free $\pi_c$-powered group of class $c + 1$ on the
  underlying set $G$. Denote by $R$ the kernel of the natural
  homomorphism $K \to G$. Note that $K_1 = K/\gamma_c(K)$, and $R$ is
  the inverse image of $R_1$ under the projection map.

  Denote by $N_1$ the free $\pi_c$-powered Lie ring of class $c$ on the
  underlying set of $L$ (which is identified with the underlying set
  of $G$). Denote by $J_1$ the kernel of the natural homomorphism $N_1
  \to L$. Denote by $N$ the free $\pi_c$-powered Lie ring of class $c +
  1$ on the underlying set of $L$. Denote by $J$ the kernel of the natural
  homomorphism $N \to L$. Note that $N_1 = N/\gamma_c(N)$ and $J$ is
  the inverse image of $J_1$ under the projection map.

  From Theorem \ref{thm:lazard-correspondence-derived} (note that the
  notation of that theorem matches the notation here), the derived
  subgroup $[K,K]$ is in Lazard correspondence with the derived
  subring $[N,N]$. From Theorems
  \ref{thm:lazard-correspondence-intersect-kernel} and
  \ref{thm:lazard-correspondence-commutator-lie} (again, note that the
  notation of that theorem matches the notation here), we see that
  this Lazard correspondence restricts to a Lazard correspondence
  between $[K,R]$ and $[N,J]$ and also to a Lazard correspondence
  between $R \cap [K,K]$ and $J \cap [N,N]$. Applying these to
  quotient groups, we obtain that:

  \begin{itemize}
  \item $[K,K]/[K,R]$ is canonically in Lazard correspondence with the
    quotient Lie ring \\ $[N,N]/[N,J]$.
  \item $[K,K]/(R \cap [K,K])$ is canonically in Lazard correspondence
    with $[N,N]/(J \cap [N,N])$. 
  \item $(R \cap [K,K])/[K,R]$ is canonically in Lazard correspondence
    with $(J \cap [N,N])/[N,J]$.
  \end{itemize}

  Moreover, these short exact sequences are also in Lazard
  correspondence:

  $$0 \to (J \cap [N,N])/[N,J] \to [N,N]/[N,J] \to [N,N]/(J \cap [N,N]) \to 0$$

  $$0 \to (R \cap [K,K])/[K,R] \to [K,K]/[K,R] \to [K,K]/(K \cap [R,R]) \to 1$$

  By the discussion in Sections
  \ref{sec:hopf-formula-pi-powered-class-one-more} and
  \ref{sec:hopf-formula-pi-powered-class-one-more-lie}, these
  correspond respectively to the short exact sequences:

  $$0 \to M(L) \to L \wedge L \to [L,L] \to 0$$

  $$0 \to M(G) \to G \wedge G \to [G,G] \to 1$$

  {\em Proof of (2)}: The fact of isomorphism follows directly from
  (1). The assertions about $M_{c+1}$ and $h_{2,c+1}$ follow from the
  fact that the Lazard correspondence between $[N,N]$ and $[K,K]$
  arises as the restriction of the Lazard correspondence between $\Q
  N$ and $\sqrt{K}$, and under this correspondence, by definition,
  $e^{M_{c+1}(x,y)} = [e^x,e^y]$ (with the group commutator appearing
  on the right) and $[\log x,\log y] = \log(h_{2,c+1}(x,y))$ (with the
  Lie bracket appearing on the left) by the respective definitions of
  $M_{c+1}$ and $h_{2,c+1}$.

  {\em Proof of (3)}: This follows from (1), and the observation that
  the Lazard correspondence coincides with the abelian Lie
  correspondence where they overlap, so $M(L)$ and $M(G)$ are
  isomorphic as abelian groups.
\end{proof}

%% {\em TONOTDO: The proof above needs to be clarified, and lemmas need to
%%   be added earlier to make it more ``obvious'' to people who have
%%   reached this point. Also, I want to include a discussion of what
%%   this means in terms of the identities used to define the exterior
%%   square of a group or a Lie ring.}


\subsection{The global Lazard correspondence up to isoclinism for extensions}\label{sec:glcuti-extensions}

Suppose $c$ is a positive integer and $\pi_c$ is the set of all primes
less than or equal to $c$. Suppose $A$ is an abelian group and $G$ is
a global class $c$ Lazard Lie group.

Denote by $L$ the Lazard Lie ring of $G$.

We have the following short exact sequence (originally described in
Section \ref{sec:ses-uct}) for the central extensions with central
subgroup $A$ and quotient group $G$:

\begin{equation*}
  0 \to \operatorname{Ext}^1_{\mathbb{Z}}(G^{\operatorname{ab}};A) \to H^2(G;A) \to \operatorname{Hom}(M(G),A) \to 0
\end{equation*}

We also have the following short exact sequence (originally described
in Section \ref{sec:ses-uct-lie}) for the central extensions with
central subring $A$ and quotient Lie ring $L$:

\begin{equation*}
0 \to \operatorname{Ext}^1_{\mathbb{Z}}(L^{\operatorname{ab}};A) \to H^2_{\text{Lie}}(L;A) \to \operatorname{Hom}(M(L), A) \to 0
\end{equation*}

By Theorem
\ref{thm:global-lazard-correspondence-preserves-schur-multipliers},
$M(L)$ and $M(G)$ are canonically isomorphic. Also,
$L^{\operatorname{ab}}$ is canonically in abelian Lie correspondence
with $G^{\operatorname{ab}}$. Thus, the downward arrows below are
canonical isomorphisms:

$$\begin{array}{ccccccccc}
  0 &\to &\operatorname{Ext}^1_{\mathbb{Z}}(G^{\operatorname{ab}};A) &\to &H^2(G;A) &\to &\operatorname{Hom}(M(G),A) &\to &0\\
  & & \downarrow & & & & \downarrow & & \\
  0 &\to &\operatorname{Ext}^1_{\mathbb{Z}}(L^{\operatorname{ab}};A) & \to & H^2_{\text{Lie}}(L;A) & \to & \operatorname{Hom}(M(L), A) & \to & 0\\
\end{array}$$
  
The middle groups $H^2(G;A)$ and $H^2_{\text{Lie}}(L;A)$ are isomorphic
to each other, because both short exact sequences split. However, we
{\em do not} in general have a {\em canonical} isomorphism between the
middle groups.

\subsubsection{The isomorphism of the $\operatorname{Ext}^1$ groups on the left side, and its relation to the global class $c$ Lazard correspondence}

We have a canonical isomorphism of groups:

$$\operatorname{Ext}^1_{\mathbb{Z}}(G^{\operatorname{ab}};A) \cong \operatorname{Ext}^1_{\mathbb{Z}}(L^{\operatorname{ab}};A)$$

This is because the abelian group $G^{\operatorname{ab}}$ is in
abelian Lie correspondence with the abelian Lie ring
$L^{\operatorname{ab}}$, so the additive groups are the same, and
$\operatorname{Ext}^1$ computation uses only the underlying additive
group.

The elements of
$\operatorname{Ext}^1_{\mathbb{Z}}(G^{\operatorname{ab}};A)$
correspond to the extension groups with subgroup $A$ and quotient
group $G$ for which the sub-extension with quotient group $G'$ splits
(as $G' \times A$) and the induced extension with subgroup $A$ and
quotient group $G^{\operatorname{ab}}$ gives an abelian group. As a
result, all extensions corresponding to elements of
$\operatorname{Ext}^1_{\mathbb{Z}}(G^{\operatorname{ab}};A)$ have the
property that the extension group is a global class $c$ Lazard Lie
group. Similarly, the extensions corresponding to elements of
$\operatorname{Ext}^1_{\mathbb{Z}}(L^{\operatorname{ab}};A)$ have the
property that the extension group is a global class $c$ Lazard Lie
ring. We thus obtain a correspondence:

\begin{center}
Group extensions with subgroup $A$ and quotient group $G$ that are in
the image of
$\operatorname{Ext}^1_{\mathbb{Z}}(G^{\operatorname{ab}};A)$
$\leftrightarrow$ Lie ring extensions with subring $A$ and quotient
ring $L$ that are in the image of $\operatorname{Ext}^1_{\mathbb{Z}}(L^{\operatorname{ab}};A)$ 
\end{center}

For each group extension and Lie ring extension that are in bijection
(in other words, each pair of elements in the two isomorphic groups
that are in bijection with each other), the corresponding extension
group is in global class $c$ Lazard correspondence with the
corresponding extension Lie ring.

\subsubsection{The isomorphism of the $\operatorname{Hom}$ groups on the right side, and its relation to the Lazard correspondence up to isoclinism}\label{sec:glcuti-extensions-main}

We have a canonical isomorphism:

$$\operatorname{Hom}(M(G),A) \cong \operatorname{Hom}(M(L), A)$$

We reviewed the meanings of the two groups in Sections
\ref{sec:beta-map} and \ref{sec:ses-uct-right-map} (for groups) and
Sections \ref{sec:beta-map-lie} and \ref{sec:ses-uct-lie-right-map}
(for Lie rings). The group $\operatorname{Hom}(M(G),A)$
classifies the central extensions with central subgroup $A$ and
quotient group $G$ up to isoclinism of extensions. The group
$\operatorname{Hom}(M(L),A)$ classifies the central extensions
with central subring $A$ and quotient Lie ring $L$ up to isoclinism of
extensions.

The two $\operatorname{Hom}$ groups are isomorphic because, as
established above (Theorem
\ref{thm:global-lazard-correspondence-preserves-schur-multipliers}),
the Schur multipliers $M(L)$ and $M(G)$ are canonically isomorphic.

The isomorphism gives a correspondence:

\begin{center}
  Equivalence classes up to isoclinism of group extensions with central
  subgroup $A$ and quotient group $G$ $\leftrightarrow$
  Equivalence classes up to isoclinism of Lie ring extensions with
  central subring $A$ and quotient Lie ring $L$
\end{center}

Any particular instance of this bijection (i.e., an equivalence class
of Lie ring extensions and an equivalence class of group extensions
that are in bijection with each other) is termed a {\em global class
  $(c + 1)$ Lazard correspondence up to isoclinism for extensions}.

We now state an important lemma that relates the global class $(c + 1)$
Lazard correspondence up to isoclinism for extensions with the global
class $(c + 1)$ Lazard correspondence up to isoclinism.

%% \begin{lemma}\label{lemma:glcuti-extensions-like-maps}
%%   Supose $A$ is an abelian group, $G$ is a global class $c$ Lazard Lie
%%   group, and $L = \log G$ is the corresponding global class $c$ Lazard
%%   Lie ring. Suppose $E$ is a group extension with central subgroup $A$
%%   and quotient group $G$. Suppose $N$ is a Lie ring extension with
%%   central subring $\log A$ (which we denote as $A$ via abuse of
%%   notation) and quotient Lie ring $L$. Suppose further that the
%%   equivalence class up to isoclinism of the group extension $E$
%%   corresponds, via the above bijection, to the equivalence class of
%%   the Lie ring extension $N$. Then, consider the commutator map:

%%   $$\omega_{E,G}:G \times G \to [E,E]$$

%%   and the Lie bracket map:

%%   $$\omega_{N,L}: L \times L \to [N,N]$$

%%   In Sections \ref{sec:lie-bracket-like-map-for-extensions} and
%%   \ref{sec:commutator-like-map-for-extensions}, we defined maps:

%%   $$\omega_{E,G}^{\text{Lie}}: G \times G \to [E,E]$$

%%   $$\omega_{N,L}^{\text{Group}}: L \times L \to [N,N]$$
%% \end{lemma}

\begin{theorem}\label{thm:glcuti-extensions-implies-glcuti}
  Suppose $A$ is an abelian group, $G$ is a global class $c$ Lazard Lie
  group, and $L = \log G$ is the corresponding global class $c$ Lazard
  Lie ring. Suppose $E$ is a group extension with central subgroup $A$
  and quotient group $G$. Suppose $N$ is a Lie ring extension with
  central subring $\log A$ (which we denote as $A$ via abuse of
  notation) and quotient Lie ring $L$. Suppose further that the
  equivalence class up to isoclinism of the group extension $E$
  corresponds, via the above bijection, to the equivalence class of
  the Lie ring extension $N$. Then, the following are true.

  \begin{enumerate}
  \item The group $[E,E]$ is in global class $c$ Lazard correspondence
    with the Lie ring $[N,N]$.
  \item The commutator-induced group homomorphism $\Omega_{E,G}: G
    \wedge G \to [E,E]$ is in global class $c$ Lazard correspondence
    with the commutator-induced group homomorphism $\Omega_{N,L}: L
    \wedge L \to [N,N]$, where we use the canonical Lazard
    correspondence between $G \wedge G$ and $L \wedge L$ described in
    Theorem
    \ref{thm:global-lazard-correspondence-preserves-schur-multipliers},
    and the Lazard correspondence between $[E,E]$ and $[N,N]$
    described in part (1).
  \item Explicitly, if $\varphi:[N,N] \to \log([E,E])$ describes the
    isomorphism of Step (1), then for all $x,y \in G$ (so that $x,y
    \in L$ as well because $L$ and $G$ have the same underlying set):

    $$\varphi(\omega_{N,L}(x,y)) = \omega^{\text{Lie}}_{E,G}(x,y)$$

    Equivalently:

    $$\varphi(\omega^{\text{Group}}_{N,L}(x,y)) = \omega_{E,G}(x,y)$$

  \item The group $E$ is in global class $(c + 1)$ Lazard
    correspondence up to isoclinism with the Lie ring $N$.
  \end{enumerate}
\end{theorem}

\begin{proof}
  {\em Proof of (1) and (2)}: We have the following map induced by the
  commutator map in $E$:

  $$\omega_{E,G}: G \times G \to [E,E]$$

  Similarly, we have the following map induced by the Lie bracket map
  in $N$:

  $$\omega_{N,L}: L \times L \to [N,N]$$

  We can define $\omega_{E,G}^{\text{Lie}}$ in terms of $\omega_{E,G}$
  as explained in Section
  \ref{sec:lie-bracket-like-map-for-extensions}. Consider the
  canonical isomorphism between $L \wedge L$ and $G \wedge G$
  (described in Theorem
  \ref{thm:global-lazard-correspondence-preserves-schur-multipliers})
  and denote the set map by $\exp:L \wedge L \to G \wedge G$.

  In Section \ref{sec:beta-map}, we
  considered short exact sequences with surjective downward maps,
  where $\beta_G$ is the element of $\operatorname{Hom}(M(G),A)$ and
  $\beta'$ is the morphism obtained by restricting $\beta_G$ to its
  image, which we call $B$:

  $$\begin{array}{ccccccccc}
    0 & \to & M(G) & \to & G \wedge G & \to & [G,G] & \to & 1\\
    &&   \downarrow^{\beta_G'}  &&  \downarrow     && \downarrow^{\text{id}} && \\
    0 & \to & B &\to & [E,E] & \to & [G,G] & \to & 1\\
  \end{array}$$

  In Section \ref{sec:beta-map-lie}, we considered a similar short
  exact sequence with surjective downward maps, where $\beta_L$ is the
  element of $\operatorname{Hom}(M(L),A)$ and $\beta'_L$ is the
  morphism obtained by restricting $\beta_L$ to its image, which we
  call $B$:

  $$\begin{array}{ccccccccc}
    0 & \to & M(L) & \to & L \wedge L & \to & [L,L] & \to & 1\\
    \downarrow &&   \downarrow^{\beta'_L}  &&  \downarrow     && \downarrow&& \downarrow\\
    0 & \to & B &\to & [N,N] & \to & [L,L] & \to & 1\\
  \end{array}$$

  Under the canonical isomorphism of $M(L)$ and $M(G)$, $\beta_L$ and
  $\beta_G$ are canonically identified, so that the subgroups $B$ in
  both cases are the same, and $\beta_L'$ and $\beta_G'$ are
  canonically identified as well.

  The left and right downward maps for the group short exact sequence
  are in Lazard correspondence respectively with the left and right
  downward maps for the Lie ring short exact sequence. Therefore, the
  middle maps for these sequences, which are determined uniquely up to
  isomorphism, are also in Lazard correspondence up to
  isomorphism.\footnote{To see this, we could apply $\exp$ to the
    diagram for Lie rings and note that that diagram is equivalent up
    to isomorphism with the diagram for groups.} In particular, $[E,E]$
  and $[N,N]$ are in Lazard correspondence up to isomorphism and the
  maps $G \wedge G \to [E,E]$ and $L \wedge L \to [N,N]$ are in
  Lazard correspondence.

  {\em Proof of (3)}: The explicit description of the Lazard
  correspondence between $G \wedge G$ and $L \wedge L$ described in
  Theorem
  \ref{thm:global-lazard-correspondence-preserves-schur-multipliers}
  gives us that, for $x,y \in L$ (so that $x,y \in G$ because $L$ and
  $G$ have the same underlying set):

  $\exp(x \wedge y) = \tilde{h}_{2,c+1}(x,y)$

  where the $x \wedge y$ on the left is interpreted as an element of
  $L \wedge L$.

  Apply $\Omega_{E,G}$ to both sides and obtain:

  $$\Omega_{E,G}(\exp(x \wedge y)) = \Omega_{E,G}(\tilde{h}_{2,c+1}(x,y))$$

  The right side is $\omega_{E,G}^{\text{Lie}}(x,y)$, as described at
  the end of Section
  \ref{sec:lie-bracket-ito-group-commutator-exteriorsquare-version}. We thus obtain:

  $$\Omega_{E,G}(\exp(x \wedge y)) = \omega_{E,G}^{\text{Lie}}(x,y)$$

  The left side involves composing the set map $\exp: L \wedge L \to G
  \wedge G$ and the group homomorphism $\Omega_{E,G}: G \wedge G
  \to[E,E]$. By Part (2), this is equivalent to composing the map
  $\Omega_{N,L}:L \wedge L \to [N,N]$ with the Lazard correspondence
  up to isomorphism between $[N,N]$ and $[E,E]$. The statement of the
  theorem uses the symbol $\varphi$ to denote the isomorphism $[N,N]
  \to \log([E,E])$ describing the correspondence, so we obtain:

  $$\varphi(\Omega_{N,L}(x \wedge y)) = \omega_{E,G}^{\text{Lie}}(x,y)$$

  $\Omega_{N,L}(x \wedge y) = \omega_{N,L}(x,y)$, and we obtain:

  $$\varphi(\omega_{N,L}(x,y)) = \omega_{E,G}^{\text{Lie}}(x,y)$$

  as desired.

  The proof of the other identity is similar.

  {\em Proof of (4)}: The image of $Z(N)$ in $L$ is precisely the set
  $\{x \in L \mid \omega_{N,L}(x,y) = 0 \forall \ y \in L \}$. The
  image of $Z(E)$ in $G$ is precisely the set $\{ x \in G \mid
  \omega_{E,G}(x,y) = 1 \ \forall \ y \in G \}$. The condition
  $\omega_{E,G}(x,y) = 1$ implies that $\omega_{E,G}^{\text{Lie}}(x,y)
  = 0$ which in turn implies that $\omega_{N,L}(x,y) = 0$. Similarly,
  the condition that $\omega_{N,L}(x,y) = 0$ implies that
  $\omega_{N,L}^{\text{Group}}(x,y) = 1$ which in turn implies that
  $\omega_{E,G}(x,y) = 1$. The upshot is that the image of $Z(N)$ in
  $L$ coincides with the image of $Z(E)$ in $G$.

  Thus, the quotient of $L$ by the image of $Z(N)$ in $L$ is in strict
  Lazard correspondence with the quotient of $G$ by the image of
  $Z(E)$ in $G$. The former is canonically isomorphic to $N/Z(N)$ and
  the latter is canonically isomorphic to $E/Z(E)$. Thus, $N/Z(N)$ is
  in Lazard correspondence with $E/Z(E)$. Finally, the maps
  $\omega_{N,L}$ and $\omega_{E,G}$ descend in the same way to maps
  $\omega_N$ and $\omega_E$, and we obtain, for all $x,y \in
  \operatorname{Inn}(N)$:

  $$\varphi(\omega_N(x,y)) = \omega_E^{\text{Lie}}(x,y)$$

  We thus obtain that $N$ and $E$ are in global class $(c+1)$ Lazard
  correspondence up to isoclinism.
\end{proof}

\subsubsection{Relation between the middle groups}\label{sec:glcuti-extensions-splitting}

We have demonstrated the existence of canonical isomorphisms between
the left groups and between the right groups in the two short exact
sequences:

$$\begin{array}{ccccccccc}
  0 &\to &\operatorname{Ext}^1_{\mathbb{Z}}(G^{\operatorname{ab}};A) &\to &H^2(G;A) &\to &\operatorname{Hom}(G \wedge G,A) &\to &0\\
  & & \downarrow & & & & \downarrow & & \\
  0 &\to &\operatorname{Ext}^1_{\mathbb{Z}}(L^{\operatorname{ab}};A) & \to & H^2_{\text{Lie}}(L;A) & \to & \operatorname{Hom}(L \wedge L, A) & \to & 0\\
\end{array}$$

As described in Sections \ref{sec:ses-uct} and
\ref{sec:ses-uct-lie}, both short exact sequences
split. Therefore, it is possible to find an isomorphism $H^2(G;A) \to
H^2_{\text{Lie}}(L;A)$ that establishes an isomorphism of the short
exact sequences:

$$\begin{array}{ccccccccc}
  0 &\to &\operatorname{Ext}^1_{\mathbb{Z}}(G;A) &\to &H^2(G;A) &\to &\operatorname{Hom}(G \wedge G,A) &\to &0\\
  & & \downarrow & & \downarrow & & \downarrow & & \\
  0 &\to &\operatorname{Ext}^1_{\mathbb{Z}}(L;A) & \to & H^2_{\text{Lie}}(L;A) & \to & \operatorname{Hom}(L \wedge L, A) & \to & 0\\
\end{array}$$

Note, however, that the middle isomorphism is not canonical. In
general, here, {\em neither} sequence splits canonically. This is in
contrast with the Baer correspondence up to isoclinism, where the
short exact sequence for Lie rings splits, as per Section
\ref{sec:bcuti-extensions-splitting}. For the Baer correspondence up
to isoclinism, specifying an isomorphism of the middle groups is
equivalent to specifying a splitting of the short exact sequence
corresponding to group extensions. For the global Lazard
correspondence up to isoclinism, however, the analogous statement is
not true.

\subsection{The global Lazard correspondence up to isoclinism: filling the details}

Suppose $c$ is a positive integer and $\pi_c$ is the set of all primes
less than or equal to $c$.

We are now in a position to flesh out the remaining details of the
global Lazard correspondence up to isoclinism, which we defined in Section
\ref{sec:glcuti-def}:

\newpage

\begin{center}
  Equivalence classes up to isoclinism of $\pi_c$-powered groups of
  nilpotency class at most $c + 1$ $\leftrightarrow$ Equivalence
  classes up to isoclinism of $\pi_c$-powered Lie rings of nilpotency
  class at most $c + 1$
\end{center}

There are two pending facts we need to establish:

\begin{enumerate}
\item For every $\pi_c$-powered group $G$ of nilpotency class at most $c
  + 1$, there exists a $\pi_c$-powered Lie ring $L$ of nilpotency class at
  most $c + 1$ such that $G$ and $L$ are in global class $(c + 1)$ Lazard
  correspondence up to isoclinism.
\item For every $\pi_c$-powered Lie ring $L$ of nilpotency class at most
  $c + 1$, there exists a $\pi_c$-powered group $G$ of nilpotency class
  at most $c + 1$ such that $G$ and $L$ are in global class $(c + 1)$ Lazard
  correspondence up to isoclinism.
\end{enumerate}

\subsubsection{Explicit construction from the group to the Lie ring}

We are given a $\pi_c$-powered group $G$ of nilpotency class at most
$c + 1$, and we need to find a $\pi_c$-powered Lie ring $L$ of
nilpotency class at most $c + 1$ such that $L$ and $G$ are in global
class $c + 1$ Lazard correspondence up to isoclinism.

\begin{enumerate}[(a)]
\item Consider $G$ as a central extension:

  $$0 \to Z(G) \to G \to G/Z(G) \to 1$$

  Consider the equivalence class up to isoclinism of this extension.

\item Based on the discussion in Section
  \ref{sec:glcuti-extensions-main}, this equivalence class corresponds
  to an equivalence class up to isoclinism of Lie ring extensions with
  central subring $\log(Z(G))$ and quotient Lie ring
  $\log(G/Z(G))$. Let $L$ be any extension Lie ring in this
  equivalence class.

\item By Theorem \ref{thm:glcuti-extensions-implies-glcuti}, $L$ and $G$
  are in global Lazard correspondence up to isoclinism.
\end{enumerate}

\subsubsection{Explicit construction from the Lie ring to the group}

We are given a $\pi_c$-powered Lie ring $L$ of nilpotency class at
most $c + 1$, and we need to find a $\pi_c$-powered group $G$ of
nilpotency class at most $c + 1$ such that $L$ and $G$ are in global
class $c + 1$ Lazard correspondence up to isoclinism.

\begin{enumerate}[(a)]
\item Consider $L$ as a central extension:

  $$0 \to Z(L) \to L \to L/Z(L) \to 0$$

  Consider the equivalence class up to isoclinism of this extension.

\item Based on the discussion in Section
  \ref{sec:glcuti-extensions-main}, this equivalence class corresponds
  to an equivalence class up to isoclinism of group extensions with
  central subgroup $\exp(Z(L))$ and quotient group $\exp(L/Z(L))$. Let
  $G$ be any extension group in this equivalence class.

\item By Theorem \ref{thm:glcuti-extensions-implies-glcuti}, $L$ and
  $G$ are in global class $(c + 1)$ Lazard correspondence up to
  isoclinism.
\end{enumerate}

\subsubsection{Preservation of order}\label{sec:glcuti-preserves-order}

In both directions, the constructions preserve the orders. In other
words, if we start with a finite group and use the construction in the
direction from groups to Lie rings, the Lie ring that we obtain has
the same order as the group that we started with. Similarly, if we
start with a finite Lie ring and use the construction in the direction
from Lie groups to groups, the group that we obtain has the same order
as the Lie ring that we started with.

This does not imply that {\em every} group and every Lie ring that are
in global Lazard correspondence up to isoclinism must have the same
order. Rather, we are saying that the answer to the existence question
continues to be affirmative even after we impose the condition that
the orders have to be equal.

In particular, given a a finite $p$-group of nilpotency class $p$, we
can find a finite $p$-Lie ring (i.e., a Lie ring whose additive group
is a finite $p$-group) of nilpotency class $p$ such that the group and
Lie ring are in global class $p$ Lazard correspondence up to
isoclinism. Similarly, given a finite $p$-Lie ring of nilpotency class
$p$, we can find a finite $p$-group of nilpotency class $p$ such that
the group and Lie ring are in global class $p$ Lazard correspondence
up to isoclinism.

\subsection{Relating the global Lazard correspondence and the global Lazard correspondence up to isoclinism}

Suppose $c$ is a positive integer and $\pi_c$ is the set of all primes
less than or equal to $c$. In Section \ref{sec:glcuti-extensions}, we
considered the case where $A$ is an abelian group and $G$ is a global
class $c$ Lazard Lie group (i.e., a $\pi_c$-powered group of nilpotency
class at most $c$) with Lazard Lie ring $L$. Recall the short exact
sequences described in Section \ref{sec:glcuti-extensions}:

$$\begin{array}{ccccccccc}
  0 &\to &\operatorname{Ext}^1_{\mathbb{Z}}(G^{\operatorname{ab}};A) &\to &H^2(G;A) &\to &\operatorname{Hom}(M(G),A) &\to &0\\
  & & \downarrow & & & & \downarrow & & \\
  0 &\to &\operatorname{Ext}^1_{\mathbb{Z}}(L^{\operatorname{ab}};A) & \to & H^2_{\text{Lie}}(L;A) & \to & \operatorname{Hom}(M(L), A) & \to & 0\\
\end{array}$$

We had noted at the time that both short exact sequences split, and
therefore the middle groups are isomorphic. However, in general,
neither splitting is canonical (the splitting on the Lie ring side is
canonical for abelian $L$, as described in Section
\ref{sec:ses-uct-lie-abelian-canonical-splitting}, but this is an exceptional
situation). Moreover, in general, there is no canonical isomorphism
between the middle groups.

Suppose now that $A$ and $G$ (and therefore also $L$) are powered over
the set of all primes less than or equal to $c + 1$. Note that in the
case that $c + 1$ is composite, this is always true, but it may also
be true for specific choices of $A$ and $G$ even in the case that $c +
1$ is prime. In this case, all the elements of $H^2(G;A)$ correspond
to global class $(c + 1)$ Lazard Lie groups and all the elements of
$H^2_{\text{Lie}}(L;A)$ correspond to global class $(c + 1)$ Lazard
Lie rings. Moreover, the global class $(c + 1)$ Lazard correspondence
induces an isomorphism between these groups that defines a canonical
isomorphism of the short exact sequences:

$$\begin{array}{ccccccccc}
  0 &\to &\operatorname{Ext}^1_{\mathbb{Z}}(G^{\operatorname{ab}};A) &\to &H^2(G;A) &\to &\operatorname{Hom}(M(G),A) &\to &0\\
  & & \downarrow & & \downarrow & & \downarrow & & \\
  0 &\to &\operatorname{Ext}^1_{\mathbb{Z}}(L^{\operatorname{ab}};A) & \to & H^2_{\text{Lie}}(L;A) & \to & \operatorname{Hom}(M(L), A) & \to & 0\\
\end{array}$$

Note that in this situation, {\em both} short exact sequences split,
{\em neither} splits canonically (unless $c = 1$), yet there is a
canonical isomorphism between them.

Another way of framing this is that, wherever applicable, the global
class $(c + 1)$ Lazard correspondence {\em refines} the global class
$(c + 1)$ Lazard correspondence up to isoclinism.

%% \subsection{Cases where the correspondence applies only up to isoclinism}

%% In order to have a situation where the global class $(c + 1)$ Lazard
%% correspondence applies {\em only} up to isoclinism, we need $c + 1$ to
%% be prime, and we need that the groups and Lie rings in question are
%% {\em not} powered over the prime $c + 1$.

%% The prototypical example of this for finite groups is as follows,
%% where $p$ is a prime number:

%% Equivalence classes up to isoclinism of finite $p$-groups of nilpotency class exactly $p$ $\leftrightarrow$ Equivalence classes up to isoclinism of finite $p$-Lie rings of nilpotency class exactly $p$

%% Examples for the case $p = 2$ (thus, $c = 1$) were described in
%% connection with the Baer correspondence up to isoclinism (see Section
%% \ref{sec:bcuti-ex}). We briefly describe an example for the case $p =
%% 3$, using the extension setup described in Section
%% \ref{sec:glcuti-extensions} in order to get a more holistic
%% picture.

%% Consider the case where $G = UT(3,3)$ (this is the unique non-abelian
%% group of order $3^3$ and exponent $3$) and $A = \mathbb{Z}_3$ is the
%% cyclic group of order three. The corresponding Lie ring $L$ is
%% $NT(3,3)$.

%% {\em TONOTDO: Insert full details of example}
