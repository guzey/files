\documentclass[10pt]{amsart}

%Packages in use
\usepackage{fullpage, hyperref, vipul}

%Title details
\title{Lie rings: automorphism structure, possibly fusion system}
\author{Vipul Naik}

%List of new commands
\newcommand{\Skew}{\operatorname{Skew}}
\newcommand{\ad}{\operatorname{ad}}
\begin{document}
\maketitle

In this article, I provide a brief outline of some thoughts on
constructing an analogue of a ``fusion system'' on a Lie ring. The
idea is to capture, as far as possible, the local structure of
automorphisms.

\section{Derivation, exponential, automorphism}

\subsection{Known result}

Suppose $R$ is a non-associative (i.e., not necessarily associative)
ring and $d$ is a derivation on $R$ and $n$ is a natural number such
that:

\begin{itemize}
\item $d^n$ is the zero map.
\item $d^i(x)d^j(y) = 0$ for all $x,y \in R$ and $i + j \ge n$.
\item $R$ is uniquely divisible by all numbers $1,2,\dots,n-1$.
\end{itemize}

Then we can define:

$$\exp(d) := \sum_{k=0}^{n-1} \frac{d^k}{k!}$$

It turns out that, due to the above conditions, $\exp(d)$ is an
automorphism. There is a nice way of thinking about this in terms of
lattice points. 
The result appears as Proposition 2.5(b) of the paper {\em Limits of abelian
subgroups of finite p-groups} by Alperin and Glauberman.

\subsection{Local version}

\begin{theorem}\label{localexpder}
Suppose $R$ is a non-associative (i.e., not necessarily associative)
ring and $d$ is a derivation on $R$, $A$ is a subring of $R$, and $n$
is a natural number such that:

\begin{itemize}
\item $d^n(x) = 0$ for all $x \in A$.
\item $d^i(x)d^j(y) = 0$ for all $x,y \in A$ and $i + j \ge n$.
\item $A$ is uniquely divisible by all numbers $1,2,\dots,n-1$.
\end{itemize}

Then we can define:

$$\exp(d) := \sum_{k=0}^{n-1} \frac{d^k}{k!}$$

It turns out that $\exp(d)$ is an injective homomorphism from $A$ to
$R$. If the image of $\exp(d)$ is a subring $B$, then $-d$ satisfies
the same conditions on $B$ and $\exp(-d)$ restricted to $B$ gives the
inverse isomorphism from $B$ to $A$. 
\end{theorem}

The proof is exactly the same as the general known version.

\subsection{$p$-Lie rings and the category of injective homomorphisms arising from derivations}

Suppose $L$ is a $p$-Lie ring, i.e., a Lie ring for which there exists
a prime number $p$ such that every element of the Lie ring has order a
power of $p$. Note that $p$-Lie rings are not always nilpotent, and we
will in practice restrict our attention to nilpotent $p$-Lie rings.

For any $u \in L$, the adjoint map $\ad u$ defines a derivation of
$L$. Set $R = L$ and $n = p$ in the result of theorem \ref{localexpder}. We get:

\begin{theorem}\label{innercategorysetup}
  For any subring $A$ of $L$ such that $(\ad u)^p(A) = 0$ and $[(\ad
  u)^i(A),(\ad u)^j(A)] = 0$ for $i + j \ge p$, we obtain an injective
  ring homomorphism $\exp(\ad u): A \to L$. If $B$ is the image of
  this homomorphism, we obtain an isomorphism $\exp(\ad u)$ from $A$
  to $B$ with inverse isomorphism given by $\exp(\ad (-u))$.
\end{theorem}

Define a {\em category on a $p$-Lie ring} $L$ to be a category whose
objects are subrings of $L$ and whose morphisms are injective Lie ring
homomorphisms.

Define the {\em inner category} (is there a better word?) on $L$ to be
the category on $L$ where the morphisms are obtained by compositions
of the following basic morphisms: injective group homomorphisms of the
form $\exp(\ad u)$ from any subring $A$ of $L$ on which the map is
defined (as per Theorem \ref{innercategorysetup}) to any subring
containing its image.

Define the {\em inner automorphism group} of $L$ as the automorphism
group of $L$ in the inner category. Clearly, the inner automorphism
group is a subgroup of the automorphism group.

\subsection{Exponentiable elements}

Suppose $L$ is a $p$-Lie ring. We call an element $u \in L$ {\em
  exponentiable} over a subring $A$ of $L$ if $\ad u$ satisfies the
conditions of Theorem \ref{innercategorysetup} relative to $A$, i.e.,
$(\ad u)^p(A) = 0$ and $[(\ad u)^i(A),(\ad u)^j(A)] = 0$ for $i + j
\ge p$. We call $u$ {\em exponentiable} over $L$ if it is
exponentiable over $L$ as a subring of itself, i.e., $(\ad u)^p = 0$
and $[(\ad u)^i(L),(\ad u)^j(L)] = 0$ for all $i + j \ge p$.

Note that whether or not an element is exponentiable depends only on
its coset modulo the center. There are hints of isoclinism here that
might connect up with another problem.

The subgroup of the automorphism group of $L$ generated by elements of
the form ($\exp(\ad u)$ where $u$ is exponentiable over $L$) is
precisely the automorphism group of $L$ as an object in its own inner
category.

\subsection{Globally determined inner category and every element being exponentiable}

We say that a category on a $p$-Lie ring $L$ is {\em globally
determined} if every homomorphism in the category can be extended to
an automorphism of the whole Lie ring in the category.

We say that a $p$-Lie ring $L$ has {\em globally determined inner
category} if its inner category is globally determined.
\section{Particular cases of nilpotent $p$-Lie rings}

\subsection{Global and $3$-local class}

It turns out that, for all the results we state here, we can replace
the conditions on the nilpotency class of the Lie ring with conditions
on the nilpotency class of {\em all subrings generated by at most $3$
elements}, which I call the $3$-local nilpotency class. To simplify
exposition, however, I state the results for Lie rings with global
bounds on the nilpotency class rather than in terms of the $3$-local
nilpotency class.
\subsection{Case of class at most $p - 1$}

For class $p - 1$ or smaller, the inner category of the Lie ring
coincides with the inner fusion system of its Lazard Lie group.

\begin{theorem}\label{lazard-p-innercategory}
  Suppose $L$ is a nilpotent $p$-Lie ring of class at most $p -
  1$. Then, under the Lazard correspondence, there exists a
  corresponding Lazard Lie group $G$ to $L$. Suppose $\alpha: L \to G$
  is the bijection from $L$ to $G$. The following theorem follows
  quite directly.  The following are true:

  \begin{enumerate}
  \item {\em All elements are exponentiable}: For every $u \in L$, the
    adjoint map $\ad u$ satisfies the conditions $(\ad u)^p = 0$ and
    $[(\ad u)^i(L),(\ad u)^j(L)] = 0$ for $i + j \ge p$.
  \item For every $u \in L$, $\exp(\ad u)$ is the same as the
    automorphism of $L$ obtained by moving back along $\alpha$ the
    conjugation automorphism of $G$ given by $\alpha(u)$.
  \item Under the element-wise identification $\alpha:L \to G$, the
    inner category on $L$ coincides with the inner fusion system on
    $G$. In particular, the inner automorphism group of $L$ coincides
    with the inner automorphism group of $G$.
  \item Every homomorphism in the inner category of $L$ extends to an
    inner automorphism of $L$.
  \end{enumerate}
\end{theorem}

\begin{proof}
  (1) is immediate from the nilpotency class condition.

  (2) is a little computationally tedious, but it was proved in Lemma
  3.3 of Glauberman's ``Partial extensions of Lazard correspondence''
  paper and also in Theorem 4.5.1 of the (mentioned in the paper). I
  plan to include some details of the proof in here.

  (3) and (4) both follow immediately once we have (1) and (2).
\end{proof}


\subsection{Case of class $p$}

We now consider the case of class exactly $p$. To make the statement
easier, we state things for class at most $p$, but exactly $p$ is what
we're interested in because smaller classes are already well covered.

\begin{theorem}\label{class-p-innercategory}
  Suppose $L$ is a nilpotent $p$-Lie ring of class at most
  $p$. $L/Z(L)$ has class at most $p - 1$. Suppose it has Lazard Lie
  group $H$ with Lazard correspondence $\beta: L/Z(L) \to H$. Then,
  the following are true:

  \begin{enumerate}
  \item {\em All elements are exponentiable}: For every $u \in L$, the
    adjoint map $\ad u$ satisfies the conditions $(\ad u)^p = 0$ and
    $[(\ad u)^i(L),(\ad u)^j(L)] = 0$ for $i + j \ge p$.
  \item For every $u \in L$, consider $\exp(\ad u)$ as an automorphism
    of $L$. The automorphism depends only on the coset $u + Z(L)$,
    which is an element of $L/Z(L)$. The automorphism descends to an
    automorphism of $L/Z(L)$. The descended automorphism corresponds,
    via $\beta$, to conjugation in $H$ by $\beta(u + Z(L))$.
  \item (Plausible, but needs to be proved!) Suppose $u,v$ are
    (possibly equal) elements of $L$. Then, there exists $w \in L$
    such that $\exp(\ad w)$ is the product $\exp(\ad u) \exp(\ad
    v)$. Moreover, $\beta(w + Z(L))$ is the product in $H$ of $\beta(u
    + Z(L))$ and $\beta(v + Z(L))$.
  \item (Follows from (3)) The automorphisms of the form $\exp(\ad
    u)$, $u \in L$, form the entire inner automorphism group of
    $L$. This is identified with $H$, the Lazard Lie group of $L/Z(L)$.
  \item Every homomorphism in the inner category of $L$ extends to an
    inner automorphism of $L$.
  \end{enumerate}
\end{theorem}

\begin{proof}
  (1) follows immediately from the nilpotency class condition (plus
      the Jacobi identity).

  The proof of (2) essentially mimics (2) from
  \ref{lazard-p-innercategory}.

  (3) still needs to be proved.

  (4) and (5) follow from (3).
\end{proof}

\subsection{Elements in small class ideals are exponentiable}

\begin{lemma}\label{small-class-ideal-exponentiable}
  Suppose $L$ is a Lie ring, $A$ is an ideal in $L$, and $u$ is an
  element of $A$. Suppose $A$ has class at most $p - 1$. Then, $\ad u$
  satisfies the nilpotency assumptions on $L$ for $n = p$, so
  $\exp(\ad u)$ defines an automorphism of $L$ by Theorem \ref{innercategorysetup}.
\end{lemma}

\begin{proof}
  We show both conditions:

  \begin{itemize}
  \item $(\ad u)^p(x) = 0$ for all $x \in L$: We note that $[u,x] \in
    A$ by the definition of ideal. Further, by the nilpotency class
    condition within $A$, $(\ad u)^{p-1}([u,x]) = 0$. Hence, $(\ad
    u)^p(x) = 0$.
  \item $[(\ad u)^i(x),(\ad u)^j(y)] = 0$ for all $x,y \in L$ (where
    $x,y$ may be equal or distinct) if $i + j \ge p$: We already
    considered the case where one of $i,j$ is zero in the previous
    case. So we will consider a situation where both are nonzero. In
    this case, we can rewrite $(\ad u)^i(x)$ as $(\ad u)^{i-1}([u,x])$
    and $(\ad u)^j(y)$ as $(\ad u)^{j-1}([u,y])$. Let $v = [u,x]$ and
    $w = [u,y]$. By the ideal condition, both $v,w$ are in $A$. Thus, we get:

    $$[(\ad u)^i(x),(\ad u)^j(y)] = [(\ad u)^{i-1}(v),(\ad u)^{j-1}(w)]$$

    The expression on the right side is an iterated Lie bracket of
    length $i - 1 + 1 + j - 1 + 1 = i + j \ge p$ and all its entries
    are within $A$. By the nilpotency class condition on $A$, it must
    be zero, completing the proof.
  \end{itemize}
\end{proof}

We now show that composing the automorphisms arising from such
elements is equivalent to performing the automorphism arising from
their group product in the Lazard Lie group of the ideal.

\begin{lemma}\label{small-class-ideal-exponentiable-group}
  Suppose $L$ is a Lie ring and $A$ is an ideal in $L$ of class at
  most $p - 1$. Suppose $u,v$ are (possibly equal, possibly distinct)
  elements of $A$. Suppose $w = u * v$ is the element of $A$ obtained
  by taking the product of the elements corresponding to $u$ and $v$
  in the Lazard Lie group of $A$, and then reinterpreting as an
  element of $L$. Then, $\exp(\ad w) = \exp(\ad u)\exp(\ad v)$.
\end{lemma}

The proof needs to be inserted.

\subsection{Engel conditions and exponentiable elements}

Consider both the conditions for an element $u$ of a Lie ring $L$ to
be exponentiable over a subring $A$ of $L$.

\begin{itemize}
\item $(\ad u)^p(x) = 0$ for all $x \in A$: This is a $p$-Engel
  condition. If $A = L$, it would mean that $u$ is a (global)
  $p$-Engel element.
\item $(\ad u)^i(x)(\ad u)^j(y) = 0$ for all $x,y \in L$ and $i + j
  \ge p$: This is an ``Engel-type condition'' (need to look up
  references to see better how this fits in).
\end{itemize}

It's true that $p$-Engel conditions imply bounds on the nilpotency
class under various circumstances. For very small class, this does
dash some hopes. 

For instance, any $2$-Engel $2$-Lie ring must have class at most
two. This follows from the fact that any $2$-Engel Lie ring that is
$3$-torsion-free must have class at most two. In other words, for $p =
2$, the only Lie rings in which every element is exponentiable are the
ones of class at most $p$, for which we were already aware of the
result from Theorem \ref{class-p-innercategory}.

\section{Case of Lie ring generated by ideals of class at most $p - 1$}

\subsection{Inspiration and setup}

This is a mirror image analogue of a situation studied by Glauberman:
the situation of a finite (hence nilpotent) $p$-group generated by
(finitely many) normal subgroups of class at most $p - 1$. We are
interested in the mirror image of Glauberman's situation: a Lie ring
$L$ generated by ideals of class at most $p - 1$. We will imitate the
procedure used by Glauberman.

First, some preliminary observations. It's easy to see that all groups
of class at most $p$ satisfy this condition (Hint: Go mod the third
member of the lower central series and reduce to showing that every
group of class at most two is generated by abelian normal subgroups,
then use that cyclic over central implies abelian). Precisely the same
reasoning works to show that any Lie ring of class at most $p$ is
generated by ideals of class at most $p - 1$, so this generalizes the
class $p$ case.

\subsection{Subgroup of automorphism group generated by exponentials of ideal family}

Suppose $L$ is generated by ideals $A_1, A_2, \dots, A_n$, all of
class at most $p - 1$. Note that we don't require $A_1, A_2, \dots, A_n$ to be {\em all} the ideals of class at most $p - 1$. Rather, we only require there to be enough of them to generate $L$. 

By lemmas \ref{small-class-ideal-exponentiable} and
\ref{small-class-ideal-exponentiable-group}, we obtain that for each
$i$, there is a homomorphism from the Lazard Lie group of $A_i$ to the
inner automorphism group of $L$. The kernel of this homomorphism is
the Lazard Lie group of $C_{A_i}(L)$, and the image is thus isomorphic
to the Lazard Lie group of $A_i/C_{A_i}(L)$.

The next natural question is: do the images of these homomorphisms
together generate the inner automorphism group of $L$? A little work
from Theorem \ref{class-p-innercategory} shows that if $L$ has class
at most $p$, this is indeed the case.

However, I suspect that it is not in general true if $L$ has class more than
$p$. The reason is that Glauberman has constructed counterexamples to
the mirror image situation (going from groups to Lie rings) and I
suspect the counterexample can be adapted to our situation.

\end{document}
    
