\documentclass[10pt]{amsart}

%Packages in use
\usepackage{fullpage, hyperref, vipul}

%Title details
\title{Division by primes: research plan}
\author{Vipul Naik (with John Wiltshire-Gordon)}

%List of new commands
\newcommand{\Skew}{\operatorname{Skew}}

\makeindex

\begin{document}
\maketitle
%\tableofcontents

This document gives a brief description of the key strands of some of
our ``division by prime'' research.

\section*{Broad goals and motivation}
Consider power series and similar formulas interpreted over fields and
rings of finite characteristic. Examples of power series:

\begin{eqnarray*}
  \exp(x) & = & \sum_{n=0}^\infty \frac{x^n}{n!}\\
  \log(1 + x) & = & \sum_{n=1}^\infty \frac{(-1)^{n-1}x^n}{n}\\
\end{eqnarray*}

A related example (not quite a power series, but similar in spirit) is
the Baker-Campbell-Hausdorff formula and its inverse formula, which
basically tells us how to pull back the multiplication in a Lie group
back to its corresponding Lie ring.

Although these formulas were originally designed/discovered in the
context of the topological/analytic structure of the real numbers and
complex numbers, many proofs of properties of these formulas are
purely algebraic, at least in cases where the formulas terminate in
finitely many steps due to suitable nilpotency conditions. We broadly
refer to this as the {\em nilpotent situation}.

Our goal is to interpret these formulas in cases where the underlying
ring has finite characteristic. The chief problem arises with
interpreting terms of the formula where we are dividing by primes that
are involved in the characteristic of the ring. Because of the Chinese
remainder theorem/Sylow's theorem and similar results, we can split
our rings up into $p$-Sylow subrings (so to speak) and can thus
restrict attention to rings of characteristic $p^n$. In other words,
we can focus attention on just one prime.

Thus, the chief problem is as follows: for a ring of characteristic
$p^n$, where, in particular, the ``multiplication by $p$'' map is
neither injective nor surjective, how do we make sense of terms in
power series and other series that involve division by $p$?

For instance, at the fifth term in the exponential formula, we must
``divide'' by $5$, but if the ring has characteristic $5^3$, then
there are two problems: (i) the multiplication by five map is not
injective, so some elements can be divided by $5$ in multiple ways,
and (ii) the multiplication by five map is not surjective, so some
elements cannot be divided by $5$ at all. [Note: there is nothing
special about $5$ -- it is used only for concreteness].

To our knowledge, the main approach used in the past has been to
simply restrict attention to those cases where the formula terminates
before division troubles arise. For instance, in the power series for
the exponential, the first time we divide by $p$ is in the term
$x^p/p!$. If the elements $x$ we are dealing with all satisfy the
property that $x^p = 0$ (or even better, $x^{p-1} = 0$) then we have
already reached $0$ by the time we encounter trouble. If we are
comfortable saying that $0$ divided by anything is $0$, then we can
truncate the exponential series and simply work with the truncated
series. More to the point, suitable variants of these nilpotency
conditions guarantee many identities that coincide with or are similar
to those in the characteristic zero nilpotent case.

This approach has yielded a number of powerful results. For instance,
there is a correspondence (known as the Lazard correspondence) between
Lie rings of order a power of $p$ with ($3$-local) nilpotency class
less than $p$ and finite $p$-groups with ($3$-local) nilpotency class
less than $p$. The correspondence makes use of the
Baker-Campbell-Hausdorff formula and notes that the terms in the
formula become zero before we need to start dividing by $p$. On a
related note, the exponential map provides a correspondence between
certain kinds of nilpotent derivations and certain kinds of unipotent
automorphisms of a Lie ring, as explored by Alperin and Glauberman.

Our key idea is to extend this correspondence a little further to some
cases where the series terminates finitely (so that we are still in a
nilpotent situation) but not before we need to divide by $p$. Thus, we
{\em do} face a problem of division by $p$. We will consider two
variants of this kind of problem:

\begin{itemize}
\item A variant where we do need to divide by $p$, and where the term
  that we need to divide by $p$ {\em is} divisible by $p$, but
  division by $p$ is not unique. For instance, we need to divide the
  element $10$ of $\Z/25\Z$ by $5$ (the answer could be $2$, $7$,
  $12$, $17$, or $22$). We believe that there is a general theory
  underlying this, and that it has a close relationship with the study
  of the $p$-adics.
\item A variant where we do need to divide by $p$, but cannot. In this
  case, we need to think of some alternative operation to division by
  $p$ that serves the same purpose. An example of this is when we have
  to divide an alternating bilinear form $\lambda$ by $2$. It turns
  out that, in some important cases, it suffices to find a $2$-cocycle
  $f$ such that $\Skew f = \lambda$. Thus, in this case, we replace
  doubling by the skew map. Something similar may also be true for
  higher primes, though we have not yet been able to uncover specific
  ideas in this direction.
\end{itemize}

\section*{Division locally and globally}

Suppose we have an abelian group $A$ with exponent $p^n$. Consider an
element $y \in pA$ (which is denoted $\mho^1(A)$ in $p$-group
terminology). The set of numbers $x$ satisfying $px = y$ is a coset of
$\Omega_1(A)$ -- the set of elements whose order divides $p$.

In most of the situations we will encounter, though, we don't just
have to divide a single element of $A$ by $p$. Rather, we have a
function (usually a group homomorphism) $f$ from some set $G$ that
takes values in $A$. The question is whether, given that the range of
$f$ lies in $pA$, we can get a new function $g$ to $A$ such that: (i)
$pg = f$ and (ii) $g$ continues to satisfy some nice properties that
we know $f$ to satisfy. 

The nice properties in (ii) should be of the kind that would be
automatically guaranteed in the case that we were working with
uniquely $p$-divisible groups. However, it is still a challenge to
figure out which of the properties that $f$ automatically satisfies
are crucial to carry over to $g$. The fewer properties we require $g$
to have, the more likely we are to find a $g$ that works.

Two broad classes of examples are stated below:

\begin{enumerate}
\item The Baer correspondence between class two $p$-groups and class
  two $p$-Lie rings relies on a formula that requires the halving of
  the commutator map. We can think of this as halving an alternating
  biadditive map $\lambda:G \times G \to A$. In the case that $p \ne
  2$, we can find a half that is also an alternating biadditive
  map. For $p = 2$, we can, under suitable conditions, find a ``half''
  that works.
\item The exponentiation map from some nilpotent derivations to some
  unipotent automorphisms requires dividing the additive endomorphism
  $d^p$ by $p$. The existence of a suitable group endomorphism $\ell$
  satisfying $p\ell = d^p$ as well as an additional condition allows
  us to carry forward some of the results.
\end{enumerate}

The most powerful results within this domain would be results of the
form that guarantee that if all the {\em values} that the function $f$
takes are in $pA$ (i.e., every particular value of the function can be
divided by $p$) then we can find a suitable function $g$ satisfying
$pg = f$ and satisfying some additional conditions. We thus need to
choose these additional conditions as weak enough for such a
local-global result to hold, but strong enough that we can draw
meaningful algebraic conclusions from them.

\section*{Repeated division and relationship with the $p$-adics}

The $p$-adic integers $\Z_{(p)}$ are obtained by taking the inverse
limit of the system of quotient maps for the groups/rings $\Z/p^n\Z$,
with $n \to \infty$. The $p$-adic integers have characteristic
zero. Multiplication by $p$ in this ring is injective but not
surjective. In particular, if an element is in $p\Z_{(p)}$, there is a
unique thing we get upon dividing it by $p$.

However, if we consider any finite stage, the number of solutions to
$px = y$ for $y$ a multiple of $p$ is exactly $p$. What's happening?
Each time we lift an element $y$ from $\Z/p^{n-1}\Z$ to $\Z/p^n\Z$,
all but one of the solutions to $px = y$ fail to extend. The one
solution that extends branches out into $p$ different solutions. This
means that if we consider solutions are keep lifting them, there is
only one infinite path. This infinite path is the answer to division
by $p$ in the $p$-adics.

Nonetheless, many important ideas related to $p$-adics have a close
bearing on what we want to do. Specifically, consider the exponential
function. For $p > 2$, this converges for any input in
$p\Z_{(p)}$. For $p = 2$, it converges for any input in
$4\Z_{(2)}$. The key idea is that under these conditions, the power of
$p$ in the numerator grows faster than that in the denominator.

This same observation is relevant in the finite case. If we start with
an element that is a multiple of $p$ (for $p > 2$) or a multiple of
$4$ (for $p = 2$) then the power of $p$ in the numerator grows much
faster than that in the denominator. This means that, roughly
speaking, if the element we start with is a multiple of $p$, and the
characteristic/exponent is $p^n$, then at some stage, the power of $p$
in the numerator exceeds that in the denominator by more than $n$, and
we can then safely declare that the power series {\em can be} terminated. 

The main question of choice arises for those terms intermediately
where the power of $p$ in the numerator is $\ge$ the power of $p$ in
the denominator, but the difference is less than $n$. Consider, for
instance, modulo $5^7$:

$$\exp(5) = 1 + 5 + \frac{5^2}{2!} + \frac{5^3}{3!} + \frac{5^4}{4!} + \frac{5^5}{5!} + \frac{5^6}{6!} + \frac{5^7}{7!} + \frac{5^8}{8!} + \frac{5^9}{9!} + \dots$$

At the $5^8/8!$ stage, the power of $5$ in the numerator exceeds that
in the denominator by $7$, so we could declare that term to be
$0$. This would allow us to concentrate on a finite chunk:

$$\exp(5) = 1 + 5 + \frac{5^2}{2!} + \frac{5^3}{3!} + \frac{5^4}{4!} + \frac{5^5}{5!} + \frac{5^6}{6!} + \frac{5^7}{7!}$$

Now, we need to make sense of the last three terms $5^5/5!$, $5^6/6!$,
and $5^7/7!$. Note that the real issue here is choosing a sound
candidate for $5^5/5$ and hoping that that choice gives consistent
results. One choice is $5^4$, but other possible choices include $5^4
+ k \cdot 5^7$ with $k = 1,2,3,4$. Choose $\ell$ to be any of
these. Note that the exponential now depends on both $5$ and $\ell$, and we get:

$$\exp(\{5, \ell \}) = 1 + 5 + \frac{5^2}{2!} + \frac{5^3}{3!} + \frac{5^4}{4!}  + \frac{\ell}{4!} + \frac{5^5}{4! \cdot 6} + \frac{5^6}{4! \cdot 6 \cdot 7}$$

(This baby situation is a somewhat special case where the choice of
$\ell$ does not affect the later terms. In general, it could affect
later terms).

We can now calculate the value of the exponential.

The situation becomes more complex if the term $d$ that we are
exponentiating is nilpotent but has very high nilpotency. Thus, we are
forced, not just to choose $d^p/p$, but also divide $d^{p^2}$ by
$(p^2)^{th}$, and so on. We believe that we have a general approach
and list of conditions for this more general requirement, and that
this can be generalized to power series in one variable other than the
exponential series.
\end{document}
