%\chapter{Powering over sets of primes}

%\newpage

\section{Groups powered over sets of primes: key results}\label{sec:group-powering}

Abelian groups can be defined as modules over $\Z$, the ring of
integers. We can therefore think of groups as the non-abelian
analogues of modules over $\Z$. In other words, we can think of group
theory as essentially happening ``over $\Z$'': we can raise group
elements only to integer powers.

Working over $\Z$ is insufficient for the Lie correspondence and its
generalizations. We saw in Section
\ref{sec:unitriangular-lie-correspondence} that the Lie correspondence
between $NT(n,\R)$ and $UT(n,\R)$ relies on the matrix exponential and
logarithm maps, which involve division. This division happens inside
the associative algebra of $n \times n$ matrices over $\R$, but it is
also related to the question of existence of rational powers of
elements in the group $UT(n,\R)$.

The purpose of this section is to develop the general theory of
$\pi$-powered groups: groups where it is possible to define $p^{th}$
roots of elements uniquely for $p$ in a specified set $\pi$ of
primes. The main purpose is to understand how ``closed'' this
collection is under various operations including taking subgroups and
quotient groups of important types. The theory will be useful in
establishing key aspects of the behavior of the Malcev and Lazard
correspondences, and their generalizations. When necessary, we will
restrict attention to nilpotent groups, where we can derive stronger
conclusions than for arbitrary groups.

\subsection{Some important intermediate rings between the integers and the rationals}

We denote the ring of integers as $\Z$ and the field of rational
numbers as $\Q$. Clearly, $\Z \subseteq \Q$. The intermediate subrings
between $\Z$ and $\Q$ can be described as follows. For any prime set
$\pi$, denote by $\Z[\pi^{-1}]$ the subring of $\Q$ comprising those
rational numbers such that, when the rational number is written as a
reduced fraction, all prime divisors of the denominator are in
$\pi$. Equivalently, it is the subring of $\Q$ generated by the
elements $1/p, p \in \pi$. The following are some special cases of
interest:

\begin{itemize}
\item The case that $\pi$ is the empty set: In this case,
  $\Z[\pi^{-1}] = \Z$.
\item The case that $\pi$ is the set of all primes: In this case,
  $\Z[\pi^{-1}] = \Q$.
\item The case that $\pi$ is a singleton set $\{ p \}$ for some prime
  number $p$: In this case, $\Z[\pi^{-1}] = \Z[1/p]$ is the subring
  generated by $1/p$.
\end{itemize}

In the language of commutative algebra, we would say that
$\Z[\pi^{-1}]$ is the localization of $\Z$ at the multiplicative
subset comprising all $\pi$-numbers. Here, a $\pi$-number is a number
all of whose prime divisors are in $\pi$.

\subsection{Background and motivation}

While building the Lazard correspondence and its generalizations, one
of the important operations we need to do is take $n^{th}$ roots of
elements (on the group side) or divide elements by $n$ (on the Lie
ring side). We need to be able to make sense of these operations.

There are two approaches to this:

\begin{enumerate}
\item The first approach is to impose conditions on the group and Lie
  ring of unique divisibility by specific primes. It suffices to
  restrict attention to primes because unique divisibility by specific
  primes gives unique divisibility by all products of powers of these,
  and conversely, unique divisibility by a number $n$ implies unique
  divisibility by all the prime divisors of $n$. In this approach, the
  operation of taking $p^{th}$ roots is not a separate operation but
  one uniquely determined by the group operations.

\item The second approach is to redefine the concept of group and/or
  Lie ring by including operations that correspond to taking $p^{th}$
  roots for specific primes $p$. On the Lie ring side, this means that
  instead of a Lie ring, we are talking now of a Lie {\em algebra}
  over the ring $\mathbb{Z}[1/p]$. If there is more than one prime, we
  adjoin all their reciprocals, so that we are considering a Lie
  algebra over the ring $\Z[\pi^{-1}]$ where $\pi$ is the set of
  primes for which we want to adjoint roots. If all primes are
  included, we simply get a Lie algebra over $\mathbb{Q}$. On the
  group side, we need to define an appropriate corresponding notion of
  group powered over a ring, then consider groups powered over
  $\mathbb{Z}[1/p]$ and similar rings. Note that in this approach,
  there is {\em additional structure} being imposed on the group
  and/or Lie ring. We would therefore revise the concept of
  ``subgroup'' and ``quotient group'' as being systems that are closed
  under the additional newly defined operations.
\end{enumerate}

Both approaches have their advantages and disadvantages. The advantage
of approach (1) is that since we are working with groups and Lie rings
as we usually understand them (over $\mathbb{Z}$) we do not need to
recheck any of the standard results, and conversely, any results we
discover here apply to abstract groups without additional
structures. The disadvantage is that we {\em do} need to verify that
the subgroups and quotient groups of interest inherit the unique
divisibility (powering) structure. This section focuses on a number of
simple lemmas designed for that goal.

Approach (2) is also reasonably straightforward in this case, but it
gets somewhat trickier when we want to deal with groups powered over
arbitrary rings. The axioms are easy to pin down for arbitrary rings
only in the case of nilpotent groups. For an exposition based on
approach (2), Thomas Weigel's monograph \cite{Weigel} and the
references therein are a good start. We will use some aspects of
approach (2) for some of the trickier results.

\subsection{Group powered over a prime}\label{sec:group-powering-def}

We begin with some definitions. Our definitions match those in
Khukhro's text \cite{Khukhro} and our treatment is quite similar to
that in Khukhro's text.

\begin{definer}[Powered for a set of primes]\label{def:poweredgroup}
  Suppose $G$ is a group and $\pi$ is a set of prime numbers. We say
  that $G$ is {\em powered over $\pi$}, or {\em $\pi$-powered}, if it satisfies
  the following equivalent conditions:

  \begin{itemize}
  \item For any $g \in G$ and any $p \in
    \pi$, there is a unique element $h \in G$ such that $h^p = g$.
  \item For any $g \in G$ and any natural number $n$ all of whose
    prime factors are in $\pi$, there is a unique element $h \in G$
    such that $h^n = g$.
  \end{itemize}
\end{definer}

For a single prime $p$, we shorten $\{ p \}$-powered to $p$-powered,
following the time-honored abuse of notation conflating elements with
singleton subsets. Note that \cite{Khukhro} uses the notation
$\mathbb{Q}_\pi$-powered for what we call $\pi$-powered.\footnote{One
  reason we avoid this notation is that $\mathbb{Q}_p$ is often used
  for the $p$-adics, which are quite different from what we wish to
  consider here.}

\begin{definer}[Local for a set of primes]\label{def:localgroup}
  Suppose $G$ is a group and $\pi$ is a set of prime numbers. We say
  that $G$ is $\pi$-local if $G$ is powered over all the primes {\em
    not} in $\pi$.
\end{definer}

For a single prime $p$, we shorten $\{ p \}$-local to $p$-local.

The term ``local'' here is used in analogy with localizations of rings
at prime ideals: when we localize at a prime ideal, we introduce
inverses {\em for all other primes}. Note that this is not directly
related to the sense of the word ``local'' in the context of {\em
  local analysis} used in the classification of finite simple groups.

For this section, we will frame all our results in terms of
$\pi$-powered groups rather than $\pi$-local groups. Obviously, each
result formulated in the language of $\pi$-powered groups can be
formulated instead in the language of $\pi$-local groups.

An extreme case is the case of a {\em rationally powered group}:

\begin{definer}[Rationally powered group]
  A group $G$ is termed a {\em rationally powered group} or a
  $\Q$-powered group if it is powered over the set of all primes.
\end{definer}

Our interest throughout this document will be on nilpotent and locally
nilpotent groups, but it is worth pointing out that there do exist
non-nilpotent rationally powered groups. The easiest example is the
group $GA^+(1,\R)$, which is defined as $\R \rtimes (\R^*)^+$, i.e.,
the group of affine maps from $\R$ to $\R$ of the form $x \mapsto ax +
b$ with $a > 0$, under composition. For any natural number $n$, every
element of the group has a unique $n^{th}$ root. Explicitly, the
unique $n^{th}$ root of $x \mapsto ax + b$ is the map:

$$x \mapsto a^{1/n}x + \frac{b}{1 + a^{1/n} + \dots + a^{(n-1)/n}}$$

This is an example of a rationally powered solvable group that is not
nilpotent.

It is also possible to construct {\em free} $\pi$-powered groups for
any set of prime numbers and any size of generating set. When the
generating set has size more than one, these are not solvable. The
free $\pi$-powered groups are extremely difficult to work with because
of the absence of an easily definable reduced form for words. We will
discuss free constructions in Section
\ref{sec:free-powered-groups-and-powering-functors}.

The majority of the results in this section can either be found in the
literature or are fairly easy to deduce, or both. The historical
origins of many individual results are hard to trace. For this reason,
we provide full proofs and avoid citations to papers for individual
results in this section. A number of the results have appeared in
\cite{Baumslagunique}, \cite{ConciseII}, \cite{Khukhro}, and other
references.

\subsection{The variety of powered groups and its forgetful functor to groups}\label{sec:variety-powered-forgetful-functor}

Suppose $\pi$ is a set of primes. The collection of $\pi$-powered
groups forms a variety of algebras. The operations in the variety
include the usual group operations (group multiplication, identity
element, inverse map) as well as operations of the form $x \mapsto
x^{1/p}$ for each prime $p \in \pi$, with the following two identities
for each $p \in \pi$:

\begin{itemize}
\item $(x^p)^{1/p} = x$: This condition shows that the $p^{th}$
  power map is {\em injective}, and that $u^{1/p}$ is the unique
  $p^{th}$ root of $u$ {\em if} $u$ is a $p^{th}$ power.
\item $(x^{1/p})^p = x$: This condition shows that the $p^{th}$ power
  map is {\em surjective}, i.e., that every element is a $p^{th}$
  power.
\end{itemize}

Suppose $\pi_1 \subseteq \pi_2$ are sets of primes. As discussed in
Section \ref{appsec:free-forgetful}, there is a forgetful functor from
the variety of $\pi_2$-powered groups to the variety of
$\pi_1$-powered groups. Each of these forgetful functors turns out to
be {\em full}. Essentially, this means that a set map $\varphi:G_1 \to
G_2$ between $\pi_2$-powered groups $G_1$ and $G_2$ is a homomorphism
of $\pi_2$-powered groups if and only if it is a homomorphism of
$\pi_1$-powered groups. The reason is that for all primes $p \in
\pi_2$, the $1/p$-powering map is preserved by the homomorphism simply
on account of the homomorphism being a group homomorphism and $p^{th}$
roots being unique in the target group $G_2$.


\subsection{The concepts of divisible and torsion-free}\label{sec:divisible-and-torsion-free}

We introduce some other useful definitions, again similar to those
found in \cite{Khukhro}.

\begin{definer}[Divisibility for a set of primes]
  Suppose $G$ is a group and $\pi$ is a set of primes. We say that $G$
  is $\pi$-{\em divisible} if it satisfies the following equivalent conditions:

  \begin{itemize}
  \item For any $g \in G$ and any $p \in \pi$, there exists $h \in G$
    (not necessarily unique) such that $h^p = g$.
  \item For any $g \in G$ and any natural number $n$ all of whose
    prime factors are in $\pi$, there exists $h \in G$ (not
    necessarily unique) such that $h^n = g$.
  \end{itemize}
\end{definer}

For a single prime $p$, we use the term $p$-divisible for $\{ p
\}$-divisible, with the usual abuse of notation conflating elements with
singleton subsets.

When we say that $G$ is {\em divisible} (without any set of primes
specified) this will be understood to mean that $G$ is divisible for
the set of all primes.  

We now define {\em torsion-free}.

\begin{definer}[Torsion-free for a set of primes]
  Suppose $G$ is a group and $\pi$ is a set of primes. We say that $G$
  is $\pi$-{\em torsion-free} if it has no element of order $p$ for
  any $p \in \pi$.
\end{definer}

When we say that $G$ is {\em torsion-free} (without any set of primes
specified) this will be understood to mean that $G$ is torsion-free
for the set of all primes.

In a short while, we will prove that for nilpotent groups, being
powered over a set of primes is equivalent to being both divisible and
torsion-free for that set of primes. However, this is not completely
obvious at the moment, and the corresponding result is false for
non-nilpotent groups.

Most of the results that follow rely on the following two crucial
observations, where $\gamma_i(G)$ denote the members of the lower
central series of $G$:

\begin{itemize}
\item Each successive quotient $\gamma_i(G)/\gamma_{i+1}(G)$
  is a homomorphic image of a tensor power of $G/G'$, the
  abelianization of $G$, via the $i$-fold iterated commutator map:

  $$G/G' \times G/G' \times \dots \times G/G' \to \gamma_i(G)/\gamma_{i+1}(G)$$

  See Lemma \ref{lemma:iterated-commutator-is-multilinear} for more
  details.
\item In the quotient $G/\gamma_{i+1}(G)$, the subgroup
  $\gamma_i(G)/\gamma_{i+1}(G)$ is central.
\end{itemize}

For simplicity, we will state and prove the results in subsequent
sections with respect to individual primes. However, the results
easily extend to sets of primes. More explicitly, for each of our
results, the corresponding result will hold if we uniformly replace
``$p$-powered'' by ``$\pi$-powered,'' ``$p$-divisible'' by
``$\pi$-divisible,'' and ``$p$-torsion-free'' by
$\pi$-torsion-free'' for an arbitrary set $\pi$ of primes.

\subsection{The case of finite groups}

If our interest is solely in finite groups, then the machinery
developed in this section is unnecessary. In particular, the finite
version of all results of interest follows from these two lemmas.

\begin{lemma}\label{lemma:finite-groups-pi-p-d-t}
  For a finite group $G$ and a prime number $p$, the following are equivalent:

  \begin{enumerate}
  \item $p$ does not divide the order of $G$.
  \item $p$ does not divide the exponent of $G$.
  \item $G$ is $p$-powered.
  \item $G$ is $p$-divisible.
  \item $G$ is $p$-torsion-free.
  \end{enumerate}
\end{lemma}

The proof is straightforward.

The next lemma builds on this.

\begin{lemma}\label{lemma:finite-groups-two-out-of-three}
  \begin{enumerate}
  \item Every subgroup, quotient group, and subquotient of a
    $p$-powered (respectively, $p$-divisible, $p$-torsion-free) finite
    group is $p$-powered (respectively, \\ $p$-divisible,
    $p$-torsion-free).
  \item If a finite group $G$ has a normal subgroup $H$ such that $H$
    and the quotient group $G/H$ are both $p$-powered (respectively,
    $p$-divisible, $p$-torsion-free), then $G$ is also $p$-powered
    (respectively $p$-divisible, $p$-torsion-free).
  \end{enumerate}
\end{lemma}

\begin{proof}
  (1) follows directly by using the characterization in terms of $p$
  not dividing the order, and using Lagrange's theorem to note that
  the order of any subgroup and quotient group divides the order of
  the group.

  (2) follows by combining the characterization in terms of $p$ not
  dividing the order, and using Lagrange's theorem to note that the
  order of a group is the product of the orders of any normal subgroup
  and the corresponding quotient group.
\end{proof}

In particular, a finite $p$-group is powered over {\em all} primes
{\em other than} $p$. Thus, it is in particular powered over all
primes less than $p$ (a very important observation) and also over all
primes greater than $p$ (a less important, but still useful,
observation). An equivalent formulation is that any finite $p$-group
is a $p$-local group.

Note that the result has some analogues for infinite groups in which
every element has finite order, but we do not need to develop these
for our purpose.

\subsection{Some general results on powering and divisibility}

We begin with some preliminary lemmas.

\begin{lemma}[Divisibility is inherited by quotient groups]\label{lemma:divisibility-inherit-quotient}
  Suppose $G$ is a group and $H$ is a normal subgroup. Suppose that
  $G$ is $p$-divisible for some prime number $p$. Then, the quotient
  group $G/H$ is also $p$-divisible. Explicitly, if $a \in G/H$, there
  exists $b \in G/H$ such that $b^p = a$.
\end{lemma}

\begin{proof}
  Let $\varphi:G \to G/H$ be the quotient map. Pick $g \in G$ such that
  $\varphi(g) = a$. There exists $h \in G$ such that $h^p = g$ due to the
  $p$-divisibility of $G$. The element $b = \varphi(h)$ satisfies $b^p = a$.
\end{proof}

We next show that powering on the quotient group implies powering on
the subgroup.

\begin{lemma}[Quotient-to-subgroup powering implication]\label{lemma:quottosub}
  Suppose $G$ is a group and $H$ is a normal subgroup of $G$. Suppose
  $p$ is a prime number such that both $G$ and $G/H$ are $p$-powered
  groups. Then, $H$ is also a $p$-powered group. In other words, for
  any $g \in H$, there exists a unique element $x \in H$ such that $x^p
  =g$.
\end{lemma}

\begin{proof}
  Let $\varphi:G \to G/H$ be the quotient map.

  Since $G$ as a whole is $p$-powered, there exists $x \in G$ such
  that $x^p = g$, and this $x$ is unique in $G$. It suffices to show
  that this unique $x$ is an element of $H$. For this, note that
  $(\varphi(x))^p = \varphi(x^p) = \varphi(g)$ is the identity element
  of $G/H$. Thus, $\varphi(x)$ is an element of order $1$ or $p$ in
  $G/H$. Since $G/H$ is $p$-powered, this forces $\varphi(x)$ to be
  the identity element of $G/H$, so $x \in H$, as desired.
\end{proof}

Theorem \ref{thm:normalsubgroupofnilpotentgroup} gives a converse
implication specific to the nilpotent context.


\subsection{Results for the center}

We begin with a lemma about the center.

\begin{lemma}\label{lemma:centerispoweringinvariant}
  Suppose $G$ is a group with center $Z(G)$. Suppose $n$ is a natural
  number. If $z \in Z(G)$ is such that there is a unique $x \in G$
  satisfying $x^n = z$, then  $x \in Z(G)$.

  In particular, if $G$ is powered over a prime $p$,
  so is the center $Z(G)$.
\end{lemma}

The key feature of the center that we use in the proof here is that it
is the fixed-point subgroup of a subgroup of $\operatorname{Aut}(G)$
(namely, $\operatorname{Inn}(G)$). The proof also works for
fixed-point subgroups of other subgroups of
$\operatorname{Aut}(G)$. In particular, the proof works for all
subgroups arising as centralizers of subgroups of $G$.

\begin{proof}
  It suffices to show that for any $y \in G$, $yxy^{-1} = y$.

  Note that:

  $$(yxy^{-1})^n = yx^ny^{-1} = yzy^{-1} = z$$

  Thus, $(yxy^{-1})^n = z = x^n$. Now, the uniqueness of $x$ (as a
  $n^{th}$ root of $z$) forces that in fact $yxy^{-1} = x$, completing
  the proof.
\end{proof}

Next:

\begin{lemma}\label{lemma:picentralimpliesqpi}
  Suppose $p$ is a prime number, $G$ is a $p$-powered group, and $H$
  is a central subgroup of $G$ that is also $p$-powered. Then, the
  quotient group $G/H$ is also $p$-powered. Explicitly, for any $a \in
  G/H$, there is a unique $b \in G/H$ satisfying $b^p = a$.
\end{lemma}

The proof below seems notationally complicated, but the idea is
simple. We first use the existence of $p^{th}$ roots in the whole
group to find a candidate $p^{th}$ root in $G/H$. We now want to show
uniqueness. Since the subgroup $H$ is in the center, we can take
$p^{th}$ roots of the subgroup elements used to translate within a
coset in order to figure out the appropriate translates on the
$p^{th}$ roots. Then, we use the uniqueness aspect to argue that all
$p^{th}$ roots of elements in a particular coset must lie in a single
coset.

\begin{proof}
  Let $\varphi:G \to G/H$ be the quotient map.

  Let $g \in G$ be such that $\varphi(g) = a$. There exists $x \in G$ such that $x^p = g$.

  For any $u \in H$, there exists $v \in H$ such that $v^p = u$ (due
  to our assumption that $H$ is $p$-powered). Thus, for any element of
  $G$ of the form $gu$ with $g$ as above and $u \in H$, we get $(xv)^p
  = x^pv^p = gu$ with $v$ as the element satisfying $v^p = u$. Note
  that rewriting $(xv)^p = x^pv^p$ uses the assumption that $H$ is
  in the center of $G$.

  Now, we claim that the element $b = \varphi(x)$ is the unique element
  satisfying $b^p = a$. First, note that $b^p = a$ follows by applying
  $\varphi$ to both sides of $x^p = g$. Suppose there is an element $c \in
  G/H$ satisfying $c^p = a$. Let $y \in G$ be such that $\varphi(y) =
  c$. Then, $y^p \in gH$, hence is of the form $gu, u \in H$, so by
  the preceding paragraph, it can also be written as $(xv)^p$, $v \in
  H$. Thus, we get $y^p = (xv)^p$ as elements of $G$. Since $G$ is
  $p$-powered, this forces $y = xv$, so $c = \varphi(y) = \varphi(xv) =
  \varphi(x)\varphi(v) = \varphi(x) = b$, thus proving the uniqueness of $b$ as
  the $p^{th}$ root of $a$.
\end{proof}

The preceding two lemmas easily give us the following.

\begin{lemma}\label{lemma:inn-aut-is-powering-invariant}
  Suppose $G$ is a group and $Z(G)$ is the center of $G$. Then, if $G$
  is $p$-powered for a prime $p$, both the center $Z(G)$ and the
  quotient group $G/Z(G)$ are $p$-powered. Hence, the inner
  automorphism group $\operatorname{Inn}(G)$, which is isomorphic to
  $G/Z(G)$, is also $p$-powered.
\end{lemma}

We are now in a position to state the main result.

\begin{theorem}\label{thm:ucsqpi}
  Suppose $G$ is a group (not necessarily nilpotent) and $Z^n(G)$ is
  the $n^{th}$ member of the upper central series of $G$. Suppose $p$
  is a prime such that $G$ is $p$-powered. Then, $Z^n(G)$ and
  $G/Z^n(G)$ are also $p$-powered.
\end{theorem}

\begin{proof}
  The fact that $G/Z^n(G)$ is $p$-powered follows by using
  mathematical induction on the preceding lemma, and noting that
  $G/Z^n(G)$ is obtained by iteration of the operation of factoring out by
  the center (starting from $G$). We can then use Lemma
  \ref{lemma:quottosub} to conclude that $Z^n(G)$ is also $p$-powered.
\end{proof}

Note that this result also extends to members of the transfinite upper
central series. For simplicity, however, we avoid dealing with
transfinite central series.

We can now state the bigger theorem. Note that this establishes a
partial converse to Lemma \ref{lemma:quottosub}.

\begin{theorem}\label{thm:normalinucspiequalsqpi}
  Suppose $G$ is a group and $H$ is a normal subgroup of $G$ that is
  contained in a member of the upper central series of $G$. Suppose
  $p$ is a prime number such that both $G$ and $H$ are
  $p$-powered. Then, $G/H$ is also $p$-powered.
\end{theorem}

Note that if $G$ is nilpotent, then the condition that $H$ is
contained in a member of the upper central series of $G$ is always
satisfied. We will return to this implication in Theorem
\ref{thm:normalsubgroupofnilpotentgroup}, which is deferred to a later
section.

\begin{proof}
  Let $Z^0(G),Z^1(G),Z^2(G),\dots$ be the upper central series of $G$
  (the trivial subgroup is $Z^0(G)$, the center is $Z^1(G)$, the
  second center is $Z^2(G)$, and so on). Suppose $H$ is contained in
  the member $Z^n(G)$ for some $n$. Intersecting with $H$, we get a
  series:

  $$1 = H \cap Z^0(G) \le H \cap Z^1(G) \le H \cap Z^2(G) \le \dots \le H \cap Z^n(G) = H$$

  All the subgroups in this series are normal (since each is an
  intersection of normal subgroups) and further, for $0
  \le i \le n - 1$, $(H \cap Z^{i+1}(G))/(H \cap Z^i(G))$ is a central
  subgroup of $G/(H \cap Z^i(G))$.

  By Theorem \ref{thm:ucsqpi}, each $Z^i(G)$ is $p$-powered. Hence, each
  $H \cap Z^i(G)$ is also $p$-powered.

  We will now prove by induction on $i$ that each $G/(H \cap Z^i(G))$
  is $p$-powered. The base case is clear. The inductive step is to
  show that if $G/(H \cap Z^i(G))$ is $p$-powered, so is $G/(H \cap
  Z^{i+1}(G))$. For this, note that by the third isomorphism theorem:

  \begin{equation*}
    G/(H \cap Z^{i+1}(G)) \cong \frac{G/(H \cap Z^i(G))}{(H \cap Z^{i+1}(G))/(H \cap Z^i(G))} \tag{*}
  \end{equation*}

  As noted above, $(H \cap Z^{i+1}(G))/(H \cap Z^i(G))$ is a central
  subgroup of $G/(H \cap Z^i(G))$. As also noted above, $H \cap
  Z^{i+1}(G)$ is $p$-powered, hence $p$-divisible. Combining this with
  Lemma \ref{lemma:divisibility-inherit-quotient}, we see that $(H \cap Z^{i+1}(G))/(H
  \cap Z^i(G))$ is also $p$-divisible. Since $(H \cap Z^{i+1}(G))/(H
  \cap Z^i(G))$ is a subgroup of the $p$-powered group $G/(H \cap
  Z^i(G))$, $(H \cap Z^{i+1}(G))/(H \cap Z^i(G))$ must be
  $p$-powered. Thus, we have a $p$-powered group $G/(H \cap Z^i(G))$
  and a $p$-powered central subgroup $(H \cap Z^{i+1}(G))/(H \cap
  Z^i(G))$. By Lemma \ref{lemma:picentralimpliesqpi}, the quotient group is
  also $p$-powered, so by (*), $G/(H \cap Z^{i+1}(G))$ is $p$-powered.

  This completes the proof of the inductive step. Thus, $G/(H \cap
  Z^i(G))$ is $p$-powered for all $i$. In particular, setting $i = n$,
  we get that $G/(H \cap Z^n(G)) = G/H$ is $p$-powered, completing the
  proof.
\end{proof}

\subsection*{A quick note on the ``duality'' between divisible and torsion-free}

There is a heuristic duality between some of the results that we will
be exploring in the coming two sections. Unfortunately, it is
difficult to make this duality rigorous. The following quick glossary
will give an idea of how the duality generally works.

\begin{itemize}
\item divisible $\leftrightarrow$ torsion-free
\item subgroup $\leftrightarrow$ quotient group
\item injective $\leftrightarrow$ surjective
\item lower central series $\leftrightarrow$ upper central series
\item derived subgroup $\leftrightarrow$ inner automorphism group
\item abelianization $\leftrightarrow$ center
\end{itemize}

The way the duality works is that for any statement involving these
concepts, we can typically consider a ``dual'' statement that replaces
each concept by its dual, and that dual statement is usually
true. Unfortunately, this duality does {\em not} always work. We shall
see examples where the ``dual'' statements to true statements are
false. Nonetheless, it is a useful guide for interpreting some of our
easier results.


\subsection{Basic results on divisible and torsion-free}

We first begin with a basic divisibility result.

\begin{lemma}\label{lemma:divisibility-extension-group}
  Suppose $G$ is a group and $H$ is a central subgroup of $G$ such
  that both $H$ and $G/H$ are $p$-divisible groups. Then, $G$ is a
  $p$-divisible group as well. In other words, for any $g \in G$,
  there exists $x \in G$ such that $x^p = g$.
\end{lemma}

\begin{proof}
  Suppose $\varphi:G \to G/H$ is the quotient map. Let $a =
  \varphi(g)$, so $a \in G/H$. There exists $b \in G/H$ such that $b^p
  = a$. Suppose $y \in G$ is such that $\varphi(y) = b$. Then,
  $y^{-p}g \in H$. Say, it is an element $u \in H$. Let $v$ be an
  element of $H$ such that $v^p = u$. Then, $y^{-p}g = v^p$, so $g =
  y^pv^p = (yv)^p$ (because $H$ is central). So, $G$ is $p$-divisible.
\end{proof}

We now turn to a result that is related to the idea of being torsion-free.

\begin{lemma}\label{lemma:powering-injective-extension-group}
  Suppose $G$ is a group and $H$ is a central subgroup of $G$. Suppose
  $p$ is a prime number such that both $H$ and the quotient group
  $G/H$ have the property that the map $x \mapsto x^p$ is injective in
  the group. Then, $G$ also has the property that the map $x \mapsto
  x^p$ is injective in $G$. Explicitly, if $a,b \in G$ are elements
  such that $a^p = b^p$, then $a = b$.
\end{lemma}

The idea is to first show that the elements are in the same coset of
$H$ (i.e., they have the same image in $G/H$) then take their quotient
and argue that that must be the identity element. The first part will
use the injectivity of the power map in $G/H$. The second part will
use the injectivity of the power map in $H$.

\begin{proof}
  Let $\varphi: G \to G/H$ be the quotient map.

  Since $a^p = b^p$, we have $\varphi(a)^p = \varphi(b)^p$. By the injectivity
  of the $p$-power map in $G/H$, we conclude that $\varphi(a) =
  \varphi(b)$. In other words, $a$ and $b$ are in the same coset of $H$,
  so the element $u = ab^{-1}$ is an element of $H$.

  Since $u = ab^{-1}$, $a = ub$. Further, $u \in H$ so $u$ is central,
  so $(ub)^p = u^pb^p$. Thus, $a^p = (ub)^p = u^pb^p$. Since $a^p =
  b^p$, we get that $u^pb^p = b^p$. Cancel $b^p$ from both sides to
  get that $u^p$ is the identity element of $G$, and hence also the
  identity element of the subgroup $H$.

  We now use the injectivity of the $p$-power map in $H$ to conclude
  that $u$ itself is the identity element of $H$, and hence, $a = b$.
\end{proof}

It easily follows that:

\begin{lemma}\label{lemma:powering-extension-group}
  Suppose $G$ is a group and $H$ is a central subgroup of $G$. Suppose
  $p$ is a prime number such that both $H$ and the quotient group
  $G/H$ are $p$-powered. Then, $G$ is $p$-powered.
\end{lemma}

\begin{proof}
  This follows from the preceding lemma and Lemma \ref{lemma:divisibility-extension-group}.
\end{proof}

\begin{lemma}\label{lemma:ucs-torsion-free}
  Suppose $G$ is a group (not necessarily nilpotent) and $p$ is a
  prime number. Suppose $i$ is a positive integer such that the
  quotient group $Z^i(G)/Z^{i-1}(G)$ is $p$-torsion-free. Then, the
  quotient group $Z^{i+1}(G)/Z^i(G)$ in $G$ is also $p$-torsion-free.
\end{lemma}

\begin{proof}
  Suppose $x$ is an element of $Z^{i+1}(G)$ whose image in
  $Z^{i+1}(G)/Z^i(G)$ has order $1$ or $p$. Our goal will be to show
  that $x \in Z^i(G)$, i.e., the order of $x$ modulo $Z^i(G)$ must be
  $1$ and cannot be $p$.

  For any $y \in G$, we have $[x,y] \in Z^i(G)$, and moreover, we
  have:

  $$[x,y]^p = [x^p,y] \pmod{Z^{i-1}(G)}$$

  Since $x^p \in Z^i(G)$, the right side is the identity element mod
  $Z^{i-1}(G)$, hence $[x,y]$ is an element of $Z^i(G)$ whose image in
  $Z^i(G)/Z^{i-1}(G)$ has $p^{th}$ power the identity. Thus, $[x,y]$
  taken modulo $Z^{i-1}(G)$ has order either $1$ or $p$. The order
  cannot be $p$ because by the inductive hypothesis,
  $Z^i(G)/Z^{i-1}(G)$ is $p$-torsion-free. Hence, $[x,y] \in Z^{i-1}(G)$.

  Since the above is true for {\em all} $y \in G$, we obtain that
  $[x,y] \in Z^{i-1}(G)$ for all $y \in G$. This forces $x \in
  Z^i(G)$, so that the order of the image of $x$ in
  $Z^{i+1}(G)/Z^i(G)$ is in fact $1$. Thus, the order can never be
  $p$, showing that $Z^{i+1}(G)/Z^i(G)$ is $p$-torsion-free.
\end{proof}

Note that the analogous result breaks down for the lower central
series. Specifically, the problem with the lower central series is
that the subgroups there are too small and the quotients too big, and
something being central modulo a quotient does not guarantee its
containment in the adjacent member.

\cite{Khukhro} gives a somewhat different proof (Lemma 3.16) that uses
the lower central series but ``augments'' it with the element of
interest, thus overcoming the problem of the lower central series being too small.

\subsection{Definition equivalence for torsion-free nilpotent groups with important corollaries}

\begin{theorem}\label{thm:torsion-free-equivalence-theorem}
  The following are equivalent for a nilpotent group $G$ and a prime
  number $p$.

  \begin{enumerate}
  \item The powering map $x \mapsto x^p$ is injective in $G$.
  \item $G$ is $p$-torsion-free.
  \item The center $Z(G)$ is $p$-torsion-free.
  \item Each of the successive quotients $Z^{i+1}(G)/Z^i(G)$ of the
    upper central series of $G$ is a $p$-torsion-free group.
  \item In any quotient of the form $Z^i(G)/Z^j(G)$, the powering map
    $x \mapsto x^p$ is injective.

  \end{enumerate}

  Moreover, the implication (1) to (2) to (3) to (4) to (5) holds in
  all groups. The only implication that relies on $G$ being nilpotent
  is the implication from (5) to (1).
\end{theorem}

\begin{proof}
  (1) implies (2): This is direct from the definition.

  (2) implies (3): This is immediate, since $Z(G)$ is a subgroup of $G$.

  (3) implies (4): This follows from Lemma \ref{lemma:ucs-torsion-free} and the
  principle of mathematical induction.

  (4) implies (5): This relies on Lemma
  \ref{lemma:powering-injective-extension-group}, the principle of
  mathematical induction, and the observation that in the base case
  (for abelian groups) being $p$-torsion-free is obviously equivalent
  to the $p$-power map being injective.

  (5) implies (1): If $G$ has class $c$, set $i = c$, $j =
  0$. Note that this is the only step where we use that $G$ is
  nilpotent.
\end{proof}

An easy corollary is as follows:

\begin{lemma}\label{lemma:nilpotent-powered-iff-d-t-f}
  If $G$ is a nilpotent group and $p$ is a prime number, then the following are equivalent:

  \begin{enumerate}
  \item $G$ is $p$-divisible and $p$-torsion-free.
  \item $G$ is $p$-powered.
  \end{enumerate}
\end{lemma}

\begin{proof}
  This is immediate, once we use the preceding theorem (Theorem
  \ref{thm:torsion-free-equivalence-theorem}) to replace $p$-torsion-free
  by ``the $p$-power map is injective.''
\end{proof}

\begin{lemma}\label{lemma:powered-subgroup-implies-torsion-free-quotient}
  If $G$ is a group and $H$ is a normal subgroup such that both $G$
  and $H$ are $p$-powered, then $G/H$ is $p$-torsion-free.
\end{lemma}

\begin{proof}
  Let $\varphi:G \to G/H$ be the quotient map and let $a \in G/H$ be such
  that $a^p$ is the identity element of $G/H$. Suppose $g \in G$ is
  such that $\varphi(g) = a$. Then, $(\varphi(g))^p = \varphi(g^p)$ is the
  identity element of $G/H$, so $g^p \in H$. Let $h = g^p$. Since $H$
  is $p$-powered, there exists an element $x \in H$ such that $x^p =
  h$. Thus, $x^p = g^p$. since $G$ is also $p$-powered, this forces $x
  = g$. Thus, $g \in H$, so $\varphi(g) = a$ is the identity element of
  $G/H$. Thus, there is no element of order $p$ in $G/H$, as desired.
\end{proof}

\begin{theorem}\label{thm:nilpotent-quotient-piequalsqpi}
  Suppose $G$ is a group and $H$ is a normal subgroup of $G$ such that
  the quotient group $G/H$ is nilpotent. Then, if $p$ is a prime such
  that both $G$ and $H$ are $p$-powered, the quotient group $G/H$ is
  also $p$-powered.
\end{theorem}

\begin{proof}
  Clearly, $G/H$ is $p$-divisible by Lemma
  \ref{lemma:divisibility-inherit-quotient}. It is also $p$-torsion-free
  because $G$ and $H$ are both $p$-powered and by the preceding lemma
  (Lemma \ref{lemma:powered-subgroup-implies-torsion-free-quotient}).

  Thus, by the lemma before last (Lemma
  \ref{lemma:nilpotent-powered-iff-d-t-f}), $G/H$ is $p$-powered.
\end{proof}

We can now state a fundamental result about normal subgroups of
nilpotent groups with {\em two} different proofs.

\begin{theorem}\label{thm:normalsubgroupofnilpotentgroup}
  Suppose $G$ is a nilpotent group and $H$ is a normal subgroup of
  $G$. Then, if $p$ is a prime number such that both $G$ and $H$ are
  $p$-powered, then the quotient group $G/H$ is also $p$-powered.
\end{theorem}

\begin{proof}
  {\em First proof alternative}: Use Theorem
  \ref{thm:normalinucspiequalsqpi}, noting that since $G$ is nilpotent,
  $H$ must lie in an upper central series member of $G$, namely $G$ itself.

  {\em Second proof alternative}: Use Theorem
  \ref{thm:nilpotent-quotient-piequalsqpi}, noting that since $G$ is
  nilpotent, so is $G/H$.
\end{proof}

\subsection{Definition equivalence for divisible nilpotent groups with important corollaries}

This result is dual to Theorem \ref{thm:torsion-free-equivalence-theorem},
the chief result of the preceding section.

\begin{theorem}\label{thm:divisibility-equivalence-theorem}
  The following are equivalent for a nilpotent group $G$ and a prime
  number $p$.

  \begin{enumerate}
  \item $G$ is $p$-divisible.
  \item The abelianization of $G$ is $p$-divisible.
  \item For every positive integer $i$, the quotient group
    $\gamma_i(G)/\gamma_{i+1}(G)$ is $p$-divisible.
  \item For all pairs of positive integers $i < j$, the quotient group
    $\gamma_i(G)/\gamma_j(G)$ is $p$-divisible.
  \end{enumerate}

  Moreover, the implications (1) to (2) to (3) to (4) hold for
  {\em all} groups. It is only the implication from (4) to (1) that
  uses that the group is nilpotent.
\end{theorem}

\begin{proof}
  (1) implies (2): This follows from Lemma
  \ref{lemma:divisibility-inherit-quotient}. Note that this step does
  not use $G$ being nilpotent.

  (2) implies (3): Note that each $\gamma_i(G)/\gamma_{i+1}(G)$ is
  $p$-divisible on account of being a homomorphic image of a tensor
  power of the abelianization of $G$. This again does not use $G$ nilpotent.

  (3) implies (4): We do mathematical induction using Lemma
  \ref{lemma:divisibility-extension-group}. This step again does not use
  that $G$ is nilpotent.

  (4) implies (1): plug in $i = 1$, $j = c + 1$ where $c$ is the
  nilpotency class of $G$. Note that this is the only step where we
  use that $G$ is nilpotent.
\end{proof}

\begin{theorem}\label{thm:powering-lcs}
  Suppose $G$ is a nilpotent group that is powered over a prime
  $p$. Then:

  \begin{enumerate}
  \item All members of the lower central series of $G$ are
    $p$-powered, i.e., $\gamma_i(G)$ is $p$-powered for all $p$.
  \item All quotients between members of the lower central series of
    $G$ are $p$-powered, i.e., $\gamma_i(G)/\gamma_j(G)$ is
    $p$-powered for all $i < j$. In particular, the abelianization of
    $G$ is $p$-powered.
  \end{enumerate}
\end{theorem}

\begin{proof}
  {\em Proof of (1)}: Since $G$ is $p$-powered, it is
  $p$-divisible. Hence, by the preceding theorem (Theorem
  \ref{thm:divisibility-equivalence-theorem}), all quotients
  $\gamma_i(G)/\gamma_j(G)$ are $p$-divisible. Let $c$ be the
  nilpotency class of $G$. Setting $j = c + 1$, we get that all the
  lower central series members $\gamma_i(G)$ are $p$-divisible. But
  since $G$ is $p$-{\em powered} (i.e., uniqueness of roots), this
  forces all of these subgroups to also be $p$-powered.

  {\em Proof of (2)}: Note that since $G$ is nilpotent, so are
  $\gamma_i(G)$ and $\gamma_j(G)$ for any $i < j$. Further,
  $\gamma_j(G)$ is characteristic in $G$, hence normal in $G$, hence
  normal in $\gamma_i(G)$. Thus, the hypotheses of Theorem
  \ref{thm:normalsubgroupofnilpotentgroup} apply, and we get the
  conclusion that the quotient group $\gamma_i(G)/\gamma_j(G)$ is
  $p$-powered.
\end{proof}

\subsection{Divisibility and the upper central series}

We now mention and prove a result about the upper central series whose
``dual'' (in the heuristic sense) fails to hold.

\begin{theorem}\label{thm:divisibility-upper-central-series}
  Suppose $G$ is a nilpotent group and $p$ is a prime number such that
  $G$ is $p$-divisible. Then, all members of the upper central series
  of $G$ are $p$-divisible.
\end{theorem}

The proof is somewhat unusual in the following sense: it proceeds
using mathematical induction starting from the {\em largest} member of
the upper central series and going down. Usually, when we use the
upper central series for induction, we move upward. This is one reason
why the proof is difficult to discover, even though it is not hard to
explain.

\begin{proof}
  Suppose $G$ has nilpotency class $c \ge 2$ (note that if $c = 1$
  there is nothing to prove).

  Consider the $c$-fold left-normed Lie bracket map of the form:

  $$T:(x_1,x_2,\dots,x_c) \mapsto [\dots [[x_1,x_2],x_3],\dots,x_c]$$

  By Lemma \ref{lemma:iterated-commutator-is-multilinear} in the
  Appendix, this map is a homomorphism in each coordinate holding the
  other coordinates fixed. Note that this fact is very specific to the
  class being $c$. It fails for higher class. Moreover, the set of
  values for $x_1$ for which the output is always the identity is
  precisely the subgroup $Z^{c-1}(G)$.

  Now, suppose $g \in Z^{c-1}(G)$. Since $G$ is $p$-divisible, there
  exists $x \in G$ such that $x^p = g$. The goal is to show that there
  exists such a value of $x$ in $Z^{c-1}(G)$ satisfying $x^p = g$. In
  fact, we will do better. We will show that any $x \in G$ satisfying
  $x^p = g$ actually lies inside $Z^{c-1}(G)$. In other words, we want
  to show that $T(x,x_2,\dots,x_c)$ is the identity element of $G$ for
  all $x_2,x_3,\dots,x_c \in G$.

  Fix the values of $x_2,x_3,\dots,x_c$ temporarily. Let $u$ be an
  element of $G$ such that $u^p = x_c$. Then, we know that:

  $$T(g,x_2,\dots,x_{c-1},u) = T(x^p,x_2,\dots,x_{c-1},u) = T(x,x_2,\dots,x_{c-1},u)^p$$

  Similarly:

  $$T(x,x_2,\dots,x_{c-1},x_c) = T(x,x_2,\dots,x_{c-1},u^p) = T(x,x_2,\dots,x_{c-1},u)^p$$

  The right sides of both equations are the same, so we get:

  $$T(g,x_2,\dots,x_{c-1},u) = T(x,x_2,\dots,x_{c-1},x_c)$$

  Since $g \in Z^{c-1}(G)$, the left side is the identity element,
  hence so is the right side. Since $x_2,\dots,x_{c-1},x_c$ are
  arbitrary, this shows that $x \in Z^{c-1}(G)$.

  The result can now be extended further down the upper central
  series. The key trick in executing the extension is to replace the
  original $c$-fold commutator with smaller fold commutators, but now
  restrict the first input to being within the member one higher. In
  general when inducting down from $Z^i(G)$ to $Z^{i-1}(G)$, we
  consider a left-normed commutator of length $i$, restricting the
  first input to be within $Z^i(G)$ and allowing all other inputs to
  vary freely within $G$. We then use the same logic. Note that this
  logic works in each stage as long as $i \ge 2$, and hence we can do
  our induction all the way down to the center. We cannot use the
  induction to get down to the trivial subgroup, but we know that the
  trivial subgroup is $p$-divisible for all primes $p$, so this is
  unnecessary.
\end{proof}

As we shall see in Example (4) in the next subsection, the naive dual
statement for torsion-free and quotient groups with the lower central
series fails to hold.

\subsection{Collection of interesting counterexamples}\label{sec:group-ctex}

For the examples below, we denote by $UT(3,R)$ the group whose
underlying set is the set of unitriangular matrices of degree three
over $R$ under matrix multiplication where $R$ is any unital ring. In
other words, $UT(3,R)$ is the set:

$$\left \{\begin{pmatrix}
1 & a_{12} & a_{13} \\
0 & 1 & a_{23}\\
0 & 0 & 1\end{pmatrix} \mid a_{12},a_{13},a_{23} \in R \right \}$$

with the usual matrix multiplication.
\begin{enumerate}
\item {\em A group may be $p$-powered and have a subgroup that is not
  $p$-powered; in fact, we can choose an abelian example}: The
  subgroup $\mathbb{Z}$ inside $\mathbb{Q}$ is an example of a
  situation where the whole group is powered over every prime but the
  subgroup is not powered over any prime. Note that it is the
  ``divisible'' aspect, not the ``torsion-free'' aspect, that
  fails. The corresponding quotient group ($\mathbb{Q}/\mathbb{Z}$) is
  also not $p$-powered for any prime $p$. For the quotient, it is the
  ``torsion-free'' aspect that fails.

  We can tweak this example a bit to construct, for any pair of prime
  sets $\pi_2 \subseteq \pi_1$, an abelian group that is powered over
  $\pi_1$ and a subgroup that is powered {\em only} over the primes
  inside $\pi_2$.
\item {\em A characteristic subgroup of a $p$-powered group need not
  be $p$-powered}: It is possible to have a characteristic subgroup of
  a group that is not powered over some primes that the whole group is
  powered over. Recall that $GA^+(1,\R) = \R \rtimes (\R^*)^+$ is
  rationally powered. The subgroup $\R \rtimes (\Q^*)^+$ is
  characteristic, but is not powered over any prime.  

  We can tweak this example a bit to construct, for any pair of prime
  sets $\pi_2 \subseteq \pi_1$, a group that is powered {\em
    precisely} over the primes in $\pi_1$ and a characteristic
  subgroup that is powered {\em precisely} over the primes inside
  $\pi_2$.

\item It is possible to have a nilpotent group $G$ (in fact, we can
  choose $G$ to have class two) such that the abelianization of $G$ is
  rationally powered (hence torsion-free), but $G$ itself is not
  torsion-free: We can take $G$ to be the quotient of $UT(3,\Q)$ by a
  subgroup $\Z$ inside its central $\Q$. The abelianization of $G$ is
  $\Q \times \Q$, while the center is $\Q/\Z$. Thus, the
  abelianization of $G$ is rationally powered, whereas $G$ has
  $p$-torsion for all primes $p$.

  Explicitly, $G$ is given as the set of matrices:

  $$\left \{ \begin{pmatrix} 1 & a_{12} & \overline{a_{13}} \\ 0 & 1 & a_{23} \\ 0 & 0 & 1 \\\end{pmatrix} \mid a_{12},a_{23} \in \mathbb{Q}, \overline{a_{13}} \in \mathbb{Q}/\mathbb{Z} \right \}$$
    
  with the matrix multiplication defined as:

  $$\begin{pmatrix} 1 & a_{12} & \overline{a_{13}} \\ 0 & 1 & a_{23} \\ 0 & 0 & 1 \\\end{pmatrix}\begin{pmatrix} 1 & b_{12} & \overline{b_{13}} \\ 0 & 1 & b_{23} \\ 0 & 0 & 1 \\\end{pmatrix} = \begin{pmatrix} 1 & a_{12} + b_{12} & \overline{a_{12}b_{23}} + \overline{a_{13}} + \overline{b_{13}} \\ 0 & 1 & a_{23} + b_{23} \\ 0 & 0 & 1 \\\end{pmatrix}$$

\item It is possible to have a nilpotent group $G$ (in fact, we can
  choose $G$ to have class two) such that $G$ is torsion-free but the
  abelianization of $G$ is not torsion-free: Let $G$ be a central
  product of $UT(3,\Z)$ by $\Q$ identifying a copy of $\Z$ inside $\Q$
  with the center of $UT(3,\Z)$. In this case, the abelianization is
  isomorphic to $\Z \times \Z \times \Q/\Z$, which has torsion for all
  primes.

  Explicitly, $G$ is the following set with matrix multiplication:

  $$\left \{\begin{pmatrix}
1 & a_{12} & a_{13} \\
0 & 1 & a_{23}\\
0 & 0 & 1\end{pmatrix} \mid a_{12},a_{23} \in \Z, a_{13} \in \Q \right \}$$

  This result is the expected dual to Theorem
  \ref{thm:divisibility-upper-central-series} that fails to hold. The
  expected dual to that result should say that all quotients of a
  $p$-torsion-free nilpotent group by its lower central series members
  are also $p$-torsion-free. This fails to be true in this situation.
\item {\em It is possible to have a non-nilpotent $p$-divisible group
  whose center is not $p$-divisible}: The simplest example is $S^3
  \cong SU(2,\mathbb{C})$, which can also be described as the group of
  unit quaternions. The center is $\{-1,1\}$. The group $S^3$ as a
  whole is $p$-divisible for all primes $p$, and in particular, every
  element in the group has a square root in the group. However, the
  center of the group is not $2$-divisible, because the element $-1$
  has no square root in the center.

  Note that this example is an opposite of sorts to potential
  generalizations of Lemma \ref{lemma:centerispoweringinvariant} (which
  rules out similar examples for the $p$-powered case) and Theorem
  \ref{thm:divisibility-upper-central-series} (which rules out similar
  examples where the whole group is nilpotent).

\item {\em It is possible to have a non-nilpotent $p$-divisible group
  whose derived subgroup is not $p$-divisible}: Consider the group
  $GL(p,\mathbb{C})$. This is $p$-divisible (and in fact, also
  divisible by all other primes). The derived subgroup
  $SL(p,\mathbb{C})$ is not $p$-divisible (however, it is divisible by
  other primes). Specifically, the element in $SL(p,\mathbb{C})$ that
  is a single Jordan block with eigenvalue a primitive $p^{th}$ root
  of unity has no $p^{th}$ root within $SL(p,\mathbb{C})$, even though
  it does have $p^{th}$ roots in $GL(p,\mathbb{C})$.

  More generally, for a finite set $\pi$ of primes, we can take $n$ as
  the product of all primes in $\pi$. Then $G = GL(n,\mathbb{C})$ is
  divisible by all primes, but $G' = SL(n,\mathbb{C})$ is divisible
  only by those primes that are not in $\pi$, and is not divisible by
  any of the primes in $\pi$.
\end{enumerate}

\subsection{Extra: a two-out-of-three theorem}

We are now ready to prove or ``two-out-of-three'' result for
powering. We begin with a lemma, which is structurally quite similar
to, and in some ways a generalization of, Theorem
\ref{thm:divisibility-equivalence-theorem}.

\begin{lemma}\label{lemma:lcs-subgroup-divisibility}
  Suppose $G$ is a nilpotent group, $p$ is a prime number, and $H$ is
  a $p$-divisible normal subgroup of $G$. Then, consider the
  descending chain:

  $$H \ge [H,G] \ge [[H,G],G] \ge [[[H,G],G],G] \ge \dots \ge 1$$

  Note that this chain reaches the trivial subgroup because $G$ is nilpotent.

  Then, the following are true:

  \begin{enumerate}
  \item Each of the quotient groups between successive members of this descending series is $p$-divisible.
  \item Each of the quotient groups between members of this descending
    series (not necessarily successive) is $p$-divisible.
  \item Each of the members of this descending series is $p$-divisible.
  \end{enumerate}
\end{lemma}

\begin{proof}
  {\em Proof of (1)}: Once we note that divisibility inherits to
  quotient groups (Lemma \ref{lemma:divisibility-inherit-quotient}), the
  situation for each quotient is similar to the situation for the last
  (final) quotient, so for notational simplicity, we prove the result
  only for the last quotient.

  Suppose the penultimate member of the series involves $c - 1$
  occurrences of $G$ and one occurrence of $H$. Thus, there is a
  $c$-fold iterated commutator map:

  $$T: H \times G \times G \times \dots \times G \to G$$

  whose image generates this subgroup. This map is multilinear, i.e.,
  it is a homomorphism in each coordinate. We can therefore use the
  $p$-divisibility of $H$ to obtain that the image set is
  $p$-divisible, and hence, so is the abelian subgroup generated by
  it.

  {\em Proof of (2)}: This follows from (1), Lemma
  \ref{lemma:divisibility-extension-group}, and mathematical induction.

  {\em Proof of (3)}: This is a special case of (2) where the lower
  end of the quotient is taken to be the trivial subgroup.
\end{proof}

This is sufficient for the following theorem:

\begin{theorem}\label{thm:divisibility-extension-group-general}
  Suppose $G$ is a nilpotent group, $H$ is a normal subgroup, and $p$
  is a prime number such that both $H$ and $G/H$ are
  $p$-divisible. Then, $G$ is also $p$-divisible.
\end{theorem}

\begin{proof}
  Consider the series:

  $$H \ge [H,G] \ge [[H,G],G] \ge \dots \ge 1$$

  For simplicity, define $H_1 = H$ and $H_{i+1} = [H_i,G]$. Then the series is:

  $$H_1 \ge H_2 \ge H_3 \ge \dots \ge 1$$

  By Lemma \ref{lemma:lcs-subgroup-divisibility} and the fact that $H$ is
  $p$-divisible, all the successive quotients $H_i/H_{i+1}$ are also
  $p$-divisible. Further, by the nature of the series, each $H_i$ is
  normal in $G$ and each quotient $H_i/H_{i+1}$ is central in the
  quotient $G/H_{i+1}$.

  We can now prove, by upward induction on $i$, that each quotient
  $G/H_i$ is $p$-divisible. The base case $i = 1$ follows from the
  stipulation that $G/H$ is $p$-divisible. For the inductive step,
  suppose $G/H_i$ is $p$-divisible and we need to show that
  $G/H_{i+1}$ is $p$-divisible. We already have that $H_i/H_{i+1}$ is
  $p$-divisible and central in $G/H_{i+1}$, and $G/H_i \cong
  (G/H_{i+1})/(H_1/H_{i+1})$ by the third isomorphism theorem, so $
  (G/H_{i+1})/(H_1/H_{i+1})$ is also $p$-divisible. Thus, by Lemma
  \ref{lemma:divisibility-extension-group}, $G/H_{i+1}$ is also
  $p$-divisible. This completes the inductive step.

  For large enough $i$, $H_i$ is the trivial subgroup of $G$, so this
  indeed gives us that $G$ is $p$-divisible.
\end{proof}

The next lemma tries to do something similar for the torsion-free
setting. It mimics and generalizes Theorem
\ref{thm:torsion-free-equivalence-theorem}.

\begin{lemma}\label{lemma:ucs-torsion-free-intersection}
  Suppose $G$ is a nilpotent group, $p$ is a prime number, and $H$ is
  a $p$-torsion-free normal subgroup of $G$. There exists some natural
  number $n$ such that $H \le Z^n(G)$. Consider the ascending chain of
  subgroups in $H$:

  $$1 \le H \cap Z(G) \le H \cap Z^2(G) \le \dots \le H \cap Z^n(G) = H$$

  We have the following:

  \begin{enumerate}
  \item Each of the quotient groups between successive members of this
    ascending series is $p$-torsion-free.
  \item Each of the quotient groups between members of this ascending
    series (not necessarily successive) is $p$-torsion-free.
  \item Each of the members of this ascending series is $p$-torsion-free.
  \end{enumerate}
\end{lemma}

\begin{proof}
  {\em Proof of (1)}: We prove this for each quotient $(H \cap
  Z^i(G))/(H \cap Z^{i-1}(G))$ by induction on $i$. The base case for
  induction, namely, the statement that $H \cap Z(G)$ is
  $p$-torsion-free, follows from the fact that $H$ is
  $p$-torsion-free. The proof method for the inductive step is similar
  to the proof method used for Lemma \ref{lemma:ucs-torsion-free}.

  Explicitly, we want to show that if $(H \cap Z^i(G))/(H \cap
  Z^{i-1}(G))$ is $p$-torsion-free, then $(H \cap Z^{i+1}(G))/(H \cap
  Z^i(G))$ is $p$-torsion-free. We do this as follows.

  Suppose $x$ is an element of $H \cap Z^{i+1}(G)$ whose image in $(H
  \cap Z^{i+1}(G))/(H \cap Z^i(G))$ has order $1$ or $p$. Our goal
  will be to show that $x \in H \cap Z^i(G)$, i.e., the order of the
  image of $x$ must be $1$ and cannot be $p$.

  For any $y \in G$, we have $[x,y] \in H \cap Z^i(G)$, and moreover, we
  have:

  $$[x,y]^p = [x^p,y] \pmod{H \cap Z^{i-1}(G)}$$

  Since $x^p \in H \cap Z^i(G)$, the right side is the identity
  element mod $H \cap Z^{i-1}(G)$, hence $[x,y]$ is an element of $H
  \cap Z^i(G)$ whose image in $(H \cap Z^i(G))/(H \cap Z^{i-1}(G))$
  has $p^{th}$ power the identity. Thus, $[x,y]$ taken modulo $H \cap
  Z^{i-1}(G)$ has order either $1$ or $p$. The order cannot be $p$
  because by the inductive hypothesis, $(H \cap Z^i(G))/(H \cap
  Z^{i-1}(G))$ is $p$-torsion-free. Hence, $[x,y] \in H \cap Z^{i-1}(G)$.

  Since the above is true for {\em all} $y \in G$, we obtain that
  $[x,y] \in H \cap Z^{i-1}(G)$ and hence $[x,y] \in Z^{i-1}(G)$ for
  all $y \in G$. This forces $x \in Z^i(G)$ and hence $x \in H \cap
  Z^i(G)$, so that the order of the image of $x$ in $(H \cap
  Z^{i+1}(G))/(H \cap Z^i(G))$ is in fact $1$. Thus, the order can
  never be $p$, showing that $(H \cap Z^{i+1}(G))/(H \cap Z^i(G))$ is
  $p$-torsion-free.

  {\em Proof of (2)}: This follows from (1) and Lemma
  \ref{lemma:powering-injective-extension-group} (note that we have already
  established, in Theorem \ref{thm:torsion-free-equivalence-theorem}, that
  being $p$-torsion-free is equivalent to the $p$ power map being
  injective).

  {\em Proof of (3)}: This is already obvious.
\end{proof}

We can now state the theorem:

\begin{theorem}\label{thm:torsion-free-extension-group-general}
  Suppose $G$ is a nilpotent group, $H$ is a normal subgroup, and $p$
  is a prime number such that both $H$ and $G/H$ are
  $p$-torsion-free. Then, $G$ is also $p$-torsion-free.
\end{theorem}

\begin{proof}
  By the preceding lemma, we have the ascending chain of subgroups in
  $H$:

  $$1 \le H \cap Z(G) \le H \cap Z^2(G) \le \dots \le H \cap Z^n(G) = H$$

  and further, we have that each successive quotient between members
  of this ascending chain is $p$-torsion-free. We can now induct
  downward on $i$ (going down from $n$ to $0$) to show that $G/(H \cap
  Z^i(G))$ is $p$-torsion-free. The base case, $i = n$, is given to
  us. To induct down from $G/(H \cap Z^i(G))$ to $G/(H \cap
  Z^{i-1}(G))$, we use Lemma \ref{lemma:powering-injective-extension-group},
  along with the observation that the abelian group $(H \cap
  Z^i(G))/(H \cap Z^{i-1}(G))$ is $p$-torsion-free if and only if the
  $p$-powering map in the group is injective.
\end{proof}

Combining the theorems for divisibility and torsion-free, we have that:

\begin{theorem}\label{thm:powering-extension-group-general}
  Suppose $G$ is a nilpotent group, $H$ is a normal subgroup, and $p$
  is a prime number such that both $H$ and $G/H$ are
  $p$-powered. Then, $G$ is also $p$-powered.
\end{theorem}

\begin{proof}
  This is a direct combination of Theorems
  \ref{thm:divisibility-extension-group-general} and
  \ref{thm:torsion-free-extension-group-general}.
\end{proof}

We can now state the two-out-of-three theorem.

\begin{theorem}\label{thm:two-out-of-three}
  Suppose $G$ is a nilpotent group, $H$ is a normal subgroup, and $p$
  is a prime number. The following are true:

  \begin{enumerate}
  \item If any two of the three groups $G$, $H$, and $G/H$ is
    $p$-powered, so is the third.
  \item For $p$-divisibility, if $G$ is $p$-divisible, so is $G/H$,
    and if $H$ and $G/H$ are $p$-divisible, so is $G$.
  \item For $p$-torsion-free, if $G$ is $p$-torsion-free, so is $H$,
    and if $H$ and $G/H$ are $p$-torsion-free, so is $G$.
  \end{enumerate}
\end{theorem}

\begin{proof}
  {\em Proof of (1)}:

  \begin{itemize}
  \item $G$ and $H$ to $G/H$: This follows from Theorem
    \ref{thm:normalsubgroupofnilpotentgroup}.
  \item $G$ and $G/H$ to $H$: This follows from Lemma \ref{lemma:quottosub}.
  \item $H$ and $G/H$ to $G$: This follows from Theorem
    \ref{thm:powering-extension-group-general}.
  \end{itemize}

  {\em Proof of (2)}:

  \begin{itemize}
  \item $G$ to $G/H$: This follows from Lemma \ref{lemma:divisibility-inherit-quotient}.
  \item $H$ and $G/H$ to $G$: This follows from Theorem \ref{thm:divisibility-extension-group-general}.
  \end{itemize}

  {\em Proof of (3)}:

  \begin{itemize}
  \item $G$ to $H$: This is obvious.
  \item $H$ and $G/H$ to $G$: This follows from Theorem \ref{thm:torsion-free-extension-group-general}.
  \end{itemize}
\end{proof}

\subsection{Is every characteristic subgroup invariant under powering?}\label{sec:charpowering}

The following is a conjecture.

\begin{conjecture}\label{conj:charpowering}
  Suppose $G$ is a nilpotent group and $H$ is a characteristic
  subgroup of $G$. Suppose $\pi$ is a set of primes such that $G$ is
  $\pi$-powered. Then, $H$ is also $\pi$-powered.
\end{conjecture}

The conjecture appears to be open.\footnote{See \url{http://mathoverflow.net/questions/124295/characteristic-subgroup-of-nilpotent-group-that-is-not-invariant-under-powering}}

The corresponding result is false for solvable groups: for instance,
the group $G = GA^+(1,\R) = \R \rtimes (\R^*)^+$ is powered over all
primes, and the subgroup $H = \R \rtimes (\Q^*)^+$ is characteristic,
but $H$ is not powered over any prime.

The corresponding result is true for {\em abelian} groups, because the
map $x \mapsto x^{1/p}$ (which in additive notation becomes $x \mapsto
(1/p)x$) is an automorphism of $G$ for every $p \in \pi$.

Apart from abelian groups, there are some important types of nilpotent
groups for which the conjecture can be demonstrated to be true. For instance:

\begin{itemize}
\item The conjecture is trivially true for all finite nilpotent
  groups, because {\em every} subgroup of a $\pi$-powered finite group
  is $\pi$-powered by Lemma
  \ref{lemma:finite-groups-two-out-of-three}. Similarly, it is also true
  for periodic groups, i.e., groups in which every element has finite order.
\item The conjecture is true for groups for which we can find an
  automorphism that behaves like a power map on the successive
  quotient groups for some central series of the group. For instance,
  suppose $\pi = \{ p \}$ and $G$ is a $p$-powered nilpotent group of
  nilpotency class two. The conjecture holds for $G$ if we can find a
  central subgroup $H$ of $G$ and an automorphism $\sigma$ of $G$ such
  that $\sigma$ induces the automorphism $x \mapsto x^{p^k}$ on $G/H$
  and $\sigma$ behaves like $x \mapsto x^{p^{2k}}$ on $H$.
\end{itemize}

In Section \ref{sec:baer-correspondence-char-sub}, we will discuss
some implications of this conjecture being true when restricted to the
groups that participate in the Baer correspondence.

%\newpage

\section{Lie rings powered over sets of primes: key results}\label{sec:lie-ring-powering}

This section covers Lie ring analogues of the material covered for
groups in the preceding section (Section
\ref{sec:group-powering}). The majority of results carry over, but
many of the proofs are notably similar. We use results of the
preceding section.

\subsection{Definitions}\label{sec:lie-ring-powering-def}

We can define notions of powered, divisible, and torsion-free for Lie
rings similar to the definitions for groups. In fact, we do not need
to redefine these terms: we simply define them by invoking the
corresponding definition for the additive group of the Lie ring. Thus,
for instance, a Lie ring is called $p$-powered for a prime $p$ if and
only if the additive group of the Lie ring is $p$-powered. Similarly,
for a prime set $\pi$, a Lie ring is called $\pi$-powered if and only
if the additive group of the Lie ring is $\pi$-powered.

$\pi$-powered Lie rings are the same as $\Z[\pi^{-1}]$-Lie
algebras. For a more detailed discussion of Lie algebras over rings
other than $\Z$, see the Appendix, Section \ref{appsec:Lie}.

We will now go over results analogous to the results we established
about groups in the preceding section. The proofs in most cases are
either the same or much simpler. For instance, we note the following
two-out-of-three result in the Lie ring context which is not true in
general for groups and took a lot of effort to establish for nilpotent groups:

\begin{lemma}\label{lemma:two-out-of-three-lie}
  Suppose $L$ is a Lie ring and $I$ is an ideal in $L$. Suppose $p$ is
  a prime number. Then, if any two of $L$, $I$, and $L/I$ are $p$-powered,
  so is the third.
\end{lemma}

\begin{proof}
  This is straightforward just from looking at the additive group
  structure and invoking the corresponding result for abelian groups,
  where it is obvious:

  \begin{itemize}
  \item $L$ and $I$ to $L/I$: This follows from Lemma
    \ref{lemma:picentralimpliesqpi}, applied to the additive group of $L$
    and the additive subgroup $I$.  Note that since $L$ is abelian,
    the subgroup $I$ is central in the additive group sense,
    regardless of whether $I$ is central as an ideal in $L$.
  \item $L$ and $L/I$ to $I$: This follows from Lemma \ref{lemma:quottosub}
    applied the additive group of $L$ and the additive subgroup $I$.
  \item $I$ and $L/I$ to $L$: This can be deduced from Lemma
    \ref{lemma:powering-extension-group} applied to the additive group of
    $L$ and the additive subgroup $I$. Note that since $L$ is abelian,
    the subgroup $I$ is central in the additive group sense,
    regardless of whether $I$ is central as an ideal in $L$.
  \end{itemize}
\end{proof}

Another basic fact is this.

\begin{lemma}
  The following are equivalent for a Lie ring $L$ and a prime number $p$:

  \begin{enumerate}
  \item The multiplication by $p$ map is injective from $L$ to itself.
  \item The additive group of $L$ is $p$-torsion-free.
  \end{enumerate}
\end{lemma}

\begin{proof}
  This follows from the additive group of $L$ being an abelian group.
\end{proof}

\subsection{Results for the center}

\begin{lemma}\label{lemma:centerispoweringinvariant-lie}
  Suppose $L$ is a Lie ring that is powered over a prime $p$ and
  $Z(L)$ is the center of $L$. Then, $Z(L)$ is also powered over
  $p$. In other words, for any $g \in Z(L)$, there exists a unique $x
  \in Z(L)$ such that $px = g$.
\end{lemma}

\begin{proof}
  Since $L$ is $p$-powered, we know there exists unique $x \in L$ such
  that $px = g$. It therefore suffices to show that this unique $x$ is
  in $Z(L)$. In other words, we need to show that for any $y \in L$,
  $[x,y] = 0$. 

  Note that $[g,y] = 0$ on account of $g$ being in the center of $L$,
  so:

  $$0 = [g,y] = [px,y] = p[x,y]$$

  Since $L$ is powered over $p$, it is in particular
  $p$-torsion-free. Thus, we must have $[x,y] = 0$, completing the proof.
\end{proof}

Note that the analogous results to Lemma \ref{lemma:picentralimpliesqpi} and
Theorem \ref{thm:normalinucspiequalsqpi} are trivially obvious due to the
two-out-of-three lemma noted above. Thus, we can jump straight to:

\begin{theorem}\label{thm:ucsqpi-lie}
  Suppose $L$ is a Lie ring (not necessarily nilpotent) and $Z^n(L)$ is
  the $n^{th}$ member of the upper central series of $L$. Suppose $p$
  is a prime such that $L$ is $p$-powered. Then, $Z^n(L)$ and
  $L/Z^n(L)$ are also $p$-powered.
\end{theorem}

\begin{proof}
  Lemma \ref{lemma:centerispoweringinvariant-lie} and the two-out-of-three
  lemma give that $L/Z(L)$ is $p$-powered. Iterating inductively, we
  get that $L/Z^n(L)$ is $p$-powered. Again using the two-out-of-three
  lemma, we get that $Z^n(L)$ is $p$-powered.
\end{proof}

We now prove the Lie ring analogue to Lemma \ref{lemma:ucs-torsion-free}.

\begin{lemma}\label{lemma:ucs-torsion-free-lie}
  Suppose $L$ is a Lie ring (not necessarily nilpotent) and $p$ is a
  prime number. Suppose $i$ is a natural number. Then, if the quotient
  ring $Z^i(L)/Z^{i-1}(L)$ is $p$-torsion-free, so is the quotient
  ring $Z^{i+1}(L)/Z^i(L)$.
\end{lemma}

\begin{proof}
  Suppose $x$ is an element of $Z^{i+1}(L)$ whose image in
  $Z^{i+1}(L)/Z^i(L)$ has order $1$ or $p$. Our goal will be to show
  that $x$ must be in $Z^i(L)$, i.e., its image must be the zero
  element of $Z^{i+1}(L)/Z^i(L)$.

  For any $y \in L$, we have $[x,y] \in Z^i(L)$, and moreover, we
  have:

  $$p[x,y] = [px,y] \pmod{Z^{i-1}(L)}$$

  Since $px \in Z^i(L)$, the right side is the zero element mod
  $Z^{i-1}(L)$, hence $[x,y]$ is an element of $Z^i(L)$ whose image in
  $Z^i(L)/Z^{i-1}(L)$, when multiplied by $p$, gives $0$. Thus,
  $[x,y]$ taken modulo $Z^{i-1}(L)$ has order either $1$ or $p$. The
  order cannot be $p$ because by the inductive hypothesis,
  $Z^i(L)/Z^{i-1}(L)$ is $p$-torsion-free. Hence, $[x,y] \in
  Z^{i-1}(L)$.

  Since the above is true for {\em all} $y \in L$, we obtain that
  $[x,y] \in Z^{i-1}(L)$ for all $y \in L$. This forces $x \in
  Z^i(L)$, so that the order of the image of $x$ in
  $Z^{i+1}(L)/Z^i(L)$ is in fact $1$. Thus, the order can never be
  $p$, showing that $Z^{i+1}(L)/Z^i(L)$ is $p$-torsion-free.
\end{proof}

\subsection{Definition equivalence for torsion-free nilpotent Lie rings}

\begin{theorem}\label{thm:torsion-free-equivalence-theorem-lie}
  The following are equivalent for a nilpotent Lie ring $L$ and a prime
  number $p$.

  \begin{enumerate}
  \item $L$ is $p$-torsion-free.
  \item The center $Z(L)$ is $p$-torsion-free.
  \item Each of the successive quotients $Z^{i+1}(L)/Z^i(L)$ of the
    upper central series of $L$ is a $p$-torsion-free Lie ring.
  \item Each of the quotients $Z^i(L)/Z^j(L)$ is a $p$-torsion-free Lie ring.
  \end{enumerate}

  Moreover, the implication (1) to (2) to (3) to (4) holds in
  all Lie rings. The only implication that relies on $L$ being nilpotent
  is the implication from (4) to (1).
\end{theorem}

\begin{proof}
  (1) implies (2): This is immediate, since $Z(L)$ is a subring, and hence
  additive subgroup, of $L$.

  (2) implies (3): This follows from Lemma
  \ref{lemma:ucs-torsion-free-lie} and the principle of mathematical
  induction.

  (3) implies (4): This relies on Lemma
  \ref{lemma:powering-injective-extension-group} and mathematical induction.

  (4) implies (1): This is follows by setting $i = c, j = 0$.
\end{proof}

\subsection{Definition equivalences for divisible nilpotent Lie rings}

\begin{theorem}\label{thm:divisibility-equivalence-theorem-lie}
  The following are equivalent for a nilpotent Lie ring $L$ and a prime
  number $p$.

  \begin{enumerate}
  \item $L$ is $p$-divisible.
  \item The abelianization of $L$ is $p$-divisible.
  \item For every positive integer $i$, the quotient group
    $\gamma_i(L)/\gamma_{i+1}(L)$ is $p$-divisible.
  \item For all pairs of positive integers $i < j$, the quotient group
    $\gamma_i(L)/\gamma_j(L)$ is $p$-divisible.
  \end{enumerate}

  Note that the implications (1) to (2) to (3) to (4) hold for {\em
    all} Lie rings. It is only the implication from (4) to (1) that
  uses that the Lie ring is nilpotent.
\end{theorem}

\begin{proof}
  (1) implies (2): This follows from Lemma
  \ref{lemma:divisibility-inherit-quotient} applied to the additive group of
  $L$. Note that this step does not use $L$ being nilpotent.

  (2) implies (3): Note that each $\gamma_i(L)/\gamma_{i+1}(L)$ is
  $p$-divisible on account of being a homomorphic image of a tensor
  power of the abelianization of $L$ (as an additive group). This
  again does not use $L$ being nilpotent.

  (3) implies (4): We do mathematical induction using Lemma
  \ref{lemma:divisibility-extension-group}. This step again does not use
  that $L$ is nilpotent.

  (4) implies (1): Plug in $i = 1$, $j = c + 1$ where $c$ is the
  nilpotency class of $L$. Note that this is the only step where we
  use that $L$ is nilpotent.
\end{proof}

Note that the following is true for {\em any} Lie ring (not
necessarily nilpotent). The corresponding statement for groups does
not hold in the general case (see Section \ref{sec:group-ctex},
Example 6).

\begin{lemma}\label{lemma:lie-ring-lcs-divisibility}
  Suppose $L$ is a Lie ring (not necessarily nilpotent) that is
  $p$-divisible for some prime number $p$. Then, the derived subring
  of $L$ and all the members of the lower central series of $L$ are
  also $p$-divisible. Moreover, if $L$ is $p$-powered, then the
  derived subring of $L$ and all the members of the lower central
  series of $L$ are also $p$-powered.
\end{lemma}

\begin{proof}
  Every Lie element of the form $[x,y]$ can be divided by $p$ to give
  a Lie element. Explicitly, if $pu = x$, then $p[u,y] = [x,y]$. Since
  the derived subring is generated additively by Lie elements, every
  element of the derived subring can be divided by $p$ within the
  derived subring.

  A similar logic applies to other members of the lower central series.

  The statement for the $p$-powered case also follows.
\end{proof}

\subsection{Divisibility and the upper central series}

The result for the upper central series for a Lie ring has a similar
formulation and a similar proof to the corresponding result for a
group.

\begin{theorem}\label{thm:divisibility-upper-central-series-lie}
  Suppose $L$ is a nilpotent Lie ring and $p$ is a prime number such that
  $L$ is $p$-divisible. Then, all members of the upper central series
  of $L$ are $p$-divisible.
\end{theorem}

The proof is analogous to that for groups.

\begin{proof}
  Suppose $L$ has nilpotency class $c \ge 2$ (note that if $c = 1$
  there is nothing to prove).

  Consider the $c$-fold left-normed commutator map of the form:

  $$T:(x_1,x_2,\dots,x_c) \mapsto [\dots [[x_1,x_2],x_3],\dots,x_c]$$

  Viewed as a map of the additive groups, this map is a homomorphism
  in each coordinate when one fixes the values of the other
  coordinates. Moreover, the set of values for $x_1$ for which the
  output is always the zero element is precisely the subring
  $Z^{c-1}(L)$.

  Now, suppose $g \in Z^{c-1}(L)$. Since $L$ is $p$-divisible, there
  exists $x \in L$ such that $px = g$. The goal is to show that there
  exists a value of $x$ in $Z^{c-1}(L)$ satisfying $px = g$. In
  fact, we will do better. We will show that any $x \in L$ satisfying
  $px = g$ actually lies inside $Z^{c-1}(L)$. In other words, we want
  to show that $T(x,x_2,\dots,x_c)$ is the zero element of $L$ for
  all $x_2,x_3,\dots,x_c \in L$.

  Fix the values of $x_2,x_3,\dots,x_c$ temporarily. Let $u$ be an
  element of $L$ such that $pu = x_c$. Then, we know that:

  $$(g,x_2,\dots,x_{c-1},u) = T(px,x_2,\dots,x_{c-1},u) = pT(x,x_2,\dots,x_{c-1},u)$$

  Similarly:

  $$T(x,x_2,\dots,x_{c-1},x_c) = T(x,x_2,\dots,x_{c-1},pu) = pT(x,x_2,\dots,x_{c-1},u)$$

  The right sides of both equations are the same, so we get:

  $$T(g,x_2,\dots,x_{c-1},u) = T(x,x_2,\dots,x_{c-1},x_c)$$

  Since $g \in Z^{c-1}(L)$, the left side is the zero element,
  hence so is the right side. Since $x_2,\dots,x_{c-1},x_c$ are
  arbitrary, this shows that $x \in Z^{c-1}(L)$.

  The result can now be extended further down the upper central
  series. The key trick in executing the extension is to replace the
  original $c$-fold commutator with smaller fold commutators, but now
  restrict the first input to being within the member one higher. In
  general when inducting down from $Z^i(L)$ to $Z^{i-1}(L)$, we
  consider a left-normed commutator of length $i$, restricting the
  first input to be within $Z^i(L)$ and allowing all other inputs to
  vary freely within $L$. We then use the same logic. Note that this
  logic works in each stage as long as $i \ge 2$, and hence we can do
  our induction all the way down to the center. We cannot use the
  induction to get down to the zero subring, but we know that the zero
  subring is $p$-divisible for all primes $p$, so this is unnecessary.
\end{proof}

\subsection{Counterexamples for Lie rings}\label{sec:lie-ring-ctex}

For the examples below, we denote by $NT(3,R)$ the set of
niltriangular matrices of degree three over $R$ under matrix
multiplication where $R$ is any unital ring. In other words, $NT(3,R)$
is the set:

$$\left \{\begin{pmatrix}
0 & a_{12} & a_{13} \\
0 & 0 & a_{23}\\
0 & 0 & 0\end{pmatrix} \mid a_{12},a_{13},a_{23} \in R \right \}$$

with the usual matrix multiplication. To make it into a Lie ring, we
define the Lie bracket as the additive commutator corresponding to the
matrix multiplication. Explicitly, we define the Lie bracket of
matrices $x$ and $y$ as $[x,y] = xy - yx$ where $xy$ and $yx$ are the
products with respect to the usual matrix multiplication.

\begin{enumerate}
\item {\em A Lie ring may be $p$-powered and have a subring that is
  not $p$-powered; in fact, we can choose an abelian example}: The
  subgroup $\mathbb{Z}$ inside $\mathbb{Q}$ is an example of a
  situation where the whole group is powered over every prime but the
  subgroup is not powered over any prime. Viewing all the group as
  abelian Lie rings (with the trivial bracket) we obtain the desired
  examples for Lie rings. Note that it is the ``divisible'' aspect, not
  the ``torsion-free'' aspect, that fails. The corresponding quotient
  group ($\mathbb{Q}/\mathbb{Z}$) is also not $p$-powered for any
  prime $p$. For the quotient, it is the ``torsion-free'' aspect that
  fails.

  We can tweak this example a bit to construct, for any pair of prime
  sets $\pi_2 \subseteq \pi_1$, an abelian group (hence an abelian Lie
  ring) that is powered over $\pi_1$ and a subgroup (hence,
  subring) that is powered {\em only} over the primes inside $\pi_2$.

\item {\em A characteristic subring of a $p$-powered Lie ring need not
  be $p$-powered}: Consider the Lie ring $L = \Q \rtimes \Q$ where the
  action of the acting $\Q$ on the other $\Q$ is by rational number
  multiplication. The ring is rationally powered, i.e., it is a
  $\Q$-Lie algebra. However, the characteristic subring $\Q \rtimes
  \Z$ is not powered over any prime.

\item It is possible to have a nilpotent Lie ring $L$ (in fact, we can
  choose $L$ to have class two) such that the abelianization of $L$ is
  rationally powered (hence torsion-free), but $L$ itself is not
  torsion-free: We can take $L$ to be the quotient of $NT(3,\Q)$ by a
  subgroup $\Z$ inside its central $\Q$. The abelianization of $L$ is
  $\Q \times \Q$, while the center is $\Q/\Z$. Thus, the
  abelianization of $L$ is rationally powered, whereas $L$ has
  $p$-torsion for all primes $p$.

  Explicitly, $L$ is given as the set of matrices:

  $$\left \{ \begin{pmatrix} 0 & a_{12} & \overline{a_{13}} \\ 0 & 0 & a_{23} \\ 0 & 0 & 0 \\\end{pmatrix} \mid a_{12},a_{23} \in \mathbb{Q}, \overline{a_{13}} \in \mathbb{Q}/\mathbb{Z} \right \}$$
    
    
    with the Lie bracket defined as the bracket arising from matrix multiplication.

\item It is possible to have a nilpotent Lie ring $L$ (in fact, we can
  choose $L$ to have class two) such that $L$ is torsion-free but the
  abelianization of $L$ is not torsion-free: Let $L$ be a central
  product of $NT(3,\Z)$ by $\Q$ identifying a copy of $\Z$ inside $\Q$
  with the center of $UT(3,\Z)$. In this case, the abelianization is
  isomorphic to $\Z \times \Z \times \Q/\Z$, which has torsion for all
  primes.

  Explicitly, $L$ is:

  $$\left \{\begin{pmatrix}
0 & a_{12} & a_{13} \\
0 & 0 & a_{23}\\
0 & 0 & 0\end{pmatrix} \mid a_{12},a_{23} \in \Z, a_{13} \in \Q \right \}$$

  This result is the expected dual to Theorem
  \ref{thm:divisibility-upper-central-series} that fails to hold. The
  expected dual to that result should say that all quotients of a
  $p$-torsion-free nilpotent group by its lower central series members
  are also $p$-torsion-free. This fails to be true in this situation.
\end{enumerate}

\subsection{Is every characteristic subring invariant under powering?}

The following is a Lie ring analogue of Conjecture \ref{conj:charpowering}.

\begin{conjecture}\label{conj:charpowering-lie}
  Suppose $L$ is a nilpotent Lie ring and $M$ is a characteristic
  subring of $L$. Suppose $\pi$ is a set of primes such that $L$ is
  $\pi$-powered. Then, $M$ is also $\pi$-powered.
\end{conjecture}

The corresponding result is false for solvable Lie rings. See (2) in
the list of counterexamples in Section \ref{sec:lie-ring-ctex}.

The corresponding result is true for {\em abelian} Lie rings, because
the map $x \mapsto (1/p)x$ is an automorphism of $L$ for every $p \in
\pi$.

Apart from abelian Lie rings, there are some important types of
nilpotent groups for which the conjecture can be demonstrated to be
true. Remars about special cases similar to those found in Section
\ref{sec:charpowering} apply here.

In Section \ref{sec:baer-correspondence-char-sub}, we will discuss
some implications of this conjecture being true when restricted to the
Lie rings that participate in the Baer correspondence.

%\newpage

\section{Free powered groups and powering functors}\label{sec:free-powered-groups-and-powering-functors}

\subsection{Construction of the free powered group}

This section uses basic terminology from universal algebra. For
background on the terminology, see Section
\ref{appsec:univalg-basic}. The section also builds on Section
\ref{sec:variety-powered-forgetful-functor}, where we described how the
collection of $\pi$-powered groups for a prime set $\pi$ is a variety
of algebras with a natural forgetful functor to the variety of groups.

For any variety of algebras, we can talk of the free algebra in that
variety on any set. In particular, we can talk of the free
$\pi$-powered group $F(S,\pi)$ on a set $S$. Working with this free
group is difficult because, unlike the usual free group, it is
difficult to work out a reduced form for elements of the free
$\pi$-powered group. It is also obvious that the canonical map from
$S$ to $F(S,\pi)$ is injective. To see this, note that the free
$\pi$-powered {\em abelian} group on $S$ is the free
$\Z[\pi^{-1}]$-module with basis indexed by $S$, and this is a
quotient group of $F(S,\pi)$. Since the free $\pi$-powered abelian
group on $S$ has the property that the natural map from $S$ to it is
injective, the natural map from $S$ to $F(S,\pi$) is also injective.

An explicit construction of the free $\pi$-powered group on a set $S$
is as follows. Start with the abstract free group $F(S)$. In each
iteration, do the following:

\begin{itemize}
\item Adjoin $p^{th}$ roots (for all $p \in \pi$) of all the elements so far.
\item Take the free group generated by all these.
\item For every pair of elements that have the same
  $p^{th}$ power (for one or more $p \in \pi$), set them
  to be equal (i.e., factor out by the relation of their being
  equal).
\end{itemize}

The group constructed at each stage has a natural homomorphism to it
from the previous group. The direct limit of this sequence is the
desired group $F(S,\pi)$.

\subsection{Free powered nilpotent groups}\label{sec:free-powered-nilpotent}

In Section \ref{sec:free-nilpotent-groups-homology}, we defined the free
nilpotent group of class $c$ on a set $S$. We now define the
$\pi$-powered analogue of that construction.

\begin{definer}[Free $\pi$-powered nilpotent group]
  Suppose $S$ is a set, $\pi$ is a set of primes, and $c$ is a
  positive integer. The {\em free $\pi$-powered nilpotent group} of
  class $c$ on $S$ is defined as the quotient group
  $F(S,\pi)/\gamma_{c+1}(F(S,\pi))$ where $F(S,\pi)$ is the free
  $\pi$-powered group on $S$. Equivalently, this group, along with the
  set map to it from $S$, is the initial object in the category of
  groups of nilpotency class at most $c$ with set maps to them from
  $S$.

  The functor sending a set to its free $\pi$-powered nilpotent group
  of class $c$ is left adjoint to the forgetful functor from
  $\pi$-powered nilpotent groups of class (at most) $c$ to sets.
\end{definer}

\subsection{$\pi$-powered words and word maps}\label{sec:pi-powered-word-maps}

Word maps (described in the Appendix, Section \ref{appsec:word-maps})
are an important tool in the study of the variety of groups and other
varieties of algebras. In particular, we can define words and word
maps relative to the variety of $\pi$-powered groups. We will use the
jargon {\em $\pi$-powered word} to describe a word relative to the
variety of $\pi$-powered groups. A $\pi$-powered word in $n$ letters
$g_1,g_2,\dots,g_n$ can be described using an expression that involves
composing the operations of multiplication, inverses, and taking
$p^{th}$ roots for primes $p \in \pi$. Two such expressions define the
same word if they give the same element in the free $\pi$-powered
group $F(S,\pi)$ where $S = \{ g_1,g_2,\dots,g_n \}$.

For any $\pi$-powered group $G$ and any $\pi$-powered word $w$ in $n$
letters, we can define the $\pi$-powered word map on $G$ corresponding
to $w$. This is a map $G^n \to G$. By abuse of notation, we will
denote this map by the letter $w$ as well, i.e., for
$x_1,x_2,\dots,x_n \in G$, we denote the image of
$(x_1,x_2,\dots,x_n)$ under $w$ by $w(x_1,x_2,\dots,x_n)$.

For a prime set $\pi$ and a positive integer $c$, we can also consider
{\em $\pi$-powered class $c$ words}. These are words with respect to
the variety of $\pi$-powered groups of nilpotency class at most
$c$. We can correspondingly considered {\em $\pi$-powered class $c$
  word maps}. For a $\pi$-powered class $c$ word $w$ in $n$ letters
and a $\pi$-powered class $c$ group $G$, the word map induced by $w$
is a set map $G^n \to G$.

\subsection{Localization and powering functors}\label{sec:localization-and-powering-functors}

For a set $\pi$ of primes, the $\pi$-powering functor is a functor
from the category of groups to the category of $\pi$-powered groups
that is left adjoint to the forgetful functor from the category of
$\pi$-powered groups to the category of groups. More explicitly, for a
group $G$, the $\pi$-powering of $G$ is a group $K$ along with a
homomorphism $\varphi:G \to K$ such that for any homomorphism
$\theta:G \to L$ from $G$ to a $\pi$-powered group $L$, there is a
unique homomorphism $\alpha:K \to L$ such that $\theta = \alpha \circ
\varphi$.

For a prime set $\pi$, the $\pi$-{\em localization} functor refers to
the powering functor for the set of primes {\em outside of} $\pi$.

We begin with a lemma.

\begin{lemma}\label{lemma:pi-powering-arbitrary-group}
  Suppose $G$ is a group, $\pi$ is a set of primes, and $K$ is the
  $\pi$-powering of $G$ with the $\pi$-powering homomorphism
  $\varphi:G \to K$. The following are true:

  \begin{enumerate}
  \item Let $N$ be the kernel of $\varphi$. Then, $N$ contains all the
    elements of $G$ whose order is a $\pi$-number.
  \item $K$ is generated as a $\pi$-powered group by the image
    $\varphi(G)$. Equivalently, $K$ does not have any proper $\pi$-powered
    subgroup containing $\varphi(G)$.
  \end{enumerate}
\end{lemma}

\begin{proof}
  {\em Proof of (1)}: If $g \in G$ has order a $\pi$-number, then
  $\varphi(g)$ also has order a $\pi$-number, since the order of
  $\varphi(g)$ divides the order of $g$. However, $K$ is
  $\pi$-powered, hence $\pi$-torsion-free, so $\varphi(g)$ is the
  identity element of $K$. Thus, $g$ is in $N$, the kernel of
  $\varphi$.

  {\em Proof of (2)}: Viewing the $\pi$-powering functor as a ``free''
  functor, we see that $K$ is generated as a $\pi$-powered group by
  the image of $G$.
\end{proof}

In general, we cannot say much more: the kernel of the homomorphism
may be a lot bigger than the subgroup generated by $\pi$-torsion
elements. However, in the case of nilpotent groups, the kernel is
precisely the set of $\pi$-torsion elements (which in fact form a
subgroup), and the $\pi$-powering is itself a nilpotent group with the
same nilpotency class. We now develop the framework that will allow us
to get to proofs. We will prove this at the end of Section
\ref{sec:minimal-powered}.

\subsection{Root set of a subgroup}\label{sec:root-set}

Suppose $G$ is a group and $H$ is a subgroup of $G$. We denote by
$\sqrt[\pi]{H}$ (relative to the ambient group $G$) the set of all
elements $x \in G$ such that $x^n \in H$ for some $\pi$-number $n$. We
begin with some lemmas. Note that if $G$ is non-nilpotent,
$\sqrt[\pi]{H}$ need not be a subgroup of $G$.\footnote{For instance,
  let $G$ be the symmetric group $S_3$ and $H$ be the trivial
  subgroup. Let $\pi$ be the set $\{ 2 \}$. Then, $\sqrt[\pi]{H}$ is a
  subset of size four comprising the identity element and the three
  elements of order two, and it is not a subgroup.} Thus, the results
below do depend on the assumption of $G$ being nilpotent.

\begin{theorem}\label{thm:nilpotent-pi-root-subgroup}
  \begin{enumerate}
  \item Suppose $G$ is a nilpotent group, $H$ is a subgroup of $G$,
    and $\pi$ is a set of primes. Then, $\sqrt[\pi]{H}$ is also a
    subgroup of $G$.
  \item Suppose $G$ is a nilpotent group, $\pi$ is a set of primes,
    and $A,B$ are subgroups of $G$ with $A$ normal in $B$. Then,
    $\sqrt[\pi]{A}$ is normal in $\sqrt[\pi]{B}$.
  \end{enumerate}
\end{theorem}

This appears as Theorem 10.19 in Khukhro's book \cite{Khukhro}. The
book does not prove this result, but provides a similar, more
specialized proof for a related result, Theorem 9.18.

\begin{proof}
  {\em Proof of (1)}: We can assume without loss of generality $G = \langle
  \sqrt[\pi]{H} \rangle$. If not, simply replace $G$ by the subgroup
  $\langle \sqrt[\pi]{H} \rangle$ and proceed. 

  With this assumption, the goal is to show that $G = \sqrt[\pi]{H}$.

  We note that for any quotient map $\varphi:G \to M$, $\varphi(G) =
  \langle \sqrt[\pi]{\varphi(H)} \rangle$. In particular, this is true for
  quotient maps by lower central series members.

  We now prove that $G = \sqrt[\pi]{H}$ by induction on the nilpotency
  class of $G$. The base case for induction, namely the case of
  abelian groups, is obviously true. For the inductive step, assume we
  have established the result for class $c - 1$, and need to prove it
  for $G$ of class $c$.

  Any element of $\gamma_c(G)$ is a product of iterated $c$-fold
  commutators involving elements of $G$. Since the $c$-fold iterated
  commutator is multilinear, the element can be expressed as a product
  of iterated $c$-fold commutators involving elements of
  $\sqrt[\pi]{H}$, which is a generating set for $G$. Each such
  iterated commutator is of the form:

  $$u = [[\dots[[x_1,x_2],x_3],\dots,x_{c-1}],x_c]$$

  with $x_i \in \sqrt[\pi]{H}$. For each $x_i$, there exists a
  $\pi$-number $n_i$ such that $x_i^{n_i} \in H$, and then, using
  multilinearity, we get that:

  $$u^{n_1n_2 \dots n_c} = [[\dots[[x_1^{n_1},x_2^{n_2}],x_3^{n_3}],\dots,x_{c-1}^{n_{c-1}}],x_c^{n_c}]$$

  The number $n_1n_2 \dots n_c$ is a $\pi$-number since each $n_i$ is
  a $\pi$-number. Hence, a suitable power of $u$ in in $H \cap
  \gamma_c(G)$, so $u \in \sqrt[\pi]{H \cap \gamma_c(G)}$. Thus,
  $\gamma_c(G)$ is abelian and is generated by elements in
  $\sqrt[\pi]{H \cap \gamma_c(G)}$. Since a product of commuting
  elements of $\sqrt[\pi]{H \cap \gamma_c(G)}$ must also be in
  $\sqrt[\pi]{H \cap \gamma_c(G)}$, we get that:

  $$\gamma_c(G) \le \sqrt[\pi]{H \cap \gamma_c(G)}$$

  In particular, we get that:

  $$\gamma_c(G) \le \sqrt[\pi]{H}$$

  By the observation regarding quotients, we also have that:

  $$G/\gamma_c(G) = \langle \sqrt[\pi]{H\gamma_c(G)/\gamma_c(G)} \rangle$$

  By the inductive hypothesis, this gives us that:

  $$G/\gamma_c(G) = \sqrt[\pi]{H\gamma_c(G)/\gamma_c(G)}$$

  We now complete the proof. Suppose $g \in G$. From the fact about
  $G/\gamma_c(G)$, there exists a $\pi$-number $m$ such that $g^m \in
  H\gamma_c(G)$. Thus, $g^m = hu$ where $h \in H$ and $u \in
  \gamma_c(G)$ (which is in particular central). Since $\gamma_c(G)
  \le \sqrt[\pi]{H}$, there exists a $\pi$-number $n$ such that $u^n
  \in H$. Thus, $g^{mn} = (g^m)^n = (hu)^n = h^nu^n$ (since $u \in \gamma_c(G)$ is
  central) and this is an element of $H$. Thus, $mn$ is a $\pi$-number
  such that $g^{mn} \in H$, so $g \in \sqrt[\pi]{H}$.

  {\em Proof of (2)}: Please see the reference (\cite{Khukhro},
  Theorem 9.18 and 10.19). %%{\em TONOTDO: Eventually transfer the details here}.
\end{proof}

Note in particular that this shows that in a nilpotent group $G$,
$\sqrt[\pi]{1}$, i.e., the set of elements whose order is a
$\pi$-number, is a subgroup of $G$.

\subsection{Minimal powered group containing a torsion-free group}\label{sec:minimal-powered}

The following is a theorem from \cite{Khukhro} (Theorem 10.20, Page
122). %{\em TONOTDO: I will eventually insert Khukhro's proof}.

\begin{theorem}\label{thm:pi-powered-envelope}
  Suppose $\pi$ is a set of primes and $G$ is a $\pi$-torsion-free
  nilpotent group of class $c$.

  \begin{enumerate}
  \item There exists a $\pi$-powered group $\hat{G}^\pi$ of nilpotency
    class $c$ containing $G$ such that $\hat{G}^\pi = \sqrt[\pi]{G}$ is
    precisely the set of elements that arise as $n^{th}$ roots of
    elements of $G$ for $n$ varying over $\pi$-numbers.
  \item The group $\hat{G}^\pi$ is uniquely determined up to
    isomorphism. In particular, any automorphism of $G$ extends to an
    automorphism of $\hat{G}^\pi$.
  \item If $F$ is free nilpotent of class $c$, then $\hat{F}^\pi$ is
    the free nilpotent $\pi$-powered group of class $c$.
  \end{enumerate}
\end{theorem}

A few comments are in order here before we proceed. Note that
$\pi$-powered groups are very nicely behaved -- all their important
characteristic subgroups, quotients, and subquotients are
$\pi$-powered. For the most part, therefore, if we start with a
$\pi$-powered group and use deterministic processes, we will stay with
$\pi$-powered groups.

The $\pi$-torsion-free groups are not so nice. In the counterexamples
section (section \ref{sec:group-ctex}), we saw situations where a
torsion-free group has an abelianization that is not
torsion-free. This means that we need to proceed with a little more
care.

We now prove a statement we made at the end of Section
\ref{sec:localization-and-powering-functors}.

\begin{theorem}
  The following are true for a set of primes $\pi$:

  \begin{enumerate}
  \item Suppose $G$ is a $\pi$-torsion-free nilpotent group. Then,
    $\hat{G}^\pi$ is the $\pi$-powering of $G$ and the inclusion map
    $G \to \hat{G}^\pi$ is the natural homomorphism.
  \item Suppose $G$ is a nilpotent group and $T$ is the set of
    elements of $G$ whose order is a $\pi$-number. The quotient group
    $G/T$ is a $\pi$-torsion-free nilpotent group, $\hat{G/T}^\pi$ is
    the $\pi$-powering of $G$, and the composite of the quotient map $G
    \to G/T$ and the inclusion $G/T \to (\hat{G/T})^\pi$ is the
    natural homomorphism to the $\pi$-powering of $G$.
  \end{enumerate}
\end{theorem}

\begin{proof}
  {\em Proof of (1)}: Suppose $\varphi:G \to K$ is the natural
  homomorphism to the $\pi$-powering of $G$. Denote by $\theta: G \to
  \hat{G}^\pi$ the canonical inclusion map of Theorem
  \ref{thm:pi-powered-envelope}. By the universality of $K$, there exists
  $\alpha:K \to \hat{G}^\pi$ such that $\theta = \alpha \circ
  \varphi$.

  Since $\theta$ is injective, $\varphi$ is also injective, so we can
  view $G$ as a subgroup of $K$. Part (2) of Theorem
  \ref{thm:pi-powered-envelope} tells us that inside $K$, $\hat{G}^\pi =
  \sqrt[\pi]{G}$. Thus, $\hat{G}^\pi$ is a $\pi$-powered subgroup of
  $K$ containing $G$. By Lemma
  \ref{lemma:pi-powering-arbitrary-group}, $\hat{G}^\pi = K$.

  {\em Proof of (2)}: By Theorem \ref{thm:nilpotent-pi-root-subgroup}, $T
  = \sqrt[\pi]{1}$ is a subgroup of $G$. If any element in $G/T$ has
  $\pi$-torsion, then any representative $g$ for it in $G$ has the
  property that $g^n \in T$ for $n$ a $\pi$-number, and therefore,
  that $(g^n)^m = 1$ for $n$ and $m$ both $\pi$-numbers, forcing $g$
  to have order a $\pi$-number, so $g \in T$. Thus, $G/T$ is
  $\pi$-torsion-free. The rest of the proof is similar to (1).
\end{proof}

\subsection{Every $\pi$-powered class $c$ word is expressible as a root of an ordinary class $c$ word}

This theorem follows from Theorem \ref{thm:pi-powered-envelope}.

\begin{theorem}\label{thm:root-outside}
  Suppose $w$ is a $\pi$-powered class $c$ word in $n$ letters. Then,
  $w$ can be expressed as $v^{1/m}$ where $v$ is an ordinary word
  (i.e., a word using the group operations only, without any powering
  operations), and $m$ is a $\pi$-number, i.e., all the prime divisors
  of $m$ are in $\pi$.
\end{theorem}

\begin{proof}
  Denote by $F$ the free group on $n$ letters of nilpotency class $c$,
  and denote by $\hat{F}^\pi$ its $\pi$-powered envelope, which is
  clearly the free $\pi$-powered group on $n$ letters. $w$ can be
  described as an element of $\hat{F}^\pi$. By Theorem
  \ref{thm:pi-powered-envelope}, $\hat{F}^\pi = \sqrt[\pi]{F}$. Thus,
  there exists a $\pi$-number $m$ such that $w^m \in F$. Let $v =
  w^m$. The result follows.
\end{proof}

\subsection{Results about isoclinisms for $\pi$-powered nilpotent groups}\label{sec:pi-powered-isoclinism-results}

We now state and prove a $\pi$-powered analogue of Theorem
\ref{thm:iterated-commutator-descends-to-inn}. We will prove the
results in the context of nilpotent groups, since this will be the
area of primary application. Similar statements can be made for
non-nilpotent groups, but the notation and proof become messier, so we
restrict attention to the nilpotent case.

\begin{theorem}\label{thm:iterated-commutator-descends-to-inn-pi-powered}
  Suppose $c \ge 1$ and $\pi$ is any set of primes. Suppose
  $w(g_1,g_2,\dots,g_n)$ is a $\pi$-powered class $c$ word in $n$
  letters with the property that $w$ evaluates to the identity element
  in {\em any} $\pi$-powered abelian group. Then, for any group $G$,
  the word map $w:G^n \to G$ obtained by evaluating $w$ descends to a
  map:

  $$\chi_{w,G}: (\operatorname{Inn}(G))^n  \to G'$$

  Any word $w$ that is an iterated commutator (with any bracketing)
  satisfies this condition.
\end{theorem}

\begin{proof}
  By Theorem \ref{thm:root-outside}, we can write $w$ as $v^{1/m}$
  where $v$ is an ordinary word and $m$ is a $\pi$-number. Moreover,
  since $w$ is guaranteed to be satisfied in any $\pi$-powered abelian
  group, so is $v$. Thus, $v$ is satisfied in the vector space over
  the rationals generated by the $n$ letters. So, $v$ is satisfied in
  the free abelian group generated by the $n$ letters, and therefore
  $v$ is satisfied in any abelian group. Thus, Theorem
  \ref{thm:iterated-commutator-descends-to-inn} applies to the word
  $v$, and we obtain that the map descends to a map:

  $$\chi_{v,G}: \operatorname{Inn}(G))^n \to G'$$

  Since $w = v^{1/m}$ and $G'$ is $\pi$-powered by Theorem
  \ref{thm:powering-lcs}, we can obtain the map:

  $$\chi_{w,G}: \operatorname{Inn}(G))^n \to G'$$
\end{proof}

The next theorem is related to Theorem \ref{thm:iterated-commutator-commutes-homoclinisms}.

\begin{theorem}\label{thm:iterated-commutator-commutes-homoclinisms-pi-powered}
  Suppose $c \ge 1$, $\pi$ is a set of primes, and $(\zeta,\varphi)$
  is a homoclinism of $\pi$-powered class $c$ groups $G_1$ and $G_2$,
  where $\zeta:\operatorname{Inn}(G_1) \to \operatorname{Inn}(G_2)$
  and $\varphi:G_1' \to G_2'$ are the component homomorphisms. Then
  for any $\pi$-powered class $c$ word $w(g_1,g_2,\dots,g_n)$ that is
  trivial in every $\pi$-powered abelian group (as described above),
  we have:

  $$\chi_{w,G_2}(\zeta(x_1),\zeta(x_2),\dots,\zeta(x_n)) = \varphi(\chi_{w,G_1}(x_1,x_2,\dots,x_n))$$

  for all $x_1,x_2,\dots,x_n \in \operatorname{Inn}(G)$.

  Any word $w$ that is an iterated commutator (with any order of
  bracketing) satisfies this condition, and the theorem applies to
  such word maps.
\end{theorem}

\begin{proof}
  By Theorem \ref{thm:root-outside}, we can write $w$ as $v^{1/m}$
  where $v$ is an ordinary word and $m$ is a $\pi$-number. Moreover,
  since $w$ is guaranteed to be satisfied in any $\pi$-powered abelian
  group, so is $v$. Thus, $v$ is satisfied in the vector space over
  the rationals generated by the $n$ letters. So, $v$ is satisfied in
  the free abelian group generated by the $n$ letters, and therefore
  $v$ is satisfied in any abelian group. Thus, Theorem
  \ref{thm:iterated-commutator-descends-to-inn} applies to the word
  $v$, and we obtain:

  $$\chi_{v,G_2}(\zeta(x_1),\zeta(x_2),\dots,\zeta(x_n)) = \varphi(\chi_{v,G_1}(x_1,x_2,\dots,x_n))$$

  Taking $m^{th}$ roots on both sides, we obtain:

  $$\chi_{w,G_2}(\zeta(x_1),\zeta(x_2),\dots,\zeta(x_n)) = \varphi(\chi_{w,G_1}(x_1,x_2,\dots,x_n))$$

  as desired.
\end{proof}


\subsection{Results for the upper central series}

We begin with a simple-looking result whose proof relies on {\em
  downward} induction with the upper central series. The technique
used in the proof is similar to the technique used in the proof of
Theorem \ref{thm:divisibility-upper-central-series}.

\begin{lemma}\label{lemma:ucs-root-commute}
  Suppose $\pi$ is a set of primes and $G$ is a $\pi$-torsion-free
  nilpotent group. Suppose $H$ is a subgroup of $G$. Then, for any
  natural number $n$, $Z^n(\sqrt[\pi]{H}) = \sqrt[\pi]{Z^n(H)}$.
\end{lemma}

\begin{proof}
  The direction $Z^n(\sqrt[\pi]{H}) \le \sqrt[\pi]{Z^n(H)}$ is
  obvious: note that any element in the group on the left has some
  $\pi$-multiple that is in $H$, and that therefore must also be in
  $Z^n(H)$ by definition. We thus concentrate on proving the opposite
  inclusion.

  Suppose $G$ has nilpotency class $c \ge 2$ (note that if $c = 1$
  there is nothing to prove).

  Consider the $c$-fold left-normed commutator map of the form:

  $$T:(x_1,x_2,\dots,x_c) \mapsto [\dots [[x_1,x_2],x_3],\dots,x_c]$$

  This map is a homomorphism in each coordinate holding the other
  coordinates fixed. Note that this fact is very specific to the class
  being $c$. It fails for higher class. Moreover, the set of values
  for $x_1 \in H$ for which the output is always the identity for the other
  inputs restricted to $H$ is precisely the subgroup $Z^{c-1}(H)$.

  Now, suppose $x \in \sqrt[\pi]{Z^{c-1}(H)}$. If we now consider:

  $$T(x,x_2,\dots,x_c)$$

  where each $x_i$ in in $\sqrt[\pi]{H}$, we see that if we replace
  each input by a suitable power of it, the first input lands inside
  $Z^{c-1}(H)$ and the remaining inputs land inside $H$. Thus, a
  suitable $\pi$-multiple of $T(x,x_2,\dots,x_c)$ is the
  identity. Since $G$ is $\pi$-torsion-free, this forces
  $T(x,x_2,\dots,x_c)$ to be the identity, so we obtain that
  $\sqrt[\pi]{Z^{c-1}(H)} \le Z^{c-1}(\sqrt[\pi]{H})$.

  The result can now be extended further down the upper central
  series. The key trick in executing the extension is to replace the
  original $c$-fold commutator with smaller-fold commutators, but now
  restrict the first input to being within the member one higher. In
  general when inducting down from $Z^i(H)$ to $Z^{i-1}(H)$, we
  consider a left-normed commutator of length $i$, restricting the
  first input to be within $Z^i(H)$ and allowing all other inputs to
  vary freely within $H$ or $\sqrt[\pi]{H}$.  
\end{proof}

We can apply this to the minimal $\pi$-powered group:

\begin{lemma}
  Suppose $G$ is a $\pi$-torsion-free nilpotent group and
  $\hat{G}^\pi$ is a minimal $\pi$-powered group containing $G$. Let
  $K = Z^n(G)$ Then, $\hat{K}^{\pi}$ is canonically isomorphic to
  $Z^n(\hat{G}^{\pi})$.
\end{lemma}

\begin{proof}
  This follows from the preceding lemma (Lemma
  \ref{lemma:ucs-root-commute}) and Theorem
  \ref{thm:pi-powered-envelope}.
\end{proof}

\subsection{Results for the lower central series}

Of the two results stated for the upper central series, only one has
an analogue for the lower central series. The analogue of the first lemma
breaks down, while the second still has a valid analogue.

To see why the analogue of the first lemma breaks down, let $G$ be the
central product of $UT(3,\Z)$ and $\Q$ where we identify the central
$\Z$ in $UT(3,\Z)$ with a $\Z$ subgroup in $\Q$. Let $H$ be the
subgroup $UT(3,\Z)$. Then, if we take $\pi$ as the set of all primes,
we have that $\sqrt[\pi]{H} = G$. Thus, $(\sqrt[\pi]{H})' = G'$, which
is the central $\Z$ of $G$ and also the center of $H$. On the other
hand $\sqrt[\pi]{H'}$ is the full center of $G$, and it is isomorphic
to $\Q$. Clearly, the two are not the same.

On the other hand, the second result, pertaining to the minimal
$\pi$-powered group, still holds.

\begin{lemma}\label{lemma:lcs-pi-envelope}
  Suppose $G$ is a $\pi$-torsion-free nilpotent group and $\hat{G}^\pi$ is
  a minimal $\pi$-powered group containing $G$. Then,
  $\hat{\gamma_i(G)}^\pi$ is canonically isomorphic to
  $\gamma_i(\hat{G}^\pi)$.
\end{lemma}

\begin{proof}
  Since $\hat{G}^\pi$ is $\pi$-powered, Theorem \ref{thm:powering-lcs} tells
  us that $\gamma_i(\hat{G}^\pi)$ is also $\pi$-powered. We already
  know that $\gamma_i(G) \le \gamma_i(\hat{G})$, and thus,
  $\sqrt[\pi]{\gamma_i(G)} \le \gamma_i(\hat{G}^{\pi})$. By Theorem
  \ref{thm:pi-powered-envelope}, the left side becomes
  $\hat{\gamma_i(G)}^\pi$. Thus, we have that:

  $$\hat{\gamma_i(G)}^\pi \le \gamma_i(\hat{G}^{\pi})$$

  The proof for the other direction is also fairly similar and
  proceeds by inducting over the lower central series, starting from
  smaller members upwards. The mechanics of the proof are quite similar
  to that of Theorem \ref{thm:nilpotent-pi-root-subgroup}. We begin by
  looking at $\gamma_c(G)$ as the image of the $c$-fold iterated left
  normed commutator map, and note that we can pull powers in and out
  of the commutators. For brevity, we omit the proof details. %%{\em
                                                              %%TODO:
                                                              %%Fill
                                                              %%this
                                                              %%in
                                                              %%later}.
\end{proof}

%\newpage

\section{Free powered Lie rings and powering functors for Lie rings}\label{sec:free-powered-lie-rings-and-powering-functors}

The final results of this section mirror those of the preceding
section (Section
\ref{sec:free-powered-groups-and-powering-functors}). However, the
proofs are much easier.
\subsection{Construction of the free powered Lie ring}

This section uses basic terminology from universal algebra. For
background on the terminology, see Section
\ref{appsec:univalg-basic}.

For any variety of algebras, we can talk of the free algebra in that
variety on any set. In particular, we can talk of the free
$\pi$-powered Lie ring on a set $S$. We already saw in Section
\ref{sec:lie-ring-powering} that $\pi$-powered Lie rings are the same
as $\Z[\pi^{-1}]$-Lie algebras. Thus, the free $\pi$-powered Lie ring
on $S$ coincides with the free $\Z[\pi^{-1}]$-Lie
algebras. Equivalently, the free $\Z[\pi^{-1}]$-Lie algebra on $S$ is
$\Z[\pi^{-1}] \otimes L$ where $L$ is the free Lie algebra on $S$. We
will denote the free $\pi$-powered Lie algebra on $S$ as Lie algebra
as $F(S,\pi)$ here to keep notation similar to the preceding section.

\subsection{Free nilpotent and free powered nilpotent Lie rings}\label{sec:free-powered-nilpotent-lie}

We begin with the definition.

\begin{definer}[Free $\pi$-powered nilpotent Lie ring]
  Suppose $S$ is a set, $\pi$ is a set of primes, and $c$ is a
  positive integer. The {\em free $\pi$-powered nilpotent Lie ring} of
  class $c$ on $S$ is defined as the quotient Lie ring
  $F(S,\pi)/\gamma_{c+1}(F(S,\pi))$ where $F(S,\pi)$ is the free
  $\pi$-powered Lie ring on $S$. Equivalently, this Lie ring, along
  with the set map to it from $S$, is the initial object in the
  category of Lie rings of nilpotency class at most $c$ with set maps to
  them from $S$.

  The functor sending a set to its free $\pi$-powered nilpotent Lie ring
  of class $c$ is left adjoint to the forgetful functor from
  $\pi$-powered nilpotent Lie rings of class (at most) $c$ to sets.
\end{definer}

\subsection{$\pi$-powered words and word maps}

We can define $\pi$-powered words and $\pi$-powered class $c$ words,
and the corresponding word maps, in a manner analogous to Section
\ref{sec:pi-powered-word-maps}. Due to the biadditivity of the Lie
bracket, we can readily deduce the Lie ring analogues of results that
took us some effort to deduce for groups. Further, we do not need the
assumption of nilpotency.

Any $\pi$-powered word $w(g_1,g_2,\dots,g_n)$ can be written in the
form $\frac{1}{m}v(g_1,g_2,\dots,g_n)$ where $v$ is an ordinary word
(a sum of Lie products) and $m$ is a $\pi$-number. The same conclusion
applies if we start with a $\pi$-powered class $c$ word.

\subsection{Localization and powering functors}\label{sec:localization-and-powering-functors-lie}

For a set $\pi$ of primes, the $\pi$-powering functor is a functor
from the category of Lie rings to the category of $\pi$-powered Lie rings
that is left adjoint to the forgetful functor from the category of
$\pi$-powered Lie rings to the category of Lie rings. More explicitly, for a
Lie ring $L$, the $\pi$-powering of $L$ is a Lie ring $K$ along with a
homomorphism $\varphi:L \to K$ such that for any homomorphism
$\theta:L \to N$ from $L$ to a $\pi$-powered Lie ring $N$, there is a
unique homomorphism $\alpha:K \to N$ such that $\theta = \alpha \circ
\varphi$.

It turns out that the $\pi$-powering functor is the same as the
functor that tensors with $\Z[\pi^{-1}]$ to change the base ring to
$\Z[\pi^{-1}]$. Explicitly, the functor takes the $\Z$-Lie algebra $L$
and returns the $\Z[\pi^{-1}]$-Lie algebra $\Z[\pi^{-1}] \otimes_\Z L$.

For a prime set $\pi$, the $\pi$-{\em localization} functor refers to
the powering functor for the set of primes {\em outside of} $\pi$.

\subsection{Results about isoclinisms for $\pi$-powered Lie rings}\label{sec:pi-powered-isoclinism-results-lie}

We now state and prove a $\pi$-powered analogue of Theorem
\ref{thm:iterated-bracket-descends-to-inn}. Note that unlike the
situation for groups, we do not restrict ourselves to the nilpotent
case, because the proofs are straightforward without the assumption of
nilpotency. However, if we wish, we {\em can} formulate the results in
the context of $\pi$-powered class $c$ Lie ring words. The proofs will
remain similar.

\begin{theorem}\label{thm:iterated-bracket-descends-to-inn-pi-powered}
  Suppose $\pi$ is a set of primes and $w(g_1,g_2,\dots,g_n)$ is a
  $\pi$-powered word in $n$ letters with the property that $w$
  evaluates to the zero element in {\em any} $\pi$-powered abelian Lie
  ring. Then, for any $\pi$-powered Lie ring $L$, the word map $w:L^n
  \to L$ obtained by evaluating $w$ descends to a map:

  $$\chi_{w,L}: (\operatorname{Inn}(L))^n  \to L'$$
\end{theorem}

\begin{proof}
  As discussed above, we can write $w = (1/m)v$ where $m$ is a
  $\pi$-number and $v$ is a Lie word (a sum of Lie products) and $m$
  is a $\pi$-number. Moreover, since $w$ is guaranteed to be satisfied
  in any $\pi$-powered abelian Lie ring, so is $v$. Thus, $v$ is
  satisfied in the vector space over the rationals generated by the
  $n$ letters. So, $v$ is satisfied in the free abelian Lie ring
  generated by the $n$ letters, and therefore $v$ is satisfied in any
  abelian Lie ring. Thus, Theorem
  \ref{thm:iterated-bracket-descends-to-inn} applies to the word
  $v$, and we obtain that the map descends to a map:

  $$\chi_{v,L}: \operatorname{Inn}(L))^n \to L'$$

  Since $w = (1/m)v$ and $L'$ is $\pi$-powered by Lemma
  \ref{lemma:lie-ring-lcs-divisibility}, we can obtain the map:

  $$\chi_{w,L}: \operatorname{Inn}(L))^n \to L'$$
\end{proof}

The next theorem is related to Theorem \ref{thm:iterated-bracket-commutes-homoclinisms}.

\begin{theorem}\label{thm:iterated-bracket-commutes-homoclinisms-pi-powered}
  Suppose $\pi$ is a set of primes and $(\zeta,\varphi)$ is a
  homoclinism of $\pi$-powered Lie rings $L_1$ and $L_2$, where
  $\zeta:\operatorname{Inn}(L_1) \to \operatorname{Inn}(L_2)$ and
  $\varphi:L_1' \to L_2'$ are the component homomorphisms. Then for
  any $\pi$-powered word $w(g_1,g_2,\dots,g_n)$ that is trivial
  in every $\pi$-powered abelian Lie ring (as described above), we have:

  $$\chi_{w,L_2}(\zeta(x_1),\zeta(x_2),\dots,\zeta(x_n)) = \varphi(\chi_{w,L_1}(x_1,x_2,\dots,x_n))$$

  for all $x_1,x_2,\dots,x_n \in \operatorname{Inn}(G)$.

  Any word $w$ that is an iterated commutator (with any order of
  bracketing) satisfies this condition, and the theorem applies to
  such word maps.
\end{theorem}

\begin{proof}
  We can write $w = (1/m)v$ where $v$ is an ordinary word and $m$ is a
  $\pi$-number. Moreover, since $w$ is guaranteed to be satisfied in
  any $\pi$-powered abelian Lie ring, so is $v$. Thus, $v$ is
  satisfied in the vector space over the rationals generated by the
  $n$ letters. So, $v$ is satisfied in the free abelian Lie ring
  generated by the $n$ letters, and therefore $v$ is satisfied in any
  abelian Lie ring. Thus, Theorem
  \ref{thm:iterated-bracket-descends-to-inn} applies to the word
  $v$, and we obtain:

  $$\chi_{v,L_2}(\zeta(x_1),\zeta(x_2),\dots,\zeta(x_n)) = \varphi(\chi_{v,L_1}(x_1,x_2,\dots,x_n))$$

  Dividing both sides by $m$, we obtain:

  $$\chi_{w,L_2}(\zeta(x_1),\zeta(x_2),\dots,\zeta(x_n)) = \varphi(\chi_{w,L_1}(x_1,x_2,\dots,x_n))$$

  as desired.
\end{proof}
