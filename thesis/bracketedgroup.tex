\documentclass[10pt]{amsart}

%Packages in use
\usepackage{fullpage, hyperref, vipul}

%Title details
\title{Bracketed groups}
\author{Vipul Naik}

%List of new commands

\makeindex

\begin{document}
\maketitle
%\tableofcontents

\begin{abstract}

\end{abstract}

\section{Definition of bracket structure}

For this document, we follow the right-action convention. Under this
convention, $x^y := y^{-1}xy$ and $[x,y] = x^{-1}y^{-1}xy$.

Let $G$ be a group with identity element $e$. We define a {\em bracket
structure} on $G$ as a binary operation $\{ , \}:G \times G \to G$
satisfying the following conditions:

\begin{enumerate}
\item {\bf Identity as nil}: $\{ g,e \} = e$.
\item {\bf Alternating property}: $\{ g,g \} = e$ and $\{ g,h \} = \{
  h,g \}^{-1}$ for all $g,h \in G$.
\item {\bf Conjugation-invariance}: $\{g,h \}^k = \{ g^k,h^k \}$.
\item {\bf Weak form of linearity}: $\{g,hk \} = \{ g,k \} \{ g,h\}^k$
  and $\{ gh,k \} = \{ g,k \}^h \{ h,k \}$.
\item The {\bf Hall-Witt-Jacobi identities}:
 
  \begin{eqnarray*}
    \{ \{g,h^{-1}\},k\}^h \{ \{h,k^{-1} \},g \}^k \{ \{k,g^{-1} \}, h \}^g & = & e\\
    \{ \{g,h \}, k^g \} \{ \{ h,k \}, g^h \} \{k,g \}, h^k \} & = & e
  \end{eqnarray*}
\end{enumerate}

We will use the term {\em bracketed group} for a group with a bracket structure.

\subsection{Quick observations}

\begin{enumerate}
\item The bracket of any two elements that generate a cyclic subgroup
  is the identity element.
\item A bracket structure on an abelian group is precisely the same
  thing as a Lie bracket, and the corresponding bracketed group is the
  same thing as a Lie ring.
\item The trivial bracket structure is a bracket structure on any
  group.
\item The commutator is a bracket structure on any group, and is a
  nontrivial bracket structure when the group is non-abelian. We shall
  call this the {\em inner bracket structure}.
\end{enumerate}

\subsection{Bracket-invariant subgroups}

For a bracketed group, it may be of interest to study all the {\em
bracket-invariant subgroups}. Note that every subgroup need not be
bracket-invariant.

\subsection{Ideals in a bracketed group}

For a bracketed group $G$, a {\em weak ideal} is a subgroup $I$ of $G$
with the property that, for any $x \in I$ and $y \in G$, $\{ x,y \} \in
I$. Note that because of the alternating property, $\{ x,y \} \in I \iff \{ y
, x \} \in I$, so there is no distinction between the notions of left and
right ideal.

An {\em ideal} is a weak ideal that is also a normal subgroup of $G$.

For the inner bracket structure on a group as well as for all Lie
rings, every weak ideal is an ideal. More generally, define a
bracket-structure to be {\em conjugacy-generating} if, for all $g,h
\in G$, we have $g^h \in \langle g, \{ g,h \} \rangle$. The inner
bracket structure as well as Lie brackets are conjugacy-generating,
and for any conjugacy-generating bracket structure, the weak ideals
are all ideals.

Given an ideal $I$ of $G$, we obtain an induced bracket structure on
$G/I$. Note that in order to check that this is well-defined, we use
the weak form of linearity. The various identities follow from the fact
that it is a quotient structure.

Note that any weak ideal, and hence any ideal, is bracket-invariant,
so we have both $I$ and $G/I$ as bracketed groups.
\subsection{Automorphism-invariant bracket structures}

We say that a bracket structure is {\em automorphism-invariant} if it
commutes with all the automorphisms of the group. The inner bracket
structure is automorphism-invariant, but most other bracket
structures, including Lie brackets, are not.

\subsection{Bracket automorphism group}

The bracket automorphism group of a bracketed group is the subgroup of
the group comprising the automorphisms that preserve the bracket
structure. It contains the inner automorphism group.

\subsection{Perp structure induced by a bracket structure}

We can define a reflexive symmetric relation on $G$ given by $g \perp
h \iff \{ g,h \} = e$. The relation is symmetric because of the
alternating property. 

This relation gives a Galois correspondence on $G$ where all closed
sets are subgroups. In other words, for every subset $A$ of $G$,
$A^\perp$ is a subgroup. This follows from the weak form of
linearity. I will use the term {\em perp-structure} for a reflexive
symmetric relation on a group such that the set of elements related to
any subset is a subgroup. Thus, every bracket structure gives a
perp-structure.

With the inner bracket structure, we get the usual notion of
centralizer, and the subgroups that arise in this way are subgroups
that are centralizers of other subsets. Such subgroups have been
called {\em c-closed subgroups}. Similarly, with a Lie ring, these are
just the centralizers of subsets in the Lie ring sense. In keeping
with the terminology of these two examples, I will use the term {\em
bracket-centralizer} for $A^\perp$ where $\perp$ is the perp-structure
arising from a bracket.

Via the Hall-Witt-Jacobi identities, we see that the
bracket-centralizer of an ideal is an ideal.

\subsection{The bracket of two ideals}

Given two ideals $I$ and $J$ of $G$, we can define $\{ I, J \}$ as the
subgroup generated by $\{ i,j \}$ where $i \in I, j \in J$. It turns
out, from the weak form of linearity and the Hall-Witt-Jacobi
identities, that this bracket is again an ideal.

\subsection{Everything that's common to groups and Lie rings works for bracketed groups}

For a bracketed group, we can define the {\em bracket-derived series}
as the series where each member is the bracket of the previous member
with itself. By induction, each member is in fact an ideal in the
preceding member and hence we can define the bracket structure on
it. Also, all members of this series are {\em bracket
automorphism-invariant}, i.e., they are invariant under the
automorphisms that preserve the bracket structure.

In a similar manner, we can define the {\em bracket-lower central
series}. We can also define a {\em bracket-upper central series}: the
bracket-center is the set of elements whose bracket with everything is
trivial. The bracket-center is clearly an ideal, hence we can take the
quotient group, and take the bracket-center of that, and so on.

We can also define a notion of a {\em bracket-central series}: a
series of ideals where each successive quotient is in the
bracket-center of the whole quotient.

Thus, we can define a notion of {\em bracket-nilpotent group} and {\em
bracket-solvable group}, talk of the {\em bracket-nilpotency class}
and the {\em bracket-derived length}, and so on.

My guess is that every statement that is clearly true both for groups
and for Lie rings has a counterpart that is true for the same reasons
for arbitrary bracketed groups.

\subsection{Derivations of a bracket structure}

I'm seeking a good definition of derivation that uses the weak form of
linearity and/or the Hall-Witt-Jacobi identities.

\section{Actions of groups and weak morphisms}

\subsection{Action of a group on a bracketed group}
Suppose we have a bracketed group $A$ and a group $G$. An action of
$G$ on $A$ is a homomorphism from $G$ to the bracket automorphism
group of $A$. Note that any group has a natural action on any bracket
structure of that group: the action by conjugation.

\subsection{Various notions of morphism}

Given two bracketed groups $A$ and $B$, the strong notion of morphism
$f:A \to B$ would be a group homomorphism from $A$ to $B$ that also
preserves the bracket structure. This would be the standard universal
algebra definition of morphism, whereby we can define notions of
isomorphism, etc. We can also prove an analogue to the fundamental
theorem of group homomorphisms and (probably) analogues to the other
isomorphism theorems.

However, we would often like weaker notions of morphism that do not
care about the precise bracket structure, or even about the precise
multiplication, but the relations that these generate.

\printindex

\end{document}
