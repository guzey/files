\documentclass[10pt]{amsart}

%Packages in use
\usepackage{fullpage, hyperref, vipul, amssymb}

%Title details
\title{Extending the Baer correspondence to 2-groups}
\author{Vipul Naik}

%List of new commands
\newcommand{\Skew}{\operatorname{Skew}}
\newcommand{\Aut}{\operatorname{Aut}}
\newcommand{\Ext}{\operatorname{Ext}}
\makeindex
\begin{document}
\maketitle

In this document, we outline a basic theoretical framework that allows
us to generalize a particular case of the Lazard correspondence -- the
{\em Baer correspondence}, to some situations beyond the original
formulation of the correspondence. To keep the length of the document
reasonable, proofs are omitted though outlines are provided for all
new results. All the proofs are either standard results or have been
written up in detail elsewhere (or both).

\section{Class two groups: 2-cocycles with trivial group action}

\subsection{Definitions of cocycle, coboundary, and cohomology class}

The following setup forms the basis of much of the discussion in this
document. Consider abelian groups $A$ and $G$. Although both $A$ and
$G$ are abelian, we will use additive notation for $A$ (denoting its
operation by $+$, inverse by $-$, and identity element by $0$) and
multiplicative notation for $G$ (denoting its multiplication by
concatenation, inverse by ${}^{-1}$, and identity element by $1$), since
some of the ideas we discuss here will generalize to the case of
non-abelian $G$.

Denote by $Z^2(G,A)$ the group of {\em $2$-cocycles} for the trivial
action of $G$ on $A$. Denote by $B^2(G,A)$ the group of {\em
$2$-coboundaries} for the trivial action of $G$ on $A$. In other
words, $Z^2(G,A)$ is the group of functions $f:G \times G \to A$
satisfying:

$$f(g_2,g_3) - f(g_1g_2,g_3) + f(g_1,g_2g_3) - f(g_1,g_2) = 0 \ \forall \ g_1,g_2,g_3 \in G$$

$B^2(G,A)$ is the subgroup comprising those $f$ for which there exists
a function $\varphi:G \to A$ such that:

$$f(g_1,g_2) = \varphi(g_1g_2) - \varphi(g_1) - \varphi(g_2) \ \forall \ g_1, g_2 \in G$$

The quotient group $Z^2(G,A)/B^2(G,A)$ is defined as the {\em
$2$-cohomology group} (or second cohomology group) $H^2(G,A)$ for the
trivial group action of $G$ on $A$. The elements of $H^2(G,A)$ are
thus cosets of $B^2(G,A)$ in $Z^2(G,A)$ and these elements are termed
{\em 2-cohomology classes}.

\subsection{Relationship with extensions: standard facts}

Here are some standard facts that we will use repeatedly. Note that
while some of these facts have analogues when $G$ is non-abelian
and/or the action is nontrivial, others do not:

\begin{enumerate}
\item Every element of $Z^2(G,A)$ can be used to define a group $E$
  with $A$ embedded as a central subgroup of $E$ and a specified
  isomorphism $E/A \to G$. As a set, we take $E = A \times G$, and the
  multiplication is defined as:

  $$(a_1,g_1)(a_2,g_2) = (a_1 + a_2 + f(g_1,g_2), g_1g_2)$$

  The corresponding embedding of $A$ in $E$ is given by:

  $$a \mapsto (a - f(1,1), 1)$$

  where $1$ denotes the identity element of $G$. The corresponding
  projection map from $E$ to $G$ is:

  $$(a,g) \mapsto g$$

  In other words, every element of $Z^2(G,A)$ gives rise to a central
  extension with base $A$ and quotient group $G$.
\item Given an extension, we can obtain a $2$-cocycle that yields that
  extension by choosing a section, i.e., a set theoretic map $s:G \to
  E$ that is a one-sided inverse to the projection map from $E$ to
  $G$. For a section $s$, the corresponding $2$-cocycle $f$ is defined
  by:

  $$f(g_1,g_2) = s(g_1)s(g_2)\{s(g_1g_2)\}^{-1}$$

\item Two elements of $Z^2(G,A)$ give rise to congruent extensions if
  and only if they are in the same coset of $B^2(G,A)$ in
  $Z^2(G,A)$. This is roughly because shifting by a $2$-coboundary
  roughly corresponds with choosing a different set theoretic section
  of the extension. Specifically, shifting a section by a function
  $\varphi:G \to A$ corresponds to shifting the corresponding
  $2$-cocycle by the boundary of $\varphi$.

  Thus, the set of congruence classes of extensions is identified with
  the $2$-cohomology group $H^2(G,A)$. Note that for all these
  extensions, the big group is a {\em group of nilpotency class at
  most two}. We will often use the term {\em nilpotency class two} to
  mean nilpotency class at most two.

\item For every extension, we can choose a {\em normalized
  $2$-cocycle} representing it. A normalized $2$-cocycle is a
  $2$-cocycle with the property that $f(1,1) = 0$. Note that $f(1,1) =
  0$ forces $f(g,1) = f(1,g) = 0$ for all $g \in G$. Specifically, if
  the section $s:G \to E$ that we pick chooses the identity element of
  $E$ as the image of the identity element of $G$, we get a normalized
  $2$-cocycle.

  In terms of the group of $2$-cocycles, this says that the normalized
  $2$-cocycles span the $2$-cocycles modulo the $2$-coboundaries. In
  other words, every $2$-cocycle is the sum of a normalized
  $2$-cocycle and a $2$-coboundary. For most practical purposes, we
  can thus restrict attention to normalized $2$-cocycles. With
  normalized $2$-coycles, the embedding of $A$ in $E$ becomes more
  straightforward:

  $$a \mapsto (a,1)$$

\item Both $\Aut(G)$ and $\Aut(A)$ act via composition on the group
  $Z^2(G,A)$ and the action takes the subgroup $B^2(G,A)$ to itself --
  hence there is an action on the quotient group
  $H^2(G,A)$.\footnote{Note: This uses the fact that the action of $G$
  on $A$ is trivial. If the action were nontrivial, we would still
  have some automorphisms acting -- however, these automorphisms would
  have to lie in suitable centralizers.} The actions commute (because
  composition is associative, and the two actions by composition are
  on opposite sides) so we get an action of $\Aut(G) \times \Aut(A)$
  on $H^2(G,A)$. Any two elements of $H^2(G,A)$ in the same orbit
  under this action correspond to extensions that are essentially the
  same up to relabeling of $G$ and $A$ -- in particular, the
  isomorphism class of the big group $E$ is the same. The quotient
  space (or the set or orbits), however, does not have a group
  structure for obvious reasons.

\item For a function $f:G \times G \to A$, define $\Skew f$ as
  $(g_1,g_2) \mapsto f(g_1,g_2) - f(g_2,g_1)$. Call a $2$-cocycle $f$
  {\em symmetric} if $\Skew f = 0$. We then have the following:

  \begin{enumerate}
  \item $\Skew$ is a linear map on the set of all functions $f:G
    \times G \to A$ under pointwise addition. Further, it maps
    $2$-cocycles to $2$-cocycles (it does a lot more, as we shall see
    later).
  \item Every $2$-coboundary is a symmetric $2$-cocycle. Thus, the
    property of being a symmetric $2$-cocycle is well defined up to
    cohomology.
  \item The group of symmetric $2$-cocycles forms a subgroup of
    $Z^2(G,A)$ (that we denote by $Z^2_{sym}(G,A)$), and its image in
    $H^2(G,A)$ forms a subgroup of $H^2(G,A)$. This subgroup
    corresponds to those extensions where the big group $E$ is also an
    abelian group. We denote this subgroup by $H^2_{Sym}(G,A)$. Hence,
    this subgroup is isomorphic to $\Ext^1(G,A)$.
  \item The subgroup $H^2_{sym}(G,A)$ is invariant (not necessarily
    pointwise) under the action of $\Aut(G) \times \Aut(A)$.
  \item $\Skew f$ represents the commutator map in the big group $E$
    as follows. For $x,y \in E$, denote by $\overline{x},
    \overline{y}$ the corresponding elements in $G$. Then, $[x,y] =
    f(\overline{x},\overline{y})$ where the element of $A$ in the
    right side is interpreted naturally as an element of $E$ via the
    embedding of $A$ in $E$.
  \item In particular, each coset of $H^2_{sym}(G,A)$ in $H^2(G,A)$
    corresponds to all the congruence classes of extensions that give
    rise to one particular commutator map.
  \item The image of $\Skew$ is in the group of alternating biadditive
    maps from $G$ to $A$. This can be seen both algebraically by
    manipulating equations involving $2$-cocycles and in terms of the
    known properties of the commutator map for groups of class
    two. More on this later.
  \end{enumerate}
\end{enumerate}

\section{Twisting into abelianness}

\subsection{The idea of twisting}
Given a group $E$ constructed as one of these extensions, we can twist
or shift $E$ by a $2$-cocycle $f$ by defining a new multiplication
$*$:

$$x * y := f(\overline{x},\overline{y})xy$$

Here, $f(\overline{x},\overline{y})$ is the element in the central
subgroup $A$ that we obtain by applying the $2$-cocycle $f$ to the
images of $x$ and $y$ in the quotient group $G$. The new group
obtained can naturally be viewed as a central extension with base $A$
and quotient group $G$, corresponding to the $2$-cocycle $c + f$. In
other words, the original $2$-cocycle has been modified by $f$.

Shifting by $-f$, or untwisting by $f$, would mean using the formula:

$$x * y := \frac{xy}{f(\overline{x},\overline{y})}$$


\subsection{Specific kinds of $2$-cocycles and twistings}

\begin{enumerate}
\item (Recall): A $2$-cocycle $f$ is a {\em normalized $2$-cocycle} if
  $f(1,1) = 0$. Note that this forces $f(g,1) = f(1,g) = 0$ for all $g
  \in G$.

  Every cohomology class contains at least one normalized
  $2$-cocycle. Thus, we will always deal with normalized $2$-cocycles
  unless otherwise specified.

  Twisting a given group extension by a normalized 2-cocycle preserves
  the identity element.

\item A $2$-cocycle is an {\em IIP 2-cocycle} if it is normalized and,
  whenever the two inputs are inverses of each other, so is the
  output. IIP 2-cocycles form a subgroup of $Z^2(G,A)$ that we denote
  by $Z^2_{IIP}(G,A)$.

  If there are no elements of order $2$, then every cohomology class
  contains an IIP 2-cocycle. However, in the presence of elements of
  order $2$ in $G$, it is possible to have $2$-cohomology classes
  without any representative IIP 2-cocycle. We define $H^2_{IIP}(G,A)$
  as the subgroup of $H^2(G,A)$ comprising those $2$-cohomology
  classes that can be represented by an IIP $2$-cocycle. As noted
  here, if $G$ has no elements of order $2$, then $H^2_{IIP}(G,A) =
  H^2(G,A)$.

  Twisting a given group extension by an IIP 2-cocycle preserves the
  identity element {\em and} preserves inverses.
\item If $f_1$ and $f_2$ are two $2$-cocycles in the same cohomology
  class, then twisting a given group extention $E$ by $f_1$ gives an
  extension equivalent to the extension obtained by twisting the same
  group extension $E$ by $f_2$.

\item A $2$-cocycle is a {\em cyclicity-preserving $2$-cocycle} if it
  is normalized and, whenever both inputs lie inside a common cyclic
  subgroup, it takes the value $0$. Cyclicity-preserving $2$-cocycles
  form a subgroup of $Z^2(G,A)$ that we denote by
  $Z^2_{CP}(G,A)$. $Z^2_{CP}(G,A)$ is a subgroup of $Z^2_{IIP}(G,A)$.
   
  Many cohomology classes do not contain cyclicity-preserving
  $2$-cocycles, so having a representative that is a
  cyclicity-preserving $2$-cocycle puts a nontrivial constraint on a
  $2$-cohomology class. We define $H^2_{CP}(G,A)$ as the subgroup of
  $H^2(G,A)$ comprising those $2$-cohomology classes that can be
  represented by a cyclicity-preserving $2$-cocycle.

  Twisting a given group extension by a cyclicity-preserving
  $2$-cocycle preserves the cyclic subgroup structure, i.e., it is a
  $1$-isomorphism. Here, we define a {\em 1-isomorphism} as a
  bijection between groups whose restriction to any cyclic subgroup on
  either side is an isomorphism.

  In particular, it preserves the directed power graph, undirected
  power graph, order statistics, and other such information about the
  group.
\end{enumerate}

\subsection{What we want to do}

We are given a group $E$ with a central subgroup $A$ and abelian
quotient group $G$. (In most cases, $A$ is taken as precisely the
center of $E$; however, it may not remain the precise center after
twisting, hence we do not include this as a condition in the general
discussion). Let $c$ be a $2$-cocycle describing the extension $E$. We
want to write:

$$c = f + s$$

where $f$ is a cyclicity-preserving $2$-cocycle and $s$ is a symmetric
$2$-cocycle. If we have such a decomposition, then twisting $E$ by
$-f$ (or the untwisting of $E$ by $f$) gives us an abelian group
described by the $2$-cocycle $s$.

Note that ambiguity in $c$ (up to $2$-coboundary) can be absorbed in
the $s$ part because $2$-coboundaries are symmetric $2$-cocycles.

The group of $2$-cocycles that we can write in this form is the sum
$Z^2_{CP}(G,A) + Z^2_{sym}(G,A)$ inside $Z^2(G,A)$. The corresponding
group of $2$-cohomology classes covered is $H^2_{CP}(G,A) +
H^2_{sym}(G,A)$.

\subsection{Uniqueness theorem}

The following uniqueness result is important:

\begin{theorem}[Uniqueness theorem]\label{uniqueness}
  Assume $A$ and $G$ are both finite abelian groups with $G$ acting
  trivially on $A$. The following (easily seen to be equivalent) are
  true:
  \begin{enumerate}
  \item $Z^2_{sym}(G,A) \cap Z^2_{CP}(G,A) \subseteq B^2(G,A)$.
  \item $H^2_{sym}(G,A) \cap H^2_{CP}(G,A)$ is trivial.
  \item For any extension group $E$ with base $A$ (central) and
    quotient group $G$, there is at most one abelian group extension
    (up to congruence) that can be obtained from it by twisting using
    a cyclicity-preserving $2$-cocycle. All the cyclicity-preserving
    $2$-cocycles that can be used are in the same cohomology class.
  \end{enumerate}
\end{theorem}

The proof is related to the idea that two finite abelian group that
are $1$-ismorphic are in fact isomorphic.

The key idea behind the proof is to exploit the fact that cocycle
groups are groups: namely, we can translate stuff to the identity
element of a group. Proving the specific statement {\em at} the
identity element (i.e., for the split extension) turns out to be a lot
easier than proving it elsewhere. However, because of the nature of
groups, the proof applies everywhere.

\begin{proof}
  (1) can be reinterpreted as follows: Let $E$ be the extension $A
  \times G$. Let $f$ be a symmetric cyclicity-preserving
  $2$-cocycle. Let $F$ be the extension obtained by twisting $E$ by
  $f$. Then, $F$ is congruent to $E$.

  We prove this as follows. First, note that since $f$ is
  cyclicity-preserving, it preserves orders of elements. If we denote
  the multiplication in $F$ by $*$ and the multiplication in $E$ by
  $+$, then:

  $$x * y := x + y + f(x,y)$$

  Since $f$ is cyclicity-preserving, $*$ and $+$ agree on cyclic
  subgroups and in particular the cyclic subgroups are the same.

  Moreover, the identification of $E$ with $F$ induces identity maps
  on the corresponding subgroups $A$ and the corresponding quotients
  $G$. Since $E$ has the property that for every $g \in G$, there
  exists $x \in E$ mapping to $G$ and having the same order as $g$,
  $F$ also has this property.

  Since $f$ is symmetric, $F$ is abelian. $F$ is thus an abelian
  extension with base $A$, quotient $G$, and the property that for
  every $g \in G$, there exists $x \in E$ mapping to $G$ and having
  the same order as $g$.

  Suppose now that we write $G$ as an internal direct product of
  cyclic subgroups $T_1, T_2, \dots, T_n$ generated by elements $t_1,
  t_2, \dots, t_n$ respectively. For each $t_i$, let $u_i$ be an
  element of $F$ in its coset and having the same order. Let $U_i$ be
  the subgroup generated by $u_i$. Then, each $U_i$ is isomorphic to
  the corresponding $T_i$. Moreover, $U_1 \times U_2 \times \dots
  \times U_n$ (in $F$) is a complement to $A$ in $F$ with the natural
  projection to $G$ being an isomorphism, and hence $F$ is congruent
  to $A \times G$. (Note: I've skipped some details here to avoid
  messiness, but have worked them out elsewhere).
\end{proof}

\subsection{Stating what we need to do}

We are confronted with the question: given a $2$-coycle $c$, how do we
go about finding a nice (i.e., cyclicity-preserving) $2$-cocycle $f$
such that $c - f$ is symmetric? Note that the symmetric $2$-cocycles
are the kernel of the skew map, so this is equivalent to requiring
that $\Skew c = \Skew f$, or equivalently, that $\Skew f$ is the
commutator map. In other words, we need to find a cyclicity-preserving
$2$-cocycle $f$ whose skew is precisely the commutator, so that when
we untwist by $f$, we land in the abelian case.

Let $\lambda = \Skew c$ be the commutator map.

\begin{lemma}
  $\lambda$ is alternating and biadditive. In particular, it is skew
  symmetric.  
\end{lemma}

This follows from the known properties of the commutator map for class
two groups. However, it can also be deduced without reference to
groups at all, using algebraic manipulations of $2$-cocycles. In other
words, the following is true and can be proved purely by cocycle
equation manipulation:

\begin{lemma}
  Let $c$ be a $2$-cocycle for a trivial group action of an abelian
  group on an abelian group. Then, $\Skew c$ is alternating and
  biadditive.
\end{lemma}

Either way, $\lambda$ is alternating and biadditive. We also note the
following easy to prove but perhaps not very well known fact:

\begin{lemma}
  Any biadditive function is a $2$-cocycle. More generally, any
  function $G^n \to A$ that is additive in each coordinate is a
  $n$-cocycle.
\end{lemma}

\section{Solving skew equations}

\subsection{Alternating biadditive maps}

\begin{lemma}\label{alternatingbiadditivesetup}
  \begin{enumerate}
  \item The group of alternating biadditive maps from $G$ to $A$ is
    isomorphic to $\operatorname{Hom}(\bigwedge^2G,A)$.
  \item The $\Skew$ map establishes an isomorphism between
    $H^2(G,A)/H^2_{sym}(G,A)$ and the subgroup of
    $\operatorname{Hom}(\bigwedge^2G,A)$ identified with the image of
    the $\Skew$ map on $H^2(G,A)$.
  \item Every alternating biadditive map is the skew of some
    $2$-coycle if and only if the above map establishes an isomorphism:

    $$\operatorname{Hom}(\bigwedge^2G, A) \cong H^2(G,A)/H^2_{sym}(G,A)$$
  \item If every alternating biaddiive map arises as the skew of a
    cyclicity-preserving $2$-cocycle, we have

    $$H^2_{CP}(G,A) \cong \operatorname{Hom}(\bigwedge^2G, A) \cong H^2(G,A)/H^2_{sym}(G,A)$$

    We thus have the internal direct sum:

    $$H^2(G,A) = H^2_{sym}(G,A) + H^2_{CP}(G,A)$$

    and thus we have:

    $$H^2(G,A) \cong H^2_{sym}(G,A) \oplus \operatorname{Hom}(\bigwedge^2G,A)$$
  \end{enumerate}
\end{lemma}

In the next two subsections, we show that for elementary abelian
$2$-groups $G$, the hypotheses of conditions (3) and (4) hold (with
(4) requiring some additional assumptions), and hence so do the
conclusions.

\subsection{Alternating biadditive maps as skew for elementary abelian $2$-groups}

{\em This subsection is new}. We prove the following:

\begin{lemma}\label{alternatingbiadditiveisskewinelab}
  Suppose $G$ is an elementary abelian $2$-group. Then, for any
  alternating biadditive map $\lambda:G \times G \to A$, there exists
  a $2$-cocycle $f:G \times G \to A$ such that $\Skew f = \lambda$.
\end{lemma}

\begin{proof}
  We sketch the main steps of the proof:

  \begin{enumerate}
  \item An alternating biadditive map arises from a homomorphism
    $\bigwedge^2G \to A$. In particular, the image of $G \times G$
    under the alternating biadditive map under $A$ lies inside the
    image of $\bigwedge^2G$ under a homomorphism. Since the latter
    group is also elementary abelian, the image of $\lambda$ is in the
    $2$-torsion (which can be denoted $\Omega_1(A)$ for $A$ a
    $2$-group). We will aim to find $f$ which also takes values in
    $\Omega_1(A)$. Thus, it suffices to consider the case of $A$
    elementary abelian.
  \item In fact, it suffices to consider the case $A$ cyclic of order
    $2$: For $A$ elementary abelian, we can write $A$ as a direct
    product of cyclic groups of order $2$. We can now find $f$
    coordinate-wise. Since the condition on being a $2$-cocycle, the
    condition of being alternating and biadditive, as well as the skew
    equation $\Skew f = \lambda$ remain preserved upon decomposing and
    reconstituting coordinate-wise, it suffices to solve the problem
    in just one coordinate, i.e., when $A$ is cyclic of order $2$.
  \item By the previous two steps, we are reduced to the case where
    $A$ is the additive group of the field of two elements and $G$ is
    a vector space over that field. Thus, $\lambda$ is an alternating
    bilinear {\em form} on the vector space $G$ over the field of two
    elements. We can thus choose a symplectic basis for
    $\lambda$. (Here, symplectic basis includes degenerate vectors and
    pairs of vectors of hyperbolic planes).
  \item We can thus decompose $G$ as a direct product of
    subspaces/subgroups $G_1, G_2, \dots, G_n$ that are pairwise
    orthogonal with respect to $\lambda$. It suffices to find $f_i$ on
    each subspace such that $\Skew f_i$ is the restriction of
    $\lambda$ to $G_i$. We then define $f$ by:

    $$f(g_1 \oplus g_2 \oplus \dots \oplus g_n, h_1 \oplus h_2 \oplus \dots \oplus h_n) = \sum_{i=1}^n f_i(g_i,h_i)$$

    In particular, $f$ takes the value $0$ whenever its inputs are
    from orthogonal subplanes.

    One way of thinking about this is that $f$ is partially biadditive
    {\em between} planes -- we can additively split $f$ across the
    hyperbolic planes -- but it is not necessarily biadditive {\em
    within} each plane. What needs to be proved, but turns out to be
    easy, is that combining cocycles biadditively still gives
    cocycles.

  \item The problem thus reduces to being able to find a $2$-cocycle
    $f$ such that $\Skew f = \lambda$, where $G$ is a Klein four-group
    with basis $v,w$, $A$ is $\Z_2$, and $\lambda$ is defined by
    $\lambda(v,w) = 1$. The requisite $f$ is the following normalized
    $2$-cocycle:

    $$f(v,v) = f(vw,vw) = f(w,w) = f(v,vw) = f(vw,w) = f(w,v) = 1, \qquad f(v,w) = f(w,vw) = f(vw,v) = 0$$

    This is a $2$-cocycle that arises from the quaternion group.
  \end{enumerate}

  The key insight in the proof is that we can, through a bunch of
  clever choices, reduce ourselves to a simple case for one specific
  $G$ and one specific $A$.
\end{proof}

\subsection{More handy results for elementary abelian groups}

When $G$ is elementary abelian, every cyclicity-preserving $2$-cocycle
is skew symmetric and takes values in the $4$-torsion of
$A$. Moreover, for $f$ a cyclicity-preserving $2$-cocycle, if $x$ and
$y$ are distinct, then $f(x,y)$ determines the value of $f$ for all
pairs of elements in $\langle x, y \rangle$, which are either equal to
$f(x,y)$ or equal to $-f(x,y)$. We state this formally:

\begin{lemma}\label{CP2cocycleonelab}
  Suppose $G$ is an elementary abelian $2$-group and $A$ is an abelian
  group with trivial action of $G$ on $A$. Then, the following are
  equivalent for a $2$-cocycle $f$:

  \begin{enumerate}
  \item $f$ is an IIP $2$-cocycle.
  \item $f$ is a cyclicity-preserving $2$-cocycle.
  \item $f$ is a skew-symmetric IIP $2$-cocycle.
  \item $f$ is a skew-symmetric cyclicity-preserving $2$-cocycle.
  \end{enumerate}

  Further, any $f$ satisfying these equivalent conditions also satisfies:

  \begin{enumerate}
  \item $$4f(x,y) = 0 \ \forall \ x,y \in G$$
  \item $$f(x,y) = f(y,xy) = f(xy,x) = -f(y,x) = -f(xy,y) = -f(x,xy) \ \forall \ x,y \in G$$
  \item $2f$ is alternating and biadditive.
  \end{enumerate}
\end{lemma}

An easy consequence of this is the following lemma:

\begin{lemma}
  Suppose $G$ is an elementary abelian $2$-group and $A$ is an abelian
  group with trivial action of $G$ on $A$. Then, the following are
  equivalent for a $2$-cocycle $f$:

  \begin{enumerate}
  \item $f$ is biadditive and IIP.
  \item $f$ is biadditive and cyclicity-preserving.
  \item $f$ is alternating and biadditive.
  \end{enumerate}

  Also, under these equivalent classes, $f$ is a symmetric
  cyclicity-preserving $2$-cocycle and hence (by the uniqueness theorem
  or otherwise) a $2$-coboundary.
\end{lemma}

\subsection{Alternating biadditive as skew of cyclicity-preserving}

We provide a necessary and sufficient condition.

\begin{lemma}\label{alternatingbiadditiveisskewinelab2}
  Suppose $G$ is an elementary abelian $2$-group. Suppose $\lambda$ is
  an alternating biadditive map $\lambda:G \times G \to A$. The following are equivalent:

  \begin{enumerate}
  \item Every element in the image of $G \times G$ under $\lambda$ is
    the double of some element of $A$.
  \item There exists a cyclicity-preserving $2$-cocycle $f: G \times G
    \to A$ such that $2f = \lambda$.
  \item There exists a cyclicity-preserving $2$-cocycle $f: G \times G
    \to A$ such that $\Skew f = \lambda$.
  \end{enumerate}
\end{lemma}

\begin{proof}
  {\em Equivalence of (2) and (3)}: By lemma \ref{CP2cocycleonelab},
  any cyclicity-preserving $2$-cocycle $f$ is skew symmetric. Thus,
  $\Skew f = 2f$, so (2) and (3) are equivalent.

  {\em (3) implies (1)}: This is obvious.

  {\em (1) implies (3)}: This is the hard part. We follow a strategy
  similar to lemma \ref{alternatingbiadditiveisskewinelab}.   

  \begin{enumerate}
  \item An alternating biadditive map arises from a homomorphism
    $\bigwedge^2G \to A$. In particular, the image of $G \times G$
    under the alternating biadditive map under $A$ lies inside the
    image of $\bigwedge^2G$ under a homomorphism. Since the latter
    group is also elementary abelian, the image of $\lambda$ is in the
    $2$-torsion (which can be denoted $\Omega_1(A)$ for $A$ a
    $2$-group). From the given assumptions, we see that the image of
    $\lambda$ must lie inside $\Omega_1(\mho^1(A))$, i.e., the set of
    elements that are doubles of elements and that have order dividing
    $2$. Consider a subgroup $B$ of $A$ which is a direct power of
    $\mathbb{Z}_4$ such that $\mho^1(B) = \Omega_1(\mho^1(A))$. Such a
    $B$ always exists (easy to show). We can thus assume that $A$
    itself is a direct power of $\mathbb{Z}_4$ and the image of
    $\lambda$ is in $\mho^1(A) = \Omega_1(A)$.
  \item In fact, it suffices to consider the case $A$ cyclic of order
    $4$ with the image of $f$ taking values that are multiples of $2$:
    We do this by writing $A$ as a direct product of $\Z_4$s and
    solving the problem coordinate-wise.
   \item The image $\mho^1(A) = 2\Z_4$ which is also the image of
    $\lambda$ is the additive group of the field of two elements and
    $G$ is a vector space over that field. Thus, $\lambda$ is an
    alternating bilinear {\em form} on the vector space $G$ over the
    field of two elements. We can thus choose a symplectic basis for
    $\lambda$. (Here, symplectic basis includes degenerate vectors and
    pairs of vectors of hyperbolic planes).
  \item We can thus decompose $G$ as a direct product of
    subspaces/subgroups $G_1, G_2, \dots, G_n$ that are pairwise
    orthogonal with respect to $\lambda$. It suffices to find $f_i$ on
    each subspace such that $\Skew f_i$ is the restriction of
    $\lambda$ to $G_i$. We then define $f$ by:

    $$f(g_1 \oplus g_2 \oplus \dots \oplus g_n, h_1 \oplus h_2 \oplus \dots \oplus h_n) = \sum_{i=1}^n f_i(g_i,h_i)$$

    In particular, $f$ takes the value $0$ whenever its inputs are
    from orthogonal subplanes.

    Note that unlike $\lambda$, which takes values only in the
    subgroup $2\Z_4$, $f$ is allowed to take values in the whole $\Z_4$ -- this is crucially necessary.
  \item The actual choice of $f$ for $G$ a Klein four-group and $A =
    \Z_4$ is as follows: Consider a hyperbolic plane with basis $v$,
    $w$, such that $\lambda(v,w) = 1$ in the additive group of the
    field. This $1$ becomes a $2$ in $\Z_4$, so $\lambda$ takes the
    value $2$ on any pair of linearly independent vectors from
    $\langle v, w \rangle$.

    $$f(v,w) = f(w,vw) = f(vw,v) = 1, f(w,v) = f(vw,w) = f(v,vw) = -1$$

    Note that this is exactly as per the constraints we obtained in
    lemma \ref{CP2cocycleonelab}.
  \end{enumerate}
\end{proof}

\subsection{Summing up in a theorem}

We combine lemmas \ref{alternatingbiadditivesetup},
\ref{alternatingbiadditiveisskewinelab}, and
\ref{alternatingbiadditiveisskewinelab} to obtain the following summary.

\begin{theorem}\label{elabsummary}
  Suppose $G$ is an elementary abelian $2$-group and $A$ is an abelian
  group. Then, the following are true:

  \begin{enumerate}
  \item The $\Skew$ map gives a canonical isomorphism:

    $$H^2(G,A)/H^2_{sym}(G,A) \cong \operatorname{Hom}(\bigwedge^2G,A)$$
  \item We have (non-canonically?):

    $$H^2(G,A) \cong H^2_{sym}(G,A) \oplus \operatorname{Hom}(\bigwedge^2G,A)$$
  \item If $A$ has the property that every element of order $2$ is the
    double of some element (for a $2$-group $A$, this reads as
    $\Omega_1(A) \subseteq \mho^1(A)$), then the direct sum
    decomposition in step (2) can be made canonical. Specifically, we
    have the internal direct sum:

    $$H^2(G,A) = H^2_{sym}(G,A) + H^2_{CP}(G,A)$$

    with the $\Skew$ map inducing an isomorphism between
    $H^2_{CP}(G,A)$ and $\operatorname{Hom}(\bigwedge^2G,A)$.
  \end{enumerate}
\end{theorem}

\begin{proof}
  {\em Proof of (1)}: Lemma \ref{alternatingbiadditiveisskewinelab}
  shows that the hypothesis of lemma \ref{alternatingbiadditivesetup},
  part (3) is satisfied, hence the conclusion holds.

  {\em Proof of (2)}: This follows from (1) and the observation that
  when $G$ is an elementary abelian $2$-group, so is $H^2(G,A)$ (need
  to insert lemma or citation -- this is not obvious) and hence the
  subgroup $H^2_{sym}(G,A)$ has a complement.

  {\em Proof of (3)}: Lemma \ref{alternatingbiadditiveisskewinelab2}
  shows that the hypothesis of lemma \ref{alternatingbiadditivesetup},
  part (4) is satisfied, hence the conclusion holds.
\end{proof}

\subsection{Going beyond elementary abelian}

In order to generalize the results of lemmas
\ref{alternatingbiadditiveisskewinelab} and
\ref{alternatingbiadditiveisskewinelab2}, we need to first understand
what alternating biadditive maps look like when the source is a
(finite) abelian $2$-group $G$. We'll also assume that $A$ is a
$2$-group. Note that this is not a loss of generality because, if $G$
is a $2$-group, the image of $G \times G$ under $b$ lies inside a
subgroup of $A$ that is a $2$-group. (As later sections shall make
clear, the $2$-group case is the most difficult and interesting case).

Suppose the exponent of $G$ is $2^k$. We claim that we can assume $A$
to be cyclic of order $2^l, l \le k$. For this, we first note that any
element in the image of $G \times G$ under $b$ has exponent at most
$2^k$. Thus, the subgroup of $A$ spanned by this image has exponent at
most $2^k$. This is thus a direct product of cyclic groups of order
$2^l, l \le k$. Since we need to consider only one coordinate at a
time, it suffices to consider $A$ to be cyclic of order $2^l, l \le
k$.

An analogous notion of symplectic basis is valid in this situation. We
decompose $G$ as a sum of subspaces orthogonal with respect to $b$:

$$G_1 \oplus G_2 \oplus \dots \oplus G_n$$

where each $G_i$ is either one-dimensional generated by a degenerate
vector or two-dimensional of the form $\langle v,w \rangle$ where
$b(v,w) = 2^r$ for some $r \le l$. (Is this really true?)

\section{Some results that help beyond elementary abelian}

\begin{lemma}
  Suppose $f$ is an IIP $2$-cocycle $G \times G \to A$, with $A$
  abelian.

  \begin{enumerate}
  \item The {\em affine line identity}: $f(x,y) = f(y,z) = f(z,x)$
    whenever $xyz$ is the identity element of $G$
  \item The {\em reverse shear identity}: $f(x^{-1},xy) = -f(x,y)$
  \item The {\em transpose inverse identity}: $f(y^{-1},x^{-1}) = -f(x,y)$.
  \end{enumerate}
\end{lemma}

Note that for the results as stated above, we do not need $G$ to be
abelian, though that is the context in which we apply the results.

\begin{proof}
  {\em Proof of (1)}: Suppose $xyz = 1$. We use the $2$-cocycle
  identity to get:

  $$f(x,yz) + f(y,z) = f(xy,z) +  f(x,y)$$

  Since $yz = x^{-1}$, we get $f(x,yz) = 0$. Similarly, since $xy
  = z^{-1}$, we get $f(xy,z) = 0$. Thus, we can drop the terms and get:

  $$f(y,z) = f(x,y)$$

  Cyclically permuting this gives the full identity.

  {\em Proof of (2)}: We apply the $2$-cocycle condition on $x^{-1}$,
  $x$, and $y$:

  $$f(x^{-1},xy) + f(x,y) = f(x^{-1},x) + f(x^{-1}x,y)$$

  The identity- and inverse-preserving conditions make the two terms
  on the right side $0$, giving the desired identity.

  {\em Proof of (3)}: From (2), we have $f(x^{-1},xy) = -f(x,y)$. From
  (1), we have $f(y^{-1},x^{-1}) = f(x^{-1},xy)$. Combining these
  gives (3).
\end{proof}

\section{Lie rings of class two}

{\em This is a new section}.

\subsection{Lie rings of class two as extensions}

Suppose $A$ and $G$ are abelian groups. We consider $A$ and $G$ both
as abelian Lie rings, i.e., we set the trivial bracket on both.

We want to study all Lie ring extensions of $A$ by $G$ with the
trivial (adjoint) action: such an extension is a group $L$ with a
central Lie subring isomorphic to $A$ and the quotient $L/A$
isomorphic to $G$. (Specific isomorphisms are part of the structure,
in analogy with group extensions).

Note that such Lie rings all have class two. There are two pieces of
information that are needed to describe such a Lie ring:

\begin{enumerate}
\item The additive group structure on $E$. This is provided by
  specified an element of $H^2_{sym}(G,A)$.
\item The Lie bracket on $E$. This is specified by providing an
  alternating bilinear map from $G$ to $A$, i.e., an element in
  $\operatorname{Hom}(\bigwedge^2G,A)$.
\end{enumerate}

Both these pieces of information are independent. The set of Lie ring
extensions can thus be identified with the direct sum of these two
groups, i.e., it is the direct sum 

$$H^2_{sym}(G,A) \oplus \operatorname{Hom}(\bigwedge^2G,A)$$

\subsection{Interpreting the twist in terms of Lie rings}

In our earlier discussion, we started out with the following: a
central extension $E$ with both pieces $A$ and $G$ abelian, defined by
a $2$-cocycle $c$. We sought to express $c$ as a sum of a
cyclicity-preserving $2$-cocycle $f$ and a symmetric $2$-cocycle
$s$. Untwisting the extension by $f$ gave us an abelian extension with
a $1$-isomorphism to the original group $E$. The uniqueness theorem
(\ref{uniqueness}) guaranteed that even though the cocycle $f$ chosen
is not unique, its cohomology class is unique if it exists.

We now give an interpretation in terms of Lie rings. Let $L$ be the
abelian group we obtain by untwisting $E$. We can then make $L$ into a
Lie ring by defining the Lie bracket on $L$ as the original commutator
operation on $E$. Since the commutator operation was alternating and
biadditive, we do get a legitimate Lie bracket, making $L$ a Lie ring
of class two with central Lie subring $A$ and quotient Lie ring $G$.

We call such a Lie ring $L$ a {\em generalized Baer CP-Lie ring} or
{\em generalized Baer cyclicity-preserving Lie ring} for the extension
$E$ of $A$ by $G$.

\subsection{Generalized Baer IIP-Lie ring}

We introduce another concept that we shall have very rare occasion to
use. This is the concept of a {\em generalized Baer IIP-Lie
ring}. Suppose $c$ is a $2$-cocycle for a central extension with
abelian base $A$ and abelian quotient $G$. We try to write:

$$c = f + s$$

where now, for $f$, we only insist that it be an IIP $2$-cocycle
rather than a cyclicity-preserving $2$-cocycle. We now untwist $E$ by
$f$, getting an abelian group with the same identity element and the
same relationship of being inverse. In particular, the set of elements
of order $2$ remains the same. We call this abelian group $L$ and give
it a Lie ring structure where the Lie bracket coincides with the
original commutator. This $L$ is termed a {\em generalized Baer
IIP-Lie ring} for the extension $E$.

Note that:

\begin{itemize}
\item In the case that there are no elements of order $2$, $H^2(G,A) =
  H^2_{IIP}(G,A)$. Thus, {\em every} abelian group can serve as the
  additive group of some generalized Baer IIP-Lie ring, and this
  concept is nearly useless.
\item In general, there is no analogue of the uniqueness theorem
  (theorem \ref{uniqueness}) for IIP $2$-cocycles: there may well be
  symmetric IIP 2-cocycles. Thus, the generalized Baer IIP-Lie ring
  need not be unique. The exception is when $G$ is an elementary
  abelian $2$-group: in this case, being IIP is equivalent to being
  cyclicity-preserving.
\item In the special case where the extension group $E$ as well as its
  generalized Baer IIP-Lie ring $L$ have exponent dividing $4$, $E$
  and $L$ have the same order statistics (i.e., the same number of
  elements of each specific order). In fact, the bijection between $E$
  and $L$ preserves orders of elements. This is because the elements
  of order $2$ are preserved, the identity is preserved, and all the
  other elements on both sides are of order $4$.
\end{itemize}

\subsection{What's really going on}

We now provide a restatement/reinterpretation of parts (3) and (4) of
lemma \ref{alternatingbiadditivesetup} in terms of Lie rings.

\begin{lemma}\label{Lieinterpretationalternatingbiadditivesetup}
  Fix $G$ and $A$ the usual way.

  \begin{enumerate}
  \item If every alternating biadditive map $G \times G \to A$ arises
    as the skew of some $2$-cocycle, then the $\Skew$ map establishes
    an isomorphism $H^2(G,A)/H^2_{sym}(G,A)$ to
    $\operatorname{Hom}(\bigwedge^2G,A)$. This in particular implies
    that the group $H^2(G,A)$ ({\em which classifies central group
    extensions}) has the same order as the group $H^2_{sym}(G,A)
    \oplus \operatorname{Hom}(\bigwedge^2G,A)$ ({\em which classifies
    central Lie ring extensions}). Thus, there are as many group
    extensions as there are Lie ring extensions.
  \item If every alternating biadditive map $G \times G \to A$ arises
    as the skew of some $2$-cocycle {\em and} $H^2_{sym}(G,A)$ has a
    complement in $H^2(G,A)$ (which happens, for instance, if
    $H^2(G,A)$ is elementary abelian), then we have the following
    (non-canonical?)  isomorphism:

    $$H^2(G,A) \cong H^2_{sym}(G,A) \oplus \operatorname{Hom}(\bigwedge^2G,A)$$

    Thus the group of central group extensions is (non-canonically)
    isomorphic to the group of central Lie ring extensions.
  \item If every alternating biadditive map $G \times G \to A$ arises
    as the skew of some cyclicity-preserving $2$-cocycle, then we have
    an internal direct sum:

    $$H^2(G,A) \cong H^2_{sym}(G,A) \oplus H^2_{CP}(G,A)$$

    and under the $\Skew$ map, $H^2_{CP}(G,A) \cong
    \operatorname{Hom}(\bigwedge^2G,A)$. Thus, we have a {\em
    canonical} isomorphism:

    $$H^2(G,A) \cong H^2_{sym}(G,A) \oplus \operatorname{Hom}(\bigwedge^2G,A)$$

    This canonical isomorphism establishes a bijection between the set
    of central group extensions and the set of central Lie ring
    extensions, and that bijection is precisely the recipe we outlined
    for taking the Lie ring of a group.
  \item If every alternating biadditive map $G \times G \to A$ arises
    as the skew of some $2$-cocycle (the same hypothesis as in part
    (1)) then, as noted in part (1), the number of Lie ring extensions
    is equal to the number of group extensions. This equality of
    numbers does not come with a canonical bijection. However, we {\em
    do} get a bijection on the subgroup $H^2_{sym}(G,A)
    +H^2_{CP}(G,A)$ of $H^2(G,A)$.
  \end{enumerate}
\end{lemma}

We now state the corresponding results with $G$ elementary abelian,
drawing on theorem \ref{elabsummary} and the new way of thinking.

\begin{theorem}\label{elabsummary2}
  Suppose $G$ is an elementary abelian $2$-group and $A$ is an abelian
  group. Then, the following are true:
  
  \begin{enumerate}
  \item The group of central group extensions with base $A$ and
    quotient $G$ is (non-canonically) isomorphic to the group of
    central Lie ring extensions with base $A$ and quotient $G$.
  \item If (and only if?) every element in $A$ of order two is a
    multiple of $2$, then the above isomorphism is canonical and the
    bijection on elements is the usual way of going from a group to
    its Lie ring.
  \item In general, the isomorphism is canonical only in part (and the
    Lie ring is defined) for the subgroup $H^2_{sym}(G,A) +
    H^2_{CP}(G,A)$ of $H^2(G,A)$.
  \end{enumerate}
\end{theorem}

\section{Generalized Baer Lie ring}

\subsection{Introduction}

Most of the results here are restatements of previous results from a
more hands-off perspective. The hardwork is over.

We say that a group $E$ has a {\em generalized Baer CP-Lie ring} or
{\em generalized Baer Lie ring} if we can find a central subgroup $A$
of $E$ with abelian quotient $G$ such that $E$ as an extension has a
generalized Baer CP-Lie ring -- in other words, the cohomology class
corresponding to $E$ is in $H^2_{sym}(G,A) + H^2_{CP}(G,A)$.

We say that a group $E$ has a {\em generalized Baer IIP-Lie ring} if
we can find a central subgroup $A$ of $E$ with abelian quotient group
$$ such that $E$ as an extension has a IIP-Lie ring -- in other words,
the cohomology class corresponding to $E$ is in $H^2_{sym}(G,A) +
H^2_{IIP}(G,A)$. Note that unlike generalized Baer CP-Lie rings,
generalized Baer IIP-Lie rings need not be unique, and in the odd
order case, the concept is almost vacuous.

Note that generalized Baer Lie ring by default means generalized Baer
CP-Lie ring.

We now make a conjecture that is probably true, but not entirely obvious:

\begin{conjecture}
  \begin{enumerate}
  \item $E$ has a generalized Baer Lie ring for {\em some} choice of
    $A$ and $G$ if and only if it has a generalized Baer Lie ring for
    the choice $A = Z(E)$ and $G = E/Z(E)$.
  \item The generalized Baer Lie ring is unique up to isomorphism
    (something stronger?) and does not depend on the choice of $A$ and
    $G$.
  \end{enumerate}
\end{conjecture}

The value of this, if true, is that it now allows us to look at groups
{\em qua} groups and ask if they have Lie rings, rather than worry
about the particular extension that we are using to think of the
group.

Interestingly, the corresponding statement is {\em not} true for
generalized Baer IIP-Lie rings. In other words, we can have situations
where a group $E$ has a generalized Baer IIP-Lie ring for a choice of
central subgroup $A$ but {\em not} for the choice $A = Z(E)$.

\subsection{Frattini-in-center groups: condition for generalized Baer Lie rings}

We state the earlier results in this new language:

\begin{theorem}
  Suppose $E$ is a class two $2$-group such that $E/Z(E)$ is
  elementary abelian (in other words, $\Phi(E) \le Z(E)$). Then, $E$
  has a generalized Baer Lie ring if and only if $[E,E] \subseteq
  \mho^1(Z(E))$.
\end{theorem}

The question of whether the hypothesis $E/Z(E)$ elementary abelian can
be dropped (i.e., whether the theorem holds for $E/Z(E)$ any abelian
$2$-group) depends on how far lemma
\ref{alternatingbiadditiveisskewinelab2} can generalize to the
non-elementary abelian case.

Note that it can definitely be extended to the case where $E/Z(E)$ is
the direct product of a cyclic and an elementary abelian group.

\subsection{An embedding theorem}

\begin{theorem}\label{embedding}
  Suppose $P$ is a class two $2$-group such that $P/Z(P)$ is
  elementary abelian. Then, $P$ can be embedded in a class two
  $2$-group $E$ such that $E$ has a generalized Baer Lie ring.
\end{theorem}

\begin{proof}
  We set $E$ as the central product of $E$ and an abelian group that
  has ``halves'' for all the elements in $[P,P]$, over the identified
  subgroup $[P,P]$.

  For instance, when $P$ is $D_8$ (dihedral of degree $4$, order $8$),
  the derived subgroup is cyclic of order $2$, so we take $P * \Z_4$
  over a commonly identified central cyclic subgroup of order $2$.
\end{proof}

Again, the question of whether this generalizes to all class two
$2$-groups hinges on how far lemma
\ref{alternatingbiadditiveisskewinelab2} can generalize to the
non-elementary abelian case.

\section{Baer's solution: the odd-order and uniquely $2$-divisible case}

We now consider Baer's solution to the problem, which is canonical and
particularly nice in the case that the group $A$ is uniquely
$2$-divisible, i.e., every element has a {\em half}.

\subsection{Baer's solution}

Note that the alternating biadditive $\lambda$ itself is
cyclicity-preserving since it is alternating and biadditive. Moreover,
it is skew symmetric, so $\Skew \lambda = 2 \lambda$. This is off by a
factor of $2$: we want $f$ such that $\Skew f = \lambda$. Thus, if we
could halve the $\lambda$, we would have the desired $f$ whereby we
could twist by $-f$.

This is the crux of Baer's idea: if $A$ is $(1/2)$-powered (in the
sense that there is a halving operation that commutes with the group
operations) then we can define $f(x,y) = \frac{1}{2}[x,y]$ in $A$. In
the big group $E$, this halving would be represented as a square root,
and we would get the formula $f(x,y) = \sqrt{[x,y]}$, giving Baer's
formula:

$$x + y := \frac{xy}{\sqrt{[x,y]}}$$

\subsection{Interpreting Baer's approach in our language}

We noticed the following fact but rarely used it: the group of
alternating biadditive maps is a subgroup of $Z^2(G,A)$. In other
words, there is a natural embedding of
$\operatorname{Hom}(\bigwedge^2G,A)$ inside $Z^2(G,A)$, because
alternating biadditive maps are $2$-cocyclesq.

The halving operation in $A$ gives a halving automorphism:

$$\operatorname{Hom}(\bigwedge^2G,A) \to \operatorname{Hom}(\bigwedge^2G,A)$$

Composing with the inclusion into $Z^2(G,A)$ gives a map:

$$\operatorname{Hom}(\bigwedge^2G,A) \to Z^2(G,A)$$

This map is a one-sided inverse (and hence a section) of the skew map
$Z^2(G,A)$. It thus provides a natural splitting of the short exact sequence:

$$0 \to Z^2_{sym}(G,A) \to Z^2(G,A) \to \operatorname{Hom}(\bigwedge^2G,A) \to 0$$

Thus, it gives a {\em canonical} direct sum decomposition:

$$Z^2(G,A) \cong Z^2_{sym}(G,A) \oplus \operatorname{Hom}(\bigwedge^2G,A)$$

This direct sum decomposition is also internal (albeit we have to keep
in mind the halving to be done at the time of the Lie ring
identification):

$$Z^2(G,A) = Z^2_{sym}(G,A) + \operatorname{Hom}(\bigwedge^2G,A)$$

Going down to the level of cohomology, we get a natural internal
direct sum decomposition:

$$H^2(G,A) = H^2_{sym}(G,A) + \operatorname{Hom}(\bigwedge^2G,A)$$

And in fact, $\operatorname{Hom}(\bigwedge^2G,A)$ is identified with
all the many other subgroups of interest to us: $H^2_{CP}(G,A)$,
$H^2(G,A)/H^2_{sym}(G,A)$, and also $H^2_{SS}(G,A)$ -- the cohomology
classes with representatives that are skew symmetric $2$-cocycles.

\subsection{How is Baer's construction stronger?}

\begin{enumerate}
\item Baer's approach gives a {\em canonical} direct sum decomposition
  at the {\em level of cocycles}, with a lot more identifications than
  our approach.
\item Baer's approach is more canonical/natural/functorial
  (elaborate/clarify?). In particular, it gives a specific $2$-cocycle
  (Rther than just a cohomology class) and this choice is covariant
  with automorphisms.
\item \item Baer's $2$-cocycle is more than just a cyclicity-preserving
  $2$-cocycle: it is alternating, biadditive, skew symmetric,
  commutativity-preserving, subgroup-preserving, and much more. We
  will find that skew symmetry and the alternating property are often
  replicated in the $2$-cocycles we manage to find as well, but the
  other properties (biadditivity, commutativity-preserving,
  subgroup-preserving) often are not.
\end{enumerate}

\section{$2$-groups: examples}

In this section, we explore in detail the various cases where $A$ and
$G$ are both $2$-groups. These examples also set the stage for the
other concerns/ideas that we have.

\subsection{The Klein four-group}

We now restrict attention to the case where $G$ is a Klein
four-group. Before proceeding further, we note that in order to be
able to tackle the general case of elementary abelian $2$-groups, it
is important to first tackle the case of the Klein four-group because
any elementary abelian $2$-group contains a bunch of Klein
four-groups.

First, what are the possible alternating biadditive maps on the Klein
four-group? Alternating biadditive maps on $G$ are equivalent to
homomorphisms from $\bigwedge^2(G)$. When $G$ is the Klein four-group,
$\bigwedge^2(G)$ is cyclic of order $2$. The only possible
homomorphisms out from this are the trivial homomorphism and a mapping
that sends the non-identity element to an element of order $2$.

In the latter case, successfully {\em halving} the $2$-cocycle
requires the image element of order $2$ to itself be double of
something. It turns out that this is sufficient. We state this with
two back-to-back lemmas.

\begin{lemma}\label{KleinfourCPcocycleclassification}
  Suppose $G$ is a Klein four-group generated by $x$ and $y$. Let $A$
  be any abelian group. Let $F$ be the subgroup of $A$ comprising
  those elements whose order divides $4$. Then consider the map $\chi$
  from $F$ to the set of functions $G \times G \to A$ where $\chi(t)$
  is the $2$-cocycle $f$ given by:

  $$f(x,y) = f(y,xy) = f(xy,x) = t$$

  and:

  $$f(y,x) = f(xy,y) = f(x,xy) = -t$$

  Finally, we define $f$ to be $0$ whenever both its inputs are
  identical or either of its inputs is $0$.

  Then, the following are true:

  \begin{enumerate}
  \item $\chi$ is an isomorphism from $F$ to $Z^2_{CP}(G,A)$. In
    particular, $\chi(t)$ is a cyclicity-preserving $2$-cocycle for
    every $t$ and $\chi(t + u) = \chi(t) + \chi(u)$.
  \item $\chi(t)$ is the {\em unique} cyclicity-preserving $2$-coycle
    $f$ satisfying $f(x,y) = t$.
  \item Let $K$ be the subgroup of $F$ comprising the elements whose
    order divides $2$. Then, for any $t \in K$, $\chi(t)$ is a
    symmetric cyclicity-preserving $2$-cocycle and hence a
    $2$-coboundary.
  \item The image of $K$ coincides with the group of alternating
    biadditive maps from $G$ to $A$.
  \item $K$ is {\em precisely} the subgroup of $F$ whose image under
    $\chi$ is in the subgroup of $2$-coboundaries. Thus, $\chi$
    induces an isomorphism from $F/K$ to $H^2_{CP}(G,A)$.
  \end{enumerate}

  In the $p$-group jargon, if $A$ is a $2$-group, we would denote $F$
  as $\Omega_2(A)$ and $K$ as $\Omega_1(A)$. Thus, in the $p$-group
  jargon, $Z^2_{CP}(G,A) \cong \Omega_2(A)$ and $H^2_{CP}(G,A) \cong
  \Omega_2(G,A)/\Omega_1(G,A)$.
\end{lemma}

An easy corollary of this is:

\begin{lemma}
  Suppose $G$ is a Klein four-group generated by distinct elements $x$
  and $y$. Suppose $\lambda$ is an alternating biadditive map from $G
  \times G$ to $A$. Suppose $\lambda(x,y) = a$ for some element $a \in
  A$. Then, the mapping $\chi$ of the previous lemma establishes a
  bijection between the sets:

  $$\{ b \in A | 2b = a \} \leftrightarrow \{ f \in Z^2_{CP}(G,A) \mid \Skew f = \lambda \}$$

  where, by lemma \ref{CP2cocycleonelab}, all elements in
  $Z^2_{CP}(G,A)$ are $2$-cocycles and hence $\Skew f = 2f$.
\end{lemma}

Finally, we can summarize the previous lemmas plus more in a
theorem:

\begin{theorem}\label{Kleinfoursummary}
  Suppose $G$ is a Klein four-group generated by distinct elements $x$
  and $y$. Let $F$ be the subgroup of $G$ comprising elements whose
  order divides $4$ (thus, $F = \Omega_2(A)$ when $A$ is a $2$-group),
  $K$ be the subgroup of $A$ comprising elements whose order divides
  $2$ (thus, $K = \Omega_1(A)$ when $A$ is a $2$-group) and $L$ be the
  subgroup $2F$ (thus, $L = \mho^1(\Omega_2(A))$ when $A$ is a
  $2$-group. Let $\chi$ be the mapping from $F$ to $Z^2_{CP}(G,A)$
  described in lemma \ref{KleinfourCPcocycleclassification}. Then:

  \begin{enumerate}
  \item $$F \cong^{\chi} Z^2_{CP}(G,A)$$

    In $2$-group jargon:

    $$\Omega_2(A) \cong^{\chi} Z^2_{CP}(G,A)$$

  \item $$F/L \cong K \cong^{\chi} B^2_{CP}(G,A) \cong
    \operatorname{Hom}(\bigwedge^2G,A) \cong^{\Skew} H^2(G,A)/H^2_{sym}(G,A)$$

    In $2$-group jargon, $K$ is $\Omega_1(A)$.

  \item $$F/K \cong L \cong H^2_{CP}(G,A) \cong (H^2_{CP}(G,A) + H^2_{sym}(G,A))/H^2_{sym}(G,A)$$

    In $2$-group jargon, $F/K$ is $\Omega_2(A)/\Omega_1(A)$ and $L$ is
    $\mho^1(\Omega_2(A))$.
  \item $$K/L \cong H^2(G,A)/(H^2_{CP}(G,A) + H^2_{sym}(G,A))$$

    In $2$-group jargon, $K/L$ is $\Omega_1(A)/(\mho^1(\Omega_2(A)))$.
  \item In particular, $H^2(G,A) = H^2_{CP}(G,A) + H^2_{sym}(G,A)$ if
    and only if $K = L$, or, every element of order $2$ has a half. In
    other words, every central extension with base $A$ and quotient
    group $G$ can be untwisted to an abelian extension if and only if
    every element of $A$ having order $2$ is twice of something.
  \end{enumerate}
\end{theorem}

\subsection{Some examples with the Klein four-group and various $A$s}

Theorem \ref{Kleinfoursummary} completes the discussion for the case
where $G$ is the Klein four-group, but some concrete examples may
help. In these concrete examples, we actually look at the extension
groups. We also note the role of the $\operatorname{Aut}(G) \times
\operatorname{Aut}(A)$ group in permuting the possibilities.

In specific examples, we restrict attention to $A$ a $2$-group,
because the $2'$-part splits off and does not interact in any
interesting way with $G$.

In the case that $A$ is a cyclic $2$-group, we get the following:

\begin{enumerate}
\item $H^2(G,A)$ is an elementary abelian group of order $8$.
\item $H^2_{sym}(G,A)$ is a Klein four-group inside this elementary
  abelian group. Its identity element is $A \times G$. Its other
  three elements all have big group isomorphic to $\Z_{2^{m+1}}
  \times \Z_2$. These three non-identity elements form one orbit
  under the action of $\Aut(G)$ and hence also of $\Aut(G) \times
  \Aut(A)$.
\item $H^2(G,A)/H^2_{sym}(G,A)$ is cyclic of order $2$. The
  non-identity coset has $4$ elements. One of these elements has big
  group isomorphic to the central product $Q_8 *_{\Z_2} \Z_{2^m}$,
  which, for $m > 1$, is also $D_8 *_{\Z_2} \Z_{2^m}$. The three other
  elements are $D_8$ when $m = 1$ and $M_{2^{m+2}}$ when $m > 1$,
  where $M_{2^n}$ is the class two $2$-group with cyclic maximal
  subgroup. These three other elements are permuted by the action of
  $\Aut(G)$, and form a single orbit under $\Aut(G) \times \Aut(A)$.
\item In the case $m = 1$, $H^2_{CP}(G,A)$ is trivial, which is
  basically because $A$ does not have elements of order $4$.
\item For $m > 1$, $H^2_{CP}(G,A)$ is cyclic of order $2$, and its
  non-identity element corresponds to the group $Q_8 *_{\Z_2}
  \Z_{2^m}$. Thus, for $m > 1$, $H^2(G,A) = H^2_{CP}(G,A) +
  H^2_{sym}(G,A)$ so every group can be untwisted to an abelian
  group. We thus get two group correspondences: 
    
  $$Q_8 *_{\Z_2} \Z_{2^m} \to \Z_{2^m} \times \Z_2 \times \Z_2$$
  
  and
  
  $$M_{2^{m + 2}} \to \Z_{2^{m+1}} \times \Z_2$$
\end{enumerate}

\subsection{Higher elementary abelian groups}

Insert all the most interesting aspects from examples from {\em
you-know-where} after working out a number of examples in detail.

\subsection{Other abelian $2$-groups}

Insert all the most interesting aspects from examples from {\em
you-know-where} after working out a number of examples in detail. {\em
Also, remember to put an explanation about why the order statistics
often match up for generalized Baer IIP-Lie rings without there being
a generalized Baer CP-Lie ring}.

\section{Conditions on $2$-cocycles}

\subsection{Things to look out for}

How nicely can we choose the $2$-cocycles that we untwist a given group
by? We want constraints that make the group and the Lie ring look as
similar as possible. We make some observations about the possibilities
for the kinds of conditions that we will encounter:

\begin{itemize}
\item Some condition that we will encounter will depend on the
  cocycle. Some will depend only on the cohomology class. Others will
  put some restriction on the cohomology class but also restrict the
  cocycle within that cohomology class.
\item Some conditions that we will encounter (such as
  commutativity-preserving) will be conditions purely on the
  untwisting cocycle as an element of $Z^2(G,A)$ and will not depend
  on the specific extension that we start with. Others (such as
  subgroup-preserving) will depend on the specific extension that we
  start with.
\end{itemize}

\subsection{Commutativity-preserving}

In the cyclicity-preserving condition, we insisted that
$f(\overline{x},\overline{y}) = 0$ whenever the elements
$\overline{x},\overline{y} \in G$ generate a cyclic subgroup. This is
equivalent to insisting that $f(\overline{x},\overline{y}) = 0$
whenever $x,y$ generate a cyclic subgroup in the extension group
$E$.\footnote{It is not true that $\langle \overline{x},\overline{y}
\rangle$ cyclic implies $\langle x,y \rangle$ cyclic. However, we can
alter the representatives while remaining in the same coset to make
the representatives generate a cyclic subgroup.}

Thus, the condition that cyclic subgroups in the big group $E$ be
unaffected by the untwisting can be checked by a condition on the
cocycle {\em without} looking at the specific extension $E$. Moreover,
this condition (cyclicity-preserving) defines an additive subgroup and
hence we can apply all the ideas of group theory to it.

Let's now look at a more demanding condition: that
$f(\overline{x},\overline{y}) = 0$ whenever the elements $x,y$ commute
in the extension group $E$. We call a $f$ satisfying this condition a
{\em commutativity-preserving $2$-cocycle}. What is the condition for
being commutativity-preserving.

If $c$ is a $2$-cocycle defining the extension $E$, then $x,y \in E$
commute if $[x,y]$ is the identity element, which is equivalent to
saying that $(\Skew c)(\overline{x},\overline{y}) = 0$. Thus, the
condition of being commutativity-preserving can be written as:

$$(\Skew c)(\overline{x},\overline{y}) = 0 \implies f(\overline{x},\overline{y}) = 0$$

If we assume that $f$ is chosen such that $c - f$ is a symmetric
$2$-cocycle (which is the kind of $f$ we want in order to untwist to
an abelian group), then $\Skew c = \Skew f$, and we get:

$$(\Skew f)(\overline{x},\overline{y}) = 0 \implies f(\overline{x},\overline{y}) = 0$$

We call such a $f$ a {\em self-commutativity-preserving $2$-cocycle}.

Note that in Baer's construction for the uniquely $2$-divisible case,
this condition is automatically satisfied, because halving $0$ always
gives $0$. In our more general setup, however, this is far from
obvious. First, note that the set of $f$s which satisfy the condition
above do not even necessarily form a subgroup, because the condition
above is not additive. Hence, we cannot deduce that the set of
cohomology classes that contain self-commutativity-preserving
$2$-cocycles is a subgroup of $H^2(G,A)$. All we can say is that this
set is a {\em subset} of $H^2_{CP}(G,A)$.

We denote the set of self-commutativity-preserving $2$-cocycles by
$Z^2_{SCP}(G,A)$ and the set of cohomology classes that contain such
$2$-cocycles by $H^2_{SCP}(G,A)$. Note that neither of these is
guaranteed to be a subgroup. Both of these are invariant under the
$\Aut G \times \Aut A$-action.

Nonetheless, we have the following:

\begin{lemma}
  The set $H^2_{sym}(G,A) + H^2_{SCP}(G,A)$ corresponds to the set of
  extensions that have generalized Baer Lie rings with the property
  that for two elements that commute, the group product is the same as
  the Lie ring sum.
\end{lemma}

\subsection{Subgroup-preserving}

We say that $f$ is a {\em subgroup-preserving $2$-cocycle} for the
extension $E$ if, for all $x,y \in E$, $f(\overline{x},\overline{y})
\in \langle x,y \rangle$. If the $2$-cocycle that we use to untwist is
subgroup-preserving, subgroups in the original group map to subrings
in the Lie ring. We call such a Lie ring a {\em subgroup-preserving
generalized Baer Lie ring}. The subgroup-preserving property depends
heavily on the extension $E$, and it can be satisfied for some
extensions but not others for the {\em same} untwisting cocycle. For
instance, with $A = \Z_4$ and $G = V_4$, the extensions $D_8 * Z_4$
and $M_{16}$ are untwisted with the same cocycle. But while the
cocycle is subgroup-preserving for $M_{16}$, it is not for $D_8 *
\Z_4$.

We note the following:

\begin{lemma}
  A necessary condition for a group to have a subgroup-preserving
  generalized Baer Lie ring is that every subgroup of the group have a
  generalized Baer Lie ring.
\end{lemma}

Is the converse true? This is something I need to investigate.

Note that there are two conflicting impulses here. On the one hand, we
understand that subgroup-preserving generalized Baer Lie rings are
good. On the other hand, given that some groups such as $D_8$ and
$Q_8$ do not have generalized Baer Lie rings, the absence of the
subgroup-preserving condition is helpful in embedding these groups in
bigger groups (theorem \ref{embedding}) that do have generalized Baer
Lie rings. The tightest we can hope to get is if the converse to the
above lemma holds -- that indicates that subgroup-preserving
generalized Baer Lie rings exist whenever there is no single witness
obstructing their existence.

\end{document}
