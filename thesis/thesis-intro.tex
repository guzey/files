\section*{Background and notation}

\subsection{Background assumed}

This document assumes that the reader is comfortable with group theory
at an advanced undergraduate or beginning graduate student level. At
minimum, the reader's knowledge should be approximately equivalent to
the first six chapters of \cite{DummitFoote}. A knowledge of the
material in \cite{RotmanGT} would make the document easy
reading. There will be particular emphasis on knowledge of the
structure of $p$-groups and nilpotent groups, including knowledge of
the interplay between the upper central series and lower central
series. A review of the most important definitions and basic results
is available in the Appendix, Section \ref{appsec:group-basic}.

Rudimentary familiarity with the ideas of universal algebra and
category theory will be helpful in understanding the motivating
ideas. A review of the most important ideas is available in the
Appendix, Sections \ref{appsec:category} and \ref{appsec:univalg-basic}.

It is assumed that the reader is familiar with the idea of Lie rings,
which can be viewed as Lie algebras over $\mathbb{Z}$, the ring of
integers. However, familiarity with Lie {\em algebras}
over the real numbers or complex numbers will also be sufficient. A
review of some basic definitions from the theory of Lie rings can be
found in the Appendix, Section \ref{appsec:Lie}.

\subsection{Group and subgroup notation}\label{sec:group-notation}

Let $G$ be a group. We will use the following notation throughout this
document.

\begin{itemize}
\item We will use $1$ to denote the trivial subgroup of $G$. Note that
  the same letter $1$ will be used to denote both the trivial group as
  an abstract group and the trivial subgroup in all groups.
\item We will also use $1$ to denote the identity element of $G$.
\item When working with groups that are known to be abelian groups, we
  will use additive notation: $0$ to denote the trivial group and $+$
  to denote the group operation. However, we will use multiplicative
  notation when dealing with abelian subgroups inside a (possibly)
  non-abelian group.
\item $H \le G$ will be understood to mean that $H$ is a sub{\em
  group} of $G$.
\item $Z(G)$ will refer to the center of $G$.
\item $G'$ and $[G,G]$ both refer to the derived subgroup of $G$.
\item $\gamma_c(G)$ refers to the $c^{th}$ member of the lower central
  series of $G$, given as follows: $\gamma_1(G) = G$, $\gamma_2(G) = G'$,
  and $\gamma_{i+1}(G) = [G,\gamma_i(G)]$.
\item $Z^c(G)$ refers to the $c^{th}$ member of the upper central
  series of $G$, given as follows: $Z^0(G)$ is the trivial subgroup,
  $Z^1(G) = Z(G)$, and $Z^{i+1}(G)/Z^i(G) = Z(G/Z^i(G))$ for
  $i \ge 1$.
\item $G^{(i)}$ denotes the $i^{th}$ member of the derived series of
  $G$, given by $G^{(0)} = G$, $G^{(1)} = G'$, and $G^{(i+1)} =
  [G^{(i)},G^{(i)}]$.
\item $\operatorname{Inn}(G)$ is the inner automorphism group of
  $G$. It is canonically isomorphic to the quotient group $G/Z(G)$,
  and we will often abuse notation by treating $\operatorname{Inn}(G)$
  as set-theoretically identical with $G/Z(G)$.
\item $\operatorname{Aut}(G)$ is the automorphism group of $G$. We
  treat $\operatorname{Inn}(G)$ naturally as a subgroup of
  $\operatorname{Aut}(G)$. In fact, $\operatorname{Inn}(G)$ is a
  normal subgroup of $\operatorname{Aut}(G)$.
\item $\operatorname{End}(G)$ is the endomorphism {\em monoid} of $G$,
  i.e., the set of endomorphisms of $G$ with the monoid structure
  given by composition.
\end{itemize}

\subsection{Lie ring and subring notation}\label{sec:lie-ring-notation}

Let $L$ be a Lie ring, i.e., a Lie algebra over $\Z$, the ring of
integers. We will use the following notation throughout this document.

\begin{itemize}
\item We will use $0$ to denote the zero subring of $L$. Note that $0$
  is used to describe both the abstract zero Lie ring and the zero
  subring in every Lie ring.
\item We will also use $0$ to denote the zero element of $L$.
\item $M \le L$ will be understood to mean that $M$ is a Lie subring
  of $L$. This means that it is an additive subgroup of $L$ and is
  closed under the Lie bracket.
\item $Z(L)$ denotes the center of $L$, i.e., the subring of $L$
  comprising those elements whose Lie bracket with any element of $L$
  is zero.
\item $L'$ and $[L,L]$ both refer to the derived subring of $L$.
\item $\gamma_c(L)$ refers to the $c^{th}$ member of the lower central
  series of $L$, given as follows: $\gamma_1(L) = L$, $\gamma_2(L) =
  L'$, and $\gamma_{i+1}(L) = [L,\gamma_i(L)]$.
\item $Z^c(L)$ refers to the $c^{th}$ member of the upper central
  series of $L$, given as follows: $Z^0(L)$ is the trivial subring,
  $Z^1(L) = Z(L)$, and $Z^{i+1}(L)/Z^i(L) = Z(L/Z^i(L))$ for
  $i \ge 1$.
\item $L^{(i)}$ denotes the $i^{th}$ member of the derived series of
  $L$, given by $L^{(0)} = L$, $L^{(1)} = L'$, and $L^{(i+1)} =
  [L^{(i)},L^{(i)}]$.
\item $\operatorname{Inn}(L)$ is the Lie ring of inner derivations of
  $L$. It is canonically isomorphic to the quotient Lie ring $L/Z(L)$,
  and we will often abuse notation by treating $\operatorname{Inn}(L)$
  as set-theoretically identical with $L/Z(L)$.
\item $\operatorname{Der}(L)$ is the Lie ring of all derivations of
  $L$. We treat $\operatorname{Inn}(L)$ naturally as a Lie subring of
  $\operatorname{Der}(L)$. In fact, $\operatorname{Inn}(L)$ is an
  ideal in $\operatorname{Der}(L)$.
\item $\operatorname{Aut}(L)$ is the automorphism group of $L$.
\item $\operatorname{End}(L)$ is the endomorphism {\em monoid} of $L$
  considered as a Lie ring. Note that this is not necessarily closed
  under addition.
\item $\operatorname{End}_{\mathbb{Z}}(L)$ is the endomorphism ring of
  the underlying additive group of $L$. To avoid confusion, we will
  explicitly specify that we are looking at all additive group
  endomorphisms whenever we use this notation.

\end{itemize}

\subsection{Other conventions}

We will adopt these conventions:

\begin{itemize}
\item As a {\em general} rule, when dealing with homomorphisms and
  other similar functions, we will apply functions on the left, in
  keeping with the convention used in most mathematics texts. Thus, $f
  \circ g$ is to be interpreted as saying that the function $g$ is
  applied first and the function $f$ is applied later.
\item For the action of a group on itself, we denote by ${}^gx$ the
  action of $g$ by conjugation on $x$ as a left action, i.e.,
  $gxg^{-1}$. We denote by $x^g$ the action of $g$ by conjugation on
  $x$ as a right action, i.e., $g^{-1}xg$. When stating results whose
  formulation is sensitive to whether we use the left-action
  convention or the right-action convention, we will explicitly state
  the result using both conventions.
\item If using the left-action convention, the group commutator
  $[x,y]$ is defined as $xyx^{-1}y^{-1}$. If using the right-action
  convention, the group commutator $[x,y]$ is defined as
  $x^{-1}y^{-1}xy$.
\end{itemize}

%\chapter{Introduction, outline, and preliminaries}

%\newpage

\section{Introduction}\label{sec:intro}

\subsection{The difference in tractability between groups and abelian groups}

The {\em structure theorem for finitely generated abelian groups},
which in turn leads to a classification of all finite abelian groups,
shows that the structure of {\em abelian} groups is fairly easy to
understand and control. On the other hand, the structure of groups in
general is wild.  Even classifying finite groups is extremely
difficult.

The difficulty is two-fold. On the one hand, the finite simple groups
(which can be thought of as the building blocks of finite groups) have
required a lot of effort to classify. While the original
classification was believed to have been completed around 1980, some
holes in parts of the proof were discovered later and it is believed
that these holes were fixed only around 2004. The only finite simple
abelian groups are the cyclic groups of order $p$. However, there are
$17$ infinite families and $26$ sporadic groups among the finite
simple non-abelian groups. For a quick background on the
classification, see \cite{Asch2004}.

At the other extreme from finite simple groups are the finite
$p$-groups. It is well known that any finite group of order $p^n$ (for
a prime $p$ and natural number $n$) must be a nilpotent group and
therefore it has $n$ composition factors that are all cyclic groups of
order $p$. In other words, there is no mystery about the building
blocks of these groups. Despite this, the multiplicity of ways of
putting the building blocks together makes it very difficult to obtain
a concise description of all the groups of order $p^n$. The general
consensus among people who have studied $p$-groups is that it is
futile to even attempt to obtain a concise description of all the
isomorphism types of groups of order $p^n$, and that it is likely that
no such description exists. Rather, the goal of the study of
$p$-groups is to identify methods that enable us to better understand
the totality of $p$-groups, including aspects that are common to all
of them and aspects that differentiate some $p$-groups from
others. For a description of the state of knowledge regarding
$p$-groups, see \cite{Mann99}.\footnote{Although the article was
  published in 1999, progress has been modest since then.}

This thesis is focused on one small part of the study of finite
$p$-groups.

\subsection{Nilpotent groups and their relation with abelian groups}

A group is termed {\em nilpotent} if it has a central series of finite
length. Nilpotent groups are considerably more diverse in nature than
abelian groups, and as alluded to in the preceding section, even the
finite nilpotent groups are difficult to classify.

A group is termed {\em solvable} if it has a normal series where all
the quotient groups are abelian groups. Solvable groups are
considerably more diverse than nilpotent groups.

Generally, statements that are true for abelian groups fall into one
of these four classes:

\begin{enumerate}
\item The statement does not generalize much further from abelian
  groups
\item The statement generalizes all the way to nilpotent groups but
  not much further
\item The statement generalizes all the way to solvable groups but not
  much further
\item The statement generalizes to all groups, or to a fairly large
  class of groups
\end{enumerate}

It might be worthwhile to attempt to understand why the properties of
being nilpotent and being solvable differ qualitatively, and why the
former is far closer to being abelian than the latter. In an abelian
group, the commutativity relation holds precisely: $ab = ba$ for all
$a$ and $b$ in the group. In general, $ab$ and $ba$ ``differ'' by a
commutator, i.e., $ab = [a,b]ba$ if we use the left action convention
for commutators.

When we consider expressions in a group and try to rearrange the terms
of the expression, the process of rearrangement introduces
commutators. These commutators themselves need to be moved past
existing terms, which introduces commutators between the commutators
and existing terms. In a nilpotent group, we eventually reach a stage
where the iterated commutators that we obtain are central, and
therefore can be freely moved past existing terms. In a solvable
group, such a stage may never arise.

An alternative perspective is that of {\em iterative algorithms}, a
common class of algorithms found in numerical analysis and other parts
of mathematics. An iterative algorithm attempts to find a solution to
a problem by guessing an initial solution and iteratively refining
the guess by identifying and correcting the error in the initial
solution. There are many iterative algorithms that are guaranteed to
terminate only for nilpotent groups, and where the number of steps in
which the algorithm is guaranteed to terminate is bounded by the
nilpotency class of the group. These algorithms work in a single step
for abelian groups, because commutativity allows for the necessary
manipulations to happen immediately. For non-abelian nilpotent groups,
the algorithms work by gradually refining guesses modulo members of a
suitable central series (such as the upper central series or lower
central series).

\subsection{The Lie correspondence: general remarks}\label{sec:lie-correspondence-intro}

The {\em non-abelianness} of groups makes it comparatively difficult
to keep track of group elements and to study the groups. It would be
very helpful to come up with an alternate description of the structure
of a group that replaces the (noncommutative) group multiplication
with a commutative group multiplication, and stores the
noncommutativity in the form of a separate operation. A {\em Lie ring}
(defined in the Appendix, Section \ref{appsec:Lie}) is an example of
such a structure.

For readers familiar with the concept of {\em Lie algebras} over $\R$
or $\mathbb{C}$, note that the definition of Lie ring is similar,
except that the underlying additive group is just an abelian group
(rather than being a $\R$-vector space or $\C$-vector space) and the
Lie bracket is just $\Z$-bilinear rather than being $\R$-bilinear or
$\C$-bilinear. In particular, any Lie algebra over $\R$ or $\C$ is a
Lie ring, but not every Lie ring is a Lie algebra over $\R$ or $\C$,
and even if it is, there may be multiple ways of giving it such a Lie
algebra structure.

The {\em Lie correspondence} is an important correspondence in the
theory of real Lie groups. For an elementary exposition of this
correspondence, see \cite{vsv}. We recall here some of the key
features of the correspondence.

To any finite-dimensional real Lie group, we can functorially
associate a $\R$-Lie algebra called the {\em Lie algebra of the Lie
  group}. The underlying vector space of the Lie algebra is the
tangent space at the identity to the Lie group, or equivalently, the
space of left-invariant vector fields, and the Lie bracket is defined
using the Lie bracket of vector fields. Note that the Lie algebra of a
Lie group depends only on the connected component of the identity.

Additionally, there exists a map, called the {\em exponential map},
from the Lie algebra to the Lie group. This map need not be bijective
globally, but it must be bijective in a small neighborhood of the
identity. The inverse of the map, again defined in a small
neighborhood of the identity, is the {\em logarithm} map. Note that
the exponential map is globally defined, but the logarithm map is
defined only locally.

The association is not quite a correspondence. The problem is that
different Lie groups could give rise to isomorphic Lie
algebras. However, if we restrict attention to {\em connected simply
  connected Lie groups}, then the association becomes a
correspondence, and we can construct a functor in the reverse
direction. Explicitly, the Lie correspondence is the following
correspondence, functorial in both directions:

\begin{center}
  Connected simply connected finite-dimensional real Lie groups
  $\leftrightarrow$ Finite-dimensional real Lie algebras
\end{center}

\subsection{The Lie algebra for the general linear group}\label{sec:lie-correspondence-gln}

Denote by $GL(n,\R)$ the general linear group of degree $n$ over the
field of real numbers, i.e., the group of all invertible $n \times n$
matrices with real entries. Denote by $\mathfrak{g}\mathfrak{l}(n,\R)$ the ``general linear
Lie algebra'' of degree $n$ over $\R$. Explicitly, $\mathfrak{g}\mathfrak{l}(n,\R)$ is the
vector space of all $n \times n$ matrices over $\R$, and the Lie
bracket is defined as $[x,y] := xy - yx$.

$\mathfrak{g}\mathfrak{l}(n,\R)$ is the Lie algebra of $GL(n,\R)$. The
exponential and logarithm maps in this case are the usual matrix
exponential and matrix logarithm maps. The exponential map:

$$\exp: \mathfrak{g}\mathfrak{l}(n,\R) \to GL(n,\R)$$

is defined as:

$$x \mapsto \sum_{i=0}^\infty \frac{x^i}{i!} = 1 + x + \frac{x^2}{2!} + \frac{x^3}{3!} + \dots$$

The matrix exponential is defined for all matrices. However, the
exponential map is neither injective nor surjective:

\begin{itemize}
\item The exponential map is not surjective for any $n$. For $n = 1$,
  this is because the exponential of any real number is a {\em
    positive} real number. A similar observation holds for larger $n$,
  once we observe that the image of the exponential map is inside
  $GL^+(n,\R)$, the subgroup of $GL(n,\R)$ comprising the matrices of
  positive determinant. However, for $n > 1$, the exponential map is
  not surjective even to $GL^+(n,\R)$. For instance, the following
  matrix is not the exponential of any matrix with real entries:

  $$\begin{pmatrix} -1 & 1 \\ 0 & -1 \\\end{pmatrix}$$

\item The exponential map is not injective for $n > 1$. For instance,
  for any positive integer $m$, the following matrix has exponential
  equal to the identity matrix:

  $$\begin{pmatrix} 0 & 1 \\ -4m^2\pi^2 & 0 \\\end{pmatrix}$$
\end{itemize}

Nonetheless, we can find an open neighborhood $U$ of the zero matrix
in $\mathfrak{g}\mathfrak{l}(n,\R)$ and an open neighborhood $V$ of
the identity matrix in $GL(n,\R)$ such that the exponential map is
bijective (and in fact, is a homeomorphism) from $U$ to $V$.

Note that $GL(n,\R)$ is not a connected simply connected Lie group, so
the above is not an instance of the Lie {\em correspondence}.
\subsection{The nilpotent case of the Lie correspondence}

The example of $\mathfrak{g}\mathfrak{l}(n,\R)$ and $GL(n,\R)$
illustrates that the exponential map does not always behave
nicely. However, it turns out that the exponential map behaves much
better when we apply the Lie correspondence in the {\em nilpotent}
case. Explicitly, the nilpotent case of the Lie correspondence is a
correspondence:

\begin{center}
  Connected simply connected finite-dimensional nilpotent real Lie
  groups $\leftrightarrow$ Finite-dimensional nilpotent real Lie
  algebras
\end{center}

In the nilpotent case, it will turn out that the exponential map is
{\em bijective}, and in fact, it defines a homeomorphism from the Lie
algebra to the Lie group. Thus, we can define its inverse, the
logarithm map, {\em globally}.

We now turn to an example.

\subsection{The example of the unitriangular matrix group}\label{sec:unitriangular-lie-correspondence}

A special case of interest for us is the correspondence between the
Lie ring $NT(n,\R)$ of $n \times n$ strictly upper triangular matrices
over $\R$ and the group $UT(n,\R)$ of $n \times n$ upper triangular
matrices over $\R$ with all the diagonal entries equal to $1$. This
correspondence gives a bijection between the underlying sets of
$NT(n,\R)$ and $UT(n,\R)$ via the exponential map. Explicitly, the
matrix exponential defines a bijective set map:

$$\exp: NT(n,\R) \to UT(n,\R)$$

given explicitly as:

$$\exp(x) = e^x = 1 + x + \frac{x^2}{2!} + \dots + \frac{x^{n-1}}{(n - 1)!}$$

Note that this coincides with the usual matrix exponential because
$x^n = 0$ and all higher powers of $x$ are therefore also zero. In
other words, this exponential map is the restriction to $NT(n,\R)$ of
the exponential map described in the preceding section:

$$\exp: \mathfrak{g}\mathfrak{l}(n,\R) \to GL(n,\R)$$

However, {\em unlike} the case of $GL(n,\R)$, the exponential map from
$NT(n,\R)$ to $UT(n,\R)$ is bijective, and in fact, is a
homeomorphism. Topologically, both $NT(n,\R)$ and $UT(n,\R)$ are
homemorphic (i.e., isomorphic in the category of topological spaces) to
the vector space $\R^{\binom{n}{2}}$.

The inverse set map is the matrix logarithm, now defined {\em
  globally}:

$$\log: UT(n,\R) \to NT(n,\R)$$

given explicitly as:

$$\log x := (x - 1) - \frac{(x - 1)^2}{2} + \frac{(x - 1)^3}{3} - \dots + \frac{(-1)^n (x - 1)^{n-1}}{n - 1}$$

\subsection{The Malcev correspondence and Lazard correspondence}\label{sec:intro-malcev-and-lazard}

The {\em Malcev correspondence} is a generalization of the nilpotent
case of the Lie correspondence that applies to algebras over the field
of rational numbers. Explicitly, the correspondence is:

\begin{center}
  Rationally powered nilpotent groups $\leftrightarrow$ Nilpotent $\Q$-Lie algebras
\end{center}

We will define ``rationally powered'' in Section
\ref{sec:group-powering}, but a quick definition for our purpose is
that every element has a unique $n^{th}$ root for every positive
integer $n$. The Malcev correspondence is a purely algebraic
correspondence that does not deal with topological structure. Note
that any $\R$-Lie algebra is a $\Q$-Lie algebra as well. It turns out
that for any nilpotent $\R$-Lie algebra, the Lie correspondence
coincides with the Malcev correspondence. Thus, for instance, under
the Malcev correspondence, the group associated with $NT(n,\R)$ is
$UT(n,\R)$.

The Malcev correspondence has a slight further generalization called
the {\em Lazard correspondence}, introduced by Lazard in
\cite{Lazardsoriginal}. The Lazard correspondence relaxes the
  assumption of being ``rationally powered'' and replaces it with the
  assumption that unique division by specific primes (namely, primes
  that are less than or equal to the nilpotency class) is possible.

If we use the Lazard correspondence in the direction from groups to
Lie rings, then it allows us to convert (a suitable type of) abstract
nilpotent group to a nilpotent Lie ring. The addition operation of the
Lie ring captures the abelian part of the group multiplication,
whereas the Lie bracket captures the non-abelian part of the group
multiplication.

Unfortunately, the Lazard correspondence applies only to {\em some}
nilpotent groups and some nilpotent Lie rings. Specifically, for
finite $p$-groups, it only works for finite $p$-groups where any
subset of size three generates a subgroup of nilpotency class at most
$p - 1$. For the bulk of this document, we will restrict our attention
to the case of small {\em global class}, i.e., the subcorrespondence
that applies to finite $p$-groups of nilpotency class at most $p - 1$.

This means that groups that have higher nilpotency class (a
way of saying that the groups are relatively more non-abelian) cannot
be studied directly using the Lazard correspondence.

We will describe the Malcev correspondence and the Lazard
correspondence in detail in Sections \ref{sec:malcev-correspondence},
\ref{sec:global-lazard-correspondence}, and
\ref{sec:lazard-correspondence}. For a textbook-style presentation of
the correspondence, see Khukhro's book \cite{Khukhro}, Chapters 9 and
10.

\subsection{Our generalization of the Lazard correspondence}

The goal of this document is to describe a generalization of the
Lazard correspondence that works for all $p$-groups of nilpotency
class at most $p$. In other words, it allows us to generalize the
Lazard correspondence to a slightly bigger collection of groups. The
limitation of this generalization is that the correspondence only
works between {\em equivalence classes of groups} and {\em equivalence
  classes of Lie rings}, with each equivalence class containing
multiple isomorphism types. The equivalence relation of interest here
is the equivalence relation of {\em isoclinism}. Informally, two
groups are isoclinic if their commutator maps are equivalent, and two
Lie rings are isoclinic if their Lie bracket maps are equivalent.

\subsection{Similarities and differences between groups and Lie rings}

The theories of groups and Lie rings are {\em structurally
  similar}. For many concepts related to groups, there are analogously
defined concepts for Lie rings. In most cases, the analogous
definition suggests itself naturally. Often, even the proofs are
similar. In some cases, proofs are easier for Lie rings than for
groups, primarily because the Lie bracket is bilinear.

There are {\em some} concepts that make sense only on the group side,
and some concepts that make sense only on the Lie ring
side. Similarly, there are some facts that are true only on the group
side, and some facts that are true only on the Lie ring side.

The closer we are to abelianness, the more structurally similar the
theory for groups is to the theory for Lie rings. In many cases, a
fact is true for nilpotent Lie rings if and only if the ``analogous''
fact is true for nilpotent groups. There are many facts that are true
in {\em general} for Lie rings and are not true in general for groups,
but they are true for nilpotent groups.

In addition to {\em structural similarity}, we will also see some
instances of {\em bijective correspondences} between certain types of
groups and certain types of Lie rings (including the Lie
correspondence and the Lazard correspondence). It will turn out that
{\em analogous concepts} become {\em bijectively correspondent} under
these correspondences. For instance, normal subgroups of groups are
analogous to ideals in Lie rings. The Lazard correspondence between
groups and Lie rings establishes a bijective correspondence between
(certain kinds of) normal subgroups of the group and (certain kinds
of) ideals of the Lie ring.

\subsection{Our central tool: Schur multipliers}

Our goal is to extend the domain of the Lazard correspondence by
relaxing its strictness (from a correspondence up to isomorphism to a
correspondence up to isoclinism). In particular, we are interested in
extending the Lazard correspondence to {\em nilpotency class one
  higher} than where it applies. Thus, the groups
(respectively, Lie rings) of interest to us arise as {\em central
  extensions} where the quotient group (respectively, quotient Lie
ring) is in the domain of the Lazard correspondence.

Rather than directly trying to study the groups and Lie rings, we
study the theory of central extensions for groups and Lie rings. We
first develop the {\em general} theory of such central
extensions. Then, we apply that general theory to the case where the
quotient group (respectively quotient Lie ring) of the central
extension lies in the domain of the Lazard correspondence. In the edge
case of interest where the group is in the domain of the Lazard
correspondence but its central extensions are ``just outside'' the
domain, we can obtain new insights. For instance, if $G$ is a
$p$-group of nilpotency class exactly $p - 1$, it is a Lazard Lie
group. The central extensions with quotient group $G$ are $p$-groups
of nilpotency class either $p - 1$ or $p$. The latter may lie outside
the domain of the Lazard correspondence.

On both the group side and the Lie ring side, the theory of central
extensions is governed by an abelian group called the {\em Schur
  multiplier}. There is a rich theory behind the Schur multiplier, and
it connects with important ideas from algebraic topology and
homological algebra. We will explore the necessary facets of this
theory. Eventually, we will prove (in Theorem
\ref{thm:global-lazard-correspondence-preserves-schur-multipliers})
that if a Lie ring and a group are in Lazard correspondence, then
their Schur multipliers are isomorphic. A version of the statement for
finite $p$-groups appeared as a conjecture in the paper
\cite{SchurmultiplierandLazard} by Eick, Horn, and Zandi in September
2012, stated informally after Theorem 2 of the paper.\footnote{The
  authors write: ``Based on various example computations, see also
  [7], we believe that Theorems 1 and 2 also hold for finite
  $p$-groups of class $p - 1$. However, our proofs do not extend to
  this case.''  The reference [7] alluded to by the authors has not
  yet been published or made available online. For a more detailed
  discussion, see Section
  \ref{sec:global-lazard-correspondence-preserves-schur-multipliers}}
Once the Schur multipliers are established to be isomorphic, it is
easy to establish the Lazard correspondence up to isoclinism.

\subsection{Globally and locally nilpotent}\label{sec:globally-and-locally-nilpotent}

The numbers $2$ and $3$ are particularly significant in the context of
the axiomatization of groups and Lie rings, and they also play an
important role in the Lazard correspondence. $2$ is the maximum of the
arities\footnote{The arity of an operation is the number of inputs it
  takes. For instance, group multiplication has arity $2$. Arity is
  discussed in more detail in the Appendix, Section
  \ref{appsec:univalg-basic}.} of the operations used in the definition
of groups. In particular, this means that if a function between groups
restricts to a homomorphism on every subgroup generated by at most $2$
elements, then the function is globally a homomorphism.

$3$ is the maximum of the number of variables that appear in the
identities that define a group. In particular, this means that if an
algebra has the same signature as a group (i.e., a $0$-ary operation
for the identity, a unary operation for the inverse map, and a binary
operation for the group multiplication), and every subalgebra of the
algebra generated by at most $3$ elements is a {\em group}, then the
algebra is globally a group.

The same is true for Lie rings: the maximum of the arities of the
operations is $2$, and the maximum of the number of terms that appear
in the defining identities is $3$. Thus, any function between Lie
rings that restricts to a homomorphism on subrings generated by sets
of size at most $2$ is globally a homomorphism. Further, given an
algebra with the same signature as a Lie ring, such that every
subalgebra generated by at most $3$ elements becomes a Lie ring with
the induced operations, the algebra as a whole is a Lie ring.

The formulas used in the Lazard correspondence describe the group
operations in terms of the Lie ring operations, and conversely
describe the Lie ring operations in terms of the group operations. The
{\em formulas} themselves refer to a maximum of two elements at a
time. However, the {\em verification} that these formulas work (i.e.,
that starting from a Lie ring, we end up with a group, or that
starting from a group, we end up with a Lie ring) relies on looking at
three elements at a time. For instance, to verify that a formula
describing group operations in terms of Lie ring operations does
indeed define a group structure, we need to verify the associativity
identity for three arbitrary elements. Similarly, to verify that a
formula describing Lie ring operations in terms of group operations
does indeed define a Lie ring structure, we need to verify the
associativity of addition, bilinearity, and Jacobi identity for the
Lie ring operations. Each of these identities requires considering
three arbitrary elements at a time.

Thus, the conditions that we work out on groups (respectively, Lie
rings) pertaining to the Lazard correspondence are $3$-local
conditions: they are conditions on what subgroups (respectively, Lie
subrings) generated by subsets of size at most three look like.

\subsection{The structure of this document}

The document is quite long despite the fact that the eventual proofs
are relatively short and simple. The reason is that the existing
literature we draw upon is fragmented. We draw on literature with
these five broad themes:

\begin{itemize}
\item Isoclinism and homoclinism.
\item Schur multiplier and the relation with group extension theory.
\item Exterior square and its generalizations.
\item The Lazard correspondence.
\item The behavior of groups and Lie rings where we can divide by
  specific primes.
\end{itemize}

Each of these themes has a well-developed body of literature. However,
the connections between these ideas are not emphasized in the
literature, and it often requires a careful reading to glean
them. Thus, it would not be sufficient to simply cite the relevant
literature. We use the next few sections to develop all the necessary
background material in preparation for our results.

Our presentation will follow these features:

\begin{itemize}
\item For the foundational sections, we will systematically alternate
  sections between groups and Lie rings. A section about groups will
  develop a concept or construct in the context of groups. The next
  section about Lie rings will develop the analogous concept or
  construct in the context of Lie rings. To the extent possible, we
  will follow parallel modes of presentation in the two
  sections. Differences between the sections will be noted at the
  beginnings of the relevant sections.
\item For the foundational sections, we will often begin by discussing
  a concept in the context of groups or Lie rings in the abstract, and
  then discuss an analogous concept in the context of extensions of
  groups or Lie rings. This will be done somewhat in reverse in the
  later sections, where we sometimes prove a result in the context of
  extensions (of groups or Lie rings) and {\em then} apply that to
  prove the result in the context of groups or Lie rings. This will be
  our {\em modus operandi} for the crucial proofs.
\item Our key results involve generalizing certain correspondences
  (the Baer correspondence and Lazard correspondence) to a larger
  domain, but with a coarser equivalence relation (of isoclinism). For
  each correspondence that we generalize, we first explicitly
  describe the known correspondence and its key attributes (in one
  or more sections), and then describe our generalization.
\end{itemize}

\subsection{For a quick reading}

For readers who wish to understand the main results without delving
into background concepts in unnecessary depth, the following reading
sequence will work:

\begin{enumerate}
\item Chapter 1 (Introduction, outline, and preliminaries):
  \begin{itemize}
  \item Section \ref{sec:outline} contains the outline of our main
    proof techniques. It is worth reading in its entirety.
  \item Section \ref{sec:abelian-lie-correspondence} (The abelian Lie
    correspondence): The contents of this section are straightforward,
    but it is worth reading because the methods used in this section
    form a template for later, more complicated, correspondences.
    \end{itemize}
\item Chapter 2 (Isoclinism and homoclinism: basic theory):
  \begin{itemize}
  \item Section \ref{sec:isoclinism-and-homoclinism} (Isoclinism and
    homoclinism of groups): It suffices to read Sections
    \ref{sec:isoclinism-definition} -- \ref{sec:homoclinism-category}, and
    the statements of the theorems in Section
    \ref{sec:homoclinism-misc-results}. Readers already familiar with
    the definitions can skip this section and return if needed.
  \item Section \ref{sec:isoclinism-and-homoclinism-lie} (Isoclinism
    and homoclinism of Lie rings): It suffices to read Section
    \ref{sec:isoclinism-definition-lie}. Readers already familiar with
    the definitions can skip this section and return if
    needed. Readers who thoroughly understand the general analogy
    between groups and Lie rings can extrapolate the definitions and
    results of this section from the preceding one, and hence may skip
    this section.
  \end{itemize}
\item Chapter 3 (Extension theory):
  \begin{itemize}
  \item Section \ref{sec:ses-group} (Short exact sequences of groups):
    Readers already familiar with the basics of short exact sequences
    and central extensions can read Sections
    \ref{sec:second-cohomology-group-classify-extensions} and
    \ref{sec:homomorphism-central-extensions}.
  \item Section \ref{sec:ses-lie} (Short exact sequences and central
    extensions of Lie rings): Readers who understood the preceding
    section (Section \ref{sec:ses-group}), and understand how the
    analogy between groups and Lie rings works, can skip this section.
  \item Section \ref{sec:cohomology-explicit} (Explicit description of
    second cohomology group): This section can be skipped without loss
    of continuity. The material in this section helps with understanding
    Section \ref{sec:baer-correspondence-cocycle-level}. However, the
    latter can also be skipped without loss of continuity.
  \item Section \ref{sec:exteriorsquare-and-homoclinism} (Exterior
    square, Schur multiplier, and homoclinism): This section is
    important to understand because it lays the foundation for later
    material, and the presentation is non-standard. Readers may skip
    proofs, many of which are tedious, and focus on the statements of
    the results.
  \item Section \ref{sec:exteriorsquare-and-homoclinism-lie} (Exterior
    square, Schur multiplier, and homoclinism for Lie rings): Apart from
    Section \ref{sec:free-lie-ring-on-abelian-group}, this section is
    mostly analogous to the preceding section. Hence, the rest of the
    section can be skipped.
  \item Section \ref{sec:schur-multiplier-and-second-cohomology}: This
    section is important to understand because it lays the foundation
    for later material, and the presentation is non-standard. Readers
    may skip proofs, many of which are tedious, and focus on the
    statements of the results.
  \item Section \ref{sec:schur-multiplier-and-second-cohomology-lie}:
    This section may be skipped by readers who have a thorough
    understanding of the preceding section and understand how the
    analogy between groups and Lie rings works.
  \item Sections \ref{sec:exterior-and-tensor-product} and
    \ref{sec:exterior-and-tensor-product-lie} (Exterior and tensor
    products for groups and Lie rings respectively): These sections can
    be skipped without loss of continuity, and interested readers can
    refer back to the explicit descriptions as needed later.
  \item Sections \ref{sec:free-nilpotent-groups-homology} and
    \ref{sec:free-nilpotent-lie-homology}: These are worth skimming for
    their main results.
  \end{itemize}
\item Chapter 4 (Powering over sets of primes):
  \begin{itemize}
  \item Section \ref{sec:group-powering} (Groups powered over sets of
    primes): Readers would benefit by reading the part of Section
    \ref{sec:group-powering} up to and including Section
    \ref{sec:divisible-and-torsion-free} in order to familiarize
    themselves with the definitions. Some of the results presented in
    the rest of the section are useful, but they can be revisited as
    necessary.
  \item Section \ref{sec:lie-ring-powering} (Lie rings powered over sets
    of primes): It suffices to read Section
    \ref{sec:lie-ring-powering-def}.
  \item Section \ref{sec:free-powered-groups-and-powering-functors}
    (Free powered groups and powering functors): The results in Sections
    \ref{sec:root-set} and \ref{sec:pi-powered-isoclinism-results} are
    the most important. The rest of the section may be skimmed.
  \item Section \ref{sec:free-powered-lie-rings-and-powering-functors}
    (Free powered Lie rings and powering functors): The results here are
    analogous to the preceding section, though the proofs are more
    straightforward. The section can be skipped and returned to as
    needed.
  \end{itemize}
\item Chapter 5 (Baer correspondence):
  \begin{itemize}
  \item Sections \ref{sec:baer-correspondence-basics} and
    \ref{sec:baer-correspondence-more} (Baer correspondence): It suffices
    to read Sections
    \ref{sec:baer-lie-definitions}-\ref{sec:baer-lie-ring-to-group} and
    Section \ref{sec:baer-p-group-case}. However, readers may benefit
    from skimming both sections in their entirety in order to get a
    better sense.
  \item Section \ref{sec:baer-correspondence-definition-relaxation} may
    be skipped without loss of continuity.
  \item Section \ref{sec:bcuti} (Baer correspondence up to isoclinism):
    Reading the whole section is strongly recommended, but readers may
    skip Section \ref{sec:baer-correspondence-cocycle-level} without
    loss of continuity.
  \item Section \ref{sec:bcuti-ex} contains interesting examples worth
  reading but may be skipped without loss of continuity.
  \end{itemize}
\item Chapter 6 (The Malcev and Lazard correspondences):
  \begin{itemize}
  \item Sections \ref{sec:adjoint-exp-log} and
    \ref{sec:free-nilpotent-exp-log} (adjoint groups, exponential and
    logarithm maps, and free nilpotent groups): These sections may be
    skimmed without reading the proofs. They provide technical
    background for Section \ref{sec:bch}.
  \item Section \ref{sec:bch} (Baker-Campbell-Hausdorff formula): This
    section should be read in its entirety. Readers may benefit from
    concentrating on the statements of the theorems and skimming the
    proofs.
  \item Section \ref{sec:inverse-bch}: This section is partly analogous
    to Section \ref{sec:bch}, so aside from the introduction, it may be
    skimmed.
  \item Sections \ref{sec:malcev-correspondence},
    \ref{sec:global-lazard-correspondence}, and
  \ref{sec:lazard-correspondence} (Malcev correspondence, global
  Lazard correspondence, and Lazard correspondence): These sections
  are worth reading, though people familiar with the correspondences
  may skim them.
  \end{itemize}
\item Chapter 7 (Generalizing the Lazard correspondence to a
  correspondence up to isoclinism):
  \begin{itemize}
  \item Section
    \ref{sec:group-commutator-and-lie-bracket-ito-each-other} (Group
    commutator and Lie bracket in terms of each other) is important.
  \item Section
    \ref{sec:lazard-correspondence-commutativity-relation-central-series}
    may be skimmed.
  \item The theorems in Section \ref{sec:malcev-lazard-free} are
    important as stepping stones for the main results. However, the
    proofs are unilluminative and may be skipped.
  \item The results in Sections
    \ref{sec:homology-of-powered-nilpotent-groups} and
    \ref{sec:homology-of-powered-nilpotent-lie-rings} are important, but
    the proofs may again be skipped.
  \item Sections \ref{sec:commutator-and-lie-bracket-like-maps} and
    \ref{sec:lcuti} are extremely important and should be read
    carefully, though the proofs may be skimmed.
  \end{itemize}
\item Chapter 8 (Applications and possible extensions): Sections
  \ref{sec:applications} and \ref{sec:possible-extensions} may be of
  interest to readers who want to understand potential applications.
\item Readers may refer to the sections in the Appendix based on their
  level of interest. Sections \ref{appsec:background-grouptheory},
  \ref{appsec:abstract-nonsense}, and \ref{appsec:nilpotent} cover
  technical background at the advanced undergraduate or beginning
  graduate level that is useful for understanding the main results of
  the thesis. Section \ref{appsec:homologism-theory} covers a general
  theory that is helpful for understanding potential generalizations
  of the results presented here.
\end{enumerate}

%\newpage

\section{Outline of our main results}\label{sec:outline}

This section provides an overview of our main results and the strategy
we will use to prove these results. Some of the technical details in
this section may be accessible only to people with a strong background
in group theory and some prior familiarity with the Lazard
correspondence. However, all readers should be able to understand the
ideas at a broad level.

\subsection{The Lazard correspondence: a rapid review}

The Lazard correspondence is a correspondence between certain kinds of
groups and certain kinds of Lie rings. The groups, called {\em Lazard
  Lie groups}, satisfy a condition relating the set of primes over
which they are powered and the nilpotency class of subgroups generated
by subsets of size at most three. The Lie rings, called {\em Lazard
  Lie rings}, satisfy a similar condition relating the set of primes
over which they are powered and the nilpotency class of Lie subrings
generated by subsets of size at most three. The precise definition of
the Lazard correspondence is in Section
\ref{sec:lazard-correspondence}. A somewhat easier case of the
correspondence, called the {\em global} Lazard correspondence, is
described in Section \ref{sec:global-lazard-correspondence}. The
global Lazard correspondence imposes a restriction on the nilpotency
class of the whole group and of the whole Lie ring. It is more narrow
than the Lazard correspondence but easier to deal with.

For a Lie ring $L$, the corresponding group, $\exp(L)$, has the {\em
  same underlying set} as $L$, and the group operations are defined in
terms of the Lie ring operations based on fixed formulas. Explicitly,
the group multiplication is defined in terms of the Lie ring
operations using the {\em Baker-Campbell-Hausdorff formula}. The
Baker-Campbell-Hausdorff formula is described in detail in Section
\ref{sec:bch}. Technically, the formula is different for different
values of the ($3$-local) nilpotency class, but we can use a single
infinite series whose truncations give all the formulas.

For a group $G$, the corresponding Lie ring, denoted $\log(G)$, has
the {\em same underlying set} as $G$, and the Lie ring operations are
defined in terms of the group operations based on fixed formulas
called the {\em inverse Baker-Campbell-Hausdorff formulas} (one
formula describing the Lie ring addition and another formula
describing the Lie bracket in terms of the group operations). The
inverse Baker-Campbell-Hausdorff formulas are described in Section
\ref{sec:inverse-bch}.

$\exp$ and $\log$ define functors between appropriately defined
subcategories of the category of groups and the category of Lie rings,
and the functors are two-sided inverses of each other. Thus, they
establish an isomorphism of categories over the category of
sets\footnote{This means an isomorphism of categories that preserves
  the underlying set} between the relevant subcategories of the
category of groups and the category of Lie rings.

\subsection{Isoclinism: a rapid review}

An {\em isoclinism of groups} is a pair of group isomorphisms, one
between their inner automorphism groups and the other between their
derived subgroups, that are compatible with the commutator
map. Intuitively, we can think of an isoclinism of groups as an
equivalence between the commutator structures of the two groups. We
will define and discuss isoclinisms in Section
\ref{sec:isoclinism-and-homoclinism}. 

There is a similar notion of {\em isoclinism of Lie rings} (that uses
the inner derivation Lie ring, the derived subring, and the Lie
bracket) that we will define and discuss in Section
\ref{sec:isoclinism-and-homoclinism-lie}. Intuitively, we can think of
an isoclinism of Lie rings as an equivalence between the Lie bracket
structures of the Lie rings.

We can use isoclinism of groups to define an {\em equivalence
  relation} on the collection of groups. Analogously, we can use
isoclinism of Lie rings to define an equivalence relation on the
collection of Lie rings.

\subsection{The Lazard correspondence up to isoclinism}

The Lazard correspondence up to isoclinism combines the idea of the
Lazard correspondence and the idea of isoclinism. For a Lie ring $L$
and a group $G$, a Lazard correspondence up to isoclinism includes two
pieces of data satisfying a compatibility condition:

\begin{itemize}
\item A Lazard correspondence up to isomorphism between
  $\operatorname{Inn}(L)$ and $\operatorname{Inn}(G)$. This can be
  viewed as an isomorphism of groups between
  $\exp(\operatorname{Inn}(L))$ and $\operatorname{Inn}(G)$ or as an
  isomorphism of Lie rings between $\operatorname{Inn}(L)$ and
  $\log(\operatorname{Inn}(G))$.
\item A Lazard correspondence up to isomorphism between $L'$ and
  $G'$. This can be viewed as an isomorphism of groups between
  $\exp(L')$ and $G'$ or as an isomorphism of Lie rings between $L'$
  and $\log(G')$.
\end{itemize}

The compatibility condition is tricky to specify. Naively, we might
expect that the compatibility condition would say that the isomorphism
converts the Lie bracket map $\operatorname{Inn}(L) \times
\operatorname{Inn}(L) \to L'$ to the commutator map
$\operatorname{Inn}(G) \times \operatorname{Inn}(G) \to G'$. The
problem with this naive specification is that even with the ordinary
Lazard correspondence, the Lie bracket of the Lie ring does not
coincide with the commutator of the group. They {\em do} coincide when
the class is at most two, and we discuss this special case, the {\em
  Baer correspondence up to isoclinism}, in Section \ref{sec:bcuti}.

%% For the Lazard correspondence in higher class, the Lie bracket and
%% commutator map are intimately related, but not equal. The relationship
%% between them has a flavor similar to composition with a unipotent
%% transformation. Explicitly, if the commutator of two elements in the
%% group is trivial, there Lie bracket in the Lie ring is zero, and vice
%% versa.

To handle higher class, we need to first derive a formula valid for
the usual Lazard correspondence that expresses the Lie bracket in
terms of the commutator, and in the reverse direction, we need to
derive a formula valid for the usual Lazard correspondence that
expresses the commutator in terms of the Lie bracket. The
compatibility condition we impose will make use of these formulas. The
formulas themselves are described in Section
\ref{sec:group-commutator-and-lie-bracket-ito-each-other}. The
compatibility condition based on these formulas is described in detail
in Section \ref{sec:lcuti}.


\subsection{The existence question}\label{sec:existence-question}

Defining the Lazard correspondence up to isoclinism is relatively
easy. The harder part is establishing sufficient conditions for the
existence of objects on the other side, i.e., establishing sufficient
conditions for the existence of groups that are in Lazard
correspondence up to isoclinism with a given Lie ring, and
establishing sufficient conditions for the existence of Lie rings that
are in Lazard correspondence up to isoclinism with a given group.

The results that we would like to aim for are:

\begin{itemize}

\item For a Lie ring $L$, if both $\operatorname{Inn}(L)$ and $L'$ are
  Lazard Lie rings, then we can find a group $G$ such that $L$ is in
  Lazard correspondence up to isoclinism with $G$.
\item For a group $G$, if both $\operatorname{Inn}(G)$ and $G'$ are
  Lazard Lie groups, then we can find a Lie ring $L$ such that $L$ is
  in Lazard correspondence up to isoclinism with $G$.
\end{itemize}

Unfortunately, the proofs of these statements at a general level
require more machinery than we can manage in this thesis. We
therefore restrict our proofs here to the case of the global Lazard
correspondence. The precise statements of the results we will prove
are in Section \ref{sec:lcuti}. Essentially, we restrict attention to
groups that satisfy global assumptions on the set of primes over which
they powered, and for which the inner automorphism group and derived
subgroup are both in the domain of the global Lazard correspondence.

The strategy that we use to demonstrate these facts is somewhat
roundabout. Instead of trying to answer the question directly, we try
to answer the question in the more general context of central
extensions of groups and Lie rings. We will then apply the results
that we obtain to the central extensions with short exact sequences:

$$0 \to Z(G) \to G \to G/Z(G) \to 1$$

and

$$0 \to Z(L) \to L \to L/Z(L) \to 0$$

We outline below the argument in the direction from Lie rings to groups.

We begin by viewing $L$ as an extension with central subring $Z(L)$
and quotient ring $L/Z(L) \cong \operatorname{Inn}(L)$. We obtain the
corresponding Lie bracket map $\operatorname{Inn}(L) \times
\operatorname{Inn}(L) \to L'$. We then obtain a desired commutator map
$\exp(\operatorname{Inn}(L)) \times \exp(\operatorname{Inn}(L)) \to
\exp(L')$ by using the formula describing the commutator map in terms
of the Lie bracket map. Finally, we demonstrate the existence of a
group $G$ that realizes this commutator map.

\subsection{The realization of isoclinism types}

We will show that equivalence classes of groups up to isoclinism can
be described by storing the commutator structure in an abstract
fashion, without reference to an actual group in that equivalence
class.

This will be useful to the final step of our proof of existence
established above: instead of directly trying to construct the groups
in the equivalence class up to isoclinism, we construct the commutator
structure. In the notation above, we construct the desired commutator
map $\exp(\operatorname{Inn}(L)) \times \exp(\operatorname{Inn}(L))
\to \exp(L')$.

Below, we provide a few more details about how we store the commutator
structure abstractly. This discussion may be accessible only to people
familiar either with group cohomology or with some other type of
cohomology theory that is structurally similar. Note also that the
group $G$ that we use here is not the same as the group $G$ used in
Section \ref{sec:existence-question}. In fact, to apply what we
discuss below to Section \ref{sec:existence-question}, we would need
to set the group $A$ below to $\exp(Z(L))$ and set the group $G$ below
to $\exp(L/Z(L))$.

{\em Technical details}: In Sections
\ref{sec:exteriorsquare-and-homoclinism} and
\ref{sec:schur-multiplier-and-second-cohomology}, we will show that we
can classify central extensions up to isoclinism using a homomorphism
from the Schur multiplier. Explicitly, when considering central
extensions with central subgroup $A$ and quotient group $G$, we can
determine the type of the extension up to isoclinism by considering
the induced map $M(G) \to A$ where $M(G)$ is the Schur multiplier of
$G$. We will relate this to the universal coefficient theorem short
exact sequence described in Section \ref{sec:ses-uct}.

\begin{equation*}
  0 \to \operatorname{Ext}^1_{\mathbb{Z}}(G^{\operatorname{ab}},A) \to H^2(G;A) \to \operatorname{Hom}(M(G),A) \to 0
\end{equation*}

The key aspect of the above short exact sequence that is relevant for
the existence question is the {\em surjectivity} of the map:

$$H^2(G;A) \to \operatorname{Hom}(M(G),A)$$

Thus, the homomorphism from $M(G)$ to $A$ describes the equivalence
class of extensions up to isoclinism, and every homomorphism from
$M(G)$ to $A$ describes some eqiuvalence class of extensions.

For results in the opposite direction, we develop a similar theory for
Lie rings.
\subsection{Powering assumptions}

One complication that arises in the discussion of the Lazard
correspondence and its generalizations is that the formulas involved
require taking $p^{th}$ roots for some primes $p$. Thus, in order to
make sense of these expressions, we need to develop a basic theory of
groups and Lie rings where these operations make sense. We develop
that basic theory in Sections \ref{sec:group-powering} and
\ref{sec:free-powered-groups-and-powering-functors} (for groups) and
in Sections \ref{sec:lie-ring-powering} and
\ref{sec:free-powered-lie-rings-and-powering-functors} (for Lie
rings).

\subsection{Global Lazard correspondence preserves Schur multipliers}

To complete the proof, we need to demonstrate that the global Lazard
correspondence behaves well with respect to the structures that we use
to classify extensions up to isoclinism. Explicitly, we need to show
that if $L = \log(G)$ and $G = \exp(L)$, then the Schur multipliers
$M(L)$ and $M(G)$ are canonically isomorphic, and also that the
exterior squares $L \wedge L$ and $G \wedge G$ are in Lazard
correspondence.  We will demonstrate these facts in Sections
\ref{sec:homology-of-powered-nilpotent-groups} and \ref{sec:lcuti}
(specifically, in Theorem
\ref{thm:global-lazard-correspondence-preserves-schur-multipliers}). A
version of the statement for finite $p$-groups appeared as a
conjecture in the paper \cite{SchurmultiplierandLazard} by Eick, Horn,
and Zandi in September 2012, stated informally after Theorem 2 of the
paper. Some technical details of our proof idea follow.

{\em Technical details}: The key idea behind our proof is to express
our group as a quotient group of a free powered nilpotent group of
class one more. Using a nilpotency class of one more allows us to use
a variant of the Hopf formula to calculate the Schur multiplier, as
described in \ref{sec:hopf-formula-class-one-more} and
\ref{sec:hopf-formula-pi-powered-class-one-more}. We can perform a
similar construction on the Lie ring side. We now show that the groups
used to compute the Schur multiplier of the group are in Lazard
correspondence with the Lie rings used to compute the Schur multiplier
of the Lie ring. The reason this is nontrivial is that the free
nilpotent group and free nilpotent Lie ring of class one more need not
themselves be in Lazard correspondence. We need to show that despite
this, the groups that we eventually use in the formula for computing
the Schur multiplier are in Lazard correspondence.


%\newpage

\section{The abelian Lie correspondence}\label{sec:abelian-lie-correspondence}

This section describes an obvious and straightforward correspondence:
the correspondence between abelian groups and abelian Lie rings. An
abelian Lie ring is a Lie ring that has trivial Lie bracket. Basic
definitions related to Lie rings can be found in the Appendix, Section
\ref{appsec:Lie}.

All assertions made here are trivial to prove. The purpose of this
section is to set up a basic prototype for the Lazard correspondence.

\subsection{Abelian groups correspond to abelian Lie rings}\label{sec:abelian-lie-correspondence-def}

We establish the abelian Lie correspondence:

\begin{center}
  Abelian groups $\leftrightarrow$ Abelian Lie rings
\end{center}

The correspondence works as follows.

\begin{itemize}
\item From groups to Lie rings: Given an abelian group $G$, the
  corresponding abelian Lie ring $\log G$ is defined as the Lie ring whose
  underlying additive group coincides with $G$, and where the Lie
  bracket is trivial.
\item From Lie rings to groups: Given an abelian Lie ring $L$, the
  corresponding abelian group $\exp L$ is defined as the underlying
  additive group of $L$.
\end{itemize}

Note that the symbols $\exp$ and $\log$ here are being used as
abstract symbols. They do not describe exponential and logarithm
maps in the conventional sense of the term. The relationship with the
usual notions of exponential and logarithm will become clearer in
subsequent sections leading up to the definition of the Lazard
correspondence.

\subsection{Preservation of homomorphisms: viewing $\exp$ and $\log$ as functors}\label{sec:abelian-lie-correspondence-homomorphism-preservation}

The following observations follow immediately from the definitions:

\begin{itemize}
\item $\log$ defines a functor from abelian groups to abelian Lie
  rings: Suppose $G_1$ and $G_2$ are abelian groups and $\varphi:G_1
  \to G_2$ is a group homomorphism. Then, there exists a unique Lie
  ring homomorphism $\log(\varphi): \log(G_1) \to \log(G_2)$ that has
  the same underlying set map as $\varphi$.
\item $\exp$ defines a functor from abelian Lie rings to abelian
  groups: Suppose $L_1$ and $L_2$ are abelian Lie rings and $\varphi:L_1
  \to L_2$ is a Lie ring homomorphism. Then, there exists a unique
  group homomorphism $\exp(\varphi): \exp(L_1) \to \exp(L_2)$ that has
  the same underlying set map as $\varphi$.
\item The $\log$ and $\exp$ functors are two-sided inverses of each
  other: This assertion has four parts:
  \begin{itemize}
    \item For every abelian group $G$, $G = \exp(\log(G))$. 
    \item For every abelian Lie ring $L$, $L = \log(\exp(L))$.
    \item For every group homomorphism $\varphi:G_1 \to G_2$ of
      abelian groups, $\exp(\log(\varphi)) = \varphi$.
    \item For every Lie ring homomorphism $\varphi:L_1 \to L_2$ of
      abelian Lie rings, $\log(\exp(\varphi)) = \varphi$.
  \end{itemize}
\end{itemize}

The upshot of these is that the category of abelian groups and the
category of abelian Lie rings are isomorphic categories, with the
$\log$ and $\exp$ functors providing the isomorphisms.
\subsection{Isomorphism over Set}

Consider the following two categories:

\begin{itemize}
\item The category of abelian groups, with the forgetful functor to
  the category of sets that sends each abelian group to its underlying
  set.
\item The category of abelian Lie rings, with the forgetful functor to
  the category of sets that sends each abelian Lie ring to its
  underlying set.
\end{itemize}

The correspondence we established above (in Sections
\ref{sec:abelian-lie-correspondence-def} and
\ref{sec:abelian-lie-correspondence-homomorphism-preservation})
establishes an {\em isomorphism of categories over Set} between the
two categories. There are two parts to this statement:

\begin{itemize}
\item The correspondence establishes an isomorphism between the
  category of abelian groups and the category of abelian Lie rings:
  The functor in the direction from groups to Lie rings is the $\log$
  functor. The functor in the direction from Lie rings to groups is
  the $\exp$ functor. The details are in the preceding section
  (Section
  \ref{sec:abelian-lie-correspondence-homomorphism-preservation}).
\item This isomorphism has the property that applying it and then
  applying the forgetful functor to the category of sets gives the
  same result as directly applying the forgetful functor to the
  category of sets. This is a category-theoretic way of saying that
  the abelian group and abelian Lie ring have the same underlying set,
  and that the set maps that are group homomorphisms are precisely the
  same as the set maps that are Lie ring homomorphisms.
\end{itemize}

\subsection{Equality of endomorphism monoids and of automorphism groups}\label{sec:abelian-lie-correspondence-aut-end}

Suppose $L$ is an abelian Lie ring and $G = \exp(L)$, so that $L =
\log(G)$. The functors $\exp$ and $\log$ are isomorphisms of
categories, hence they induce isomorphisms between the endomorphism
monoids. Further, since these isomorphisms of categories preserve the
underlying set, the isomorphism between the endomorphism monoids sends
each Lie ring endomorphism to a corresponding group endomorphism that
is {\em the same as a set map}. Explicitly, the map
$\exp:\operatorname{End}(L) \to \operatorname{End}(G)$ is an
isomorphism. Further, for $\varphi \in \operatorname{End}(L)$, the
corresponding map $\exp(\varphi) \in \operatorname{End}(G)$ coincides
with $\varphi$ as a set map. The isomorphism induced by $\exp$ between
the endomorphism monoids $\operatorname{End}(L)$ and
$\operatorname{End}(G)$ restricts to an isomorphism between the
automorphism groups $\operatorname{Aut}(L)$ and
$\operatorname{Aut}(G)$.

\subsection{The correspondence up to isomorphism}\label{sec:abelian-lie-correspondence-up-to-isomorphism}

We have so far considered the correspondence at the level of
individual groups and Lie rings:

\begin{center}
  Abelian groups $\leftrightarrow$ Abelian Lie rings
\end{center}

The correspondence defines an isomorphism of categories, and thus it
descends to a correspondence between equivalence classes up to
isomorphism on both sides, giving a correspondence:

\begin{center}
  Isomorphism classes of abelian groups $\leftrightarrow$ Isomorphism classes of abelian Lie rings
\end{center}

Suppose $L$ is an abelian Lie ring and $G$ is an abelian
group. Specifying an abelian Lie correspondence {\em up to
  isomorphism} between $L$ and $G$ amounts to specifying one of the
following two equivalent pieces of data:

\begin{itemize}
\item An isomorphism of groups from $\exp L$ to $G$.
\item An isomorphism of Lie rings from $\log G$ to $L$.
\end{itemize}

A common convention used to provide this data is to provide one of these:

\begin{itemize}
\item A {\em set} map $\exp:L \to G$ that, viewed as a set map from
  $\exp(L)$ to $G$, becomes a group isomorphism.
\item A {\em set} map $\log:G \to L$ that, viewed as a set map from
  $\log(G)$ to $L$, becomes a Lie ring isomorphism.
\end{itemize}

In other words, we can specify the data in the form of one of these set maps:

$$\exp: L \to G, \log: G \to L$$

The set maps $\log$ and $\exp$ are two-sided inverses of each other.

It will turn out, later, that actual exponential and logarithm maps,
with the usual power series expansions, occurring inside an
associative ring, provide examples of an abelian Lie correspondence up
to isomorphism.

In cases where we want to emphasize that we are talking of the abelian
Lie correspondence and not the abelian Lie correspondence up to
isomorphism, we will talk of the {\em strict} abelian Lie
correspondence. In this section, our focus will be on the strict
abelian Lie correspondence because that provides for an easier way to
formulate our statements.

\subsection{Isomorphism of categories versus equivalence of categories}

When discussing what it means for two categories to be essentially the
same, category theorists typically rely on a weaker notion than
isomorphism of categories. An {\em equivalence of categories}
$\mathcal{C}$ and $\mathcal{D}$ and a pair of functors $\mathcal{F}:\mathcal{C}
\to \mathcal{D}$ and $\mathcal{G}:\mathcal{D} \to \mathcal{C}$ along with
natural isomorphism $\varepsilon:\mathcal{F} \circ \mathcal{G} \to
\operatorname{Id}_{\mathcal{D}}$ and $\eta:\mathcal{G} \circ \mathcal{F} \to
\operatorname{Id}_{\mathcal{C}}$. Two categories $\mathcal{C}$ and
$\mathcal{D}$ are said to be equivalent if there exists an equivalence
of categories between them. An alternative characterization is that
two categories $\mathcal{C}$ and $\mathcal{D}$ are equivalent if there
exists a functor $\mathcal{F}: \mathcal{C} \to \mathcal{D}$ such that $\mathcal{F}$ is
full, faithful, and essentially surjective. Here, {\em essentially
  surjective} means that for every object $B \in \mathcal{D}$, there
exists $A \in \mathcal{C}$ such that $\mathcal{F}(A)$ is isomorphic to $B$.

The difference between the definitions of isomorphism of categories
and equivalence of categories arises from the distinction between a
functor being bijective (in the sense that {\em every} object is in
the image of the functor and has a unique pre-image under the functor)
and the functor being essentially surjective (in the sense that every
object is {\em isomorphic} to an object in the image of the
functor). Equivalence of categories is a more robust and useful notion
because it is less sensitive to how strictly we define equality of
objects. Thus, even though the correspondences we define are
isomorphisms of categories over the category of sets, it will often be
more helpful to think of them as equivalences of categories.

Note that any equivalence of categories establishes a bijective
correspondence between isomorphism classes of objects in the two
categories (in this case, the two categories are respectively the
category of abelian groups and the category of abelian Lie
rings). However, the equivalence of categories also includes
additional data that allows us to identify homomorphism sets on both
sides (in this case, identify abelian group homomorphisms with abelian
Lie ring homomorphisms).

\subsection{Subgroups, quotients, and direct products}\label{sec:abelian-lie-correspondence-sub-quot-dp}

The collection of abelian groups is a subvariety of the variety of
groups (see the Appendix, Section \ref{appsec:univalg-basic} for the
definition of variety). There are three parts to this assertion:

\begin{itemize}
\item Every subgroup of an abelian group is abelian.
\item Every quotient group of an abelian group is abelian.
\item A direct product of (finitely or infinitely many) abelian groups
  is abelian.
\end{itemize}

Similarly, the collection of abelian Lie rings is a subvariety of the
variety of Lie rings. There are three parts to this assertion:

\begin{itemize}
\item Every subring of an abelian Lie ring is abelian.
\item Every quotient ring of an abelian Lie ring is abelian.
\item A direct product of (finitely or infinitely many) abelian Lie
  rings is abelian.
\end{itemize}

A natural question is whether the abelian Lie correspondence behaves
nicely with respect to taking subalgebras (subgroups and subrings
respectively), quotient algebras (quotient groups and quotient rings
respectively), and direct products. The answer is {\em
  yes}. Specifically, the following are true:

\begin{itemize}
\item {\em Subgroups correspond to subrings}: Suppose an abelian Lie
  ring $L$ is in abelian Lie correspondence with an abelian group $G$,
  i.e., $L = \log(G)$ and $G = \exp(L)$. Then, for every subgroup $H$
  of $G$, $\log(H)$ is a subring of $L$, and the inclusion map of
  $\log(H)$ in $L$ is obtained by applying the $\log$ functor to the
  inclusion map of $H$ in $G$. In the opposite direction, for every
  subring $M$ of $L$, $\exp(M)$ is a subgroup of $G$, and the
  inclusion map of $\exp(M)$ in $G$ is obtained by applying the $\exp$
  functor to the inclusion map of $M$ in $L$. The abelian Lie
  correspondence thus gives rise to a correspondence:

  \begin{center}
    Subgroups of $G$ $\leftrightarrow$ Subrings of $L$
  \end{center}

\item {\em Quotient groups correspond to quotient rings}: Suppose an
  abelian Lie ring $L$ is in abelian Lie correspondence with an
  abelian group $G$. Then, for every normal subgroup $H$ of
  $G$,\footnote{Note that since $G$ is abelian, every subgroup is
    normal. However, we deliberately state the result in this fashion
    so that parallels with later generalizations are clearer.}
  $\log(G/H)$ is a quotient Lie ring of $L$, and the quotient map $L
  \to \log(G/H)$ is obtained by applying the $\log$ functor to the
  quotient map $G \to G/H$. In the opposite direction, for every ideal
  $I$ of $L$, $\exp(L/I)$ is a quotient group of $G$, and the quotient
  map $G \to \exp(L/I)$ is obtained by applying the $\exp$ functor to
  the quotient map $L \to L/I$. The abelian Lie correspondence thus
  gives rise to correspondences:

  \begin{center}
    Normal subgroups of $G$ $\leftrightarrow$ Ideals of $L$
  \end{center}
  
  \begin{center}
    Quotient groups of $G$ $\leftrightarrow$ Quotient rings of $L$
  \end{center}

\item {\em Direct products correspond to direct products}: Suppose $I$
  is an indexing set, and $G_i, i \in I$ is a collection of abelian
  groups. For each $i \in I$, let $L_i = \log(G_i)$. Then, the
  external direct product $\prod_{i \in I} L_i$ is in abelian Lie
  correspondence with the external direct product $\prod_{i \in I}
  G_i$. Moreover, the projection maps from the direct product to the
  individual factors are in abelian Lie correspondence. Also, the
  inclusion maps of each direct factor in the direct product are in
  abelian Lie correspondence.
\end{itemize}

\subsection{Characteristic and fully invariant}

Suppose an abelian group $G$ is in abelian Lie correspondence with an
abelian Lie ring $L$. In Section
\ref{sec:abelian-lie-correspondence-aut-end}, we saw that $G$ and $L$
have the same automorphism group as each other and the same
endomorphism monoid as each other (where ``same'' here means that the
actions agree on the underlying set). In Section
\ref{sec:abelian-lie-correspondence-sub-quot-dp}, we saw that the
abelian Lie correspondence induces a bijective correspondence between
subgroups of $G$ and subrings of $L$. Combining these ideas, we obtain
two additional bijective correspondences:

\begin{center}
  Characteristic subgroups of $G$ $\leftrightarrow$ Characteristic
  subrings of $L$
\end{center}

\begin{center}
  Fully invariant subgroups of $G$ $\leftrightarrow$ Fully invariant
  subrings of $L$
\end{center}

Here, {\em characteristic} means invariant under all automorphisms and
{\em fully invariant} means invariant under all endomorphisms.

\subsection{How the template will be reused}\label{sec:isocat-template}

The steps that we have outlined above will be used to construct and
study a number of similar correspondences. The steps will be as
follows:

\begin{itemize}
\item We will describe a way of writing group operations in terms of
  Lie ring operations and a way of describing Lie ring operations in
  terms of group operations, such that the formulas used satisfy the
  axioms for groups and Lie rings by definition, and such that the
  formulas are inverses of each other.
\item We will then use this to construct a correspondence that defines
  an isomorphism over the category of sets between a full subcategory
  of the category of groups and a full subcategory of the category of
  Lie rings. We will use $\log$ to denote the functor from the group
  side to the Lie ring side, and $\exp$ to denote the functor from the
  Lie ring side to the group side.
\item We will deduce that if a group and Lie ring are in
  correspondence, then their endomorphism monoids are naturally
  isomorphic, and their automorphism groups are naturally isomorphic.
\item The correspondence can be weakened to a correspondence between
  isomorphism classes in the full subcategories.
\item An instance of the correspondence up to isomorphism between a
  group $G$ and a Lie ring $L$ can be described by specifying the
  isomorphism from $\log(G)$ to $L$ or by specifying the isomorphism
  from $\exp(L)$ to $G$. We will describe the correspondence in terms
  of the set map $\log:G \to L$ or, equivalently, the set map $\exp:L
  \to G$.
\item Of the results in Section
  \ref{sec:abelian-lie-correspondence-sub-quot-dp}, the results for
  the direct product generalizes for each correspondence (note that
  this does not follow category-theoretically, but rather, it follows
  from the nature of the correspondence). The results for the
  correspondence between subgroups and subrings and the correspondence
  between quotient groups and quoitent rings generalize but only after
  we impose restrictions on the types of subgroups, subrings, quotient
  groups, and quotient rings under consideration.
\end{itemize}

For brevity, we will not repeat these steps in every instance. Rather,
our focus will be on the first step: establishing that the formulas
used make sense, satisfy the axioms, and are inverses of each other.
