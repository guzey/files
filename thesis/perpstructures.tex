\documentclass[a4paper]{amsart}

%Packages in use
\usepackage{fullpage, hyperref, vipul, diagrams}


%Title details
\title{Exponentials and the log category}
\author{Vipul Naik}
\thanks{\copyright Vipul Naik, Ph.D. student, University of Chicago}

%List of new commands
\newcommand{\logcategory}[1]{\textsc{Log-Category}\left(#1\right)}
\newcommand{\enrichedlogcategory}[2]{\textsc{Enriched-Log-Category}\left(#1 \, ; \, #2\right)}
\newcommand{\characteristiclogcategory}[1]{\textsc{Characteristic-Log-Category}\left(#1 \right)}
\newcommand{\aut}[1]{\text{Aut}\left(#1\right)}
\newcommand{\inn}[1]{\text{Inn}\left(#1\right)}
\makeindex

\begin{document}
\maketitle
%\tableofcontents

\begin{abstract}
  In this write-up, I jot down some ideas I have about a ``log
  category'', a category such that the exponential maps we usually
  encounter in the theory of Lie algebras actually become
  homomorphisms between objects of the category. 
\end{abstract}

\section{Groups with perp-structure}

\subsection{Definition of  $\perp$-structure}

We begin with the notion of a perp-structure on a group.

\begin{definer}[Group with perp-structure]
  A \definedind{perp-structure}, or \definedind{$\perp$-structure} on
  a group $G$ is a symmetric binary relation $\perp$ on $G$ such that for any
  $a \in G$, the set of $b$ such that $a \perp b$, is a subgroup of $G$.
\end{definer}

Given two structures $\perp$ and $\perp'$,we say that $\perp$ is coarser
than $\perp'$ if $a \perp b \implies a \perp' b$.

\begin{definer}[$\perp$-morphism, weak $\perp$-morphism]
  Let $G$ and $H$ be groups with $\perp$-structures. A
  \definedind{perp-morphism} from $G$ to $H$, also called a
  $\perp$-morphism from $G$ to $H$, is a group homomorphism $f: G \to
  H$ such that if $a, b \in G$ are such that $a \perp b$, then $f(a)
  \perp f(b)$.

  A \definedind{weak perp-morphism} from $G$ to $H$ is a map $f: G \to
  H$ (not necessarily a group homomorphism) such that if $a, b \in G$
  are such that $a \perp b$, then $f(ab) = f(a)f(b)$ and $f(a) \perp f(b)$.
\end{definer}

We can form some categories:

\begin{enumerate}

\item The category whose objects are groups with perp structure, and
  whose morphisms are perp-morphisms.

\item The category whose objects are groups with perp structure, and
  whose morphisms are weak perp-morphisms.

\end{enumerate}

\subsection{Some typical examples of $\perp$-structures}

Given an abstract group $G$, there are the following natural
$\perp$-structures on $G$:

\begin{definer}[Natural $\perp$-structures on a group]
  The following are some natural $\perp$-structures on a group $G$:

  \begin{itemize}

  \item The \definedind{trivial structure} is the finest possible
    $\perp$-structure: $a \perp b$ for all $a,b \in G$.

  \item The \definedind{discrete structure} is the coarsest possible
    $\perp$-structure: $a \perp b$ iff one of $a$ and $b$ is the
    identity element of $G$.

  \item The \definedind{reflexivity structure} is the coarsest
    $\perp$-structure such that $a \perp a $ for all $a \in G$.  Under
    the reflexivity structure, whenever $a$ and $b$ generate a cyclic
    subgroup, $a \perp b$. However, it is not in general true that $a
    \perp b$ implies that $a$ and $b$ generate a cyclic subgroup. To
    see why this is not true, observe that it may happen that $\langle
    a , b \rangle$ and $\langle a, c\rangle$ may be cyclic but
    $\langle a, bc \rangle$ is not cyclic.

  \item The \definedind{commutation structure} is defined as follows:
    $a \perp b \iff ab = ba$. Under the commutation structure, $a
    \perp b$ iff the subgroup generated by $a$ and $b$ is Abelian.
    The commutation structure is in general finer than the reflexivity

  \end{itemize}
\end{definer}

A $\perp$-structure is termed
\adefinedproperty{perp-structure}{reflexive} if it is finer than the
reflexivity structure; in other words $a \perp a$ for any $a$. Many of
the $\perp$-structures we shall encounter are reflexive (in fact, all
of the above, except the discrete structure, are reflexive).

Here are some properties for $\perp$-structures:

\begin{definer}[Properties for $\perp$-structures]
  A $\perp$-structure on a group $G$ is defined to be:

  \begin{itemize}

  \item \adefinedproperty{perp-structure}{conjugation-invariant} or
    \adefinedproperty{perp-structure}{normal} if $a \perp b \implies
    gag^{-1} \perp gbg^{-1}$ for any $a,b,g \in G$.

  \item \adefinedproperty{perp-structure}{characteristic} or
    \adefinedproperty{perp-structure}{automorphism-invariant} if $a
    \perp b \implies \varphi(a) \perp \varphi(b)$ for any $a,b \in G$
    and any $\varphi \in \aut(G)$.

  \item \adefinedproperty{perp-structure}{fully invariant} or
    \adefinedproperty{perp-structure}{endomorphism-invariant} if $a
    \perp b \implies \varphi(a) \perp \varphi(b)$ for any $a,b \in G$
    and any endomorphism $\varphi$ of $G$.

  \end{itemize}
\end{definer}

All the four $\perp$-structures defined above are characteristic, and
are in fact fully invariant.

\subsection{Relation between category of groups and perp-category}

Here are the relationships at the functor level:

\begin{itemize}

\item There is a full and faithful functor from the category of groups
  to the perp-category, that sends any group to the same group, with
  $\perp$-structure being the commutation structure.

\item The $\perp$-category is a subcategory of the weak
  $\perp$-category.

\item Composing the full and faithful functor from groups to the
  $\perp$-category, with the embedding of the $\perp$-category in the
  weak $\perp$-category, gives a faithful functor that is no longer
  necessarily full.
\end{itemize}

\section{Perp-structures on Abelian groups}

In this section, we restrict ourselves to perp-structures on Abelian
groups. First, we describe some natural ``sources'' of
perp-structures. Next, we look at some specific Abelian groups, and
classify all the perp-structures on them.

\subsection{Perp structure coming from a symmetric or alternating bilinear map}

Suppose $A$ and $B$ are Abelian groups, and:

$$b: A \times A \to B$$

is a biadditive map. Then, we can define:

$$a_1 \perp a_2 \iff b(a_1,a_2) = 0$$

Clearly, for any $a_1 \in A$, the set of $a_2$ such that $a_1 \perp
a_2$, is a subgroup. However, for arbitrary $b$, the relation $\perp$
need not be symmetric. There are two special cases:

\begin{itemize}

\item When $b$ is symmetric, the relation $\perp$ is
  symmetric. However, $\perp$ is not reflexive (in fact, it is
  reflexive iff it is trivial). Thus, we get a (not reflexive)
  $\perp$-structure on the group $A$.

\item When $b$ is alternating, the relation $\perp$ is both reflexive
  and symmetric. Thus, we get a reflexive $\perp$-structure on the
  group $A$.

\end{itemize}

This raises questions: does every perp structure arise from a
symmetric or alternating bilinear form? And does every reflexive perp
structure arise from an alternating bilinear form? We'll be looking at
those questions a little later.

\subsection{Some special cases of perp structures}

It's now time to consider some special kinds of
$\perp$-structures. These can be viewed as the real motivation behind
the theory of $\perp$-structures.

\begin{enumerate}

\item Suppose $R$ is a noncommutative (and not necessarily
  associative or unital) ring. Consider $R$ as an additive group, and for
  $x,y\in R$, define $x \perp y$ iff $xy = yx$. This gives a reflexive
  $\perp$-structure.

  The alternating bilinear map here is the map $R \times R \to R$
  given by $(x,y) \mapsto xy - yx$. It is alternating because of the
  distributivity laws.

\item Suppose $L$ is a Lie ring. Consider $L$ as an additive group,
  and for $x,y \in L$ define $x \perp y \iff [x,y] = 0$. This gives a
  reflexive $\perp$-structure.

  The alternating bilinear map here is the Lie bracket $[ \ , \ ]$.

\item Suppose $V$ is a vector space with an alternating bilinear form $b$.
  (i.e., a symplectic vector space). Then, consider $V$ as an additive
  group, and for $x,y \in V$, define $x \perp y \iff b(x,y) = 0$.

  This is a special case where the bilinear map is a bilinear form,
  i.e. it takes values in the base field.
\end{enumerate}

\end{document}
