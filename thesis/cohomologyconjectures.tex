\documentclass[10pt]{amsart}

%Packages in use
\usepackage{fullpage, hyperref, vipul, amssymb, graphicx}

%Title details
\title{Group cohomology and halving conjectures}
\author{Vipul Naik}
\thanks{Based on joint work with John Wiltshire-Gordon}

%List of new commands
\newcommand{\Skew}{\operatorname{Skew}}
\makeindex

\begin{document}
\maketitle
%\tableofcontents

The purpose of this document is to flesh out, in clear cohomology
language, some of the main conjectures that arise when trying to find
Lazard Lie rings of 2-groups of class two. Once the conjectures are
stated purely in the language of cohomology, they become susceptible
to methods of cohomology.

\section{The main conjectures and statements related to cocycles}

\subsection{Setup and notation}

We are given two abelian groups $G$ and $A$ with $G$ acting trivially
on $A$. Note that the abelianness of $G$ is perhaps not essential to
everything we do here. For this reason and also for clarity when
dealing with group actions, we use multiplicative notation for $G$ and
additive notation for $A$.

For convenience, we denote the identity element of $G$ as $1$. Also,
we use the left action convention since this is more common when
dealing with cohomology. Note that this does not affect matters much
in the final analysis because the action is trivial anyway.

Throughout this document, whenever we speak of 2-cocycles, we always
assume the action to be trivial.

\subsection{IIP 2-cocycles and CP 2-cocycles}

We are considering functions $f: G \times G \to A$ and we are
interested in the following conditions:

\begin{enumerate}
\item $f$ is a 2-cocycle for the trivial group action. In other words,
  for all $x,y,z \in G$, we have:

  $$f(x,yz) + f(y,z) = f(xy,z) + f(x,y)$$

\item $f$ is {\em identity-preserving}. In other words, $f(x,1) =
  f(1,x) = 0$ for all $x \in G$. (In the context of group extensions,
  identity-preserving 2-cocycles are also called normalized
  2-cocycles, because they correspond to normalized choices of coset
  representatives. Since we want to avoid thinking in terms of group
  extensions, I use this different term).
\item $f$ is {\em inverse-preserving}: In other words, $f(x,x^{-1}) =
  0$ for all $x \in G$.
\item $f$ is {\em cyclicity-preserving}: In other words, $f(x,y) = 0$
  whenever $\langle x,y \rangle$ is a cyclic subgroup of $G$.
\item $f$ is {\em skew-symmetric}: In other words, $f(x,y) = -f(y,x)$
  for all $x,y \in G$.
\end{enumerate}

Note that condition (4) implies conditions (2) and (3). The converse
is true when $G$ is an elementary abelian $2$-group.

Note also that condition (4) does {\em not} imply condition (5) in the
general situation, though it does when $f$ is a bihomomorphism (see
the next subsection for a discussion of bihomomorphisms). However, (4)
and (5) together imply that $f$ is alternating: $f(g,g) = 0$ and
$f(g,h) = -f(h,g)$ for all $g,h \in G$.

Each of these conditions cuts out a subgroup of the group of all
functions $G \times G \to A$ under pointwise addition. There are two
important notions that we consider by combining these conditions:

\begin{enumerate}
\item An {\em IIP 2-cocycle} is a 2-cocycle that is both
  identity-preserving and inverse-preserving, i.e., conditions (1) --
  (3) are satisfied.
\item A {\em CP 2-cocycle} is a 2-cocycle that is
  cyclicity-preserving. Equivalently, it satisfies all of conditions
  (1) -- (4).
\item A {\em skew-symmetric IIP 2-cocycle} is an IIP 2-cocycle that is
  also skew-symmetric.
\item A {\em skew-symmetric CP 2-cocycle} is a CP 2-cocycle that is
  also skew-symmetric.
\end{enumerate}

The IIP 2-cocycles form a subgroup of the group of all functions $G
\times G \to A$ under pointwise addition. The CP 2-cocycles form a
further subgroup of that. In the case that $G$ is an elementary
abelian 2-group, these two subgroups coincide. Skew symmetry brings us
down to potentially smaller subgroups.


\subsection{Bihomomorphisms and alternating bihomomorphisms}

A function $f: G \times G \to A$ is termed a {\em bihomomorphism} if,
for any fixed $x \in G$, the functions $y \mapsto f(x,y)$ and $y
\mapsto f(y,x)$ are homomorphisms from $G$ to $A$. This is a
generalization of the notion of bilinear map. Note that we are
assuming $G$ and $A$ to be abelian. If $G$ were not abelian, the
bihomomorphism would descend to a bihomomorphism from the
abelianization of $G$.

Here are some important facts about bihomomorphisms:

\begin{enumerate}
\item A bihomomorphism $f:G \times G \to A$, for $A$ abelian, is
  automatically an identity-preserving 2-cocycle for the trivial
  action of $G$ on $A$.
\item For a bihomomorphism, being inverse-preserving is equivalent to
  being cyclicity-preserving, which is equivalent to being
  alternating: If $f(x,x^{-1}) = 0$, then $f(x,x) = 0$ as well by the
  bihomomorphism property. We can now use the bihomomorphism property
  to prove that $f(x,y) = -f(y,x)$ -- the proof being essentially the
  same as the linear algebra proof (use $f(x,y) + f(y,x) = f(x+y,x+y)
  - f(x,x) - f(y,y)$).
\end{enumerate}

We shall use the term {\em alternating bihomomorphism} for a
bihomomorphism that satisfies the equivalent conditions in (2). The
group of alternating bihomomorphisms is thus a subgroup of the group
of skew-symmetric CP 2-cocycles.

Note that a bihomomorphism $f:G \times G \to A$ is equivalent to a
homomorphism $G \otimes_\Z G \to A$. An alternating bihomomorphism $f:
G \times G \to A$ is equivalent to a homomorphism $\bigwedge^2 G \to
A$.

The inclusions that we have discussed so far are summarized in the figure below.

\includegraphics[width=7in]{cocyclesubgroupinclusions.png}

\subsection{Doubles and skews}

Suppose $f:G \times G \to A$ is a function. We define the skew of $f$,
denoted $\Skew f$, as a new function $G \times G \to A$ defined as:

$$(\Skew f)(x,y) = f(x,y) - f(y,x)$$

Here are some petty observations:

\begin{itemize}
\item If $g = \Skew f$, then $g$ is an {\em alternating function}:
  $g(x,x) = 0$ and $g(x,y) = -g(y,x)$. {\em Please note: Most of the
  functions of interest to us are not bilinear, so both aspects of
  alternating -- $g(x,x) = 0$ and $g(x,y) = -g(y,x)$, need to be
  stated.}
\item $\Skew f = 0$ if and only if $f$ is a {\em symmetric function}:
  $f(x,y) = f(y,x)$ for all $x,y \in G$.
\item $\Skew f = 2f$ if and only if $f$ is a {\em skew-symmetric
  function}: $f(x,y) = -f(y,x)$ for all $x,y \in G$.
\end{itemize}

\subsection{The main skewing result}

\begin{lemma}[Skew of 2-cocycle is bihomomorphism]
  Suppose $f$ is a 2-cocycle from $G$ to $A$ for the trivial
  action. Then, if $G$ is abelian, $\Skew f$ is a bihomomorphism.
\end{lemma}

\begin{proof}
  Since $f$ is a 2-cocycle from $G$ to $A$, we have:

  \begin{equation}
    f(x,yz) + f(y,z) = f(xy,z) + f(x,y) \label{1}
  \end{equation}

  Since this is true for all $x,y \in G$, the corresponding statement
  with $x$ and $y$ interchanged is also true, yielding:

  \begin{equation}
    f(y,xz) + f(x,z) = f(yx,z) + f(y,x) \label{2}
  \end{equation}

  Starting with the original expression and interchanging $y$ and $z$,
  we get:

  \begin{equation}
    f(x,zy) + f(z,y) = f(xz,y) + f(x,z) \label{3}
  \end{equation}
  
  Since $G$ is abelian, $xy = yx$ and $xz = zx$. Do \ref{1} - \ref{2}
  - \ref{3} to get:

  \begin{equation*}
    f(y,z) - f(y,xz) - f(x,z) - f(z,y) = f(x,y) - f(y,x) - f(xz,y) - f(x,z)
  \end{equation*}

  Rearranging, we get:

  \begin{equation*}
    [f(y,z) - f(z,y)] + [f(y,x) - f(x,y)] = f(y,xz) - f(xz,y)
  \end{equation*}

  which is equivalent to:

  \begin{equation*}
    (\Skew f)(y,z) + (\Skew f)(y,x) = (\Skew f)(y,xz)
  \end{equation*}

  Thus, for every fixed value of $y$, $x \maspto \Skew f(y,x)$ is a
  homomorphism from $G$ to $A$. Because of the left-right symmetry in
  the definition of 2-cocycle for trivial aciton, we can similarly
  show that the map is a 2-homomorphism in the first coordinate.
\end{proof}

\section{The question of interest}

We have shown in the previous subsection that the skew of any
2-cocycle on an abelian group is a bihomomorphism. What we are
interested in is the question: does every bihomomorphism arise as the
skew of a 2-cocycle? More specifically, we are interested in whether
every bihomomorphism arises as the skew of an IIP 2-cocycle.

\appendix

\section{Old stuff}

\subsection{The halving conjectures}

The general format of the halving conjectures is as follows:

\begin{conjecture}[$\alpha \to \beta$ halving conjecture]
  Suppose $c: G \times G \to A$ is a 2-cocycle satisfying condition
  $\alpha$. Suppose, further, that for all $x,y \in G$, there exists
  $a \in A$ such that $c(x,y) = 2a$. Then, we can find a 2-cocycle $f$
  satisfying $\beta$ such that $2f = c$.
\end{conjecture}

We now state what's known about the halving conjectures and what would
follow if they were true:

\begin{enumerate}
\item The $\alpha \to \alpha$ halving conjecture holds when $A$ is a
  uniquely $2$-divisible group for each of the following conditions
  $\alpha$ separately: alternating bihomomorphism, bihomomorphism,
  skew-symmetric cyclicity-preserving 2-cocycle, skew-symmetric IIP
  2-cocycle, cyclicity-preserving 2-cocycle, IIP 2-cocycle,
  identity-preserving 2-cocycle, and 2-cocycle. Moreover, the solution
  $f$ is {\em unique} in each case. In the case that $A$ has bounded
  exponent $d$ (which is odd), we can in fact define $f = \frac{d +
  1}{2}c$.

  More explicitly, if $c:G \times G \to A$ is a function, there
  exists a unique function $f: G \times G \to A$ such that $2f = c$. Moreover:
  \begin{itemize}
  \item If $c$ is an alternating bihomomorphism, so is $f$.
  \item If $c$ is a bihomomorphism, so is $f$.
  \item If $c$ is a skew-symmetric cyclicity-preserving 2-cocycle, so is $f$.
  \item If $c$ is a skew-symmetric IIP 2-cocycle, so is $f$.
  \item If $c$ is a cyclicity-preserving 2-cocycle, so is $f$.
  \item If $c$ is an IIP 2-cocycle, so is $f$.
  \item If $c$ is an identity-preserving 2-cocycle, so is $f$.
  \item if $c$ is a 2-cocycle, so is $f$.
  \end{itemize}
\item The alternating bihomomorphism $\to$ alternating bihomomorphism
  halving conjecture fails in many cases where both $G$ and $A$ are
  finite $2$-groups. For instance, set $A = \Z_4$ and $G = \Z_2 \times
  \Z_2$ and consider the alternating bihomomorphism that sends any
  pair of linearly independent inputs in $G$ to the unique element of
  order $2$ in $A$. This can be locally halved but not globally
  halved.
\item The following halving conjecture, if true, would be enough for
  the widespread existence of Lazard Lie rings: (alternating
  bihomomorphism $\to$ skew-symmetric IIP 2-cocycle). A somewhat
  stronger halving conjecture, namely (alternating bihomomorphism
  $\to$ skew-symmetric CP 2-cocycle), would be enough for the
  existence of Lazard Lie rings where the correspondence is a
  1-isomorphism.
\item The following even stronger halving conjectures are open and are
  false, as we shall see later using the Klein four-group: $\alpha \to
  \alpha$ for $\alpha$ being: skew-symmetric CP 2-cocycle,
  skew-symmetric IIP 2-cocycle.
\end{enumerate}

\section{The case of elementary abelian $2$-groups}

\subsection{Collapse}

In this section, we restrict attention to the case where $G$ is an
elementary abelian $2$-group. This case is more tractable than the
other cases for two reasons:

\begin{enumerate}
\item IIP and CP become equivalent conditions, because $x = x^{-1}$
  and the only other power of $x$ is the identity element.
\item Skew symmetry becomes redundant since it turns out to follow
  from IIP and the $2$-cocycle condition (we shall elaborate on this
  in the next subsection).
\end{enumerate}

Thus, the four conditions: IIP, CP, skew-symmetric IIP, and
skew-symmetric CP, all become equivalent for $2$-cocycles when $G$ is
an elementary abelian $2$-group.

\subsection{Solving it for Klein four-groups}

We begin with the first simple lemma.

\begin{lemma}[IIP $2$-cocycle on Klein four-subgroups]
  Suppose $G$ is an elementary abelian $2$-group and $A$ is an abelian
  group. Suppose $f: G \times G \to A$ is an IIP $2$-cocycle for the
  trivial action. Suppose $x$ and $y$ are two distinct non-identity
  elements of $G$, so that $\langle x,y \rangle$ is a Klein
  four-group. Let $z = xy$. Then $f(x,y) = f(y,z) = f(z,x)$, $f(y,x) =
  f(z,y) = f(x,z)$, and $f(x,y) = -f(y,x)$. In particular, knowing
  $f(x,y)$ determines all the other values. In particular, $f$ is
  skew symmetric.

  Further, we have $4f(x,y) = 0$ for all $x,y \in G$.
\end{lemma}

\begin{proof}
  We write the cocycle condition on $x,x,y$:

  \begin{equation*}
    f(x,xy)+ f(x,y) = f(x^2,y) + f(x,x)
  \end{equation*}

  We note that by the IIP conditions, both terms on the right side are
  zero, so since $z = xy$, we get:

  \begin{equation}
    f(x,z) + f(x,y) = 0 \label{quasialternatingkleinfour}
  \end{equation}

  Now we write the cocycle condition on $x,y,z$:

  \begin{equation*}
    f(x,yz) + f(y,z) = f(xy,z) + f(x,y)
  \end{equation*}

  Since $yz = x$ and $xy = z$, we have, from inverse-preservation (and
  the fact that elements have order $2$), that $f(x,yz) = f(xy,z) = 0$, so we get:

  \begin{equation}
    f(y,z) = f(x,y) \label{cyclicpermkleinfour-prelim}
  \end{equation}

  By cyclic permutations of \ref{cyclicpermkleinfour-prelim}, we get:

  \begin{equation}
    f(x,y) = f(y,z) = f(z,x)\label{cyclicpermkleinfour}
  \end{equation}
  
  Combining this with \ref{quasialternatingkleinfour}, we get the
  desired result relating all six $f$-values.

  Note that the remaining $10$ values for inputs in $\langle x,y
  \rangle$ are already zero by identity-preservation and
  inverse-preservation.

  We now plug in $x$, $y$, and $x$ into the cocycle condition, to get:

  \begin{equation*}
    f(x,yx) + f(y,x) = f(xy,x) + f(x,y)
  \end{equation*}

  Simplifying using the relations we already have, we get $4f(x,y) =
  0$.
  
  In other words, the image of $f$ must land inside the $4$-torsion of
  $A$.
\end{proof}

\subsection{$2$-torsion and $4$-torsion?}

The fact that the image of an IIP 2-cocycle must land inside the
$4$-torsion means that an IIP 2-cocycle cannot be halved to get
another IIP 2-cocycle even if every element can be halved locally. In
other words, IIP 2-cocycle $\to$ IIP 2-cocycle halving does not always
work, nor does skew-symmetric IIP 2-cocycle $\to$ skew-symmetric IIP
2-cocycle halving.

However, the conjecture that we really care about is alternating
bihomomorphism $\to$ skew-symmetric IIP 2-cocycle halving. As far as
this is concerned, we are in good shape, at least locally with respect
to Klein four-groups. This is because if $G$ is an elementary abelian
$2$-group, the image under a bihomomorphism must land inside the
$2$-torsion, so halving it lands us inside $4$-torsion, which is still
safe.
\end{document}