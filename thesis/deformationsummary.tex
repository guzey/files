\documentclass[10pt]{amsart}

%Packages in use
\usepackage{fullpage, hyperref, vipul, amssymb, graphicx}

%Title details
\title{Deformation results: summary}
\author{Vipul Naik}
\thanks{Based on joint work with John Wiltshire-Gordon}

%List of new commands
\newcommand{\Skew}{\operatorname{Skew}}
\makeindex

\begin{document}
\maketitle
%\tableofcontents

This is a summary of important results. For more, see
lazardlieringof2groups and cohomologyconjectures.

Start with a Lie ring $L$ of nilpotency class two. Define a
multiplication $*$ by:

\begin{equation}
  x * y := x + y + f(x,y)
\end{equation}

where $\overline{f}:L/Z(L) \times L/Z(L) \to Z(L)$ is lifted naturally
to $f:L \times L \to L$. Here are all the important results:

\begin{enumerate}
\item If $f$ is a 2-cocycle for the trivial group action, then $*$ is associative.
\item If $f$ is a identity-preserving, i.e., $f(x,0) = 0$ and $f(0,x)
  = 0$, then $0$ is the identity element for $*$.
\item If $f$ is identity-preserving and inverse-preserving, i.e.,
  $f(x,0) = f(0,x) = f(x,-x) = 0$ for all $x$, then $0$ is the
  identity element for $*$ and the inverse operation for $*$ is the
  $-$ operation of the Lie ring.
\item Combining the above, if $f$ is an IIP 2-cocycle, then $*$ gives
  a group structure.
\item If $f$ is an IIP 2-cocycle and the skew of $f$ equals the Lie
  bracket of $L$, then $*$ gives a group of nilpotency class two whose
  Lie bracket equals that of $L$.
\item If $f$ is 2-locally a 2-cocycle (i.e., the restriction of $f$ to
  any subgroup generated by two elements is a 2-cocycle), then $*$ is
  diassociative. In particular, if $f$ is IIP and is 2-locally a
  2-cocycle, then $*$ gives (check!) a diassociative loop.
\item If $f$ is cyclicity-preserving (i.e., $f(x,y) = 0$ whenever
  $\langle x,y \rangle$ is cyclic) and a 2-cocycle, then $*$ gives a
  group structure that is 1-isomorphic to the original Lie ring.
\end{enumerate}