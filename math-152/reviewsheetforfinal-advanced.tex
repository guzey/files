\documentclass[10pt]{amsart}
\usepackage{fullpage,hyperref,vipul, graphicx}
\title{Review sheet for final: advanced}
\author{Math 152, Section 55 (Vipul Naik)}

\begin{document}
\maketitle

\section{Area computations}

No error-spotting exercises.

\section{Volume computations}

Error-spotting exercises ...

\begin{enumerate}
\item Consider the function $f(x) := \sin x$ and $g(x) := -\sin x$,
  for $x \in [0,\pi]$. The region between the graphs of these
  functions is revolved about the $x$-axis. The volume of the solid of
  revolution is:

  $$\pi \int_0^\pi (f(x) - g(x))^2 \, dx = \pi \int_0^\pi 4 \sin^2 x \, dx = 2\pi^2$$

\item If a right angled triangle is revolved about any of its sides,
  then we get a right circular cone.

\item If a rectangle is revolved about one of its diagonals, then we
  get a union of two right circular cones.

\item The volume of the solid of revolution obtained by revolving a
  region of area $A$ is proportional to $A$.
\end{enumerate}

\section{One-one functions and inverses}

\subsection{Vague generalities}

Error-spotting exercises ...

\begin{enumerate}
\item If $f$ and $g$ are one-one functions, then so is $f \circ g$ and
  $(f \circ g)^{-1} = f^{-1} \circ g^{-1}$.
\end{enumerate}

\subsection{In the real world}

Error-spotting exercises ...

\begin{enumerate}
\item Consider the function:

  $$f(x) := x^3 + x$$

  We have $f'(x) = 3x^2 + 1 > 0$ is always positive, so $f$ is
  one-one. Also, note that the inverse function to $x \mapsto x^3$ is
  $x \mapsto x^{1/3}$ and the inverse function to $x \mapsto x$ is $x
  \mapsto x$. Thus, we get:

  $$f^{-1}(x) := x^{1/3} + x$$

\item If $f$ is a one-one function on $\R$, then it must be either an
  increasing function or a decreasing function on $\R$.
\item If $f$ is a differentiable function on $\R$, then $f$ is one-one
  if and only if $f'(x) > 0$ for all $x \in \R$.
\item Suppose $f$ and $g$ are continuous one-one functions on
  $\R$. Then, clearly, they are either increasing or decreasing
  functions on $\R$. Thus, the sum $f + g$ is also either an
  increasing or decreasing function on $\R$, and hence it must be
  one-one.
\item Suppose $f$, $g$, and $h$ are continuous one-one functions on
  $\R$. Then, the pairwise sums $f + g$, $g + h$, and $f + h$ are all
  one-one functions.
\item Suppose $f$ is a one-one function such that the graph of $f$ is
  concave up. Then, $f^{-1}$ is also a one-one function and its graph
  is concave down.
\item If $c$ is a point in the domain of a function $f$ such that the
  left hand derivative and right hand derivative of $f$ at $c$ do not
  agree, then the left hand derivative and right hand derivative of
  $f^{-1}$ at $c$ also do not agree.
\item Consider the function:

  $$f(x) := \left\lbrace\begin{array}{rl} x + 1, & x \text{ rational } \\ x^3, & x \text{ irrational } \\\end{array}\right.$$

  We know that both $x + 1$ and $x^3$ are one-one functions on their
  respective domains. Thus, $f$ is a one-one function.

\end{enumerate}

\section{Logarithm, exponential, derivative, and integral}

\subsection{Logarithm and exponential: basics}

\begin{enumerate}
\item We have that $\ln(xy) = \ln x + \ln y$. Thus, $\ln((-1)^2) =
  \ln(-1) + \ln(-1)$. The left side is $\ln 1 = 0$, so $\ln(-1) = 0$.
\item If $f$ is a function on $\R \setminus \{ 0 \}$ such that $f'(x)
  = 1/x$ for all $x \ne 0$, then $f(x) = \ln|x| + C$ for some fixed
  constant $C$.
\item Using that $\ln 2 \sim 0.7$ and $\ln 3 \sim 1.1$, we obtain that
  $\ln 5 = \ln(2 + 3) = \ln 2 + \ln 3 \sim 0.7 + 1.1 = 1.8$.
\item We have $\exp((\ln x)^2) = \exp(\ln(x^2)) = x^2$.
\end{enumerate}

\subsection{Integration involving logarithms and exponents}

\begin{enumerate}
\item Consider the integration $\int_0^\pi \tan x \, dx$. This is:

  $$\int_0^\pi \tan x \, dx = [\ln|\cos x|]_0^\pi = 0$$

\item Consider the indefinite integration $\int e^{x + \ln(\sin x)} \, dx$. This becomes:

  $$\int e^{x + \ln(\sin x)} \, dx = \int e^x \, dx + \int e^{\ln(\sin x)} \, dx = e^x + \int \sin x \, dx = e^x - \cos x +C = -e^x \cos x + C$$
  
\end{enumerate}

\subsection{Exponentiation with arbitrary bases, exponents}

No error-spotting exercises

\section{Miscellaneous error-spotting exercises}

\begin{enumerate}
\item Consider the function $x \mapsto x^{1/3} + x^{2/3}$. The
  $x^{1/3}$ part has a vertical tangent at $x = 0$ and $x^{2/3}$ part
  has a vertical cusp at $x = 0$. The tangent and cusp cancel and thus
  overall we get neither a vertical tangent nor a vertical cusp at
  $0$.
\item Consider the integration:

  $$\int \frac{x^2}{x + 1} \, dx = \int \frac{x^2}{x} \, dx + \int \frac{x^2}{1} \, dx = \int x + 1 \, dx = x^2/2 + x + C$$
\item Consider the function:

  $$f(x) := x \sin (\pi/(x^2 + x))$$

  This is undefined at $x = 0$ and $x = 1$. At both these points, the
  graph of $f$ has a vertical asymptote.
\end{enumerate}




\section{Quickly}

This ``Quickly'' list improves upon previous ``Quickly'' lists.

\subsection{Our common values}

Preferably remember these (or be capable of computing quickly) to at
least two digits.

\begin{enumerate}
\item Square roots of $2$, $3$, $5$, $6$, $7$, $10$.
\item Natural logarithms of $2$, $3$, $5$, $7$, and $10$.
\item Value of $\pi$, $1/\pi$, $\sqrt{\pi}$, and $\pi^2$.
\item Value of $e$, $1/e$.
\item Some relative logarithms, such as $\log_23$ or
  $\log_2(10)$. Although you don't need these values to a significant
  degree of precision, it is useful to have some idea of their
  magnitude.
\end{enumerate}

\subsection{Adding things up: arithmetic}

You should be able to:

\begin{enumerate}
\item Do quick arithmetic involving fractions.
\item Sense when an expression will simplify to $0$.
\item Compute approximate values for square roots of small numbers,
  $\pi$ and its multiples, etc., so that you are able to figure out,
  for instance, whether $\pi/4$ is smaller or bigger than $1$, or two
  integers such that $\sqrt{39}$ is between them.
\item Know or quickly compute small powers of small positive
  integers. This is particularly important for computing definite
  integrals. For instance, to compute $\int_2^3 (x + 1)^3 \, dx$, you
  need to know/compute $3^4$ and $4^4$.
\end{enumerate}

\subsection{Computational algebra}

You should be able to:

\begin{enumerate}
\item Add, subtract, and multiply polynomials.
\item Factorize quadratics or determine that the quadratic cannot be
  factorized.
\item Factorize a cubic if at least one of its factors is a small and
  easy-to-spot number such as $0$, $\pm 1$, $\pm 2$, $\pm 3$. {\em
  This could be an area for potential improvement for many people.}
\item Factorize an even polynomial of degree four. {\em This could be
  an area for potential improvement for many people.}
\item Do polynomial long division (not usually necessary, but helpful).
\item Solve simple inequalities involving polynomial and rational
  functions once you've obtained them in factored form.
\end{enumerate}

\subsection{Computational trigonometry}

You should be able to:

\begin{enumerate}
\item Determine the values of $\sin$, $\cos$, and $\tan$ at multiples
  of $\pi/2$.
\item Determine the intervals where $\sin$ and $\cos$ are positive and
  negative.
\item Remember the formulas for $\sin(\pi \pm x )$ and $\cos(\pi \pm x)$,
  as well as formulas for $\sin(-x)$ and $\cos(-x)$.
\item Recall the values of $\sin$ and $\cos$ at $\pi/6$, $\pi/4$, and
  $\pi/3$, as well as at the corresponding obtuse angles or other
  larger angles.
\item Reverse lookup for these, for instance, you should quickly
  identify the acute angle whose $\sin$ is $1/2$.
\item Formulas for double angles, half angles: $\sin(2x)$, $\cos(2x)$
  in terms of $\sin$ and $\cos$; also the reverse: $\sin^2x$ and
  $\cos^2x$ in terms of $\cos(2x)$.
\item {\em More advanced}: Remember the formulas for $\sin(A + B)$,
  $\cos(A + B)$, $\sin(A - B)$, and $\cos(A - B)$.
\item {\em More advanced}: Convert between products of $\sin$ and
  $\cos$ functions and their sums: for instance, the identity $2\sin A
  \cos B = \sin(A + B) + \sin (A - B)$.
\end{enumerate}

\subsection{Computational limits}

You should be able to: size up a limit, determine whether it is of the
form that can be directly evaluated, of the form that we already know
does not exist, or indeterminate.

\subsection{Computational differentiation}

You should be able to:

\begin{enumerate}
\item Differentiate a polynomial (written in expanded form) on sight
  (without rough work).
\item Differentiate a polynomial (written in expanded form) twice
  (without rough work).
\item Differentiate sums of powers of $x$ on sight (without rough
  work).
\item Differentiate rational functions with a little thought.
\item Do multiple differentiations of expressions whose derivative
  cycle is periodic, e.g., $a \sin x + b \cos x$ or $a \exp(-x)$.
\item Do multiple differentiations of expressions whose derivative
  cycle is periodic up to constant factors, e.g. $a \exp(mx + b)$ or $a
  \sin(mx + \varphi)$.
\item Differentiate simple composites without rough work (e.g.,
  $\sin(x^3)$).
\item Differentiate $\ln$, $\exp$, and expressions of the form $f^g$
  and $\log_f(g)$.
\end{enumerate}

\subsection{Computational integration}

You should be able to:

\begin{enumerate}
\item Compute the indefinite integral of a polynomial (written in
  expanded form) on sight without rough work.
\item Compute the definite integral of a polynomial with very few
  terms within manageable limits quickly.
\item Compute the indefinite integral of a sum of power functions
  quickly.
\item Know that the integral of sine or cosine on any quadrant is $\pm
  1$.
\item Compute the integral of $x \mapsto f(mx)$ if you know how to
  integrate $f$. In particular, integrate things like $(a + bx)^m$.
\item Integrate $\sin$, $\cos$, $\sin^2$, $\cos^2$, $\tan^2$,
  $\sec^2$, $\cot^2$, $\csc^2$, any odd power of $\sin$, any odd power
  of $\cos$, any odd power of $\tan$.
\item Integrate on sight things such as $x\sin(x^2)$, getting the
  constants right without much effort.
\end{enumerate}
\subsection{Being observant}

You should be able to look at a function and:

\begin{enumerate}
\item Sense if it is odd (even if nobody pointedly asks you whether it
  is).
\item Sense if it is even (even if nobody asks you whether it is).
\item Sense if it is periodic and find the period (even if nobody asks
  you about the period).
\end{enumerate}

\subsection{Graphing}

You should be able to:

\begin{enumerate}
\item Mentally graph a linear function.
\item Mentally graph a power function $x^r$ (see the list of things to
  remember about power functions). Sample cases for $r$: $1/3$, $2/3$,
  $4/3$, $5/3$, $1/2$, $1$, $2$, $3$, $-1$, $-1/3$ $-2/3$.
\item Graph a piecewise linear function with some thought.
\item Mentally graph a quadratic function (very approximately) --
  figure out conditions under which it crosses the axis etc.
\item Graph a cubic function after ascertaining which of the cases for
  the cubic it falls under.
\item Mentally graph $\sin$ and $\cos$, as well as functions of the $A
  \sin(mx)$ and $A\cos(mx)$.
\item Graph a function of the form linear + trigonometric, after doing
  some quick checking on the derivative.
\end{enumerate}

\subsection{Graphing: transformations}

Given the graph of $f$, you should be able to quickly graph the following:

\begin{enumerate}
\item $f(mx)$, $f(mx + b)$: pre-composition with a linear function;
  how does $m < 0$ differ from $m > 0$?
\item $Af(x) + C$: post-composition with a linear function, how does
  $A > 0$ differ from $A < 0$?
\item $f(|x|)$, $|f(x)|$, $f(x^+)$, and $(f(x))^+$: pre- and
  post-composition with absolute value function and positive part
  fnuction.
\item More slowly: $f(1/x)$, $1/f(x)$, $\ln(|f(x)|)$, $f(\ln|x|)$,
  $\exp(f(x))$, and other popular composites.
\end{enumerate}

\subsection{Fancy pictures}

Keep in mind approximate features of the graphs of:

\begin{enumerate}
\item $\sin(1/x)$, $x\sin(1/x)$, $x^2 \sin(1/x)$ and $x^3\sin(1/x)$,
  and the corresponding $\cos$ counterparts -- both the behavior near
  $0$ and the behavior near $\pm \infty$.
\item The Dirichlet function and its variants -- functions defined
  differently for the rationals and irrationals.
\end{enumerate}

\end{document}