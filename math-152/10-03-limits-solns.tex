\documentclass[10pt]{amsart}

%Packages in use
\usepackage{fullpage, hyperref, vipul, enumerate}

%Title details
\title{Class quiz solutions: October 3: Limits}
\author{Math 152, Section 55 (Vipul Naik)}
%List of new commands

\begin{document}
\maketitle

\section{Performance review}

$12$ people took this quiz. The score distribution was as follows:

\begin{itemize}
\item Score of $0$: $1$ person
\item Score of $1$: $2$ people
\item Score of $2$: $5$ people
\item Score of $3$: $3$ people
\item Score of $4$: $1$ person
\end{itemize}

Here are the answers:

\begin{enumerate}
\item Option (C): $8$ people
\item Option (C): $2$ people. {\em Please review!}
\item Option (B): $8$ people
\item Option (C): $7$ people
\end{enumerate}

\section{Solutions}

\begin{enumerate}

\item Which of these is the correct interpretation of $\lim_{x \to c} f(x)
  = L$ in terms of the definition of limit?

  \begin{enumerate}
  \item For every $\alpha > 0$, there exists $\beta > 0$ such that if
    $0 < |x - c| < \alpha$, then $|f(x) - L| < \beta$.
  \item There exists $\alpha > 0$ such that for every $\beta > 0$, and
    $0 < |x - c| < \alpha$, we have $|f(x) - L| < \beta$.
  \item For every $\alpha > 0$, there exists $\beta > 0$ such that if
    $0 < |x - c| < \beta$, then $|f(x) - L| < \alpha$.
  \item There exists $\alpha > 0$ such that for every $\beta > 0$ and
    $0 < |x - c| < \beta$, we have $|f(x) - L| < \alpha$.
  \end{enumerate}

  {\em Answer}: Option (C)

  {\em Explanation}: $\alpha$ plays the role of $\epsilon$ and $\beta$
  plays the role of $\delta$.

  {\em Performance review}: $8$ out of $12$ got this correct. $2$
  chose (B), $1$ each chose (A) and (E).

  {\em Historical note (last year)}: $9$ out of $12$ people got this
  correct. $2$ people chose (B) and $1$ person chose (D).

  {\em Action point}: If you got this correct, that means that you are
  not completely fixated on the letters $\epsilon$ and $\delta$. This
  is good news, because it is important to concentrate on the
  substantive meaning rather than get caught up with a name. If you
  had difficulty with this, make sure you can understand it now.

\item Suppose $f:\R \to \R$ is a function. Which of the following says
  that $f$ does not have a limit at any point in $\R$ (i.e., there is
  no point $c \in \R$ for which $\lim_{x \to c} f(x)$ exists)?

  \begin{enumerate}[(A)]
  \item For every $c \in \R$, there exists $L \in \R$ such that for
    every $\epsilon > 0$, there exists $\delta > 0$ such that for all
    $x$ satisfying $0 < |x - c| < \delta$, we have $|f(x) - L| \ge
    \epsilon$.
  \item There exists $c \in \R$ such that for every $L \in \R$, there
    exists $\epsilon > 0$ such that for every $\delta > 0$, there exists
    $x$ satisfying $0 < |x - c| < \delta$ and $|f(x) - L| \ge \epsilon$.
  \item For every $c \in \R$ and every $L \in \R$, there exists
    $\epsilon > 0$ such that for every $\delta > 0$, there exists $x$
    satisfying $0 < |x - c| < \delta$ and $|f(x) - L| \ge \epsilon$.
  \item There exists $c \in \R$ and $L \in \R$ such that for
    every $\epsilon > 0$, there exists $\delta > 0$ such that for all
    $x$ satisfying $0 < |x - c| < \delta$, we have $|f(x) - L| \ge
    \epsilon$.
  \item All of the above.
  \end{enumerate}

  {\em Answer}: Option (C)

  {\em Explanation}: Our statement should be that {\em every} $c$ has
  no limit. In other words, for {\em every} $c$ and {\em every} $L$,
  it is {\em not} true that $\lim_{x \to c} f(x) = L$. That's exactly
  what option (C) says.

  {\em Performance review}: $2$ out of $12$ got this correct. $7$
  chose (B), $2$ chose (D), $1$ chose (A).

  {\em Historical note (last year)}: $10$ out of $12$ people got this
  correct. $1$ person each chose (B) and (E).

  {\em Note on performance discrepancy with last year}: I now remember
  that last year I had discussed the idea behind this very question in
  the same class as the quiz was administered, so students last year
  had an advantage in doing the quiz.

  {\em Action point}: If you got this correct, great! Do make sure you
  understand this thoroughly -- review the $\epsilon-\delta$
  definition yet another time and try to understand how this follows
  from that definition.

\item In the usual $\epsilon-\delta$ definition of limit for a given
  limit $\lim_{x \to c} f(x) = L$, if a given value $\delta > 0$ works
  for a given value $\epsilon > 0$, then which of the following is
  true?
  \begin{enumerate}[(A)]
  \item Every smaller positive value of $\delta$ works for the same
    $\epsilon$. Also, the given value of $\delta$ works for every
    smaller positive value of $\epsilon$.
  \item Every smaller positive value of $\delta$ works for the same
    $\epsilon$. Also, the given value of $\delta$ works for every
    larger value of $\epsilon$.
  \item Every larger value of $\delta$ works for the same
    $\epsilon$. Also, the given value of $\delta$ works for every
    smaller positive value of $\epsilon$.
  \item Every larger value of $\delta$ works for the same
    $\epsilon$. Also, the given value of $\delta$ works for every
    larger value of $\epsilon$.
  \item None of the above statements need always be true.
  \end{enumerate}

  {\em Answer}: Option (B)

  {\em Explanation}: This can be understood in multiple ways. One is
  in terms of the prover-skeptic game. A particular choice of $\delta$
  that works for a specific $\epsilon$ also works for larger
  $\epsilon$s, because the function is already ``trapped'' in a
  smaller region. Further, smaller choices of $\delta$ also work
  because the skeptic has fewer values of $x$.

  Rigorous proofs are being skipped here, but you can review the
  formal definition of limit notes if this stuff confuses you.

  {\em Performance review}: $8$ out of $12$ got this correct. $3$
  chose (A), $1$ chose (E).

  {\em Historical note (last year)}: $17$ out of $26$ people
  got this correct. $5$ people chose (A), $3$ chose (C), and $1$ chose
  (D).

\item Which of the following is a correct formulation of the statement
  $\lim_{x \to c} f(x) = L$, in a manner that avoids the use of
  $\epsilon$s and $\delta$s? {\em Not appeared in previous years}

  \begin{enumerate}[(A)]
  \item For every open interval centered at $c$, there is an open
    interval centered at $L$ such that the image under $f$ of the open
    interval centered at $c$ (excluding the point $c$ itself) is
    contained in the open interval centered at $L$.
  \item For every open interval centered at $c$, there is an open
    interval centered at $L$ such that the image under $f$ of the open
    interval centered at $c$ (excluding the point $c$ itself) contains
    the open interval centered at $L$.
  \item For every open interval centered at $L$, there is an open
    interval centered at $c$ such that the image under $f$ of the open
    interval centered at $c$ (excluding the point $c$ itself) is
    contained in the open interval centered at $L$.
  \item For every open interval centered at $L$, there is an open
    interval centered at $c$ such that the image under $f$ of the open
    interval centered at $c$ (excluding the point $c$ itself) contains
    the open interval centered at $L$.
  \item None of the above.
  \end{enumerate}

  {\em Answer}: Option (C)

  {\em Explanation}: The ``open interval centered at $L$'' describes
  the ``$\epsilon > 0$'' part of the definition (where the open
  interval is the interval $(L - \epsilon, L + \epsilon)$). The ``open
  interval centered at $c$'' describes the ``$\delta > 0$'' part of
  the definition (where the open interval is the interval $(c -
  \delta, c + \delta)$). $x$ being in the open interval centered at
  $c$ (except the case $x = c$) is equivalent to $0 < |x - c| <
  \delta$, and $f(x)$ being in the open interval centered at $L$ is
  equivalent to $|f(x) - L| < \epsilon$.

  {\em Performance review}: $7$ out of $12$ got this correct. $2$
  chose (A), $1$ each chose (B), (D), and (E).

  {\em Action point}: You should master this way of thinking. This is
  actually the ``correct'' way of thinking of the definition. You'll
  see what I mean in future math classes (153 and beyond).

\end{enumerate}
\end{document}