\documentclass[10pt]{amsart}

%Packages in use
\usepackage{fullpage, hyperref, vipul, enumerate}

%Title details
\title{Class quiz: October 21: Max-min problems}
\author{Math 152, Section 55 (Vipul Naik)}
%List of new commands

\begin{document}
\maketitle

Your name (print clearly in capital letters): $\underline{\qquad\qquad\qquad\qquad\qquad\qquad\qquad\qquad\qquad\qquad}$

Note: Questions 5-7 have only four options.
\begin{enumerate}

\item Consider all the rectangles with perimeter equal to a fixed
  length $p > 0$. Which of the following {\bf is true} for the unique
  rectangle which is a square, compared to the other rectangles? {\em
  Last year: $15/15$ correct}

  \begin{enumerate}[(A)]
  \item It has the largest area and the largest length of diagonal.
  \item It has the largest area and the smallest length of diagonal.
  \item It has the smallest area and the largest length of diagonal.
  \item It has the smallest area and the smallest length of diagonal.
  \item None of the above.
  \end{enumerate}

  \vspace{0.1in}
  Your answer: $\underline{\qquad\qquad\qquad\qquad\qquad\qquad\qquad}$
  \vspace{0.6in}

\item Suppose the total perimeter of a square and an equilateral
  triangle is $L$. (We can choose to allocate all of $L$ to the
  square, in which case the equilateral triangle has side zero, and we
  can choose to allocate all of $L$ to the equilateral triangle, in
  which case the square has side zero). Which of the following
  statements {\bf is true} for the sum of the areas of the square and
  the equilateral triangle? (The area of an equilateral triangle is
  $\sqrt{3}/4$ times the square of the length of its side). {\em Last
  year: $9/15$ correct}
  \begin{enumerate}[(A)]
  \item The sum is minimum when all of $L$ is allocated to the square.
  \item The sum is maximum when all of $L$ is allocated to the square.
  \item The sum is minimum when all of $L$ is allocated to the
    equilateral triangle.
  \item The sum is maximum when all of $L$ is allocated to the
    equilateral triangle.
  \item None of the above.
  \end{enumerate}

  \vspace{0.1in}
  Your answer: $\underline{\qquad\qquad\qquad\qquad\qquad\qquad\qquad}$
  \vspace{0.6in}

\item Suppose $x$ and $y$ are positive numbers such as $x + y =
  12$. For {\bf what values} of $x$ and $y$ is $x^2y$ maximum? {\em
  Last year: $12/15$ correct}

  \begin{enumerate}[(A)]
  \item $x = 3$, $y = 9$
  \item $x = 4$, $y = 8$
  \item $x = 6$, $y = 6$
  \item $x = 8$, $y = 4$
  \item $x = 9$, $y = 3$
  \end{enumerate}

  \vspace{0.1in}
  Your answer: $\underline{\qquad\qquad\qquad\qquad\qquad\qquad\qquad}$
  \vspace{0.6in}

\item (**) Consider the function $p(x) := x^2 + bx + c$, with $x$
  restricted to {\em integer inputs}. Suppose $b$ and $c$ are
  integers. The minimum value of $p$ is attained either at a single
  integer or at two consecutive integers. Which of the following is a
  {\bf sufficient condition} for the minimum to occur at two
  consecutive integers? {\em Last year: $4/15$ correct}

  \begin{enumerate}[(A)]
  \item $b$ is odd
  \item $b$ is even
  \item $c$ is odd
  \item $c$ is even
  \item None of these conditions is sufficient.
  \end{enumerate}

  \vspace{0.1in}
  Your answer: $\underline{\qquad\qquad\qquad\qquad\qquad\qquad\qquad}$
  \vspace{0.6in}

\item (**) Consider a hollow cylinder with no top and bottom and total
  curved surface area $S$. What can we say about the {\bf maximum and
  minimum} possible values of the {\bf volume}? (for radius $r$ and height
  $h$, the curved surface area is $2\pi rh$ and the volume is $\pi
  r^2h$). {\em Last year: $6/15$ correct}

  \begin{enumerate}[(A)]
  \item The volume can be made arbitrarily small (i.e., as close to
    zero as we desire) and arbitrarily large (i.e., as large as we
    want).
  \item There is a positive minimum value for the volume, but it can
    be made arbitrarily large.
  \item There is a finite maximum value for the volume, but it can be
    made arbitrarily small.
  \item There is both a finite positive minimum and a finite positive
    maximum for the volume.
  \end{enumerate}

  \vspace{0.1in}
  Your answer: $\underline{\qquad\qquad\qquad\qquad\qquad\qquad\qquad}$
  \vspace{0.6in}

\item Consider a hollow cylinder with a bottom but no top and total
  surface area (curved surface plus bottom) $S$. What can we say about
  the {\bf maximum and minimum} possible values of the {\bf volume}?
  (for radius $r$ and height $h$, the curved surface area is $2\pi rh$
  and the volume is $\pi r^2h$). {\em Last year: $8/15$ correct}

  \begin{enumerate}[(A)]
  \item The volume can be made arbitrarily small (i.e., as close to
    zero as we desire) and arbitrarily large (i.e., as large as we
    want).
  \item There is a positive minimum value for the volume, but it can
    be made arbitrarily large.
  \item There is a finite maximum value for the volume, but it can be
    made arbitrarily small.
  \item There is both a finite positive minimum and a finite positive
    maximum for the volume.
  \end{enumerate}

  \vspace{0.1in}
  Your answer: $\underline{\qquad\qquad\qquad\qquad\qquad\qquad\qquad}$
  \vspace{0.6in}

\item (**) Consider a hollow cylinder with a bottom and a top and
  total surface area (curved surface plus bottom and top) $S$. What
  can we say about the {\bf maximum and minimum} possible values of
  the {\bf volume}?  (for radius $r$ and height $h$, the curved
  surface area is $2\pi rh$ and the volume is $\pi r^2h$). {\em Last
  year: $5/15$ correct}

  \begin{enumerate}[(A)]
  \item The volume can be made arbitrarily small (i.e., as close to
    zero as we desire) and arbitrarily large (i.e., as large as we
    want).
  \item There is a positive minimum value for the volume, but it can
    be made arbitrarily large.
  \item There is a finite maximum value for the volume, but it can be
    made arbitrarily small.
  \item There is both a finite positive minimum and a finite positive
    maximum for the volume.
  \end{enumerate}

  \vspace{0.1in}
  Your answer: $\underline{\qquad\qquad\qquad\qquad\qquad\qquad\qquad}$
  \vspace{0.6in}

\end{enumerate}

\end{document}