\documentclass[10pt]{amsart}

%Packages in use
\usepackage{fullpage, hyperref, vipul, enumerate}

%Title details
\title{Class quiz solutions: November 16: Volume}
\author{Math 152, Section 55 (Vipul Naik)}
%List of new commands

\begin{document}
\maketitle

\section{Performance review}

$11$ people took this $11$ question quiz. The score distribution was
as follows:

\begin{itemize}
\item Score of $4$: $1$ person.
\item Score of $6$: $2$ people.
\item Score of $7$: $3$ people.
\item Score of $8$: $2$ people.
\item Score of $10$: $3$ people.
\end{itemize}

The mean score was $6.92$. The problem wise answers were as follows:

\begin{enumerate}
\item Option (B): $11$ people
\item Option (D): $5$ people
\item Option (B): $10$ people
\item Option (A): $8$ people
\item Option (B): $8$ people.
\item Option (C): $6$ people.
\item Option (C): $4$ people.
\item Option (C): $9$ people.
\item Option (E): $7$ people.
\item Option (A): $8$ people.
\item Option (B): $7$ people.
\end{enumerate}

\section{Solutions}

\begin{enumerate}
\item Oblique cylinder:Right cylinder::

  \begin{enumerate}[(A)]
  \item Rectangle:Square
  \item Parallelogram:Rectangle
  \item Disk:Circle
  \item Triangle:Rectangle
  \item Triangle:Square
  \end{enumerate}

  {\em Answer}: Option (B).

  {\em Explanation}: A right cylinder is obtained by translating a
  region of the plane along a direction perpendicular to the plane,
  while an oblique cylinder is obtained by translating along some
  direction, not necessarily perpendicular to the plane. Similarly, a
  rectangle is obtained by translating a line segment along a
  direction perpendicular to the line segment, while a parallelogram
  is obtained by translating a line segment along some direction, not
  necessarily perpendicular to the line segment.

  {\em Performance review}: Everybody got this correct.

  {\em Historical note (last year)}: $14$ out of $16$ people got this
  correct. $2$ people chose (D).

\item Right circular cone:Right circular cylinder::

  \begin{enumerate}[(A)]
  \item Triangle:Square
  \item Rectangle:Square
  \item Isosceles triangle:Equilateral triangle
  \item Isosceles triangle:Rectangle
  \item Isosceles triangle:Square
  \end{enumerate}

  {\em Answer}: Option (D).

  {\em Explanation}: There are two ways of seeing this. One is that if
  we look at a cross section containing the axis of symmetry, the
  cross section of a right circular cone is an isosceles triangle and
  the cross section of a right circular cylinder is a
  rectangle. Another way of thinking about this is that a rectangle is
  obtained by translating a line segment of fixed length in a
  perpendicular direction, and an isosceles triangle is obtained by
  translating it in a perpendicular direction while shirinking it
  symmetrically in a fixed proportion. Similarly, a right circular
  cylinder is obtained by translating a fixed disk, and a right
  circular cone is obtained by translating it while shrinking it.

  {\em Performance review}: $5$ out of $11$ got this. $4$ chose (A),
  $1$ chose (E), $1$ left the question blank.

  {\em Historical note (last year)}: $13$ out of $16$ people got this
  correct. $2$ people chose (A) and $1$ person chose (E).
\item Circular disk:Circle::

  \begin{enumerate}[(A)]
  \item Hollow cylinder:Solid cylinder
  \item Solid cylinder:Hollow cylinder
  \item Cube:Cuboid (cuboid is a term for rectangular prism)
  \item Cube:Square
  \item Cube:Sphere
  \end{enumerate}
  
  {\em Answer}: Option (B).
  
  {\em Explanation}: The circular disk is the region enclosed by the
  circle. Similarly, the solid cylinder is the region enclosed by the
  hollow cylinder.

  {\em Performance review}: $10$ out of $11$ people got this. $1$
  chose (D).

  {\em Historical note (last year)}: $8$ out of $16$ people
  got this correct. $4$ people chose (A) (the inverted option), $3$
  people chose (D), and $1$ person chose (E).

\item Circular disk:Line segment::

  \begin{enumerate}[(A)]
  \item Solid sphere:Circular disk
  \item Circle:Rectangle
  \item Sphere:Cube
  \item Cube:Right circular cylinder
  \item Square:Triangle
  \end{enumerate}

  {\em Answer}: Option (A).

  {\em Explanation}: There are many many ways of seeing this, none of
  which is obvious.

  In terms of cross sections, a line segment is a one-dimensional
  cross section (i.e., intersection with a line) of a circular disk,
  while a circular disk is a two-dimensional cross section (i.e.,
  intersection with a plane) of a solid sphere. Thus, a circular disk
  is to a solid sphere what a line segment is to a circular disk.

  Better, a line segment can be described as the set of points on a
  line (one-dimensional space) of distance at most a certain length
  from a fixed point. A circular disk is the analogue in two
  dimensions, and a solid sphere is the analogue in three
  dimensions. Thus, a solid sphere is to a circular disk what a
  circular disk is to a line segment.

  {\em Performance review}: $8$ out of $11$ got this. $2$ chose (A),
  $1$ chose (D).

  {\em Historical note (last year)}: $14$ out of $16$ got this
  correct. $1$ person chose (D) and $1$ person left the question
  blank. The number of correct answers was surprisingly large given
  that the logic of the analogy is far from obvious, but many people
  probably used the difference of dimensions.
\item Suppose a filled triangle $ABC$ in the plane is revolved about
  the side $AB$. Which of the following best describes the solid of
  revolution thus obtained if both the angles $A$ and $B$ are acute
  (ignoring issues of boundary inclusion/exclusion)?

  \begin{enumerate}[(A)]
  \item It is a right circular cone.
  \item It is the union of two right circular cones sharing a common
    disk as base.
  \item It is the set difference of two right circular cones sharing a
    common disk as base.
  \item It is the union of two right circular cones sharing a common
    vertex.
  \item It is the set difference of two right circular cones sharing a
    common vertex.
  \end{enumerate}

  {\em Answer}: Option (B)

  {\em Explanation}: Let $D$ be the foot of the perpendicular from $C$
  to $AB$. Since both the angles $A$ and $B$ are cute, $D$ lies on the
  line segment $AB$. Then, the triangle $ABC$ is the union of the
  right triangles $ACD$ and $BCD$, sharing a common side $CD$. Each of
  these right triangles gives as its solid of revolution a right
  circular cone, with the base disk being the disk corresponding to
  $CD$ in both. Thus, the overall solid is the union of the two right
  circular cones with a common disk.

  {\em Performance review}: $8$ out of $11$ got this. $2$ chose (A),
  $1$ chose (E).

  {\em Historical note (last year)}: $13$ out of $16$ people got this
  correct. $2$ people chose (D) and $1$ person chose (A).
\item Suppose a filled triangle $ABC$ in the plane is revolved about
  the side $AB$. Which of the following best describes the solid of
  revolution thus obtained if the angle $A$ is obtuse (ignoring issues
  of boundary inclusion/exclusion)?

  \begin{enumerate}[(A)]
  \item It is a right circular cone.
  \item It is the union of two right circular cones sharing a common
    disk as base.
  \item It is the set difference of two right circular cones sharing a
    common disk as base.
  \item It is the union of two right circular cones sharing a common
    vertex.
  \item It is the set difference of two right circular cones sharing a
    common vertex.
  \end{enumerate}

  {\em Answer}: Option (C) (this is not quite precise language,
  because we need to be careful about boundaries, but it is basically correct).

  {\em Explanation}: Let $D$ be the foot of the perpendicular from $C$
  to the line $AB$. Unlike the previous case, $D$ does {\em not} lie
  on the line segment $AB$, because the angle $A$ is obtuse. In fact,
  it lies on the $A$-side of the line segment. Thus, the triangle
  $ABC$ is the set difference of the right triangles $BCD$ and $ACD$,
  sharing a common side $CD$ (modulo some boundaries getting
  re-included). The corresponding solid of revolution is the set
  difference of the corresponding right circular cones, both of which
  have a common base disk corresponding to the side $CD$.

  {\em Performance review}: $6$ out of $11$ got this. $3$ chose (E),
  $1$ chose (B), $1$ chose (D).

  {\em Historical note (last year)}: $9$ out of $16$ people got this
  correct. $5$ people chose (D) and $1$ person each chose (B) and (E).
\item What is the volume of the solid of revolution obtained by
  revolving the filled triangle $ABC$ about the side $AB$, if the
  length of the base $AB$ is $b$ and the height corresponding to this
  base is $h$?

  \begin{enumerate}[(A)]
  \item $(1/6) \pi b^{3/2}h^{3/2}$
  \item $(1/3) \pi b^2h$
  \item $(1/3) \pi bh^2$
  \item $(2/3) \pi b^2h$
  \item $(2/3) \pi bh^2$
  \end{enumerate}

  {\em Answer}: Option (C).

  {\em Explanation}: The region is a union or difference of two right
  circular cones, as seen in some earlier multiple choice
  questions. Both these cones have {\em radius} $h$. The sum or
  difference of the heights of these cones is $b$. Thus, the formula
  gives (C).

  For the next two questions, suppose $\Omega$ is a region in a plane
  $\Pi$ and $\ell$ is a line on $\Pi$ such that $\Omega$ lies
  completely on one side of $\ell$ (in particular, it does not
  intersect $\ell$). Let $\Gamma$ be the solid of revolution obtained
  by revolving $\Omega$ about $\ell$. Suppose further that the
  intersection of $\Omega$ with any line perpendicular to $\ell$ is
  either empty or a point or a line segment.

  {\em Performance review}: $4$ out of $11$ got this. $3$ chose (C),
  $3$ chose (E), $1$ left the question blank.

  {\em Historical note}: $10$ out of $16$ people got
  this correct. $2$ people each chose (B) and (D) and $1$ person each
  chose (A) and (E).
\item (*) What is the intersection of $\Gamma$ with $\Pi$ (your answer
  should be always true)?

  \begin{enumerate}[(A)]
  \item It is precisely $\Omega$.
  \item It is the union of $\Omega$ and a translate of $\Omega$ along
    a direction perpendicular to $\ell$.
  \item It is the union of $\Omega$ and the reflection of $\Omega$
    about $\ell$.
  \item It is either empty or a rectangle whose dimensions depend on
    $\Omega$.
  \item It is either empty or a circle or an annulus whose inner and
    outer radius depend on $\Omega$.
  \end{enumerate}

  {\em Answer}: Option (C).

  {\em Explanation}: The intersection of $\Gamma$ with $\pi$ comprises
  those regions obtained by revolving $\Omega$ that land inside
  $\pi$. This is precisely $\Omega$ and the region obtained by
  revolving $\Omega$ by the {\em angle} $\pi$ (180 degrees;
  unfortunately there's symbol overloading here), which is equivalent
  to the reflection of $\Omega$ about $\ell$.

  {\em Performance review}: $9$ out of $11$ got this correct. $2$
  chose (E).

  {\em Historical note (last year)}: $6$ out of $16$ people got this
  correct. $4$ people chose (A), $3$ people chose (E), $2$ people
  chose (D), and $1$ person chose (B).

\item What is the intersection of $\Gamma$ with a plane
  perpendicular to $\ell$ (your answer should be always true)?

  \begin{enumerate}[(A)]
  \item It is precisely $\Omega$.
  \item It is the union of $\Omega$ and a translate of $\Omega$ along
    a direction perpendicular to $\ell$.
  \item It is the union of $\Omega$ and the reflection of $\Omega$
    about $\ell$.
  \item It is either empty or a rectangle whose dimensions depend on
    $\Omega$.
  \item It is either empty or a circle or an annulus whose inner and
    outer radius depend on $\Omega$.
  \end{enumerate}

  {\em Answer}: Option (E) (Oops, {\em circle} should have been {\em
  circular disk})

  {\em Explanation}: This is precisely the annulus procedure used to
  justify the washer method.

  Note that the description given by option (D) requires $\Omega$ to
  satisfy the condition given in the correction, which essentially
  says that the slices perpendicular to $\ell$ are nice enough. 

  {\em Performance review}: $7$ out of $11$ go this correct. $2$ chose
  (B) and $2$ chose (C).

  {\em Historical note}: $9$ out of $16$ people got this correct. $3$
  people chose (C), $2$ people chose (B), $1$ person chose (D), and
  $1$ person left the question blank.

\item (*) Consider a fixed equilateral triangle $ABC$. Now consider,
  for any point $D$ outside the plane of $ABC$, the solid tetrahedron
  $ABCD$. This is the solid bounded by the triangles $ABC$, $BCD$,
  $ACD$, and $ABD$. The volume of this solid depends on $D$. What
  specific information about $D$ completely determines the volume?

  \begin{enumerate}[(A)]
  \item The perpendicular distance from $D$ to the plane of the
    triangle $ABC$.
  \item The minimum of the distances from $D$ to points in the filled
    triangle $ABC$.
  \item The location of the point $E$ in the plane of triangle $ABC$
    that is the foot of the perpendicular from $D$ to $ABC$.
  \item The distance from $D$ to the center of $ABC$ (here, you can
    take the center as any of the notions of center since $ABC$ is
    equilateral).
  \item None of the above.
  \end{enumerate}

  {\em Answer}: Option (A).

  {\em Explanation}: In fact, the volume is $1/3$ times the area of
  the base (which is fixed) times this perpendicular distance.

  {\em Performance review}: $8$ out of $11$ got this correct. $2$
  chose (D) and $1$ chose (E).

  {\em Historical note (last year)}: $7$ out of $16$ people got this
  correct. $3$ people chose (C), $3$ people chose (D), and $1$ person
  each chose (B) and (E).
\item (**) For $r > 0$, consider the region $\Omega_r(a)$ bounded by
  the $x$-axis, the curve $y = x^{-r}$, and the lines $x = 1$ and $x =
  a$ with $a > 1$. Let $V_r(a)$ be the volume of the region obtained
  by revolving $\Omega_r(a)$ about the $x$-axis. What is the precise
  set of values of $r$ for which $\lim_{a \to \infty} V_r(a)$ is
  finite?

  \begin{enumerate}[(A)]
  \item All $r > 0$
  \item $r > 1/2$
  \item $r > 1$
  \item $r > 2$
  \item No value of $r$
  \end{enumerate}

  {\em Answer}: Option (B).

  {\em Explanation}: $V_r(a) = \pi \int_1^a x^{-2r} \, dx$. The limit
  is finite iff $2r > 1$, which is equivalent to $r > 1/2$.

  {\em Performance review}: $7$ out of $11$ got this correct. $3$
  chose (A) and $1$ chose (C).

  {\em Historical note (last year)}: $3$
  out of $16$ people got this correct. $7$ people chose (C), $4$
  people chose (A), and $2$ people chose (E).

\end{enumerate}
\end{document}