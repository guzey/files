\documentclass[10pt]{amsart}
\usepackage{fullpage,hyperref,vipul}
\title{Derivative as rate of change, implicit derivatives: rough approximation of lecture}
\author{Math 152, Section 55 (Vipul Naik)}

\begin{document}
\maketitle

{\bf Corresponding material in the book}: Sections 3.4 and 3.7

{\bf Difficulty level}: Easy to moderate, since most of these should
be familiar to you and there are no new subtleties being added here.

{\bf What students should definitely get}: The notion of derivative as
a rate of change, handling word problems that ask for rates of
change. The main idea and procedure of implicit differentiation.

{\bf What students should hopefully get}: The distinction between
conceptual and computational, the significance of implicit
differentiation, understanding the relative rates concept and its
intuitive relationship with the chain rule.

\section*{Executive summary}

\subsection*{Derivative as rate of change}

Words...

\begin{enumerate}
\item The derivative of $A$ with respect to $B$ is the rate of change
  of $A$ with respect to $B$. Thus, to determine rates of change of
  various quantities, we can use the techniques of differentiation.
\item If there are three linked quantities that are changing together
  (e.g., different measures for a circle such as radius, diameter,
  circumference, area) then we can use the chain rule.
\end{enumerate}

Most of the actions in this case are not more than a direct
application of the words.

\subsection*{Implicit differentiation}

Words...

\begin{enumerate}
\item Suppose there is a curve in the plane, whose equation cannot be
  manipulated easily to write one variable in terms of the other. We
  can use the technique of implicit differentiation to determine the
  derivative, and hence the slope of the tangent line, at different
  points to the curve.
\item For a curve where neither variable is expressible as a function
  of the other, the notion of derivative still makes sense as long as
  {\em locally}, we can get $y$ as a function of $x$. For instance,
  for the circle $x^2 + y^2 = 1$, $y$ is not a function of $x$, but if
  we restrict attention to the part of the circle above the $x$-axis,
  then on this restricted region, $y$ is a function of $x$.
\item In some cases, even when one variable is expressible as a
  function of the other, implicit differentiation is easier to handle
  as it may involve fewer messy squareroot symbols.
\end{enumerate}

Actions ...

\begin{enumerate}
\item To determine the derivative using implicit differentiation,
  write down the equations of both curves, differentiate both sides
  with respect to $x$, and simplify using all the differentiation
  rules, to get everything in terms of $x$, $y$, and $dy/dx$. Isolate
  the $dy/dx$ term in terms of $x$ and $y$, and compute it at whatever
  point is needed.
\item This procedure can be iterated to compute higher order
  derivatives at specific points on the curve where the curve locally
  looks like a function.
\end{enumerate}

\section{Conceptual versus computational}

Back in the first lecture, I defined the concept of function. A
function is some kind of machine that takes an input and gives an
output. And the important thing about functions is that {\em equal
inputs give equal outputs}.

The interesting thing about functions is that this way of thinking
about functions is a sort of {\em black box}, {\em hands-off}
approach. If you think of the function as this box machine which sucks
in an input from one side and spits out the output on the other side,
we don't really care {\em how the black box works}. It doesn't matter
what is happening inside, as long as we are guaranteed that equal
inputs give equal outputs. 

With this abstract concept of function, we defined the notion of
limit, which was the $\epsilon-\delta$ definition, and this definition
didn't really depend on how you compute $f$. Then we defined the
notion of derivative, which is a particular kind of limit, namely, the
limit of the different quotient. And in all this, how to {\em compute}
things wasn't the focus. And simply thinking of things conceptually,
we got a lot of insights. We understood what limits mean and we
understood what derivatives mean, and we saw the qualitative
significance.

Complementing this conceptual understanding of the concepts of
functions, limits, continuity, derivatives, and differentiation, there
is the computational aspect. The computational aspect tells us how,
for functions with specific functional forms or expressions, we can
calculate limits and derivatives. And in order to do this, we use
general theorems (limits for sums, differences, ...; derivatives for
sums, differences ...) and specific tricks and formulas.

What you should remember, though, is that {\em just because you cannot
compute something, doesn't mean that it cannot be understood
qualitatively}. So, if you encounter a function and there's no formula
to differentiate it, that's not the same as saying that it isn't
differentiable. Computation is one tool among many to get a conceptual
understanding of ideas.

This is really important because a lot of the places where you'll see
these mathematical ideas applied are cases where the functions
involved are inherently {\em unknown} or {\em unknowable} -- there
aren't explicit expressions for them. Still, we want to talk about the
broad qualitative properties -- is the function continuous? Is it
differentiable? Is it twice differentiable? Is it increasing or
decreasing, is it oscillating? Often, we can answer these qualitative
questions without having explicit expressions for the functions.

\section{Derivative as a rate of change}

Recall that if $f$ is a function, the derivative $f'$ is the {\em rate
of change} of the output of $f$ relative to the input. Or, if we are
thinking of two quantities $x$ and $y$, where $y$ is functionally
dependent on $x$, then the rate of change of $y$ with respect to $x$
is $dy/dx$. That is the limit of the {\em difference quotient} $\Delta
y/\Delta x$.

This means that if we want to ask the question: {\em if the rate of
change of $x$ is this much, what is the rate of change of $y$}, we
should think of derivatives.

For instance, we know that the area of a circle of radius $r$ is $\pi
r^2$. We may ask the question: what is the rate of change of the area
with respect to the radius? This is the derivative of $\pi r^2$ with
respect to $r$, and that turns out to be $2\pi r$.

For instance, if $r = 5$, the rate of change of the area with respect
to the radius is $10\pi$.

Now, suppose the radius is changing at the rate of $5 m/hr$. That
means that every hour, the radius increases by $5m$. What is the rate
of increase of the area with respect to time, when the radius is
$100m$. Well, here we have three quantities, the area $A$, the radius
$r$, and the time $t$. $r$ is a function of $t$, and $dr/dt = 5 m/hr$
and $dA/dr = 2\pi r$. So by the chain rule, we have $dA/dt =
(dA/dr)(dr/dt = (2\pi r) (5m/hr)$. And since $r = 100m$, we get
$1000\pi m^2/hr$.

\section{Implicit differentiation}

\subsection{Introduction}

So far, when trying to differentiate one quantity with respect to
another quantity, what we do is to write one as a function of the
other, and then differentiate that function. This is all very good
when we have an explicit expression for the function. Sometimes,
however, we do not really have a functional expression for one
quantity in terms of the other, but we do know of a {\em relation}
between the two quantities.

Let's think of this a little differently. One importance of
differentiation is that it allows us to find tangent lines to curves
that arise as the graph of a function. This has some geometric
significance, if we are trying to understand the geometry of a curve
that arises as the graph of a function. But what about the curves that
don't arise from explicit functions? Or, where we don't have explicit
functional expressions?

For instance, let's look at the circle of radius $1$ centered at the
origin. This is given by the equation $x^2 + y^2 = 1$. Note that in
this case, $y$ is {\em not} a function of $x$, because for many values
of $x$, there are two values of $y$. For instance, for $x = 0$, we
have $y = 1$ and $y = -1$. For $x = 1/2$, we have $y = \sqrt{3}/2$ and
$y = -\sqrt{3}/2$. So, $y$ is not a function of $x$.

However, {\em locally} $y$ is still a function of $x$, in the
following sense. If you just restrict yourself to the part above the
$x$-axis, then you do get $y$ as a function of $x$. This is the
function $y := \sqrt{1 - x^2}$ for $-1 \le x \le 1$.  If we restrict
ourselves to the part below the $x$-axis, we consider the function $y
:= -\sqrt{1 - x^2}$ for $-1 \le x \le 1$.

Now, how do we calculate $dy/dx$? Well, it depends on whether we are
interested in the part above the $x$-axis or in the part below the
$x$-axis. For the part above the $x$-axis, we have the function
$\sqrt{1 - x^2}$, and we get that the derivative is:

$$\frac{d(\sqrt{1 - x^2})}{dx} = \frac{d(\sqrt{1 - x^2})}{d(1 - x^2)} \frac{d(1 - x^2)}{dx} = \frac{1}{2\sqrt{1 - x^2}} \cdot (2x) = \frac{-x}{\sqrt{1 - x^2}}$$

If we are interested in the lower side, we get $x/\sqrt{1 - x^2}$.

Now, in this case, we have to split into two cases, and do a painful
calculation involving differentiating a square root via the chain rule.

Here's another way of handling this differentiation, that does not
involve a messy square root.

We start with the original expression:

$$x^2 + y^2 = 1$$

This is an {\em identity}, which means that it's true for every point
on the curve. When we have an equation that is identically true, it is
legitimate to differentiate both sides and still get an
identity. Differentiating both sides with respect to $x$, we get:

$$\frac{d(x^2)}{dx} + \frac{d(y^2)}{dx} = 0$$

Simplifying and using the chain rule, we get:

$$2x + 2y\frac{dy}{dx} = 0$$

We thus get:

$$\frac{dy}{dx} = \frac{-x}{y}$$

Notice that with this method, we get $-x/y$, which works in {\em both}
cases. When $y = \sqrt{1 - x^2}$, we get $-x/\sqrt{1 - x^2}$, and when
$y = -\sqrt{1 - x^2}$, we get $x/\sqrt{1 - x^2}$. The method that we
used is called {\em implicit differentiation}.

So the idea of implicit differentiation is that, instead of writing $y
= f(x)$ and then differentiating both sides, we differentiate the
messy mixed-up expression on both sides with respect to $x$. Next, we
use the various rules (sum rule, difference rule, product rule,
quotient rule) to keep splitting things up into smaller and smaller
pieces, and in the final analysis, we get everything in terms of $x$,
$y$, and $dy/dx$. Then, we try to separate $dy/dx$ completely to one
side.

Let's look at another example:

$$\sin(x + y) = xy$$

So, what we do is differentiate both sides:

$$\frac{d(\sin(x + y))}{dx} = \frac{d(xy)}{dx}$$

Now, how would we handle something like $\sin(x + y)$? It is something
in terms of $x + y$, so we use the chain rule on the left side,
thinking of $v = x + y$ as the intermediate function:

$$\frac{d(\sin(x + y))}{d(x + y)} \frac{d(x + y)}{dx} = x \frac{dy}{dx} + y \frac{dx}{dx}$$

This simplifies to:

$$ \cos (x + y) \left[1 + \frac{dy}{dx}\right] = x \frac{dy}{dx} + y$$

Opening up the parentheses, we get:

$$\cos(x + y) + \cos(x + y) \frac{dy}{dx} = x \frac{dy}{dx} + y$$

Now, we move stuff together to one side, to get:

$$(\cos(x + y) - x)\frac{dy}{dx} = y - \cos(x + y)$$

And we now isolate $dy/dx$:

$$\frac{dy}{dx} = \frac{y - \cos(x + y)}{\cos(x + y) - x}$$

\subsection{Implicit differentiation: understood better}

So, in implicit differentiation, what we're doing is, instead of
thinking of an explicit functional form, we are using a relation that
is true for every point in the curve, then {\em differentiating both
sides}. Next, we keep trying to simplify the expression we have using
the various rules until we land up with something that just involves
$x$, $y$, and $dy/dx$. Till this point, it's usually smooth
sailing. Now, it may be the case that we can {\em isolate} $dy/dx$ and
hence get an expression for it in terms of $x$ and $y$. If that's the
case, then we're in good shape.

Note the following key difference: when $y$ is an explicit function of
$x$, then the expression we get for $dy/dx$ only involves $x$ and does
not have the letter $y$ appearing in it. However, in the implicit
case, the expression we get for $dy/dx$ involves both $x$ and $y$
together.

\subsection{Higher derivatives using implicit differentiation}

We can also use implicit differentiation to compute second derivatives
and higher derivatives. Here's what we do. First, we get the
expression for $dy/dx$. In other words, we write:

$$\frac{dy}{dx} = \text{Some expression in terms of $x$ and $y$}$$

We now differentiate both sides with respect to $x$. Again, this
differentiation is valid because the above relation holds as an
identity, and not just as an isolated point.

The left side becomes $d^2y/dx^2$. For the right side, we again use
the same idea: we split as much as possible using the sum rule,
product rule, etc. For the expressions that purely involve $x$, we
differentiate the usual way. For the expressions that purely involve
$y$, we differentiate with respect to $y$ and multiply by $dy/dx$.
The upshot is that we get:

$$\frac{d^2y}{dx^2} = \text{Some expression in terms of $x$, $y$, and $\frac{dy}{dx}$}$$

Now, we plug back the earlier expression for $dy/dx$ in terms of $x$
and $y$ into this expression, and get an expression for $d^2y/dx^2$.

\end{document}

