\documentclass[10pt]{amsart}
\usepackage{fullpage,hyperref,vipul, graphicx}
\title{Trigonometric limits and derivatives}
\author{Math 152, Section 55 (Vipul Naik)}

\begin{document}
\maketitle

{\bf Corresponding material in the book}: Section 3.6.

{\bf Difficulty level}: Easy to moderate, particularly if you remember
corresponding stuff from AP level calculus.

{\bf Covered in class?}: Yes, but not necessarily all examples.

{\bf What students should definitely get}: The key trigonometric
limits. The key differentiation formulas for trigonometric functions.

{\bf What students should eventually get}: Techniques for computing
limits and derivatives involving composites of trigonometric functions
with each other and with polynomial and rational functions.

\section*{Executive summary}

Words ...

\begin{enumerate}
\item The following three important limits form the foundation of
  trigonometric limits: $\lim_{x \to 0} (\sin x)/x = 1$, $\lim_{x \to
  0} (\tan x)/x = 1$, and $\lim_{x \to 0} (1 - \cos x)/x^2 = 1/2$.
\item The derivative of $\sin$ is $\cos$, the derivative of $\cos$ is
  $-\sin$. The derivative of $\tan$ is $\sec^2$, the derivative of
  $\cot$ is $-\csc^2$, the derivative of $\sec$ is $\sec \cdot \tan$,
  and the derivative of $\csc$ is $-\csc \cdot \cot$.
\item The second derivative of any function of the form $x \mapsto
  a\sin x + b \cos x$ is the negative of that function, and the fourth
  derivative is the original function.
\end{enumerate}

Actions ...

\begin{enumerate}
\item Substitution is one trick that we use for trigonometric limits:
  we translate $\lim_{x \to c}$ to $\lim_{h \to 0}$ where $x = c + h$.
\item Multiplicative splitting, chaining, and stripping are some
  further tricks that we often use.
\item For derivatives of functions that involve composites of
  trigonometric and polynomial functions, we {\em have} to use the
  chain rule as well as rules for sums, differences, products, and
  quotients when simplifying expressions.
\end{enumerate}

\section{Some critical trigonometric limits}

\subsection*{Sinning by degrees is costly}

For all applications of trigonometry to limits and calculus, {\em all
angles are expressed in radians}. The radian measurement is the
natural measurement for an angle.

\subsection{The sinc function}

There are many other minor matters related to trigonometric functions
that we need to address, but for now, let's get back and focus on one
very important function -- the so-called sinc function. This function
is defined as $(\sin x)/x$ for $x \ne 0$.

Now, the function isn't defined at $x = 0$, and it isn't immediately
clear what the limit is. Because when we just try to substitute the
value at $0$, we get a $0/0$ form. So it seems we need some angel to
come and help us out.

Anyway, here's what the angel tells us:

$$\lim_{x \to 0} \frac{\sin x}{x} = 1$$

\subsection{Demystifying our angel}

For an acute angle $\theta$, $\sin \theta$ is the vertical height (the
$y$-coordinate of the point we get after rotating counter-clockwise by
an angle of $\theta$ from $(1,0)$ along the unit circle, while
$\theta$ is the arc length. You can see, from this picture, that the
arc length $\theta$ is greater than $\sin \theta$. That's because
$\sin \theta$ falls straight down while the arc moves both
horizontally and vertically. However, and this is the crucial point --
as $\theta$ gets smaller and smaller, you see that the arc length and
the vertical line seem to get closer and closer. And this suggests
that, perhaps, as $\theta$ tends to zero, $\sin \theta/\theta$ tends
to $1$.

Now, this is hardly a proof, because two numbers getting really close
does not necessarily mean that their ratio tends to $1$. But it is
suggestive. So, with this suggestivity, let's believe that the limit,
as $x$ tends to $0$, of the fraction $\sin x / x$ is $1$.

\subsection{The substitution idea}

We know that $\lim_{x \to 0} (\sin x)/x = 1$. More generally, it is
true that if $f$ is continuous at $c$ and $f(c) = 0$, then as $x \to c$,
we have:

$$\lim_{x \to c} \frac{\sin(f(x))}{f(x)} = 1$$

For instance:

$$\lim_{x \to 0} \frac{\sin(x^2)}{x^2} = 1$$

Similarly:

$$\lim_{x \to 0} \frac{\sin(2x)}{2x} = 1$$

and:

$$\lim_{x \to 0} \frac{\sin(x/2)}{x/2} = 1$$

On the other hand, if we consider the limit:

$$\lim_{x \to 0} \frac{\sin(x + (\pi/3))}{x + (\pi/3)}$$

This limit is not $1$ -- the inner expression does {\em not} go to $0$
as $x$ goes to $0$.
\subsection{Chaining limit computations}

Let's consider the computation:

$$\lim_{x \to 0} \frac{\sin(\sin x)}{x}$$

This is a special case of a more general limit computation that you
have seen in Question 2 of the October 4 quiz. Let's first do this
specific example, and then return to how it relates to that question.

For the specific example, we note that the limit in question involves
a composite function. For such problems, we typically {\em chain} the
limit by multiplying and dividing by the inner function. We get:

$$\lim_{x \to 0} \frac{\sin(\sin x)}{\sin x} \frac{\sin x}{x}$$

We now use that the limit of products is the product of limits, and
obtain:

$$\lim_{x \to 0} \frac{\sin(\sin x)}{\sin x} \lim_{x \to 0} \frac{\sin x}{x}$$

The second limit is clearly $1$. The first limit is also $1$, because
it is of the form $[\sin(f(x))]/f(x)$ where $f(x) \to 0$.

Now, the quiz question was that if $\lim_{x \to 0} g(x)/x = A \ne 0$
with $g$ continuous, then what is $\lim_{x \to 0} g(g(x))/x$?

The same chaining idea applies:

$$\lim_{x \to 0} \frac{g(g(x))}{x} = \lim_{x \to 0} \frac{g(g(x))}{g(x)} \frac{g(x)}{x}$$

We again split the limit multiplicatively, and argue that both
component limits are $A$. For one of them, we have to argue that as $x
\to 0$, $g(x) \to 0$ -- an argument that we make in a somewhat
indirect fashion. Go back to the quiz solution for more.

\subsection{Easier chainings}

Here are some easier examples:

$$\lim_{x \to 0} \frac{\sin(mx)}{x} = m$$

Here, we chain via $mx$.

$$\lim_{x \to 0} \frac{\sin(mx^n)}{x^n} = m$$

where $n$ is positive.

\subsection{The $(1 - \cos x)/x^2$ limit}

We now show another fundamentally important trigonometric limit:

$$\lim_{x \to 0} \frac{1 - \cos x}{x^2} = \frac{1}{2}$$

We first show how to obtain this limit by multiplying both numerator
and denominator by $1 + \cos x$. We get:

$$\lim_{x \to 0} \frac{1 - \cos^2 x}{x^2(1 + \cos x)}$$

The $1 + \cos x$ in the denominator pulls out by evaluation, and we get:

$$\frac{1}{2} \lim_{x \to 0} \frac{1 - \cos^2 x}{x^2}$$

We now use $1 - \cos^2x = \sin^2x$ and get:

$$\frac{1}{2}\lim_{x \to 0} \frac{\sin^2x}{x^2}$$

This becomes:

$$\frac{1}{2} \lim_{x \to 0} \left(\frac{\sin x}{x}\right)^2$$

The limit of the square is the square of the limit (basically, a
special case of the fact that the limit of the product is the product
of the limit, so the inner limit is $1$, and we get a $1/2$.

We can also obtain this limit using a double angle formula as
described below:

The idea here is to use the identity we saw last time, which was that
$\cos 2A = 1 - 2\sin^2 A$. So $1 - \cos 2A = 2\sin^2 A$. Here $A =
x/2$, so we get:

$$\lim_{x \to 0} \frac{2\sin^2(x/2)}{x^2}$$

We can pull the $2$ out, and we get $\sin^2(x/2)/x^2$ inside. Now, the
thing with calculating these limits is that we only know how to
calculate the limit of the form $\sin \theta/\theta$, and here,
$\theta = x/2$. So, we rewrite the denominator as $4(x/2)^2$, and we
pull out the $4$, so we get:

$$\frac{1}{2} \lim_{x \to 0} \left(\frac{\sin(x/2)}{(x/2)}\right)^2$$

Now, using the {\em limit of product equals product of limits} meme, we get:

$$\frac{1}{2} \lim_{x \to 0} \frac{\sin(x/2)}{(x/2)} \lim_{x \to 0} \frac{\sin(x/2)}{(x/2)}$$

Now, in both cases, we have $x \to 0$, so $x/2 \to 0$, so both limits
are $1$, and hence, our final answer is $1/2$.

\subsection*{The $\tan x/x$ limit}

Let's now calculate the limit:

$$\lim_{x \to 0} \frac{\tan x}{x} $$

What we do is to write $\tan x = \sin x/\cos x$:

$$\lim_{x \to 0} \frac{\sin x}{x \cos x}$$

We split this as a product:

$$\lim_{x \to 0} \frac{1}{\cos x} \lim_{x \to 0} \frac{\sin x}{x}$$

Both limits are $1$, so the overall limit is $1$.

\subsection{Corollaries}

We can now state some easy corollaries of the above results:

$$\lim_{x \to 0} \frac{1 - \cos(mx)}{x^2} = \frac{m^2}{2}$$

We obtain by chaining via $(mx)^2$.

Similarly:

$$\lim_{x \to 0} \frac{\tan(mx)}{x} = m$$

\subsection{Substitution that involves translation}

There is a {\em substitution of variables} trick that we can use for
computing limits. When we were computing limits for rational
functions, we never really needed that trick, primarily because we
knew how to handle rational functions anyway. But this trick comes in
useful for trigonometric functions. 

The trick is:

$$\lim_{x \to c} f(x) = \lim_{h \to 0} f(c + h)$$

Similarly:

$$\lim_{x \to c^+} f(x) = \lim_{h \to 0^+} f(c + h)$$

and:

$$\lim_{x \to c^-} f(x) = \lim_{h \to 0^-} f(c + h) = \lim_{h \to 0^+} f(c - h)$$

Why is this useful for trigonometric functions? Because for
trigonometric functions, the {\em only} nontrivial limit that we know
is the one I just told you: $\lim_{x \to 0} \frac{\sin x}{x} = 1$. So,
we need to basically use this for any nontrivial limit that we need to
compute. For instance, consider the limit:

$$\lim_{x \to \pi/4} \frac{\sin x - (1/\sqrt{2})}{x - \pi/4}$$

Now, we want to change the thing that's limiting to $h$, so we rewrite
this as:

$$\lim_{h \to 0} \frac{\sin(\pi/4 + h) - (1/\sqrt{2})}{h}$$

We simplify the numerator using the $\sin(A + B)$ formula, which we
know is $\sin A \cos B + \cos A \sin B$, and we get:

$$\lim_{h \to 0} \frac{(1/\sqrt{2})(\sin(h) + \cos(h) - 1)}{h}$$

We take out the $1/\sqrt{2}$ factor and now try to split the inner
limit additively:

$$\frac{1}{\sqrt{2}}\left( \lim_{h \to 0} \frac{\sin h}{h} - \lim_{h \to 0} \frac{1 - \cos h}{h} \right)$$

The first limit is $1$. For the second limit, it can be written as
$\lim_{h \to 0} h \cdot \lim_{h \to 0} (1 - \cos h)/h^2 = 0(1/2) =
0$. So the second limit is $0$. Overall, the limit is $1/\sqrt{2}$.

We'll see this same calculation a little later when we try to
calculate the derivative of the $\sin$ function.

\section{Stripping: a sneak peek}

We will cover this somewhat delicate operation a little later in 153,
when students are more mature (in terms of having seen more kinds of
functions) and can handle the issue with the requisite care. But some
of you may be equipped to use this approach to compute trigonometric
limits more intuitively and rapidly now. [NOTE: To my knowledge, this
particular intuitive approach to stripping is not found in any high
school or college calculus text and can be considered my own
invention, though most people who compute limits on a regular basis do
this all the time.]

To motivate stripping, let us look at a fancy example:

$$\lim_{x \to 0} \frac{\sin(\tan(\sin x))}{x}$$

This is a composite of three functions, so if we want to chain it, we
will chain it as follows:

$$\lim_{x \to 0} \frac{\sin(\tan(\sin x))}{\tan(\sin x)}\frac{\tan(\sin x)}{\sin x}\frac{\sin x}{x}$$

We now split the limit as a product, and we get:

$$\lim_{x \to 0} \frac{\sin(\tan(\sin x))}{\tan(\sin x)}\lim_{x \to 0} \frac{\tan(\sin x)}{\sin x}\lim_{x \to 0} \frac{\sin x}{x}$$

Now, we argue that each of the inner limits is $1$. The final limit is
clearly $1$. The middle limit is $1$ because the inner function $\sin
x$ goes to $0$. The left most limit is $1$ because the inner function
$\tan(\sin x)$ goes to $0$. Thus, the product is $1 \times 1 \times 1$
which is $1$.

If you are convinced, you can further convince yourself that the same
principle applies to a much more convoluted composite:

$$\lim_{x \to 0} \frac{\sin(\sin(\tan(\sin(\tan(\tan x)))))}{x}$$

However, {\em writing that thing out takes loads of time}. Wouldn't it
be nice if we could just strip off those $\sin$s and $\tan$s? In fact,
we can do that.

The key stripping rule is this: {\em in a multiplicative situation}
(i.e. there is no addition or subtraction happening), if we see
something like $\sin(f(x))$ or $\tan(f(x))$, and $f(x) \to 0$ in the
relevant limit, then we can strip off the $\sin$ or $\tan$. In this
sense, both $\sin$ and $\tan$ are {\em strippable} functions. A
function $g$ is strippable if $\lim_{x \to 0} g(x)/x = 1$.

The reason we can strip off the $\sin$ from $\sin(f(x))$ is that we
can multiply and divide by $f(x)$, just as we did in the above
examples.

Stripping can be viewed as a special case of the l'Hopital rule as
well, but it's a much quicker shortcut in the cases where it works.

Thus, in the above examples, we could just have stripped off the
$\sin$s and $\tan$s all the way through.

Here's another example:

$$\lim_{x \to 0} \frac{\sin(2 \tan (3x))}{x}$$

As $x \to 0$, $3x \to 0$, so $2 \tan 3x \to 0$. Thus, we can strip off
the outer $\sin$. We can then strip off the inner $\tan$ as well,
since its input $3x$ goes to $0$. We are thus left with:

$$\lim_{x \to 0} \frac{2(3x)}{x}$$

Cancel the $x$ and get a $6$. We could also do this problem by
chaining or the l'Hopital rule, but stripping is quicker and perhaps
more intuitive.

Here's yet another example:

$$\lim_{x \to 0} \frac{\sin (x \sin (\sin x))}{x^2}$$

As $x \to 0$, $x \sin(\sin x) \to 0$, so we can strip off the
outermost $\sin$ and get:

$$\lim_{x \to 0} \frac{x \sin(\sin x)}{x^2}$$

We cancel a factor of $x$ and get:

$$\lim_{x \to 0} \frac{\sin(\sin x)}{x}$$

Two quick $\sin$ strips and we get $x/x$, which becomes $1$.

Yet another example:

$$\lim_{x \to 0} \frac{\sin(ax)\tan(bx)}{x}$$

where $a$ and $b$ are constants. Since this is a multiplicative
situation, and $ax \to 0$ and $bx \to 0$, we can strip the $\sin$ and
$\tan$, and get:

$$\lim_{x \to 0} \frac{(ax)(bx)}{x}$$

This limit becomes $0$, because there is a $x^2$ in the numerator and
a $x$ in the denominator, and cancellation of one factor still leaves
a $x$ in the numerator.

Here is yet another example:

$$\lim_{x \to 0} \frac{\sin^2(ax)}{\sin^2(bx)}$$

where $a,b$ are nonzero constants. We can pull the square out of the
whole expression, strip the $\sin$s in both numerator and denominator,
and end up with $a^2/b^2$.

\subsection*{When you can't strip}

The kind of situations where we are not allowed to strip are where the
expression $\sin(f(x))$ is not just multiplied but is being added to
or subtracted from something else. For instance, in order to calculate
the limit:

$$\lim_{x \to 0} \frac{x - \sin x}{x^3}$$

stripping off the $\sin$ would be a sin. The intuition behind what is
wrong with this will have to wait till next quarter.

\subsection*{What about $1 - \cos(f(x))$?}

If we see something like $1 - \cos(f(x))$ with $f(x) \to 0$ in the
limit, then we know that it ``looks like'' $f(x)^2/2$ and we can
replace it by $f(x)^2/2$. This can be thought of as a sophisticated
version of stripping. For instance:

$$\lim_{x \to 0} \frac{(1 - \cos(5x^2))}{x^4}$$

Here $f(x) = 5x^2$, so the numerator is like $(5x^2)^2/2$, and the
limit just becomes $25/2$.

\section{Differentiating the trigonometric functions}

\subsection{Differentiation formulas}

The differentiation formulas are as follows:

\begin{eqnarray*}
  \sin' & = & \cos\\
  \cos' & = & -\sin \\
  \tan' & = & \sec^2\\
  \sec' & = & \sec \cdot \tan \\
  \cot' & = & -\csc^2\\
  \csc' & = & -\csc \cdot \cot
\end{eqnarray*}

We can obtain all these just from the differentiation formulas for
$\sin$ and $\cos$, using the quotient rule:

$$\frac{d}{dx}[\sin x] = \cos x$$

and:

$$\frac{d}{dx}[\cos x] = \sin x$$

In fact, the differentiation formula for $\cos$ can be obtained from
that of $\sin$ using the chain rule as follows:

$$\frac{d}{dx}[\cos x] = \frac{d}{dx}[\sin((\pi/2) - x)] = -\cos((\pi/2) - x) = - \sin x$$

So, in order to obtain all these formulas, all we need to do is obtain
the differentiation formula for $\sin$. The derivation of this formula
is given below. Essentially, it uses the fact that $\lim_{h \to 0}
(\sin h)/h = 1$:

\begin{align*}
  & \lim_{h \to 0} \frac{\sin(x + h) - \sin x}{h} \\
  = & \lim_{h \to 0} \frac{\sin x \cos h + \cos x \sin h - \sin x}{h}\\
  = & \sin x \lim_{h \to 0} \frac{\cos h - 1}{h} + \cos x \lim_{h \to 0} \frac{\sin h}{h}\\
  = & \sin x \cdot 0 + \cos x \cdot 1 \\
  = & \cos x
\end{align*}

The fact that $(\cos h - 1)/h \to 0$ can be deduced by multipled by $1
+ \cos h$ and simplifying. Basically, $\cos h - 1$ goes to $0$ at the
rate of $h^2$, and the denominator is only $h$.

\subsection{Graphical interpretation of these derivatives}

\includegraphics[width=3in]{sinecosinegraphs.png}

The fascinating thing about the $\sin$ function is that its derivative
looks just like the function -- except it's shifted over by
$\pi/2$. Let's see what this means graphically.

At the point $x = 0$, the derivative is $\cos 0 = 1$. This means that
the tangent is the line $y = x$. And indeed, close to the point $0$,
the sine graph looks a lot like the line $y = x$. However, there's an
added subtlety that you may not be able to appreciate right now but
will be able to later. The graph of the sine function on the right of
the point $0$ falls slightly on the lower side of the $y = x$ line,
and the graph to the left of $0$ falls slightly on the upper side of
the line. In other words, the curve actually passes from one side of
the tangent line at $0$ to the other side.

This is a very unusual situation, because most of the time that you
think of tangent lines, the graph of the function {\em close to the
point} lies entirely on one side of the tangent line. But here, the
graph is {\em crossing} the tangent line. These kinds of points are
called {\em points of inflection}, and believe it or not, for a twice
differentiable function, a point of inflection is a point where the
{\em second} derivative is zero (though there could be points where
the second derivative is zero that are not points of inflection --
funny things could happen with the {\em third} derivative). We'll talk
more about inflections later.

Anyway, so at zero, the derivative is $1$. What happens as $x$
increases from $0$ to $\pi/2$? The cosine function keeps decreasing
form $1$ to $0$. But since the cosine function is positive, the
tangent to the sine function is still upward-sloping, so the sine
function is increasing, but the slope of the tangent line is
falling. Finally at $x = \pi/2$, the sine function reaches its peak of
$1$, and its derivative, the cosine function, becomes $0$.

So here's the beautiful things you see, and for which you now have a
powerful explanation based on the derivative:

\begin{enumerate}
\item The points where the sine function attains its maxima and minima
  are precisely the points where the cosine function is $0$. Namely,
  these points are odd multiples of $\pi/2$. The points where the
  cosine function attains its maxima and minima are precisely the
  points where the sine function is $0$. Namely, these points are the
  multiples of $\pi$.
\item The regions where the sine function is increasing (respectively,
  decreasing) are the same as the regions where the cosine function is
  positive (respectively, negative). The regions where the cosine
  function is increasing (respectively, decreasing) are the same as
  the regions where the sine function is negative (respectively,
  positive).
\end{enumerate}

\subsection{We've done it!}

What I've given you is the complete toolkit with which you can
calcuate any derivative that involves a mix of trigonometric,
polynomial, and rational functions. Let's consider an example:
$\sin(x^2)$.

How do you think of this function? To evaluate this function at a
particular value of $x$, what you first do is calculate $x^2$, and
then apply the $\sin$ function to that. So, you're doing two
operations in sequence, whereby, you're feeding the output of one
operation as the input of the other. So what you're doing is
essentially {\em function composition}.

Which brings us back to the chain rule. Remember the way the chain
rule works? We set $v = x^2$, first differentiate the function with
respect to $v$, and then multiply by $dv/dx$. So, we get that the
derivative is $(\cos(x^2)) \cdot 2x$.

For composites of trigonometric functions and polynomial, the chain
rule is {\em indispensable} -- if you don't want to use the chain
rule, there's no way of calculating the derivative without {\em
actually calculating it using the limit definition}. In the case of
polynomials and rational functions, the chain rule was a
convenience. Now it's a necessity.

\subsection{Periodicity of the sequence of derivative functions}

The derivative of $\sin$ is $\cos$. The derivative of $\cos$ is
$-\sin$. The derivative of $-\sin$ is $-\cos$. The derivative of
$-\cos$ is $\sin$.

In other words, the $\sin$ function equals its {\em fourth}
derivative. If we consider the sequence of derivative functions of
$\sin$, this {\em sequence} has a {\em period} of four. (We'll
formally define sequence and period of a sequence in 153, but you know
what this means). Thus, for instance, the $101^{th}$ derivative of
$\sin$ is the same as the first derivative (because the remainder on
dividing $101$ by $4$ is $1$) and is thus $\cos$.

[Aside: This is significant in many ways. For instance, when we study
antidifferentiation (indefinite integration) we'll notice that the
above basically tells us that we can keep taking antiderivatives of
$\sin$ or $\cos$ and still remain $\sin$ or $\cos$ (up to a plus or
minus). This is significant when, for instance, we study integration
by parts. There, we will choose $\sin$ or $\cos$ as the {\em part to
repeatedly integrate} precisely because repeated integration does not
increase the complexity of the function.]

\subsection{Second derivative same with a minus sign}

Also note that for the $\sin$ function, the second derivative is the
{\em negative} of the function. This is also true for the $\cos$
function. It is also true of all {\em linear combinations} of $\sin$
and $\cos$, i.e., all functions of the form $x \mapsto a \sin x + b
\cos x$ where $a, b \in \R$.\footnote{Formally, this is the vector
space of functions generated by $\sin$ and $\cos$.} For any function
$f$ of this form, $f''(x) = -f(x)$.

What's so special about this fact? {\em This is the reason why
trigonometric functions, specifically linear combinations of $\sin$
and $\cos$, arise in nature.} What happens is that some basic
physical/biological/chemical/ecological law or constraint forces the
solution function we have to satisfy the equation $f''(x) =
-\omega^2f(x)$ or something like that. Now, functions of the above
form (with appropriate scaling) pop up naturally. More on this when we
study differential equations.

\end{document}
