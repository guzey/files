\documentclass[10pt]{amsart}

%Packages in use
\usepackage{fullpage, hyperref, vipul, enumerate}

%Title details
\title{Class quiz solutions: October 14: Derivatives}
\author{Math 152, Section 55 (Vipul Naik)}
%List of new commands

\begin{document}
\maketitle

\section{Performance review}

$11$ people took this quiz. Everybody got all questions correct!

The problem wise answers and performance review are below:
\begin{enumerate}
\item Option (E): Everybody
\item Option (C): Everybody
\item Option (D): Everybody
\item Option (D): Everybody
\end{enumerate}

Good job!

\section{Solutions}
\begin{enumerate}

\item Suppose $f$ and $g$ are functions from $\R$ to $\R$ that are
  everywhere differentiable. Which of the following functions is/are
  guaranteed to be everywhere differentiable?

  \begin{enumerate}[(A)]

  \item $f + g$
  \item $f - g$
  \item $f \cdot g$
  \item $f \circ g$
  \item All of the above
  \end{enumerate}

  {\em Answer}: Option (E)

  {\em Explanation}: In fact, we have explicit formulas for the
  derivatives of all of these in terms of the derivatives of $f$ and
  $g$. We have $(f + g)' = f' + g'$ and $(f - g)' = f' - g'$. We also
  have the product rule and chain rule for options (C) and (D).

  Note that for the composition, we are using something more: since
  these are functions on the whole real line $\R$, the value $g(x)$
  also lies in the domain of $f$, hence it makes sense to compose.

  {\em Performance review}: Everybody got this correct

  {\em Historical note (last year)}: $13$ out of $14$ people got this
  correct. $1$ person chose a multitude of options. {\em Note: The
  answer should always be exactly one option.}

\item Suppose $f$ and $g$ are both twice differentiable functions
  everywhere on $\R$. Which of the following is the correct formula
  for $(f \cdot g)''$?

  \begin{enumerate}[(A)]
  \item $f'' \cdot g + f \cdot g''$
  \item $f'' \cdot g + f' \cdot g' + f \cdot g''$
  \item $f'' \cdot g + 2f' \cdot g' + f \cdot g''$
  \item $f'' \cdot g - f' \cdot g' + f \cdot g''$
  \item $f'' \cdot g - 2f' \cdot g' + f \cdot g''$
  \end{enumerate}

  {\em Answer}: Option (C)

  {\em Explanation}: We differentiate once to get:

  $$(f \cdot g)' = f' \cdot g + f \cdot g'$$

  Now we differentiate both sides. The left side becomes $(f \cdot
  g)''$. The right side is a sum of two terms, so we get:

  $$(f \cdot g)'' = (f' \cdot g)' + (f \cdot g')'$$

  We now apply the product rule to each piece on the right side to get:

  $$(f \cdot g)'' = [f'' \cdot g + f' \cdot g'] + [f' \cdot g' + f \cdot g'']$$

  Combining terms, we get option (C).

  {\em Remark}: In general, there is a binomial theorem-like formula
  for the $n^{th}$ derivative of $f \cdot g$. I've given the formula
  below, but it will make sense only to people who have seen summation
  notation and the binomial coefficients, which we have not yet done:

  $$(f \cdot g)^{(n)} = \sum_{k=0}^n \binom{n}{k} f^{(k)}g^{(n - k)}$$

  This is a lot like the binomial theorem expansion for $(a +
  b)^n$. It can be proved purely formally using induction from the
  product rule.\footnote{One of the things I'm doing research on has
  to do with the fact above, albeit with a completely different notion
  of differentiation.}

  {\em Performance review}: Everybody got this correct

  {\em Historical note (last year)}: $13$ out of $14$ people got this
  correct. $1$ person chose option (E), though that person's rough work
  gave option (C).
\item Suppose $f$ and $g$ are both twice differentiable functions
  everywhere on $\R$. Which of the following is the correct formula
  for $(f \circ g)''$?

  \begin{enumerate}[(A)]

  \item $(f'' \circ g) \cdot g''$
  \item $(f'' \circ g) \cdot (f' \circ g') \cdot g''$
  \item $(f'' \circ g) \cdot (f' \circ g') \cdot (f \circ g'')$
  \item $(f'' \circ g) \cdot (g')^2 + (f' \circ g) \cdot g''$
  \item $(f' \circ g') \cdot (f \circ g) + (f'' \circ g'')$
  \end{enumerate}

  {\em Answer}: Option (D)

  {\em Explanation}: This question is tricky because it requires the
  application of both the product rule and the chain rule, with the
  latter being used twice. We first note that:

  $$(f \circ g)' = (f' \circ g) \cdot g'$$

  Now, we differentiate both sides:

  $$(f \circ g)'' = [(f' \circ g) \cdot g']'$$

  The expression on the right side that needs to be differentiated is
  a product, so we use the product rule:

  $$(f \circ g)'' = [(f' \circ g)' \cdot g'] + [(f' \circ g) \cdot g'']$$

  Now, the inner composition $f' \circ g$ needs to be
  differentiated. We use the chain rule and obtain that $(f' \circ g)'
  = (f'' \circ g) \cdot g'$. Plugging this back in, we get:

  $$(f \circ g)'' = (f'' \circ g) \cdot (g')^2 + (f' \circ g) \cdot g''$$

  {\em Remark}: What's worth noting here is that in order to
  differentiate composites of functions, you need to use both
  composites {\em and} products (that's the chain rule). And in order
  to differentiate products, you need to use both products {\em and}
  sums (that's the product rule). Thus, in order to differentiate a
  composite twice, we need to use composites, products, {\em and}
  sums.

  {\em Performance review}: Everybody got this correct

  {\em Historical note (last year)}: $14$ out of $14$ people got this
  correct. This is great! I had expected that many of you would be put
  off by the messy computation, but apparently you were unfazed.

\item Suppose $f$ is an everywhere differentiable function on
  $\R$ and $g(x) := f(x^3)$. What is $g'(x)$?

  \begin{enumerate}[(A)]
  \item $3x^2f(x)$
  \item $3x^2f'(x)$
  \item $3x^2f(x^3)$
  \item $3x^2f'(x^3)$
  \item $f'(3x^2)$
  \end{enumerate}

  {\em Answer}: Option (D)

  {\em Explanation}: Put $h(x) := x^3$. Then $g = f \circ h$. Thus,
  $g'(x) = f'(h(x))h'(x) = f'(x^3) \cdot (3x^2)$, giving option (D).

  {\em Performance review}: Everybody got this correct

  {\em Historical note (last year)}: $13$ out of $14$ people got this
  correct. $1$ person chose option (B), which is a close distractor if
  you're not paying attention.
\end{enumerate}

\end{document}