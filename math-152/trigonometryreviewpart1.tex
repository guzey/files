\documentclass{amsart}
\usepackage{fullpage,hyperref,vipul,graphicx}
\title{Trigonometry review (part 1)}
\author{Math 152, Section 55 (Vipul Naik)}

\begin{document}
\maketitle

{\bf Difficulty level}: Easy to moderate, given that you are already
familiar with trigonometry.

{\bf Covered in class?}: Probably not (for the most part). Some small
segments may be covered in class or in problem session if it helps
with some problems. Please go through this if you experience
difficulties while doing trigonometry problems.

{\bf Corresponding material in the book}: Section 1.6 (part).

{\bf Corresponding material in homework problems}: Homework 1,
advanced homework problem 6.

\section{Trigonometric functions for acute angles}

Earlier we talked about the fact that sometimes you know that
something is a function (because it sends every element in the domain
to a unique output -- and satisfies the condition that equal inputs
give equal outputs) but you don't have an expression for it. It's like
you know somebody is a person but you don't know that person's
name. So what do you do? You just make up a name. Well, that's what
we're going to do.

So, let's consider an angle, that I'll call $\theta$, and assume that
$\theta$ is {\em strictly} between $0$ and $\pi/2$ ($90\,^\circ$, a
right angle). By the way, the word {\em strictly} when used in
mathematics means that the equality case (the {\em trivial} or {\em
degenerate} case) is excluded. So, in this case, it means $0 < \theta
< \pi/2$. The high school term for an angle strictly between $0$ and
$\pi/2$ is {\em acute angle}.

So now I define the following function whose domain is the set of
acute angles. $f(\theta)$ is the ratio of the height of a right-angled
triangle with base angle $\theta$ to the hypotenuse of that
triangle. In other words, it is the ratio of the opposite side to the
angle $\theta$ to the hypotenuse.

So, you may say, why is this a function at all? Why does it make
sense? There are infinitely many triangles of different sizes with
base angle $\theta$. Could different choices of triangle give
different values of $f(\theta)$? And if so, doesn't that undermine the
claim that $f$ is a function? For instance, in the two triangles
$\triangle ABC$ and $\triangle DEF$, the base angles are equal. Should
the corresponding side ratios also be equal?

\includegraphics[width=3in]{righttriangleABC.png}

\includegraphics[width=2in]{righttriangleDEF.png}

By the way, this first question that I asked is the kind of question
you should ask whenever a {\em function} is defined in a roundabout
manner, with some arbitrary choices in between. People often phrase
this as {\em is the function well-defined?} though a more precise
formulation is: {\em is the so-called function a function at all?}

In this case, the answer is {\em yes}, and the reason is the notion of
{\em similarity} of triangles. For those of you who've taken some high
school geometry, you've probably seen this notion. For those who
haven't, the idea is that if two triangles have the same angles, they
essentially have the same {\em shape}, even if they have different
{\em sizes}, so the {\em ratios} of side lengths are the same.

So $f$ is a function. But so what? Do we have an expression for it?
Well, yes and no. There's no expression for $f$ as a polynomial or
rational function, because it isn't that kind of function. But we can
give $f$ its own name, and then we'll be happy. So what's a good name?
$f$? No, $f$, is too plain and all too common. We need a special
name. The name we use is sine, written {\em sine} in English and
$\sin$ in mathematics. So $f(\theta)$ is written as $\sin(\theta)$. By
the way, when the $\sin$ is being taken of a single letter variable or
constant, we don't usually put parentheses. So we just say $\sin
\theta$.

Okay, so what is the domain of the $\sin$ function as we've defined
it? It is $(0,\pi/2)$. What is the range? In other words, what are the
possible values that $\sin \theta$ can take? Well, think about it this
way. Think of a ladder that you have placed with one end touching a
vertical wall and the other end on the floor. Now, imagine this ladder
sliding down. When the ladder becomes horizontal, the triangle has base
angle of zero. When it's almost touched down, the base angle is really
small and the opposite side is, too. So $\sin \theta \to 0$ as $\theta
\to 0$. That $\to$ here means {\em tends to} or {\em approaches} --
it's a concept we'll be looking at in more detail when we do limits.

\includegraphics[width=3in]{slidingladder.png}

At the other extreme, when the ladder is almost upright, the opposite
side is almost equal to $1$, so $\sin \theta \to 1$ as $\theta \to
\pi/2$. So what we're seeing is that $\sin$ is an increasing function
starting off from just about the right of zero and ending at just
about the left of $1$. So the range of this function is $(0,1)$.

So here's one more point that is worth thinking about. In high school,
if you started looking at trigonometric functions before the radian
measure was introduced, then you might have seen that angles are
denoted differently, e.g., by Greek letters. Why's that? Well, one
reason to think of that is that angles aren't ordinary numbers. They
are measurements, and denoting an angle by a common letter like $x$ is
debasing, because angles come in {\em degrees}. But after you switch
to the radian measure, an angle (in radians) is just any old real
number. So we feel free to use $x$ and $y$ to denote angles. We've
demystified angles.

Now, let's recall the definitions of cosine. The cosine, denoted
$\cos$, is the ratio of the adjacent side to the hypotenuse. So if
$\theta$ is the base angle of a right triangle, $\cos \theta$ is the
ratio of the base to the hypotenuse.

And then there is the tangent function. This is denoted $\tan$, and it
is the ratio of the opposite side to the adjacent side. Now,
mathematicians can be very creative with naming sometimes but
sometimes they just copy names without much deep meaning. So this
tangent function doesn't have any deep relation with the concept of
{\em tangent} to a circle or a curve. Yes, they are loosely related,
but the relation isn't strong enough to merit the same name. Call that
an accident of history.

The other three trigonometric functions are the reciprocals of
these. The reciprocal of the sine function is the cosecant function,
denoted in mathematical shorthand as $\csc$. The reciprocal of the
cosine function is the secant function, denoted in mathematical
shorthand as $\sec$. The reciprocal of the tangent function is the
cotangent function, denoted in mathematical shorthand as $\cot$.

We summarize the important definitions here.

\includegraphics[width=3in]{righttriangleABC.png}

Next, we define the six trigonometric functions of $\theta$ as ratios
of side lengths in this triangle:

\begin{align*}
  \sin \theta & = \frac{\text{Opposite leg}}{\text{Hypotenuse}} = \frac{BC}{AC}\\
  \cos \theta & = \frac{\text{Adjacent leg}}{\text{Hypotenuse}} = \frac{AB}{AC}\\
  \tan \theta & = \frac{\text{Opposite leg}}{\text{Adjacent leg}} = \frac{BC}{AB}\\
  \cot \theta & = \frac{\text{Adjacent leg}}{\text{Opposite leg}} =\frac{AB}{BC}\\
  \sec \theta & = \frac{\text{Hypotenuse}}{\text{Adjacent leg}} = \frac{AC}{AB}\\
  \csc \theta & = \frac{\text{Hypotenuse}}{\text{Opposite leg}} =\frac{AC}{BC}
\end{align*}

\section{Relation between trigonometric functions}

The six trigonometric functions are related via three broad classes of
relationships. Each of these relationships pairs up the six
trigonometric functions into three pairs. We discuss each of these
pairings.

\subsection{Complementary angle relationships}

The right triangle $\triangle ABC$ has two acute angles. The ratios of
side lengths of this triangle give the trigonometric function values
for both acute angles. However, a leg that's opposite to one angle
becomes adjacent to the other. Thus, the trigonometric functions for
$(\pi/2) - \theta$ are related to the trigonometric functions for
$\theta$ as follows:

\begin{eqnarray*}
  \sin((\pi/2) - \theta) & = & \cos \theta\\
  \cos((\pi/2) - \theta) & = & \sin \theta\\
  \tan((\pi/2) - \theta) & = & \cot \theta\\
  \cot((\pi/2) - \theta) & = & \tan \theta\\
  \sec((\pi/2) - \theta) & = & \csc \theta\\
  \csc((\pi/2) - \theta) & = & \sec \theta
\end{eqnarray*}

The prefix {\em co-} indicates a complementary angle
relationship. Thus, the functions sine and {\em co}sine have a
complementary angle relationship. The functions tangent and {\em
co}tangent have a complementary angle relationship. The functions
secant and {\em co}secant have a complementary angle relationship.

It is an easy but useful exercise to verify the complementary angle
relationships from the definitions of the trigonometric functions.

\subsection{Reciprocal relationships}

Reciprocal relationships between the trigonometric functions are as follows:

\begin{enumerate}
\item $\sin \theta$ and $\csc \theta$ are reciprocals. In other words,
  $(\sin \theta)(\csc \theta) = 1$ for all acute angles $\theta$.
\item $\cos \theta$ and $\sec \theta$ are reciprocals. In other words,
  $(\cos \theta)(\sec \theta) = 1$ for all acute angles $\theta$.
\item $\tan \theta$ and $\cot \theta$ are reciprocals. In other words,
  $(\tan \theta)(\cot \theta) = 1$ for all acute angles $\theta$.
\end{enumerate}

It is an easy but useful exercise to verify the complementary angle
relationships from the definitions of the trigonometric functions.

\subsection{Square sum and difference relationships}

These are the trickiest and the most important of the
relationships. We consider the most important of these first: the
square sum relationship between $\sin$ and $\cos$.

By the Pythagorean theorem for the right triangle $\triangle ABC$ with
the angle at $B$ being the right angle, we have:

$$AB^2 + BC^2 = AC^2$$

Or, in terms of the angle $\theta$:

$$\text{(Adjacent leg)}^2 + \text{(Opposite leg)}^2 = \text{(Hypotenuse)}^2$$

Dividing both sides by $AC^2$, we obtain:

$$\frac{AB^2}{AC^2} + \frac{BC^2}{AC^2} = \frac{AC^2}{AC^2}$$

Simplifying, we obtain:

$$\left(\frac{AB}{AC}\right)^2 + \left(\frac{BC}{AC}\right)^2 = 1$$

Recall that $\cos \theta = AB/AC$ and $\sin \theta = BC/AC$, so we get:

$$(\cos \theta)^2 + (\sin \theta)^2 = 1$$

With trigonometric functions, it is a typical convention to write the
exponent before the angle, so we write $(\cos \theta)^n$ as
$\cos^n\theta$. Using this convention, we can rewrite the above relationship as:

$$\cos^2\theta + \sin^2\theta = 1$$

Two other square sum and difference relationships of importance are:

\begin{eqnarray*}
  \tan^2 \theta + 1 & = & \sec^2 \theta\\
  \cot^2 \theta + 1 & = & \csc^2 \theta
\end{eqnarray*}

It is a good exercise to prove both of these using the Pythagorean theorem.

\subsection{Everything in terms of $\sin$ and $\cos$}

It is often useful to deal with $\sin$ and $\cos$ only, so it is
helpful to know how to write the other trigonometric functions in
terms of $\sin$ and $\cos$. The expressions are given below:

It is a good exercise to verify that these expressions are correct
using the definitions of the trigonometric functions.

\begin{eqnarray*}
  \tan \theta & = & \frac{\sin \theta}{\cos \theta} \\
  \cot \theta & = & \frac{\cos \theta}{\sin \theta} \\
  \sec \theta & = & \frac{1}{\cos \theta}\\
  \csc \theta & = & \frac{1}{\sin \theta}
\end{eqnarray*}


\subsection{Everything in terms of $\sin$ or $\cos$}

Finally, when it comes to acute angles, we can write all the
trigonometric functions in terms of $\sin$ alone or $\cos$ alone. The
key is to use the fact that $\sin^2\theta + \cos^2 \theta = 1$. Since
both $\sin \theta$ and $\cos \theta$ are positive for an acute angle
$\theta$, we can use this to get the expressions:

\begin{eqnarray*}
  \cos \theta & = & \sqrt{1 - \sin^2 \theta}\\
  \sin \theta & = & \sqrt{1 - \cos^2 \theta}
\end{eqnarray*}

Once we have this, we can get expressions for all the other
trigonometric functions in terms of $\sin$. We can also get
expressions for all the other trigonometric functions in terms of
$\cos$.

We give here all the expressions in terms of $\sin$. In all these, we
just take the previos expressions and replace every occurrence of
$\cos \theta$ by $\sqrt{1 - \sin^2\theta}$.

\begin{eqnarray*}
  \cos \theta & = & \sqrt{1 - \sin^2\theta}\\
  \tan \theta & = & \frac{\sin \theta}{\sqrt{1 - \sin^2\theta}}\\
  \cot \theta & = & \frac{\sqrt{1 - \sin^2\theta}}{\sin \theta}\\
  \sec \theta & = & \frac{1}{\sqrt{1 - \sin^2 \theta}}\\
  \csc \theta & = & \frac{1}{\sin \theta}
\end{eqnarray*}

You can work out how the other five trigonometric functions look in
terms of $\cos \theta$ by yourself.

{\em Note that these expressions are valid only for acute angles}. The
problem is that for bigger angles, as we shall soon see, the value of
$\sin$ or $\cos$ can be negative, so even thought $\sin^2\theta +
\cos^2\theta = 1$, we cannot write $\sin \theta = \sqrt{1 -
\cos^2\theta}$ because the $\sqrt{}$ symbol always gives a nonnegative
output and $\sin \theta$ may well be negative.

\section{Unit circle trigonometry}

\subsection{The unit circle}

The unit circle centered at the origin is defined as the set of points
$(x,y)$ in the coordinate plane that satisfy $x^2 + y^2 = 1$. This is
a circle of radius $1$ centered at the origin.

The unit circle intersects the $x$-axis at the points $(1,0)$ and
$(-1,0)$. It intersects the $y$-axis at the points $(0,1)$ and
$(0,-1)$.

\includegraphics[width=3in]{unitcircle.png}

\subsection{Sine and cosine using the unit circle}

Suppose $\theta$ is an angle. We define $\sin \theta$ and $\cos
\theta$ using the unit circle as follows. Start on the unit circle at
the point $(1,0)$. Move an angle of $\theta$ in the counter clockwise
direction on the unit circle. Call the point you finally reach
$(x_0,y_0)$. Then, $\cos \theta$ is defined as $x_0$ and $\sin \theta$
is defined as $y_0$.

When $\theta$ is an acute angle, then the point $(x_0,y_0)$ is in the
first quadrant. We see that the definitions of $\cos \theta$ and $\sin
\theta$ match the definitions we gave earlier in terms of
triangles.

\section{Qualitative behavior of sine and cosine}

\subsection{For acute angles}

Note that prior to the introduction of unit circle trigonometry, we
defined $\sin$ and $\cos$ only for acute angles. We first discuss the
qualitative behavior of these functions for acute angles, but also
include the limiting cases of $0$ and $\pi/2$. After
that, we discuss the behavior for obtuse angles.

For acute angles, we have the following:

\begin{enumerate}
\item $\sin$ is a strictly increasing function for acute angles,
  starting off with $\sin 0 = 0$ and $\sin (\pi/2) =
  1$. This can be seen graphically in many ways. For instance, imagine
  a ladder that is initially vertical along a wall, and gradually
  slides down. The angle that the foot of the ladder makes with the
  floor decreases from $\pi/2$ to $0$, and the vertical
  height of the top of the ladder also decreases. [More class
  discussion on this]

  \includegraphics[width=3in]{slidingladder.png}

  \vspace{2in}
\item $\cos$ is a strictly decreasing function for acute angles,
  starting off with $\cos 0 = 1$ and $\cos (\pi/2) =
  0$. The fact that the behavior of $\cos$ is the mirror opposite of
  that of $\sin$ is not surprising -- this essentially follows from
  the complementary angle relationship.
  \vspace{1in}
\item $\tan$ is a strictly increasing function for acute angles,
  starting off with $\tan 0 = 0$. $\tan (\pi/2)$ is not
  defined, and as an acute angle $\theta$ comes closer and closer to
  being $\pi/2$, $\tan \theta$ approaches $\infty$. (The precise
  meaning and explanation of this statement involve familiarity with
  ideas of limits, which are beyond the current scope of discussion).
\item $\cot$ is a strictly decreasing function for acute angles, and
  its behavior mirrors that of $\tan$ because of the complementary
  angle relationship. $\cot 0$ is undefined and $\cot
  (\pi/2) = 0$.
\end{enumerate}

\subsection{For angles between $0$ and $\pi$}

Using the unit circle trigonometry definition, we can see that:

\begin{eqnarray*}
  \sin(\pi - \theta) & = & \sin \theta\\
  \cos(\pi - \theta) & = & -\cos \theta
\end{eqnarray*}

In particular, $\sin$ is positive for all angles $\theta$ that are
strictly between $0$ and $\pi$, including both acute
and obtuse angles. In terms of unit circle trigonometry, this is
because the first half of the unit circle if we go counter-clockwise
from $(1,0)$ has positive $y$-coordinate. On the other hand, $\cos$ is
negative for obtuse angles, and this can again be seen from the unit
circle trigonometry.

Note that $\tan \theta$ is defined as $\sin \theta/\cos \theta$
wherever $\cos \theta \ne 0$, so we get the formula:

\begin{equation*}
  \tan(\pi - \theta) = -\tan \theta
\end{equation*}

\subsection{The graphs}

Here are the graphs of the sine and cosine function from $0$
to $2\pi$.

\includegraphics[width=3in]{sinecosinegraphs.png}

We will study the graphs of the other trigonometric functions later.

\subsection{General facts: value taking}

\begin{enumerate}
\item $\sin$ takes its peak value of $1$ at numbers of the form
  $2n\pi + (\pi/2)$, where $n$ is an integer, and its trough values of
  $-1$ at numbers of the form $2n\pi - (\pi/2)$, where $n$ is an
  integer. $\sin$ takes the value $0$ at multiples of $\pi$, i.e.,
  numbers of the form $n\pi$ where $n$ is an integer.
\item The solutions to $\sin x = \sin \alpha$ come in two families: $x
  = 2n\pi + \alpha$ and $x = 2n\pi + (\pi - \alpha)$, with $n$ ranging
  over the integers.
\item $\cos$ takes its maximum value of $1$ at multiples of $2\pi$,
  i.e., numbers of the form $2n\pi$, $n$ ranging over integers. It
  takes its minimum value of $-1$ at odd multiples of $\pi$, i.e.,
  numbers of the form $(2n + 1)\pi$, $n$ ranging over integers. it
  takes the value $0$ at odd multiples of $\pi/2$, i.e., numbers of
  the form $n\pi + (\pi/2)$, $n$ ranging over integers.
\item The solutions to $\cos x = \cos \alpha$ come in two families: $x
  = 2n\pi + \alpha$ and $x = 2n\pi - \alpha$, $n$ ranging over
  integers. These two families can be combined compactly by writing
  the general expression as $x = 2n\pi \pm \alpha$.
\end{enumerate}

\subsection{Even, odd, mirror symmetry, half turn symmetry}

Here are some general facts about the sine and cosine functions:

\begin{enumerate}
\item The sine function is an {\em odd function}, i.e., $\sin(-\theta)
  = -\sin \theta$. In particular, the graph of the sine function
  enjoys a half turn symmetry about the origin. In fact, the graph of
  the sine function enjoys a half turn symmetry about any point of the
  form $(n\pi,0)$.
\item The graph of the sine function enjoys mirror symmetry about any
  vertical line through a peak or trough, i.e., any line of the form
  $x = n\pi + (\pi/2)$.
\item The cosine function is an {\em even function}, i.e.,
  $\cos(-\theta) = \cos \theta$. In particular, the graph of the
  cosine function enjoys a mirror symmetry about the origin. In fact,
  the graph enjoys a mirror symmetry about all vertical lines through
  a peak or trough, i.e., any line of the form $x = n\pi$.
\item The graph of the cosine function enjoys a half turn symmetry
  about any point of the form $(n\pi + (\pi/2),0)$.
\item The tangent, cotangent, and cosecant functions are odd functions
  on their domains of definition. The secant function is even on its
  domain of definition.
\end{enumerate}

\subsection{Periodicity}

Here are some important facts:

\begin{enumerate}
\item The sine and cosine are periodic functions and they both have
  period $2\pi$.
\item The tangent and cotangent are periodic functions and they both
  have period $\pi$.
\item The secant and cosecant are periodic functions and they both
  have period $2\pi$.
\end{enumerate}
\section{Values of trigonometric functions for important angles}

\subsection{$\pi/4$}

To determine the values of trigonometric functions for $\pi/4$,
we need to examine closely the right isosceles triangle. In this
triangle, both legs have the same length, and by the Pythagorean
theorem, the length of the hypotenuse is $\sqrt{2}$ times this
length. Thus, we obtain the following:

\begin{align*}
  \sin (\pi/4) & = \cos (\pi/4)  = \frac{1}{\sqrt{2}} = \frac{\sqrt{2}}{2} & \approx 0.707\\
  \tan (\pi/4) & = \cot (\pi/4) = 1 \\
  \sec (\pi/4) & = \csc (\pi/4) = \sqrt{2} & \approx 1.414
\end{align*}

Notice something else too. For the angle $\pi/4$, the values of
any two trigonometric functions that are related by the complementary
angle relationship are equal. Thus, $\sin (\pi/4) = \cos
(\pi/4)$, $\tan (\pi/4) = \cot (\pi/4)$, and $\sec (\pi/4)
= \csc (\pi/4)$. Geometrically, this is because we are working with
a right isosceles triangle, and there is a symmetry between the two
legs. Algebraically, this is because the angle $(\pi/4)$ is its own
complement, i.e., $(\pi/4) = (\pi/2) - (\pi/4)$.

\subsection{$\pi/6$ and $\pi/3$}

We now consider the triangle where one of the angles is $\pi/6$
and the other angle is $\pi/3$. Consider the figure below:

\includegraphics[width=3in]{equilateraltrianglewithaltitude.png}

In the figure, the triangle $\triangle ABC$ is an equilateral triangle
and the line $AD$ is an altitude so $AD \perp BC$. Since $\triangle
ABC$ is equilateral, all its angles are also equal and hence all the
angles are $\pi/3$. Also, some elementary geometry tells us that
$AD$ bisects $BC$ so $DC = (1/2)BC$, so $DC/BC = 1/2$.

Now consider the triangle $\triangle ADC$. The angle at $C$ is
$\pi/3$ and the angle $\angle CAD$ is $\pi/6$. Thus, we
obtain that $\cos (\pi/3) = \sin (\pi/6) = DC/AC$. Since
$\triangle ABC$ is equilateral, $AC = BC$, so $DC/AC = DC/BC$, which
is $1/2$.

We thus have the first important fact: $\cos (\pi/3) = \sin
(\pi/6) = 1/2$. Now, using the relations between trigonometric
functions, we can obtain the other values. The full list is given
below:

\begin{align*}
  \sin (\pi/6) & = \cos (\pi/3)  = \frac{1}{2}& = 0.5\\
  \cos (\pi/6) & = \sin (\pi/3) = \frac{\sqrt{3}}{2} & \approx 0.866 \\
  \tan (\pi/6) & = \cot (\pi/3) = \frac{1}{\sqrt{3}} & \approx 0.577 \\
  \cot (\pi/6) & = \tan (\pi/3) = \sqrt{3} & \approx 1.732\\
  \sec (\pi/6) & = \csc (\pi/3) = \frac{2}{\sqrt{3}} & \approx 1.154\\
  \csc (\pi/6) & = \sec (\pi/3) = 2
\end{align*}

You should be able to reconstruct all these values. You can choose
either to memorize all of them, or to memorize the first two rows and
reconstruct the rest from them on the spot. Alternatively, you can
remember the side length configuration of the triangle $\triangle ADC$
and read off the trigonometric function values by looking at that
triangle.

\end{document}

