\documentclass[10pt]{amsart}

%Packages in use
\usepackage{fullpage, hyperref, vipul, enumerate}

%Title details
\title{Class quiz solutions: November 28: Logarithm and exponential}
\author{Math 152, Section 55 (Vipul Naik)}
%List of new commands

\begin{document}
\maketitle

\section{Performance review}

$11$ people took this $6$-question quiz. The score distribution was as
follows:

\begin{itemize}
\item Score of $2$: $1$ person
\item Score of $4$: $6$ people
\item Score of $5$: $4$ people
\end{itemize}

The mean score was $3.83$. The answers and performance review are as
follows:

\begin{enumerate}
\item Option (B): $3$ people
\item Option (A): $9$ people
\item Option (A): $10$ people
\item Option (C): $2$ people
\item Option (B): $11$ people
\item Option (B): $11$ people
\end{enumerate}

\section{Solutions}

\begin{enumerate}
\item Consider the function $f(x) := \exp(5 \ln x)$ defined for $x \in
  (0,\infty)$. How does $f(x)$ grow as a function of $x$?

  \begin{enumerate}[(A)]
  \item As a linear function
  \item As a polynomial function but faster than a linear function
  \item Faster than a polynomial function but slower than an
    exponential function
  \item As an exponential function, i.e., $x \mapsto \exp(kx)$ for some $k > 0$
  \item Faster than an exponential function
  \end{enumerate}

  {\em Answer}: Option (B)

  {\em Explanation}: $\exp(5 \ln x) = (\exp (\ln x))^5 = x^5$ is a
  polynomial in $x$.

  {\em Performance review}: $3$ people got this correct. $3$ chose
  (C), $3$ chose (D), $1$ chose (E).
\item Consider the function $f(x) := \ln (5 \exp x)$ for $x \in
  (0,\infty)$. How does $f(x)$ grow as a function of $x$?

  \begin{enumerate}[(A)]
  \item As a linear function
  \item As a polynomial function but faster than a linear function
  \item Faster than a polynomial function but slower than an
    exponential function
  \item As an exponential function, i.e., $x \mapsto \exp(kx)$ for some $k > 0$
  \item Faster than an exponential function
  \end{enumerate}

  {\em Answer}: Option (A)

  {\em Explanation}: $\ln(5 \exp x) = \ln 5 + \ln(\exp x) = \ln 5 +
  x$, which is a linear function.

  {\em Performance review}: $9$ people got this correct. $2$ people
  chose (B).
\item Consider the function $f(x) := \ln((\exp x)^5)$ defined for $x \in
  (0,\infty)$. How does $f(x)$ grow as a function of $x$?

  \begin{enumerate}[(A)]
  \item As a linear function
  \item As a polynomial function but faster than a linear function
  \item Faster than a polynomial function but slower than an
    exponential function
  \item As an exponential function, i.e., $x \mapsto \exp(kx)$ for some $k > 0$
  \item Faster than an exponential function
  \end{enumerate}

  {\em Answer}: Option (A)

  {\em Explanation}: $\ln((\exp x)^5) = \ln(\exp (5x)) = 5x$ is a
  linear function of $x$.

  {\em Performance review}: $10$ people got this correct. $1$ chose
  (C).

\item Consider the function $f(x) := \exp((\ln x)^5)$ defined for $x \in
  (0,\infty)$. How does $f(x)$ grow as a function of $x$?

  \begin{enumerate}[(A)]
  \item As a linear function
  \item As a polynomial function but faster than a linear function
  \item Faster than a polynomial function but slower than an
    exponential function
  \item As an exponential function, i.e., $x \mapsto \exp(kx)$ for some $k > 0$
  \item Faster than an exponential function
  \end{enumerate}

  {\em Answer}: Option (C)

  {\em Explanation}: This is a little tricky, so we break it down into
  two parts.

  First, note that any polynomial function (with positive leading
  coefficient) grows like $x^n$ for $n$ a positive integer. The log of
  that grows like $\ln(x^n) = n \ln x$, i.e., as a linear function of
  $\ln x$. This is slower than $(\ln x)^5$. Thus, the polynomial
  function grows slower than $\exp((\ln x)^5)$.

  Second, note that an exponential function in $x$ is something like
  $\exp x$ or $\exp(kx)$, and $x$ or $kx$ grows faster than any power
  of $\ln x$, so the function $\exp x$ grows faster than $\exp((\ln
  x)^5)$.

  {\em Performance review}: $2$ people got this correct. $6$ chose
  (E), $2$ chose (D), $1$ chose (B).

\item {\em Consumption smoothing}: A certain measure of happiness is
  found to be a logarithmic function of consumption, i.e., the
  happiness level $H$ of a person is found to be of the form $H = a +
  b \ln C$ where $C$ is the person's current consumption level, and
  $a$ and $b$ are positive constants independent of the consumption
  level.

  The person has a certain total consumption $C_{tot}$ to be split
  within two years, year 1 and year 2, i.e., $C_{tot} = C_1 +
  C_2$. Thus, the person's happiness level in year 1 is $H_1 = a + b
  \ln C_1$ and the person's happiness level in year 2 is $H_2 = a + b
  \ln C_2$. How would the person choose to split consumption between
  the two years to maximize average happiness across the years?

  \begin{enumerate}[(A)]
  \item All the consumption in either one year
  \item Equal amount of consumption in the two years
  \item Consume twice as much in one year as in the other year
  \item Consumption in the two years is in the ratio $a:b$
  \item It does not matter because any choice of split of consumption
    level between the two years produces the same average happiness
  \end{enumerate}

  {\em Answer}: Option (B)

  {\em Explanation}: Basically, happiness is logarithmic in
  consumption, so if consumption is unequal, then it can be
  distributed from the higher consumption year to the lower
  consumption year. The {\em fractional} loss in the higher
  consumption year is lower than the {\em fractional} gain in the
  lower consumption year. The nature of logarithmis means that the
  {\em absolute} loss in the higher consumption year is lower than the
  {\em absolute} gain in the lower consumption year. The process
  continues till consumption in both years is exactly equal.

  We can also do this formally. We are basically using the fact that
  the logarithm function is concave down.

  {\em Performance review}: Everybody got this correct.

\item {\em Income inequality and subjective well being}: Subjective
  well being {\em across} individuals is found to be logarithmically
  related to income. Every doubling of income is found to increase an
  individuals' measured subjective well being by $0.3$ points on a
  certain scale. {\em Holding total income across two individuals
  constant}, how should that income be divided between the two
  individuals to maximize their average subjective well being?

  \begin{enumerate}[(A)]
  \item All the income goes to one person
  \item Both earn the exact same income
  \item One person earns twice as much as the other
  \item One person earns $0.3$ times as much as the other
  \item It does not matter because the average subjective well being
  is independent of the distribution of income.
  \end{enumerate}

  {\em Answer}: Option (B)

  {\em Explanation}: The logic is {\em exactly} the same as the
  preceding question, except that instead of an individual
  distributing consumption between years, income is being
  ``distributed'' between individuals in the same year, and instead of
  happiness, we are measuring subjective well being.

  {\em Performance review}: Everybody got this correct.
\end{enumerate}
\end{document}