\documentclass[10pt]{amsart}

%Packages in use
\usepackage{fullpage, hyperref, vipul, enumerate}

%Title details
\title{Class quiz: November 2: Integration}
\author{Math 152, Section 55 (Vipul Naik)}
%List of new commands

\begin{document}
\maketitle

Your name (print clearly in capital letters): $\underline{\qquad\qquad\qquad\qquad\qquad\qquad\qquad\qquad\qquad\qquad}$

\begin{enumerate}

\item Suppose $f$ and $g$ are both functions on $\R$ with the property
  that $f''$ and $g''$ are both everywhere the zero function. For
  which of the following functions is the second derivative {\em not
  necessarily} the zero function everywhere? {\em Last year: $14/15$
  correct}

  \begin{enumerate}[(A)]
  \item $f + g$, i.e., the function $x \mapsto f(x) + g(x)$
  \item $f \cdot g$, i.e., the function $x \mapsto f(x)g(x)$
  \item $f \circ g$, i.e., the function $x \mapsto f(g(x))$
  \item All of the above, i.e., the second derivative need not be
    identically zero for any of these functions.
  \item None of the above, i.e., for all these functions, the second
    derivative is the zero function.
  \end{enumerate}

  \vspace{0.1in}
  Your answer: $\underline{\qquad\qquad\qquad\qquad\qquad\qquad\qquad}$
  \vspace{1in}

\item Suppose $f$ and $g$ are both functions on $\R$ with the property
  that $f'''$ and $g'''$ are both everywhere the zero function. For
  which of the following functions is the third derivative {\em
  necessarily} the zero function everywhere? {\em Last year: $12/15$
  correct}

  \begin{enumerate}[(A)]
  \item $f + g$, i.e., the function $x \mapsto f(x) + g(x)$
  \item $f \cdot g$, i.e., the function $x \mapsto f(x)g(x)$
  \item $f \circ g$, i.e., the function $x \mapsto f(g(x))$
  \item All of the above, i.e., the third derivative is identically
    zero for all of these functions.
  \item None of the above, i.e., the third derivative is not
    guaranteed to be the zero function for any of these.
  \end{enumerate}

  \vspace{0.1in}
  Your answer: $\underline{\qquad\qquad\qquad\qquad\qquad\qquad\qquad}$
  \vspace{1in}

\item Suppose $f$ is a function on an interval $[a,b]$, that is
  continuous except at finitely many interior points $c_1 < c_2 <
  \dots < c_n$ ($n \ge 1)$, where it has jump discontinuities (hence,
  both the left-hand limit and the right-hand limit exist but are not
  equal). Define $F(x) := \int_a^x f(t) \, dt$. Which of the following
  {\bf is true}? {\em Last year: $8/15$ correct}

  \begin{enumerate}[(A)]
  \item $F$ is continuously differentiable on $(a,b)$ and the
    derivative equals $f$ wherever $f$ is continuous.
  \item $F$ is differentiable on $(a,b)$ but the derivative is not
    continuous, and $F' = f$ on the entire interval.
  \item $F$ has one-sided derivatives on $(a,b)$ and the left-hand
    derivative of $F$ at any point equals the left-hand limit of $f$
    at that point, while the right-hand derivative of $F$ at any point
    equals the right-hand limit of $f$ at that point.
  \item $F$ has one-sided derivatives on all points of $(a,b)$ except
    at the points $c_1, c_2, \dots, c_n$; it is continuous at all
    these points but does not have one-sided derivatives.
  \item $F$ is continuous at all points of $(a,b)$ except at the
  points $c_1, c_2, \dots, c_n$.
  \end{enumerate}

  \vspace{0.1in}
  Your answer: $\underline{\qquad\qquad\qquad\qquad\qquad\qquad\qquad}$
  \vspace{1in}

\item (**) For a continuous function $f$ on $\R$ and a real number $a$,
  define $F_{f,a}(x) = \int_a^x f(t) \, dt$. Which of the following
  is {\bf true}? {\em Last year: $5/15$ correct}

  \begin{enumerate}[(A)]
  \item For every continuous function $f$ and every real number $a$,
    $F_{f,a}$ is an antiderivative for $f$, and every antiderivative
    of $f$ can be obtained in this way by choosing $a$ suitably.
  \item For every continuous function $f$ and every real number $a$,
    $F_{f,a}$ is an antiderivative for $f$, but it is not necessary
    that every antiderivative of $f$ can be obtained in this way by
    choosing $a$ suitably. (i.e., there are continuous functions $f$
    where not every antiderivative can be obtained in this way).
  \item For every continuous function $f$, every antiderivative of $f$
    can be written as $F_{f,a}$ for some suitable $a$, but there may
    be some choices of $f$ and $a$ for which $F_{f,a}$ is not an
    antiderivative of $f$.
  \item There may be some choices for $f$ and $a$ for which $F_{f,a}$
    is not an antiderivative for $f$, and there may be some choices of
    $f$ for which there exist antiderivatives that cannot be written
    in the form $F_{f,a}$.
  \item None of the above.
  \end{enumerate}

  \vspace{0.1in}
  Your answer: $\underline{\qquad\qquad\qquad\qquad\qquad\qquad\qquad}$
  \vspace{1in}

\item (**) Suppose $F$ is a differentiable function on an open interval
  $(a,b)$ and $F'$ is not a continuous function. Which of these
  discontinuities can $F'$ have? {\em Last year: $0/15$ correct}

  \begin{enumerate}[(A)]
  \item A removable discontinuity (the limit exists and is finite but
    is not equal to the value of the function)
  \item An infinite discontinuity (one or both the one-sided limits is
    infinite)
  \item A jump discontinuity (both one-sided limits exist and are
    finite, but not equal)
  \item All of the above
  \item None of the above
  \end{enumerate}

  \vspace{0.1in}
  Your answer: $\underline{\qquad\qquad\qquad\qquad\qquad\qquad\qquad}$
  \vspace{1in}

\end{enumerate}

\end{document}