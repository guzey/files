\documentclass[10pt]{amsart}

%Packages in use
\usepackage{fullpage, hyperref, vipul, enumerate}

%Title details
\title{Class quiz: October 24: Concave, inflections, tangents, cusps, asymptotes}
\author{Math 152, Section 55 (Vipul Naik)}
%List of new commands

\begin{document}
\maketitle

Your name (print clearly in capital letters): $\underline{\qquad\qquad\qquad\qquad\qquad\qquad\qquad\qquad\qquad\qquad}$

\begin{enumerate}

\item Consider the function $f(x) := x^3(x - 1)^4(x - 2)^2$. Which of
  the following {\bf is true}? {\em Last year: $11/15$ correct}

  \begin{enumerate}[(A)]
  \item $0$, $1$, and $2$ are all critical points and all of them are
    points of local extrema.
  \item $0$, $1$, and $2$ are all critical points, but only $0$ is a
    point of local extremum.
  \item $0$, $1$, and $2$ are all critical points, but only $1$ and
    $2$ are points of local extrema.
  \item $0$, $1$, and $2$ are all critical points, and none of them is
    a point of local extremum.
  \item $1$ and $2$ are the only critical points.
  \end{enumerate}

  \vspace{0.1in}
  Your answer: $\underline{\qquad\qquad\qquad\qquad\qquad\qquad\qquad}$
  \vspace{0.6in}

\item Suppose $f$ and $g$ are continuously differentiable functions on
  $\R$. Suppose $f$ and $g$ are both concave up. Which of the
  following is {\bf always true}? {\em Last year: $8/15$ correct}

  \begin{enumerate}[(A)]
  \item $f + g$ is concave up.
  \item $f - g$ is concave up.
  \item $f \cdot g$ is concave up.
  \item $f \circ g$ is concave up.
  \item All of the above.
  \end{enumerate}

  \vspace{0.1in}
  Your answer: $\underline{\qquad\qquad\qquad\qquad\qquad\qquad\qquad}$
  \vspace{0.6in}

\item Consider the function $p(x) := x(x-1) \dots (x - n)$, where $n
  \ge 1$ is a positive integer. How many points of inflection does $p$
  have? {\em Last year: $7/15$ correct}

  \begin{enumerate}[(A)]
  \item $n - 3$
  \item $n - 2$
  \item $n - 1$
  \item $n$
  \item $n + 1$
  \end{enumerate}

  \vspace{0.1in}
  Your answer: $\underline{\qquad\qquad\qquad\qquad\qquad\qquad\qquad}$
  \vspace{0.6in}

\item Suppose $f$ is a polynomial function of degree $n \ge 2$. What
  can you say about the sense of concavity of the function $f$ for
  {\bf large enough inputs}, i.e., as $x \to +\infty$? (Note that if
  $n \le 1$, $f$ is linear so we do not have concavity in either
  sense). {\em Last year: $12/15$ correct}

  \begin{enumerate}[(A)]
  \item $f$ is eventually concave up.
  \item $f$ is eventually concave down.
  \item $f$ is eventually either concave up or concave down, and which
    of these cases occurs depends on the sign of the leading
    coefficient of $f$.
  \item $f$ is eventually either concave up or concave down, and which
    of these cases occurs depends on whether the degree of $f$ is even
    or odd.
  \item $f$ may be concave up, concave down, or neither.
  \end{enumerate}

  \vspace{0.1in}
  Your answer: $\underline{\qquad\qquad\qquad\qquad\qquad\qquad\qquad}$
  \vspace{0.6in}

\item (**) Suppose $f$ is a continuously differentiable function on $[a,b]$
  and $f'$ is continuously differentiable at all points of $[a,b]$
  except an interior point $c$, where it has a vertical cusp. What can
  we say is {\bf definitely true} about the behavior of $f$ at $c$?
  {\em Last year: $3/15$ correct}

  \begin{enumerate}[(A)]
  \item $f$ attains a local extreme value at $c$.
  \item $f$ has a point of inflection at $c$.
  \item $f$ has a critical point at $c$ that does not correspond to a
    local extreme value.
  \item $f$ has a vertical tangent at $c$.
  \item $f$ has a vertical cusp at $c$.
  \end{enumerate}

  \vspace{0.1in}
  Your answer: $\underline{\qquad\qquad\qquad\qquad\qquad\qquad\qquad}$
  \vspace{0.6in}

\item (**) Suppose $f$ and $g$ are continuous functions on $\R$, such
  that $f$ attains a vertical tangent at $a$ and is continuously
  differentiable everywhere else, and $g$ attains a vertical tangent
  at $b$ and is continuously differentiable everywhere else. Further,
  $a \ne b$. What can we say is {\bf definitely true} about $f - g$?
  {\em Last year: $5/15$ correct}

  \begin{enumerate}[(A)]
  \item $f - g$ has vertical tangents at $a$ and $b$.
  \item $f - g$ has a vertical tangent at $a$ and a vertical cusp at $b$.
  \item $f - g$ has a vertical cusp at $a$ and a vertical tangent at $b$.
  \item $f - g$ has no vertical tangents and no vertical cusps.
  \item $f - g$ has either a vertical tangent or a vertical cusp at
    the points $a$ and $b$, but it is not possible to be more specific
    without further information.
  \end{enumerate}

  \vspace{0.1in}
  Your answer: $\underline{\qquad\qquad\qquad\qquad\qquad\qquad\qquad}$
  \vspace{0.6in}

\item (**) Suppose $f$ and $g$ are continuous functions on $\R$, such that
  $f$ is continuously differentiable everywhere and $g$ is
  continuously differentiable everywhere except at $c$, where it has a
  vertical tangent. What can we say is {\bf definitely true} about $f
  \circ g$? {\em Last year: $3/15$ correct}

  \begin{enumerate}[(A)]
  \item It has a vertical tangent at $c$.
  \item It has a vertical cusp at $c$.
  \item It has either a vertical tangent or a vertical cusp at $c$.
  \item It has neither a vertical tangent nor a vertical cusp at $c$.
  \item We cannot say anything for certain.
  \end{enumerate}

  \vspace{0.1in}
  Your answer: $\underline{\qquad\qquad\qquad\qquad\qquad\qquad\qquad}$
  \vspace{0.6in}

\end{enumerate}

\end{document}