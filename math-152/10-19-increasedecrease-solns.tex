\documentclass[10pt]{amsart}

%Packages in use
\usepackage{fullpage, hyperref, vipul, enumerate}

%Title details
\title{Class quiz solutions: October 19: Increase/decrease and maxima/minima}
\author{Math 152, Section 55 (Vipul Naik)}
%List of new commands

\begin{document}
\maketitle

\section{Performance review}

$11$ people took the quiz. The score distribution was as follows:

\begin{itemize}
\item Score of $2$: $8$ people
\item Score of $3$: $3$ people
\end{itemize}

The mean score was $2.08$. The problem wise performance was:

\begin{enumerate}
\item Option (D): $10$ people
\item Option (B): $6$ people
\item Option (B): $2$ people
\item Option (B): $7$ people
\end{enumerate}

\section{Solutions}

\begin{enumerate}

\item Suppose $f$ is a function defined on a closed interval
  $[a,c]$. Suppose that the left-hand derivative of $f$ at $c$ exists
  and equals $\ell$. Which of the following implications is {\bf true
  in general}?

  \begin{enumerate}[(A)]
  \item If $f(x) < f(c)$ for all $a \le x < c$, then $\ell < 0$.
  \item If $f(x) \le f(c)$ for all $a \le x < c$, then $\ell \le 0$.
  \item If $f(x) < f(c)$ for all $a \le x < c$, then $\ell > 0$.
  \item If $f(x) \le f(c)$ for all $a \le x < c$, then $\ell \ge 0$.
  \item None of the above is true in general.
  \end{enumerate}

  {\em Answer}: Option (D)

  {\em Explanation}: If $f(x) \le f(c)$ for all $a \le x < c$, then
  all difference quotients from the left are nonnegative. The limiting
  value, which is the left-hand derivative, is thus also
  nonnegative. See the lecture notes for more details.

  {\em The other choices}: Options (A) and (B) predict the wrong
  sign. Option (C) is incorrect because even though the difference
  quotients are all strictly positive, their limiting value could be
  $0$. For instance, $\sin x$ on $[0,\pi/2]$ or $x^3$ on $[-1,0]$.

  {\em Performance review}: $10$ out of $11$ got this correct. $1$
  person chose (E).

  {\em Historical note (last year)}: $8$ people got this correct. $5$
  people chose option (B) and $2$ people chose option (E). It is
  likely that the people who chose option (B) made a sign computation
  error.

  {\em Action point}: On the plus side, most of you seem to have
  understood the fact that strict inequality does not guarantee strict
  positivity or negativity of the one-sided derivative. But, please
  sort out your sign issues while the quarter is still young! Getting
  the right sign is a good sign for the future.

\item Suppose $f$ and $g$ are increasing functions from $\R$ to
  $\R$. Which of the following functions is {\em not} guaranteed to be
  an increasing function from $\R$ to $\R$?

  \begin{enumerate}[(A)]

  \item $f + g$
  \item $f \cdot g$
  \item $f \circ g$
  \item All of the above, i.e., none of them is guaranteed to be increasing.
  \item None of the above, i.e., they are all guaranteed to be increasing.
  \end{enumerate}

  {\em Answer}: Option (B)

  {\em Explanation}: The problem with option (B) arises when one or
  both functions take negative values. For instance, consider the case
  $f(x) := x$ and $g(x) := x$. Both are increasing functions on all of
  $\R$. However, the pointwise product is the function $x \mapsto
  x^2$, which is a decreasing function for negative $x$.

  Formally, the issue is that we cannot multiply inequalities of the
  form $A < B$ and $C < D$ unless we are guaranteed to be working with
  positive numbers.

  {\em The other choices}:

  Option (A): For any $x_1 <
  x_2$, we have $f(x_1) < f(x_2)$ and $g(x_1) < g(x_2)$. Adding up, we
  get $f(x_1) + g(x_1) < f(x_2) + g(x_2)$, so $(f + g)(x_1) < (f + g)(x_2)$.

  Option (C): For any $x_1 < x_2$, we have $g(x_1) < g(x_2)$ since $g$
  is increasing. Now, we use the factthat $f$ is increasing to compare
  its values at the two points $g(x_1)$ and $g(x_2)$, and we get
  $f(g(x_1)) < f(g(x_2))$. We thus get $(f \circ g)(x_1) < (f \circ
  g)(x_2)$.

  {\em Performance review}: $6$ out of $11$ got this correct. $2$
  chose (C) and $3$ chose (E).

  {\em Historical note (last year)}: Only $1$ person got this correct!
  $8$ people chose option (E), $4$ people chose option (C), $1$ person
  chose option (D), and $1$ person chose (A)+(B). Note that you'll
  always have exactly one correct answer option.

  From the rough work shown by a few people, it
  seems that a lot of people were trying to reason this problem using
  derivatives. Using derivatives is {\em not} a sound approach to
  tackling this problem because it is not given that the function is
  differentiable or even continuous. Nonetheless, it is possible to
  obtain the correct answer using the flawed approach of derivatives,
  and it is sad that so few of you did so.

  Others seem to have used examples. With examples, you should have
  found the counterexample rather easily, if you'd chosen $f(x) = g(x)
  = x$. However, most of you don't seem to have considered a
  sufficiently wide range of examples and to have settled with a few
  random ones. This is {\em not} the right way to use examples. When
  searching for counterexamples, you should look systematically and
  try to vary the essential features in a meaningful manner. More on
  this if we get time to cover this material in problem session.

  {\em Action point}: Please, please make sure you understand this
  kind of problem so well that in the future, you're puzzled that this
  ever confused you. Unlike formulas for differentiating complicated
  functions, which you may forget a few years after doing calculus,
  the reasoning methods for these kinds of questions should stick with
  you for a lifetime.

\item Suppose $f$ is a continuous function defined on an open interval
  $(a,b)$ and $c$ is a point in $(a,b)$. Which of the following
  implications is {\bf true}?
  \begin{enumerate}[(A)]

  \item If $c$ is a point of local minimum for $f$, then there is a
    value $\delta > 0$ and an open interval $(c - \delta, c + \delta)
    \subseteq (a,b)$ such that $f$ is non-increasing on $(c -
    \delta,c)$ and non-decreasing on $(c,c+\delta)$.
  \item If there is a value $\delta > 0$ and an open interval $(c -
    \delta, c + \delta) \subseteq (a,b)$ such that $f$ is
    non-increasing on $(c - \delta,c)$ and non-decreasing on
    $(c,c+\delta)$, then $c$ is a point of local minimum for $f$.
  \item If $c$ is a point of local minimum for $f$, then there is a
    value $\delta > 0$ and an open interval $(c - \delta, c + \delta)
    \subseteq (a,b)$ such that $f$ is non-decreasing on $(c -
    \delta,c)$ and non-increasing on $(c,c+\delta)$.
  \item If there is a value $\delta > 0$ and an open interval $(c -
    \delta, c + \delta) \subseteq (a,b)$ such that $f$ is
    non-decreasing on $(c - \delta,c)$ and non-increasing on
    $(c,c+\delta)$, then $c$ is a point of local minimum for $f$.
  \item All of the above are true.
  \end{enumerate}

  {\em Answer}: Option (B).

  {\em Explanation}: Since $f$ is continuous, being non-increasing on
  $(c - \delta, c)$ implies being non-increasing on $(c -
  \delta,c]$. Similarly on the right side. In particular, this means
  that $f(c) \le f(x)$ for all $x \in (c - \delta, c + \delta)$,
  establishing $c$ as a point of local minimum.

  {\em The other choices}: Options (C) and (D) have the wrong kind of
  increase/decrease. Option (A) is wrong, though counterexamples are
  hard to come by. The reason Option (A) is wrong is the core of the
  reason that the first-derivative test does not always work: the
  function could be oscillatory very close to the point $c$, so that
  even though $c$ is a point of local minimum, the function does not
  steadily become non-increasing to the left of $c$. The example
  discussed in the lecture notes is $|x|(2 + \sin(1/x))$.

  {\em Performance review}: $2$ out of $11$ got this. $6$ chose (A)
  and $1$ each chose (C), (D), and (E).

  {\em Historical note (last year)}: $5$ people got this correct. $5$
  people chose (A), which is the converse of the statement. $2$ people
  chose (D) and $1$ person each chose (C) and (E). Thus, most people
  got the sign/direction part correct but messed up on which way the
  implication goes.

  {\em Action point}: This is tricky to get when the lecture material
  is still very new to you. However, you should consistently get this
  kind of question correct once you have reviewed and mastered the
  lecture material.
\item Suppose $f$ is a continuously differentiable function on $\R$
  and $f'$ is a periodic function with period $h$. (Recall that
  periodic derivative implies that the original function is a sum of
  ...). Suppose $S$ is the set of points of local maximum for $f$, and
  $T$ is the set of local maximum values. Which of the following is
  {\bf true in general} about the sets $S$ and $T$?

  \begin{enumerate}[(A)]
  \item The set $S$ is invariant under translation by $h$ (i.e., $x
    \in S$ if and only if $x + h \in S$) and all the values in the set
    $T$ are in the image of the set $[0,h]$ under $f$.
  \item The set $S$ is invariant under translation by $h$ (i.e., $x
    \in S$ if and only if $x + h \in S$) but all the values in the set
    $T$ need not be in the image of the set $[0,h]$ under $f$.
  \item Both the sets $S$ and $T$ are invariant under translation by $h$.
  \item Both the sets $S$ and $T$ are finite.
  \item Both the sets $S$ and $T$ are infinite.
  \end{enumerate}

  {\em Answer}: Option (B).

  {\em Explanation}: For a differentiable function, whether a point is
  a point of local maximum or not depends only on the derivative
  behavior near the point. Since the derivative is periodic with
  period $h$, $S$ is invariant under translation by $h$.

  The set of values $T$ may be finite or infinite. If $f$ itself is
  periodic, then the set of maximum values over a single period is the
  same as the set of maximum values overall. Thus, for instance, in
  the case of the function $f(x) := \sin x$, $T = \{ 1 \}$ and $S = \{
  \pi/2 + 2n\pi : n \in \Z\}$. On the other hand, for the function
  $f(x) := 2\sin x - x$, the derivative is $f'(x) = 2\cos x - 1$,
  which is zero at $2n\pi \pm \pi/3$. The local maxima are attained at
  $2n\pi + \pi/3$ (as we can see from either derivative
  test). However, the set of local maximum {\em values} is infinite --
  in fact, each $n$ gives a different local maximum value, and the set
  of local maximum values is infinite and unbounded. For instance, for
  $n = 0$, the local maximum value is $f(\pi/3) = \sqrt{3} -
  \pi/3$. For $n = 1$, the local maximum value is $f(7\pi/3) =
  \sqrt{3} - (7\pi/3)$. And so on.

  {\em Performance review}: $7$ out of $11$ got this correct. $3$
  chose (E) and $1$ chose (C).

  {\em Historical note (last year)}: $3$ people got this correct. $3$
  people each chose (A) and (D), $4$ people chose (C), and $2$ people
  chose (E).

  {\em Action point}: This was a fairly hard problem. However, it
  should become easy after we've seen a little more about functions
  with periodic derivative.
\end{enumerate}

\end{document}