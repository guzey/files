\documentclass[10pt]{amsart}
\usepackage{fullpage,hyperref,vipul,graphicx}
\title{Informal introduction to limits}
\author{Math 152, Section 55 (Vipul Naik)}

\begin{document}
\maketitle

{\bf Corresponding material in the book}: Section 2.1, parts of
Sections 2.4.

{\bf Corresponding material in homework problems}: Homework 2,
Routine problems 1--4, 7--9.

{\bf Difficulty level}: Easy to moderate, assuming you have seen some
intuitive concept of limits before.

{\bf Covered in class?}: Probably not. We may go over some small part
of this quickly before covering $\epsilon-\delta$ definitions of
limits.

{\bf Things that students should definitely get}: To define limits,
you need to get really really close. There are two directions from
which to approach a real number: left and right. The notation for
limits and one-sided limits. The intuitive meaning of the existence of
limits and of continuity,, one-sided continuity, continuity on
intervals. The notions of removable and jump discontinuity.

{\bf Things that students should hopefully get}: The allusion to why
taking limits for functions on a plane is harder because of multiple
directions of approach.

\section*{Executive summary}

Words ...

\begin{enumerate}

\item On the real line, there are two directions from which to
  approach a point: the {\em left} direction and the {\em right}
  direction.
\item For a function $f$, $\lim_{x \to c} f(x)$ is read as ``the limit
  as $x$ approaches $c$ of $f(x)$. Equivalently, as $x$ approaches
  $c$, $\lim_{x \to c} f(x)$ is the value that $f(x)$ approaches.
\item $\lim_{x \to c} f(x)$ makes sense only if $f$ is defined {\em
  around} $c$, i.e., both to the immediate left and to the immediate
  right of $c$.
\item We have the notion of the {\em left hand limit} $\lim_{x \to
  c^-} f(x)$ and the {\em right hand limit} $\lim_{x \to c^+}
  f(x)$. The {\em limit} $\lim_{x \to c} f(x)$ exists if and only if
  (both the left hand limit and the right hand limit exist and they are
  both equal).
\item $f$ is termed {\em continuous} at $c$ if $c$ is in the domain of
  $f$, the limit of $f$ at $c$ exists, and $f(c)$ equals the
  limit. $f$ is termed {\em left continuous} at $c$ if the left hand
  limit exists and equals $f(c)$. $f$ is termed {\em right continuous}
  at $c$ if the right hand limit exists and equals $f(c)$.
\item $f$ is termed {\em continuous} on an interval $I$ in its domain
  if $f$ is continuous at all points in the interior of $I$,
  continuous from the right at any left endpoint in $I$ (if $I$ is
  closed from the left) and continuous fromthe left at any right
  endpoint in $I$ (if $I$ is closed from the right).
\item A {\em removable discontinuity} for $f$ is a discontinuity where
  a two-sided limit exists but is not equal to the value. A {\em jump
  discontinuity} is a discontinuity where both the left hand limit and
  right hand limit exist but they are not equal.
\end{enumerate}

Actions: See the procedure in the last subsection on computing limits
for polynomial and rational functions.

{\em Note}: These notes cover only the informal and intuitive concept
of limits that you should be familiar with, and do not include the
$\epsilon-\delta$ definitions. The $\epsilon-\delta$ definitions are
covered in subsequent notes which we will go through very carefully in
class.
\section{Intuitive conception of limits}

\subsection{The real numbers and two-sidedness of approach}

The first thing you need to know is that in order to understand
limits, you really need to appreciate the real numbers. There's
something particularly interesting about the real numbers: you can get
{\em really really close} to a real number without equaling it. That's
not something you can do with more sparse sets such as the integers.

It is this ability to sneak really close to something without being
equal to it that allows us to talk of limits. This is something you
should keep in mind -- we'll get back to it later again when we talk
of another kind of limit in 153 -- the limit of a sequence. That's a
very different but in some ways remarkably similar notion, but we'll
come to it in due course.

\includegraphics[width=3in]{twosidedapproach.png}

Now, in the picture, I sneaked up on this number from one side -- the
left side. But I could sneak up to it from another side -- the right
side. This two-sidedness makes things pretty interesting. By the way,
this is one advantage of dealing with a line -- there are only two
directions to worry about. Imagine, just imagine, if you were dealing
with a plane. Then there would be the left side, the right side, the
up side, the down side, this side, that side -- too many! Luckily for
us, we can postpone all those headaches for multivariable calculus,
which is beyond the scope of the 150s. So we focus right now on this
simple real line.

Here's the kind of picture you can avoid thinking about for now:

\includegraphics[width=3in]{multidirectionalapproach.png}

\subsection{Beginning and verbal gymnastics}

\includegraphics[width=3in]{monotoniccontinuouslimit.png}

So let's be really abstract. Suppose $f$ is a function from a subset
of the reals to a subset of the reals and $c$ is a real number. What
we would like to know is the answer to this question: as $x$ gets
really close to $c$, what does $f(x)$ get close to? If $f(x)$ is
heading towards a specific destination, that's called its limit, and
it has the notation:

$$\lim_{x \to c} f(x)$$

This is read as ``the limit as $x$ approaches $c$ of $f(x)$''. An
equation such as:

\begin{equation*}
  \lim_{x \to c} f(x) = b
\end{equation*}

can be read in two ways: ``the limit as $x$ approaches $c$ of $f(x)$
is $b$'' or ``as $x$ approaches $c$, $f(x)$ approaches $b$''. By the
way, some people say {\em tends to} instead of {\em approaches}. Some
people say {\em goes to} and those who're living at the point $c$ may
say {\em comes to}.

Now, let's take some examples. Suppose $f(x) = x$. So $f$ is what is
called the {\em identity function}. It is like a mirror that gives
back what is put into it. Well, what then is $\lim_{x \to 0} f(x)$?
Well, $f(x) = x$, so this is $\lim_{x \to 0} x$. So this reads like a
word puzzle: ``as $x$ tends to $0$, what does $x$ tend to?'' Of
course, $0$. In fact, more generally,

\begin{equation*}
  \lim_{x \to c} x = c
\end{equation*}

\subsection{Beyond our reach: can't limit to what you can't approach}

Okay, here's the next question: what is $\lim_{x \to -1} \sqrt{x}$. By
the way, remember that $\sqrt{x}$ is defined as the nonnegative
square root. So what is this limit? In other words, as $x$ approaches,
gets really really close to $-1$, what does $\sqrt{x}$ approach?

Verbal gymnastics doesn't work the same way as it does for the
previous limit, so this one requires some serious thought. Okay, to
simplify matters, let's first tackle the right side and then the left
side.

Okay, let's try the right side. Let's start from far off. What is the
square root of $4$? It's $2$. What is the square root of $3$? $\sqrt{3}$
is approximately $1.732 \dots$. Square Root of $2$ is $1.414 \dots$,
square root of $1$ is $1$, square root of $0$ is $0$. Hmmm. So the
square root seems to be decreasing. So you might guess right now that
the limit is some negative number.

But to check this guess, you need to go down the negative aisle. And
because there's a paucity of integers, we need to use fractional
numbers. So let's try some negative number between $0$ and $-1$. Say
$-1/4$. What's the square root of $-1/4$?

\includegraphics[width=3in]{squarerootfunctionlonelyminusone.png}

It doesn't {\em have a square root}. It's a negative number. In fact,
the domain of the square root function is the nonnegative reals, the
interval $[0,\infty)$. And that's bad. Which means that as we get even
a little close to $-1$, we cannot evaluate the function from the right
side. So the limit from the right side doesn't make sense.

The limit from the left side doesn't make sense either, because the
function isn't defined {\em anywhere} to the left of $-1$.

What's the message to take from this? It makes sense to talk of the
limit of a function at a point, if the function is defined at places
very close to the point. If it isn't, it's like, as some people say,
putting ``lipstick on a pig.'' You can take the limit of a function
that doesn't exist, and it still doesn't exist.

\subsection{One-sided limits}

There are two further notions, that are mirror images of each
other: the {\em left hand limit} and the {\em right hand limit}. The
left hand limit at $a$ is denoted as $\lim_{x \to a^-} f(x)$ and
$\lim_{x \to a^+} f(x)$.

The left hand limit of a function is the limit as you approach the
domain value from the left side. The right hand limit of a function is
the limit as you approach the domain value from the right side. If a
function is defined on both the left and the right side of a point,
there are five possibilities:

\begin{enumerate}

\item Neither the left hand limit nor the right hand limit exist.

\item The left hand limit exists but the right hand limit does not
  exist.

\item The left hand limit does not exist but the right hand limit
  exists.

\item Both the left hand and the right hand limit exist, but they are
  not equal.

\item Both the left hand and the right hand limit exist, and they are
  equal. In this case, we say that the function {\em has a limit} and
  the limit is equal to both these values.

\end{enumerate}

Phew! What a wide range of possibilities! But you should be happy that
there are only two directions of approach: left and right. If and when
you study multivariable calculus, you'll be studying functions on a
plane, where you have not two, but {\em infinitely many}
directions. If computing limits from two directions is a headache,
computing limits from infinitely many directions is like enduring
torture for eternity.

Things are a little different for values that are at extreme ends of
the domain. For instance, think about the function $f:[0,1] \to [0,1]$
given by $f(x) = \sqrt{1 - x^2}$. Now, at the point $-1$, a left hand
limit doesn't make sense because the function is not defined to the
left of $-1$. So, only the right hand limit does. Similarly at the
point $1$, the right hand limit doesn't make sense but the left hand
limit does.

There's a little confusion about conventions in what I'm going to say,
but I'll just stick with the book on this one: if the point $c$ is at
the boundary of the domain and so the function isn't defined on one
side, the book says that talking of the limit at $c$, or writing
$\lim_{x \to c} f(x)$, is not meaningful. However, we can talk of the
one-sided limit from the side that the function is defined. You may
see a different convention at other places, but we'll just stick to
the book for now. That means that if the point is at the boundary of
the domain, you should clearly specify a one-sided limit instead of
just taking {\em the limit}.

\section{Continuity}
\subsection{The concept of continuity}

Suppose $f$ is a function and it is defined {\em around} a point $a$,
i.e., $f$ is defined at the point $a$ and is also defined in some open
interval containing $a$. Then $f$ is continuous at $a$ if the limit of
$f$ exists at $a$ and equals $f(a)$. This means that the left hand
limit and the right hand limit of $f$ exist at $a$ and are equal to
$f(a)$. In symbols:

$$f \text{ continuous at } a \iff \lim_{x \to a^-} f(x) = \lim_{x \to
a^+} f(x) = f(a)$$

If $f$ is defined at $a$ and on the {\em immediate right} of $a$, then
we say that $f$ is {\em right continuous} or {\em continuous from the
right} at $a$ if the right hand limit of $f$ at $a$ equals $f(a)$. In
symbols:

$$f \text{ right continuous at } a \iff \lim_{x \to a^+}f(x) = f(a)$$

If $f$ is defined at $a$ and on the {\em immediate left} of $a$, then
we say that $f$ is {\em left continouus} or {\em continuous from the
left} at $a$ if the left hand limit of $f$ at $a$ equals $f(a)$. In symbols:

$$f \text{ left continuous at } a \iff \lim_{x \to a^-}f(x) = f(a)$$

\subsection{Continuity on an interval}

Suppose $I$ is an interval (open, closed, half-open half-closed,
possibly infinite in one or both directions). A function $f$ whose
domain contains $I$ is termed {\em continuous} on $I$ if $f$ is
continuous for all {\em interior} points of $I$ (i.e., all points of
$I$ that are not at the boundary of $I$) and is continuous from the
appropriate side at all boundary points. We consider all cases below:

\begin{enumerate}
\item If $I = (a,b)$, $f$ needs to be continuous at all points of $I$.
\item If $I = [a,b]$, $f$ needs to be continuous at all points of
  $(a,b)$, right continuous at $a$, and left continuous at $b$.
\item If $I = [a,b)$, $f$ needs to be continuous at all points of
  $(a,b)$ and right continuous at $a$.
\item If $I = (a,b]$, $f$ needs to be continuous at all points of
  $(a,b)$ and left continuous at $b$.
\item If $I = (a,\infty)$ or $(-\infty,b)$ or $(-\infty,\infty)$, $f$
  needs to be continuous at all points of $I$.
\item if $I = [a,\infty)$, $f$ needs to be continuous at all points of
  $(a,\infty)$ and right continuous at $a$.
\item If $I = (-\infty,b]$, $f$ needs to be continuous at all points
  of $(-\infty,b)$ and left continuous at $b$.
\end{enumerate}

\section{Plumbing leaks}

\subsection{Filling in the hole in the FORGET function}

I hope you remember the $FORGET$ function that we defined a little earlier:

\begin{equation*}
  FORGET(x) = \frac{x}{x}
\end{equation*}

When I defined this function, we discussed that the function is {\em
not} defined at zero. Why? Because at $0$, when we plug in, we get a
$0$ in the numerator and a $0$ in the denominator. Zero in the
denominator is bad! This expression makes no sense. So forget about
evaluating this function at $0$.

However, it definitely makes sense to ask whether the {\em limit}
exists at $0$:

\begin{equation*}
  \lim_{x \to 0} FORGET(x)
\end{equation*}

Why does it make sense? Because the function is defined at all points
other than $0$, it is defined at all points that are close to $0$ but
not equal to it. It's defined at all points to the left of $0$ and at
all points to the right of $0$. The {\em only} point where it is not
defined is $0$. So, it makes sense to evaluate the limit at $0$.

It makes sense, but can we actually do this? Well, let's use the
graph.

\includegraphics[width=3in]{forgetfunctionlimit.png}

Okay, so we see that the function is $1$ to the
left of zero, $1$ to the right of zero. What should the limit be? If
there's any justice in the world, it should be $1$. And it is.

Let's see this mathematically:

\begin{equation*}
  FORGET(x) = \frac{x}{x}, \qquad x \ne 0
\end{equation*}

Thus:

\begin{equation*}
  \lim_{x \to 0} FORGET(x) = \lim_{x \to 0} \frac{x}{x} = \lim_{x \to 0} 1 = 1
\end{equation*}

Now you may say: {\em why can we cancel now?} The reason why we can
cancel $x$ in that step above is that now, since we are {\em only
approaching} $0$ and {\em are not equal to it}, $x$ can be
canceled. And that is the beauty of limits. The thing that gives you
trouble {\em at} a point doesn't give you any trouble {\em near} the
point, and because you are sneaking up from nearby rather than
evaluating at the point, you evade trouble. It's the roundabout
maneuver when a direct assault fails.


\subsection{Looking back at the signum function}

The signum function, denoted $\operatorname{sgn}$, is the function
defined as $x/|x|$ when $x \ne 0$. Under some conventions, it is
considered undefined at $x = 0$. Under other conventions, its value at
$0$ is defined to be $0$.

The signum function is continuous -- in fact, locally constant -- at
all nonzero points in the domain. At the point $0$, it takes the value
$-1$ everywhere on the left and it takes the value $1$ everywhere on
the right. Thus, the left hand limit is $-1$ and the right hand limit
is $1$. Thus, the function jumps from the value $-1$ to $1$ at $0$.

\includegraphics[width=3in]{signumfunctionlimit.png}

The signum function differs from the FORGET function in this important
respect: for the FORGET function, we could {\em fix} or {\em remove}
the discontinuity at $0$ by filling in the value $1$, because the {\em
limit at $0$ exists}. However, for the signum function, there is no
way of fixing the discontinuity at $0$ because the limit does not
exist, which happens in turn because the left and right hand limits
differ.

\subsection{Kinds of discontinuities}

We now consider some important kinds of {\em discontinuities} for
functions, i.e., situations where a function $f$ is defined aound a
point $c$ but is not continuous at $c$. Here are two important kinds
of discontinuities:

\begin{enumerate}
\item {\em Removable discontinuities} are discontinuities where the
  limit of $f$ at $c$ exists but is not equal to $f(c)$. There could
  be two reasons for this: either $f(c)$ is not defined (as for the
  FORGET function) or it is defined but is not equal to the limit.
\item {\em Jump discontinuities} are discontinuities where both the
  left hand limit and the right hand limit exist and are finite, but
  they are not equal. Note that the value of the function at $c$ can
  be changed to make the function continuous from the left at $c$. It
  can also be changed to make the function continuous from the right
  at $c$. But we cannot choose a value $f(c)$ such that both things
  happen simultaneously. An example is the signum function.
\end{enumerate}

These are not the only possibilities. There are also infinite
discontinuities (where the left hand limit or the right hand limit or
both is/are $\pm \infty$) and oscillatory discontinuities. We will
return to this topic later.

\subsection*{Aside:Left and right are based on source, not target}

The left hand limit is the limit where the approach is {\em from} the
left, but the direction in which the approach happens is {\em toward}
the right. In other words, the choice of hand is based on the
direction {\em from} which approach is being made rather than the
direction in which the approach is happening.

A similar convention is followed when specifying the direction of
winds. A {\em northern} wind is a wind that blows {\em from} north to
south. If you took geography in school or read weather forecasts, you
should be familiar with this.

[I have a guess as to the reason for choosing this convention, but
it's purely speculative, so I won't include it here.]


\section{Our nice cocooned world}

Living on Planet Earth in an age of affluence, we are used to taking
niceties for granted. Of course, if we consider the whole world
throughout history, poverty is more the default condition of humans
than affluence.

In the same way, the functions you've been dealing with so far are
nice and sweet. For all their complications and complexities, they
don't throw tantrums. In this course, we'll continue to deal, for the
most part, with nice functions, but we'll explore the more rugged
terrain every once in a while to appreciate our good fortune and the
boundaries of our understanding. For now, let's review how good we
have it and how to handle the occasional hiccup.

\subsection{Limits of polynomial and rational functions}

All polynomial functions are continuous, so the limit of a polynomial
function at a point equals the value of the polynomial function at
that point. In other words, if $f$ is a polynomial function and $c$ is
a number, then $\lim_{x \to c} f(x) = f(c)$.

For rational functions, we evaluate the limit of a rational function
$f$ at a point $c$ using the following rules:

\begin{enumerate}
\item First, try to evaluate the numerator and the denominator at the
  point. If the denominator is nonzero at the point, the limit equals
  the value. If the denominator evaluates to $0$ at the point, and the
  numerator evaluates to something nonzero at the point, then the
  limit is not defined. If both the numerator and the denominator
  evaluate to $0$ at the point, then {\em more work is needed}.
\item If both the numerator and the denominator are $0$ at the point
  $c$, then there is a factor of $x - c$ in both the numerator and the
  denominator. Cancel this factor. This is permissible because we are
  only doing this cancellation {\em near} the point $c$, not {\em at}
  the point $c$. Keep doing such cancellations till there are no
  common factors of $x - c$ in the numerator and the denominator, and
  go back to step (1).
\end{enumerate}

Here are some examples:

The function $f(x) = (x^2 - 3x + 2)/(x - 3)$ has $\lim_{x \to 1} f(x)
= 0/(-2) = 0$.

The function $f(x) = (x^2 - 3x + 2)/(x - 1)$ has $\lim_{x \to 1} f(x)
= ?$? Well, evaluation gives $0$ for both the numerator and the
denominator, so we cancel the $(x - 1)$ factor, and we get:

$$\lim_{x \to 1} \frac{x^2 - 3x + 2}{x - 1} = \lim_{x \to 1} \frac{(x-1)(x-2)}{x-1} = \lim_{x \to 1} x - 2 = 1 - 2 = -1$$

Similarly, if $f(x) = (x - 1)/(x^2 - 3x + 2)$, we have:

$$\lim_{x \to 1} \frac{x - 1}{x^2 - 3x + 2} = \lim_{x \to 1} \frac{1}{x - 2} = \frac{1}{1 - 2} = -1$$

And here's another example:

$$\lim_{x \to 1} \frac{x - 3}{x^2 - 3x + 2}$$

This limit is not defined, because the numerator approaches $-2$ and
the denominator approaches $0$.

\end{document}