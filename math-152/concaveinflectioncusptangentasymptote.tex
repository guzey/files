\documentclass[10pt]{amsart}
\usepackage{fullpage,hyperref,vipul, graphicx}
\title{Concavity, inflections, cusps, tangents, and asymptotes}
\author{Math 152, Section 55 (Vipul Naik)}

\begin{document}
\maketitle

{\bf Corresponding material in the book}: Section 4.6, 4.7.

{\bf Difficulty level}: Moderate to hard. If you have seen these
topics in AP Calculus, then moderate difficulty; if you haven't, then
hard.

{\bf What students should definitely get}: The definitions of {\em
concave up}, {\em concave down}, and {\em point of inflection}. The
strategies to determine {\em limits at infinity}, {\em limits valued
at infinity}, {\em vertical tangents}, {\em cusps}, {\em vertical
asymptotes}, and {\em horizontal asymptotes}.

{\bf What students should hopefully get}: The intuitive meanings of
these concepts, important examples and boundary cases, the
significance of concavity in determining local extrema, the use of
higher derivative tests. Important tricks for calculating limits at
infinity.

\section*{Executive summary}

\subsection{Concavity and points of inflection}

Words ...

\begin{enumerate}
\item A function is called {\em concave up} on an interval if it is
  continuous and its first derivative is continuous and increasing on
  the interval. If the function is twice differentiable, this is
  equivalent to requiring that the second derivative be positive
  except possibly at isolated points, where it can be zero. (Think
  $x^4$, whose first derivative, $4x^3$, is increasing, and the second
  derivative is positive everywhere except at $0$, where it is zero).
\item A function is called {\em concave down} on an interval if it is
  continuous and its first derivative is continuous and decreasing on
  the interval. If the function is twice differentiable, this is
  equivalent to requiring that the second derivative be negative
  except possibly at isolated points, where it can be zero.
\item A {\em point of inflection} is a point where the sense of
  concavity of the function changes. A point of inflection for a
  function is a point of local extremum for the first derivative.
\item Geometrically, at a point of inflection, the tangent line to the
  graph of the function {\em cuts through} the graph.
\end{enumerate}

Actions ...

\begin{enumerate}

\item To determine points of inflection, we first find critical points
  for the first derivative (which are points where this derivative is
  zero or undefined) and then use the first or second derivative test
  at these points. Note that these derivative tests are applied to
  the first derivative, so the first derivative here is the second
  derivative and the second derivative here is the third derivative.
\item In particular, if the second derivative is zero and the third
  derivative exists and is nonzero, we have a point of inflection.
\item A point where the first two derivatives are zero could be a
  point of local extremum or a point of inflection. To find out which
  one it is, we either use sign changes of the derivatives, or we use
  higher derivatives.
\item Most importantly, the second derivative being zero does {\em
  not} automatically imply that we have a point of inflection.
\end{enumerate}
\subsection{Tangents, cusps, and asymptotes}

Words...

\begin{enumerate}

\item We say that $f$ has a horizontal asymptote with value $L$ if
  $\lim_{x \to \infty} f(x) = L$ or $\lim_{x \to -\infty} f(x) =
  L$. Sometimes, both might occur. (In fact, in almost all the
  examples you have seen, the limits at $\pm \infty$, if finite, are
  both equal).
\item We say that $f$ has a vertical asymptote at $c$ if $\lim_{x \to
  c^-} f(x) = \pm \infty$ and/or $\lim_{x \to c^+} f(x) = \pm
  \infty$. Note that in this case, it usually also happens that $f'(x)
  \to \pm \infty$ on the relevant side, with the sign the same as that
  of $f(x)$'s approach if the approach is from the left and opposite
  to that of $f(x)$'s approach if the approach is from the
  right. However, this is not a foregone conclusion.
\item We say that $f$ has a vertical tangent at the point $c$ if $f$
  is continuous (and finite) at $c$ and $\lim_{x \to c} f'(x) =
  \pm \infty$, with the {\em same sign of infinity} from both sides. If
  $f$ is increasing, this sign is $+\infty$, and if $f$ is decreasing,
  this sign is $-\infty$. Geometrically, points of vertical tangent
  behave a lot like points of inflection (in the sense that the
  tangent line cuts through the graph). Think $x^{1/3}$.
\item We say that $f$ has a vertical cusp at the point $c$ if $f$ is
  continuous (and finite) at $c$ and $\lim_{x \to c^-} f'(x)$ and
  $\lim_{x \to c^+} f'(x)$ are infinities of opposite sign. In other
  words, $f$ takes a sharp about-turn at the $x$-value of $c$. Think
  $x^{2/3}$.
\item We say that $f$ is asymptotic to $g$ if $\lim_{x \to \infty}
  f(x) - g(x) = \lim_{x \to -\infty} f(x) - g(x) = 0$. In other words,
  the graphs of $f$ and $g$ come progressively closer as $|x|$ becomes
  larger. (We can also talk of one-sided asymptoticity, i.e.,
  asymptotic only in the positive direction or only in the negative
  direction). When $g$ is a {\em nonconstant linear function}, we say
  that $f$ has an {\em oblique asymptote}. Horizontal asymptotes are a
  special case, where one of the functions is a constant function.
\end{enumerate}

Actions...

\begin{enumerate}
\item Finding the horizontal asymptotes involves computing limits as
  the domain value goes to infinity. Finding the vertical asymptotes
  involves locating points in the domain, or the boundary of the
  domain, where the function limits off to infinity. For both of
  these, it is useful to remember the various rules for limits related
  to infinities.
\item Remember that for a vertical tangent or vertical cusp at a
  point, it is necessary that the function be continuous (and take a
  finite value). So, we not only need to find the points where the
  derivative goes off to infinity, we also need to make sure those are
  points where the function is continuous. Thus, for the function
  $f(x) = 1/x$, $f'(x) \to - \infty$ on both sides as $x \to 0$, but
  we do {\em not} obtain a vertical tangent -- rather, we obtain a
  vertical asymptote.
\end{enumerate}

\section{Concavity and points of inflection}

\subsection{Concavity}

{\em Concave up} means that the derivative of the function (which
measures its rate of change) is itself increasing. Formally, a
function $f$ differentiable on an open interval $I$ is termed {\em
concave up} on $I$ if $f'$ is increasing on $I$. I hope you remember
the definition of an {\em increasing function}: it means that for two
points $x_1, x_2 \in I$, with $x_1 < x_2$, we have $f'(x_1) < f'(x_2)$.

Here's three points:

\begin{enumerate}
\item If $f$ itself is increasing (so that $f'$ is positive), then
  being concave up means that $f$ is increasing {\em at an increasing
  rate}. In other words, the slope of the tangent line to the graph of
  $f$ becomes steeper and steeper (up) as we go from left to
  right. Here's a typical picture:

  \includegraphics[width=3in]{increasingandconcaveup.png}
\item If $f$ itself is decreasing (so that $f'$ is negative), then
  being concave up means that $f$ is decreasing {\em at a decreasing
  rate}. In other words, the slope of the tangent line to the graph is
  negative, but it is becoming less and less steep as we go from left
  to right. Here's a typical picture:

  \includegraphics[width=3in]{decreasingandconcaveup.png}
\item If $f$ is {\em twice differentiable}, i.e., $f'$ is
  differentiable, then we can deduce whether $f'$ is increasing by
  looking at $f''$. Specifically, if $f'$ is continuous on $I$, and
  $f'' > 0$ everywhere on $I$ except at a few isolated points, then
  $f$ is concave up throughout.
\end{enumerate}

Similarly, if $f$ is differentiable on an open interval $I$, we say
that $f$ is {\em concave down} on $I$ if $f'$ is {\em decreasing} on
the interval $I$. I hope you remember the definition
of a {\em decreasing function}: it means that for two points $x_1,
x_2 \in I$, with $x_1 < x_2$, we have $f'(x_1) > f'(x_2)$.

Here's three points:

\begin{enumerate}
\item If $f$ itself is decreasing (so that $f'$ is negative), then
  being concave down means that $f$ is decreasing {\em at an increasing
  rate}. In other words, the slope of the tangent line to the graph of
  $f$ becomes steeper and steeper (downward) as we go from left to right.

  \includegraphics[width=3in]{decreasingandconcavedown.png}

\item If $f$ itself is increasing (so that $f'$ is positive), then
  being concave down means that $f$ is increasing {\em at a decreasing
  rate}. In other words, the slope of the tangent line to the graph is
  positive, but it is becoming less and less steep as we go from left
  to right.

  \includegraphics[width=3in]{increasingandconcavedown.png}

\item If $f$ is {\em twice differentiable}, i.e., $f'$ is
  differentiable, then we can deduce whether $f'$ is increasing by
  looking at $f''$. Specifically, if $f'$ is continuous on $I$, and
  $f'' < 0$ everywhere on $I$ except at a few isolated points, then
  $f$ is concave down throughout.
\end{enumerate}

\subsection{Points of inflection}

A point of inflection is a point $c$ in the interior of the domain of
a differentiable function (i.e., the function is defined and
differentiable on an open interval containing that point) such that
the function is concave in one sense to the immediate left of $c$ and
concave in the other sense to the immediate right of $c$.

Another way of thinking of this is that points of inflection of a
function are points where the derivative is increasing to the
immediate left and decreasing to the immediate right, or decreasing to
the immediate left and increasing to the immediate right. In other
words, it is a point of local maximum or a point of local minimum for
the derivative of the function.

Recall that earlier, we noted that for a point of local maximum or a
point of local minimum, either the derivative is zero or the
derivative does not exist. Since everything we're talking about now is
related to $f'$, we have that for a point of inflection, either $f'' =
0$ or $f''$ does not exist.

So the upshot: concave up means the derivative is increasing, concave
down means the derivative is decreasing, point of inflection means the
sense in which the derivative is changing changes at the point.

\subsection{A point of inflection where the first two derivatives are zero}

We now consider one kind of point of inflection: where the first
derivative and the second derivative are both zero. Let's begin with
the example.

Consider the function $f(x):= x^3$. Recall first that since $f$ is a
cubic function, it has odd degree, so as $x \to -\infty$, $f(x) \to
-\infty$, and as $x \to \infty$, we also have $f(x) \to
\infty$. Further, if we compute $f'(x)$, we get $3x^2$. Note that the
function $3x^2$ is positive for $x \ne 0$, and is $0$ at $x = 0$. So,
from our prior discussion of increasing and decreasing functions, we
see that $f$ is increasing on $(-\infty,0]$ and then again on
$[0,\infty)$. And since the point $0$ is common to the two intervals,
$f$ is in fact increasing everywhere on $(-\infty,\infty)$.

If you remember, this was an important and somewhat weird example
because, although $f'(0) = 0$, $f$ does {\em not} attain a local
extreme value at $0$. This is because the derivative of $f$ is
positive {\em on both sides} of $0$.

\includegraphics[width=3in]{cubefunction.png}

This was the picture we had from our earlier analysis. But now, with
the concepts of concave up, concave down, and points of inflection, we
can get a better understanding of what's going on. Specifically, we
see that the second derivative $f''$ is $6x$, which is negative for $x
< 0$, zero for $x = 0$, and positive for $x > 0$. Thus, $f$ is concave
down for $x < 0$, $x = 0$ is a point of inflection, and $f$ is concave
up for $x > 0$.

So, here's the picture: for $x < 0$, $f$ is negative, $f'$ is
positive, and $f''$ is again negative. Thus, the graph of $f$ is below
the $x$-axis (approaching $0$), it is going upward, and it is going up
at a decreasing rate. So, as $x \to 0$, the graph becomes flatter and
flatter.

For $x > 0$, $f$ is positive, $f'$ is positive, and $f''$ is again
positive. Thus, the graph of $f$ is above the $x$-axis (starting from
the origin), it is going upward, and it is going up at an increasing
rate. The graph starts out from flat and becomes steeper and steeper.

So, $x = 0$ is a {\em no sign change} point for $f'$, which is why it
is not a point of local maximum or local minimum. This is because on
both sides of $0$, $f'$ is positive. What happens is that it is going
down from positive to zero and then up again from zero to
positive. But on a related note, because $f'$ itself dips down to zero
and then goes back up, the point $0$ is a point of local minimum for
$f'$, so it is a point of inflection for $f$.

The main thing you should remember is that when we have a critical
point for a function, where the derivative is zero, but it is neither
a point of local maximum nor a point of local minimum, then it is {\em
likely to be} a point of inflection. In other words, this idea of
something that is increasing (or decreasing) and momentarily stops in
its tracks, is the picture of neither a local maximum nor a local
minimum but a point of inflection.

\subsection{Points of inflection where the derivative is not zero}

\includegraphics[width=3in]{sinegraph.png}

Let's review the graph of the sine function. The sine function starts
with $\sin(0) = 0$, goes up from $0$ to $\pi/2$, where it reaches the
value $1$, then drops down to $0$, drops down further to $-1$ at
$3\pi/2$, and then turns back up to reach $0$ at $2\pi$. And this
pattern repeats periodically.

So far, you've taken me on faith about the way the graph curves. But
we can now start looking at things in terms of concave up and concave
down.

\includegraphics[width=3in]{sinecosinegraphs.png}

The derivative of the $\sin$ function is the $\cos$ function. Let's
graph the $\cos$ function. This starts with the value $1$ at $x = 0$,
goes down to zero at $x = \pi/2$, dips down to $-1$ at $x = \pi$, goes
back up to $0$ at $x = 3\pi/2$, and then goes up to $1$ at $x = 2\pi$.

We see that $\cos$ is positive from $0$ to $\pi/2$, and $\sin$ is
increasing on that interval. $\cos$ is negative from $\pi/2$ to
$3\pi/2$, and $\sin$ is decreasing on that interval. $\cos$ is again
positive from $3\pi/2$ to $2\pi$, and $\sin$ is again increasing on
that interval.

Next, we want to know where $\sin$ is concave up and where it is
concave down. And for this, we look at the second derivative of
$\sin$, which is the function $-\sin$. As you know, the graph of this
function is the same as the graph of $\sin$, but flipped about the
$x$-axis. This means that where $\sin$ is positive, its second
derivative is negative, and where $\sin$ is negative, its second
derivative is positive.

So, from the interval between $0$ and $\pi$, $\sin$ is concave down
and on the interval between $\pi$ and $2\pi$, $\sin$ is concave
up. Breaking the interval down further, $\sin$ is increasing and
concave down on $(0,\pi/2)$, decreasing and concave down on
$(\pi/2,\pi)$, decreasing and concave up on $(\pi,3\pi/2)$, and
increasing and concave up on $(3\pi/2,2\pi)$. The behavior repeats
periodically.

\includegraphics[width=3in]{sinegraphwithtangentatorigin.png}

Now, let's concentrate on the points of inflection. Note that the
sense of concavity changes at multiples of $\pi$ -- at the point $0$,
the function changes from concave up to concave down. At the point
$\pi$, the function changes from concave down to concave up. Another
way of thinking about this is that just before $\pi$, the function is
decreasing at an increasing rate -- it is becoming progressively
steeper. But from $\pi$ onwards, it starts decreasing at a decreasing
rate, in the sense that it starts becoming less steep. So $\pi$ is the
point where the way the tangent line is turning starts changing.

\subsection{A graphical characterization of inflection points}

Inflection points are graphically special because they are points
where the way the tangent line is turning changes sense. There's a
related characterization. If you draw the tangent line through an
inflection point, the tangent line {\em cuts through} the
curve. Equivalently, the curve {\em crosses} the tangent line. This is
opposed to any other point, where the curve {\em locally} lies to one
side of the tangent line.

For instance, for the cube function $f(x) := x^3$, the tangent line is
the $x$-axis, and the curve crosses the $x$-axis at $x = 0$. We see
something similar for the tangent lines at the points of inflection
$0$ and $\pi$ for the $\sin$ function.

\subsection{Third and higher derivatives: exploration}

(I may not get time to cover this in class).

A while ago, we had developed criteria to determine whether a critical
point is a point where a local extreme value is attained. We discussed
two tests that could be used: the {\em first derivative test} and the
{\em second derivative test}. The first derivative test said that if
$f'$ changes sign across the critical point, it is a point where a
local extreme value occurs: a local maximum if the sign change is from
positive on the left to negative on the right, and a local minimum if
the sign change is from negative on the left to positive on the right.

The second derivative test was a test specially suited for functions
that are twice differentiable at the critical point. This test states
that if the second derivative at a critical point is negative, the
function attains a local maximum, and if the second derivative is
positive, the function attains a local minimum. This leaves one case
open: what happens if the second derivative at the critical point is
zero?

In this case, things are inconclusive. We might have a point of local
maximum, a point of local minimum, an inflection point, or none of the
above. How do we figure this out? I will give two general principles
of alternation, and then we will look at some examples:

\begin{enumerate}
\item If $c$ is a point of inflection for $f'$ and and $f'(c) = 0$,
  then $c$ is a point of local extremum for $f$. If the point of
  inflection is a change from concave up to concave down, we get a
  local maximum and if the change is from concave down to concave up,
  we get a local minimum.
\item If $c$ is a point of local maximum or minimum for $f'$, then $c$
  is a point of inflection for $f$. Local maximum implies a change
  from concave up to concave down and local minimum implies a change
  from concave down to concave up.
\end{enumerate}

Let's illustrate this with the function $f(x) := x^5$ and the point $c
= 0$. Let's also assume you knew nothing except differentiation and
applying the derivative tests. We have $f'(x) = 5x^4$, $f''(x) =
20x^3$, $f'''(x) = 60x^2$, $f^{(4)}(x) = 120x$, and $f^{(5)}(x) =
120$. At $c = 0$, $f^{(5)}$ is the first nonzero derivative.

Now, $0$ is a point at which $f^{(4)} = 0$ and $f^{(5)} > 0$. Thus, by
the second derivative test, $0$ is a point of local minimum for
$f^{(3)}$. So, $0$ is a point of inflection for $f^{(2)}$, by point
(2) above. Thus, $0$ is a point of local minimum for $f'$, by point
(1) above. Thus, $0$ is a point of inflection for $f$, by point (2) above.

So, the upshot of this is the alternating behavior between
derivatives.

\subsection{Higher derivative tests}

The discussion above gives a practical criterion to simply use
evaluation of derivatives to determine whether a critical point, where
a function is infinitely differentiable, is a point of local maximum,
point of local minimum, point of inflection, or none of these.

Suppose $f$ is an infinitely differentiable function around a critical
point $c$ for $f$.. Let $k$ be the smallest integer for which
$f^{(k)}(c) \ne 0$ and let $L$ be the nonzero value of the $k^{th}$
derivative. Then:

\begin{enumerate}
\item If $k$ is odd, then $c$ is a point of inflection for $f$ and
  hence neither a point of local maximum nor a point of local minimum.
\item If $k$ is even and $L > 0$, then $c$ is a point of local minimum
  for $f$.
\item If $k$ is even and $L < 0$, then $c$ is a point of local maximum
  for $f$.
\end{enumerate}

For instance, for the power function $f(x) := x^n, n \ge 2$. $0$ is a
critical point and $f$ is infinitely differentiable. In this case, $c
= 0$, $k = n$, and $L = n! > 0$. Thus, if $n$ is even, then $f$ does
attain a local minimum at $0$. if $n$ is odd, $0$ is a point of
inflection. In this simple situation, we could have deduced this
directly from the first derivative test -- for $n$ even, the first
derivative changes sign fron negative to positive at $0$, because $n -
1$ is odd. For $n$ odd, $n - 1$ is even, so the first derivative has
positive sign on both sides of $0$. However, the good news is that
this general method is applicable for other situations where the first
derivative test is harder to apply.

\subsection{Notion of concave up and concave down for one-sided differentiable}

So far, we have defined the notion of concave up and concave down on
an interval assmuing the function is differentiable everywhere on the
interval. In higher mathematics, a somewhat more general definition is
used, and this makes sense for functions that have one-sided
derivatives everywhere.

{\em Note}: Please, please, please, please make sure you understand
this clearly: we can calculate the left-hand derivative and right-hand
derivative using the formal expressions {\em only} after we have
checked that the function is continuous from that side! If there is a
piecewise description of the function and it is not continuous from
one side where the definition is changing, then the corresponding
one-sided derivative is {\em not}, repeat {\em not} defined.

Suppose $f$ is a function on an interval $I = (a,b)$ such that both
the left-hand derivative and the right-hand derivative of $f$ are
defined everywhere on $I$. Note that {\em both} one-sided derivatives
being defined at every point in particular means that the function is
continuous at each point, and hence on $I$. However, $f$ need not be
differentiable at every point, because it is possible that the
left-hand derivative and right-hand derivative differ at different
points.

We then say that:

\begin{enumerate}
\item $f$ is concave up if, at every point, the right-hand derivative
  is greater than or equal to the left-hand derivative, and both
  one-sided derivative are increasing functions on $\R$.
\item $f$ is concave down if, at every point, the right-hand
  derivative is less than or equal to the left-hand derivative, and
  both one-sided derivatives are decreasing functions on $\R$.
\end{enumerate}

For instance, first consider the function $f$ on $(0,\infty)$:

$$f(x) := \lbrace\begin{array}{rl} x^2, & 0 < x \le 1\\x^3, &1 < x \\\end{array}$$

{\em Before we proceed further}, we check/note that the function is
continuous at $1$. Indeed it is. Hence, to calculate the left-hand
derivative and right-hand derivative at $1$, we can formally
differentiate the expressions at $1$. We obtain that the left-hand
derivative at $1$ is $2 \cdot 1 = 2$ and the right-hand derivative at
$1$ is $3 \cdot 1^2 = 3$. We thus obtain the following piecewise
definitions for the left-hand derivative and right-hand derivative:

$$\text{LHD of $f$ at } x = \lbrace\begin{array}{rl} 2x, 0 < x \le 1 \\3x^2, &1 < x \\\end{array}$$

and:

$$\text{RHD of $f$ at } x = \lbrace\begin{array}{rl} 2x, 0 < x < 1 \\3x^2, &1 \le x \\\end{array}$$

The derivative is undefined at $1$. Note that both one-sided
derivatives are increasing everywhere, and at the point $1$, where the
function is not differentiable, the right-hand derivative is
bigger. Thus, the function is concave up on $(0,\infty)$.

Here's the graph, with dashed lines indicating the one-sided derivatives:

\includegraphics[width=3in]{squarecubemax.png}

On the other hand, consider the function $g$ on $(0,\infty)$:

$$g(x) := \lbrace\begin{array}{rl} x^3, & 0 < x \le 1\\x^2, & 1 < x \\\end{array}$$

The function $g$ is continuous and has one-sided derivatives
everywhere. Also note that on the intervals $(0,1)$ and $(1,\infty)$,
$g$ is concave up. However, at the critical point $1$ where $g'$ is
undefined, the right-hand derivative is smaller than the left-hand
derivative. Thus, the function is not concave up overall on
$(0,\infty)$, because at the critical point, its rate of increase
takes a plunge for the worse. Here's the picture of $g$:

\includegraphics[width=3in]{squarecubemin.png}

\subsection{Graphical properties of concave functions}

Here are some properties of the graphs of functions that are concave
up, which are particularly important in the context of
optimization. You should be able to do suitable role changes and
obtain corresponding properties for concave down functions. In all the
points below, $f$ is a continuous function on an interval $[a,b]$ and
is concave up on the interior $(a,b)$.

\begin{enumerate}
\item The only possibilities for the increase-decrease behavior of $f$
  are: increasing throughout, decreasing throughout, or decreasing
  first and then increasing.
\item In particular, $f$ either has exactly one local minimum or
  exactly one endpoint minimum, and this local or endpoint minimum is
  also the absolute minimum.
\item Also, $f$ cannot have a local maximum in its interior. It has
  exactly one endpoint maximum, and this is also the absolute maximum.
\item For any two points $x_1, x_2$ in the domain of $f$, the part of
  the graph of $f$ between $(x_1,f(x_1))$ and $(x_2,f(x_2))$ lies
  below the chord joining the points $(x_1,f(x_1))$ and
  $(x_2,f(x_2))$.
\item If we assume that $f$ is differentiable on $(a,b)$, the tangent
  line through any point $(x,f(x))$ for $a < x < b$ does {\em not}
  intersect the curve at any other point. In the more general notion
  where $f$ has one-sided derivatives, both the left and right tangent
  line satisfy this property.
\end{enumerate}

For concave down functions, the role of minimum and maximum gets
interchanged, and in point (3) above, the graph is now above the chord
rather than below.

\subsection{Addendum: concave and convex}

The book uses the terminology {\em concave up} and {\em concave down},
but it's worth knowing that in much of mathematics as well as
applications of mathematics, the term {\em convex} is used for concave
up and the term {\em concave} is used for concave down. However, there
is some confusion about this since some people use {\em concave} for
concave up and {\em convex} for concave down.

\section{Infinity and asymptotes}

\subsection{Limits to infinity and vertical asymptotes}

We have already discussed what it means to say $\lim_{x \to c} f(x) =
+\infty$, but here's a friendly reminder. It means that as $x$ comes
closer and closer to $c$ (from either side), $f(x)$ goes above every
finite value and does not then come back down. You can similarly
understand what it means to say that $\lim_{x \to c} f(x) = -
\infty$. You should also be able to understand the {\em one-sided
versions} of these concepts: $\lim_{x \to c^-} f(x) = \infty$,
$\lim_{x \to c^+} f(x) = \infty$, $\lim_{x \to c^-} f(x) = -\infty$,
and $\lim_{x \to c^+} f(x) = -\infty$.

Now, here's a little guesswork question: if $\lim_{x \to c^-} f(x) =
\infty$, what can you say about $\lim_{x \to c^-} f'(x)$? As a general
rule, nothing, but in most of the situations that we see, it turns out
that $f'(x)$ also approaches $+\infty$\footnote{More precisely, it
turns out that if $f'$ is continuous and {\em does} approach
something, that something must be $+\infty$. However, there are weird
examples where it oscillates}. The reason is easy to see graphically:
for $f(x)$ to head to $+\infty$ as $x$ approaches a finite value, $f$
needs to climb faster and faster and faster.

Similarly, it is usually the case that if $\lim_{x \to c^-} f(x) =
-\infty$, then $\lim_{x \to c^-} f'(x) = -\infty$. Also, if $\lim_{x
  \to c^+} f(x) = \infty$, then it is likely that $\lim_{x \to c^+}
f'(x) = -\infty$ (because it has to drop very quickly down from
infinity) and if $\lim_{x \to c^+} f(x) = -\infty$, then $\lim_{x \to
  c^+} f'(x) = \infty$ (because it has to rise very quickly back up
from $-\infty$). Again, these things are not always true, but for most
of the typical examples you'll see, they will be.

Now here's the meaning of {\em vertical asymptote}. If $\lim_{x \to
c^-} f(x) = \pm \infty$ and/or $\lim_{x \to c^+} f(x) = \pm \infty$,
then the line $x = c$ is termed a {\em vertical asymptote} for
$f$. This is because the graph of $f$ is approaching the vertical line
$x = c$. In some sense, if we think of $f(c) = +\infty$ or $-\infty$
as the case may be, the vertical line becomes the tangent line to the
curve at that infinite point.

Some of the typical situations worth noting are:

\begin{enumerate}
\item $\lim_{x \to c} f(x) = +\infty$ from both sides. An example of
  this is the function $f(x) = 1/x^2$ with $c = 0$. The vertical
  asymptote is the $y$-axis, i.e., the line $x = 0$. In this case, and
  in most other representative examples, $\lim_{x \to c^-} f'(x) =
  +\infty$, and $\lim_{x \to c^+} f'(x) = -\infty$.
  
  \includegraphics[width=3in]{1byxsquare.png}

\item $\lim_{x \to c} f(x) = -\infty$ from both sides. An example of
  this is the function $f(x) = -1/x^2$ with $c = 0$. The vertical
  asymptote is the $y$-axis, i.e., the line $x = 0$. In this case, and
  in most other representative examples, $\lim_{x \to c^-} f(x) =
  -\infty$, and $\lim_{x \to c^+} f(x) = +\infty$.

  \includegraphics[width=3in]{-1byxsquare.png}

\item $\lim_{x \to c^-} f(x) = \infty$ and $\lim_{x \to c^+} f(x) =
  -\infty$. Examples include $f = \tan$ at $c = \pi/2$ (vertical
  asymptote $x = \pi/2$) and $f(x) = -1/x$ at $c = 0$ (vertical
  asymptote $x = 0$). In both these cases, as in most others, $\lim_{x
    \to c} f'(x) = +\infty$.

  \includegraphics[width=3in]{tangraphwithasymptote.png}

  \includegraphics[width=3in]{-1byx.png}
\item $\lim_{x \to c^-} f(x) = -\infty$ and $\lim_{x \to c^+} f(x) =
  \infty$. Examples include $f = \cot$ at $c = 0$ and $f(x) := 1/x$ at
  $c = 0$. In both these cases, as in most others, $\lim_{x \to c}
  f'(x) = -\infty$.

  \includegraphics[width=3in]{-1byxsquare.png}

\end{enumerate}

\subsection{Horizontal asymptotes}

Horizontal asymptotes are horizontal lines that the graph comes closer
and closer to, just as vertical asymptotes are vertical lines that the
graph comes closer and closer to.

We saw that vertical asymptotes arose when the {\em range value} was
approaching $\pm \infty$ for a finite limiting value of the
domain. Horizontal asymptotes arise where the {\em domain value}
approaches $\pm \infty$ for a finite limiting value of the range.

Explicitly, if $\lim_{x \to \infty} f(x) = L$ (with $L$ a finite
number), then the line $y = L$ is a horizontal asymptote for the graph
of $f$, because as $x \to \infty$, the graph comes closer and closer
to this horizontal line. Similarly, if $\lim_{x \to -\infty} f(x) =
M$, then the line $y = M$ is a horizontal asymptote for the graph of
$f$. Thus, a function whose domain extends to infinity in both
directions could have zero, one, or two horizontal asymptotes.

\includegraphics[width=3in]{twohorizontalasymptotes.png}

\includegraphics[width=3in]{onehorizontalasymptote.png}

We will discuss some of the computational aspects of vertical and
horizontal asymptotes in the problem sessions. Later in the lecture,
and in the addendum, we look at some computational tips and guidelines
over and above what is there in the book.

\subsection{Vertical tangents}

A vertical tangent to the graph of a function $f$ occurs at a point
$(c,f(c))$ if $f$ is continuous but not differentiable at $c$, and
$\lim_{x \to c} f'(x) = +\infty$ or $\lim_{x \to c} f'(x) =
-\infty$. {\em It is important that the sign of infinity in the limit
is the same from both the left and the right side.} 

An example is the function $f(x) := x^{1/3}$ at the point $c = 0$. The
function is continuous at $0$. The derivative functions is
$(1/3)x^{-2/3}$, and the limit of this as $x \to 0$ (from either side)
is $+\infty$. Graphically, what this means is that the tangent is
vertical. In this case, the vertical tangent coincides with the
$y$-axis, because it is attained at the point $0$.

\includegraphics[width=3in]{cuberootfunction.png}

Points of vertical tangent are points of inflection, as we can see
from the $x^{1/3}$ example. Recall that the horizontal tangent case of
the point of inflection was typified by $x^3$, and the general slogan
was that the function slows down for an instant to speed zero. For
vertical tangents, we can think of it as the function speeding up
instantaneously to speed infinity before returning to the realm of
finite speed.

It is important to note that the situation of a vertical tangent
requires that the function itself be defined and continuous, and hence
finite-valued, at the point. Thus, for instance, the function $f(x) :=
1/x$ satisfies $\lim_{x \to 0} f'(x) = -\infty$ but does {\em not}
have a vertical tangent at zero because the function is undefined at
zero.

\subsection{Vertical cusps}

A vertical cusp in the graph of $f$ occurs at a point $c$ if $f$ is
continuous at $c$, and both one-sided limits of $f'$ at $c$ are
infinities of {\em opposite} sign. There are two possibilities:

\begin{enumerate}
\item The left-hand limit of the derivative is $+\infty$ and the
  right-hand limit of the derivative is $-\infty$. Then, $(c,f(c))$ is
  a point of local maximum. An example is $f(x) := -x^{2/3}$ and $c =
  0$ What happens in this situation is the the graph has a sharp peak
  (picking up to speed infinity) at the point $c$, after which it
  rapidly starts dropping.
\item The left-hand limit of the derivative is $-\infty$ and the
  right-hand limit of the derivative is $+\infty$. In this case we get
  a local minimum. An example is $f(x) := x^{2/3}$ and $c = 0$.
\end{enumerate}

There is a special curve called the {\em astroid curve} (I had planned
to put this on the homework, but it went on the chopping block when I
needed to trim the homeworks to size), given by the equation $x^{2/3}
+ y^{2/3} = a^{2/3}$. This curve is not the graph of a function, since
every value of $x$ in $(-a,a)$ has two corresponding values of
$y$. Nonetheless, the curve is a good illustration of the concept of
cusps: there are two vertical cusps at the points $(0,a)$ and $(0,-a)$
respectively, and two horizontal cusps at the points $(a,0)$ and
$(-a,0)$ respectively.

Shown below is the astroid curve for $a = 1$:

\includegraphics[width=3in]{astroidcurve.png}

{\em Note that for the graph of a function, the only kind of cusp that
can occur is a vertical cusp, because a horizontal or oblique cusp
would result in the curve intersecting a vertical line at multiple
points, which would contradict the meaning of a function.}

It is important to note that the situation of a vertical cusp requires
that the function itself be defined and continuous, and hence
finite-valued, at the point. Thus, for instance, the function $f(x) :=
1/x^2$ satisfies $\lim_{x \to 0^-} f'(x) = +\infty$ and $\lim_{x \to
0^+} f'(x) = -\infty$ but does {\em not} have a vertical tangent at
zero because the function is undefined at zero.

\section{Computational aspects}

\subsection{Computing limits at infinity: a review}

We review the main results that you have probably seen and add some
more:

\begin{enumerate}
\item $(\to \infty)(\to \infty) = \to \infty$. In other words, if
  $\lim_{x \to c} f(x) = \infty$ and $\lim_{x \to c} g(x) = \infty$,
  then $\lim_{x \to c} f(x)g(x) = \infty$. $c$ could be finite or $\pm
  \infty$ here, and we could take one-sided limits instead.
\item $(\to \infty)(\to -\infty) = \to -\infty$, and $(\to
  -\infty)(\to -\infty) = \to \infty$.
\item $(\to a)(\to \infty) = \to \infty$ if $a > 0$ and $(\to a)(\to
  \infty) = \to -\infty$ if $a < 0$. Similarly, $(\to a)(\to -\infty) =
  \to -\infty$ if $a > 0$ and $(\to a)(\to -\infty) = \to \infty$ if
  $a < 0$.
\item The previous point can be generalized somewhat: $(\to \infty)$,
  times a function that eventually has a positive lower bound (even if
  it keeps oscillating), is also $\to \infty$. Analogous results hold
  for negative upper bounds.
\item $(\to 0)(\to \infty)$ is an indeterminate form: it is not clear
  what it tends to without doing more work.
\item $(\to \infty) + (\to \infty) = \to \infty$.
\item $(\to \infty) + (\to a) = \to \infty$ where $a$ is finite. More
  generally, $\to \infty$ plus anything that is bounded from below is
  also $\to \infty$.
\item $(\to \infty) - (\to \infty)$ and $(\to \infty) + (\to -\infty)$
  are indeterminate forms.
\end{enumerate}

Apart from this, the main facts you need to remember are that if
$a > 0$, then $\lim_{x \to \infty} x^a = \infty$ and $\lim_{x \to
\infty} x^{-a} = \lim_{x \to -\infty} x^{-a} = 0$. Note that this
holds regardless of whether $a$ is an integer.

When $a$ is an odd integer or a rational number with odd numerator and
odd denominator, $\lim_{x \to -\infty} x^a = -\infty$. When $a$ is an
even integer or a rational number with even numerator and odd
denominator, $\lim_{x \to -\infty} x^a = \infty$. 

Also worth noting: $\lim_{x \to 0^+} x^{-a} = \infty$ for $a > 0$ and
$\lim_{x \to 0^-} x^{-a} = \lim_{x \to -\infty} x^a$, which is
computed by the rule above.

We can use these facts to explain most of the limits involving
polynomial and rational functions. Earlier, we had noted that when
calculating the limit of a polynomial, it is enough to calculate the
limit of its leading monomial. Let's now see why.

Consider the function $f(x) := x^7 - 5x^5 + 3x + 2$. Then, we can
write $f(x) = x^7\left[ 1 - 5x^{-2} + 3x^{-6} + 2x^{-7}\right]$. The
expression on the inside is $1$ plus various negative powers of
$x$. Each of those negative powers of $x$ goes to $0$ as $x \to
\infty$. So, we obtain:

$$\lim_{x \to \infty} [1 - 5x^{-2} + 3x^{-6} + 2x^{-7}] = 1$$

We also have $\lim_{x \to \infty} x^7 = \infty$. Thus, the limit of
the product is $\infty$.

Let's now consider an example of a rational function:

$$\frac{9x^3 - 3x + 2}{103x^2 - 17x - 99}$$

Earlier, we had discussed that when computing such limits at $\pm
\infty$, we can simply calculate thel imits of the leading terms and
ignore the rest. We now have a better understanding of the rationale
behind this. Formally:

\begin{align*}
  & \lim_{x \to \infty} \frac{9x^3 - 3x + 2}{103x^2 - 17x - 99}\\
  = & \lim_{x \to \infty} \frac{x^3(9 - 3x^{-2} + 2x^{-3})}{x^2(103 - 17x^{-1} - 99x^{-2})}\\
  = & \lim{x \to \infty} x \lim_{x \to \infty} \frac{9 - 3x^{-2} + 2x^{-3}}{103 - 17x^{-1} - 99x^{-2}}\\
  = & \lim_{x \to \infty} x \cdot \frac{9}{103}\\
  = & +\infty
\end{align*}

More generally, we see that if the degree of the numerator is greater
than the degree of the denominator, the fraction approaches $\pm
\infty$ as $x \to \pm \infty$, with the sign depending on the signs of
the leading coefficients and the parity (even versus odd) of the
exponents.

If the numerator and denominator have equal degree, the limit is a
finite number. For $x \to \pm \infty$, it is the ratio of the leading
coefficients (notice that it is the same on both sides). This is the
case where we get horizontal asymptotes. In this case, the horizontal
asymptotes on both ends coincide.

For instance:

\begin{align*}
  & \lim_{x \to -\infty} \frac{2x^2 - 3x + 5}{23x^2 - x - 1}\\
  = & \lim_{x \to -\infty} \frac{x^2(2 - 3x^{-1} + 5x^{-2})}{x^2(23 - x^{-1} - x^{-2})}\\
  = & \lim_{x \to -\infty} \frac{x^2}{x^2} \lim_{x \to -\infty} \frac{2 - 3x^{-1} + 5x^{-2}}{23 - x^{-1} - x^{-2}}\\
  = & 1 \cdot \frac{2}{23}\\
  = & \frac{2}{23}
\end{align*}

Finally, when the degree of the numerator is less than the degree of
the denominator, then the fraction tends to $0$ as $x \to \infty$ and
also tends to $0$ as $x \to -\infty$. Thus, in this case, we get the
$x$-axis as the horizontal asymptote on both sides.
\subsection{The $1/x$ substitution trick}

\includegraphics[width=3in]{xsin1byx-3to3.png}

Consider the limit:

$$\lim_{x \to \infty} x \sin(1/x)$$

This limit cannot be computed by plugging in values, because $x \to
\infty$ and $1/x \to 0$, so $\sin(1/x) \to 0$, and we get the
indeterminate form $(\to \infty)(\to 0)$. The approach we use here is
to set $t = 1/x$. As $x \to \infty$, $t \to 0^+$. Since $t = 1/x$, we
get $x = 1/t$. Plugging in, we get:

$$\lim_{t \to 0^+} \frac{\sin t}{t}$$

This limit is $1$, as we know well.

Note that with this general substitution, limits to infinity
correspond to right-hand limits at $0$ for the reciprocal and limits
at $-\infty$ correspond to left-hand limits at $0$ for the
reciprocal. If there is a two-sided limit at $0$ for the reciprocal,
the limits at $\pm \infty$ are the same. In fact, in the $x\sin(1/x)$
example, the limits at $\infty$ and $-\infty$ are both $1$ since
$\lim_{t \to 0} \sin t/t = 1$.

\subsection{Difference of square roots}

Consider the limit:

$$\lim_{x \to \infty}(\sqrt{x + 1} - \sqrt{x})$$

There are many ways to compute this limit, but the easiest is to use
the general $1/x$ substitution trick. Let $t = 1/x$. Then the above
limit becomes:

$$\lim_{t \to 0^+} \frac{\sqrt{t + 1} - 1}{\sqrt{t}}$$

This is an indeterminate form (specifically, a $0/0$ form). However,
we can do the rationalization trick and rewrite this as:

$$\lim_{t \to 0^+} \frac{t}{\sqrt{t}(\sqrt{t + 1} + 1)}$$

The $t$ and $\sqrt{t}$ cancel to give a $\sqrt{t}$ in the numerator,
and we can evaluate and find the limit to be $0$.

A similar approach can be used to handle, for instance, a difference
of cube roots.

\subsection{Combinations of polynomial and trigonometric functions}

We illustrate using some examples:

\begin{enumerate}
\item Consider the function $f(x) := x + 25 \sin x$. As $x \to
  \infty$, this is the sum of a function that tends to infinity and a
  function that oscillates. The oscillating component, however, has a
  finite lower bound, and hence, $\lim_{x \to \infty} f(x) =
  \infty$. Similarly, $\lim_{x \to -\infty} f(x) = -\infty$.
\item Consider the function $f(x) := x\sin x$. As $x \to \infty$, this
  is the product of a function that goes to $\infty$ and a function
  that oscillates between $-1$ and $1$. The oscillating part causes
  the sign of the whole expression to shift, and so as $x \to \infty$,
  $f(x)$ is oscillating with an ever-increasing magnitude of
  oscillation. A similar observation holds for $x \to -\infty$.
\item Consider the function $f(x) := x(3 + \sin x)$ As $x \to \infty$,
  this is the product of a function that tends to $\infty$ and a
  function that oscillates between $2$ and $4$. The important point
  here is that the latter oscillation has a {\em positive lower
    bound}, so the product still tends to $\infty$.
\item Consider the function $f(x) := x \sin(1/x)$. As $x \to \infty$,
  this is the product of a function that tends to $\infty$ and a
  function that tends to $0$, so it is an indeterminate form. We
  already discussed above how this particular indeterminate form can
  be handled.
\end{enumerate}
\end{document}