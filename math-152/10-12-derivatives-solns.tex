\documentclass[10pt]{amsart}

%Packages in use
\usepackage{fullpage, hyperref, vipul, enumerate}

%Title details
\title{Class quiz solutions: October 12: Derivatives}
\author{Math 152, Section 55 (Vipul Naik)}
%List of new commands

\begin{document}
\maketitle

\section{Performance review}

$12$ people took this quiz. The score distribution is as follows:

\begin{itemize}
\item Score of $1$: $6$ people.
\item Score of $2$: $5$ people.
\item Score of $3$: $1$ person.
\end{itemize}

The mean score was $1.58$.

Here are the problem-wise answers and scores:

\begin{enumerate}
\item Option (C): $0$ people
\item Option (A): $4$ people
\item Option (E): $12$ people. {\em Everybody got this correct!}
\item Option (C): $3$ people
\end{enumerate}

\section{Solutions}

\begin{enumerate}

\item (**) Suppose $f$ is a differentiable function on $\R$. Which of
  the following implications is {\bf false}?

  \begin{enumerate}[(A)]
  \item If $f$ is even, then $f'$ is odd.
  \item If $f$ is odd, then $f'$ is even.
  \item If $f'$ is even, then $f$ is odd.
  \item If $f'$ is odd, then $f$ is even.
  \item None of the above, i.e., they are all true.
  \end{enumerate}

  {\em Answer}: Option (C)

  {\em Explanation}: The function $f(x) := 3x + 1$ has derivative
  $f'(x) = 3$, which is even, but the original function $f$ is not
  odd.

  The key idea is that being an odd function has an additional
  condition, namely, that $f(0) = 0$, and the derivative provides no
  control over the value at a point, because we ca nadd a constant to
  a function and still retain the same derivative.

  {\em Performance review}: {\em Nobody} got this correct. $7$ chose
  (E), $3$ chose (A), $1$ chose (B), $1$ chose (D).

  {\em Historical note (last year)}: {\em Nobody} got this correct. $11$
  people chose (E), $2$ people chose (A), and $1$ person chose (D).

  {\em Action point}: This is a tricky problem. The reason why you
  were all led astray is that you were simply using examples, but did
  not have a wide enough repertoire of examples. You needed to think
  of examples of functions $f$ where $f(0) \ne 0$ -- examples as given
  above. Alternatively, you can try to do the theoretical derivation
  and stumble on the key insight that way.

  The problem will probably become easier to think about when we reach
  indefinite integration.

\item (*) A function $f$ on $\R$ is said to satisfy the {\em intermediate
  value property} if, for any $a < b \in \R$, and any $d$ between
  $f(a)$ and $f(b)$, there exists $c \in [a,b]$ such that $f(c) =
  d$. Which (one or more) of the following functions satisfies the
  intermediate value property?

  \begin{enumerate}[(A)]

  \item $f(x) := \lbrace\begin{array}{rl} \sin(1/x), & x \ne 0\\ 0, & x = 0 \\\end{array}$
  \item $f(x) := \lbrace\begin{array}{rl} 1, & x \text{ rational} \\0, & x \text{ irrational}\\\end{array}$
  \item $f(x) := \lbrace\begin{array}{rl} x, & x \text{ rational} \\0, & x \text{ irrational}\\\end{array}$
  \item All of the above
  \item None of the above
  \end{enumerate}

  {\em Answer}: Option (A)

  {\em Explanation}: The crucial thing we use is that the range of $f$
  is $[-1,1]$, and it takes all values in the range on any nonempty
  open interval. From this, it is easy to see that $f$ satisfies the
  intermediate value property.

  {\em The other choices}:

  Option (B): This is not correct. The function takes the values $0$
  and $1$ only. In particular, it does not take any of the
  intermediate values between $0$ and $1$.

  Option (C): This is not correct. For instance, suppose $a =
  \sqrt{2}$ and $b = 1.44$. Then $f(a) = 0$ and $f(b) = 1.44$. The
  value $0.264$ lies between $f(a)$ and $f(b)$. But there is nothing
  between $a$ and $b$ to which applying $f$ gives $0.264$.

  {\em Remark}: The intermediate value theorem can be interpreted as
  the statement that any continuous function (on an interval)
  satisfies the intermediate value property. The example (A), however,
  illustrates that the converse to the intermediate value theorem does
  not hold, i.e., that there are functions that satisfy the
  intermediate value property but are not continuous.

  {\em Performance review}: $4$ out of $12$ people got this
  correct. $4$ chose (D), $3$ chose (E), $1$ chose (C).

  {\em Historical note (last year)}: $7$ out of $14$ people got this
  correct. $4$ people chose (D) and $3$ people chose (E). It is likely
  that the people who chose (D) first tried and found that (A) works,
  and assumed (incorrectly) that the other options work similarly.

\item Which (one or more) of the following functions have a period of
  $\pi$?

  \begin{enumerate}[(A)]
  \item $x \mapsto \sin^2 x$
  \item $x \mapsto |\sin x|$
  \item $x \mapsto \cos^2 x$
  \item $x \mapsto |\cos x|$
  \item All of the above
  \end{enumerate}

  {\em Answer}: Option (E)

  {\em Explanation}: We have $\sin(x + \pi) = -\sin x$ and $\cos(x +
  \pi) = \cos x$. Thus, when we square or take the absolute value, we
  see that the function value repeats after an interval of $\pi$. It
  is also clear from the graph or by inspection that no smaller thing
  works as the period.

  {\em Performance review}: {\em Everybody} got this correct.

  {\em Historical note (last year)}: $12$ out of $14$ people got this
  correct. $1$ person chose (B) and (D) and $1$ person chose (A).

\item Suppose $f$ is a function defined on all of $\R$ such that $f'$
  is a periodic function defined on all of $\R$. What can we conclude
  is {\bf definitely true} about $f$?

  \begin{enumerate}[(A)]

  \item $f$ must be a linear function.
  \item $f$ must be a periodic function.
  \item $f$ can be expressed as the sum of a linear and a periodic
    function, but $f$ need not be either linear or periodic.
  \item $f$ can be expressed as the product of a linear and periodic
    function, but $f$ need not be either linear or periodic.
  \item $f$ can be expressed as a composite of a linear and a periodic
    function, but $f$ need not be either linear or periodic.
  \end{enumerate}

  {\em Answer}: Option (C)

  {\em Explanation}: The function $x + \sin x$ provides an example for
  option (C) that does not satisfy any of the descriptions of the
  other options. Thus, by eliminating other choices, we see that (C)
  is correct. A more formal explanation of why (C) works will have to
  wait till we reach the material leading up to indefinite
  integration.

  {\em Remark}: Graphically, a sum of a linear function and a periodic
  function has a graph that repeats itself, but shifted over both
  vertically and horizontally. We'll talk quite a bit about such
  functions when we study graphing techniques. Another way of thinking
  of it is that the linear function represents the {\em secular trend}
  and the periodic function represents the {\em seasonal
  variation}. For instance, a graph of the daily sales revenue at a
  supermarket will have a secular trend (increasing, if the
  supermarket and its customer base are expanding over time, and
  decreasing if the customer base is shrinking) and a seasonal
  variation (spikes during Black Friday and Christmas season, lows at
  some times of the year). If the secular trend is linear, and the
  secular and seasonal trend interact additively, then the sales
  revenues are approximately represented as the sum of a linear and a
  periodic function. If they interact multiplicatively, then the sales
  revenues are approximately represented as the product of a linear
  and a periodic function.

  {\em Performance review}: $3$ out of $12$ people got this
  correct. $9$ chose (B).

  {\em Historical note (last year)}: $8$ out of $14$ people got this
  correct. $5$ people chose (B), and $1$ person chose (E).

  {\em Action point}: It {\em is} true that if $f$ is a periodic
  differentiable function, then $f'$ is also periodic. However, that's
  not what the question is asking for. The question is asking for the
  reverse: if $f'$ is a periodic function, what can we conclude about
  $f$? Is $f$ also periodic? In fact, as the answer above makes clear,
  it is possible for a non-periodic function to have a periodic
  derivative.
\end{enumerate}

\end{document}