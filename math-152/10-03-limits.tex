\documentclass[10pt]{amsart}

%Packages in use
\usepackage{fullpage, hyperref, vipul, enumerate}

%Title details
\title{Class quiz: October 3: Limits}
\author{Math 152, Section 55 (Vipul Naik)}
%List of new commands

\begin{document}
\maketitle

Your name (print clearly in capital letters): $\underline{\qquad\qquad\qquad\qquad\qquad\qquad\qquad\qquad\qquad\qquad}$

\begin{enumerate}

\item Which of these is the correct interpretation of $\lim_{x \to c}
  f(x) = L$ in terms of the definition of limit? {\em Last year's
  performance: $9/12$ correct}

  \begin{enumerate}[(A)]
  \item For every $\alpha > 0$, there exists $\beta > 0$ such that if
    $0 < |x - c| < \alpha$, then $|f(x) - L| < \beta$.
  \item There exists $\alpha > 0$ such that for every $\beta > 0$, and
    $0 < |x - c| < \alpha$, we have $|f(x) - L| < \beta$.
  \item For every $\alpha > 0$, there exists $\beta > 0$ such that if
    $0 < |x - c| < \beta$, then $|f(x) - L| < \alpha$.
  \item There exists $\alpha > 0$ such that for every $\beta > 0$ and
    $0 < |x - c| < \beta$, we have $|f(x) - L| < \alpha$.
  \item None of the above
  \end{enumerate}

  \vspace{0.1in}
  Your answer: $\underline{\qquad\qquad\qquad\qquad\qquad\qquad\qquad}$
  \vspace{0.5in}

\item Suppose $f:\R \to \R$ is a function. Which of the following says
  that $f$ does not have a limit at any point in $\R$ (i.e., there is
  no point $c \in \R$ for which $\lim_{x \to c} f(x)$ exists)? {\em
  Last year's performance: $10/12$ correct}

  \begin{enumerate}[(A)]
  \item For every $c \in \R$, there exists $L \in \R$ such that for
    every $\epsilon > 0$, there exists $\delta > 0$ such that for all
    $x$ satisfying $0 < |x - c| < \delta$, we have $|f(x) - L| \ge
    \epsilon$.
  \item There exists $c \in \R$ such that for every $L \in \R$, there
    exists $\epsilon > 0$ such that for every $\delta > 0$, there exists
    $x$ satisfying $0 < |x - c| < \delta$ and $|f(x) - L| \ge \epsilon$.
  \item For every $c \in \R$ and every $L \in \R$, there exists
    $\epsilon > 0$ such that for every $\delta > 0$, there exists $x$
    satisfying $0 < |x - c| < \delta$ and $|f(x) - L| \ge \epsilon$.
  \item There exists $c \in \R$ and $L \in \R$ such that for
    every $\epsilon > 0$, there exists $\delta > 0$ such that for all
    $x$ satisfying $0 < |x - c| < \delta$, we have $|f(x) - L| \ge
    \epsilon$.
  \item All of the above.
  \end{enumerate}

  \vspace{0.1in}
  Your answer: $\underline{\qquad\qquad\qquad\qquad\qquad\qquad\qquad}$
  \vspace{0.5in}

\item In the usual $\epsilon-\delta$ definition of limit for a given
  limit $\lim_{x \to c} f(x) = L$, if a given value $\delta > 0$ works
  for a given value $\epsilon > 0$, then which of the following is
  true? {\em Last year's performance: $17/26$ correct, appeared in 153
  quiz}

  \begin{enumerate}[(A)]
  \item Every smaller positive value of $\delta$ works for the same
    $\epsilon$. Also, the given value of $\delta$ works for every
    smaller positive value of $\epsilon$.
  \item Every smaller positive value of $\delta$ works for the same
    $\epsilon$. Also, the given value of $\delta$ works for every
    larger value of $\epsilon$.
  \item Every larger value of $\delta$ works for the same
    $\epsilon$. Also, the given value of $\delta$ works for every
    smaller positive value of $\epsilon$.
  \item Every larger value of $\delta$ works for the same
    $\epsilon$. Also, the given value of $\delta$ works for every
    larger value of $\epsilon$.
  \item None of the above statements need always be true.
  \end{enumerate}

  \vspace{0.1in}
  Your answer: $\underline{\qquad\qquad\qquad\qquad\qquad\qquad\qquad}$
  \vspace{0.5in}

\item Which of the following is a correct formulation of the statement
  $\lim_{x \to c} f(x) = L$, in a manner that avoids the use of
  $\epsilon$s and $\delta$s? {\em Not appeared in previous years}

  \begin{enumerate}[(A)]
  \item For every open interval centered at $c$, there is an open
    interval centered at $L$ such that the image under $f$ of the open
    interval centered at $c$ (excluding the point $c$ itself) is
    contained in the open interval centered at $L$.
  \item For every open interval centered at $c$, there is an open
    interval centered at $L$ such that the image under $f$ of the open
    interval centered at $c$ (excluding the point $c$ itself) contains
    the open interval centered at $L$.
  \item For every open interval centered at $L$, there is an open
    interval centered at $c$ such that the image under $f$ of the open
    interval centered at $c$ (excluding the point $c$ itself) is
    contained in the open interval centered at $L$.
  \item For every open interval centered at $L$, there is an open
    interval centered at $c$ such that the image under $f$ of the open
    interval centered at $c$ (excluding the point $c$ itself) contains
    the open interval centered at $L$.
  \item None of the above.
  \end{enumerate}

  \vspace{0.1in}
  Your answer: $\underline{\qquad\qquad\qquad\qquad\qquad\qquad\qquad}$
  \vspace{0.5in}

\end{enumerate}
\end{document}