\documentclass[10pt]{amsart}

%Packages in use
\usepackage{fullpage, hyperref, vipul, enumerate}

%Title details
\title{Class quiz: October 10: Derivatives}
\author{Math 152, Section 55 (Vipul Naik)}
%List of new commands

\begin{document}
\maketitle

Your name (print clearly in capital letters): $\underline{\qquad\qquad\qquad\qquad\qquad\qquad\qquad\qquad\qquad\qquad}$

Write your answer in the space provided. In the space below, you can
explain your work if you want (this will not affect scoring). I may or
may not get time to look at the work you have done, but it may help
you recall how you arrived at a particular answer.

You are expected to take about one minute per question.

\begin{enumerate}

\item Consider the expression $x^2 + t^2 + xt$. What is the derivative
  of this with respect to $x$ (with $t$ assumed to be a constant)?
  {\em Last year: $11/12$ correct}

  \begin{enumerate}[(A)]

  \item $2x + 2t + x + t$
  \item $2x + 2t + 1$
  \item $2x + 2t$
  \item $2x + t + 1$
  \item $2x + t$
  \end{enumerate}

  \vspace{0.1in}
  Your answer: $\underline{\qquad\qquad\qquad\qquad\qquad\qquad\qquad}$
  \vspace{1.5in}

\item Which of the following verbal statements is {\bf not valid as a
  general rule}? {\em Last year: $10/12$ correct}

  \begin{enumerate}[(A)]
  \item The derivative of the sum of two functions is the sum of the
    derivatives of the functions.
  \item The derivative of the difference of two functions is the
    difference of the derivatives of the functions.
  \item The derivative of a constant times a function is the same
    constant times the derivative of the function.
  \item The derivative of the product of two functions is the product
    of the derivatives of the functions.
  \item None of the above, i.e., they are all valid as general rules.
  \end{enumerate}

  \vspace{0.1in}
  Your answer: $\underline{\qquad\qquad\qquad\qquad\qquad\qquad\qquad}$
  \vspace{1.5in}

  {\bf PLEASE TURN OVER FOR THE THIRD AND FOURTH QUESTION.}

\newpage
\item (*) Which of the following statements is {\bf definitely true} about
  the tangent line to the graph of an everywhere differentiable
  function $f$ on $\R$ at the point $(a,f(a))$ (Here, ``everywhere
  differentiable'' means that the derivative of $f$ is defined and
  finite for all $x \in \R$)? {\em Last year: $6/12$ correct}

  \begin{enumerate}[(A)]

  \item The tangent line intersects the curve at precisely one point,
    namely $(a,f(a))$.
  \item The tangent line intersects the $x$-axis.
  \item The tangent line intersects the $f(x)$-axis (the $y$-axis).
  \item Any line through $(a,f(a))$ other than the tangent line
    intersects the graph of $f$ at at least one other point.
  \item None of the above need be true.
  \end{enumerate}

  \vspace{0.1in}
  Your answer: $\underline{\qquad\qquad\qquad\qquad\qquad\qquad\qquad}$
  \vspace{1.5in}

\item (*) For a function $f: (0,\infty) \to (0,\infty)$, denote by
  $f^{(k)}$ the $k^{th}$ derivative of $f$. Suppose $f(x) := x^r$ with
  domain $(0,\infty)$, and $r$ a rational number. What is the {\bf
  precise set of values} of $r$ satisfying the following: there exist
  a positive integer $k$ (dependent on $r$) for which $f^{(k)}$ is
  identically the zero function. {\em Last year: $4/12$ correct}

  \begin{enumerate}[(A)]
  \item $r$ should be an integer.
  \item $r$ should be a nonnegative integer.
  \item $r$ should be a positive integer.
  \item $r$ should be a nonnegative rational number.
  \item $r$ should be a positive rational number.
  \end{enumerate}

  \vspace{0.1in}
  Your answer: $\underline{\qquad\qquad\qquad\qquad\qquad\qquad\qquad}$
  \vspace{1.5in}

\end{enumerate}

\end{document}