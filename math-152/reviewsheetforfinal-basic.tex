\documentclass[10pt]{amsart}
\usepackage{fullpage,hyperref,vipul, graphicx}
\title{Review sheet for final: basic}
\author{Math 152, Section 55 (Vipul Naik)}

\begin{document}
\maketitle

{\em With minor exceptions, this document does not re-review material
already covered in the review sheet for midterm 1 and midterm 2. It is
your responsibility to go through that review sheet again and make
sure you have mastered all the material there.}

See the advanced version for error-spotting exercises and the quickly
list.
 
\section{Area computations}

Note that this section partially repeats material from the prevoius
midterm review, because part of the area computations syllabus was in
the previous midterm syllabus.

Words ...

\begin{enumerate}

\item We can use integration to determine the area of the region
  between the graph of a function $f$ and the $x$-axis from $x = a$ to
  $x = b$: this integral is $\int_a^b f(x) \, dx$. The integral
  measures the signed area: parts where $f \ge 0$ make positive
  contributions and parts where $f \le 0$ make negative
  contributions. The magnitude-only area is given as $\int_a^b |f(x)|
  \, dx$. The best way of calculating this is to split $[a,b]$ into
  sub-intervals such that $f$ has constant sign on each sub-interval,
  and add up the areas on each sub-interval.
\item Given two functions $f$ and $g$, we can measure the area between
  $f$ and $g$ between $x = a$ and $x = b$ as $\int_a^b |f(x) - g(x)|
  \,dx$. For practical purposes, we divide into sub-intervals so that
  on each sub-interval one function is bigger than the other. We then
  use integration to find the magnitude of the area on each
  sub-interval and add up. If $f$ and $g$ are both continuous, the
  points where the functions {\em cross} each other are points where
  $f = g$.
\item Sometimes, we may want to compute areas against the
  $y$-axis. The typical strategy for doing this is to interchange the
  roles of $x$ and $y$ in the above discussion. In particular, we try
  to express $x$ as a function of $y$.
\item An alternative strategy for computing areas against the $y$-axis
  is to use formulas for computing areas against the $x$-axis, and
  then compute differences of regions.
\item A general approach for thinking of integration is in terms of
  slicing and integration. Here, integration along the $x$-axis is
  based on the following idea: divide the region into vertical slices,
  and then integrate the lengths of these slices along the horizontal
  dimension. Regions for which this works best are the regions called
  {\em Type I regions}. These are the regions for which the
  intersection with any vertical line is either empty or a point or a
  line segment, hence it has a well-defined length.
\item Correspondingly, integration along the $y$-axis is based on
  dividing the region into horizontal slices, and integrating the
  lengths of these slices along the vertical dimension. Regions for
  which this works best are the regions called {\em Type II regions}.
  These are the regions for which the intersection with any horizontal
  line is either empty or a point or a line segment, hence it has a
  well-defined length.
\item Generalizing from both of these, we see that our general
  strategy is to choose two perpendicular directions in the plane, one
  being the direction of our slices and the other being the direction
  of integration.
\end{enumerate}

Actions ...

\begin{enumerate}
\item In some situations we are directly given functions and/or curves
  and are asked to find areas. In others, we are given real-world
  situations where we need to find areas of regions. Here, we have to
  find functions and set up the integration problem as an intermediate
  step.
\item In all these situations, it is important to draw the graphs in a
  reasonably correct way. This brings us to all the ideas that are
  contained in graph drawing. Remember, here we may be interested in
  simultaneously graphing more than one function. Thus, in addition to
  being careful about each function, we should also correctly estimate
  where one function is bigger than the other, and find (approximately
  or exactly) the intersection points. (Go over the notes on
  graph-drawing, and some additional notes on graphing that weren't
  completely covered in class).
\item In some situations, we are asked to find the area(s) of
  region(s) bounded by the graphs of one, two, three, or more
  functions. Here, we first need to sketch the figure. Then, we need
  to find the interval of integration, and if necessary, split this
  interval into sub-intervals, such that on each sub-interval, we know
  exactly what integral we need to do. For instance, consider the
  region between the graphs of $\sin$, $\cos$, and the
  $x$-axis. Basically, the idea is to find, for all the vertical
  slices, the upper and lower limits of the slice.
\end{enumerate}

\section{Volume computations}
Words ...

\begin{enumerate}
\item The cross section method for computing volume is an analogue of
  the two-dimensional area computation method: our slices are replaced
  by cross sections by planes parallel to a fixed plane, and the line
  of integration is a line perpendicular to the
  planes. One-dimensional slices are replaced by two-dimensional cross
  sections.
\item Suppose $\Omega$ is a region in the plane. We can construct a
  right cylinder with base $\Omega$ and height $h$. This is obtained
  by translating $\Omega$ in a direction perpendicular to its plane by
  a length of $h$. The cross section of this right cylinder along any
  plane parallel to the original plane looks like $\Omega$ if that
  plane is within range. The volume is the product of the area of
  $\Omega$ and the height $h$. This is also called the right cylinder
  with constant cross section $\Omega$.
\item We can also construct an oblique cylinder. Here, the direction
  of translation is not perpendicular to the original plane. The total
  volume is the product of the area of $\Omega$ and the height
  perpendicular to $\Omega$. Oblique cylinders are to right cylinders
  what parallelograms are to rectangles.
\item More generally, the volume of a solid can be computed using the
  cross section method. Here, we choose a direction. We measure areas
  of cross sections along planes perpendicular to that direction, and
  integrate these areas along that direction.
\item This general approach has another special case that is perhaps
  as important as right cylinders. These are the {\em cones} (there
  are right cones and oblique cones). A cone is obtained by taking a
  region in a plane and connecting all points in it to a point outside
  the plane. It is a right cone if that point is directly above the
  center of the region. The volume of a cone is $1/3$ times the
  product of the base area and the height, i.e., the perpendicular
  distance from the outside point to the plane. In particular, a cone
  has one-third the volume of a cylinder of the same base and height.
\item A solid of revolution is a solid obtained by revolving a region
  in a plane about a line (called the axis of revolution). The volume
  of a solid of revolution can be computed by choosing the axis as the
  axis of integration and using the planes of cross section as planes
  perpendicular to it. These cross sections are either circular disks
  or annuli in the nice cases. {\em Added}: In nastier cases, the cross
  sections could be unions of multiple annuli.
\item The {\em disk method} is a special case of the above, where the
  region being revolved is supported on the axis of revolution. For
  instance, consider the region between the $x$-axis, the graph of a
  function $f$, and the lines $x = a$ and $x = b$. The volume of the
  corresponding solid of revolution is $\pi \int_a^b [f(x)]^2 \,
  dx$. This is because the radius of the cross section disk at $x=
  x_0$ is $|f(x_0)|$.
\item The {\em washer method} is the more general case where the
  region need not adhere to the axis of revolution. For instance,
  consider two nonnegative functions $f,g$ and suppose $0 \le g \le
  f$. Consider the region bounded by the graphs of these two functions
  and the lines $x = a$ and $x = b$. The volume of the corresponding
  solid of revolution is $\pi \int_a^b ([f(x)]^2 - [g(x)]^2) \,
  dx$. Note that in the more general case where the functions cross
  each other, we may need to split into sub-intervals so that we can
  apply the washer method on each sub-interval.
\item The shell method works for situations where we revolve about the
  $y$-axis the region made between the graph of a function and the
  $x$-axis. The formula here is $2\pi \int_a^b xf(x) \, dx$ for $f$
  nonnegative and $0 < a < b$. If $f$ could be positive or negative,
  we use $2 \pi \int_a^b x|f(x)| \, dx$.. More generally, if we are
  looking at the region between the graphs of $f$ and $g$ (vertically)
  with $g \le f$, we get $2\pi \int_a^b x[f(x) - g(x)] \, dx$. If we
  don't know which one is bigger where, we use $2\pi \int_a^b x|f(x) -
  g(x)| \, dx$.
\end{enumerate}

Actions ...

\begin{enumerate}
\item To compute the volume using cross sections, we first need to set
  things up so that we know the cross section areas as a function of
  the position of the plane. For this, it is usually necessary to use
  either coordinate geometry or basic trigonometry, or a combination.
\item A solid occurs as a solid of revolution if it has complete
  rotational symmetry about some axis. In that case, that axis is the
  axis of revolution and the original region that we need is obtained
  by taking a cross section in any plane containing the axis of
  revolution and looking at the part of that cross section that is on
  one side of the axis of revolution.
\item For solids of revolution, be particularly wary if the original
  figure being revolved has parts on both sides of the axis of
  revolution. If it is symmetric about the axis of revolution, delete
  one side. {\em Added}: In general, fold the figure about the axis of
  revolution -- folding does not affect the final solid of revolution
  we obtain.
\item Be careful about the situations where you have to be
  sign-sensitive and the situations where you do not. In the disk
  method sensitivity to signs is not important. In the washer method
  and shell method, it is. {\em Added}: Also be careful about applying
  the disk, washer, and shell methods when the axis of revolution is
  {\em not} the $x$- or $y$-axis but is parallel to one of them.
\item The farther the shape being revolved is from the axis, the
  greater the volume of the solid of revolution.
\item The average value point of view is sometimes useful for
  understanding such situations.
\end{enumerate}

\section{One-one functions and inverses}

\subsection{Vague generalities}

Words...

\begin{enumerate}
\item Old hat: Given two sets $A$ and $B$, a function $f:A \to B$ is
  something that takes inputs in $A$ and gives outputs in $B$. The
  {\em domain} of a function is the set of possible inputs, while the
  {\em range} of a function is the set of possible outputs. The
  notation $f:A \to B$ typically means that the domain of the function
  is $A$. However, the whole of $B$ need not be the range; rather, all
  we know is that the range is a {\em subset} of $B$. One way of
  thinking of functions is that {\em equal inputs give equal outputs}.
\item A function $f$ is one-to-one if $f(x_1) = f(x_2) \implies x_1 =
  x_2$. In other words, {\em unequal inputs give unequal
  outputs}. Another way of thinking of this is that {\em equal outputs
  could only arise from equal inputs}. Or, {\em knowledge of the
  output allows us to determine the input uniquely}. One-to-one
  functions are also called one-one functions or injective functions.
\item Suppose $f$ is a function with domain $A$ and range $B$. If $f$
  is one-to-one, there is a {\em unique} function $g$ with domain $B$
  and range $A$ such that $f(g(x)) = x$ for all $x \in B$. This
  function is denoted $f^{-1}$.  We further have that $g$ is also
  one-to-one, and that $f = g^{-1}$. Note that $f^{-1}$ differs from
  the reciprocal function of $f$.
\item Suppose $f: A \to B$ and $g:B \to C$ are one-to-one
  functions. Then $g \circ f$ is also one-to-one, and its inverse is
  the function $f^{-1} \circ g^{-1}$.
\end{enumerate}

Actions ...

\begin{enumerate}
\item To determine whether a function is one-to-one, solve $f(x) =
  f(a)$ for $x$ in terms of $a$. If, for every $a$ in the domain, the
  only solution is $x = a$, the function is one-to-one. If, on the
  other hand, there are some values of $a$ for which there is a
  solution $x \ne a$, the function is not one-to-one.
\item To compute the inverse of a one-to-one function, solve $f(x) =
  y$ and the expression for $x$ in terms of $y$ is the inverse function.
\end{enumerate}

\subsection{In graph terms}

Thousand words ...

\begin{enumerate}
\item A picture in a coordinatized plane is the graph of a function if
  every vertical line intersects the picture at most once. The
  vertical lines that intersect it exactly once correspond to the
  $x$-values in the domain. This is known as the {\em vertical line
  test}.
\item A function is one-to-one if and only if its graph satisfies the
  {\em horizontal line test}: every horizontal line intersects the
  graph at most once. The horizontal lines that intersect the graph
  exactly once correspond to $y$-values in the range.
\item For a one-to-one function, the graph of the inverse function is
  obtained by reflecting the graph of the function about the $y = x$
  line. In particular, a function equals its own inverse iff its graph
  is symmetric about the $y = x$ line.
\item Many of the results on inverse functions and their properties
  have graphical interpretations. For instance, the fact that the
  derivative of the inverse function is the reciprocal of the
  derivative corresponds to the geometrical fact that reflection about
  the $y = x$ line inverts slopes of tangent lines. Similarly, the
  results relating increase/decrease and concave up/down for a
  function and its inverse function can all be deduced graphically.
\end{enumerate}

\subsection{In the real world}

Words... (from now on, we restrict ourselves to functions whose domain
and range are both subsets of the real numbers)

\begin{enumerate}

\item An increasing function is one-to-one. A decreasing function is
  one-to-one.
\item A {\em continuous} function on an {\em interval} is one-to-one
  if and only if it is either increasing throughout the interval or
  decreasing throughout the interval.
\item If the derivative of a continuous function on an interval is of
  constant sign everywhere, except possibly at a few isolated points
  where it is either zero or undefined, then the function is
  one-to-one on the interval. Note that we need the function to be
  continuous {\em everywhere} on the interval, even though it is
  tolerable for the derivative to be undefined at a few isolated
  points.
\item In particular, a one-to-one function cannot have local extreme
  values.
\item A continuous one-to-one function is increasing if and only if
  its inverse function is increasing, and is decreasing if and only if
  its inverse function is decreasing.
\item {\em Point added, not present in original executive summary of
  lecture notes}: If a one-to-one function on an interval satisfies
  the intermediate value property, then it is continuous. This is
  because the function cannot {\em jump} suddenly since it needs to
  cover all intermediate values. Note that the analogous statement is
  {\em not} true if we drop either the assumption of one-to-one or the
  assumption of the intermediate value property. (Think of
  $\sin(1/x)$, for instance).
\item If $f$ is one-to-one {\em and differentiable at a point $a$}
  (emphasis added) with $f'(a) \ne 0$, with $f(a) = b$, then
  $(f^{-1})'(b) = 1/f'(a)$. This agrees with the previous point and
  also shows that the rates of relative increase are inversely
  proportional.
\item Two extreme cases of interest are: $f'(a) = 0$, $f(a) = b$. In
  this case, $f$ has a horizontal tangent at $a$ and $f^{-1}$ has a
  vertical tangent at $b$. The horizontal tangent is typically also a
  point of inflection. It is definitely {\em not} a point of local
  extremum. Similarly, if $(f^{-1})'(b) = 0$, then $f^{-1}$ has a
  horizontal tangent at $b$ and $f$ has a vertical tangent at $a$.
\item A slight complication occurs when $f$ has one-sided derivatives
  but is not differentiable. If both one-sided derivatives of $f$
  exist and are nonzero, then both one-sided derivatives of $f^{-1}$
  (at the image point) exist and are nonzero. When $f$ is increasing,
  the left hand derivative of $f^{-1}$ is the reciprocal of the left
  hand derivative of $f$, and the right hand derivative of $f^{-1}$ is
  the reciprocal of the right hand derivative of $f$. When $f$ is
  decreasing, the right hand derivative of $f^{-1}$ is the reciprocal
  of the left hand derivative of $f$, and the left hand derivative of
  $f^{-1}$ is the reciprocal of the right hand derivative of $f$.
\item The second derivative of $f^{-1}$ at $f(a)$ is
  $-f''(a)/(f'(a))^3$. In particular, the second derivative of the
  inverse function at the image point depends on the values of both
  the first and the second derivatives of the function at the point.
\item If $f$ is increasing, the sense of concavity of $f^{-1}$ is
  opposite to that of $f$. If $f$ is decreasing, the sense of
  concavity of $f^{-1}$ is the same as that of $f$.
\end{enumerate}

Actions ...

\begin{enumerate}
\item For functions on intervals, {\em to check if the function
  is one-to-one}, we can compute the derivative and check if it has
  constant sign everywhere except possibly at isolated points.
\item In order to find $(f^{-1})'$ at a particular point, given an
  explicit description of $f$, it is {\em not} necessary to find an
  explicit description of $f^{-1}$. Rather, it is enough to find
  $f^{-1}$ at that particular point and then calculate the derivative
  using the above formula. The same is true for $(f^{-1})''$, except
  that now we need to compute the values of both $f'$ and $f''$.
\item The idea can be extended somewhat to finding $(f^{-1})'$ when $f$
  satisfies a differential equation that expresses $f'(x)$ in terms of
  $f(x)$ (with no direct appearance of $x$).
\end{enumerate}

\section{Logarithms, exponents, derivatives and integrals}

\subsection{Logarithm and exponential: basics}

\begin{enumerate}
\item The {\em natural logarithm} is a one-to-one function with domain
  $(0,\infty)$ and range $\R$, and is defined as $\ln(x) := \int_1^x
  (dt/t)$.
\item The natural logarithm is an increasing function that is concave
  down. It satisfies the identities $\ln(1) = 0$, $\ln(ab) = \ln(a) +
  \ln(b)$, $\ln(a^r) = r\ln a$, and $\ln(1/a) = -\ln a$.
\item The limit $\lim_{x \to 0} \ln(x)$ is $-\infty$ and the limit
  $\lim_{x \to \infty} \ln(x)$ is $+\infty$. Note that $\ln$ goes off
  to $+\infty$ at $\infty$ even though its derivative goes to zero as
  $x \to + \infty$.
\item The derivative of $\ln(x)$ is $1/x$ and the derivative
  of $\ln(kx)$ is also $1/x$. The derivative of $\ln(x^r)$
  is $r/x$.
\item The antiderivative of $1/x$ is $\ln |x| + C$. What this really
  means is that the antiderivative is $\ln(-x) + C$ when $x$ is
  negative and $\ln(x) + C$ when $x$ is positive. If we consider $1/x$
  on both positive and negative reals, the constant on the negative
  side is unrelated to the constant on the positive side.
\item $e$ is defined as the unique number $x$ such that $\ln(x) =
  1$. $e$ is approximately $2.718$. In particular, it is between $2$ and
  $3$.
\item The inverse of the natural logarithm function is denoted $\exp$,
  and $\exp(x)$ is also written as $e^x$. When $x$ is a rational
  number, $e^x = e^x$ (i.e., the two definitions of exponentiation
  coincide). In particular, $e^1 = e$, $e^0 = 1$, etc.
\item The function $\exp$ equals its own derivative and hence also its
  own antiderivative. Further, the derivative of $x \mapsto e^{mx}$ is
  $me^{mx}$. Similarly, the integral of $e^{mx}$ is $(1/m)e^{mx} +C$.
\item We have $\exp(x + y) = \exp(x)\exp(y)$, $\exp(rx) =
  (\exp(x))^r$, $\exp(0) = 1$, and $\exp(-x) = 1/\exp(x)$. All of
  these follow from the corresponding identities for $\ln$.
\end{enumerate}

Actions...

\begin{enumerate}
\item We can calculate $\ln(x)$ for given $x$ by using the usual
  methods of estimating the values of integrals, applied to the
  function $1/x$. We can also use the known properties of logarithms,
  as well as approximate $\ln$ values for some specific $x$ values, to
  estimate $\ln x$ to a reasonable approximation. For this, it helps
  to remember $\ln 2$, $\ln 3$, and $\ln 5$ or $\ln 10$.
\item Since both $\ln$ and $\exp$ are one-to-one, we can {\em cancel}
  $\ln$ from both sides of an equation and similarly {\em cancel}
  $\exp$. Technically, we cancel $\ln$ by applying $\exp$ to both
  sides, and we cancel $\exp$ by applying $\ln$ to both sides.
\end{enumerate}

\subsection{Integrations involving logarithms and exponents}

Words/actions ...

\begin{enumerate}
\item If the numerator is the derivative of the denominator, the
  integral is the logarithm of the (absolute value of) the
  denominator. In symbols, $\int g'(x)/g(x) \, dx = \ln|g(x)| + C$.
\item More generally, whenever we see an expression of the form
  $g'(x)/g(x)$ inside the integrand, we should consider the
  substitution $u = \ln |g(x)|$. Thus, $\int f(\ln|g(x)|)g'(x)/g(x) \,
  dx = \int f(u) \, du$ where $u = \ln|g(x)|$.
\item $\int f(e^x) e^x \, dx = \int f(u) \, du$ where $u = e^x$.
\item $\int e^x[f(x) + f'(x)] \, dx = e^x f(x) + C$.
\item $\int e^{f(x)} f'(x) \, dx = e^{f(x)} + C$.
\item Trigonometric integrals: $\int \tan x \, dx = -\ln|\cos x| + C$,
  and similar integration formulas for $\cot$, $\sec$ and $\csc$:
  $\int \cot x \, dx = \ln|\sin x| + C$, $\int \sec x = \ln|\sec x +
  \tan x| + C$, and $\int \csc x \, dx = \ln |\csc x - \cot x| + C$.
\end{enumerate}

\subsection{Exponents with arbitrary bases, exponents}

Words ...

\begin{enumerate}

\item For $a > 0$ and $b$ real, we define $a^b := \exp(b \ln a)$. This
  coincides with the usual definition when $b$ is rational.
\item All the laws of exponents that we are familiar with for integer
  and rational exponents continue to hold. In particular, $a^0 = 1$,
  $a^{b + c} = a^b \cdot a^c$, $a^1 = a$, and $a^{bc} = (a^b)^c$.
\item The exponentiation function is continuous in the exponent
  variable. In particular, for a fixed value of $a > 0$, the function
  $x \mapsto a^x$ is continuous. When $a \ne 1$, it is also one-to-one
  with domain $\R$ and range $(0,\infty)$, with inverse function $y
  \mapsto (\ln y)/(\ln a)$, which is also written as $\log_a(y)$. In
  the case $a > 1$, it is an increasing function, and in the case $a <
  1$, it is a decreasing function.
\item The exponentiation function is also continuous in the base
  variable. In particular, for a fixed value of $b$, the function $x
  \mapsto x^b$ is continuous. When $b \ne 0$, it is a one-to-one
  function with domain and range both $(0,\infty)$, and the inverse
  function is $y \mapsto y^{1/b}$. In case $b > 0$, the function is
  increasing, and in case $b < 0$, the function is decreasing.
\item Actually, we can say something stronger about $a^b$ -- it is
  {\em jointly} continuous in both variables. This is hard to describe
  formally here, but what it approximately means is that if $f$ and
  $g$ are both continuous functions, and $f$ takes positive values
  only, then $x \mapsto [f(x)]^{g(x)}$ is also continuous.
\item The derivative of the function $[f(x)]^{g(x)}$ is
  $[f(x)]^{g(x)}$ times the derivative of its logarithm, which is
  $g(x)\ln(f(x))$. We can further simplify this to obtain the formula:

  $$\frac{d}{dx} \left([f(x)]^{g(x)}\right) = [f(x)]^{g(x)}\left[\frac{g(x)f'(x)}{f(x)} + g'(x)\ln(f(x))\right]$$
\item Special cases worth noting: the derivative of $(f(x))^r$ is
  $r(f(x))^{r-1}f'(x)$ and the derivative of $a^{g(x)}$ is
  $a^{g(x)}g'(x) \ln a$.
\item Even further special cases: the derivative of $x^r$ is
  $rx^{r-1}$ and the derivative of $a^x$ is $a^x \ln a$.
\item The antiderivative of $x^r$ is $x^{r+1}/(r + 1) + C$ (for $r \ne
  -1$) and $\ln|x| + C$ for $r = -1$. The antiderivative of $a^x$ is
  $a^x/(\ln a) + C$ for $a \ne 1$ and $x + C$ for $a = 1$.
\item The logarithm $\log_a(b)$ is defined as $(\ln b)/(\ln a)$. This
  is called the logarithm of $b$ to base $a$. Note that this is
  defined when $a$ and $b$ are both positive and $a \ne 1$. This
  satisfies a bunch of identities, most of which are direct
  consequences of identities for the natural logarithm. In particular,
  $\log_a(bc) = \log_a(b) + \log_a(c)$, $\log_a(b)\log_b(c) =
  \log_a(c)$, $\log_a(1) = 0$, $\log_a(a) = 1$, $\log_a(a^r) = r$,
  $\log_a(b) \cdot \log_b(a) = 1$ and so on.
\item {\em Added}: The derivative of $\log_{f(x)}(g(x))$ is given by:

  $$\frac{d}{dx}\left[\log_{f(x)}(g(x))\right] = \frac{\ln(f(x))g'(x)/g(x) - \ln(g(x))f'(x)/f(x)}{(\ln(f(x)))^2}$$
\end{enumerate}

Actions...

\begin{enumerate}
\item We can use the formulas here to differentiate expressions of the
  form $f(x)^{g(x)}$, and even to differentiate longer exponent towers
  (such as $x^{x^x}$ and $2^{2^x}$).
\item To solve an integration problem with exponents, it may be most
  prudent to rewrite $a^b$ as $\exp(b \ln a)$ and work from there
  onward using the rules mastered earlier. Similarly, when dealing
  with relative logarithms, it may be most prudent to convert all
  expressions in terms of natural logairthms and then use the rules
  mastered earlier.
\end{enumerate}


\end{document}
