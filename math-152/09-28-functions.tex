\documentclass[10pt]{amsart}

%Packages in use
\usepackage{fullpage, hyperref, vipul, enumerate}

%Title details
\title{Class quiz: September 28; Topic: Functions}
\author{Vipul Naik}
%List of new commands

\begin{document}
\maketitle

Your name (print clearly in capital letters): $\underline{\qquad\qquad\qquad\qquad\qquad\qquad\qquad\qquad\qquad\qquad}$

Write your answer in the space provided. In the space below, you can
explain your work if you want (this will not affect scoring). I may or
may not get time to look at the work you have done, but it may help
you recall how you arrived at a particular answer.

You are expected to take about one minute per question.

Questions marked with a (*) are questions that are somewhat trickier,
with the probability of getting the question correct being about
$50\%$ or less. For these questions, you are free to discuss the questions
with others while making your attempt.

\begin{enumerate}

\item Suppose $f$ and $g$ are functions from $\R$ to $\R$. Suppose
  both $f$ and $g$ are even, i.e., $f(x) = f(-x)$ for all $x \in \R$
  and $g(x) = g(-x)$ for all $x \in \R$. Which of the following is
  {\em not} guaranteed to be an even function from the given
  information? {\em Last year's performance}: 11/15 correct

  {\em Note: For this question, it is possible to solve the question
  by taking a few simple examples. You're free to do this, but the
  recommended method for tackling the question is to handle it {\bf
  abstractly}, i.e., try to prove or disprove in general for each
  function whether it is even.}

  \begin{enumerate}[(A)]
  \item $f + g$, i.e., the function $x \mapsto f(x) + g(x)$
  \item $f - g$, i.e., the function $x \mapsto f(x) - g(x)$
  \item $f \cdot g$, i.e., the function $x \mapsto f(x)g(x)$
  \item $f \circ g$, i.e., the function $x \mapsto f(g(x))$
  \item None of the above, i.e., they are all guaranteed to be even
    functions.
  \end{enumerate}

  \vspace{0.1in}
  Your answer: $\underline{\qquad\qquad\qquad\qquad\qquad\qquad\qquad}$
  \vspace{1.5in}
  
\item (*) Suppose $f$ and $g$ are functions from $\R$ to $\R$. Suppose
  both $f$ and $g$ are odd, i.e., $f(-x) = -f(x)$ for all $x \in \R$
  and $g(-x) = -g(x)$ for all $x \in \R$. Which of the following is
  {\em not} guaranteed to be an odd function from the given
  information? {\em Last year's performance}: 7/15 correct

  {\em Note: For this question, it is possible to solve the question
  by taking a few simple examples. You're free to do this, but the
  recommended method for tackling the question is to handle it {\bf
  abstractly}, i.e., try to prove or disprove in general for each
  function whether it is odd.}

  \begin{enumerate}[(A)]
  \item $f + g$, i.e., the function $x \mapsto f(x) + g(x)$
  \item $f - g$, i.e., the function $x \mapsto f(x) - g(x)$
  \item $f \cdot g$, i.e., the function $x \mapsto f(x)g(x)$
  \item $f \circ g$, i.e., the function $x \mapsto f(g(x))$
  \item None of the above, i.e., they are all guaranteed to be odd functions.
  \end{enumerate}

  \vspace{0.1in}
  Your answer: $\underline{\qquad\qquad\qquad\qquad\qquad\qquad\qquad}$
  \vspace{1.5in}

\newpage
\item For which of the following pairs of polynomial functions $f$ and
  $g$ is it true that $f \circ g \ne g \circ f$? {\em Last year's
    performance}: 14/15 correct

  \begin{enumerate}[(A)]
  \item $f(x) := x^2$ and $g(x) := x^3$
  \item $f(x) := x + 1$ and $g(x) := x + 2$
  \item $f(x) := x^2 + 1$ and $g(x) := x^2 + 1$
  \item $f(x) := -x$ and $g(x) := x^2$
  \item $f(x) := -x$ and $g(x) := x^3$
  \end{enumerate}

  \vspace{0.1in}
  Your answer: $\underline{\qquad\qquad\qquad\qquad\qquad\qquad\qquad}$
  \vspace{1.5in}

\item (*) Which of the following functions is {\em not} periodic? {\em
    Last year's performance}: 8/15 correct

  \begin{enumerate}[(A)]
  \item $\sin(x^2)$
  \item $\sin^2x$
  \item $\sin(\sin x)$
  \item $\sin(x + 13)$
  \item $(\sin x) + 13$
  \end{enumerate}

  \vspace{0.1in}
  Your answer: $\underline{\qquad\qquad\qquad\qquad\qquad\qquad\qquad}$
  \vspace{1.5in}

\item (*) What is the domain of the function $\sqrt{1 - x} + \sqrt{x -
    2}$? Here, domain refers to the {\em largest subset of the reals}
  on which the function can be defined. {\em Last year's performance}:
  8/15 correct

  {\em Hint: Think clearly, first about what the domain of each of the
    two functions being added is, and then about whether you need to
    take the union or the intersection of the domains of the
    individual functions.}

  \begin{enumerate}[(A)]
  \item $(1,2)$
  \item $[1,2]$
  \item $(-\infty,1) \cup (2,\infty)$
  \item $(-\infty,1] \cup [2,\infty)$
  \item None of the above
  \end{enumerate}

  \vspace{0.1in}
  Your answer: $\underline{\qquad\qquad\qquad\qquad\qquad\qquad\qquad}$
  \vspace{1.5in}

\end{enumerate}
\end{document}