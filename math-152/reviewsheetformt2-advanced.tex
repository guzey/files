\documentclass[10pt]{amsart}
\usepackage{fullpage,hyperref,vipul, graphicx}
\title{Review sheet for midterm 2: advanced}
\author{Math 152, Section 55 (Vipul Naik)}

\begin{document}
\maketitle

This is the part of the review sheet that we will concentrate on during the review session.

\section{Left-overs from differentiation basics}

\subsection{Derivative as rate of change}

No error-spotting exercises
\subsection{Implicit differentiation}

No error-spotting exercises

\section{Increase/decrease, maxima/minima, concavity, inflection, tangents, cusps, asymptotes}

\subsection{Rolle's, mean value, increase/decrease, maxima/minima}

Error-spotting exercises

\begin{enumerate}
\item If a function $f$ has a local maximum at a point $c$ in its
  domain, then $f$ is increasing on the immediate left of $c$ and
  decreasing on the immediate right of $c$.
\item Consider the function:

  $$f(x) := \lbrace\begin{array}{rl} x^3 - 12x + 14, & x \le 1 \\ x^2 - 6x + 8, & x > 1 \\\end{array}$$

  The derivative is:

  $$f'(x) = \lbrace\begin{array}{rl} 3x^2 - 12, & x \le 1 \\ 2x - 6, & x > 1 \\\end{array}$$

  The solutions for $f'(x) = 0$ are $x = -2$ and $x = 2$ (for the $x
  \le 1$ case) and $x = 3$ (the $x > 1$ case). Thus, the critical
  points are at $x = -2$, $x = 2$, and $x = 3$.
\item Consider the function:

  $$f(x) := x^4 - x + 1$$

  The derivative is:

  $$f'(x) = 4x^3 - 1$$

  Solve $f'(x) = 0$ and we get $x = (1/4)^{1/3}$. Thus, $f$ has a
  local maximum at $x = (1/4)^{1/3}$. The local maximum value is:

  $$4((1/4)^{1/3})^3 - 1$$

  which is $0$.

\item Consider the function

  $$f(x) := \frac{1}{x^3 - 1}$$

  The derivative is:

  $$f'(x) = \frac{3x^2}{(x^3 - 1)^2}$$

  The derivative is zero at $x = 0$, so that gives a critical
  point. Also, the derivative is undefined at $x = 1$, so that gives
  another critical point for $f$.

\item An everywhere differentiable function $f$ on $\R$ has critical
  points at $2$, $5$, and $9$ with corresponding function values $11$,
  $16$, and $3$ respectively. Thus, the absolute maximum value of $f$
  is $16$ and the absolute minimum value is $3$.
\end{enumerate}
 
\subsection{Concave up/down and points of inflection}

Error-spotting exercises ...

\begin{enumerate}
\item Consider the function

  $$f(x) := 3x^5 - 5x^4 + 12x + 17$$

  The derivative is:

  $$f'(x) = 15x^4 - 20x^3 + 12$$

  The second derivative is:

  $$f''(x) = 60x^3 - 60x^2$$

  The zeros of this are $x = 0$ and $x = 1$. The function thus has
  points of inflection at the points on the graph corresponding to $x
  = 0$ and $x = 1$.
\item To check whether a critical point for the first derivative gives
  a point of inflection for the graph of the function, we need to
  check the sign of the third derivative. If the third derivative is
  nonzero, we get a point of inflection. If the third derivative is
  zero, then we {\em do not} get a point of inflection.
\end{enumerate}

\subsection{Tangents, cusps, and asymptotes}

Cute fact: Rational functions are asymptotically polynomial, and the
polynomial to which a given rational function is asymptotic (both
directions) is obtained by doing long division and looking at the
quotient. If the degree of the numerator is one more than that of the
denominator, we get an oblique (linear) asymptote. If the numerator
and denominator have equal degree, we get a horizontal asymptote (both
directions) with nonzero value. If the numerator has smaller degree,
the $x$-axis is the horizontal asymptote (both directions).

Error-spotting exercises

\begin{enumerate}
\item If $\lim_{x \to \infty} f(x) = L$ with $L$ a finite number, then
  $\lim_{x \to \infty} f'(x) = 0$.
\item If $\lim_{x \to \infty} f'(x) = 0$, then $\lim_{x \to \infty}
  f(x) = L$, with $L$ a finite number.
\item If $f'$ has a vertical tangent at a point $a$ in its domain,
  then $f$ has a point of inflection at $(a,f(a))$.
\item If $f'$ has a vertical cusp at a point $a$ in its domain, then
  $f$ has a local extreme value at $a$.
\item Suppose $f$ and $g$ are functions defined on all of
  $\R$. Suppose $f$ has a vertical tangent at a point $a$ in its
  domain and $g$ has a vertical tangent at a point $b$ in its
  domain. Then $f + g$ has a vertical tangent at $a + b$ and $f - g$
  has a vertical tangent at $a - b$.
\item Suppose $f$ and $g$ are functions, both defined on $\R$. Suppose
  $f$ and $g$ both have vertical tangents at a point $a$ in their
  domain (i.e., at the same point in the domain). Then, the sum $f +
  g$ also has a vertical tangent at $a$.
\item Suppose $f$ and $g$ are functions, both defined on $\R$. Suppose
  $f$ and $g$ both have vertical tangents at a point $a$ in their domain (i.e.,
  at the same point in the domain). Then, the pointwise product $f
  \cdot g$ also has a vertical tangent at $a$. {\em This is trickier
  than it looks!}
\end{enumerate}
\section{Max-min problems}

Smart thoughts for smart people ...

\begin{enumerate}
\item Before getting started on the messy differentiation to find
  critical points, think about the constraints and the endpoints. Is
  it obvious that the function will attain a minimum/maximum at one of
  the endpoints? What are the values of the function at the endpoints?
  (If no endpoints, take limiting values as you go in one direction of
  the domain). Is there an intuitive reason to believe that the
  function attains its optimal value somewhere {\em in between} rather
  than at an endpoint? Is there some kind of trade-off to be made? Are
  there some things that can be said qualitatively about where the
  trade-off is likely to occur?
\item Feel free to convert your function to an equivalent function
  such that the two functions rise and fall together. This reduces the
  burden of messy expressions.
\item {\em Cobb-Douglas production}: For $p,q > 0$, the function $x
  \mapsto x^p(1 - x)^q$ attains a local maximum at $p/(p + q)$. In
  fact, this is the absolute maximum on $[0,1]$, and the function
  value is $p^pq^q/(p + q)^{p + q}$. This is important because this function
  appears in disguise all the time (e.g., maximizing area of rectangle
  with given perimeter, etc.)
\item A useful idea is that when dividing a resource into two
  competing uses, and one use is hands-down better than the other, the
  {\em best} use happens when the entire resource is devoted to the
  better use. However, the {\em worst} may well happen somewhere in
  between, because divided resources often perform even worse than
  resources devoted wholeheartedly to a bad use. This is seen in
  perimeter allocation to boundaries with the objective function being
  the total area, and area allocation to surfaces with the objective
  function being the total volume.
\item When we want to {\em maximize} something subject to a collection
  of many constraints, the most relevant constraint is the {\em
  minimum} one. Think of the ladder-through-the-hallway problem, or
  the truck-going-under-bridges problem. 
\end{enumerate}

Error-spotting exercises

\begin{enumerate}
\item The absolute maximum among the values of a (?) function (of reals)
  at integers is attained at the integer closest to the point at which
  it attains its absolute maximum among all reals.
\item The absolute maximum among the values of a (?) function (of reals)
  at integers is attained at one of the integers closest to the point
  at which it attains a local maximum.
\item To maximize the sum of two functions is equivalent to maximizing
  each one separately and then finding the common point of maximum.
\item If $f$ is a function that is continuous and concave up on an
  interval $[a,b]$, then the absolute minimum of $f$ always occurs at
  an interior point and the absolute maximum of $f$ always occurs at
  an endpoint. {\em This is a little subtle, because it's almost but
  not completely correct. Think through it clearly!}
\item Consider the function:

  $$f(x) := \lbrace\begin{array}{rl} x^3, & 0 \le x \le 1 \\ x^2, & 1 < x \le 2 \\\end{array}$$

  Then, $f'$ is increasing on $[0,2]$, so $f$ is concave up on
  $[0,2]$.
\end{enumerate}

\section{Definite and indefinite integration}

\subsection{Definition and basics}

Error-spotting exercises ...

\begin{enumerate}
\item If $P_1$ and $P_2$ are partitions of $[a,b]$ and $\| P_2 \| \le
  \| P_1 \|$, then $P_2$ is finer than $P_1$.
\item If $P_1$ and $P_2$ are partitions of $[a,b]$ such that $P_2$ is
  finer than $P_1$, and $f$ is a bounded function on $[a,b]$, then
  $L_f(P_2) \le L_f(P_1)$ and $U_f(P_2) \le U_f(P_1)$.
\item For any continuous function $f$ on $[a,b]$, the number of parts
  $n$ we need in a regular partition of $[a,b]$ so that the integral
  is bounded in an interval of length $L$ is proportional to $1/n$.
\end{enumerate}
\subsection{Definite integral, antiderivative, and indefinite integral}

Error-spotting exercises ...

\begin{enumerate}
\item Consider the function $f(x) := \int_x^{x^2} \sin x \, dx$. Then
  $f'(x) = \sin(x^2) - \sin(x)$.
\item Suppose $f$ is a function on the nonzero reals such that $f'(x)
  = 1/x^2$ for all $x \in \R$. Then, we must have $f(x) = 1/x + C$ for
  some constant $C$.
\end{enumerate}

\subsection{Higher derivatives, multiple integrals, and initial/boundary conditions}

Error-spotting exercises ...

\begin{enumerate}
\item Suppose $F$ and $G$ are everywhere $k$ times differentiable
  functions for $k$ a positive integer. If the $k^{th}$ derivatives of
  the functions $F$ and $G$ are equal, then $F - G$ is a polynomial of
  degree $k$.
\item Suppose $F$ is a function defined on nonzero reals and $F''(x) =
  1/x^3$ for all $x$. Then, $F$ is of the form $F(x) = 1/x + C$ where
  $C$ is a real constant.
\end{enumerate}
\subsection{Reversing the chain rule}

No error-spotting exercises

\subsection{$u$-substitutions for definite integrals}

No error-spotting exercises

\subsection{Symmetry and integration}

No error-spotting exercises

\subsection{Mean-value theorem}

No error-spotting exercises

\subsection{Application to area computations}

No error-spotting exercises

\section{Graphing and miscellanea on functions}

\subsection{Symmetry yet again}

Words...

\begin{enumerate}
\item All mathematics is the study of symmetry (well, not all).
\item One interesting kind of symmetry that we often see in the graph
  of a function is {\em mirror symmetry} about a vertical line. This
  means that the graph of the function equals its reflection about the
  vertical line. If the vertical line is $x = c$ and the function is
  $f$, this is equivalent to asserting that $f(x) = f(2c - x)$ for all
  $x$ in the domain, or equivalently, $f(c + h) = f(c - h)$ whenever
  $c + h$ is in the domain. In particular, the domain itself must be
  symmetric about $c$.
\item A special case of mirror symmetry is the case of an {\em even
  function}. An even function is a function with mirror symmetry about
  the $y$-axis. In other words, $f(x) = f(-x)$ for all $x$ in the
  domain. (Even also implies that the domain should be symmetric about $0$).
\item Another interesting kind of symmetry that we often see in the
  graph of a function is {\em half-turn symmetry} about a point on the
  graph. This means that the graph equals the figure obtained by
  rotating it by an angle of $\pi$ about that point. A point $(c,d)$
  is a point of half-turn symmetry if $f(x) + f(2c - x) = 2d$ for all
  $x$ in the domain. In particular, the domain itself must be
  symmetric about $c$. If $f$ is defined at $c$, then $d = f(c)$.
\item A special case of half-turn symmetry is an odd function, which
  is a function having half-turn symmetry about the origin.
\item Another symmetry is {\em translation symmetry}. A function is
  {\em periodic} if there exists $h > 0$ such that $f(x + h) = f(x)$
  for all $x$ in the domain of the function (in particular, the domain
  itself should be invariant under translation by $h$). If a smallest
  such $h$ exists, then such an $h$ is termed the period of $f$.
\item A related notion is that of a function that is {\em periodic
  with shift}. A function is periodic with shift if there exists $h >
  0$ and $k \in \R$ such that $f(x + h) - f(x) = k$ for all $x \in
  \R$. Note that if $k$ is nonzero, the function isn't periodic.

  If $f$ is differentiable for all real numbers, then $f'$ is periodic
  if and only if $f$ is periodic with shift. In particular, if $f'$ is
  periodic with period $h$, then $f(x + h) - f(x)$ is constant. If
  this constant value is $k$, then the graph of $f$ has a
  two-dimensional translational symmetry by $(h,k)$ and its multiples.

  A function that is periodic with shift can be expressed as the sum
  of a linear function (slope $k/h$) and a periodic function. The
  linear part represents the secular trend and the periodic part
  represents the seasonal variation.
\end{enumerate}

Derivative facts...

\begin{enumerate}
\item The derivative of an even function, if defined everywhere, is
  odd. Any antiderivative of an odd function is even.
\item The derivative of an odd function is even. Any antiderivative of
  an even function is an odd function plus a constant.
\item The derivative of a function with mirror symmetry has half turn
  symmetry about the corresponding $x$-value and has value $0$ at that
  $x$-value. (For a more detailed description of these, see the
  solutions to the November 12 whoppers).
\item Assuming that $f'$ is defined and does not change sign
  infinitely often on a neighborhood of $c$, we have that if $x = c$
  is an axis of mirror symmetry for the graph of $f$, then $c$ is a
  point of local extremum. The reason is that if $f$ is increasing on
  the immediate left, it must be decreasing on the immediate right,
  and similarly ...
\item Assuming that $f''$ is defined and does not change sign
  infinitely often on a neighborhood of $c$, we have that if
  $(c,f(c))$ is a point of half-turn symmetry for the graph of $f$,
  then it is also a point of inflection for the graph. The reason is
  that if $f$ is concave up on the immediate left, it must be concave
  down on the immediate right, and similarly ...
\item The conver statements to the above two do not hold: most points
  of inflection do not give points of half-turn symmetry, and most
  local extrema do not give axes of mirror symmetry.
\item If $f$ has more than one axis of mirror symmetry, then it is
  periodic. Conversely, if $f$ is periodic with period $h$, and has an
  axis of mirror symmetry $x = c$, then all $x = c + (nh/2)$, $n$ an
  integer, are axes of mirror symmetry.
\item If $f$ has more than one point of half-turn symmetry, then it is
  periodic with shift. Conversely, if $f$ is periodic with shift and
  has a point of half-turn symmetry, it has infinitely many points of
  half-turn symmetry.
\end{enumerate}

Cute facts...

\begin{enumerate}
\item Constant functions enjoy mirror symmetry about every vertical
  line and half-turn symmetry about every point on the graph.
\item Nonconstant linear functions enjoy half-turn symmetry about
  every point on their graph. They do not enjoy any mirror symmetry
  (in the sense of mirror symmetry about vertical lines) because they
  are everywhere increasing or everywhere decreasing. (They do have
  mirror symmetry about {\em oblique} lines, but this is not a kind of
  symmetry that we are considering).
\item Quadratic (nonlinear) functions enjoy mirror symmetry about the
  line passing through the vertex (which is the unique absolute
  maximum/minimum, depending on the sign of the leading
  coefficient). They do not enjoy any half-turn symmetry.
\item Cubic functions enjoy half-turn symmetry about the point of
  inflection, and no mirror symmetry. Either the first derivative does
  not change sign anywhere, or it becomes zero at exactly one point,
  or there is exactly one local maximum and one local minimum,
  symmetric about the point of inflection.
\item Functions of higher degree do not necessarily have either
  half-turn symmetry or mirror symmetry.
\item More generally, we can say the following for sure: a nonconstant
  polynomial of even degree greater than zero can have at most one
  line of mirror symmetry and no point of half-turn symmetry. A
  nonconstant polynomial of odd degree greater than one can have at
  most one point of half-turn symmetry and no line of mirror symmetry.
\item The sine function is an example of a function where the points
  of inflection and the points of half-turn symmetry are the same: the
  multiples of $\pi$. Similarly, the points with vertical axis of
  symmetry are the same as the points of local extrema: odd multiples
  of $\pi/2$.
\item A polynomial is an even function iff all its terms have even
  degree. Such a polynomial is termed an {\em even polynomial}. A
  polynomial is an odd function iff all its terms have odd
  degree. Such a polynomial is termed an {\em odd polynomial}.
\end{enumerate}

Actions ...

\begin{enumerate}
\item Worried about periodicity? Don't be worried if you only see
  polynomials and rational functions. Trigonometric functions should
  make you alert. Try to fit in the nicest choices of period. Check if
  smaller periods can work (e.g., for $\sin^2$, the period is
  $\pi$). Even if the function in and of itself is not periodic, it
  might have a periodic derivative or a periodic second
  derivative. The sum of a linear function and a periodic function has
  periodic derivative, and the sum of a quadratic function and a
  periodic function has a periodic second derivative.
\item Want to milk periodicity? Use the fact that for a periodic
  function, the behavior everywhere is just the behavior over one
  period translates over and over again. If the first derivative is
  periodic, the increase/decrease behavior is periodic. If the second
  derivative is periodic, the concave up/down behavior is periodic.
\item Worried about even and odd, and half-turn symmetry and mirror
  symmetry? If you are dealing with a quadratic polynomial, or a
  function constructed largely from a quadratic polynomial, you are
  probably seeing some kind of mirror symmetry. For cubic polynomials
  and related constructions, think half-turn symmetry.
\item Use also the cues about even and odd polynomials.
\end{enumerate}

\subsection{Graphing a function}

Actions ...

\begin{enumerate}
\item To graph a function, a useful first step is finding the domain
  of the function.
\item It is useful to find the intercepts and plot a few additional points.
\item Try to look for symmetry: even, odd, periodic, mirror symmetry,
  half-turn symmetry, and periodic derivative.
\item Compute the derivative. Use that to find the critical points,
  the local extreme values, and the intervals where the function
  increases and decreases.
\item Compute the second derivative. Use that to find the points of
  inflection and the intervals where the function is concave up and
  concave down.
\item Look for vertical tangents and vertical cusps. Look for vertical
  asymptotes and horizontal asymptotes. For this, you may need to
  compute some limits.
\item Connect the dots formed by the points of interest. Use the
  information on increase/decrease and concave up/down to join these
  points. To make your graph a little better, compute the first
  derivative (possibly one-sided) at each of these points and start
  off your graph appropriately at that point.
\end{enumerate}

Subtler points...

\begin{enumerate}
\item When graphing a function, there may be many steps where you need
  to do some calculations and solve equations and you are unable to
  carry them out effectively. You can skip some of the steps and come
  back to them later.
\item If you cannot solve an equation exactly, try to approximate the
  locations of roots using the intermediate value theorem or other
  results such as Rolle's theorem.
\item In some cases, it is helpful to graph multiple functions
  together, on the same graph. For instance, we may be interested in
  graphing a function and its second and higher derivatives. There are
  other examples, such as graphing a function and its translates, or a
  function and its multiplicative shifts.
\item A graph can be used to suggest things about a function that are
  not obvious otherwise. However, the graph should not be used as
  conclusive evidence. Rather, the steps used in drawing the graph
  should be retraced and used to give an algebraic proof.
\item We are sometimes interested in sketching curves that are not
  graphs of functions. This can be done by locally expressing the
  curve piecewise as the graph of a function. Or, we could use many
  techniques similar to those for graphing functions.
\item For a function with a piecewise description, we plot each piece
  within its domain. At the points where the definition changes,
  determine the one-sided limits of the function and its first and
  second derivatives. Use this to make the appropriate open circles,
  asymptotes, etc.
\end{enumerate}

\section{Tricky topics}

\subsection{Piecewise definition by interval: new issues}

Before looking at these, please review the corresponding material on
piecewise definition by interval in the previous midterm review sheet.

\begin{enumerate}
\item Composition involving piecewise definitions is tricky. The
  limit, continuity and differentiation theorems for composition do
  not hold for one-sided approach. If one of the functions is
  decreasing, then things can get flipped. For piecewise definitions,
  when composing, we need to think clearly about how the intervals
  transform.

  Please review the midterm question on composition (midterm 1,
  question 7) of piecewise definitions. The key idea is as follows:
  for the composition $f \circ g$, we make cases to determine the
  values of $g$ for which the image under $g$ would land in a
  particular piece for the definition of $f$. Considering all cases is
  extremely painful and we are usually able to take shortcuts based on
  the nature of the problem.
\item For a function with piecewise definition, the points where the
  definition changes are endpoints for each definition, and hence,
  these points are possible candidates for critical points, points of
  inflection, and local extreme. They're just {\em candidates} (so
  they may not be any of these) but they're worth checking out.
\item A related helpful concept is that of {\em how smoothly} a
  function transitions at a point where its definition changes. 
\item At the one extreme are the discontinuous transitions, where the
  function has a non-removable discontinuity at the point. Such a
  transition may be a jump discontinuity (if both one-sided limits are
  defined but unequal) or something even worse, such as an infinite or
  oscillatory discontinuity.

  For functions with a discontinuity at a point, it makes sense to
  talk of one-sided derivatives only from the side where the function
  is continuous; of course, this one-sided derivative may still not
  exist.
\item A somewhat smoother transition occurs where the function is
  continuous but not differentiable at the point where it changes
  definition. This is a particular kind of {\em critical point} for
  the function definition. Critical points could arise in the form of
  vertical tangents, vertical cusps, or just plain points of turning
  such as for $|x|$ or $x^+$ at $x = 0$. At such points, it makes
  sense to try to compute the one-sided derivatives, and these can be
  computed just by differentiating the piece functions and plugging in
  at the point. The second derivative does not exist at such
  points. Also, there is an abrupt change in the nature of concavity
  at these points.
\item An even smoother transition occurs if the first derivative is
  defined at the point. If the first derivative is also defined around
  the point, then we can start thinking about the second derivative.
\item More generally, we could think of situations where we want the
  first $k$ derivatives to be defined at or around the point.
\item To integrate a function with a piecewise definition, partition
  the interval of integration in a manner that each part lies within
  one definition piece. Please review the following two routine
  problems from Homework 6: Exercise 5.4.55 and 5.4.60. You might want
  to do a few more suggested problems of the same type.
\end{enumerate}

\subsection{The $\sin(1/x)$ examples}

\begin{enumerate}
\item The $\sin(1/x)$ and related examples are somewhat tricky because
  the function definition differs at an {\em isolated point}, namely $0$.
\item To calculate any limit or derivative at a point other than $0$,
  we can do formal computations. However, to calculate the derivative
  at $0$, we {\em must} use the definition of derivative as a limit of
  a difference quotient.
\item For all the facts below, the qualitative conclusions at finite
  places hold if we replace $\sin$ by $\cos$. Those at $\infty$ change
  qualitatively.
\item The function $f_0(x) := \lbrace \begin{array}{rl} \sin(1/x), & x
  \ne 0 \\ 0, & x = 0 \\\end{array}$ is odd and satisfies the intermediate
  value property but is not continuous at $0$. Its limit at $\pm
  \infty$ is $0$, i.e., it has horizontal asymptote the $x$-axis in
  both directions.
\item The function $f_1(x) := \lbrace \begin{array}{rl} x\sin(1/x), &
  x \ne 0 \\ 0, & x = 0 \\\end{array}$ is even and continuous but not
  differentiable at $0$. We can see this from the pinching theorem --
  it is pinched between $|x|$ and $-|x|$. $f_1$ is infinitely
  differentable at all points other than $0$. Its limit at $\pm
  \infty$ is $1$, and it approaches this from below in both
  directions.
\item The function $f_2(x) := \lbrace \begin{array}{rl} x^2\sin(1/x), &
  x \ne 0 \\ 0, & x = 0 \\\end{array}$ is differentiable at $0$, and
  infinitely differentiable everywhere other than $0$, but the
  derivative is not continuous at $0$. The limit $\lim_{x \to 0}
  f_2'(x)$ does not exist. Note that $f_2'$ is defined everywhere and
  satisfies the intermediate value property but is not continuous at $0$.

  $f_2$ is asymptotic to the line $y = x$ both additively and
  multiplicatively, as $x \to \pm \infty$.

\item The function $f_3(x) := \lbrace \begin{array}{rl} x^3\sin(1/x),
  & x \ne 0 \\ 0, & x = 0 \\\end{array}$ is continuously
  differentiable but not twice differentiable at $0$, and infinitely
  differentiable everywhere other than $0$.

  $f_3$ is asymptotic to the line $y = x^2 + C$ as $x \to \pm \infty$,
  where $C$ is an actual constant (whose value you were supposed to
  compute in a homework problem).

\item More generally, consider something such as $p(x)
  \sin(1/(q(x)))$. This function is not defined at the zeros of
  $q$. However, it does not have vertical asymptotes at these
  points. If $a$ is a root of $q$ and also of $p$, then the limiting
  value as $x \to a$ is $0$. Otherwise, the limit is undefined but the
  function oscillates between finite bounds.

  In the limit as $x \to \pm \infty$, if the degree of $p$ is less
  than that of $q$, the function has horizontal asymptote the
  $y$-axis. If their degrees are equal, it has asymptote a finite
  nonzero value, namely $\lim_{x \to \infty} p(x)/q(x)$. If the degree
  of $p$ is bigger, it is asymptotic to a polynomial.

  For $p(x)\cos(1/(q(x)))$, the behavior at points where the function
  isn't defined is the same as for $\sin$, but the behavior at $\pm
  \infty$ is different -- the $\cos$ part goes to $1$, so the function
  is asymptotically polynomial, albeit not necessarily to $p$ itself.
\item Fun exercise: Consider $x\tan(1/x)$. What can you say about this?
\end{enumerate}

\subsection{Power functions}

We here consider exponents $r = p/q$, $q$ odd. When $q$ is even, or
when $r$ is irrational, the conclusions drawn here continue to hold
for $x > 0$; however, the function isn't defined for $x < 0$.

For each of these, you should be able to provide ready
justifications/reasoning based on derivatives.

\begin{enumerate}
\item Case $r < 0$: $x^r$ is undefined at $0$. It is decreasing and
  concave up on $(0,\infty)$, with vertical asymptote at $x = 0$ going
  to $+\infty$ and horizontal asymptote as $x \to \infty$ going to $y
  = 0$. If $p$ is even, it is increasing and concave up on
  $(-\infty,0)$ with horizontal asymptote as $x \to -\infty$ going to
  $y = 0$ and vertical asymptote $+\infty$ at $0$. If $p$ is odd, it
  is decreasing and concave down on $(-\infty,0)$ with horizontal
  asymptote as $x \to -\infty$ going to $y = 0$ and vertical asymptote
  $-\infty$ at $0$.
\item Case $r = 0$: We get a constant function with value $1$.
\item Case $0 < r < 1$: $x^r$ is increasing and concave down on
  $(0,\infty)$. If $p$ is even, it is decreasing and concave down on
  $(-\infty,0)$ and has a downward-facing vertical cusp at $(0,0)$. If
  $p$ is odd, it is increasing and concave up on $(-\infty,0)$ and has
  an upward vertical tangent at $(0,0)$.
\item Case $r = 1$: A straight line $y = x$.
\item Case $1 < r$: $x^r$ is increasing and concave up on
  $(0,\infty)$. If $p$ is even, it is decreasing and concave up on
  $(-\infty,0)$ and has a local and absolute minimum and critical
  point at $(0,0)$. If $p$ is odd, it is increasing and concave down
  on $(-\infty,0)$ and has a point of inflection-type critical point
  (no local extreme value) at $(0,0)$.
\end{enumerate}

\subsection{Local behavior heuristics: multiplicative}

You have a complicated looking function such that $(x -
\alpha_1)^{r_1}(x - \alpha_2)^{r_2} \dots (x - \alpha_k)^{r_k}$. What
is the local behavior of the function near $x = \alpha_1$?

The answer: For determining the qualitative nature of this local
behavior, you can just concentrate on $(x - \alpha_i)^{r_i}$ and
ignore the rest. In particular, {\em just} looking at $r_1$, you can
figure out whether you have a critical point, local extremum, point of
inflection, vertical tangent, or vertical cusp. The other things {\em
do} matter if you are further interested in, say, whether we have a
local maximum or minimum, or in whether the vertical tangent is an
increasing tangent or a decreasing tangent, or which direction the
vertical cusp points in.

Overall, if $r_i > 0$ and $r_i = p_i/q_i$, $q_i$ odd, and both $p_i$,
$q_i$ positive, then:

\begin{itemize}
\item Critical point iff $r_i \ne 1$.
\item Local extremum iff $p_i$ even.
\item Point of inflection/vertical tangent iff $p_i$ odd.
\item Vertical cusp iff $p_i$ even and $p_i < q_i$, both positive,
  i.e., $r_i < 1$. Note that vertical cusp is a special kind of local extremum.
\item Vertical tangent iff $p_i$ odd, $p_i < q_i$, both positive,
i.e., $r_i < 1$.
\end{itemize}

So, if we look at, say $(x - \pi)^2(x - \sqrt{6})^3(x - 2)^{1/3}(x -
3)^{2/3}$, it has a local extremum at $\pi$, a point of inflection at
$\sqrt{6}$, a vertical tangent at $2$, and a vertical cusp at $3$.

\subsection{Local behavior heuristics: additive}

If you have something of the form $f + g$, and the vertical
tangent/cusp points for $f$ are disjoint from those of $g$, then the
vertical tangent/cusp points for $f + g$ include both lists. Further,
the nature (tangent versus cusp) is inherited from the corresponding
piece.

For instance, for $x^{1/3} + (x - 131.4)^{2/3}$, there is a vertical
tangent at $x = 0$ and a vertical cusp at $x = 131.4$.

In particular, if $g$ is everywhere differentiable, then the vertical
tangent/cusp behavior of $f + g$ is the same as that of $f$.

\section{High yield practice}

Here are the areas that you should focus on if you have a thorough grasp of the basics:

\begin{enumerate}
\item Everything to do with piecewise definitions (differentiation,
  integration, reasoning).
\item Vertical tangents and cusps in sophisticated cases.
\item Horizontal, oblique, and weird asymptotes.
\item Trigonometric integrations, particularly $\sin^2$, $\cos^2$, and
  $\tan^2$ and their variants.
\item Tricky integration problems that involve the use of symmetry
  and/or the chain rule.
\end{enumerate}

\section{Quickly}

This ``Quickly'' list is a bit of a repeat and augmentation of the
``Quicky'' list given out for the previous midterm.

\subsection{Arithmetic}

You should be able to:

\begin{itemize}
\item Do quick arithmetic involving fractions.
\item Remember $\sqrt{2}$, $\sqrt{3}$, and $\pi$ to at least two
  digits.
\item Sense when an expression will simplify to $0$.
\item Compute approximate values for square roots of small numbers,
  $\pi$ and its multiples, etc., so that you are able to figure out,
  for instance, whether $\pi/4$ is smaller or bigger than $1$, or two
  integers such that $\sqrt{39}$ is between them.
\item Know or quickly compute small powers of small positive
  integers. This is particularly important for computing definite
  integrals. For instance, to compute $\int_2^3 (x + 1)^3 \, dx$, you
  need to know/compute $3^4$ and $4^4$.
\end{itemize}

\subsection{Computational algebra}

You should be able to:

\begin{enumerate}
\item Add, subtract, and multiply polynomials.
\item Factorize quadratics or determine that the quadratic cannot be
  factorized.
\item Factorize a cubic if at least one of its factors is a small and
  easy-to-spot number such as $0$, $\pm 1$, $\pm 2$, $\pm 3$.
\item Do polynomial long division (not usually necessary, but helpful).
\item Solve simple inequalities involving polynomial and rational
  functions once you've obtained them in factored form.
\end{enumerate}

\subsection{Computational trigonometry}

You should be able to:

\begin{enumerate}
\item Determine the values of $\sin$, $\cos$, and $\tan$ at multiples
  of $\pi/2$.
\item Determine the intervals where $\sin$ and $\cos$ are positive and
  negative.
\item Remember the formulas for $\sin(\pi - x)$ and $\cos(\pi - x)$,
as well as formulas for $\sin(-x)$ and $\cos(-x)$.
\item Recall the values of $\sin$ and $\cos$ at $\pi/6$, $\pi/4$, and
  $\pi/3$, as well as at the corresponding obtuse angles.
\item Reverse lookup for these, for instance, you should quickly
  identify the acute angle whose $\sin$ is $1/2$.
\end{enumerate}

\subsection{Computational limits}

You should be able to: size up a limit, determine whether it is of the
form that can be directly evaluated, of the form that we already know
does not exist, or indeterminate.

\subsection{Computational differentiation}

You should be able to:

\begin{enumerate}
\item Differentiate a polynomial (written in expanded form) on sight
  (without rough work).
\item Differentiate a polynomial (written in expanded form) twice
  (without rough work).
\item Differentiate sums of powers of $x$ on sight (without rough
  work).
\item Differentiate rational functions with a little thought.
\item Do multiple differentiations of expressions whose derivative
  cycle is periodic, e.g., $a \sin x + b \cos x$.
\item Differentiate simple composites without rough work (e.g.,
  $\sin(x^3)$).
\end{enumerate}

\subsection{Computational integration}

You should be able to:

\begin{enumerate}
\item Compute the indefinite integral of a polynomial (written in
  expanded form) on sight without rough work.
\item Compute the definite integral of a polynomial with very few
  terms within manageable limits quickly.
\item Compute the indefinite integral of a sum of power functions
  quickly.
\item Know that the integral of sine or cosine on any quadrant is $\pm
  1$.
\item Compute the integral of $x \mapsto f(mx)$ if you know how to
  integrate $f$. In particular, integrate things like $(a + bx)^m$.
\item Integrate $\sin$, $\cos$, $\sin^2$, $\cos^2$, $\tan^2$,
  $\sec^2$, $\cot^2$, $\csc^2$,.
\end{enumerate}
\subsection{Being observant}

You should be able to look at a function and:

\begin{enumerate}
\item Sense if it is odd (even if nobody pointedly asks you whether it
  is).
\item Sense if it is even (even if nobody asks you whether it is).
\item Sense if it is periodic and find the period (even if nobody asks
  you about the period).
\end{enumerate}

\subsection{Graphing}

You should be able to:

\begin{enumerate}
\item Mentally graph a linear function.
\item Mentally graph a power function $x^r$ (see the list of things to
  remember about power functions). Sample cases for $r$: $1/3$, $2/3$,
  $4/3$, $5/3$, $1/2$, $1$, $2$, $3$, $-1$, $-1/3$ $-2/3$.
\item Graph a piecewise linear function with some thought.
\item Mentally graph a quadratic function (very approximately) --
  figure out conditions under which it crosses the axis etc.
\item Graph a cubic function after ascertaining which of the cases for
  the cubic it falls under.
\item Mentally graph $\sin$ and $\cos$, as well as functions of the $A
  \sin(mx)$ and $A\cos(mx)$.
\item Graph a function of the form linear + trigonometric, after doing
  some quick checking on the derivative.
\end{enumerate}

\subsection{Fancy pictures}

Keep in mind approximate features of the graphs of:

\begin{enumerate}
\item $\sin(1/x)$, $x\sin(1/x)$, $x^2 \sin(1/x)$ and $x^3\sin(1/x)$,
  and the corresponding $\cos$ counterparts -- both the behavior near
  $0$ and the behavior near $\pm \infty$.
\item The Dirichlet function and its variants -- functions defined
  differently for the rationals and irrationals.
\end{enumerate}

\end{document}
