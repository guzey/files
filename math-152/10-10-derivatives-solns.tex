\documentclass[10pt]{amsart}

%Packages in use
\usepackage{fullpage, hyperref, vipul, enumerate}

%Title details
\title{Class quiz solutions: October 10: Derivatives}
\author{Math 152, Section 55 (Vipul Naik)}
%List of new commands

\begin{document}
\maketitle

\section{Performance review}

$12$ people took the quiz. The score distribution was as follows:

\begin{itemize}
\item Score of $2$: $2$ people
\item Score of $3$: $4$ people
\item Score of $4$: $6$ people
\end{itemize}

The mean score was $3.33$. Here are the problem-wise scores:

\begin{enumerate}
\item Option (E): $11$ people
\item Option (D): $6$ people
\item Option (C): $11$ people
\item Option (B): $12$ people
\end{enumerate}

More details below.
\section{Solutions}

\begin{enumerate}

\item Consider the expression $x^2 + t^2 + xt$. What is the derivative
  of this with respect to $x$ (with $t$ assumed to be a constant)?


  \begin{enumerate}[(A)]

  \item $2x + 2t + x + t$
  \item $2x + 2t + 1$
  \item $2x + 2t$
  \item $2x + t + 1$
  \item $2x + t$
  \end{enumerate}

  {\em Answer}: Option (E)

  {\em Explanation}: When we differentiate with respect to $x$,
  keeping $t$ constant, the $x^2$ differentiates to $2x$, the $t^2$
  differentiates to $0$ (because it is constant) and the $xt$
  differentiates to $t$.

  Note that there's something else called {\em implicit
  differentiation} where we do not assume $t$ to be a constant but
  rather an {\em unknown function of $x$}. In that case, we
  differentiate the functions of $t$ with respect to $t$ and tag along
  a $dt/dx$. With that interpretation, the derivative would be $2x +
  2t(dt/dx) + x(dt/dx) + t$. However, we're assuming $t$ constant, so
  $dt/dx = 0$, and so the $dt/dx$ terms vanish and we are just left
  with $2t + x$.

  {\em The other choices}:

  Option (A) is what you get if you just differentiate each thing with
  respect to its own variable naively: $x^2$ gives $2x$, $t^2$ gives
  $2t$, $xt$, by the product rule, gives $x + t$. This is {\em
  completely wrong} because we are differentiating only with respect
  to $x$. (See the note on implicit differentiation right above this).

  Options (B), (C) and (D) are also possible results of incorrect
  differentiation.

  {\em Performance review}: $11$ out of $12$ people got this
  correct. $1$ person chose (B).

  {\em Historical note (last year)}: $11$ out of $12$ people got this
  correct. $1$ person chose (B).

  History repeats itself!

\item Which of the following verbal statements is {\bf not valid as a
  general rule}?

  \begin{enumerate}[(A)]
  \item The derivative of the sum of two functions is the sum of the
    derivatives of the functions.
  \item The derivative of the difference of two functions is the
    difference of the derivatives of the functions.
  \item The derivative of a constant times a function is the same
    constant times the derivative of the function.
  \item The derivative of the product of two functions is the product
    of the derivatives of the functions.
  \item None of the above, i.e., they are all valid as general rules.
  \end{enumerate}

  {\em Answer}: Option (D)

  {\em Explanation}: The correct replacement of option (D) is the
  product rule for derivatives, which, in words, states that: ``the
  derivative of the product of two functions is the sum of the product
  of the derivative of the first function with the second function and
  the product of the first function with the derivative of the second
  function.'' If that seems cumbersome to you, feel grateful for the
  power of algebraic symbols to capture this compactly:

  $$(f \cdot g)' = (f' \cdot g) + (f \cdot g')$$

  {\em Performance review}: $6$ out of $12$ people got this
  correct. $4$ chose (E) and $2$ chose (C).

  {\em Historical note (last year)}: $10$ ouf to $12$ people got this
  correct. $1$ person chose (C) and $1$ person chose (E).

  {\em Note on better performance last year}: I think I got time to
  review the product rule for differentiation {\em before} the quiz
  last year, which was probably helpful.

\item Which of the following statements is {\bf definitely true} about
  the tangent line to the graph of an everywhere differentiable
  function $f$ on $\R$ at the point $(a,f(a))$ (Here, ``everywhere
  differentiable'' means that the derivative of $f$ is defined and
  finite for all $x \in \R$)?

  \begin{enumerate}[(A)]

  \item The tangent line intersects the curve at precisely one point,
    namely $(a,f(a))$.
  \item The tangent line intersects the $x$-axis.
  \item The tangent line intersects the $f(x)$-axis (the $y$-axis).
  \item Any line through $(a,f(a))$ other than the tangent line
    intersects the graph of $f$ at at least one other point.
  \item None of the above need be true.
  \end{enumerate}

  {\em Answer}: Option (C)

  {\em Explanation}: If the function is differentiable, then the
  tangent line has finite slope, and hence cannot be vertical. Thus,
  it is not parallel to the $y$-axis, and hence must intersect the
  $y$-axis.

  {\em The other choices}:

  Option (A) is not true, as discussed in class. For instance, for the
  $\sin$ function, the tangent line through any of the peak points is
  $y = 1$, and passes through all the peak points, hence it intersects
  the graph infinitely often. We can graphically construct a lot of
  examples where the tangent line at one point in the graph intersects
  the graph elsewhere. There are certain classes of functions for
  which the statement of option (A) {\em is} true, and we'll talk more
  about this when we discuss concave up and concave down.

  Option (B) is not true. The tangent line to $(\pi/2,1)$ for the
  $\sin$ function is $y = 1$ -- a horizontal line. Thus, it does not
  intersect the $x$-axis. In general, the tangent line does not
  intersect the $x$-axis iff it is horizontal, which happens iff the
  derivative at the point is zero.

  {\em Performance review}: $11$ out of $12$ people got this
  correct. $1$ chose (E).

  {\em Historical note (last year)}: $6$ out of $12$ people got this
  correct. $5$ people chose option (E) (in other words, they were not
  convinced by any of the options (A)-(D)) and $1$ person chose option
  (D).

  {\em Note on better performance this year}: The question was starred
  and hence people were able to convince each other of the correct
  ideas through discussion.
\item (*) For a function $f: (0,\infty) \to (0,\infty)$, denote by
  $f^{(k)}$ the $k^{th}$ derivative of $f$. Suppose $f(x) := x^r$ with
  domain $(0,\infty)$, and $r$ a rational number. What is the {\bf
  precise set of values} of $r$ satisfying the following: there exist
  a positive integer $k$ (dependent on $r$) for which $f^{(k)}$ is
  identically the zero function.

  \begin{enumerate}[(A)]
  \item $r$ should be an integer.
  \item $r$ should be a nonnegative integer.
  \item $r$ should be a positive integer.
  \item $r$ should be a nonnegative rational number.
  \item $r$ should be a positive rational number.
  \end{enumerate}

  {\em Answer}: Option (B)

  {\em Explanation}: If $r$ is $0$, then the derivative of the
  function is zero. For $r$ a positive integer, the $(r + 1)^{th}$
  derivative is $0$. See also Routine Problem 14 of Homework 3
  (Exercise 3.3.64 of the book, Page 129).

  For any other value of $r$, the power of $x$ keeps going down by $1$
  each time we differentiate. However, since we didn't start with a
  nonnegative integer, the power of $x$ never becomes $0$, so we keep
  going down and never stop. For instance, if $r = 5/3$, we have:

  $$f(x) = x^{5/3}, f^{(1)}(x) = (5/3)x^{2/3}, f^{(2)}(x) = (10/9)x^{-1/3}, \dots$$
  
  Note that the powers of $x$ in $f$ and its derivatives are $5/3$,
  $2/3$, $-1/3$, $-4/3$ and so on, going down by $1$ each time. Note
  that going down from $2/3$ to $-1/3$, the power skips right past
  $0$.
 
  {\em Future teaser}: This idea will come back at us with a vengeance
  when we study a technique called integration by parts in Math
  153. For those who're brave enough, here's the key idea: one
  application of integration by parts is to integrate functions of the
  form (polynomial) times (function easy to repeatedly integrate). The
  main reason why this technique works for (polynomial)s is that if we
  differentiate a polynomial often enough, it becomes zero. If we try
  to apply the same idea replacing a polynomial by a power function
  such as $x^{5/3}$ (e.g., try $\int x^{5/3}\sin x \, dx$) then this
  does not work precisely because when repeatedly differentiating
  $x^{5/3}$, we skip right past the zero exponent.

  {\em Performance review}: Everybody got this correct!

  {\em Historical note (last year)}: $4$ out of $12$ people got this
  correct. $2$ people each chose (A), (C), (D) and (E).

  {\em Note on better performance this year}: The question was starred
  allowing discussion, so people had the opportunity to convince each
  other of the right ideas. Nonetheless, everybody getting this
  correct is reasonably impressive, particularly considering that we
  haven't talked about these ideas in class so far.
\end{enumerate}

\end{document}