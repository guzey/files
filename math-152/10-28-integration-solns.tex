\documentclass[10pt]{amsart}

%Packages in use
\usepackage{fullpage, hyperref, vipul, enumerate}

%Title details
\title{Class quiz solutions: October 28: Integration basics}
\author{Math 152, Section 55 (Vipul Naik)}
%List of new commands

\begin{document}
\maketitle

\section{Performance review}

This quiz actually happened on November 4.

$12$ people took this quiz. The score distribution was as follows:

\begin{itemize}
\item Score of $0$: $3$ people
\item Score of $1$: $2$ people
\item Score of $2$: $4$ people
\item Score of $3$: $2$ people
\item Score of $4$: $1$ person
\end{itemize}

The mean score was $1.67$.

Here are the problem-wise answers and performance:

\begin{enumerate}
\item Option (C): $4$ people
\item Option (D): $7$ people
\item Option (D): $4$ people
\item Option (A): $5$ people
\end{enumerate}

\section{Solutions}

\begin{enumerate}

\item Consider the function(s) $[0,1] \to \R$. {\bf Identify the
  function}s for which the integral (using upper sums and lower sums)
  is not defined.

  \begin{enumerate}[(A)]
  \item $f_1(x) := \lbrace\begin{array}{rl} 0, & 0 \le x < 1/2 \\ 1, &
    1/2 \le x \le 1\end{array}$.
  \item $f_2(x) := \lbrace\begin{array}{rl} 0, & x \ne 0 \text{ and } 1/x \text{ is an
    integer } \\ 1, & \text{otherwise} \end{array}$.
  \item $f_3(x) := \lbrace\begin{array}{rl} 0, & x \text{ rational }\\
    1, & x \text{ irrational}\end{array}$
  \item All of the above
  \item None of the above
  \end{enumerate}

  {\em Answer}: Option (C)

  {\em Explanation}: For option (C), the lower sum for any partition is $0$
  and the upper sum is $1$. Thus, the integral is not
  well-defined. 

  For option (A), the function is piecewise continuous with
  only jump discontinuities, hence the integral is well-defined: in
  fact, it is $1/2$. 

  For (B), the integral is zero. We can see this by
  noting that the points where the function is $0$ are all isolated
  points, so if in our partition the intervals surrounding each of
  these points is small enough, we can make the upper sums tend to
  zero. (This is hard to see. You should, however, be able to easily
  see that (A) has an integral and (C) does not. This forces the
  answer to be (C)).

  {\em Performance review}: $4$ out of $12$ got this correct. $3$ each
  chose (B) and (D), $2$ chose (E).

  {\em Historical note (last year)}: Everybody got this correct.

\item (**) Suppose $a < b$. Recall that a {\em regular partition}
  into $n$ parts of $[a,b]$ is a partition $a = x_0 < x_1 < \dots <
  x_{n-1} < x_n = b$ where $x_i - x_{i-1} = (b - a)/n$ for all $1 \le
  i \le n$. A partition $P_1$ is said to be a {\em finer partition}
  than a partition $P_2$ if the set of points of $P_1$ contains the
  set of points of $P_2$. Which of the following is a {\bf necessary
  and sufficient condition} for the regular partition into $m$ parts
  to be a {\em finer partition} than the regular partition into $n$
  parts? (Note: We'll assume that any partition is finer than itself).

  \begin{enumerate}[(A)]
  \item $m \le n$
  \item $n \le m$
  \item $m$ divides $n$ (i.e., $n$ is a multiple of $m$)
  \item $n$ divides $m$ (i.e., $m$ is a multiple of $n$)
  \item $m$ is a power of $n$
  \end{enumerate}

  {\em Answer}: Option (D)

  {\em Explanation}: If $n$ divides $m$, then the partition into $m$
  pieces is obtained by further subdividing the partition into $m$
  parts, with each part divided into $n/m$ parts.

  {\em The other choices}: Option (B) is a {\em necessary} condition
  but is not a sufficient condition. For instance, the regular
  partition of $[0,1]$ into two parts corresponds to $\{ 0, 1/2, 1 \}$
  and the partition into three parts corresponds to $\{ 0, 1/3, 2/3, 1
  \}$. These partitions are incomparable, i.e., neither is finer than
  the other.

  {\em Performance review}: $7$ out of $12$ got this correct. $4$
  chose(C), $1$ chose (B).

  {\em Historical note (last year)}: $5$ out of $15$ people got this
  correct. $6$ people chose (B), which is necessary but not
  sufficient, as indicated above. $2$ people chose (C), which is the
  reverse option. $1$ person chose (A) and $1$ chose (A)+(D).

  {\em Action point}: If you chose option (B), please make sure you
  understand the distinction between the options. Also review the
  concept of regular partitions and finer partition till you find this
  question obvious.

\item (**) For a partition $P = x_0 < x_1 < x_2 < \dots < x_n$ of
  $[a,b]$ (with $x_0 = a$, $x_n = b$) define the norm $\| P \|$ as the
  maximum of the values $x_i - x_{i-1}$. Which of the following {\bf
  is always true} for any continuous function $f$ on $[a,b]$?

  \begin{enumerate}[(A)]
  \item If $P_1$ is a finer partition than $P_2$, then $\| P_2 \| \le
    \| P_1 \|$ (Here, {\em finer} means that, as a set, $P_2 \subseteq
    P_1$, i.e., all the points of $P_2$ are also points of $P_1$).
  \item If $\| P_2 \| \le \| P_1 \|$, then $L_f(P_2) \le L_f(P_1)$
    (where $L_f$ is the lower sum).
  \item If $\| P_2 \| \le \| P_1 \|$, then $U_f(P_2) \le U_f(P_1)$
    (where $U_f$ is the upper sum).
  \item If $\| P_2 \| \le \| P_1 \|$, then $L_f(P_2) \le U_f(P_1)$.
  \item All of the above.
  \end{enumerate}

  {\em Answer}: Option (D).

  {\em Explanation}: Option (D) is true for the rather trivial reason
  that any lower sum of $f$ over any partition cannot be more than any
  upper sum of $f$ over any partition. The norm plays no role.

  Option (A) is incorrect because the inequality actually goes the
  other way: the finer partition has the smaller norm. Options (B) and
  (C) are incorrect because a smaller norm does not, in and of itself,
  guarantee anything about how the lower and upper sums compare.

  {\em Performance review}: $4$ out of $12$ got this correct. $4$
  chose (A), $2$ chose (B), $1$ chose (C), $1$ chose (E).

  {\em Historical note (last year)}: $4$ out of $15$ people got this
  correct. $8$ people chose (C), presumably with the intuition that
  the smaller the norm of a partition, the smaller its upper
  sums. While this intuition is right in a broad sense, it is not
  correct in the precise sense that would make (C) correct. It is
  possible that a lot of people did not read (D) carefully, and
  stopped after seeing (C), which they thought was a correct
  statement. $1$ person each chose (A), (E), and (C)+(D).

  {\em Historical note (two years ago)}: This question appeared in a
  152 midterm two years ago, and $6$ of $29$ people got this
  right. Many people chose (C) in that test too (though I haven't
  preserved numerical information on number of wrong choices
  selected). {\em History repeats itself, despite my best efforts!} On
  the plus side, though, this is just a quiz after your first college
  exposure to the material, as opposed to what was a midterm last
  year, which happened after homeworks and a midterm review session.

  {\em Action point}: Please make sure you understand very clearly the
  relation between the ``finer'' partition notion, norms of
  partitions, upper sums, and lower sums, to the point where you
  wondered how you could have ever got this question wrong.
\item (**) Suppose $F$ and $G$ are continuously differentiable
  functions on all of $\R$ (i.e., both $F'$ and $G'$ are
  continuous). Which of the following is {\bf not necessarily true}?

  \begin{enumerate}[(A)]
  \item If $F'(x) = G'(x)$ for all integers $x$, then $F - G$ is a
    constant function when restricted to integers, i.e., it takes the
    same value at all integers.
  \item If $F'(x) = G'(x)$ for all numbers $x$ that are not integers,
    then $F - G$ is a constant function when restricted to the set of
    numbers $x$ that are not integers.
  \item If $F'(x) = G'(x)$ for all rational numbers $x$, then $F - G$
    is a constant function when restricted to the set of rational
    numbers.
  \item If $F'(x) = G'(x)$ for all irrational numbers $x$, then $F -
    G$ is a constant function when restricted to the set of irrational
    numbers.
  \item None of the above, i.e., they are all necessarily true.
  \end{enumerate}

  {\em Answer}: Option (A).

  {\em Explanation}: The fact that the derivatives of two functions
  agree at integers says nothing about how the derivatives behave
  elsewhere -- they could differ quite a bit at other places. Hence,
  (A) is not necessarily true, and hence must be the right option. All
  the other options are correct as statements and hence cannot be the
  right option. This is because in all of them, the set of points
  where the derivatives agree is {\em dense} -- it intersects every
  open interval. So, continuity forces the functions $F'$ and $G'$ to
  be equal everywhere, forcing $F - G$ to be constant everywhere.

  {\em Performance review}: $5$ out of $12$ got this correct. $3$
  chose (D), $2$ chose (C), $2$ chose (E).

  {\em Historical note (last year)}: Nobody got this correct. $14$ people
  chose (E) and $1$ person chose (B).

  {\em Action point}: This was a devilish question, because answering
  it correctly requires a knowledge of (or at least intuition about)
  ideas that you have not yet encountered. Nonetheless, it is a
  question whose solution you should be able to understand after
  having read it. Please make sure you understand why (A) is correct,
  i.e., how the set of integers is different from the sets in options
  (B), (C), and (D).
\end{enumerate}

\end{document}