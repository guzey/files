\documentclass[10pt]{amsart}

%Packages in use
\usepackage{fullpage, hyperref, vipul, enumerate}

%Title details
\title{Class quiz: October 7: Limit theorems}
\author{Math 152, Section 55 (Vipul Naik)}
%List of new commands

\begin{document}
\maketitle

Your name (print clearly in capital letters): $\underline{\qquad\qquad\qquad\qquad\qquad\qquad\qquad\qquad\qquad\qquad}$

Questions marked with a (*) are questions that are somewhat trickier,
with the probability of getting the question correct being about
$50\%$ or less. For these questions, you are free to discuss the questions
with others while making your attempt.

Questions marked with a (**) are questions where, in a previous
administration of this quiz, a specific incorrect option was chosen by
as many people as or more people than the correct option. For these
questions, you are free to discuss the questions with others while
making your attempt.

\begin{enumerate}

\item (**) Which of the following statements is {\bf always true}? {\em
  Last year's performance: $2/11$ correct}

  \begin{enumerate}[(A)]

  \item The range of a continuous nonconstant function on a closed
    bounded interval (i.e., an interval of the form $[a,b]$) is a
    closed bounded interval (i.e., an interval of the form $[m,M]$).
  \item The range of a continuous nonconstant function on an open
    bounded interval (i.e., an interval of the form $(a,b)$) is an
    open bounded interval (i.e., an interval of the form $(m,M)$).
  \item The range of a continuous nonconstant function on a closed
    interval that may be bounded or unbounded (i.e., an interval of
    the form $[a,b]$, $[a,\infty)$, $(-\infty,a]$, or
    $(-\infty,\infty)$) is also a closed interval that may be bounded
    or unbounded.
  \item The range of a continuous nonconstant function on an open
    interval that may be bounded or unbounded (i.e., an interval of
    the form $(a,b)$,$(a,\infty)$, $(-\infty,a)$, or
    $(-\infty,\infty)$), is also an open interval that may be bounded
    or unbounded.
  \item None of the above.
  \end{enumerate}

  \vspace{0.1in}
  Your answer: $\underline{\qquad\qquad\qquad\qquad\qquad\qquad\qquad}$
  \vspace{0.5in}

\item (**) Suppose $g:\R \to \R$ is a continuous function such that
  $\lim_{x \to 0} g(x)/x = A$ for some constant $A \ne 0$. What is
  $\lim_{x \to 0} g(g(x))/x$? {\em Last year's performance: $1/12$ correct}

  \begin{enumerate}[(A)]
  \item $0$
  \item $A$
  \item $A^2$
  \item $g(A)$
  \item $g(A)/A$
  \end{enumerate}

  \vspace{0.1in}
  Your answer: $\underline{\qquad\qquad\qquad\qquad\qquad\qquad\qquad}$
  \vspace{0.5in}

\item Suppose $I = (a,b)$ is an open interval. A function $f:I \to \R$
  is termed {\em piecewise continuous} if there eixst points $a_0 <
  a_1 < a_2 < \dots < a_n$ (dependent on $f$) with $a = a_0$ and $a_n
  = b$, such that $f$ is continuous on each interval $(a_i,
  a_{i+1})$. In other words, $f$ is continuous at every point in
  $(a,b)$ except possibly the $a_i$s.

  Suppose $f$ and $g$ are piecewise continuous functions on the same
  interval $I$ (with possibly different sets of $a_i$s). Which of the
  following is/are guaranteed to be piecewise continuous on $I$? {\em
  Last year's performance: $9/11$ correct}

  \begin{enumerate}[(A)]
  \item $f + g$, i.e., the function $x \mapsto f(x) + g(x)$
  \item $f - g$, i.e., the function $x \mapsto f(x) - g(x)$
  \item $f \cdot g$, i.e., the function $x \mapsto f(x)g(x)$
  \item All of the above
  \item None of the above
  \end{enumerate}

  \vspace{0.1in}
  Your answer: $\underline{\qquad\qquad\qquad\qquad\qquad\qquad\qquad}$
  \vspace{0.5in}

\item Suppose $f$ and $g$ are everywhere defined and $\lim_{x \to 0}
  f(x) = 0$. Which of these pieces of information is {\bf not
  sufficient} to conclude that $\lim_{x \to 0} f(x)g(x) = 0$? {\em
  Last year's performance: $8/11$ correct}

  \begin{enumerate}[(A)]
  \item $\lim_{x \to 0} g(x) = 0$.
  \item $\lim_{x \to 0} g(x)$ is a constant not equal to zero.
  \item There exists $\delta > 0$ and $B > 0$ such that for $0 < |x| <
    \delta$, $|g(x)| < B$.
  \item $\lim_{x \to 0} g(x) = \infty$, i.e., for every $N > 0$ there
    exists $\delta > 0$ such that if $0 < |x| < \delta$, then $g(x) >
    N$.
  \item None of the above, i.e., they are all sufficient to conclude
    that $\lim_{x \to 0} f(x)g(x) = 0$.
  \end{enumerate}

  \vspace{0.1in}
  Your answer: $\underline{\qquad\qquad\qquad\qquad\qquad\qquad\qquad}$
  \vspace{0.5in}

\item $f$ and $g$ are functions defined for all real values. $c$ is a
  real number. Which of these statements is {\bf {\em not} necessarily
  true}? {\em Last year's performance: $9/11$ correct}

  \begin{enumerate}[(A)]
  \item If $\lim_{x \to c^-} f(x) = L$ and $\lim_{x \to c^-} g(x) =
    M$, then $\lim_{x \to c^-} (f(x) + g(x))$ exists and is equal to
    $L + M$.
  \item If $\lim_{x \to c^-} g(x) = L$ and $\lim_{x \to L^-} f(x) =
    M$, then $\lim_{x \to c^-} f(g(x)) = M$.
  \item If there exists an open interval containing $c$ on which $f$
    is continuous and there exists an open interval containing $c$
    on which $g$ is continuous, then there exists an open interval
    containing $c$ on which $f + g$ is continuous.
  \item If there exists an open interval containing $c$ on which $f$
    is continuous and there exists an open interval containing $c$ on
    which $g$ is continuous, then there exists an open interval
    containing $c$ on which the product $f \cdot g$ (i.e., the
    function $x \mapsto f(x)g(x)$) is continuous.
  \item None of the above, i.e., they are all necessarily true.
  \end{enumerate}

  \vspace{0.1in}
  Your answer: $\underline{\qquad\qquad\qquad\qquad\qquad\qquad\qquad}$
  \vspace{0.5in}


\end{enumerate}

\end{document}