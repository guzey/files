\documentclass[10pt]{amsart}

%Packages in use
\usepackage{fullpage, hyperref, vipul, enumerate}

%Title details
\title{Class quiz solutions: September 28; Topic: Functions}
\author{Vipul Naik}
%List of new commands

\begin{document}
\maketitle

\section{Performance review}

$12$ students submitted solutions. Here is the score distribution:

\begin{enumerate}
\item Score of $0$: Nobody.
\item Score of $1$: Nobody.
\item Score of $2$: $2$ persons.
\item Score of $3$: $2$ persons.
\item Score of $4$: $5$ persons.
\item Score of $5$: $3$ persons.
\end{enumerate}

The mean score was $3.75$ and the median was $4$.

If you scored poorly on this quiz, there is no need to get
discouraged, since these question types are probably new for you even
if you're familiar with the material. Please take the time to go
through the solutions so that you are able to get similar questions
correct if you see them in the future.

{\em Note}: You are allowed and encouraged to discuss/collaborate with
others on some questions (the star-marked ones) but you should
ultimately put the solution that {\em best fits your conscience} and
not simply go with the crowd against what you believe to be the
correct answer.

Here is the summary of problem wise scores:

\begin{enumerate}
\item Option (E): $9/12$ correct.
\item (*) Option (C): $11/12$ correct. {\em Magic of collaboration?}
\item Option (D): $12/12$ correct.
\item (*) Option (A): $8/12$ correct.
\item (*) Option (E): $5/12$ correct. {\em Needs review!}
\end{enumerate}

More details in the next section.

\section{Solutions}
\begin{enumerate}

\item Suppose $f$ and $g$ are functions from $\R$ to $\R$. Suppose
  both $f$ and $g$ are even, i.e., $f(x) = f(-x)$ for all $x \in \R$
  and $g(x) = g(-x)$ for all $x \in \R$. Which of the following is
  {\em not} guaranteed to be an even function from the given information?

  {\em Note: For this question, it is possible to solve the question
  by taking a few simple examples. You're free to do this, but the
  recommended method for tackling the question is to handle it {\bf
  abstractly}, i.e., try to prove or disprove in general for each
  function whether it is even.}

  \begin{enumerate}[(A)]
  \item $f + g$, i.e., the function $x \mapsto f(x) + g(x)$
  \item $f - g$, i.e., the function $x \mapsto f(x) - g(x)$
  \item $f \cdot g$, i.e., the function $x \mapsto f(x)g(x)$
  \item $f \circ g$, i.e., the function $x \mapsto f(g(x))$
  \item None of the above, i.e., they are all guaranteed to be even
    functions.
  \end{enumerate}

  {\em Answer}: Option (E)

  {\em Explanation}: We show that $f + g$, $f - g$, $f \cdot g$, and
  $f \circ g$ are all even. Below are given short versions of
  ``proofs'' of these facts.

  $f + g$ is even because 
  
  $$(f + g)(-x) = f(-x) + g(-x) = f(x) + g(x) = (f + g)(x)$$

  Hence $(f + g)(-x) = (f + g)(x)$ for all $x$.

  $f - g$ is even because

  $$(f - g)(-x) = f(-x) - g(-x) = f(x) - g(x) = (f - g)(x)$$

  $f \cdot g$ is even because

  $$(f \cdot g)(-x) = f(-x)g(-x) = f(x)g(x) = (f \cdot g)(x)$$

  $f \circ g$ is even because:

  $$(f \circ g)(-x) = f(g(-x)) = f(g(x)) = (f \circ g)(x)$$

  Note that $f \circ g$ is somewhat special, because for this case, we
  only use that $g$ is even -- we do not use or require $f$ to be even.

  {\em Performance review}: $9$ out of $12$ people got this
  correct. $3$ chose (D).

  {\em Historical note (last year)}: $11$ out of $15$ people got this
  correct. $3$ people chose (D) and $1$ person chose (B).

\item (*) Suppose $f$ and $g$ are functions from $\R$ to $\R$. Suppose
  both $f$ and $g$ are odd, i.e., $f(-x) = -f(x)$ for all $x \in \R$
  and $g(-x) = -g(x)$ for all $x \in \R$. Which of the following is
  {\em not} guaranteed to be an odd function from the given information?

  {\em Note: For this question, it is possible to solve the question
  by taking a few simple examples. You're free to do this, but the
  recommended method for tackling the question is to handle it {\bf
  abstractly}, i.e., try to prove or disprove in general for each
  function whether it is odd.}

  \begin{enumerate}[(A)]
  \item $f + g$, i.e., the function $x \mapsto f(x) + g(x)$
  \item $f - g$, i.e., the function $x \mapsto f(x) - g(x)$
  \item $f \cdot g$, i.e., the function $x \mapsto f(x)g(x)$
  \item $f \circ g$, i.e., the function $x \mapsto f(g(x))$
  \item None of the above, i.e., they are all guaranteed to be odd functions.
  \end{enumerate}

  {\em Answer}: Option (C)

  {\em Explanation}: An example is when $f(x) := x$ and $g(x) :=
  x$. Both $f$ and $g$ are odd functions. But the product $f \cdot g$
  the function $x \mapsto x^2$, which is an even function.
  
  {\em The other choices}:

  Option (A): If $f$ and $g$ are both odd, then $f + g$ has to be
  odd. Here's why:

  $$(f + g)(-x) = f(-x) + g(-x) = -f(x) + (-g(x)) = -[f(x) + g(x)] = -(f + g)(x)$$

  Option (B): If $f$ and $g$ are both odd, then $f - g$ has to be
  odd. Here's why:

  $$(f - g)(-x) = f(-x) - g(-x) = -f(x) - (-g(x)) = -[f(x) - g(x)] = -(f - g)(x)$$

  Option (D): If $f$ and $g$ are both odd, then $f \circ g$ has to be
  odd. Here's why:

  $$(f \circ g)(-x) = f(g(-x)) = f(-g(x)) = -f(g(x)) = -(f \circ g)(x)$$

  Option (E) is ruled out because option (C) works.

  {\em Performance review}: $11$ out of $12$ got this correct. $1$
  chose (D). {\em Performance this year was much better than last
  year, possibly due to the benefits of collaboration?}

  {\em Historical note (last year)}: $7$ out of $15$ people got this
  correct. $5$ people chose (E) and $3$ people chose (D).

\item For which of the following pairs of polynomial functions $f$ and
  $g$ is it true that $f \circ g \ne g \circ f$?

  \begin{enumerate}[(A)]
  \item $f(x) := x^2$ and $g(x) := x^3$
  \item $f(x) := x + 1$ and $g(x) := x + 2$
  \item $f(x) := x^2 + 1$ and $g(x) := x^2 + 1$
  \item $f(x) := -x$ and $g(x) := x^2$
  \item $f(x) := -x$ and $g(x) := x^3$
  \end{enumerate}

  {\em Answer}: Option (D)

  {\em Explanation}: For option (D), $f(g(x)) = -x^2$ and $g(f(x)) =
  (-x)^2 = x^2$. The two polynomials take distinct values for all
  nonzero $x$, hence they are not equal as functions.

  {\em The other choices}:

  Option (A): $f(g(x)) = (x^3)^2 = x^{3 \cdot 2} = x^6$. Similarly
  $g(f(x)) = (x^2)^3 = x^{2 \cdot 3} = x^6$. So in this case $f \circ
  g = g \circ f$.

  Option (B): $f(g(x)) = (x + 2) + 1 = x + 3$ and $g(f(x)) = (x + 1) +
  2 = x + 3$. So in this case $f \circ g = g \circ g$.

  Option (C): Here, $f = g$ so both $f \circ g$ and $g \circ f$ are
  equal to $f \circ f$. Note that we do not need to explicitly compute
  $f \circ g$ in this case.

  Option (E): Here, $f(g(x)) = -x^3$ and $g(f(x)) = (-x)^3 = (-1)^3x^3
  = -x^3$.

  {\em Additional remark}: If $f \circ g = g \circ f$, we say that the
  functions $f$ and $g$ {\em commute}. You may have heard about the
  commutativity law for addition and multiplication. In the case of
  function composition, commutativity is {\em not} a law. It holds for
  some pairs of functions (such as options (A), (B), (D), (E) here)
  and not for others (such as option (C)).

  Note that for the function $f(x) := -x$, a function $h$ commutes
  with $f$ if and only if $h$ is an odd function. Thus, in option (E),
  the cube map is an odd function. And option (D) fails because the
  square map is {\em not} an odd function.

  {\em Performance review}: Everybody($12$ out of $12$) got this
  correct.

  {\em Historical note (last year)}: $14$ out of $15$ people got this
  correct. $1$ person chose (B).

  {\em Action point}: Even if you got this correct, it may be helpful
  to try to understand how to more quickly see that for all the other
  pairs, $f \circ g = g \circ f$.

\item (*) Which of the following functions is {\em not} periodic?

  \begin{enumerate}[(A)]
  \item $\sin(x^2)$
  \item $\sin^2x$
  \item $\sin(\sin x)$
  \item $\sin(x + 13)$
  \item $(\sin x) + 13$
  \end{enumerate}

  {\em Answer}: Option (A)

  {\em Explanation}: It's somewhat hard to show that $\sin(x^2)$ is
  not a periodic function. (This will be an advanced problem in a
  subsequent homework).

  On the other hand, it is easy to solve this problem by elimination,
  since the other four options are periodic, as explained below.

  {\em The other choices}:

  Option (B): $\sin^2$ is periodic and has period $\pi$. This follows
  from the fact that $\sin(\pi + x) = -\sin x$. Even if you don't
  notice that the period is $\pi$, you can still deduce
  $2\pi$-periodicity from the fact that $\sin$ is $2\pi$-periodic.

  Option (C): $\sin \circ \sin$ is $2\pi$-periodic. 

  Note that options (B) and (C) both from a more general fact: if $f,g :\R
  \to \R$ are functions, and $g$ is periodic, so is $f \circ g$, and
  any choice of $h > 0$ that works for $g$ also works for $f \circ g$
  (though the period for $f \circ g$ may be smaller than that for $g$)

  Option (D): This is $2\pi$-periodic. In graphical terms, it is
  obtained by shifting the graph of the $\sin$ function $13$ units to
  the left, and the periodicity pattern is unaffected.

  Option (E): This is $2\pi$-periodic. Adding a constant function (in
  this case, $13$) to a periodic function (in this case, $\sin$) gives
  a periodic function with the same period.

  {\em Performance review}: $8$ out of $12$ got this correct. $2$ each
  chose (B) and (C).

  {\em Historical note (last year)}: $8$ out of $15$ people got this
  correct. $3$ people chose (B), $3$ people chose (C), and $1$ person
  chose (D).

  {\em Action point}: Please make sure you understand the reasoning
  for why the other functions are periodic. You may also experiment
  with plotting the graphs of these functions using Mathematica or a
  graphing calculator.
\item (*) What is the domain of the function $\sqrt{1 - x} + \sqrt{x -
  2}$? Here, domain refers to the {\em largest set} on which the
  function can be defined.

  \begin{enumerate}[(A)]
  \item $(1,2)$
  \item $[1,2]$
  \item $(-\infty,1) \cup (2,\infty)$
  \item $(-\infty,1] \cup [2,\infty)$
  \item None of the above
  \end{enumerate}

  {\em Hint: Think clearly, first about what the domain of each of the
    two functions being added is, and then about whether you need to
    take the union or the intersection of the domains of the
    individual functions.}

  {\em Answer}: Option (E)

  {\em Explanation}: For $\sqrt{1 - x}$ to be defined, we need $1 - x
  \ge 0$, so $x \le 1$. For $\sqrt{x - 2}$ to be defined, we need $2 -
  x \ge 0$, so $x \ge 2$. Thus, we require that $x \le 1$ and $x \ge
  2$ hold {\em simultaneously}. In set theory terms, we need to take
  the {\em intersection} of the solution sets to $x \le 1$ (which is
  $(-\infty,1]$) and to $x \ge 2$ (which is $[2,\infty)$).

  The two conditions cannot hold together, i.e., the intersection of
  the solutions for the two constraints is empty. Hence, the domain of
  the function is in fact empty, i.e., the function is defined {\em
  nowhere}.

  {\em The other choices}: Options (C) and (D) are the most
  sophisticated distractors. Option (D) is the union of the domains of
  $\sqrt{1 - x}$ and $\sqrt{x - 2}$. However, what we need here is the
  {\em intersection} of the domains, not the union.

  {\em Performance review}: $5$ out of $12$ got this correct. $7$
  chose (D), which is the {\em union} rather than the {\em
  intersection}. So, people fell for the sophisticated distractors,
  not the silly ones.

  {\em Historical note (last year)}: $8$ out of $15$ people got this
  correct. $5$ people chose (D), $1$ person chose (B), and $1$ person
  chose (C).

  {\em Action point}: Please make sure you understand why we need to
  {\em intersect} the domains of $f$ and $g$ rather than take the
  union. This goes back to the fact discussed in class that the domain
  of $f + g$ is the intersection of the domains of $f$ and $g$.

\end{enumerate}
\end{document}