\documentclass{amsart}
\usepackage{fullpage,hyperref,vipul,graphicx}
\title{Functions: a rapid review (part 1)}
\author{Math 152, Section 55 (Vipul Naik)}

\begin{document}
\maketitle

{\bf Difficulty level}: Easy to moderate. Most of these are ideas you
should have encountered either implicitly or explicitly in the past.

{\bf Covered in class?}: This will roughly correspond to material
covered on Monday September 26. Most of the trickier aspects of this
will be covered in class, but many small points will be omitted due to
time constraints. Hence, it is recommended that you read through these
notes either before or after lecture.

{\bf Corresponding material in the book}: Sections 1.5, 1.6. Note that
the book covers the same material with somewhat different language and
different examples of functions, so you should go through it before or
while doing the homework problems.

{\bf Corresponding material in homework problems}: Homework 1, routine
problems 1--6 (all from section 1.5), advanced problem 1.
 
{\bf Things that students should get immediately}: The concepts of
function, domain, range, expression for function, table of values for
a function, graph of a function, the notion of piecewise defined
function, polynomials, rational functions, absolute value function,
signum function, positive part function.

{\bf Things that students should get with effort}: How to obtain a
piecewise definition for a maximum or minimum of two functions, how to
determine the domain and range of a function.

\section*{Executive review}

This review will probably be reproduced (with minor modifications) in
the midterm review sheet. It is meant as a {\em review summary} of
these lecture notes -- capturing those aspects of these notes that are
important on a second reading, and {\em ignoring} those things that
are significant for first time learning but not so important later on.

For first time reading, skip to the next section.

Words ...

\begin{enumerate}

\item The {\em domain} of a function is the set of possible
  inputs. The {\em range} is the set of possible outputs. When we say
  $f:A \to B$ is a function, we mean that the domain is $A$, and the
  range is a {\em subset} of $B$ (possibly equal to $B$, but also
  possibly a proper subset).
\item The main fact about functions is that {\em equal inputs give
  equal outputs}. We deal here with functions whose domain and range
  are both subsets of the real numbers.
\item We typically define a function using an algebraic expression,
  e.g. $f(x) := 3 + \sin x$. When an algebraic expression is given
  without a specified domain, we take the domain to be the largest
  possible subset of the real numbers for which the function makes
  sense.
\item Functions can be defined piecewise, i.e., one definition on one
  part of the domain, another definition on another part of the
  domain. Interesting things happen where the function changes
  definition.
\item Functions involving absolute values, max of two functions, min
  of two functions, and other similar constructions end up having
  piecewise definitions.
\end{enumerate}

Actions (think back to examples where you've dealt with these issues)...

\begin{enumerate}
\item To find the (maximum possible) domain of a function given using
  an expression, exclude points where:
  
  \begin{enumerate}
  \item Any denominator is zero.
  \item Any expression under the square root sign is negative.
  \item Any expression under the square root sign in the denominator is
    zero or negative.
  \end{enumerate}

\item To find whether a given number $a$ is in the range of a function
  $f$, try solving $f(x) = a$ for $x$ in the domain.
\item To find the range of a given function $f$, try solving $f(x) =
  a$ with $a$ now being an {\em unknown constant}. Basically, solve
  for $x$ in terms of $a$. The set of $a$ for which there exists one
  or more value of $x$ solving the equation is the range.
\item To write a function defined as $H(x) := \max \{ f(x), g(x) \}$
  or $h(x) := \min\{ f(x), g(x) \}$ using a piecewise definition, find
  the points where $f(x) - g(x)$ is zero, find the points where it is
  positive, and find the points where it is inegative. Accordingly,
  define $h$ and $H$ on those regions as $f$ or $g$.
\item To write a function defined as $h(x) := |f(x)|$ piecewise, split
  into regions based on the sign of $f(x)$.
\item To solve an equation for a function with a piecewise definition,
  solve for each definition within the piece (domain) for which that
  definition is satisfied.
\end{enumerate}

\section{What is a function?}\label{whatisafunction}

\subsection{Inputs and outputs, or so they say}

We're going to begin by talking about functions. You've probably
already seen functions in some form in calculus and precalculus. You
may have seen both the general concept of function and lots of
specific examples. In this course, we try to be a lot more precise
about what a function means. This precision will be very important
because functions are used for modeling purposes throughout
mathematics and mathematically based disciplines.

A function is something that ``takes in'' (or {\em eats} or {\em
gobbles}) an input and ``gives out'' (or {\em spits}) an output. Some
people think of a function as a {\em black box} or {\em machine} into
which you feed in input and get output. For instance, you put in money
into a cola vending machine and get out a cola. In today's computer
age, you might enter an input value onto a computer screen and get an
output. We say that a function {\em maps} the input to the output, so
functions are also called {\em mappings} or {\em maps}. Some people
say that a function {\em sends} an input to an output. Functions can
also be thought of as {\em rules} or {\em assignments}.

\subsection{The real bite: equal inputs give equal outputs}

So what's missing from this description? Well, the most important
thing about a function is that when you put in one input, you get one
output, and the output depends only on the input. In other words, {\em
equal inputs should give equal outputs}. So it doesn't depend on who
feeds the input or how the machine is feeling at the time it is fed
in. The output depends on the input, and only on the input. This
dependence is what we call a {\em functional} dependence.

So is Google a function? It takes in your query and outputs a bunch of
search results. But in another sense, it isn't a function, because
Google's results keep changing with time and other factors. What about
temperature? Is temperature a function of time?  No, because what the
temperature is at a given time depends on where you measure it. On the
other hand, the temperature at a particular point in space {\em is} a
function of time.

So, in order to make something a function, you have to specify enough
input so that the output is determined based on that.

In this course sequence, we will not be looking at functions with
weird inputs and weird outputs. For the most part, our inputs will be
single real numbers and our outputs will be single real numbers. So,
although the concept of function is very general, we will be
restricting to very particular kinds of functions to which we can
apply tools specifically developed for real numbers.

There are two important concepts related to functions: the {\em
domain} and {\em range}. We'll talk about these in more detail as we
proceed, but here's the rough description: the {\em domain is the set
of all possible (sensible) inputs to the function} and the {\em range
is the set of all possible outputs of the function}.

\subsection{Of circumferences and diameters: an illustrative example}

Let's first consider a ``real-world'' problem. The wheel of your
bicycle has diameter $d$. You want to find out the circumference of
your wheel. In more abstract jargon, you want to find out the
circumference of a circle {\em in terms of} its diameter.

The first question you should ask is: does the circumference {\em depend
only on} the diameter? That's not a silly question, even if in this
case it seems intuitively clear to some people. What does my question
mean? What it means is that if two circles have the same diameter, is
it necessary that they have the same circumference? If that isn't the
case, then we don't have a function.

In this case, the answer is {\em yes}. If the diameter is the same,
the circles are congruent -- you can translate one over to cover the
other. So their circumference should be the same. So yes, we do have a
function. But what kind of function is it?

Now, here's the thing. We have this loose reasoning that says that
there is a function that takes in an input $d$ and spits out an
input. Let's call this function $f$. The value obtained by applying
$f$ to $d$ is denoted $f(d)$. What this really means is $f$ evaluated
at $d$ or the value of $f$ at $d$. [Sidenote: In geometric settings, I
will abuse matters a little bit and conflate real numbers with length
measurements. The implicit idea everywhere is to choose a unit of
measurement, and all inputs are measured in those units. I will
perform this abuse on a regular basis. I could be more precise, but
you'll find this abuse everywhere so might as well get used to it.] So
$f(1)$ denotes the circumference of a circle with diameter $1$, and
$f(2)$ denotes the circumference of a circle with diameter $2$.

So far so good. Happy? Not quite. We've just shown there is a
function, but from a computational point of view, we haven't done
anything. What we would like to do is have some expression for $f$
that makes it easy to calculate. As it happens, various ancient
civilizations (the Greeks and Indians) showed that $f(d) = \pi d$,
where $\pi$ is some number. They also calculated the first few digits
of $\pi$, $\pi = 3.141592 \dots$. So this finally gives a formula.

So $f(1) = \pi$, $f(2) = 2\pi$, $f(3) = 3\pi$, $f(\pi) = \pi^2$, and
so on. What about $f(a + b)$? What's that? Well, to calculate $f$ of
something, you do $\pi$ times that thing, that's what the formula
tells you. So $f(a + b)$ is $\pi(a + b)$. What's that? That's $\pi a +
\pi b$, by the distributivity laws you learned in primary school. 

When you apply $f$ to something, you should make sure that you apply
$f$ to {\em the whole thing}. A common error that students make
is to just write $f(a + b) = \pi a + b$. That's wrong, because the
{\em whole expression} $a + b$ has to be multipled by $\pi$. So, at
the first stage of simplifying a function where the input is itself an
expression, {\em please put parentheses around the input} wherever you
write it down.

Hey, but $\pi a = f(a)$ and $\pi b = f(b)$, so we have this really
cool fact:

\begin{equation*}
  f(a + b) = f(a) + f(b) \ \forall \ a,b
\end{equation*}

The $\forall$ symbol above means {\em for all}. What
we have is a rule that holds for all values you plug in for $a$ and
$b$. Is this rule true for any function $f$, or only for the $f$ that
we wrote down? Well, it turns out that this is true for $f$ because
$f$ is a {\em linear function}: it is of the form $f(x) = cx$ for some
constant $c$.

What is $f(-1)$? The formula tells you it is $-\pi$. But hang
on. What does it even mean to have a circle of diameter $-1$?
Nothing. It's nonsense. It doesn't make sense. The diameter of a
circle cannot be negative, even though the formula makes perfect sense
for negative diameters.

Which brings us to the concept of {\em domain}. The domain of a
function is the set of values you can feed in as inputs. What's the
domain of $f$? It is the set of {\em positive real numbers}. There are
two ways to write this set: $\{ x \in \R \mid x > 0 \}$ and
$(0,\infty)$. [Explain both, if many students don't understand.]

So $f$ is a function {\em from} the positive reals {\em to} something
-- where? The set of values that $f$ can possibly take is termed the
{\em range} [SIDENOTE: There is a related concept of {\em co-domain}
  that we will discuss later.] For this choice of function $f$, the
range is also the set of positive real numbers.

We write this as follows:

\begin{equation*}
  f: (0,\infty) \to (0,\infty), \qquad f(x) := \pi x
\end{equation*}

Note that I changed the letter from $d$ to $x$. That was bad board
technique, but it is not mathematically a problem at all. Why? Because
that's just a name, and there's nothing in a name. May be $d$'s real
name is $d$, but I prefer to call it by the nickname (or alias)
$x$. The main thing to take care of is that the letter inside the
parentheses is the one used on the right side where the input should
go.

And by the way, that earlier equation was not quite correct. We should
really have:

\begin{equation*}
  f(a + b) = f(a) + f(b) \ \forall \ a,b \in \operatorname{dom}(f)
\end{equation*}

or:

\begin{equation*}
  f(a + b) = f(a) + f(b) \ \forall \ a,b \in (0,\infty)
\end{equation*}

[SIDENOTE, may not be covered in class: By the way, what we're using
here is the fact that, for this function $f$, if $a$ and $b$ are in
the domain of $f$, so is $a + b$. Why is that? This basically goes
back to the fact that the sum of two positive numbers is positive.]

[SIDENOTE: Local memory, don't give two functions the same letter name
in the same context, but feel free to reuse letters in different
contexts. Function name letters are just like variables in this
respect.]

Let's look at some other functions coming from geometry:

\begin{enumerate}

\item Area of a rectangle as a function of the perimeter? No,
  sorry. The perimeter does not give enough information to calculate
  the area of the rectangle. For instance, we can have a long and thin
  rectangle and a square of the same perimeter but very different
  areas.
\item Area of a square as a function of the perimeter? Yes, it is
  $g(x) = x^2/16$, where $g$ has domain and range the set of positive
  real numbers.

\end{enumerate}

Before we go into examples of functions, I just want to reiterate the
following: {\em whether something is a function} is a very different
question from {\em whether we have an expression to compute it}. You may be
able to successfully prove that something is a function but be
completely at a loss to actually compute it.

\section{Some important classes of functions}

\subsection{Constant functions}

A constant function is a function where the output is the same for all
inputs. A constant function can be identified by the constant value of
the output. For instance, the {\em zero} function is the function that
sends all its inputs to $0$.
\subsection{Polynomial functions}

The general expression for a polynomial looks like:

\begin{equation*}
  p(x) = a_0 + a_1x + \dots + a_nx^n
\end{equation*}

Here, the $a_i$ are all real numbers, and they're termed the {\em
coefficients} of the polynomial. If $a_n \ne 0$, $n$ is the {\em
degree} of the polynomial. $a_0$ is termed the {\em constant term} of
the polynomial and $a_n$ is termed the {\em leading coefficient} of the
polynomial.

The coefficients of the polynomial are constants, in the sense that
they do not depend on $x$. However, we haven't specified beforehand
the values of these constants. So they're {\em unknown knowns}.

Here are some concrete examples of polynomials:

\begin{itemize}
\item $2x - 5$ is a polynomial of degree $1$, with the constant term
  $a_0 = -5$ and the leading coefficient $a_1 = 2$.
\item $x^2 - 7x + 3$ is a polynomial of degree $2$ with the constant
  term $a_0 = 3$, the middle coefficient $a_1 = -7$, and the leading
  coefficient $a_2 = 1$.
\item $x^3 - 2$ is a polynomial of degree $3$ with constant term $a_0
  = -2$, leading coefficient $a_3 = 1$, and $a_1 = a_2 = 0$.
\end{itemize}


Polynomials are {\em globally defined functions}. In other
words, they have domain the whole real numbers $\R$. This is just a
fancy way of saying that you can evaluate a polynomial at any real
number without getting into trouble.

However, even though a function may be globally defined, we may
sometimes be interested in restricting it to a smaller domain. For
instance, in the circle example, we had the linear function $f(x) =
\pi x$. That expression is defined for all real numbers. However, the
real-world context from which we were getting the function required us
to {\em restrict the function} to a smaller domain: the set of
positive real numbers.

[SIDENOTE, cover in class only if somebody raises the question: What
about the range of a polynomial function? That turns out to be a
trickier question. We will need to build more machinery before we can
answer that question for arbitrary polynomials.]

\subsection{Rational functions}

Next, we consider rational functions. Here, we have the problem of
vanishing denominators. So, the {\em largest possible domain} for a
rational function is the real numbers minus all the points where the
denominator vanishes. By the way, the points where a polynomial
vanishes are called its {\em zeros} or {\em roots}. [SIDENOTE: See
``Convention on domains'', Page 28, in the book.]

For instance, consider the rational function $T(x) = x/(x^2 +
1)$. What is the largest possible domain for this function? To answer
this, first ask: where does the denominator vanish? Now, those of you
who've not seen complex numbers may say -- nowhere. And those of you
who've seen complex numbers will say $\pm i$. Yes, $\pm i$ are roots
of the polynomial $x^2 + 1$. But in this course we are dealing with
the real world. In all our discussions, whether I say it or not, all
numbers that we deal with are real numbers. And $x^2 + 1$ has no real
roots. So the denominator does not vanish anywhere and this rational
function is globally defined on $\R$.

Okay, what about this function called $FORGET$? $FORGET$ is defined by:

\begin{equation*}
  FORGET(x) = \frac{x}{x}
\end{equation*}

You may be tempted to cancel the $x$ and the $x$ and say that
$FORGET(x) = 1$ and so it is always defined. But one of the things
about rational functions is that you need to look at the rational
function {\em as it is written}. You cannot cancel something unless it
is guaranteed to be nonzero.

So, at the point $0$, the function becomes $0/0$, which is
undefined. At any other point, the function takes the value $0$. So,
we find that the domain is the set of nonzero real numbers. How do we
express this?

We can use the set difference notation. The domain is written as
$\R \setminus \{ 0 \}$. Or, we can think of it as the union of the
negative and the positive numbers. In this case, the domain is
$(-\infty,0) \cup (0,\infty)$. 

\includegraphics[width=3in]{forgetfunction.png}

\section{Computational tools}

\subsection{The domain}

When the domain is not explicitly specified or clear from the
situational context, the convention (cf. Page 28, ``conventions on
domains'', subtopic of Section 1.5) is to define the domain as the
largest possible subset of $\R$ where the function {\em as given} can
be evaluated. Some of the things you need to check for are:

\begin{itemize}
\item The denominator should not vanish. In other words, {\em we need
  to exclude from the domain all points where any denominator becomes
  zero}. For instance, $1/(x(x-1))$ is not defined at the points $0$
  and $1$ because the denominator vanishes at these points.
\item When you are taking the square root of some expression as a
  sub-expression of the function, then the thing under the square root
  should be nonnegative. For instance, for the function $\sqrt{x} +
  \sqrt{1-x}$, we should have both $x \ge 0$ and $1 - x \ge 0$.
\item When an expression under the square root is in the denominator,
  then the thing under the square root should be positive.
\end{itemize}

When we introduce new functions such as the logarithm and exponents
with arbitrary bases and exponents, we will introduce further rules
for determining the domain of the function based on the domain
properties of these functions.

\subsection{The range}

Let's try to translate the statement ``$a$ is in the range of $f$''
into a form that is computationally tractable. What does it mean
for $a$ to be in the range of $f$? It means that there exists a value
of $x$ such that $a = f(x)$.

For instance, consider the function $f(x) = 1/(x-1)$. How do we
determine whether a given $a \in \R$ is in the range of $f$? Okay, now
this might be a little too abstract and symbolic for some people, so I
would urge you to do this little trick. Imagine that $a$ is some
constant, some number {\em known to you} but not to me. So in your
mind, instead of $a$, you see a specific number. But since that number
is secret, you cannot reveal it to me and you have to call it $a$.

So you have this number $a$ that's known to you and you need to find
$x$ such that $f(x) = a$. Well, let's solve. We have:

\begin{eqnarray*}
  1/(x-1) & = &a \\
  \implies 1/a & = & x - 1 \\
  \implies x & = & 1 + (1/a)
\end{eqnarray*}

The goal is to determine whether there exists a $x$ such that $f(x) =
a$. What we've actually done is obtained a formula for $x$ in terms of
$a$. So for those $a$ where this formula makes sense, we actually do
have a value of $x$. What are those $a$? Well, all nonzero reals. So
when $a \ne 0$, we can find a $x$ such that $f(x) = a$. What about
when $a = 0$? In this case, it's clear that $f(x) = a$ has no
solution.

Okay, so this is the rough idea. But other situations can be a little
trickier. In some cases, you may get multiple values of $x$ mapping to
a single value of $a$.

Also see Examples 1 and 2, Page 28-29, and look at Exercises 18-30,
Page 30 (all of these are within Section 1.5).

\section{Describing a function}

\subsection{Description by algebraic expression}

So far we've discussed functions. Now, we want to discuss ways
of describing functions. One way of describing functions is using an
expression. We discussed examples of this last time, such as:

\begin{align*}
  f: (0, \infty) \to (0, \infty),& \qquad f(x) = \pi x \text{ sends diameter of circle to circumference}\\
  g: (0, \infty) \to (0, \infty),& \qquad g(x) = x^2/16 \text{ sends perimeter of square to area} \\
\end{align*}

But, unless you have a deep algebraic understanding of the expression,
this doesn't give a very good {\em feel} for the function. And in many
cases, an expression may not exist, or we may not know what it is. So
we employ two other tools: tabular listing and graphs.

\subsection{Description by tabular listing and graphs}

So let's do this tabular listing thing. Let's look at the function
$f(x) = x^2 - x + 3$. We want to get a feel for this function. Where
is it going up and down? How does it change? Let's try some numbers.

\begin{tabular}{|r|r|}
  \hline
  $x$ & $f(x)$ \\
  \hline
  $-2$ & $9$ \\
  $-1$ & $5$ \\
  $0$ & $3$ \\
  $1$ & $3$ \\
  $2$ & $5$ \\
  $3$ & $9$ \\
  $4$ & $15$ \\
  \hline
\end{tabular}

Do you see a picture here? Let's try to draw a graph for this
function. What's a graph? Well, it is a picture we draw in the plane
that allows us to read the value of the function at any point. We
choose a system of coordinate axes. The horizontal axis is
conventionally chosen to be the $x$-axis, with right positive, and the
vertical axis is the $f(x)$-axis, with up positive. We then plot the
points $(x,f(x))$ for all values of $x$ in the domain.

\includegraphics[width=3in]{parabolagraph.png}

Now, there are lots of real numbers, and we cannot plot the values for
all of them. So what I'm going to do here is a little imprecise. We'll
just plot the values at a few numbers (the ones we calculated in the
table) and then try to find an easy-to-draw curve that passes through
all those points. 

So we draw this graph. Now notice that I sort of assumed that the
graph moves smoothly, it doesn't have any unexpected kinks, like, it
doesn't jump wildly in between the points I plotted. You should take
that with a grain of salt. I haven't presented any evidence. To really
check that the graph I have drawn represents reality, you need to
check a lot of intermediate values.

So, by the way, now that we have drawn the graph, two questions
emerge: what's that bottom point for the graph? Or another way of
putting it: what is the minimum value of $f(x)$ and at what value of
$x$ is it attained? For the case of the quadratic, there is a neat
algebraic manipulation trick that can give us the answer. But since
you have seen some basic calculus, you are also aware of a general
procedure/approach to answering that kind of question.

\subsection*{Scaling issues with graphs}

For most of the graphs that we will draw in class, we will use the
same scale for both the $x$-axis and the $y$-axis. However, when using
graphs to study functions in practice, this is not useful. Indeed, for
many of the pictures of functions in these lecture notes using
Mathematica, the scale used for the two axes is different.

In addition, it is also sometimes helpful, when drawing graphs, to
shift the origin. Mathematica, and some other graphing softwares, may
do this automatically for many graphs. However, for graphs drawn in
class (as well as the Mathematica pictures included here) we will
assume that there is no shifting of origin.

\subsection{The vertical line test, domain and range}

Given a picture in the coordinate plane, we have the following:

\begin{itemize}
\item The picture represents the graph of a function if every vertical
  line intersects it at most once.
\item The domain of the function is the set of values of $x$ such that
  the vertical line for that value of $x$ intersects the graph.
\item The range of the function is the set of values of $y$ such that
  the horizontal line for that value of $y$ intersects the graph.
\end{itemize}

We will return to these points a little later when we study techniques
for drawing graphs.
\subsection{Functions defined piecewise}

Now, we're going to consider functions that have explicit expressions,
but they have different expressions for different values. In other
words, the domain of the function is split into parts and the
definition of the function is different for each part. We will say
that such functions are {\em piecewise defined}.

Can you think of an example? Let's think about taxes. Now, in a simple
world, the tax you pay is a (non-decreasing) function of your
income. The real world is a lot more complicated, with the tax you pay
being a function of many other factors. But let's ignore all
this. Let's consider the simplest tax system, which is called a {\em
flat tax}. \footnote{For instance, income tax in the state of Illinois
is a flat tax, with a tax rate of $3\%$ or $0.03$. Eastern European
countries such as Estonia, Latvia, Russia, and Bulgaria have flat or
near-flat tax systems.}

So here's how a simple version of the flat tax works. There is a {\em
basic exemption} amount, which I'll call $B$, and a tax rate $r$ for
all income earned over and above $B$. In other words, the first $B$
units of money that you earn don't get taxed, and of the remaining
money you earn, a fraction $r$ is taxed. By the way, $0 < r < 1$, and
if you write the tax rate as a percentage, you have to divide it by
$100$ to get $r$. So, for instance, a tax rate of $10\%$ means
that $r$ is $0.1$.

[SIDENOTE: So, why did I put {\em strict inequality} (a $<$ sign
instead of a $\le$ sign and a $>$ sign instead of a $\ge$ sign)?
Well, what does $r = 0$ mean? It means that there is effectively no
tax at all for any income, which isn't a case of interest here. And
what does $r = 1$ mean? It means that all money you earn beyond $B$
belongs to the government, and that doesn't provide people with much
incentive to earn. So in fact $r$ should be between $0$ and $1$. What
the optimal value of $r$ should be is a question beyond the scope of
this discussion.]

So if $T$ is the tax function, we have:

\begin{equation*}
  T(I) = r(I - B)
\end{equation*}

This formula is correct for people who earn as much as or more than
$B$. But what about people who earn less than $B$? What if, for
instance, your income is $0$? The formula then says that your tax is
$T(0) = -rB$, which means you have a {\em negative tax}. But that's
not the way flat tax systems usually work. So, the real formula is:

\begin{equation*}
  T(I) = \lbrace \begin{array}{lr}
               0 &\text{ if } I < B \\
	       r(I - B) &\text{ if } I \ge B
	       \end{array}
\end{equation*}

In other words, $T(I)$ is zero for income up to $B$, and then rises
{\em linearly} (or proportionally) with income.

So let's draw the graph. In the graph, the $x$-axis is now the
$I$-axis, or income axis, because the income is the input
variable. And the $y$-axis is the $T$-axis or the tax axis, because
that's the output variable. Note that the tax function goes from
$[0,\infty)$ to $[0,\infty)$, i.e., both the income and the tax are
nonnegative.

\includegraphics[width=3in]{flattaxfunction.png}

The graph starts off along the horizontal axis (the $I$-axis) from $0$
to $B$. Then, at $B$, it takes a turn and goes in a straight line
forever. This line points northeast. Now, what can you say about how
steep that line can be? It depends on the rate $r$, but can you say
something in general? Sure. Since $r < 1$, the largest angle that line
can make with the horizontal axis is $\pi/4$ -- that's the angle when
$r = 1$. The smaller the $r$, the smaller the angle.

Note that the function takes a turn at the value $B$, but it does not
jump in value. In other words, you can draw the graph without lifting
your pencil. So what's happening is that when you cross $B$, there is
a shift in the tax regime, but your tax function doesn't jump
suddenly. Since most of you have some idea of what continuous and
differentiable means, you can probably make this precise in those
terms: the tax function is continuous everywhere (including at the
point $B$ where it changes definition) but it is not differentiable at
$B$.

So, just as a fun question, what happens with a {\em progressive tax
function}?\footnote{The term ``progressive'' here is used in a strictly
mathematical, rather than a political sense, even though
self-identified political progressives on average tend to favor more
progressive tax systems.} By the way, the United States, and most
countries, have progressive tax systems. What that would mean is that
in addition to the base exemption, there are likely to be other income
cutoff values at which the graph takes turns. So you might start off
with an almost horizontal line, then turn to a slightly steeper slope,
then an even steeper slope and so on. Of course, the tax rate should
never exceed $1$, so you'll never get steeper than an angle of $\pi/4$
with the horizontal axis.

And what happens with a {\em regressive tax function}? For instance,
the payroll tax in the United States is a regressive tax. Here, the
graph becomes {\em less steep} as the income increases.

\subsection{Back to mathematics}
Coming back to mathematics, here are some important functions with
piecewise definitions:

\begin{enumerate}
\item {\bf Absolute value function}: The function is denoted, not by a
  letter, but by bars. For a real number $x$, the absolute value of
  $x$, denoted $|x|$, is defined as $x$ if $x \ge 0$ and as $-x$ if $x
  < 0$. It is also termed the {\em magnitude}.

  \includegraphics[width=3in]{absolutevalue.png}

\item {\bf Signum function}: The sign function or signum function
  sends all positive numbers to $+1$, $0$ to $0$, and all negative
  numbers to $-1$. For $x \ne 0$, the signum function is thus equal to
  the function that sends $x$ to $x/|x|$. In an alternate conventions,
  the signum function is considered undefined at $0$, so its domain is
  $\R \setminus \{ 0 \}$.

  \includegraphics[width=3in]{signumfunction.png}

\item {\bf Positive part function}: This function applied to a real
  number $x$ is denoted $x^+$, and is defined as $\max \{ 0, x
  \}$. The positive part of $x$ is equal to $0$ if $x \le 0$ and equal
  to $x$ if $x > 0$.

  \includegraphics[width=3in]{positivepartfunction.png}

\end{enumerate}

How do we think of functions defined piecewise? The main thing to
remember is that most of the action happens at the place where the
function changes definition. Think of it like switching gears or
taking a turn. That's what happens literally with the tax function,
and the absolute value function and the signum function. And whenever
you have piecewise defined functions and you know that the functions
on each of the parts are very well-behaved, these turning points are
the ones where mischief is most likely to occur.

\section{The max and min operators}

Among the constructs used to create piecewise functions, that arise
naturally, are the $\max$ and $\min$ operators. For instance, suppose
$f$ and $g$ are functions on the real numbers. Then, consider the
function:

\begin{equation*}
  h(x) := \max\{f(x), g(x) \}
\end{equation*}

What does this mean? Here is how we evaluate $h(x)$ at a given value
of $x$. We compute both $f(x)$ and $g(x)$. If $f(x) > g(x)$, then we
set $h(x) = f(x)$. If $g(x) > f(x)$, then we set $h(x) = g(x)$. If
both are equal, we set $h(x)$ to be that equal value. In other words:

\begin{equation*}
  h(x) := \lbrace\begin{array}{lr}f(x) & \qquad \text{if } f(x) > g(x) \\ g(x) & \qquad \text{otherwise} \\\end{array}
\end{equation*}

For instance, consider the function:

\begin{equation*}
  h(x) := \max \{ x + 1, 2x \}
\end{equation*}

Here are the graphs of the two functions:

\includegraphics[width=3in]{xplusoneandtwox.png}

At $x = 0$, $x + 1 = 1$ and $2x = 0$, and the maximum of these two
values is $1$. So $h(0) = 1$.

At $x = 1$, $x + 1 = 2$ and $2x = 2$. Both are equal to $2$, so $h(1)
= 2$.

At $x = 2$, $x + 1 = 3$ and $2x = 4$. The maximum of these is $4$, so
$h(2) = 4$.

A similar approach works for calculating the $\min$ of two functions.

Here are the graphs of the $\max$ and $\min$ functions for $x + 1$ and
$2x$.

\includegraphics[width=3in]{xplusoneandtwox.png}

Many of the piecewise defined functions that we encounter can
naturally be described using the $\max$ operator. For instance:

\begin{enumerate}

\item The flat tax function with base exemption $B$ and rate $r$ can
  be defined as the function $T(I) = \max \{ 0, r(I - B) \}$.

\item The absolute value function, that sends $x$ to $|x|$, can be
  defined as $\max \{ x, -x \}$.

\item The positive part function, that sends $x$ to $x^+$, can be
  defined as $\max \{ x, 0 \}$.
\end{enumerate}

Now, here's a way of thinking of the maximum of two functions. What we
need to do is determine, at every point, which one of them is
bigger. Think of it as a race. The two functions are constantly racing
against each other. At some values of $x$, one function might come on
top, and at other values of $x$, the other function might come on top.

For instance, think of the absolute value function. Let's first plot
the graph of the function that sends $x$ to $x$ and the function that
sends $x$ to $-x$. The first is a straight line pointing north-east to
soth-west, the second is a straight line pointing north-west to
south-east.

Now, if you start off at $-\infty$ and move right, the function $-x$
dominates, as is obvious both graphically and algebraically. And it
keeps dominating up till the point $x = 0$, where the two functions
become equal. After that the function $x$ dominates. Thus, we see that
$|x| = -x$ for $x < 0$ and $|x| = x$ for $x \ge 0$.

So the main point is that the places where the functions switch roles
are the places where the two functions become equal. [SIDENOTE:
Technically, this statement requires both functions to be continuous.]
So if we have $h(x) := \max \{ f(x),g(x) \}$, then the first thing we
should do is find the points where $f(x) = g(x)$. Then, in the
intervals between these points, we can try to find out which of the
two functions is greater.

For instance, consider the function:

\begin{equation*}
  f(x) := \max \{ x, x^2 + 1 \}
\end{equation*}

\includegraphics[width=3in]{xandxsquareplusone.png}

The first thing you want to know is when those two functions become equal. So you try to solve:

\begin{equation*}
  x = x^2 + 1
\end{equation*}

which simplifies to:

\begin{equation*}
  x^2 - x + 1 = 0
\end{equation*}

Now, that function that we just wrote down there has no real
roots. You can check this by evaluating the {\em discriminant} -- the
$b^2 - 4ac$ term. The discriminant is negative, hence there are no
real roots. So this function is never zero.

Thus, one of the two functions, $x$ or $x^2 + 1$, always has the upper
hand. Now you can just plug in one value of $x$ and see that $x^2 + 1$
is in fact the upper hand. So, in fact:

\begin{equation*}
  f(x) = x^2 + 1 \ \forall \ x \in \R
\end{equation*}

By the way, there is another way of showing that $x^2 + 1 > x$ for all
$x \in \R$. This is by writing:

\begin{equation*}
  x^2 - x + 1 = (x - 1/2)^2 + (3/4) \ge 3/4 > 0
\end{equation*}

The secret I used here is {\em completing the square using the middle
term}. This technique will be of importance later when we study
integration techniques.

\end{document}
