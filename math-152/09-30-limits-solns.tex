\documentclass[10pt]{amsart}

%Packages in use
\usepackage{fullpage, hyperref, vipul, enumerate}

%Title details
\title{Class quiz solutions: September 30: Limits}
\author{Vipul Naik}
%List of new commands

\begin{document}
\maketitle

\section{Performance review}

$12$ people took this quiz. The score distribution was as follows:

\begin{itemize}
\item Score of $2$: $8$ people.
\item Score of $3$: $4$ people.
\end{itemize}

The median score was $2$ and the mean score was $2.33$.

Here is the information by question:

\begin{enumerate}
\item Option (A): $11$ people got this correct.
\item Option (D): $8$ people got this correct. {\em Although many of
  you got this correct, it's likely that some of you did so through
  educated guesswork, so I recommend you go through the solution,
  which is not completely straightforward.}
\item Option (B): $9$ people got this correct.
\end{enumerate}

More details in the next section.

\section{Solutions}
\begin{enumerate}

\item (**) We call a function $f$ left continuous on an open interval $I$
  if, for all $a \in I$, $\lim_{x \to a^-} f(x) = f(a)$. Which of the
  following is an example of a function that is left continuous but
  not continuous on $(0,1)$?

  \begin{enumerate}[(A)]
  \item $f(x) := \lbrace\begin{array}{rl}x, & 0 < x \le 1/2 \\ 2x, & 1/2 < x < 1 \\\end{array}$
  \item $f(x) := \lbrace\begin{array}{rl}x, & 0 < x < 1/2 \\ 2x, & 1/2 \le x < 1 \\\end{array}$
  \item $f(x) := \lbrace\begin{array}{rl}x, & 0 < x \le 1/2 \\ 2x - (1/2), & 1/2 < x < 1 \\\end{array}$
  \item $f(x) := \lbrace\begin{array}{rl}x, & 0 < x < 1/2 \\ 2x - (1/2), & 1/2 \le x < 1 \\\end{array}$
  \item All of the above
  \end{enumerate}

  {\em Answer}: Option (A)

  {\em Explanation}: Note that in all four cases, the two pieces of
  the function are continuous. Thus, the relevant questions are: (i) do
  the two definitions agree at the point where the definition changes
  (in all four cases here, $1/2$)? and (ii) is the point (in all cases,
  $1/2$) where the definition changes included in the left or the
  right piece?

  For options (C) and (D), the definitions on the left and right piece
  agree at $1/2$. Namely the function $x$ and $2x - (1/2)$ both take the
  value $1/2$ at the domain point $1/2$. Thus, options (C) and (D)
  both define continuous functions (in fact, the same continuous
  function).

  This leaves options (A) and (B). For these, the left definition $x$
  and the right definition $2x$ do not match at $1/2$: the former
  gives $1/2$ and the latter gives $1$. In other words, the function
  has a jump discontinuity at $1/2$. Thus, (ii) becomes relevant: is
  $1/2$ included in the left or the right definition?

  For option (A), $1/2$ is included in the left definition, so $f(1/2)
  = 1/2 = \lim_{x \to 1/2^-} f(x)$. On the other hand, $\lim_{x \to
  1/2^+} f(x) = 1$. Thus, the $f$ in option (A) is left continuous but
  not right continuous.

  For option (B), $1/2$ is included in the right definition, so
  $f(1/2) = 1$ and $f$ is right continuous but not left continuous at
  $1/2$.

  {\em Performance review}: $11$ out of $12$ people got this
  correct. $1$ chose (C).

  {\em Historical note (last year)}: $6$ out of $13$ people got this
  correct. $6$ people chose option (E) -- I'm not sure why this option
  was so popular. $1$ person chose option (C) but was quite close to
  choosing (A).

  {\em Action point}: Please read through the lecture notes on Chalk
  (title ``informal introduction to limits'' -- these roughly
  correspond to today's lecture), as well as any notes you took during
  class discussion today, very carefully, till you are completely
  confusion-free on the issues of left and right limits and
  continuity. This is the kind of question that, once you are thorough
  with the definitions, you should be able to get correctly. In other
  words, while the question is ``hard'' today, I expect it to be in
  the ``moderately easy'' category by the time you take the first
  midterm.

\item (**) Suppose $f$ and $g$ are functions $(0,1)$ to $(0,1)$ that are
  both left continuous on $(0,1)$. Which of the following is {\em not}
  guaranteed to be left continuous on $(0,1)$?

  \begin{enumerate}[(A)]
  \item $f + g$, i.e., the function $x \mapsto f(x) + g(x)$
  \item $f - g$, i.e., the function $x \mapsto f(x) - g(x)$
  \item $f \cdot g$, i.e., the function $x \mapsto f(x)g(x)$
  \item $f \circ g$, i.e., the function $x \mapsto f(g(x))$
  \item None of the above, i.e., they are all guaranteed to be left
  continuous functions
  \end{enumerate}

  {\em Answer}: Option (D)

  {\em Explanation}: We need to construct an explicit example, but we
  first need to do some theoretical thinking to motivate the right
  example. The full reasoning is given below.

  {\em Motivation for example}: Left hand limits split under addition,
  subtraction and multiplication, so options (A)-(C) are guaranteed to
  be left continuous, and are thus false. This leaves the option $f
  \circ g$ for consideration. Let us look at this in more detail.

  For $c \in (0,1)$, we want to know whether:

  $$\lim_{x \to c^-} f(g(x)) \stackrel{?}{=} f(g(c))$$

  We do know, by assumption, that, as $x$ approaches $c$ from the
  left, $g(x)$ approaches $g(c)$. However, we do not know whether
  $g(x)$ approaches $g(c)$ from the left or the right or in
  oscillatory fashion. If we could somehow guarantee that $g(x)$
  approaches $g(c)$ from the left, then we would obtain that the above
  limit holds. However, the given data does not guarantee this, so (D)
  is false.

  We need to construct an example where $g$ is {\em not} an increasing
  function. In fact, we will try to pick $g$ as a decreasing function,
  so that when $x$ approaches $c$ from the left, $g(x)$ approaches
  $g(c)$ from the right. As a result, when we compose with $f$, the
  roles of left and right get switched. Further, we need to construct
  $f$ so that it is left continuous but not right continuous.

  {\em Explanation with example}: Consider the case where, say:

  $$f(x) := \lbrace\begin{array}{rl}1/3,& 0 < x \le 1/2 \\ 2/3, & 1/2 < x < 1 \\\end{array}$$

  and

  $$g(x) := 1 - x$$

  Note that both functions have range a subset of $(0,1)$.

  Composing, we obtain that:

  $$f(g(x)) = \lbrace\begin{array}{rl}2/3,& 0 < x < 1/2\\ 1/3, & 1/2 \le x < 1 \\\end{array}$$

  $f$ is left continuous but not right continuous at $1/2$, whereas $f
  \circ g$ is right continuous but not left continuous at $1/2$.

  {\em Performance review}: $8$ out of $12$ got this correct. $4$
  chose (E).

  {\em Historical note (last year)}: $4$ out of $13$ people got this correct. $9$
  people chose option (E). This is understandable, because if you look
  only at the obvious examples (all of which are increasing
  functions), you are likely to think that $f \circ g$ must be left
  continuous. If you got this question right for the right reasons,
  congratulate yourself.

  {\em Action point}: We will emphasize the moral of this problem in a
  class in the near future. When we discuss the theorems involving
  limits, we will note that the theorems on sums, differences,
  products, etc. also hold for one-sided limits (i.e., each of the
  theorems holds for left hand limits and each of the theorems holds
  for right hand limits). However, the theorem on compositions for
  limits does not hold for one-sided limits, unless we make additional
  assumptions. I hope you will never forget this point (or at least,
  not till the end of the winter quarter).

\item (*) Consider the function:

  $$f(x) := \lbrace\begin{array}{rl} x, & x \text{ rational }\\1/x, & x \text{ irrational }\\\end{array}$$

  What is the set of all points at which $f$ is continuous?

  \begin{enumerate}[(A)]
  \item $\{ 0, 1 \}$
  \item $\{ -1,1 \}$
  \item $\{-1,0 \}$
  \item $\{ -1,0,1 \}$
  \item $f$ is continuous everywhere
  \end{enumerate}

  {\em Answer}: Option (B)

  {\em Explanation}: In this interesting example, instead of a {\em
  left} versus {\em right} split, we are splitting the domain into
  rationals and irrationals. For the overall limit to exist at $c$, we
  need that: (i) the limit for the function as defined for rationals
  exists at $c$, (ii) the limit for the function as defined for
  irrationals exists at $c$, and (iii) the two limits are equal.

  Note that regardless of whether the point $c$ is rational or
  irrational, we need {\em both} the rational domain limit and the
  irrational domain limit to exist and be equal at $c$. This is
  because rational numbers are surrounded by irrational numbers and
  vice versa -- both rational numbers and irrational numbers are dense
  in the reals -- hence at any point, we care about the limits
  restricted to the rationals as well as the irrationals.

  The limit for rationals exists for all $c$ and equals the value
  $c$. The limit for irrationals exists for all $c \ne 0$ and equals
  the value $1/c$. For these two numbers to be equal, we need $c =
  1/c$. Solving, we get $c^2 = 1$ so $c = \pm 1$.

  {\em Performance review}: $9$ out of $12$ got this correct. $3$
  chose (E).

  {\em Historical note (last year)}: $5$ out of $13$ people got this correct. $3$
  people chose (D), $3$ people chose (A), and $1$ person each chose
  (C) and (E). Some of the people who chose (D) wrote ``all
  rationals'', so they probably thought that the correct answer is
  ``all rationals'' but it was not one of the options.

  {\em Action point}: Getting this correct requires a thorougher
  definition of limit than the purely intuitive one. Like the
  $\sin(1/x)$-based functions, these functions are fascinating
  precisely because of the lack of clarity in what it means for such a
  function to have a limit. In a couple of weeks, after you have dealt
  more with functions defined differently for the rationals and
  irrationals, {\em and} seen the $\epsilon-\delta$ definition of
  limit, you will be in a much better position to tackle this
  question. By the time of the first midterm, this question should be
  in the ``moderately easy'' category.
\end{enumerate}
\end{document}