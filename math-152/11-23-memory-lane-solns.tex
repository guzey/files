\documentclass[10pt]{amsart}

%Packages in use
\usepackage{fullpage, hyperref, vipul, enumerate}

%Title details
\title{Class quiz solutions: November 23: Memory lane}
\author{Math 152, Section 55 (Vipul Naik)}
%List of new commands

\begin{document}
\maketitle

\section{Performance review}

$12$ people took this quiz. The score distribution was as follows:

\begin{itemize}
\item Score of $2$: $5$ people
\item Score of $3$: $3$ people
\item Score of $4$: $2$ people
\item Score of $6$: $1$ person
\item Score of $7$: $1$ person
\end{itemize}

The mean score was $3.33$. The problem wise answers and performance
are as follows:

\begin{enumerate}
\item Option (B): $5$ people
\item Option (E): $7$ people
\item Option (D): $9$ people
\item Option (C): $7$ people
\item Option (D): $5$ people
\item Option (D): $2$ people
\item Option (A): $5$ people
\end{enumerate}

\section{Solutions}

\begin{enumerate}
\item For which of the following specifications is there {\bf no
  continuous function} satisfying the specifications?

  \begin{enumerate}[(A)]
  \item Domain $[0,1]$ and range $[0,1]$
  \item Domain $[0,1]$ and range $(0,1)$
  \item Domain $(0,1)$ and range $[0,1]$
  \item Domain $(0,1)$ and range $(0,1)$
  \item None of the above, i.e., we can get a continuous function for
    each of the specifications.
  \end{enumerate}

  {\em Answer}: Option (B)

  {\em Explanation}: By the extreme value theorem, any continuous
  function on a closed bounded interval must attain its maximum and
  minimum, and hence its image cannot be an open interval.

  {\em The other choices}:

  For options (A) and (D), we can pick the identity functions $f(x) :=
  x$ on the respective domains.

  For option (C), we can pick the function $f(x) := \sin^2(2\pi x)$ on
  the domain $(0,1)$. 

  {\em Performance review}: $5$ out of $12$ got this. $6$ chose (C),
  $1$ chose (D).

  {\em Historical note (last year)}: $7$ out of $14$ people got this
  correct. $5$ people chose (C) and $2$ people chose (E).

  {\em Action point}: Any question that involves feasible options for
  the range of a function should remind you of the {\em intermediate
  value theorem} and {\em extreme value theorem}. It seems likely that
  the people who got this question wrong (and perhaps some of the
  point who got it right too!) did not even think of the extreme value
  theorem.
\item Suppose $f$ and $g$ are continuous functions on $\R$, such that
  $f$ is continuously differentiable everywhere and $g$ is
  continuously differentiable everywhere except at $c$, where it has a
  vertical tangent. What can we say is {\bf definitely true} about $f
  \circ g$?

  \begin{enumerate}[(A)]
  \item It has a vertical tangent at $c$.
  \item It has a vertical cusp at $c$.
  \item It has either a vertical tangent or a vertical cusp at $c$.
  \item It has neither a vertical tangent nor a vertical cusp at $c$.
  \item We cannot say anything for certain.
  \end{enumerate}

  {\em Answer}: Option (E).

  {\em Explanation}: Consider $g(x) := x^{1/3}$. This has a vertical
  tangent at $c = 0$. If we choose $f(x) = x$, we get (A). If we
  choose $f(x) = x^2$, we get (B). If we choose $f(x) = x^3$, we get
  neither a vertical tangent nor a vertical cusp. Hence, (E) is the
  only viable option.

  {\em Performance review}: $7$ out of $12$ got this correct. $4$
  chose (C), $1$ chose (D).

  {\em Historical note (last year)}: $5$ out of $14$ people got this
  correct. Other choices were (A) (3), (C) (4), (B) (1),and (D) (1).

  {\em Historical note}: In an earlier quiz where this question
  appeared, $3$ out of $15$ people got this correct. Other choices were
  (A) (7), (C) (4), and (D) (1). The main thing that people had
  trouble with was thinking of possibilities for $f$ that could play
  the role of converting the vertical tangent behavior of the original
  function $g$ into vertical cusp or ``neither'' behavior for the
  composite function.

  {\em Action point}: Performance this time was a little better than
  earlier, but it seems that many of you either did not read the
  original solution or it did not register properly in your
  minds. Well, there's always a second chance! Take it this time.

\item Consider the function $p(x) := x^{2/3}(x - 1)^{3/5} +
  (x-2)^{7/3}(x - 5)^{4/3}(x - 6)^{4/5}$. For what values of $x$ does
  the graph of $p$ have a vertical cusp at $(x,p(x))$?

  \begin{enumerate}[(A)]
  \item $x = 0$ only.
  \item $x = 0$ and $x = 5$ only.
  \item $x = 5$ and $x = 6$ only.
  \item $x = 0$ and $x = 6$ only.
  \item $x = 0$, $x = 5$, and $x = 6$.
  \end{enumerate}

  {\em Answer}: Option (D)

  {\em Explanation}: This uses local behavior heuristics, both
  additive and multiplicative. We need the exponent on top to be $p/q$
  where $0 < p < q$ with $p$ even and $q$ odd.

  {\em Performance review}: $9$ out of $12$ got this correct. $1$ each
  chose (A), (C), and (E).

  {\em Historical note (last year)}: $3$ out of $14$ people got this
  correct. $5$ people chose (E) (indcating that they probably forgot
  the condition that $p < q$), $4$ people chose (C), $3$ people chose
  (B), and $1$ person chose (A).

  {\em Action point}: Review the local behavior heuristics section of
  the review sheet for midterm 2. Or, if this was just a careless
  error about not noting that a particular number was bigger than $1$,
  don't make the careless error again.

\item Consider the function $f(x) := \lbrace\begin{array}{rl}x, & 0
  \le x \le 1/2 \\ x^2, & 1/2 < x \le 1\end{array}$. What is $f \circ f$?

  \begin{enumerate}[(A)]
  \item $x \mapsto \lbrace\begin{array}{rl} x, & 0 \le x \le 1/2\\ x^4, & 1/2 < x \le 1\end{array}$
  \item $x \mapsto \lbrace\begin{array}{rl} x, & 0 \le x \le 1/2\\ x^2, & 1/2 < x \le 1\end{array}$
  \item $x \mapsto \lbrace\begin{array}{rl} x, & 0 \le x \le 1/2\\ x^2, & 1/2 < x \le 1/\sqrt{2}\\ x^4, & 1/\sqrt{2} < x \le 1\end{array}$
  \item $x \mapsto \lbrace \begin{array}{rl} x, & 0 \le x \le 1/\sqrt{2}\\ x^2,& 1/\sqrt{2} < x \le 1\end{array}$
  \item $x \mapsto \lbrace\begin{array}{rl} x, & 0 \le x \le 1/\sqrt{2}\\ x^4, &1/\sqrt{2} < x \le 1\end{array}$
  \end{enumerate}

  {\em Answer}: Option (C)

  {\em Explanation}: If $0 \le x \le 1/2$, then $f(x) = x$, so
  $f(f(x)) = x$. If $1/2 < x \le 1$, then $f(x) = x^2$. What happens
  when we apply $f$ to that depends on where $x^2$ falls. If $0 \le
  x^2 \le 1/2$, then $f(x^2) = x^2$, so $f(f(x)) = x^2$. This covers
  $1/2 < x \le 1/\sqrt{2}$. Otherwise $f(x^2) = x^4$, so $f(f(x)) = x^4$.

  {\em Performance review}: $7$ out of $12$ got this correct. $3$ chose
  (A), $1$ chose (A), $1$ chose (B).

  {\em Historical note (last year)}: $4$ out of $14$ people got this
  correct. $4$ people chose (E), $4$ people chose (A), $1$ person
  chose (D), and $1$ person left the question blank.

  {\em Action point}: It seems that many people don't have the correct
  conceptual picture of how to compose functions with piecewise
  definitions. {\em You need to spend some time to understand this --
  please do!} We will talk briefly about this in one of the subsequent
  review opportunities.
\item Suppose $f$ and $g$ are functions $(0,1)$ to $(0,1)$ that are
  both right continuous on $(0,1)$. Which of the following is {\em not}
  guaranteed to be right continuous on $(0,1)$?

  \begin{enumerate}[(A)]
  \item $f + g$, i.e., the function $x \mapsto f(x) + g(x)$
  \item $f - g$, i.e., the function $x \mapsto f(x) - g(x)$
  \item $f \cdot g$, i.e., the function $x \mapsto f(x)g(x)$
  \item $f \circ g$, i.e., the function $x \mapsto f(g(x))$
  \item None of the above, i.e., they are all guaranteed to be right
    continuous functions
  \end{enumerate}

  {\em Answer}: Option (D)

  {\em Explanation}: See the explanation for Question 2 on the October
  1 quiz. Note that that quiz uses left continuity, but the example
  can be adapted to right continuity.

  {\em Performance review}: $5$ out of $12$ got this correct. $4$
  chose (E), $3$ chose (C).

  {\em Historical note (last year)}: $9$ out
  of $14$ people got the question correct. $3$ people chose (E) and
  $1$ person each chose (B) and (C).

\item For a partition $P = x_0 < x_1 < x_2 < \dots < x_n$ of $[a,b]$
  (with $x_0 = a$, $x_n = b$) define the norm $\| P \|$ as the maximum
  of the values $x_i - x_{i-1}$. Which of the following {\bf is always
  true} for any continuous function $f$ on $[a,b]$? (5 points)

  \begin{enumerate}[(A)]
  \item If $P_1$ is a finer partition than $P_2$, then $\| P_2 \| \le
    \| P_1 \|$ (Here, {\em finer} means that, as a set, $P_2 \subseteq
    P_1$, i.e., all the points of $P_2$ are also points of $P_1$).
  \item If $\| P_2 \| \le \| P_1 \|$, then $L_f(P_2) \le L_f(P_1)$
    (where $L_f$ is the lower sum).
  \item If $\| P_2 \| \le \| P_1 \|$, then $U_f(P_2) \le U_f(P_1)$
    (where $U_f$ is the upper sum).
  \item If $\| P_2 \| \le \| P_1 \|$, then $L_f(P_2) \le U_f(P_1)$.
  \item All of the above.
  \end{enumerate}

  {\em Answer}: Option (D).

  {\em Explanation}: Option (D) is true for the rather trivial reason
  that any lower sum of $f$ over any partition cannot be more than any
  upper sum of $f$ over any partition. The norm plays no role.

  Option (A) is incorrect because the inequality actually goes the
  other way: the finer partition has the smaller norm. Options (B) and
  (C) are incorrect because a smaller norm does not, in and of itself,
  guarantee anything about how the lower and upper sums compare.

  {\em Performance review}: $2$ out of $12$ got this correct. $5$
  chose (C), $3$ chose (B), $1$ each chose (A) and (E).

  {\em Historical note (last year)}: $9$ out of $14$ people got this
  correct. $3$ people chose (B) and $1$ person each chose (A) and (C).

  {\em Historical note 1}: In the previous quiz appearance, $4$ out of
  $15$ prople got this correct. $8$ people chose (C), presumably with
  the intuition that the smaller the norm of a partition, the smaller
  its upper sums. While this intuition is right in a broad sense, it
  is not correct in the precise sense that would make (C) correct. It
  is possible that a lot of people did not read (D) carefully, and
  stopped after seeing (C), which they thought was a correct
  statement. $1$ person each chose (A), (E), and (C)+(D).

  {\em Historical note 2}: This question appeared in a 152 midterm two
  years ago, and $6$ of $29$ people got this right. Many people chose
  (C) in that test too (though I haven't preserved numerical
  information on number of wrong choices selected).

\item A disk of radius $r$ in the $xy$-plane is translated parallel to
  itself with its center moving in the $yz$-plane along the semicircle
  $y^2 + z^2 = R^2, y \ge 0$. The solid thus obtained can be thought
  of as a {\em cylinder of bent spine} with cross sections being disks
  of radius $r$ along the $xy$-plane and the centers forming a
  semicircle of radius $R$ in the $yz$-plane, with the $z$-value
  ranging from $-R$ to $R$. What is the volume of this solid?

  \begin{enumerate}[(A)]
  \item $2\pi r^2R$
  \item $\pi^2r^2R$
  \item $2\pi rR^2$
  \item $\pi^2rR^2$
  \item $\pi^2R^3$
  \end{enumerate}

  {\em Answer}: Option (A)

  {\em Explanation}: For $-R \le z \le R$, the cross section in the
  $xy$-plane has constant area with value $\pi r^2$. Thus, the total
  volume is $\pi r^2 \times (R - (-R)) = \pi r^2 \times 2R = 2\pi
  r^2R$.

  {\em Performance review}: $5$ out of $12$ got this correct. $3$
  chose (C), $2$ chose (B), $1$ each chose (D) and (E).

  {\em Historical note (last year)}: $3$ out of $14$ people got this
  correct. $5$ people chose (C), $4$ people chose (B), and $2$ people
  chose (E).

  {\em Action point}: Make sure you understand this really really
  well! In particular, make sure you understand why this is {\em not},
  repeat {\em not}, a solid of revolution but rather a {\em cylinder
  with bent spine}.
\end{enumerate}
\end{document}