\documentclass[10pt]{amsart}

%Packages in use
\usepackage{fullpage, hyperref, vipul, enumerate}

%Title details
\title{Class quiz solutions: November 30: Integration}
\author{Math 152, Section 55 (Vipul Naik)}
%List of new commands

\begin{document}
\maketitle

\section{Performance review}

$11$ people took this $7$-question quiz. The score distribution was as follows:

\begin{itemize}
\item Score of $2$: $1$ person
\item Score of $3$: $2$ people
\item Score of $4$: $2$ people
\item Score of $5$: $2$ people
\item Score of $6$: $4$ people
\end{itemize}

The mean score was $4.17$ out of $7$. Here are the problem answers:

\begin{enumerate}
\item (A): $7$ people
\item (B): $8$ people
\item (C): $6$ people
\item (D): $8$ people
\item (A): $10$ people
\item (B): $5$ people
\item (B): $6$ people
\end{enumerate}

\section{Solutions}

\begin{enumerate}
\item What is the limit $\lim_{x \to \infty} \left[\left(\int_0^x
  \sin^2 \theta d\theta\right)/x\right]$?

  \begin{enumerate}[(A)]
  \item $1/2$
  \item $1$
  \item $1/\pi$
  \item $2/\pi$
  \item $1/(2\pi)$
  \end{enumerate}

  {\em Answer}: Option (A)

  {\em Explanation}: The average over a period is $1/2$. Thus, this is
  also the limit of the average over intervals of arbitrarily large
  length. See the notes on the mean value of a periodic function over
  a period.

  {\em Performance review}: $7$ out of $11$ got this correct. $3$
  chose (B) and $1$ chose (D).

  {\em Historical note (last year)}: $13$ out of $16$ people got this
  correct. $1$ person each chose (C) and (E) and $1$ person left the
  question blank.

\item Consider the substitution $u = -1/x$ for the integral $\int
  \frac{dx}{x^2 + 1}$. What is the {\bf new integral}?

  \begin{enumerate}[(A)]
  \item $\int \frac{du}{u(u^2 + 1)}$
  \item $\int \frac{du}{u^2 + 1}$
  \item $\int \frac{ u du}{u^2 + 1}$
  \item $\int \frac{u^2 du}{u^2 + 1}$
  \item $\int \frac{u^2 du}{(u^2 + 1)^2}$ 
  \end{enumerate}

  {\em Answer}: Option (B)

  {\em Explanation}: Setting $u = -1/x$, we get $x = -1/u$, so $dx/du
  = 1/u^2$. Plugging in, we get:

  $$\int \frac{(1/u^2) \, du}{(-1/u)^2 + 1} = \int \frac{du}{1 + u^2}$$

  {\em Performance review}: $8$ out of $11$ got this correct. $2$
  chose (D), $1$ chose (A).

  {\em Historical note (last year)}: $8$ out of $16$ people got this
  correct. $6$ people chose (D), $1$ person chose (C), and $1$ person
  left the question blank.

  {\em Action point}: Those who chose (D) either made a calculation
  error or forgot the relative derivative $1/u^2$ in the
  numerator. Please make sure you review $u$-substitutions and get
  this kind of question correct in the future.
\item {\em Hard}: What is the {\bf value} of $c \in (0,\infty)$ such
  that $\int_0^c \frac{dx}{x^2 + 1} = \lim_{a \to \infty} \int_c^a
  \frac{dx}{x^2 + 1}$?

  \begin{enumerate}[(A)]
  \item $\frac{1}{\sqrt{3}}$
  \item $\frac{1}{\sqrt{2}}$
  \item $1$
  \item $\sqrt{2}$
  \item $\sqrt{3}$
  \end{enumerate}

  {\em Answer}: Option (C)

  {\em Explanation}: By the previous question, we get:

  $$\lim_{a \to \infty} \int_c^a \frac{dx}{x^2 + 1} = \int_{-1/c}^0 \frac{du}{u^2 + 1}$$

  Because the integrand is even, we get that:

  $$\lim_{a \to \infty} \int_c^a \frac{dx}{x^2 + 1} = \int_0^{1/c} \frac{dx}{x^2 + 1}$$

  Thus, from the given data, we get:

  $$\int_0^c \frac{dx}{x^2 + 1} = \int_0^{1/c} \frac{dx}{x^2 + 1}$$

  This gives:

  $$\int_c^{1/c} \frac{dx}{x^2 + 1} = 0$$

  But the function in question is a positive function, hence the only
  way the above can hold is if $c = 1/c$, giving $c = 1$ (since $c$ is
  positive).

  Note that the problem can also be solved using the ``fact'' that
  $\arctan$ is an indefinite integral, so we note that the integral
  from $0$ to $1$, as well as the integral from $1$ to $\infty$, are
  both $\pi/4$. However, that solution requires a knowledge of the
  antiderivative and of the properties of inverse trigonometric
  functions, whereas this proof does not require any development of
  that theory.

  {\em Performance review}: $6$ out of $11$ got this correct. $3$
  chose (E), $1$ chose (B), $1$ chose (D).

  {\em Historical note (last year)}: $8$ out of $16$ people got this
  correct. $6$ people chose (B), for reasons unclear. $1$ person chose
  (A) and $1$ person left the question blank.
\item Suppose $f$ is a continuous nonconstant even function on
  $\R$. Which of the following is {\bf true}?
  
  \begin{enumerate}[(A)]
  \item Every antiderivative of $f$ is an even function.
  \item $f$ has exactly one antiderivative that is an even function.
  \item Every antiderivative of $f$ is an odd function.
  \item $f$ has exactly one antiderivative that is an odd function.
  \item None of the antiderivatives of $f$ is either an even or an odd
    function.
  \end{enumerate}

  {\em Answer}: Option (D)

  {\em Explanation}: The odd function will be the unique
  antiderivative that takes the value $0$ at $0$. Specifically, if $F$
  is an antiderivative of $f$, we can easily check that:

  $$F(x) - F(0) = \int_0^x f(t) \, dt = \int_{-x}^0 f(t) \, dt = F(0) - F(-x)$$

  Thus, we get that:

  $$F(0) = \frac{F(x) + F(-x)}{2}$$

  Thus, $F$ has half-turn symmetry about $(0,F(0))$. It is odd iff
  $F(0) = 0$. We see that, among the family of antiderivatives, there
  is a unique one with the property.

  {\em Additional note}: A little while ago, you proved that a cubic
  function enjoys half-turn symmetry about its point of inflection,
  and were supposed to give a computational proof thereof. The fact
  can actually be deduced without any computation using the fact that
  the derivative function, the quadratic, has {\em mirror symmetry}
  about the same $x$-value. The proof of that fact follows the same
  lines as the proof given above.

  {\em Performance review}: $8$ out of $11$ got this correct. $3$
  chose (C).

  {\em Historical note (last year)}: $4$ out of $16$ people got this
  correct. $9$ people chose (C), apparently forgetting the fact that
  an odd function must be $0$ at $0$. $1$ person chose (A) and $2$
  people chose (B).


\item Suppose $f$ is a continuous nonconstant odd function on
  $\R$. Which of the following is {\bf true}?

  \begin{enumerate}[(A)]
  \item Every antiderivative of $f$ is an even function.
  \item $f$ has exactly one antiderivative that is an even function.
  \item Every antiderivative of $f$ is an odd function.
  \item $f$ has exactly one antiderivative that is an odd function.
  \item None of the antiderivatives of $f$ is either an even or an odd
    function.
  \end{enumerate}

  {\em Answer}: Option (A)

  {\em Explanation}: Fill this in yourself; it is similar to the
  previous exercise. Note that the key difference here is that an even
  function does not have to be $0$ at $0$, and adding a constant
  preserves the property of being even.

  {\em Performance review}: $10$ out of $11$ got this correct. $1$
  chose (E).

  {\em Historical note (last year)}: $13$ out of $16$ people got this
  correct. $1$ person each chose (B), (C), and (D).

\item Suppose $f$ is a continuous nonconstant periodic function on
  $\R$ with period $h$. Which of the following is {\bf true}?

  \begin{enumerate}[(A)]
  \item Every antiderivative of $f$ is a periodic function with period
    $h$, regardless of the choice of $f$.
  \item For some choices of $f$, every antiderivative of $f$ is a
    periodic function; for all others, $f$ has no periodic antiderivative.
  \item $f$ has exactly one periodic antiderivative for every choice of $f$.
  \item For some choices of $f$, $f$ has exactly one periodic
    antiderivative; for all others, $f$ has no periodic
    antiderivative.
  \item Regardless of the choice of $f$, no antiderivative of $f$ can
    be periodic.
  \end{enumerate}

  {\em Answer}: Option (B)

  {\em Explanation}: What this crucially depends on is the mean value
  of $f$ over a period. If this mean value is $0$ (e.g., for $\sin$
  and $\cos$), then every antiderivative is periodic. If the mean
  value is nonzero (e.g., $x \mapsto 1 + \sin x$ or $\sin^2$) then the
  antiderivative is (linear + periodic), and that mean value is the
  slope of the linear component of any antiderivative. For instance,
  $1 + \cos x$ has mean value $1$, and its antiderivative, $x + \sin
  x$, has linear part $x$ of slope $1$ and periodic part $\sin x$.

  {\em Performance review}: $5$ out of $11$ got this correct. $4$
  chose (A), $2$ chose (D).

 {\em Historical note (last year)}: $5$ out of $16$ people
  got this correct. $9$ people chose (A), suggesting that they didn't
  remember the ideas about non-periodic functions with periodic
  derivatives. $2$ people chose (A).
  
  {\em Action point}: Review the material on functions that are
  ``periodic with shift'' -- discussed when we covered graphing of
  functions.
\item Consider a continuous increasing function $f$ defined on the
  nonnegative real numbers. Define $m_f(a)$, for $a > 0$, as the
  unique value $c \in [0,a]$ such that $f(c)$ is the mean value of $f$
  on the interval $[0,a]$.
  
  If $f(x) := x^n$, $n$ an integer greater than $1$, what kind of
  function is $m_f$ (your answer should be valid for all $n$)?

  \begin{enumerate}[(A)]
  \item $m_f(a)$ is a constant $\lambda$ dependent on $n$ but
    independent of $a$.
  \item It is a function of the form $m_f(a) = \lambda a$, where
    $\lambda$ is a constant depending on $n$.
  \item It is a function of the form $m_f(a) = \lambda a^{n-1}$, where
    $\lambda$ is a constant depending on $n$.
  \item It is a function of the form $m_f(a) = \lambda a^n$, where
    $\lambda$ is a constant depending on $n$.
  \item It is a function of the form $m_f(a) = \lambda a^{n+1}$, where
    $\lambda$ is a constant depending on $n$.
  \end{enumerate}

  {\em Answer}: Option (B)

  {\em Explanation}: The integral on the interval $[0,a]$ is
  $a^{n+1}/(n + 1)$. The mean value is $a^n/(n + 1)$. The value $c$ is
  thus $(a^n/(n + 1))^{1/n} = a/(n + 1)^{1/n}$. Setting $\lambda =
  1/(n + 1)^{1/n}$, we see that option (B) works.
  
  {\em Performance review}: $6$ out of $11$ got this correct. $4$
  chose (C), $1$ chose (D).

  {\em Historical note (last year)}: $1$ out of $16$ people got this
  correct. $5$ people chose (D), $4$ people chose (E), $3$ people
  chose (C), $2$ people chose (A), and $1$ person left the question
  blank. Most probably, people forgot the step of raising to the power
  of $1/n$, and of course, many people just guessed.

  {\em Action point}: Make sure that you can solve the problem under
  fewer time constraints.
\end{enumerate}
\end{document}
