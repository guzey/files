\documentclass[a4paper]{amsart}

%Packages in use
\usepackage{fullpage, hyperref, vipul}

%Title details
\title{Duality, manifolds and some algebraic topology}
\author{Vipul Naik}
%\thanks{\copyright Vipul Naik, B.Sc. (Hons) Math and C.S., Chennai Mathematical Institute}

%List of new commands

\newcommand{\field}{\mathbf{k}}
\newcommand{\Quat}{\mathbb{H}}
\makeindex

\begin{document}
\maketitle
%\tableofcontents

\begin{abstract}
  This is a short note intended to explore the applications of duality
  theory to the study of manifolds. I discuss Alexander duality,
  Lefschetz duality and Poincare duality, along with applications to
  the study of compact connected orientable manifolds.
\end{abstract}

\section{Manifolds and points}

\subsection{The core question}

One of the questions we shall be interested in is:

\begin{quote}
  Given two manifolds $M$ and $N$, what are the ways in which $N$
  embeds as a submanifold of $M$? In other words, what are the
  submanifolds of $M$ homeomorphic to $N$?
\end{quote}

Roughly speaking, we want to know how $N$ ``sits inside'' $M$ merely
from the data of what $M$ and $N$ look like abstractly. First, we need
to define what it means for ``ways in which $N$ embeds''.

\begin{definer}[Equivalently embedded subsets]
  Given a topological space $X$ and subspaces $Y_1$ and $Y_2$, we say
  that $Y_1$ and $Y_2$ are \definedind{equivalently embedded
    subsets} if there is a homeomorphism of $X$ under which $Y_1$ maps
  homeomorphically to $Y_2$, or equivalently, there is a homeomorphism
  of the pair $(X,Y_1)$ and $(X,Y_2)$.
\end{definer}

The question we want to ask, more precisely is: what are the various
ways of embedding one manifold inside another,  upto equivalence?

The answer in general could be {\em lots}. Moreover, non-equivalent
embeddings may also look very similar to the algebraic
topologist. Algebraic topology tries to partially solve this problem
by looking at an ``invariant'' of embeddings:

\begin{quote}
  Given a topological space $X$ and a subspace $Y$, describe all
  possibilities for the induced maps of homology:

  $$H_i(X \setminus Y) \to H_i(X)$$

  Given only the datum of what $Y$ looks like as an abstract
  topological space.
\end{quote}

\subsection{Points in manifolds}

There are some special cases where we can prove that just knowing the
abstract homeomorphism type of the manifold and the submanifold,
determines a unique way of embedding the submanifold. More precisely,
there are special cases of manifolds $M$ and $N$ such that any two
embeddings of $N$ in $M$ ar eequivalent. Before beginning this, let's
define a manifold:

\begin{definer}[Manifold]
  A \definedind{manifold} of dimension $n$ is a topological space such that:

  \begin{itemize}
  \item Every point in the manifold is contained in an open set
    homeomorphic to Euclidean space.
  \item The space is Hausdorff
  \item The space has a countable basis of open sets, viz., it is
    second-countable.
  \end{itemize}

  A topological space which satisfies only the first condition is
  called a \definedind{locally Euclidean space}.
\end{definer}

Now the big theorem:

\begin{theorem}[Connected manifolds are homogeneous]
  Let $M$ be a connected manifold. Then any two embeddings of a point
  in $M$ are equivalent. In other words, given any two points $p,q \in
  M$, there is a homeomorphism of $M$ which sends $p$ to $q$.
\end{theorem}

\begin{proof}
  Consider the group of self-homeomorphisms of $M$. This group acts on
  $M$, and $M$ is partitioned into orbits. We will prove that each
  orbit is open. Since $M$ is connected, this will force the action to
  be transitive.

  To prove that each orbit is open, it suffices toshow that every
  point is contained in an open set such that all points in the open
  set are in its orbit.

  Let $p \in M$ be a point. Then there exists a neighbourhood $U$ of
  $p$ which is homeomorphic to $\R^n$ via $f:\R^n \to U$. Let $V$ be
  the image of the interior of the unit disc under $f$, and $D$ be the
  image of the unit disc under $f$. Now, we claim that any point $q
  \in V$ is in the same orbit as $p$. To see this:

  \begin{itemize}

  \item Define a map $g_1:D \to D$ which sends $p$ to $q$, and is
    identity on $D \setminus V$ (this can be done; it's a problem in
    $\R^n$).

  \item Define a map $g_2:M \setminus V \to M \setminus V$ given by
    the identity map.

  \end{itemize}

  Now note that:

  \begin{itemize}

  \item $D$ is homeomorphic to the unit disc, so it is
    compact. Moreover, $M$ is {\em Hausdorff}, so $D$ is closed in
    $M$.

  \item $V$ is open in $U$, and $U$ is open in $M$, so $V$ is open in
    $M$. Hence $M \setminus V$ is closed in $M$.

  \item $g_1$ and $g_2$ agree on $(M \setminus V) \cap D$.

  \end{itemize}

  Thus the gluing lemma allows us to get a map $g:M \to M$ which
  restricts to $g_1$ on $D$ and $g_2$ on $M \setminus V$. $g(p) = q$,
  and we are done.
\end{proof}

The proof crucially relies on the Hausdorffness of $M$. In general,
Hausdorffness is necessary for us to be able to conclude that compact
subsets are closed, and this is something we shall keep needing as we
proceed.

A locally Euclidean space for which the above proposition fails is the
``line with two origins'': the real line with two copies of zero.

The proof does not depend on the second-countability assumption. Many
of the proofs we shall see do not depend on
second-countability. However, second-countability turns out to be
necessary to force stronger separation and metrizability properties.

Another theorem, which can be proved along similar lines:

\begin{theorem}[Multiple transitivity]
  Let $M$ be a connected manifold of dimension greater than $1$, and
  $\{ p_1, p_2, \ldots, p_r \}$ and $\{ q_1, q_2, \ldots, q_r \}$ be
  two sets of points.  Then there is a homeomorphism $\phi$ of $M$
  such that $\phi(p_i) = q_i$ for every $i$.
\end{theorem}

\subsection{Relative homology for a point}

Let's study the point-deletion inclusion problem, viz., the problem of
how the inclusion maps look:

$$H_i(M \setminus p) \to H_i(M)$$

Since (for a connected manifold) any two points in a manifold are
equivalently embedded, the nature of the above map should {\em not}
depend on the choice of point. Let's prove our first theorem.

\begin{theorem}[Homology of pair with point removed]
  Let $M$ be a topological space and $p$ a closed point in $M$, such
  that $p$ is inside an open neighbourhood $U$ homeomorphic to $\R^n$. Then:

  $$H_n(M, M \setminus p) = \Z, \qquad H_i(M, M \setminus p) = 0 \ \forall \ i \ne n$$
\end{theorem}

\begin{proof}
  Note that since $p$ is closed, $M \setminus p$ is open. Excision at
  $p$ yields:

  $$H_i(M, M \setminus p) \cong H_i(U, U \setminus p)$$

  The result holds for $U$, since $U$ is homeomorphic to $\R^n$, and
  hence we are done.
\end{proof}

The proof does not require Hausdorffness. In fact, since locally
Euclidean spaces are $T_1$, the proof works for every point in a
locally Euclidean space.

Also, the proof requires a Euclidean neighbourhood only at the
particular point. Thus, for manifolds with boundary, the result holds
for any point not in the boundary. For CW-complexes, the result holds
for points in an attached $n$-cell, if there is no cell of higher
dimension whose boundary contains that point.

The homology of the pair gives a first foot into the problem we're
after. We write down the long exact sequence of homology of the
pair. We see immediately that for $i \ne n, n-1$, the inclusion:

$$H_i(M \setminus p) \to H_i(M)$$

is an isomorphism. For $i = n,n-1$, we have the picture:

$$0 \to H_n(M \setminus p) \to H_n(M) \to H_n(M,M \setminus p) \to H_{n-1}(M \setminus p) \to H_{n-1}(M) \to 0$$

What happens at $n$ and at $n-1$ depends on the nature of $M$. This is
part of a more general principle: the homology of a pair cares only
about local information, whereas the actual inclusion maps care a bit
about the global structure as well.

\subsection{Some particular cases}

We assume here the following results:

\begin{itemize}

\item If $M$ is any manifold, then $H_i(M) = 0$ for $i$ greater than
  the dimension of $M$.
\item If $M$ is a compact connected orientable $n$-manifold, then the map
  $H_n(M) \to H_n(M, M \setminus p)$ is an isomorphism for all $p$.

\item If $M$ is a compact connected non-orientable $n$-manifold, $H_n(M) =
  0$.

\item All the homology groups of a compact connected orientable
  $n$-manifold are finitely generated.
\end{itemize}

Let's see some consequences of this. First, the fact that higher
homologies of $M$ being zero, also tells us that higher homologies of
$M \setminus p$ are zero.

For a compact connected orientable manifold, we obtain that:

\begin{itemize}

\item $H_n(M \setminus p) = 0$

\item The map from $M \setminus p$ to $M$ induces an isomorphism on
  $(n-1)^{th}$ homology

\end{itemize}

On the other hand, when $M$ is a compact connected non-orientable
manifold, we get:

\begin{itemize}

\item $H_n(M \setminus p) = 0$

\item There is a short exact sequence:

  $$0 \to H_n(M, M \setminus p) \to H_{n-1}(M \setminus p) \to H_{n-1}(M) \to 0$$

\end{itemize}

In sharp contrast to these cases, when $M = \R^n$, then $H_n(M
\setminus p) = 0$ and $H_{n-1}(M \setminus p) = \Z$.

\section{Alexander duality}

\subsection{Statement of Alexander duality}

Alexander duality is a tool for computing the relative homology of the
pair $(M, M \setminus K)$ where $M$ is a connected orientable manifold
and $K$ is a compact subset of $M$. We state the result in two parts:

\begin{itemize}

\item There is a natural isomorphism:

  $$\overline{H}^i(K) \to H_{n-i}(M, M \setminus K)$$

\item When $K$ is a submanifold, or when $K$ is a strong deformation
  retract of an open neighbourhood, we have:

  $$H^i(K) \cong \overline{H}^i(K)$$

\end{itemize}

The precise statement of Alexander duality also gives us the maps,
with a very concrete interpretation of those maps. However, we shall
not be too interested in the specific maps as of now. Our main
objective in looking at Alexander duality is to study the problem of
how submanifolds can be embedded inside orientable manifolds.

Actually, we do not require a global condition of orientability or a
global manifold property for $M$; we only require that $K$ has a
neighbourhood which looks like an orientable manifold. Thus, for
instance, the Alexander duality statement works out for points in
arbitrary manifolds.

Alexander duality does for compact submanifolds what out explicit
excision argument did for points. In fact, applying Alexander duality
for a {\em point} in a manifold gives exactly the answer we got in the
previous subsection.

\subsection{The general problem formulation}

Suppose $M$ is a connected orientable manifold and $K$ is a compact connected
submanifold, and we know $K$ upto homeomorphism, but have no clue
about how $K$ sits inside $M$. We want to compute the maps:

$$H_i(M \setminus K) \to H_i(M)$$

Clearly, if $K$ embeds in $M$ in only one possible way (as happens
when $K$ is a point) then the maps are uniquely determined. However,
there are situations where different possible embeddings of $K$ give
rise to the same maps on homology.

In the coming sections, we shall see what can be said under special
cases where:

\begin{itemize}

\item $M$ is a compact connected orientable manifold

\item $K$ is a sphere

\item $M$ is a sphere

\item $M$ is highly connected, and $K$ has small codimension

\end{itemize}

\subsection{When the manifold is compact connected orientable}

\begin{claimer}
  Suppose $M$ is a compact connected orientable manifold and $K$ is a
  compact connected submanifold of $M$. Then the map:

  $$H_n(M) \to H_n(M, M \setminus K)$$

  is an isomorphism.
\end{claimer}

\begin{proof}
  Let $p \in K$. Then consider:

  $$H_n(M) \to H_n(M, M \setminus K) \to H_n(M, M \setminus p)$$

  Note that $H_n(M, M \setminus K)$ is, by Alexander duality, the same
  as $H^0(K)$, which is $\Z$.

  Thus, all three groups are $\Z$, and the composite map is an
  isomorphism, so both maps are isomorphisms.
\end{proof}

The consequence is that $H_n(M \setminus K) = 0$, and we can start our
homology chase from $n-1$ instead of $n$.

\subsection{When the submanifold is a sphere}

Suppose $M$ is a connected orientable $n$-manifold and $K$ is
homeomorphic to $S^m$. Then Alexander duality tells us that:

$$H_n(M, M \setminus K) = H_{n-m}(M, M \setminus K) = \Z$$

and all other relative homologies are $0$. Thus the inclusion of $M
\setminus K$ in $M$ induces isomorphisms on all homologies except
possibly $n,n-1$, and $n-m,n-m-1$.

If we are also given that $M$ is compact connected orientable, then
$H_n(M \setminus K) = 0$ and we have isomorphisms everywhere else except
possibly at $n-m$ and $n-m-1$.

\subsection{When the manifold is a sphere}

If $M$ itself is a sphere, then all the homologies of $M$ vanish
except the $0^{th}$ and $n^{th}$ homology. Also, $M$ is compact
connected orientable, so what we get is:

\begin{itemize}

\item $H_n(M \setminus K) = 0$

\item $H_i(M, M \setminus K) \cong \tilde{H}_{i-1}(M \setminus K)$ for
  $i \le n-1$.

\end{itemize}

\subsection{When both the manifold and submanifold are spheres}

Combining all the results above, we see that when $M = S^n$ and $K$ is
homeomorphic to $S^m$, then $\tilde{H}_i(M \setminus K)$ is $\Z$ for
$i = n - m - 1$ and $0$ elsewhere.

Thus, we have the surprising result that the Alexander duality method
cannot distinguish between different ways that $S^m$ may sit inside
$S^n$. This might lead to the question: are any two $S^m$s inside
$S^n$ equivalently embedded? The answer is {\em no}: The Alexander
Horned sphere inside $S^3$ is homeomorphic to $S^2$, but it is not
embedded in the same way as an equatorial $S^2$.

There are other interesting consequences. Note that nowhere did we use
that $K$ is actually homeomorphic to $S^m$: all we used was that $K$
is compact and has the same homology groups as $S^m$. Thus, the
results hold for any compact $m$-manifold with the same homology
groups as $S^m$. Such manifolds are called {\em homology spheres}, and
there exists a homology sphere of dimension $3$ which are not
homotopy-equivalent to $S^3$.

\subsection{The circle inside the torus}

Let's now look at a situation where we have multiple ways of embedding
a submanifold inside a manifold. Suppose $M$ is the torus, and $K$ is
homeomorphic to a circle. There are two obviously distinct ways in
which $K$ can be embedded inside $M$:

\begin{itemize}

\item $K$ can be embedded as a circle inside a Euclidean open subset. In
  this case, $M \setminus K$ has two connected components, one of
  which is contractible, and the other is homotopy-equivalent to a
  wedge of two circles.

\item $K$ can be embedded as one of the $S^1$s of $S^1 \times
  S^1$. The complement of $K$ is now a single connected component,
  homotopy-equivalent to a circle.

\end{itemize}

Let's see if Alexander duality gives us both these possibilities, and
if it gives any others. Since $M$ is compact connected orientable, the
homology chase can start from $1$, so we get the following long exact
sequence:

$$0 \to H_1(M \setminus K) \to H_1(M) \to H_1(M, M \setminus K) \to \tilde{H}_0(M\setminus K) \to 0$$

Putting known values, we get:

$$0 \to H_1(M \setminus K) \to \Z \oplus \Z \to \Z \to \tilde{H}_0(M \setminus K) \to 0$$

Some homological algebra yields two possibilities:

\begin{itemize}

\item $H_1(M \setminus K) = \Z \oplus \Z$, with the map from it being
  an isomorphism, and $\tilde{H}_0(M \setminus K) = \Z$.

\item $H_1(M \setminus K) = \Z$, with the map from it being a direct
  factor embedding, and $\tilde{H}_0(M \setminus K) = Z$.

\end{itemize}

These are the two possibilities we have already sketched ``visually''.

\subsection{Connectivity and embeddings}

Here is an interesting result:

\begin{theorem}[Compact connected submanifolds are separating]
  Suppose $M$ is a path-connected simply connected $n$-manifold, and
  $K$ is a compact simply connected submanifold of codimension
  $1$. Then $K$ is a separating submanifold: $M \setminus K$ has
  exactly two connected components. Moreover, $K$ must be orientable.
\end{theorem}

\begin{proof}
  We would like to use Alexander duality, but the problem is that we do
  not know whether the highest cohomology of $K$ is $\Z$. The trick is
  to use Alexander duality with $\Z/2\Z$-coefficients, because every
  manifold is $\Z/2\Z$-orientable. 

  Since $K$ is $\Z/2\Z$-orientable, its top homology is $\Z/2\Z$.
  Also, since $M$ is $\Z/2\Z$-orientable, Alexander duality applies to
  $M$, and we get:

  $$\Z/2\Z \cong H^{n-1}(K;\Z/2\Z) \cong H_1(M, M \setminus K; \Z/2\Z)$$
  
  We now use a similar argument with the pair $(M, M \setminus
  K)$. First note that since $M$ is simply connected, $H_1(M) =
  0$. For first homology, $H_1(M;R) = H_1(M) \otimes R$, so we get
  $H_1(M;\Z/2\Z) = 0$. Also $\tilde{H}_0(M;\Z/2\Z) = 0$ since $M$ is
  connected. Thus the long exact sequence of homology yields an
  isomorphism:

  $$H_1(M, M \setminus K; \Z/2\Z) \to \tilde{H}_0(M \setminus K; \Z/2\Z)$$

  Combining with the previous observation, we see that the reduced
  homology of $M \setminus K$ is $\Z/2\Z$.

  Now the zeroth reduced homology is free on $1$ less than the number
  of connected components; hence, $M \setminus K$ has exactly two
  connected components. But now we can go back and put
  $\Z$-coefficients, and we'll have:

  $$\tilde{H}_0(M \setminus K; \Z) = \Z$$

  Proceeding backwards, we get that the top cohomology of $K$ is $\Z$;
  hence $K$ is orientable. Note that to proceed backwards, we use the
  following facts:

  \begin{itemize}

  \item $H_1(M) = 0$ because $M$ is simply connected

  \item Alexander duality applies, because $M$ is simply connected,
    and hence orientable

  \end{itemize}

\end{proof}

This proves, among other things, results like the Jordan separation
theorem (set $M = \R^n$, $K = S^{n-1}$). Note that the assumption of
$M$ simply connected was crucial; in the previous section, we saw that
when $M$ is not simply connected, then we can embed curves of
codimension $1$ that do not separate $M$.

\section{Homology, cohomology and connected sums}

\subsection{Facts}

This is a list of well-known ``facts'':

\begin{enumerate}

\item If a map from a topological space $X$ to a topological space $Y$
  induces isomorphisms on all homologies for $1 \le i \le r$, then it
  also induces isomorphisms on all cohomologies for $1 \le i \le r$,
  and moreover, it preserves the cup product structure for $i,j,i + j
  \in \oneton{r}$. The first part follows from naturality of the
  universal coefficient theorem; the second part follows from the fact
  that the cup product commutes with homomorphisms, which in turn
  follows from naturality of the Alexander-Whitney map.

\item The cohomology ring of a wedge sum is the direct sum of the
  cohomology rings (preserving gradation) modulo an identification of
  the zeroth cohomology group. One can see this, for instance, from
  the fact that the map from the disjoint union to the wedge sum
  induces isomorphisms on all positive homologies.

\item The cohomology ring of a connected sum of compact connected
  orientable manifolds is the direct sum of the cohomology rings
  (preserving gradation) modulo an identification of the zeroth
  cohomology group, and an identification of the $n^{th}$ cohomology
  group. We shall discuss this in a little more detail.

\end{enumerate}

\subsection{Homology of a connected sum}

For now, let $M_1$ and $M_2$ be connected manifolds.

There are many approaches to computing the homology of a connected
sum; perhaps the most elementary is a Mayer-Vietoris. The upshot of
the initial Mayer-Vietoris computation is that for $i \ne n, n-1$, we
have isomorphisms:

$$H_i(M_1 \sharp M_2) \cong H_i(M_1) \oplus H_i(M_2)$$

To prove this, we use two basic facts:

\begin{itemize}

\item The inclusion of $M_i \setminus p$ in $M_i$ induces isomorphism
  on all homologies except possibly $n, n - 1$.

\item The only place where the intersection of the open sets (which is
  effectively $S^{n-1}$) has nonzero reduced homology is at
  $n-1$. Hence it can affect only $n,n-1$.

\end{itemize}

Again, the behaviour of the connected sum at $n,n-1$ depends on the
nature of the manifolds. When $M_i$ is compact connected orientable,
then the inclusion of the glued $S^{n-1}$ into $M_i$ is nullhomotopic
(since $S^{n-1}$ is assumed to live inside a Euclidean neighbourhood,
and the inclusion factors through that). Hence, we get that the
inclusion of $S^{n-1}$ in the manifold minus a point is also
nullhomotopic, and using this, we see that if both manifolds are
compact connected orientable:

\begin{itemize}

\item The connected sum is also compact connected orientable

\item The $(n-1)^{th}$ homology of the connected sum is the sum of the
  $(n-1)^{th}$ homology of the pieces.

\end{itemize}

It turns out that the $(n-1)^{th}$ homology of the connected sum
remains the sum of the homologies of the pieces, even if {\em one} of
the manifolds is non-orientable; however, the result fails if both are
non-orientable.

\subsection{Maps from the connected sum to the pieces}

Suppose $M$ and $N$ are manifolds. Then there is a map from $M \sharp
N$ to $M$, given by collapsing the whole of $N$ to a point. (The
manifold we get is not $M$ itself, but is homeomorphic to $M$ via
shrinking inside a Euclidean neighbourhood). There is a similar map to
$N$.

These maps have some very nice properties, as we shall see. First, a
little definition:

\begin{definer}[Degree of a map]
  Let $M$ and $N$ be compact connected orientable manifolds, and let
  $[M]$ and $[N]$ be fundamental classes for $M$ and $N$ respectively
  (viz., generators of the top homologies). Then if $f:M \to N$ is a
  continuous map, the degree of $f$ is defined as the unique integer
  $d$ such that $[M]$ maps to $d[N]$.
\end{definer}

Given compact connected orientable manifolds, one question is whether
there exists a map of degree $\pm 1$ between them (a map of degree $1$
can be converted to a map of degree $-1$ by reversing the orientation
on one side). It turns out that the existence of degree one maps poses
fairly strong restrictions; in some sense $M$ must have ``more
complexity'' than $N$ for there to exist a degree $1$ map from $M$ to
$N$.

\begin{theorem}[Degree one maps from connected sum]
  The map $M \sharp N \to M$ obtained by pinching $N$ to a point, is a
  degree one map.
\end{theorem}

\begin{proof}
  Pick a point $p$ anywhere in $M$ outside the part which got pinched.
  Then the inverse image of $p$ in $M \sharp N$ is also one point, and
  the map at the inverse image is a local homeomorphism.  Call the
  inverse image $q$. Consider the following commutative diagram:

  \begin{eqnarray*}
    M \sharp N & \rightarrow & (M \sharp N, M \sharp N \setminus q)\\
    \downarrow & & \downarrow \\
    M & \rightarrow & (M, M \setminus p)
  \end{eqnarray*}

  The map on the right is an orientation-preserving local
  homeomorphism so it induces isomorphism on top homology. The
  horizontal arrows induce isomorphism on top homology, so the left
  vertical arrow should have degree $1$ (if we gave our orientations
  correctly).
\end{proof}

The proof idea generalizes to computing local degrees: given any point
$p$ in the target space, look at $f^{-1}(p)$. If $f^{-1}(p)$ is a
discrete set, we can use an excision argument to see that the total
degree is the sum of the local degrees at each point. If the map is a
local homeomorphism at each point in $f^{-1}(p)$, then the total
degree is the size of the inverse image. This is what happens, for
instance, in the case of covering maps. Incidentally, this also shows
that any map of nonzero degree must be surjective.

A particular case of a degree one map from the connected sum is the
case where we take $M$ to be $S^n$. Then we get a degree one map from
$S^n \sharp N$ to $S^n$, and since $N$ is the same as $S^n \sharp N$,
we get a degree one map from any compact connected orientable manifold
to $S^n$. In fact, we shall see soon that $S^n$ is a terminal object
in the category of compact connected orientable manifolds with degree
one maps.

To get a better understanding of compact connected orientable
manifolds, we need to look a bit at Poincare duality.

\section{Poincare duality}

\subsection{Statement of Poincare duality}

Poincare duality is a special case of Alexander duality when the
manifold $M$ itself is compact, and hence we can set $K = M$. Thus,
Poincare duality tells us that if $M$ is a compact connected
orientable manifold:

$$H^i(M) \cong H_{n-i}(M)$$

Here are some corollaries of Poincare duality:

\begin{enumerate}

\item The top homology and the top cohomology of $M$ are $\Z$.
  This can be proved by taking $i = 0$ and $i = n$ respectively,
  and using the fact that $M$ is connected.

\item The second highest homology group of $M$ is free Abelian.  This
  is because it is isomorphic to the first cohomology group, which is
  free Abelian, since it is obtained as the group of homomorphisms
  from an Abelian group to $\Z$.

\item The Betti numbers of $M$ are symmetric. In other words, the rank
  of the $k^{th}$ homology group equals the rank of the $(n-k)^{th}$
  homology group.

\end{enumerate}

More is true. To see this, we need to look at the Poincare duality map
carefully.

The Poincare duality map is the map from $H^i(M)$ to $H_{n-i}(M)$. The
map sends $a \in H^i(M)$ to $a \cap [M]$, where $[M]$ is a fundamental
class of $M$. This map is unique up to a choice of fundamental class.

To understand the Poincare duality map, we note some facts about cap
products.

\subsection{Cap products}

The cap product is a structure by which homology becomes a module over
cohomology; it is a collection of maps:

$$H^i(X) \otimes H_j(X) \to H_{j-i}(X)$$

The cap product satisfies a ``naturality'' condition, which is
interesting because one of the functors is covariant and the other is
contravariant. Suppose $f:X \to Y$ is a continuous map.

Then $H^*(X)$ admits $H_*(X)$ as a module. Via the induced map from
$H^*(Y)$ to $H^*(X)$, we see that $H_*(X)$ becomes a
$H^*(Y)$-module. $H_*(Y)$ is already a $H^*(Y)$-module, and the
naturality condition is that the natural map from $H_*(X)$ to $H_*(Y)$
is a map of $H^*(Y)$-modules.

We now prove an easy result:

\begin{claimer}
  Consider a topological space obtained as $A \vee B$. Then the cap
  product of any nonzero cohomology class coming from $A$, with any
  nonzero homology class coming from $B$, is $0$.
\end{claimer}

\begin{proof}
  Let $X = B$, $Y = A \vee B$ and $f: X \to Y$ be the
  inclusion. Suppose $h$ is a cohomology class coming from $A$. From
  observations we made about the cohomology ring of a wedge sum, it is
  clear that $h$ gets mapped to $0$ under $f^*$.

  Now consider the action of $H^*(Y)$ on $H_*(X)$. $h$ must act
  trivially, because its image $f^*(h)$ in $H^*(X)$ is $0$. But now
  $f_*$ is a $H^*(Y)$-module map, and hence $h \in H^*(Y)$ acts
  trivially on any element of $H_*(Y)$ coming from $H_*(X)$. This
  proves the claim.
\end{proof}

This immediately rules out ``wedge-type'' structures from being
homotopy-equivalent to compact connected orientable manifolds. For
instance, $S^2 \vee S^6 \vee S^8$ cannot have the homotopy type of a
compact connected orientable manifold, because the cap product of the
second cohomology with the top cohomology is trivial.

\subsection{More from cap products}

Cap products can actually tell us a fair deal about compact connected
orientable manifolds. Let us return to the theme of studying the
degrees of maps between compact connected orientable manifolds. So
far, we have noted the following:

\begin{enumerate}

\item Given two compact connected orientable manifolds, we can compute
  the degree of a map between them via a ``local degree'' computation:
  pick a point, look at the set of its inverse images, and compute the
  degree of the map in neighbourhoods at each of the inverse
  images. This works if the set of inverse images is discrete.

\item In particular, a covering map of finite degree $d$ has degree $d$ as a
  map of compact connected orientable manifolds.

\item The map from a connected sum to either piece obtained by
  pinching the other piece, has degree $1$.

\item Given any manifold, there is a map from the manifold to the
  sphere, of degree $1$.

\end{enumerate}

We are now interested in developing tools whereby one can conclude
that there does {\em not} exist a degree one map between compact
connected orientable manifolds. Cap products give us such a tool.
Suppose $M$ and $N$ are compact connected orientable manifolds with
chosen fundamental classes $[M]$ and $[N]$, and suppose $f:M \to N$ is
a continuous map which sends $[M]$ to $[N]$. Then we get the following
maps:

$$H^i(N) \to H^i(M) \to H_{n-i}(M) \to H_{n-i}(N)$$

Here, the middle map involves capping with the fundamental class of $M$.

Moreover, what we said earlier about the ``naturality'' of cap
products {\em and} the fact that the fundamental class $[M]$ gets
mapped to the fundamental class of $N$, tells us that the composite is
capping with $[N]$. But since the composite is an isomorphism, we see
that the left-most map must be injective and the right-most map must be
surjective. 

Thus, the cohomology of $N$ is ``smaller'' than the cohomology of $M$
and the homology of $M$ is ``bigger'' than the homology of $N$ (both
statements are consistent with each other). We can view degree one
maps as a kind of ``going-down'' in complexity, as far as the amount
of homology is concerned. Some more facts which can be deduced along
similar lines:

\begin{itemize}

\item If $f:M \to N$ is a degree one map of compact connected
  orientable manifolds, then the induced map on the fundamental groups
  is also surjective.

\item Suppose there exists a degree one map from $M$ to $N$ and a
  degree one map from $N$ to $M$. Then the composite of these maps
  induces a surjective map from $H_*(M)$ to $H_*(M)$. Since all the
  homology groups are finitely generated, the surjective map must be
  an isomorphism, and hence the composite induces isomorphisms on the
  homology of $M$. It also induces an isomorphism on the fundamental
  group of $M$, and hence turns out to be a homotopy equivalence from
  $M$ to itself (by Whitehead's theorem). Similar reasoning shows that
  the composite at the $N$-end is also a homotopy-equivalence, so $M$
  and $N$ are homotopy-equivalent.

\item Thus, we can define a partial order on homotopy classes of
  compact connected orientable manifolds, where $M \ge N$ iff there
  exists a degree one map from $M$ to $N$.

\item If $M \ge N$, then each Betti number of $M$ is greater than or
  equal to the corresponding Betti number of $N$.
\end{itemize}

\subsection{The partial order for surfaces}

For compact connected orientable surfaces, we get a total ordering. A
classification theorem tells us that the only compact connected
orientable surfaces are the surfaces of genus $g$, where a surface of
genus $g$ is a connected sum of $g$ tori (the surface of genus $0$ is
the sphere).

Let $M_g$ denote the surface of genus $g$. Then for $g \ge h$, there
exists a degree $1$ map from $M_g$ to $M_h$, because $M_g = M_h \sharp
M_{g-h}$. Moreover, there cannot exist a degree one map from $M_h$ to
$M_g$ if $g > h$, because the first Betti number of $M_g$ is $2g$, and
the first Betti number of $M_h$ is $2h$.

Thus, the partial order we described in the previous subsection yields
a total ordering for surfaces. In fact, more is true. The argument of
the previous subsection generalizes to any ring of coefficients;
hence, we conclude that there does not even exist a map from $M_h$ to
$M_g$ which induces an isomorphism on the top {\em rational} homology
if $g > h$. Equivalently, there does not exist a map of nonzero degree
from $M_h$ to $M_g$ if $g > h$.

\section{Constructing and studying compact connected orientable manifolds}

We now enter the murky waters of compact connected orientable
manifolds in higher dimensions than $2$. While anything resembling a
classification is far beyond the scope here, we would at least like to
get a reasonable number of manifolds and study their properties. Let
us recall the various techniques for constructing compact connected
orientable manifolds:

\begin{enumerate}

\item Connected sums: This is a technique for {\em adding up} the
  homological complexity (as in, adding up the number of holes in each
  dimension). To take a connected sum, we need two manifolds of the
  same dimension, and we would like both of them to be orientable.

\item Product: A product of compact connected orientable manifolds is
  compact connected orientable. The dimensions add up in a product, so
  this can be a tool for constructing higher-dimensional compact
  connected orientable manifolds from lower-dimensional ones.

\item Spheres: The starting point of all homology theory is the
  spheres. These are the ``simplest'' compact connected orientable
  manifolds.

\item Gluing cells: A more complicated construction than taking
  products is to glue cells in non-conventional ways. This is how, for
  instance, complex projective spaces are constructed. That takes us
  into the realm of CW-complexes.

\end{enumerate}

These constructions can be used to construct manifolds that are
non-compact and non-orientable as well (provided we start out with
some that aren't) but we'll restrict attention to the relatively smaller
class of compact connected orientable manifolds.

Our goal will be to relate the geometric, algebraic and topological
effect of each of the operations, and hopefully gain some intuition
into the way each operation works.

\subsection{Products}

There are many ways to understand how products work. Let's first do it
using cell structures (as in CW-complex structures).

Suppose $X$ and $Y$ are (sufficiently nice) topological spaces, each
of which is given a certain cell structure. Then $X \times Y$ also
gets a natural cell structure. What happens is that for each $i$-cell
of $X$ and each $j$-cell of $Y$, we get a corresponding $(i+j)$-cell
of $X \times Y$. A convenient way to compute the number of $n$-cells
in the product is by considering generating functions of the number of
cells in $X$ and the number of cells in $Y$; specifically, define:

$$p_c(X) = \sum_i \alpha_i x^i$$

where $\alpha_i$ is the number of $i$-dimensional cells of $X$. Then we have:

$$p_c(X \times Y) = p_c(X)p_c(Y)$$

The following fact is true (we will defer the proof for later, because
it would take us too far afield):

\begin{quote}
  If all the homologies of a CW-complex are free Abelian, then we can
  construct a homotopy-equivalent CW-complex, where all the boundary
  maps are zero. In fact, we can give the CW-complex a cell structure
  where the boundary of every $k$-cell lies inside the
  $(k-2)$-skeleton (except at $k=1$, of course). With this new
  CW-complex structure, the cellular chain groups are precisely the
  homology groups.
\end{quote}

Thus, a good model for spaces with free homology are spaces where all
the boundary maps go into the $(k-2)$-skeleton. For such a cell
structure, the number of cells in dimension $i$ is precisely equal to
the rank of the $i^{th}$ homology group.

\begin{definer}[Poincare polynomial]
  The Poincare polynomial of a topological space with finitely
  generated homology, is the generating function for the Betti
  numbers.  In other words, it is a polynomial where the coefficient
  of $x^j$ is the Betti number $b_j$.
\end{definer}

For spaces with free homology, the Poincare polynomials multiply.  For
instance, the Poincare polynomial of the sphere $S^n$ is $1 + x^n$,
and the Poincare polynomial for a product of spheres $S^{m_1} \times
S^{m_2} \times S^{m_r}$ is:

$$(1 + x^{m_1})(1 + x^{m_2}) \ldots (1 + x^{m_r})$$

In particular, the Poincare polynomial for $S^1 \times S^1 \times \ldots S^1$
is:

$$(1 + x)^n$$

Hence the $k^{th}$ Betti number is $\binom{n}{k}$.

There is a general theory for what happens if the spaces involved do
not have free homology, but in that case, we need to actually look at
the homology groups, and cannot simply work with Betti numbers. The
formula that we get is termed the ``Kunneth formula'', which involves
taking tensor products of the homology groups and adding. The ``tensor
product'' operation is analogous to multiplication of monomials and
the ``adding up'' is analogous to adding up monomials to get a
polynomial.

Note that the Euler characteristic is the value taken by the Poincare
polynomial at $-1$. This shows us immediately that for a product of
circles, the Euler characteristic is always zero. On the other hand,
for a product of arbitrary spheres, the Euler characteristic is zero
iff at least one of the spheres is odd-dimensional.

As we proceed, we shall see an interesting even-versus-odd bifurcation
in properties.

Let's also describe the cohomology ring of a product. The cohomology
ring of a product is a graded tensor product of the cohomology rings
of each of the pieces. In the particular case where the pieces are $S^{m_1},S^{m_2},\ldots,S^{m_r}$, the cohomology ring looks like:

$$\Z\langle e_1,e_2,\ldots,e_r\rangle/\langle \langle e_ie_j - (-1)^{m_im_j}e_je_i, e_i^2 \rangle \rangle$$

We're essentially taking the associative algebra generated by the
cohomology rings, modulo the commutation relations that need to be
satisfied because of degree constraints. Each $e_i$ lives in degree $m_i$,
and is the generator for the cohomology that ``comes from'' the projection
onto the sphere $S^{m_i}$.

Some further points:

\begin{itemize}

\item For a product of spheres, whenever $m_i$ is even, $e_i$ is in
  the center. Thus, if the product is only of even-dimensional
  spheres, then we get a quotient of the polynomial ring in $r$
  variables, by the ideal generated by the squares of the variables:

  $$\Z\langle e_1, e_2, \ldots, e_r \rangle/\langle \langle e_i^2 \rangle \rangle$$

\item If all the $m_i$s are odd, then we get an exterior algebra on
  the $e_i$s -- an algebra generated by anticommuting elements, namely:

  $$\Z \langle e_1, e_2, \ldots, e_r \rangle/\langle \langle e_i^2, e_ie_j + e_je_i \rangle \rangle$$

  The $e_i$ here lives in degree $m_i$.

\end{itemize}
 
\subsection{Connected sum}

When we take a connected sum, we add up the homologies in each of the
dimensions, except in the {\em top} dimension, where the fundamental
classes coming from each piece get identified.

Intuitively, we are ``adding up'' the number of holes in all
dimensions except the top dimension. The reason why the number of
holes in the top dimension doesn't get added up is because we are
gluing along a $S^{n-1}$.

Thus a connected sum is a good way to ``add up'' the complexity of two
manifolds.

One point deserves mention here. When we take a connected sum of two
manifolds, there are two possible ways of gluing together the sphere
of one manifold with the sphere of the other. These two ways could
give manifolds which are not even homotopy-equivalent.

This non-equivalence is reflected in the structure of the cohomology
ring.  The choice of how to glue together the $S^{n-1}$s is related to
the choice of how to identify the fundamental class coming from one
piece, with the fundamental class coming from the other piece. For
certain kinds of manifolds, both choices give the same ring, whereas
for certain choices of manifolds, we could get two non-isomorphic ring.

Let's first see what's happening for products of spheres, in which
case connected sum {\em is} unique -- this is essentially because
spheres possess orientation-reversing homeomorphisms. Consider the connected sum of $S^2 \times S^2$ and $S^1 \times S^3$.

The cohomology ring of $S^2 \times S^2$ looks like:

$$\Z[e_1,e_2]/(e_1^2,e_2^2) $$

Here both $e_1$ and $e_2$ live in degree $2$.

The cohomology ring of $S^3 \times S^1$ looks like:

$$\Z\langle e_3,e_4\rangle/\langle \langle e_3^2,e_4^2, e_3e_4 + e_4e_3 \rangle \rangle$$

When we take a connected sum, we need to ensure that a product of
anything from one piece with anything from the other piece is zero; we
also need to identity the top cohomology classes. The end result is this:

$$\Z \langle e_1, e_2, e_3, e_4 \rangle/I$$

where $I$ is the two-sided ideal generated by:

$$e_1^2, e_2^2, e_3^2, e_4^2, e_1e_2 - e_2e_1, e_3e_4 + e_4e_3, e_1e_3, e_2e_3, e_1e_4, e_2e_4, e_1e_2 - e_3e_4$$

The last relation identifies the top cohomology classes in both pieces.

Taking products is like multiplying complexity; taking connected sums
is like adding complexity within the same dimension. We now consider
the third construction, which is all about gluing things in a way that
cannot be ``decomposed''.

\subsection{The projective spaces}

Let $\field$ be one of $\R$, $\C$ or $\Quat$. Then one can talk of
$\field^{n+1}$ as the standard $n+1$-dimensional vector space over
$\field$ (for the case of $\Quat$, one must talk of ``left vector
space'').  Inside $\field^{n+1}$, one can look at the ``sphere'' $S^n(\field)$,
which is the set of elements $(z_1, z_2, \ldots, z_{n+1})$ such that:

$$\sum_{i=1}^n |z_i|^2 = 1$$

The following are true:

\begin{itemize}

\item By treating $\R$, $\C$ and $\Quat$ as $\R^1$, $\R^2$ and $\R^4$,
  we can identify $S^n(\R)$ with $S^n$, $S^n(\C)$ with $S^{2n+1}$, and
  $S^n(\Quat)$ with $S^{4n+3}$.

\item $S^0(\field)$ is the group of units in $\field$ which have norm
  $1$. $S^0(\R)$ is the group $S^0 = \Z/2\Z$, $S^0(\C)$ is the group $S^1$,
  and $S^0(\Quat)$ is the group $S^3$ of unit quaternions.

\item There is a natural action of $S^0(\field)$ on $S^n(\field)$ by
  coordinate-wise multiplication, and the quotient of $S^n(\field)$ by
  this action is the so-called ``projective space'' $\field \mathbb{P}^n$.
  In other words, we get fiber bundles:

  \begin{eqnarray*}
    S^0 & \to S^n & \to \R\mathbb{P}^n\\
    S^1 & \to S^{2n+1} & \to \C\mathbb{P}^n\\
    S^3 & \to S^{4n+3} & \to \Quat \mathbb{P}^n
  \end{eqnarray*}

\item The quotient $\field\mathbb{P}^1$ is the one-point
  compactification of $\field$, which is a sphere: $\R\mathbb{P}^1 =
  S^1$, $\C\mathbb{P}^1 = S^2$ and $\Quat \mathbb{P}^1 = S^4$. We thus
  get three fiber bundles of the form:

  \begin{eqnarray*}
    S^0 & \to S^1 & \to S^1 \\
    S^1 & \to S^3 & \to S^2 \\
    S^3 & \to S^7 & \to S^4
  \end{eqnarray*}

\end{itemize}

Equipped with this knowledge, let us try to realize projective spaces
over $\R$, $\C$ and $\Quat$ in terms of CW-complexes. Explicitly:


\begin{itemize}

\item The projective space $\field \mathbb{P}^n$ is constructed from
  $\field \mathbb{P}^{n-1}$ via the attaching map $S^n(\field) \to
  \field\mathbb{P}^{n-1}$.

\item For real projective space, we are attaching a $n$-cell to the
  $(n-1)$-skeleton. For complex projective space, we are attaching a
  $2n+2$-cell to the $2n$-skeleton, and for Hamiltonian projective
  space, we are attaching a $4n + 4$-cell to the $4n$-skeleton.

\item Thus, for complex projective space, the cellular chain complex
  has homology group $\Z$ in even dimensions upto $2n$ and $0$ in odd
  dimensions. For Hamiltonian projective space, the cellular chain
  complex has homology group $\Z$ in dimensions $4,8,\ldots,4n$ and
  $0$ elsewhere. Real projective spaces require a bit more of
  computation for homology which we won't do.

\item All projective spaces are compact. This is because they are
  quotients of spheres, which are already compact.

\item In fact, complex and Hamiltonian projective spaces are simply
  connected, because they have no $1$-cells. Thus, they are
  orientable. This gives us some genuinely new examples of compact
  connected orientable manifolds.

\item Real projective spaces are orientable iff the dimension is odd.
  One way of seeing this is that the real projective space is obtained
  by quotienting the sphere by the antipodal identification; and the
  antipodal identification is orientation-preserving iff the dimension
  is odd.

\item The cellular structure of projective spaces is pretty
  rigid: each new cell that we are adding has its boundary mapping
  surjectively to the previous skeleton. We shall see that this is
  closely related to the fact that in the cohomology ring, the lowest
  cohomology generates all the higher cohomologies.

\end{itemize}

So far, we have discussed the operation of taking products and
connected sums. Products had more complexity, but the complexity was
multiplicatively decomposed; connected sums had complexity, but the
complexity was additively decomposed. However, constructions like
those of the complex projective space are {\em indecomposable} because
each layer of complexity is built into the previous one.

Let's look at the cohomology rings. The cohomology ring of
$\C\mathbb{P}^n$ is $\Z[x]/(x^{n+1})$ where $x$ is a generator of the
second cohomology. In other words, generators of all the higher
cohomology groups, are obtained by taking cup powers of
$\C\mathbb{P}^n$. Some remarks:

\begin{itemize}

\item Any map from $\C\mathbb{P}^n$ to itself, which is multiplication
  by $d$ on the second cohomology, is multiplication by $d^k$ on the
  $2k^{th}$ cohomology. Hence its Lefschetz number is $1 + d + d^2 +
  \ldots + d^n$ and its degree (as a map of compact connected
  orientable manifolds) is $d^k$.

\item If a map from $\C\mathbb{P}^n$ to itself takes the $2$-skeleton
  to itself, then the effect of the map on the cohomology of the
  $2$-skeleton, determines the effect of the map on the whole
  cohomology ring. In other words, if it is multiplication by $d$ on
  the cohomology ring of the $2$-skeleton, then it is multiplication
  by $d^k$ on the $2k^{th}$ cohomology.

\item The first observation tells us that $\C\mathbb{P}^n$ can possess
  an orientation-reversing homeomorphism only if $n$ is odd. Also,
  combined with the Lefschetz fixed-point theorem, it tells us that
  $\C\mathbb{P}^n$ has the fixed-point property if $n$ is even. That
  the converses of these hold requires explicit constructions of maps.

\end{itemize}

These two facts may seem surprising, but what's going on is that what
happens on the ``outermost'' part determines more or less what happens
everywhere, as far as the cohomology ring is concerned.

We see that looking at the cohomology ring, and the cup product
structure, already gives us significantly more constraints than merely
looking at the effect on the homology groups individually.

\section{Manifolds and maps}

In this section we study the problem of manifolds, self-maps of
manifolds, and maps between manifolds. 

Some of the questions we shall be concerned with:

\begin{itemize}

\item Given a topological space, what do its homology groups and
  cohomology ring look like?

\item Given a topological space, what are the possible graded ring
  endomorphisms of its cohomology ring? Which of these can be realized
  by a continuous map, and to what extent does the effect of the map
  on the cohomology ring, determine its homotopy type?

\item Similar questions, but for maps between different topological
  spaces.

\end{itemize}

The typical approach we will use is to restrict our attention to what
the map does in the dimensions which generate the cohomology ring. For
instance, if the cohomology ring is, as a ring, generated by the first
cohomology group, then we shall restrict attention to possible
endomorphisms of the first cohomology group, and ask which such
endomorphisms extend to ring endomorphisms.

\subsection{A review of compact connected orientable surfaces}

Essentially there was only one kind of compact connected orientable
surface: the surface of genus $g$. The cohomology ring of this surface
in general looks like:

$$\Z \langle x_1,y_1,\ldots,x_g,y_g \rangle/ \langle \langle x_iy_i + y_ix_i, x_i^2, y_i^2, x_ix_j, x_iy_j, y_iy_j, x_iy_i - x_jy_j \rangle \rangle$$

All the $x_i$ and $y_i$ live in degree $1$.

Any self-map of a topological space induces an endomorphism of the
cohomology ring as a graded ring. Some thought reveals that the
endomorphisms of this graded ring correspond to maps which preserve an
alternating nondegenerate bilinear form on $\Z^{2n}$. Here the
$(x_i,y_i)$ are paired together under the bilinear form.

\subsection{The product of circles}

For the product of many copies of the circle, the corresponding
cohomology algebra is the exterior algebra in the generators for each
circle, and once again, any map is determined by its effect on first
cohomology. In this case, it turns out that {\em any} $\Z$-module map
on the first cohomology, can be achieved via a continuous map from the
product of circles to itself.

\subsection{Product of even-dimensional spheres}

For $S^2 \times S^2$, the cohomology ring looks like:

$$\Z[e_1,e_2]/(e_1^2,e_2^2)$$

Something interesting happens here: not every $\Z$-module map of
second cohomology extends to a continuous map from $S^2 \times S^2$ to
itself. The constraint is that since $e_1$ and $e_2$ are nilpotent of
order $2$, they must get sent to nilpotents of order $2$, and the only
nilpotents of order $2$ are multiples of $e_1$ and $e_2$. Thus, the
map must send each one to a multiple of one of them.

The crucial difference with the odd-dimensional case is that the
generators here {\em commute}, and thus not every element's square is
zero. When they {\em anti-commute}, we could deduce that the square of
any combination of them is zero.


\subsection{Some more compact connected orientable manifolds}

We have seen most of the basic operations that go into constructing
compact connected orientable manifolds; it's now time to put some of
these operations together and see what are some possibilities for
$3$-manifolds:

\begin{enumerate}

\item There's $S^3$ (which has the minimum possible homology, and
  trivial fundamental group, so it's the ``smallest'' compact
  connected orientable manifold). The cohomology ring of $S^3$ is
  $\Z[x]/x^2$ where $x$ is in degree $3$.

  The only endomorphisms of this ring are those which take $x$ to
  $\lambda x$: each of these can be realized by an explicit continuous
  map; in fact, we can construct the map inductively using the fact
  that $S^n$ is the suspension of $S^{n-1}$.

  $S^3$ is, as a topological group, the same as $SU(2)$. It is also
  the unit quaternion group, and is also the spin group $Sp(3)$.

\item There's $S^1 \times S^2$. The cohomology ring is
  $\Z[x_1,x_2]/(x_1^2, x_2^2)$. As one can see from the cohomology
  ring, the top homology (and cohomology) are controlled by what
  happens in the first and second cohomology. Moreover, the first
  cohomology must go to itself, and the second cohomology must go to itself.

  Any pair of $\Z$-module maps separately on the first and second
  cohomology can be realized by a continuous map.

\item There's $S^1 \times S^1 \times S^1$, which is a product of
  circles. The cohomology ring is an exterior algebra in three
  variables, and any linear map on the first cohomology can be
  realized by a continuous map.

\item There's $\R\mathbb{P}^3$, which does {\em not} have free
  homology and is not simply connected. $\R\mathbb{P}^3$ is the
  topological group $SO(3)$, and its double cover $S^3$ is also its
  universal cover. We shall not concern ourselves with the study of
  $\R\mathbb{P}^3$ for now.
\end{enumerate}

What I've listed above are some examples of {\em prime} manifolds:
manifolds which cannot be expressed as connected sums of two
``smaller'' manifolds. In the partial ordering between them, $S^3$ is
at the bottom. Slightly higher is $S^2 \times S^1$, and $S^1 \times
S^1 \times S^1$ is even higher (note that a distributivity law holds
between connected sums and products, so the connected sum of $S^2
\times S^1$ and $S^1 \times S^1 \times S^1$ is just $S^1 \times S^1
\times S^1$).

Incomparable with all of these is $\R\mathbb{P}^3$ (this can be seen
from criteria of surjectivity on homology for degree one maps).

There are other examples of compact connected orientable manifolds,
many of which arise by taking quotients of $S^3$ by finite subgroups
of itself. Some of these ar equotients of
$\R\mathbb{P}^3$ by finite subgroups of that, viz. quotients of
$SO(3)$ by finite subgroups thereof. The classification of finite
subgroups of $SO(3)$ gives a reasonable list of compact connected
orientable manifolds (all of them are orientable because action of a
group on itself by left multiplication is orientation-preserving).\footnote{In fact, all the manifolds have trivial tangent bundle} The
quotient of $SO(3)$ by the action of $A_5$ is also the quotient of
$S^3$ by the action of the group $\Gamma(2,3,5) = SL(2,5)$.

\begin{theorem}[Homology sphere]
  The quotient of $S^3$ by the action of $SL(2,5) = \Gamma(2,3,5)$,
  which is the double cover of $A_5 \subset SO(3)$, is a homology sphere.
\end{theorem}

\begin{proof}
  Note that the quotient is compact connected orientable.

  The fundamental group of the quotient is $SL(2,5)$. Since $SL(2,5)$
  equals its own commutator subgroup, the first homology of the
  quotient is zero. Hence, its first cohomology is $0$. Poincare
  duality tells us that the second homology is $0$, so the quotient is
  a homology sphere.
\end{proof}

The famous lens spaces are also achieved as finite quotients of $S^3$
by the action of certain subgroups.
\section{A return to embedding problems}

Let's return to the original problem that we were considering: given
an abstract manifold $M$ (say, connected and orientable) and an
abstract compact connected manifold $K$, what are the possible ways
(upto equivalence) in which we can embed $K$ in $M$?

Here are some constraints we have derived (or which are ``obvious''):

\begin{itemize}

\item The dimension of $K$ has to be strictly less than that of $M$
  for there to exist any embedding. This follows from Alexander
  duality, or from more direct arguments.

\item If $K$ is one-dimensional (viz., a point) then there is only one
  embedding of $K$ in $M$ (upto equivalence). The induced maps on
  homology by the inclusion of the complement at $n,n-1$ depend upon
  the nature of $M$. Elsewhere, they are isomorphisms.

\item If $K$ has codimension one, and $M$ is {\em simply connected},
  then $K$ must be orientable. Further, $K$ must separate $M$.

\item If $K$ is a sphere of dimension $m$, the only possibilities for
  homology of the complement are at $n,n-1,n-m,n-m-1$. If $M$ is also
  compact, connected and orientable, we have homology only at $n-m$
  and $n-m-1$.

\item If $M$ is a sphere, the $(n-i-1)^{th}$ homology of the complement
  of $K$ equals the $i^{th}$ cohomology of $K$, for a suitable range
  of $i$.

\end{itemize}

Let us now concentrate on a very specific class of embedding problems:
what are the ways in which a compact connected $n$-manifold embeds
inside $\R^{n+1}$? Note that this is equivalent to studying embeddings
inside $S^{n+1}$, since any $n$-manifold in $S^{n+1}$ must miss a
point.

Since $\R^{n+1}$ (or $S^{n+1}$) is simply connected, we see that any
submanifold of dimension $n$ must be orientable, and must separate the
space into two pieces. Thus, manifolds like $\R\mathbb{P}^2$ or the
Klein bottle, which are non-orientable, cannot be embedded in the real
world.

Further, we have that if $K$ is the submanifold:

$$H_{n-i-1}(S^{n+1} \setminus K) \cong H^i(K)$$

Thus, even if there are many inequivalent embeddings, the homology
groups of the complement are determined uniquely by the homology
groups of the manifold. For instance, if we try to embed compact
connected orientable surfaces inside $S^3$, then the above formula
yields:

$$H_1(S^3 \setminus K) \cong H^1(K) = H_1(K)$$

This is the ``intuitive'' statement that the number of one-dimensional
holes in a surface of genus $g$ is the number of one-dimensional holes
in its complement.

However, homology with $\Z$-coefficients seems to have limitations:
while it seems to yield the homology groups readily, it does not
impose any further constraints on which manifolds can be
embedded. Recall that the proof that any codimension one submanifold
must be separating, and hence orientable, involved a mix of $\Z$ and
$\Z/2\Z$-coefficients. Thus, it is sometimes helpful to look at other
coefficient rings to get more constraints on which submanifolds can be
embedded, and how.

Thus, we are currently unable to answer questions like:

\begin{itemize}

\item Can real projective space of odd dimension $n$ be embedded in
  $S^{n+1}$(the answer is {\em no}, for $n > 1$)?

\item Can complex projective space be embedded inside a sphere, with
  codimension $1$ (the answer is again {\em no})?

\end{itemize}

\section{Special kinds of compact connected orientable manifolds}

\subsection{Parallelizable manifolds}

\begin{definer}[Parallelizable manifold]
  A differential manifold is termed
  \adefinedproperty{manifold}{parallelizable} if its tangent bundle is
  trivial; viz, it can be given a global coordinate system.
\end{definer}

Parallelizability is a condition that makes sense only for manifolds
with a differential structure; however, it implies orientability,
which makes sense for a manifold even without giving it a differential
structure. All the manifolds that we have constructed do have natural
differential structures, so the question of parallelizability makes
sense. It turns out that the following are true:

\begin{itemize}

\item Any Lie group is parallelizable (we shall see Lie groups a
  little more in coming sections)

\item A product of parallelizable manifolds is parallelizable

\item Any quotient of a connected Lie group by the action of a finite
  subgroup by left multiplication is again parallelizable

\end{itemize}

Since $S^1$ and $S^3$ are Lie groups, we get a wide range of
parallelizable manifolds: all products of $S^1$s and $S^3$, as well as
all quotients of $S^3$ by finite subgroups, which includes $SO(3)$,
the homology $3$-sphere, and the lens spaces.

\subsection{Lie groups}

\begin{definer}[Lie group]
  A \definedind{Lie group} is a manifold with a compatible group structure.  It
  turns out that any Lie group can be equipped with a differential
  structure such that the group multiplication is smooth for that
  structure (this is far from obvious).
\end{definer}

The remarkable thing about Lie groups, as opposed to arbitrary
manifolds, is that they are much more homogeneous, and they have a lot
more of symmetry. Here are some applications:

\begin{theorem}[Euler characteristic of Lie group]
  Any compact connected nontrivial Lie group, has zero Euler characteristic.
\end{theorem}

We assume here the fact that any compact connected nontrivial Lie
group admits a simplicial complex structure.

\begin{proof}
  Let $G$ be the group, and $e$ its identity element. Pick $g \ne
  e$. Then left multiplication by $g$ is in the same homotopy class as
  the identity map (a homotopy is given by a path from $e$ to $g$).
  Also, left multiplication by $g$ has no fixed points. Hence, the
  identity map is homotopy-equivalent to a map which has no fixed
  points, and thus the Lefschetz number of the identity map is zero.

  (Here we are applying Lefschetz fixed-point theorem, and using the
  fact that any compact connected Lie group admits a triangulation)

  The Euler characteristic is the same as the Lefschetz number of the
  identity map; thus we have proved that the Lefschetz number of the
  identity map is zero.
\end{proof}

Also, another result:

\begin{theorem}[Lie groups are parallelizable]
  Any Lie group is parallelizable.
\end{theorem}

Here we are using the definition of a Lie group as a differential
manifold with a compatible group structure. The proof uses the fact
that left multiplication by group elements can be used to translate a
basis of the tangent space at the identity, to a basis at all points.

Thus, compact connected Lie groups are examples of compact connected
orientable manifolds with Euler characteristic zero. The Euler
characteristic does not impose additional constraints for {\em
  odd}-dimensional manifolds, because for a compact connected
orientable manifold, the Euler characteristic is anyway zero (the
Betti numbers match up by Poincare duality). On the other hand, we do
get a restriction on the Betti numbers for even-dimensional manifolds.
For instance, we see trivially that even-dimensional spheres, and
even-dimensional real projective spaces, cannot be Lie groups.

Also, complex and quaternionic (Hamiltonian) projective spaces cannot
be Lie groups.


Here is yet another fact about Lie groups:

\begin{theorem}
  A compact connected nontrivial Lie group possesses an
  orientation-reversing homeomorphism. Further, if there are precisely
  $d$ $d^{th}$ roots of unity, then the map $x \mapsto x^d$ induces
  multiplication by $d$ on the fundamental class.
\end{theorem}

Thus compact connected (nontrivial) Lie groups are parallelizable
(hence orientable), have Euler characteristic zero, and possess
orientation-reversing homeomorphisms.

\subsection{Compact simply connected manifolds}

Another important class of compact connected orientable manifolds is
ithose which are {\em simply connected}. For instance, any manifold
which admits a cell structure with no one-dimensional cells is simply
connected. For instance, complex and quaternionic projective spaces,
spheres of dimension at least $2$, products of spheres where each has
dimension at least $2$, and so on.

Given a compact connected orientable manifold with finite fundamental
group, we can pass to the universal cover of the manifold -- that
again is compact, connected and orientable. Conversely, starting with
a compact simply connected manifold, we may be interested in what
manifolds arise as quotients of this manifold by properly
discontinuous actions. The quotient manifold is orientable iff the action
is orientation-preserving.

For instance, for an odd-dimensional sphere, the antipodal map is
orientation-preserving, so the quotient, which is real projective
space, is orientable. For even-dimensional spheres the quotient is
non-orientable.

Thus, when trying to classify all compact connected orientable manifolds
with {\em finite} fundamental group, we can proceed as follows:

\begin{itemize}

\item First, classify all compact simply connected manifolds

\item For each compact simply connected manifold, study the possible
  properly discontinuous group actions on it, which preserve
  orientation. The quotients give new compact connected orientable manifolds

\end{itemize}

For odd-dimensional spheres, the antipodal map is
orientation-preserving, while for even-dimensional spheres, it is
orientation-reversing. The Lefschetz fixed-point theorem tells us
something better: there does not exist a fixed-point free map from the
even-dimensional sphere to itself, that is orientation-preserving.

\begin{theorem}
  Any orientation-preserving self-homeomorphism of an even-dimensional
  sphere, must have a fixed point.
\end{theorem}

\begin{proof}
  Since the map is orientation-preserving, the trace on top dimension
  is $1$, so the Lefschetz number is $2$. Applying the Lefschetz
  fixed-point theorem, we see that there is a fixed point.
\end{proof}

In some sense, all the manifolds which have the same universal cover
could be studied ``together'' because they share common properties
(for instance, their higher homotopy groups are all equal). However,
they may be very different homologically.

\subsection{Compact connected orientables: quick review}

We have seen that for compact connected orientable manifolds, there
are the following properties of importance. Most of the properties
listed here are properties that embody greater transitivity,
homogeneity, and freedom of motion. The opposites of these (like
having the fixed-point property, not possessing an
orientation-reversing homeomorphism, being even-dimensional) embody
greater rigidity and inflexibility:

\begin{itemize}

\item Being a compact connected (nontrivial) Lie group

\item Being parallelizable

\item Having Euler characteristic zero

\item Being odd-dimensional

\item Having Euler characteristic zero

\item Possessing fixed-point-free self-maps

\item Possessing orientation-reversing homeomorphisms

\item Possessing self-maps of every possible degree

\item Being simply connected

\end{itemize}

Here are the implications among them:

\begin{itemize}

\item Compact connected Lie group $\implies$ possesses
  fixed-point-free self-maps (in fact possesses fixed-point-free
  self-maps homotopic to the identity)

\item Compact connected Lie group $\implies$ possesses
  orientation-reversing homeomorphisms

\item Compact connected Lie group $\implies$ has Euler characteristic zero

\item Compact connected Lie group $\implies$ parallelizable

\item Odd-dimensional $\implies$ has Euler characteristic zero

\item Quotient of compact connected Lie group by finite subgroup
  $\implies$ parallelizable

\end{itemize}

On the rigid side, we have spaces like $S^{2n}$, which does not
possess a fixed-point-free orientation-preserving homeomorphism, and
whose Euler characteristic is not zero. It is also far from
parallelizable. The quotient of $S^{2n}$ by the antipodal map is not
even orientable.

Even more rigid than $S^{2n}$ are complex projective spaces of {\em
  even} complex dimension. Although orientable, these spaces do not
possess orientation-reversing homeomorphisms, nor do they possess
fixed-point-free self-maps. Note that odd-dimensional complex
projective spaces do possess orientation-reversing homeomorphisms and
also possess fixed-point-free self-maps; in fact they possess maps
which are both fixed-point-free and orientation-reversing.

\section{Constructing manifolds upto homotopy}

\subsection{A quick sketch for 3-manifolds}

Assuming that every compact connected orientable manifold can be given
a CW-complex structure, we would like to classify all compact
connected orientable manifolds upto homotopy using what we know of the
theory of CW-complexes. Let us first concentrate on the problem of
classifying compact {\em simply connected} manifolds, for which we
already have a headstart for applying Hurewicz's theorem. The idea:

\begin{itemize}

\item Classify all compact simply connected manifolds upto homotopy

\item Using the classification upto homotopy, find all compact simply
  connected manifolds upto homeomorphism

\item Study all possible properly discontinuous orientation-preserving
  group actions on these, and thus find all compact connected
  orientable manifolds with finite fundamental group.

\end{itemize}

For a compact simply connected manifold $M$, $H_1(M) = H^1(M) = 0$ and
thus, by Poincare duality, $H_{n-1}(M) = H^{n-1}(M) = 0$. We've almost
completed the classification of compact simply connected manifolds
upto homotopy:

\begin{theorem}
  Any compact simply connected $3$-manifold is homotopy-equivalent to
  the $3$-sphere.
\end{theorem}

\begin{proof}
  Let $M$ be a compact simply connected $3$-manifold. $M$ is
  orientable, and above remarks show that $H_0(M) = H_3(M) = \Z$ and
  all $H_1(M) = H_2(M) = 0$. Since $M$ is simply connected, Hurewicz's
  theorem tells us that $\pi_3(M) = \Z$. Pick a generator for
  $\pi_3(M)$, and consider a representative map $(S^3,p) \to
  (M,x)$. This map induces isomorphism on all homologies, and by a
  combination of Whitehead's and Hurewicz's theorem, is a homotopy
  equivalence.
\end{proof}

It is acutally true that any compact simply connected $3$-manifold is
homeomorphic to the $3$-sphere, but the proof of this is well beyond
the scope of this write-up (it has everything to do with the
``Poincare conjecture'').

We also get compact connected orientable manifolds with finite
fundamental group as quotients of $S^3$ by finite subgroups. It is not
clear if all the properly discontinuous actions are actually realized
by subgroups of $S^3$, but at any rate we get a reasonable number of
compact connected orientable manifolds with finite fundamental group.

In contrast, compact connected orientable manifolds like $S^1 \times
S^1 \times S^1$ and $S^1 \times S^2$ elude this classification because
they have infinite fundamental groups.

\subsection{Generalities}

Given a simply connected topological space, we can construct a
CW-complex with a map from the CW-complex to the topological space,
which induces isomorphism on all homologies, and is hence a weak
homotopy equivalence by Whitehead's plus Hurewicz's. Suppose $M$ is
the starting topological space. The idea is to, at the $k^{th}$ stage,
choose the attaching maps in such a way as to ``kill'' the relative
homotop between $M$ and the $k$-skeleton (precisely, we need to take
mapping cylinders at each stage).

The construction involves keeping track of two things:

\begin{itemize}

\item We need to, at each stage, kill the homology at that stage.

\item If the homologies are not free Abelian, then we cannot achieve
  the required $k^{th}$ homology in the $k$-skeleton itself -- we need
  to go one skeleton further.

\end{itemize}

When all the homologies are free Abelian; we do not have the second
headache -- essentially, there is no interaction between the $k$-cells
and the $k-1$-cells (the boundary of the $k$-cells can be chosen
inside the $k-2$-cells). When all the nonzero homology groups are
spaced at a distance of at least $2$, we again have no problem,
because at each stage we are either killing the ``current'' homology,
or destroying the ``carry'' from the previous stage. However, when we
have torsion {\em as well as} homology groups in adjacent dimensions,
we have to choose attaching maps to take care of both factors.

In all the examples we have seen so far, we do not simultaneously have
both headaches. For spheres and complex projective spaces, we have
free Abelian homology. In contrast, for real projective spaces, we
have torsion in the homology but we have empty homologies in between
which we can use to destroy the ``carries''.

\subsection{Ways of pasting}

Given a specific topological space, the CW-complex which we construct
for it is unique upto homotopy equivalence. However, if we are {\em
  only} given the sequence of homology groups of the topological
space, there may be many possible homotopy types of CW-complexes. The
homology groups essentially only measure the interaction between the
$k$-cells and the $k-1$-cells; they do not capture any information
about how the attaching map behaves on the $k-2$-skeleton.

Thus, if we want to classify all homotopy types of CW-complexes with a
certain sequence of homotopy groups, we need to classify all
possibilities for the attaching maps. 

For instance, suppose we know the $k-1$-skeleton of a simply connected
CW-complex with exactly one $k$-cell. We want to find what are the
possible ways in which the $k$-cell can be attached. Attaching a
$k$-cell is equivalent to choosing a map from $S^{k-1}$ to the
$k-1$-skeleton, and such maps are classified upto homotopy by
$\pi_{k-1}(X^{k-1})$. Every element of this homotopy group gives a
different possibility for the CW-complex.

If we have multiple $k$-cells, then we are effectively choosing
subsets of $\pi_{k-1}(X^{k-1})$ (the homotopy classes of the attaching
maps) and things become more complicated.

\subsection{For 4-manifolds}

Let us now classify all compact simply connected 4-manifolds upto
homotopy.  Let $M$ be a compact simply connected $4$-manifold. Then
$H_1(M) = H^1(M) = 0$ and by Poincare duality, $H_3(M) = H^3(M) = 0$.
Further $H_2(M) \cong H^2(M)$, and $H^2(M)$ is free Abelian (because
$H_1(M) = 0$), hence $H_2(M)$ is free Abelian. Moreover, $H_2(M)$ is
finitely generated. Let $k$ be the number of generators.

In the construction of the CW-complex, the $2$-skeleton is a wedge of
$k$ copies of $S^2$. Denote this wedge as $X^2$. There is exactly one
$4$-cell, and the number of ways we can paste the $4$-cell is
parametrized by elements of the group $\pi_3(X^2)$. Thus, we see that
the third homotopy group of a wedge of spheres controls the possible
homotopy types of CW-complexes.

Of course, not all these homotopy types may be equivalent to compact connected
orientable manifolds.

Let us consider a few particular cases of this. Suppose $b_2 = k = 1$,
viz., $H_2(M) = \Z$. Then the ways of attaching the $4$-cell are
governed by elements of $\pi_3(S^2)$. It turns out that $\pi_3(S^2) =
\Z$, and a generator for this is the Hopf fibration $S^3 \to S^2 =
\C\mathbb{P}^1$.

Some observations:

\begin{itemize}

\item Choosing the generator $1 \in \pi_3(S^2)$, namely the Hopf map itself,
  yields $\C\mathbb{P}^2$.

\item Choosing the element $0 \in \pi_3(S^2)$ yields $S^2 \vee
  S^4$. This is not homotopy-equivalent to a compact connected
  orientable manifold.

\item I believe that choosing other elements in $\pi_3(S^2)$ also
  gives a complex projective plane, but I'm not sure how one would
  prove that.

\end{itemize}

Let's now consider the case $b_2 = 2$. The $3$-skeleton is now $S^2
\vee S^2$, and the maps are parametrized by elements of $\pi_3(S^2
\vee S^2)$. Although we do not explicitly know what this group is,
we can still pick some elements from it:

\begin{itemize}

\item Picking the $0$ element here gives $S^2 \vee S^2 \vee S^4$. This
  is {\em not} homotopy-equivalent to a compact connected orientable
  manifold, because the cap product here is trivial.

\item We can pick the Hopf map from $S^3$ to one of the $S^2$s -- this
  yields $\C\mathbb{P}^2 \vee S^2$. This is again not
  homotopy-equivalent to a compact connected orientable manifold,
  because the cap product of the $S^2$-piece with the top cohomology is zero.

\item We can pick an element of $\pi_3(S^2 \vee S^2)$ which
  corresponds to the boundary of $S^2 \times S^2$, and the CW-complex
  we get is $S^2 \times S^2$.

\item We can first consider the pinch map $S^3 \to S^3 \vee S^3$ and
  then apply the Hopf fibration on each piece. This gives a connected
  sum of two copies of $\C\mathbb{P}^2$. The relative orientations we
  choose for the two pieces determine which connected sum we do get.

  Thus, the {\em wedge} in the $2$-skeleton corresponds to the {\em
    connected sum} as far as the manifolds are concerned. This is
  because the ``point'' at which the wedging is done blows up to a
  $S^3$ after we paste the $4$-cell.

\end{itemize}

In general, an element of $\pi_3(X^2)$ can give rise to a compact
connected orientable manifold only if it maps essentially into each of
the $2$-cells; if it can be made to miss even one $2$-cells, Poincare
duality is contradicted.

As we increase $b_2$, we get various possible connected sums of
$\C\mathbb{P}^2$ and $S^2 \times S^2$, by choosing combinations of
pinching, the Hopf map, and the map from $S^3$ to $S^2 \vee S^2$ for
pasting $S^2 \times S^2$.




\printindex

\end{document}
