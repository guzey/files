\documentclass[10pt]{amsart}

%Packages in use
\usepackage{fullpage, hyperref, vipul, enumerate}

%Title details
\title{Take-home class quiz solutions: due Monday November 25: Stochastic matrices}
\author{Math 196, Section 57 (Vipul Naik)}
%List of new commands

\begin{document}
\maketitle

\section{Performance review}

22 people took this 10-question quiz. The score distribution was as follows:

\begin{itemize}
\item Score of 4: 2 people
\item Score of 5: 2 people
\item Score of 6: 5 people
\item Score of 7: 4 people
\item Score of 8: 5 people
\item Score of 9: 1 person
\item Score of 10: 3 people
\end{itemize}

The mean score was about 7.05.

The question-wise answers and performance review were as follows:

\begin{enumerate}
\item Option (C): 11 people%$14$ people
\item Option (C): 13 people%$12$ people
\item Option (C): 18 people%$11$ people
\item Option (E): 20 people%$19$ people
\item Option (B): 21 people%$15$ people
\item Option (D): 11 people%$9$ people
\item Option (A): 16 people%$7$ people
\item Option (C): 19 people%$18$ people
\item Option (B): 15 people%$8$ people
\item Option (B): 11 people%$3$ people
\end{enumerate}

\section{Solutions}

{\bf PLEASE FEEL FREE TO DISCUSS {\em ALL} QUESTIONS.}

This quiz can be viewed as a continuation of the quiz on linear
dynamical systems. The book defines column-stochastic matrices using
the jargon ``transition matrix'' on Page 53 (Definition 2.1.4) and
uses them throughout the text when describing (a simplified version
of) Google's PageRank algorithm. The quiz questions are self-contained
and do not require you to read the book, but you may benefit from
skimming through the book's discussion of PageRank to complement these
questions. Note that the 4th Edition does not include the discussion
of transition matrices and PageRank.

In this quiz, we discuss the dynamics of a very special type of linear
transformation. A $n \times n$ matrix $A$ is termed a {\em
  row-stochastic matrix} if all its entries are in the interval
$[0,1]$ and all the row sums are equal to $1$. A $n \times n$ matrix
is termed a {\em column-stochastic matrix} if all its entries are in
the interval $[0,1]$ and all the column sums are equal to $1$. A $n
\times n$ matrix $A$ is termed a {\em doubly stochastic matrix} if it
is both row-stochastic and column-stochastic, i.e., all the entries
are in the interval $[0,1]$, all the row sums are equal to $1$, and
all the column sums are equal to $1$.

\begin{enumerate}

\item Suppose $A$ and $B$ are two $n \times n$ row-stochastic
  matrices. Which of the following is {\em guaranteed} to be
  row-stochastic? Please see Options (D) and (E) before answering.

  \begin{enumerate}[(A)]
  \item $A + B$
  \item $A - B$
  \item $AB$
  \item All of the above
  \item None of the above
  \end{enumerate}

  {\em Answer}: Option (C)

  {\em Explanation}: Suppose we are trying to compute the $(ik)^{th}$
  entry of $AB$. This is the sum:

  $$\sum_{j=1}^n a_{ij}b_{jk}$$

  We now want to sum up all such entries in the $i^{th}$ row of
  $AB$. Thus, the sum is:

  $$\sum_{k=1}^n \sum_{j=1}^n a_{ij}b_{jk}$$

  The sum can be rearranged as:

  $$\sum_{j=1}^n \left(a_{ij} \sum_{k=1}^n b_{jk}\right)$$

  Each of the inner sums is $1$, on account of being a row sum of
  $B$. Thus, the sum simplifies to:

  $$\sum_{j=1}^n a_{ij}$$

  This is $1$, on account of being a row sum of $A$.

  Thus, every row sum of $AB$ is $1$. Further, because of the way we
  define matrix multiplication, all the entries of $AB$ are
  nonnegative. Combined with the condition on sums, we get that all
  entries are in $[0,1]$ with all row sums $1$. Thus, the matrix $AB$
  is row-stochastic.

  As for Options (A) and (B), note that for Option (A), the row sums
  will become $2$ and for Option (B), the row sums will become $0$.

  {\em Performance review}: 11 out of 22 got this. 10 chose (E), 1 chose (A).

  {\em Historical note (last time)}: $14$ out of $24$ got this. $9$ chose (E),
  $1$ chose (D).

\item Suppose $A$ and $B$ are two $n \times n$ column-stochastic
  matrices. Which of the following is {\em guaranteed} to be
  column-stochastic? Please see Options (D) and (E) before answering.

  \begin{enumerate}[(A)]
  \item $A + B$
  \item $A - B$
  \item $AB$
  \item All of the above
  \item None of the above
  \end{enumerate}

  {\em Answer}: Option (C)

  {\em Explanation}: Suppose we are trying to compute the $(ik)^{th}$
  entry of $AB$. This is the sum:

  $$\sum_{j=1}^n a_{ij}b_{jk}$$

  We now want to sum up all such entries in the $k^{th}$ column of
  $AB$. Thus, the sum is:

  $$\sum_{i=1}^n \sum_{j=1}^n a_{ij}b_{jk}$$

  The sum can be rearranged as:

  $$\sum_{j=1}^n \left(b_{jk} \sum_{i=1}^n a_{ij}\right)$$

  Each of the inner sums is $1$, on account of being a row sum of
  $A$. Thus, the sum simplifies to:

  $$\sum_{j=1}^n b_{jk}$$

  This is $1$, on account of being a column sum of $B$.

  Thus, every column sum of $AB$ is $1$. Further, because of the way we
  define matrix multiplication, all the entries of $AB$ are
  nonnegative. Combined with the condition on sums, we get that all
  entries are in $[0,1]$ with all row sums $1$. Thus, the matrix $AB$
  is column-stochastic.

  As for Options (A) and (B), note that for Option (A), the column sums
  will become $2$ and for Option (B), the column sums will become $0$.

  {\em Performance review}: 13 out of 22 got this. 8 chose (E), 1 chose (A).

  {\em Historical note (last time)}: $12$ out of $24$ got this. $11$ chose (E),
  $1$ chose (A).

\item Suppose $A$ and $B$ are two $n \times n$ doubly stochastic
  matrices. Which of the following is {\em guaranteed} to be
  doubly stochastic? Please see Options (D) and (E) before answering.

  \begin{enumerate}[(A)]
  \item $A + B$
  \item $A - B$
  \item $AB$
  \item All of the above
  \item None of the above
  \end{enumerate}

  {\em Answer}: Option (C)

  {\em Explanation}: This follows by combining the two preceding questions.

  {\em Performance review}: 18 out of 22 got this. 3 chose (E), 1 chose (A).
 
  {\em Historical note (last time)}: $11$ out of $24$ got this. $8$ chose (D),
  $4$ chose (E), $1$ chose (A).

  \vspace{0.6in}

  We now consider the case $n = 2$. In this case, the doubly
  stochastic matrices have the form:

  $$\left[ \begin{matrix} a & 1 - a \\ 1 - a & a \\\end{matrix}\right]$$

  where $a \in [0,1]$. Denote this matrix by $D(a)$ for short.

\item Suppose $a,b \in [0,1]$ (they are allowed to be equal). The
  product $D(a)D(b)$ equals $D(c)$ for some $c \in [0,1]$. What is
  that value of $c$?

  \begin{enumerate}[(A)]
  \item $a + b$
  \item $ab$
  \item $2ab + a + b$
  \item $(1 - a)(1 - b)$
  \item $1 - a - b + 2ab$
  \end{enumerate}

  {\em Answer}: Option (E)

  {\em Explanation}: We carry out the multiplication:

  $$\left[\begin{matrix} a & 1 - a \\ 1 - a & a \\\end{matrix}\right]\left[\begin{matrix} b & 1 - b \\ 1 - b & b \\\end{matrix}\right] = \left[ \begin{matrix} ab + (1 - a)(1 - b) & a(1 - b) + (1 - a)b \\ a(1- b) + (1 - a)b & ab + (1 - a)(1 - b)\\\end{matrix}\right] = D(ab + (1 - a)(1 - b)) = D(1 - a - b + 2ab)$$

  {\em Performance review}: 20 out of 22 got this. 1 each chose (A) and (D).

  {\em Historical note (last time)}: $19$ out of $24$ got this. $2$ chose (B),
  $1$ each chose (A), (C), and (D).

\item For what value(s) of $a$ is the matrix $D(a)$ non-invertible?
  Note that when judging invertibility, we do not insist that the
  inverse matrix also be doubly stochastic.

  \begin{enumerate}[(A)]
  \item $a = 0$ only
  \item $a = 1/2$ only
  \item $a = 1$ only
  \item $0 < a < 1$ (i.e., $D(a)$ is invertible only at $a = 0$ and $a
    = 1$)
  \item $a \ne 1/2$
  \end{enumerate}

  {\em Answer}: Option (B)

  {\em Explanation}: For the matrix to be non-invertible, we need the
  rows to be scalar multiples of each other. Since both row sums are
  $1$, this can happen only if the rows are identical, which happens
  iff $a = 1/2$. Equivalently, we can note that invertibility requires
  a nonzero determinant, and the determinant is $a^2 - (1 - a)^2 = 2a -
  1$, which is $0$ iff $a = 1/2$.

  {\em Performance review}: 21 out of 22 got this. 1 chose (A).

  {\em Historical note (last time)}: $15$ out of $24$ got this. $4$ chose (A),
  $2$ each chose (C) and (D), $1$ chose (E).


\item For what value(s) of $a$ is it true that the matrix $D(a)$ does
  not have an inverse that is a doubly stochastic matrix? In other
  words, either $D(a)$ should be non-invertible or it should be
  invertible but the inverse is not a doubly stochastic matrix.

  \begin{enumerate}[(A)]
  \item $a = 0$ only
  \item $a = 1/2$ only
  \item $a = 1$ only
  \item $0 < a < 1$ (i.e., $D(a)$ has an inverse that is also doubly
    stochastic only if $a = 0$ or $a = 1$)
  \item $a \ne 1/2$
  \end{enumerate}

  {\em Answer}: Option (D)

  {\em Explanation}: The inverse of $D(a)$, for $a \ne 1/2$, is:

  $$\left[\begin{matrix} a/(2a - 1) & (a - 1)/(2a - 1)\\ (a - 1)/(2a - 1) & a/(2a - 1)\\\end{matrix}\right]$$

  For this to be doubly stochastic, we need that $0 \le a/(2a - 1) \le
  1$. We make cases:

  \begin{itemize}
  \item $a = 0$: In this case, $D(a)$ is its own inverse.
  \item $a \ne 0$ (excluding $a = 1/2$): In this case, since $0 \le
    a/(2a - 1)$, we obtain that $2a - 1 > 0$. So, starting with $a/(2a
    - 1) \le 1$ we get $a \le 2a - 1$, which simplifies to $1 \le a$,
    forcing $a = 1$ (since we are constrained by $0 \le a \le 1$). The
    choice $a = 1$ works (in the sense that $D(1)$ is its own
    inverse).
  \end{itemize}

  The upshot is that the only cases where the inverse is doubly
  stochastic are the cases $a = 0$ and $a = 1$. Otherwise, the inverse
  either does not exist (case $a = 1/2$) or exists but is not doubly
  stochastic.

  {\em Performance review}: 11 out of 22 got this. 9 chose (B), 2 chose (C).

  {\em Historical note (last time)}: $9$ out of $24$ got this. $9$ chose (A),
  $2$ each chose (B), (C), and (E).

  \vspace{0.6in}

  For the next few questions, denote by $T_a$ the linear
  transformation whose matrix is $D(a)$. For any vector $\vec{x} \in
  \R^2$, we can consider the sequence:

  $$\vec{x}, T_a(\vec{x}), T_a^2(\vec{x}), \dots$$

  Note that if we were to start with a vector $\vec{x} \in \R^2$ with
  both coordinates equal, it would be invariant under $T_a$.

  Thus, for the questions below, assume that we start with a nonzero
  vector $\vec{x} \in \R^2$ for which the two coordinates are not
  equal to each other.

\item For what value of $a$ is it the case that $\lim_{r \to \infty}
  T_a^r(\vec{x})$ does {\em not} exist?

  \begin{enumerate}[(A)]
  \item $a = 0$ only
  \item $a = 1/2$ only
  \item $a = 1$ only
  \item $0 < a < 1$
  \item $a \ne 1/2$
  \end{enumerate}

  {\em Answer}: Option (A)

  {\em Explanation}: In this case, applying $T_a$ interchanges the
  coordinates. A second application interchanges them back. The
  sequence thus cycles between the vector $\vec{x}$ and the vector
  obtained by interchanging its coordinates.

  The failure of the remaining options will become clear from the
  answers to the rest of the questions.

  {\em Performance review}: 16 out of 22 got this. 2 chose (B), 1 each
  chose (C), (D), and (E). 1 left the question blank.

  {\em Historical note (last time)}: $7$ out of $24$ got this. $11$ chose (B),
  $4$ chose (C), $1$ each chose (D) and (E).


\item For what value of $a$ is it the case that the sequence

  $$\vec{x}, T_a(\vec{x}), T_a^2(\vec{x}), \dots$$

  is a constant sequence?

  \begin{enumerate}[(A)]
  \item $a = 0$ only
  \item $a = 1/2$ only
  \item $a = 1$ only
  \item $0 < a < 1$
  \item $a \ne 1/2$
  \end{enumerate}

  {\em Answer}: Option (C)

  {\em Explanation}: $a = 1$ gives the identity transformation. Any
  other choice of $a$ replaces the first coordinate by a combination
  of the two coordinates that cannot be equal to it unless the two
  coordinates were equal to begin with. Explicitly:

  $$D(a)\left[\begin{matrix} x_1 \\ x_2 \\\end{matrix}\right] = \left[ \begin{matrix}ax_1 + (1 - a)x_2 \\ (1 - a)x_1 + ax_2 \\\end{matrix}\right]$$

  For the output to equal the input, we need that:

  \begin{eqnarray*}
    x_1 & = & ax_1 + (1 - a)x_2\\
    x_2 & = & (1 - a)x_1 + ax_2 \\
  \end{eqnarray*}

  Solving the first equation alone gives $x_1 = x_2$ or $a = 1$. By
  assumption, $x_1 \ne x_2$, so we get $a = 1$. Note that the second
  equation yields a similar conclusion.

  {\em Performance review}: 19 out of 22 got this. 2 chose (B), 1 chose (A).

  {\em Historical note (last time)}: $18$ out of $24$ got this. $3$ chose (D),
  $2$ chose (E), $1$ chose (B).

\item For what value of $a$ is it the case that the sequence

  $$\vec{x}, T_a(\vec{x}), T_a^2(\vec{x}), \dots$$

  is not a constant sequence but becomes constant from $T_a(\vec{x})$
  onward?

  \begin{enumerate}[(A)]
  \item $a = 0$ only
  \item $a = 1/2$ only
  \item $a = 1$ only
  \item $0 < a < 1$ 
  \item $a \ne 1/2$
  \end{enumerate}

  {\em Answer}: Option (B)

  {\em Explanation}: If the sequence is not constant, then $a \ne
  1$. However, it becomes constant from $T_a(\vec{x})$ onward. Thus,
  $T_a(\vec{x})$ has both coordinates equal by the previous
  question. Hence, we get that:

  $$ax_1 + (1 - a)x_2 = (1 - a)x_1 + ax_2$$

  This simplifies to:

  $$(2a - 1)(x_1 - x_2) = 0$$

  Thus, either $a = 1/2$ or $x_1 = x_2$. Since $x_1 \ne x_2$ by
  assumption, we get $a = 1/2$.

  {\em Performance review}: 15 out of 22 got this. 3 chose (A), 2 each
  chose (C) and (D).

  {\em Historical note (last time)}: $8$ out of $24$ got this. $10$ chose (A),
  $4$ chose (C), $1$ each chose (D) and (E).

\item For $a$ other than $0$, $1/2$, or $1$, what is the limit
  $\lim_{r \to \infty} (D(a))^r$? Here, when we talk of taking the
  limit of a sequence of matrices, we are taking the limit entry-wise.

  \begin{enumerate}[(A)]
  \item The matrix $D(0)$
  \item The matrix $D(1/2)$
  \item The matrix $D(1)$
  \item The matrix $D(a)$
  \item The matrix $D(1 - a)$
  \end{enumerate}

  {\em Answer}: Option (B)

  {\em Explanation}: Intuitively, what's happening is that we start
  off with the point:

  $$\left[\begin{matrix} x_1 \\ x_2 \\\end{matrix}\right]$$
  
  We now make the coordinates ``come closer to each other'' by converting to:

  $$\left[\begin{matrix} ax_1 + (1 - a)x_2 \\ (1 - a)x_1 + ax_2 \\\end{matrix}\right]$$

  We then iterate. Each time, the coordinates are coming closer to
  each other, but note also that the sum of the coordinates remains
  fixed. Thus, we hope to eventually converge to the vector with both
  coordinates $(x_1 + x_2)/2$. This is $D(1/2)\left[\begin{matrix} x_1
      \\ x_2 \\\end{matrix}\right]$.

  Here is a more formal demonstration: By one of the previous
  questions, we can verify that if $D(a)D(b) = D(c)$, then $(2a -
  1)(2b - 1) = 2c - 1$. An immediate corollary is that if $(D(a))^r =
  D(u)$, then $(2a - 1)^r = 2u - 1$. As a result, if $\lim_{r \to
    \infty} (D(a))^r = D(u)$, then $2u - 1 = \lim_{r \to \infty} (2a -
  1)^r =0$, forcing $u = 1/2$.

  The geometric intuition for why we look at $2a - 1$ is because that
  is the ratio of the signed difference between the output coordinates
  to the signed difference between the input coordinates. It describes
  an contraction factor, and the contraction factors multiply when we
  compose.

  {\em Performance review}: 11 out of 22 got this. 5 chose (A), 4
  chose (E), 1 each chose (C) and (D).

  {\em Historical note (last time)}: $3$ out of $24$ got this. $10$ chose (D),
  $7$ chose (E), $2$ each chose (A) and (C).

\end{enumerate}
\end{document}
