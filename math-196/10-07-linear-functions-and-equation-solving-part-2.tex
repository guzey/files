\documentclass[10pt]{amsart}

%Packages in use
\usepackage{fullpage, hyperref, vipul, enumerate}

%Title details
\title{Take-home class quiz: due Monday October 7: Linear functions and equation-solving (part 2)}
\author{Math 196, Section 57 (Vipul Naik)}
%List of new commands

\begin{document}
\maketitle

Your name (print clearly in capital letters): $\underline{\qquad\qquad\qquad\qquad\qquad\qquad\qquad\qquad\qquad\qquad}$

{\bf PLEASE DO {\em NOT} DISCUSS ANY QUESTIONS EXCEPT THE STARRED OR DOUBLE-STARRED QUESTIONS.}

This quiz covers some basics involving linear functions and
equation-solving (notes at {\tt Linear functions: a primer} and {\tt
  Equation-solving with a special focus on the linear case}). The quiz
tests for the following:

\begin{itemize}
\item The distinction between behavior relative to the variables (the
  inputs) and behavior relative to the parameters.
\item Counting the number of parameters by creating the explicit
  general functional form from a verbal description (with a special
  focus on polynomial functional forms).
\item Figuring out how to ``ask the right questions'' with respect to
  input choices, so that the answers provide meaningful
  information. This builds towards the ideas of hypothesis testing,
  rank, and overdetermination that we will see in the future.
\end{itemize}

\begin{enumerate}
\item {\em Do not discuss this!}: Suppose $f$ is a polynomial function
  of $x$ of degree at most a {\em known number} $n$. What is the
  minimum number of (input,output) pairs that we need in order to
  determine $f$ uniquely? {\em Extra information: Somewhat
    surprisingly, in this case, we do not need to be judicious about
    our input choices. Any set of distinct inputs of the required
    number will do. This has something to do with the ``Vandermonde
    matrix'' and ``Vandermonde determinant'' and is also related to
    the Lagrange interpolation formula.}

  \begin{enumerate}[(A)]
  \item $n - 1$
  \item $n$
  \item $n + 1$
  \item $2n$
  \item $n^2$
  \end{enumerate}

  \vspace{0.1in}
  Your answer: $\underline{\qquad\qquad\qquad\qquad\qquad\qquad\qquad}$
  \vspace{0.6in}

\item {\em Do not discuss this!}: $f$ is a polynomial function of two
  variables $x$ and $y$ of total degree at most $2$. In other words,
  for each monomial occurring in $f$, the total of the degrees of $x$
  and $y$ in that monomial is at most $2$. No other information is
  given about $f$. What is the minimum number of judiciously chosen
  (input,output) pairs we need in order to determine $f$ uniquely?

  \begin{enumerate}[(A)]
  \item $2$
  \item $3$
  \item $4$
  \item $6$
  \item $7$
  \end{enumerate}

  \vspace{0.1in}
  Your answer: $\underline{\qquad\qquad\qquad\qquad\qquad\qquad\qquad}$
  \vspace{0.6in}

\item {\em Do not discuss this!}: $f$ is a polynomial function of two
  variables $x$ and $y$ of total degree at most $3$. In other words,
  for each monomial occurring in $f$, the total of the degrees of $x$
  and $y$ in that monomial is at most $3$. No other information is
  given about $f$. What is the minimum number of judiciously chosen
  (input,output) pairs we need in order to determine $f$ uniquely?

  \begin{enumerate}[(A)]
  \item $3$
  \item $6$
  \item $8$
  \item $9$
  \item $10$
  \end{enumerate}

  \vspace{0.1in}
  Your answer: $\underline{\qquad\qquad\qquad\qquad\qquad\qquad\qquad}$
  \vspace{0.6in}

\item (*) {\em The perils of overfitting; see also Occam's Razor}:
  Suppose we are trying to model a function that we expect to behave
  in a polynomial-like manner, though we don't really have a good
  reason to believe this. Additionally, there is a possibility for
  measurement error in our observations. Our goal is to find the
  parameters so that we can both predict unmeasured values and do a
  qualitative analysis of the nature of the function and its
  derivatives and integrals.

  We have a large number of observations (say, several thousands). We
  could attempt to ``fit'' the function using a polynomial of degree
  $n$ for some fixed $n$ using all those data points, and we will get
  a certain ``best fit'' that minimizes the deviation between the
  curve used for fitting and the function being fit. For instance, for
  $n = 1$, we are trying to find the best fit by a straight line
  function. For $n = 2$, we are trying to find the best fit by a
  polynomial of degree at most $2$. We could try fitting using
  different values of $n$. Which of the following is true?

  {\em If you are interested in more on this, look up
    ``overfitting''. A revealing quote is by mathematician and
    computer scientist John von Neumann: ``With four parameters I can
    fit an elephant. And with five I can make him wiggle his trunk.''
    Another is by prediction guru Nate Silver: ``The wide array of
    statistical methods available to researchers enables them to be no
    less fanciful and no more scientific than a child finding animal
    patterns in clouds.''}

  \begin{enumerate}[(A)]
  \item Larger values of $n$ give better fits, therefore the larger
    the value of $n$ we use, the better.
  \item Smaller values of $n$ give better fits, therefore the smaller
    the value of $n$ we use, the better.
  \item Larger values of $n$ give better fits, therefore the larger
    the value of $n$ we use, the less impressive a good fit (i.e., low
    deviation between the polynomial and the actual set of
    observations) should be.
  \item Smaller values of $n$ give better fits, therefore the smaller
    the value of $n$ we use, the less impressive a good fit (i.e., low
    deviation between the polynomial and the actual set of
    observations) should be.
  \item The value of $n$ we use for trying to get a good fit is
    irrelevant. A good fit is a good fit, regardless of the type of
    function used.
  \end{enumerate}

  \vspace{0.1in}
  Your answer: $\underline{\qquad\qquad\qquad\qquad\qquad\qquad\qquad}$
  \vspace{0.6in}

\item (*) $F$ is an affine linear function of two variables $x$ and
  $y$, i.e., it has the form $F(x,y) := ax + by + c$ with $a$, $b$,
  and $c$ real numbers. We want to determine the values of the
  parameters $a$, $b$, and $c$ by using input-output pairs. It is,
  however, costly to find input-output pairs. We have already found
  $F(1,3)$ and $F(3,7)$. We want to find $F$ for one other pair of
  inputs to determine $a$, $b$, and $c$. Which of these will {\em not}
  be a good choice?

  \begin{enumerate}[(A)]
  \item $F(2,2)$, i.e., the input $x = 2$, $y = 2$
  \item $F(2,3)$, i.e., the input $x = 2$, $y = 3$
  \item $F(2,4)$, i.e., the input $x = 2$, $y = 4$
  \item $F(2,5)$, i.e., the input $x = 2$, $y = 5$
  \item $F(2,6)$, i.e., the input $x = 2$, $y = 6$
  \end{enumerate}

  \vspace{0.1in}
  Your answer: $\underline{\qquad\qquad\qquad\qquad\qquad\qquad\qquad}$
  \vspace{0.6in}
\end{enumerate}

\end{document}
