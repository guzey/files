\documentclass[10pt]{amsart}

%Packages in use
\usepackage{fullpage, hyperref, vipul, enumerate}

%Title details
\title{Diagnostic in-class quiz: originally due Friday November 15, delayed to Wednesday November 20: Linear dependence, bases, and subspaces}
\author{Math 196, Section 57 (Vipul Naik)}
%List of new commands

\begin{document}
\maketitle

Your name (print clearly in capital letters): $\underline{\qquad\qquad\qquad\qquad\qquad\qquad\qquad\qquad\qquad\qquad}$

{\bf PLEASE DO NOT DISCUSS {\em ANY} QUESTIONS}

The purpose of this quiz is to review some basic ideas from part of
the lecture notes titled {\tt Linear dependence, bases, and
  subspaces}. The corresponding sections of the book are Sections 3.2
and 3.3.

\begin{enumerate}
\item {\em Do not discuss this!}: Suppose $S$ is a finite nonempty set
  of vectors in $\R^n$, and $T$ is a nonempty subset of $S$. What can
  we say about $S$ and $T$?

  \begin{enumerate}[(A)]
  \item $S$ is linearly dependent if and only if $T$ is linearly
    dependent. $S$ is linearly independent if and only if $T$ is
    linearly independent.
  \item If $S$ is linearly dependent, then $T$ is linearly
    dependent. If $S$ is linearly independent, then $T$ is linearly
    independent. However, we cannot deduce anything about the linear
    dependence or independence of $S$ from the linear dependence or
    independence of $T$.
  \item If $T$ is linearly dependent, then $S$ is linearly
    dependent. If $T$ is linearly independent, then $S$ is linearly
    independent. However, we cannot deduce anything about the linear
    dependence or independence of $T$ from the linear dependence or
    independence of $S$.
  \item If $S$ is linearly dependent, then $T$ is linearly
    dependent. If $T$ is linearly independent, then $S$ is linearly
    independent. We cannot make either of the two other deductions.
  \item If $T$ is linearly dependent, then $S$ is linearly
    dependent. If $S$ is linearly independent, then $T$ is linearly
    independent. We cannot make either of the other two deductions.
  \end{enumerate}
    
  \vspace{0.1in}
  Your answer: $\underline{\qquad\qquad\qquad\qquad\qquad\qquad\qquad}$
  \vspace{0.1in}

\item {\em Do not discuss this!}: Suppose $S$ is a finite set of
  vectors in $\R^n$. Consider the three statements: (i) $S$ is
  linearly independent, (ii) $S$ does not contain the zero vector,
  (iii) $S$ does not contain any two vectors that are scalar multiples
  of one another. Which of the following options best describes the
  relationship between these statements?

  \begin{enumerate}[(A)]
  \item (i) is equivalent to (ii), and both imply (iii), but the
    reverse implication does not hold.
  \item (i) is equivalent to (iii), and both imply (ii), but the
    reverse implication does not hold.
  \item (i) is equivalent to (ii) and (iii) combined.
  \item (i) implies both (ii) and (iii), but (ii) and (iii), even if
    combined, do not imply (i).
  \end{enumerate}

  \vspace{0.1in}
  Your answer: $\underline{\qquad\qquad\qquad\qquad\qquad\qquad\qquad}$
  \vspace{0.1in}

\item {\em Do not discuss this!}: Suppose $V$ is a linear subspace of
  $\R^n$ for some $n$, and $W$ is a linear subspace of $V$. Assume
  also that $W \ne V$, i.e., $W$ is a {\em proper} subspace of
  $V$. Which of the following correctly describes the relationship
  between bases of $V$ and bases of $W$?

  \begin{enumerate}[(A)]
  \item Given a basis of $V$, we can find a subset of that basis that
    is a basis of $W$. Also, given a basis of $W$, we can find a set
    containing that basis that is a basis of $V$.
  \item Given a basis of $V$, we can find a subset of that basis that
    is a basis of $W$. However, given a basis of $W$, we may not
    necessarily be able to find a set containing that basis that is a
    basis of $V$.
  \item Given a basis of $V$, we may not necessarily be able to find a
    subset of that basis that is a basis of $W$. However, given a
    basis of $W$, we can find a set containing that basis that is a
    basis of $V$.
  \end{enumerate}

  \vspace{0.1in}
  Your answer: $\underline{\qquad\qquad\qquad\qquad\qquad\qquad\qquad}$
  \vspace{0.1in}

\end{enumerate}
\end{document}
