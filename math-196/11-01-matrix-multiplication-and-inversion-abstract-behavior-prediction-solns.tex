\documentclass[10pt]{amsart}

%Packages in use
\usepackage{fullpage, hyperref, vipul, enumerate}

%Title details
\title{Take-home class quiz solutions: due Friday November 1: Matrix multiplication and inversion: abstract behavior prediction}
\author{Math 196, Section 57 (Vipul Naik)}
%List of new commands

\begin{document}
\maketitle

\section{Performance review}

26 people took this 6-question quiz. The score distribution was as follows:

\begin{itemize}
\item Score of 0: 1 person
\item Score of 1: 3 people
\item Score of 2: 3 people
\item Score of 3: 7 people
\item Score of 4: 5 people
\item Score of 5: 6 people
\item Score of 6: 1 person
\end{itemize}

The mean score was 3.3.

The question-wise answers and performance review are as follows:

\begin{enumerate}
\item Option (B): 12 people
\item Option (C): 12 people
\item Option (E): 21 people
\item Option (E): 19 people
\item Option (A): 16 people
\item Option (E): 6 people
\end{enumerate}

{\em Note}: Question 6 erroneously had ``5 points'' printed in front
of it in the print copy handed out to students. That was because the
question was copy-pasted to the quiz from a previous year's test. We
are not giving it additional weight.

{\em Note on comparison with last time}: Last time, I gave these
questions (all except Question 6) {\em before} I gave the ``linear
transformations and finite state automata'' questions. This might
explain why the overall performance was somewhat worse last time
compared to this time.

\section{Solutions}


{\bf PLEASE FEEL FREE TO DISCUSS {\em ALL} QUESTIONS.}

This quiz tests for {\em abstract behavior prediction} related to the
structure of matrices defined based on the operations of matrix
multiplication and inversion. It is based on part of the {\tt Matrix
  multiplication and inversion} notes and is related to Sections 2.3
and 2.4. It does not, however, test all aspects of that material.

To understand this abstract behavior, we will consider {\em
  nilpotent}, {\em invertible}, and {\em idempotent} matrices.

\begin{enumerate}

\item Suppose $A$ and $B$ are $n \times n$ matrices such that $B$ is
  invertible. Suppose $r$ is a positive integer. What can we say that
  $(BAB^{-1})^r$ definitely equals?

  \begin{enumerate}[(A)]
  \item $A^r$
  \item $BA^rB^{-1}$
  \item $B^rA^rB^{-r}$
  \item $B^rAB^{-r}$
  \item $BAB^{-1-r}$
  \end{enumerate}

  {\em Answer}: Option (B)

  {\em Explanation}: Write:

  $$BAB^{-1}BAB^{-1} \dots BAB^{-1}$$

  Each $B^{-1}$ and subsequent $B$ multiply to the identity matrix,
  which disappears. So, we are left with:

  $$BAA\dots AB^{-1}$$

  where $A$ appears $r$ times. We thus get $BA^rB^{-1}$.

  This is related to some deep facts about group structure. We will
  hint at these later in the course, but will not be able to
  appreciate the full depth of these.
  
  {\em Performance review}: 12 out of 26 got this. 10 chose (A), 4 chose (C).

  {\em Historical note (last time)}: $8$ out of $26$ got this. $13$ chose (A),
  $5$ chose (C).
\item Suppose $A$ and $B$ are $n \times n$ matrices ($n$ not too
  small) such that $(AB)^2 = 0$. What is the smallest $r$ for which we
  can conclude that $(BA)^r$ is definitely $0$?

  \begin{enumerate}[(A)]
  \item $1$
  \item $2$
  \item $3$
  \item $4$
  \item $5$
  \end{enumerate}

  {\em Answer}: Option (C)

  {\em Explanation}: We have:

  $$(BA)^3 = BABABA = B(ABAB)A = B(AB)^2A = B(0)A = 0$$

  Thus, $(BA)^3$ is definitely $0$. There are examples of matrices $A$
  and $B$ such that $(AB)^2 = 0$ but $(BA)^2 \ne 0$. The smallest
  examples are $3 \times 3$. For instance:

  $$A = \left[\begin{matrix} 0 & 1 & 0 \\ 0 & 0 & 1 \\ 0 & 0 & 0 \\\end{matrix}\right], B = \left[\begin{matrix} 1 & 0 & 0 \\ 0 & 1 & 0 \\ 0 & 0 & 0 \\\end{matrix}\right]$$

  We obtain:

  $$AB = \left[\begin{matrix} 0 & 1 & 0 \\ 0 & 0 & 0 \\ 0 & 0 & 0 \\\end{matrix}\right], BA = \left[\begin{matrix} 0 & 1 & 0 \\ 0 & 0 & 1 \\ 0 & 0 & 0 \\\end{matrix}\right]$$

  We have $(AB)^2 = 0$ but $(BA)^2 \ne 0$. However, $(BA)^3 = 0$.

  {\em How did we construct this example?}: Building on the ``linear
  transformations and finite state automata'' framework, the idea is
  to choose $f,g: \{ 0,1,2,3 \} \to \{ 0,1,2,3 \}$ (with $f(0) = g(0)
  = 0$) such that $f \circ g$ is a function whose composite with itself
  sends everything to zero, whereas $g \circ f$ is a function whose
  composite with itself does not send everything to zero. The above
  matrices arise from one such example:

  $$f(0) = 0, f(1) = 0, f(2) = 1, f(3) = 2, \qquad g(0) = 0, g(1) = 1, g(2) = 2, g(3) = 0$$

  The composites are:

  $$(f \circ g)(0) = 0, (f \circ g)(1) = 0, (f \circ g)(2) = 1, (f \circ g)(3) = 0$$

  and:

  $$(g \circ f)(0) = 0, (g \circ f)(1) = 0, (g \circ f)(2) = 1, (g \circ f)(3) = 2$$

  Notice that $f \circ g$ sends $2$ to $1$ and everything else to $0$,
  hence its composite with itself sends everything to zero. On the
  other hand, the composite of $g \circ f$ with itself sends $3$ to
  $1$, and therefore does not send everything to zero.

  In symbols, $A = M_f$, $B = M_g$, $AB = M_{f \circ g}$, and $BA =
  M_{g \circ f}$.

  {\em Performance review}: 12 out of 26 got this. 12 chose (B), 1
  each chose (A) and (E).

  {\em Historical note (last time)}: $6$ out of $26$ got this. $16$
  chose (B), $3$ chose (A), $1$ chose (E).
\item Suppose $n > 1$. A $n \times n$ matrix $A$ is termed {\em
  nilpotent} if there exists a positive integer $r$ such that $A^r$ is
  the zero matrix. It turns out that if $A$ is nilpotent, then $A^n =
  0$. Which of the following describes correctly the relationship
  between being invertible and being nilpotent for $n \times n$
  matrices?

  \begin{enumerate}[(A)]
  \item A matrix is nilpotent if and only if it is invertible.
  \item Every nilpotent matrix is invertible, but not every invertible matrix is nilpotent.
  \item Every invertible matrix is nilpotent, but not every nilpotent matrix is invertible.
  \item An invertible matrix may or may not be nilpotent, and a
    nilpotent matrix may or may not be invertible.
  \item A matrix cannot be both nilpotent and invertible.
  \end{enumerate}

  {\em Answer}: Option (E)

  {\em Explanation}: Suppose $A^r = 0$ and $A$ has inverse
  $A^{-1}$. We have $(A^{-1})^rA^r = I_n$ (where $I_n$ denotes the $n
  \times n$ identity matrix), but it also equals $(A^{-1})^r0 = 0$, so
  $I_n = 0$ is a contradiction.

  {\em Performance review}: 21 out of 26 got this. 3 chose (B), 1 each
  chose (A) and (D).

  {\em Historical note (last time)}: $9$ out of $26$
  got this. $9$ chose (C), $6$ chose (B), $2$ chose (D).
\item Suppose $A$ and $B$ are $n \times n$ matrices. Which of the
  following is true? Please see Option (E) before answering.

  \begin{enumerate}[(A)]
  \item $AB$ is nilpotent if and only if $A$ and $B$ are both nilpotent.
  \item $AB$ is nilpotent if and only if at least one of $A$ and $B$ is nilpotent.
  \item If both $A$ and $B$ are nilpotent, then $AB$ is nilpotent, but
    $AB$ being nilpotent does not imply that both $A$ and $B$ are
    nilpotent.
  \item If $AB$ is nilpotent, then both $A$ and $B$ are
    nilpotent. However, both $A$ and $B$ being nilpotent does not
    imply that $AB$ is nilpotent.
  \item None of the above.
  \end{enumerate}

  {\em Answer}: Option (E)

  {\em Explanation}: Here is an example where $A$ and $B$ are both
  nilpotent but $AB$ is not:

  $$A = \left[ \begin{matrix} 0 & 1 \\ 0 & 0 \\\end{matrix}\right], B = \left[ \begin{matrix} 0 & 0 \\ 1 & 0 \\\end{matrix}\right], AB = \left[ \begin{matrix} 1 & 0 \\ 0 & 0 \\\end{matrix}\right]$$

  Here is an example where neither $A$ nor $B$ is nilpotent but $AB$
  is the zero matrix, and therefore, is nilpotent:

  $$A = \left[ \begin{matrix} 1 & 0 \\ 0 & 0 \\\end{matrix}\right], B = \left[ \begin{matrix} 0 & 0 \\ 0 & 1 \\\end{matrix}\right]$$

  {\em How did we construct these examples}: These examples fall out
  of the ``linear transformations and finite state automata''
  framework. In fact, there are questions on that quiz that directly
  correspond to the construction of examples for this situation. Can
  you locate them?

  {\em Performance review}: 19 out of 26 got this. 5 chose (C), 1 each
  chose (A) and (B).

  {\em Historical note (last time)}: $6$ out of $26$ got this. $9$
  chose (D), $5$ each chose (B) and (C), $1$ chose (A).
\item Suppose $A$ and $B$ are $n \times n$ matrices. Which of the
  following is true? Please see Option (E) before answering.

  \begin{enumerate}[(A)]
  \item $AB$ is invertible if and only if $A$ and $B$ are both invertible.
  \item $AB$ is invertible if and only if at least one of $A$ and $B$ is invertible.
  \item If both $A$ and $B$ are invertible, then $AB$ is invertible, but
    $AB$ being invertible does not imply that both $A$ and $B$ are
    invertible.
  \item If $AB$ is invertible, then both $A$ and $B$ are
    invertible. However, both $A$ and $B$ being invertible does not
    imply that $AB$ is invertible.
  \item None of the above.
  \end{enumerate}

  {\em Answer}: Option (A)

  {\em Explanation}: If $A$ and $B$ are both invertible, then
  $(AB)^{-1} = B^{-1}A^{-1}$. if $AB$ is invertible with inverse $C$,
  then $C(AB) = I_n$, so $CA$ is an inverse for $B$, and $(AB)C =
  I_n$, so $BC$ is an inverse for $A$. We are using the (somewhat
  nontrivial fact) that if a square matrix has a one-sided inverse,
  that inverse is actually a two-sided inverse.

  {\em Performance review}: 16 out of 26 got this. 5 each chose (C)
  and (D).

  {\em Historical note (last time)}: $11$ out of $26$ got this. $8$
  chose (D), $3$ each chose (C) and (E), $1$ chose (B).

\item Suppose $A$ and $B$ are $n \times n$ matrices. Which of the
  following is true? We call a $n \times n$ matrix {\em idempotent} if
  it equals its own square. Please see Option (E) before answering.

  \begin{enumerate}[(A)]
  \item $AB$ is idempotent if and only if $A$ and $B$ are both idempotent.
  \item $AB$ is idempotent if and only if at least one of $A$ and $B$ is idempotent.
  \item If both $A$ and $B$ are idempotent, then $AB$ is idempotent, but
    $AB$ being idempotent does not imply that both $A$ and $B$ are
    idempotent.
  \item If $AB$ is idempotent, then both $A$ and $B$ are
    idempotent. However, both $A$ and $B$ being idempotent does not
    imply that $AB$ is idempotent.
  \item None of the above.
  \end{enumerate}

  {\em Answer}: Option (E)

  {\em Explanation}: The following is an example where both $A$ and
  $B$ are idempotent but $AB$ is not idempotent:

  $$A = \left[\begin{matrix} 1 & 1 \\ 0 & 0 \\\end{matrix}\right], B = \left[ \begin{matrix} 0 & 0 \\ 0 & 1 \\\end{matrix}\right]$$

  The product is:

  $$AB = \left[\begin{matrix} 0 & 1 \\ 0 & 0 \\\end{matrix}\right]$$

  The following is an example where neither $A$ nor $B$ is idempotent
  but $AB$ is idempotent:

  $$A= \left[\begin{matrix} 0 & 1 \\ 0 & 0 \\\end{matrix}\right], B = \left[\begin{matrix} 0 & 0 \\ 1 & 0 \\\end{matrix}\right]$$

  The product matrix is:

  $$AB = \left[\begin{matrix} 1 & 0 \\ 0 & 0 \\\end{matrix}\right]$$

  {\em Performance review}: 6 out of 26 got this. 8 each chose (C) and
  (D), 3 chose (A), and 1 chose (B).

  {\em Historical note (last time: {\bf final})}: This question
  appeared in last year's final. 9 out of 30 got it correct at the
  time (note that students had significantly more exposure to the
  concepts by the time of the final). 12 chose (C), 4 each chose (A)
  and (D), 1 chose (B).
\end{enumerate}
\end{document}
