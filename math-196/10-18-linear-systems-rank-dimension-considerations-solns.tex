
\documentclass[10pt]{amsart}

%Packages in use
\usepackage{fullpage, hyperref, vipul, enumerate}

%Title details
\title{Take-home class quiz solutions: due Friday October 18: Linear systems: rank and dimension considerations}
\author{Math 196, Section 57 (Vipul Naik)}
%List of new commands

\begin{document}
\maketitle

\section{Performance review}

27 people took this 8-question quiz. The score distribution was as
follows:

\begin{itemize}
\item Score of 3: 1 person
\item Score of 4: 4 people
\item Score of 5: 7 people
\item Score of 6: 6 people
\item Score of 7: 8 people
\item Score of 8: 1 person
\end{itemize}

The question-wise answers and performance review are as follows:

\begin{enumerate}
\item Option (E): 24 people
\item Option (B): 21 people
\item Option (E): 10 people
\item Option (E): 24 people
\item Option (B): 24 people
\item Option (B): 20 people
\item Option (A): 17 people
\item Option (C): 14 people
\end{enumerate}

\section{Solutions}

The questions here consider a wide range of theoretical and practical
settings where linear systems appear, and prompt you to think about
the notion of rank and its relationship with whether we can uniquely
acquire the information that we want. It relates approximately with
the material in the {\tt Linear systems and matrix algebra} notes (the
corresponding section in the book is Section 1.3).

\begin{enumerate}
\item (*) Let $m$ and $n$ be positive integers. It turns out that {\em
  almost all} $m \times n$ matrices over the real numbers have a
  particular rank. What is that rank? (Unfortunately, it is beyond our
  current scope to define ``almost all'').

  \begin{enumerate}[(A)]
  \item $m$ (regardless of whether $m$ or $n$ is bigger)
  \item $n$ (regardless of whether $m$ or $n$ is bigger)
  \item $(m + n)/2$
  \item $\max \{ m,n \}$
  \item $\min \{ m,n \}$
  \end{enumerate}

  {\em Answer}: Option (E)

  {\em Explanation}: In some sense, this is the only viable option,
  because we know that the rank is at {\em most} $\min \{ m,n \}$. The
  intuitive reason why the theoretical maximum is usually attained is
  that ``random things are most likely to be unrelated.'' 

  {\em Performance review}: 24 out of 27 got this. 2 chose (A), 1
  chose (D).

  {\em Historical note (last time)}: $23$ out of $25$ got this. $1$ each chose
  (C) and (D).

\item (*) A container has a mix of two known gases that do not react
  with each other. The temperature and pressure of the container are
  known. Assume that $PV = nRT$. The volume of the container is also
  known, and so is the total mass of the gases in the container. Under
  what conditions can we predict the amount (say, in the form of the
  number of moles) of each gas that is present from this information?

  \begin{enumerate}[(A)]
  \item It is possible if both gases have the same molecular mass,
    because in that case, the coefficient matrix of the linear system
    has full rank $2$.
  \item It is possible if both gases have different molecular
    masses, because in that case, the coefficient matrix of the linear
    system has full rank $2$.
  \item It is possible if both gases have the same molecular mass,
    because in that case, the coefficient matrix of the linear system
    has rank $1$.
  \item It is possible if both gases have different molecular
    masses, because in that case, the coefficient matrix of the linear
    system has rank $1$.
  \item It is not possible to deduce the amount of each gas from the
    given information.
  \end{enumerate}

  {\em Answer}: Option (B)

  {\em Explanation}: Let $n_1$ and $n_2$ be the number of moles of
  each gas. Let $m_1$ and $m_2$ be their respective molecular
  masses. Then, we have two equations, where $M$ is the total mass:

  \begin{eqnarray*}
    n_1 + n_2 & = & PV/(RT)\\
    n_1m_2 + n_2m_2 & = & M\\
  \end{eqnarray*}

  The coefficient matrix for this is:

  $$\left[\begin{matrix} 1 & 1 \\ m_1 & m_2 \\\end{matrix}\right]$$

  This has rank one iff $m_1 = m_2$, and rank two otherwise.

  If the coefficient matrix has rank one, then the second equation is
  redundant (note that it must be consistent since we are getting
  these numbers from an actual situation). In other words, we get a
  one-dimensional solution space, with a freely varying parameter. The
  nonnegativity of the number of moles of each gas does constrain the
  parameter to a closed bounded interval instead of all reals, but it
  still has infinitely many candidate values.

  {\em Performance review}: 21 out of 27 got this. 4 chose (A), 2 chose (C).

  {\em Historical note (last time)}: $17$ out of $25$ got this. $4$ chose (C),
  $3$ chose (E), $1$ chose (A).

\item (*) A container has a mix of three known gases with no reactions
  between the gases. The temperature and pressure of the container are
  known. Assume that $PV = nRT$. The volume of the container is also
  known, and so is the total mass of the gases in the container. Under
  what conditions can we predict the amount (say, in the form of the
  number of moles) of each gas that is present from this information?

  \begin{enumerate}[(A)]
  \item It is possible if all three gases have the same molecular mass,
    because in that case, the coefficient matrix of the linear system
    has full rank $3$.
  \item It is possible if all three gases have different molecular
    masses, because in that case, the coefficient matrix of the linear
    system has full rank $3$.
  \item It is possible if all three gases have the same molecular mass,
    because in that case, the coefficient matrix of the linear system
    has rank $2$.
  \item It is possible if all three gases have different molecular
    masses, because in that case, the coefficient matrix of the linear
    system has rank $2$.
  \item It is not possible to deduce the amount of each gas from the
    given information.
  \end{enumerate}

  {\em Answer}: Option (E)

  {\em Explanation}: We have two equations in three
  variables. Explicitly, if the total mass is $M$, the number of moles
  of the three gases are $n_1$, $n_2$, and $n_3$, and the molecular
  masses are $m_1$, $m_2$, and $m_3$, then we have:

  \begin{eqnarray*}
    n_1 + n_2 + n_3 & = & PV/(RT)\\
    n_1m_1 + n_2m_2 + n_3m_3 & = & M \\
  \end{eqnarray*}

  The coefficient matrix is:

  $$\left[\begin{matrix} 1 & 1 & 1 \\ m_1 & m_2 & m_3 \\\end{matrix}\right]$$

  This matrix can have rank either one or two. The rank is one if all
  three gases have the same molecular mass. The rank is two
  otherwise. Note that in either case, the rank is less than three,
  i.e., the system does not have full column rank. Thus, it will not
  be possible to solve the system and uniquely determine the values of
  $n_1$, $n_2$, and $n_3$.

  What if we include the nonnegativity constraints? Even in the
  presence of these constraints, a unique solution is not possible if
  there are nonzero amounts of all gases in the actual solution.

  {\em Performance review}: 10 out of 27 got this. 10 chose (B), 4
  chose (C), 3 chose (A).

  {\em Historical note (last time)}: $10$ out of $25$ got this. $13$ chose (B),
  $1$ each chose (A) and (C).

The branch of chemistry called quantitative analysis has historically
used stoichiometric methods to determine the proportions of various
chemicals present in a given mix. The idea is to use information about
the amounts needed and produced in various reactions to estimate the
quantities of chemicals present (the possible chemicals are first
identified via ``qualitative analysis'' techniques). We generally find
that these conditions give linear systems, and the coefficient
matrices of these systems have (or can be written in a manner as to
have) small integer entries. 

\item (*) Consider a situation where we have a material that is a mix
  (in fixed proportion) of three known chemicals $X$, $Y$, and
  $Z$. Our goal is to find the amount of $X$, $Y$, and $Z$
  present. Suppose we want to set up a collection of experiments so
  that the coefficient matrix is diagonal, i.e., we are effectively
  solving a diagonal system of equations and can recover the
  quantities of each of $X$, $Y$, and $Z$. Which of the following is
  the best approach?  Assume that we can measure, for each reagent,
  the amount of the reagent that gets used up for the reaction(s) to
  proceed to completion, but cannot isolate or separate the outputs
  from each other.

  \begin{enumerate}[(A)]
  \item Choose a single reagent that reacts with all of $X$, $Y$, and
    $Z$.
  \item Choose a single reagent that reacts with only one of $X$, $Y$,
    and $Z$.
  \item Choose three separate reagents, each of which reacts with {\em
    all} of $X$, $Y$, and $Z$.
  \item Choose three separate reagents, each of which reacts only with
    $X$.
  \item Choose three separate reagents, one of which reacts only with
    $X$, one of which reacts only with $Y$, and one of which reacts
    only with $Z$.
  \end{enumerate}

  {\em Answer}: Option (E)

  {\em Explanation}: Consider a matrix with the columns indexed by
  $X$, $Y$, and $Z$ and with the rows indexed by the reagents. The
  entry in each cell of the matrix is the amount of the row reagent
  needed to react with a unit amount of the column substance. It turns
  out that this is the coefficient matrix of the system of
  simultaneous linear equations that we construct.

  In order to get a diagonal matrix, we need to have three reagents
  (so the number of rows equals the number of columns) and further, we
  need to choose them so that the off-diagonal entries are zero, i.e.,
  our first reagent should react only with $X$, our second reagent
  should react only with $Y$, and our third reagent should react only
  with $Z$.

  {\em Performance review}: 24 out of 27 got this. 2 chose (C), 1 chose (A).

  {\em Historical note (last time)}: $22$ out of $25$ got this. $1$ each chose
  (A), (B), and (D).

\item (*) Suppose we are given an aqueous solution with two known
  dissolved substances. There are two different types of
  reactions. One is an acid-base reaction and the other is a redox
  reaction. For both reactions, we can use titrations (separately) to
  deduce the quantity of reagent needed. What type of system should we
  expect to get if only one of the solutes participates in the redox
  reaction but both participate in the acid-base reaction?

  \begin{enumerate}[(A)]
  \item A diagonal system, i.e., the coefficient matrix is a diagonal matrix.
  \item A triangular system, i.e., the coefficient matrix is a
    triangular matrix (whether it is upper or lower triangular depends
    on the order in which we write the rows).
  \item A system of rank one, i.e., the coefficient matrix has rank one.
  \end{enumerate}

  {\em Answer}: Option (B)

  {\em Explanation}: The coefficient matrix has columns corresponding
  to the solutes, and rows corresponding to the reagents, with the
  matrix entries describing the amount of the row reagent consumed per
  unit of the solute. The row for the redox reagent has one entry
  zero. If we write that solute on the left, and that row as the
  second row, we get a matrix of the form:

  $$\left[\begin{matrix} * & * \\ 0 & * \\\end{matrix}\right]$$

  This is upper triangular.

  {\em Performance review}: 24 out of 27 got this. 2 chose (C), 1 chose (A).

  {\em Historical note (last time)}: $20$ out of $25$ got this. $4$ chose (A),
  $1$ chose (C).

\item (*) Suppose we are given an aqueous solution with two known
  dissolved substances. There are two different types of
  reactions. One is an acid-base reaction and the other is a redox
  reaction. For both reactions, we can use titrations (separately) to
  deduce the quantity of reagent needed.Suppose we are given an
  aqueous solution with two known dissolved substances. Suppose both
  solutes participate in both reactions. What should we desire if we
  want to use the data from the two titrations to determine the
  amounts of each of the substances?

  \begin{enumerate}[(A)]
  \item The proportions in which the two substances react should be
    the same for the two reactions.
  \item The proportions in which the two substances react should
    differ for the two reactions.
  \item It does not matter; we will be able to determine the amounts
    of each of the substances in both cases.
  \item It does not matter; we will not be able to determine the
    amounts of each of the substances in either case.
  \end{enumerate}

  {\em Answer}: Option (B)

  {\em Explanation}: If they react in the same proportions, then the
  coefficient matrix will have rank one, i.e., the equations we get
  from the two reactions will convey the same algebraic information.

  {\em Performance review}: 20 out of 27 got this. 4 chose (A), 2
  chose (D), 1 left the question blank.
 
  {\em Historical note (last time)}: $12$ out of $25$ got this. $7$ chose (C),
  $4$ chose (D), $2$ chose (A).

\item {\em Do not discuss this!}: A consumer price index is obtained
  from a ``goods basket'' by multiplying the price of each good in the
  basket by a fixed weight, and then adding up all the price X weight
  products. The weights are kept fixed, but the prices vary from year
  to year. Thus, the consumer price index value itself fluctuates from
  year to year.
  
  What is a good way of modeling this?

  \begin{enumerate}[(A)]
  \item The prices of the various goods in various years are stored
    in a matrix, the weights used in the index are stored in a
    vector, and the consumer price index values arise as the output vector
    of the matrix-vector product.
  \item The weights used in the index are stored in a matrix, the
    prices of the various goods in various years are stored in a
    vector, and the consumer price index values arise as the output
    vector of the matrix-vector product.
  \item The prices of the various goods in various years are stored in
    a matrix, the consumer price index values are stored as a vector,
    and the weights used in the index arise as the output vector
    of the matrix-vector product.
  \item The weights used in the index are stored in a matrix, the
    consumer price index values are stored in a vector, and the prices
    of the various goods in various years arise as the output vector
    of the matrix-vector product.
  \item The consumer price index values are stored in a matrix, the
    prices of the various goods in various years are stored in a
    vector, and the weights used in the index arise as the output
    vector of the matrix-vector product.
  \end{enumerate}

  {\em Answer}: Option (A)

  {\em Explanation}: The weight vector is fixed. There are many price
  vectors, one for each year. It makes sense to store these as
  different rows of a matrix, that is then multiplied with the weight
  vector to give the vector of consumer price index values. The
  intuition is that the fixed vector is chosen as the vector and the
  many different vectors that need to be dotted (in the sense of
  ``taking the dot product'') with the fixed vector are taken as rows
  of a matrix.

  {\em Performance review}: 17 out of 27 got this. 10 chose (B).

  {\em Historical note (last time)}: $4$ out of $25$ got this. $20$ chose (B),
  $1$ chose (E).

\item Amelia wants to choose a healthy balanced diet. She has access
  to $30$ different types of foods. There are $400$ different
  nutrients that she wants a good amount of. Each of the foods that
  Amelia consumes offers a positive amount of each nutrient per unit
  foodstuff. Amelia is interested in meeting the daily value
  requirements for all nutrients. For some nutrients, her daily value
  requirements specify only a minimum. For some nutrients, both a
  minimum and a maximum are specified. Assume that the total amount of
  any nutrient can be obtained by adding up the amounts obtained from
  each of the foodstuffs Amelia consumes. Amelia wants to determine
  how much of each foodstuff she should consume. How should she model
  the situation?

  \begin{enumerate}[(A)]
  \item The matrix with information on the nutritional contents of the
    foodstuffs is a $400 \times 400$ matrix, and the vector of amounts
    of each foodstuff consumed is a $400 \times 1$ column vector.
  \item The matrix with information on the nutritional contents of the
    foodstuffs is a $30 \times 30$ matrix, and the vector of amounts
    of each foodstuff consumed is a $30 \times 1$ column vector.
  \item The matrix with information on the nutritional contents of the
    foodstuffs is a $400 \times 30$ matrix, and the vector of amounts
    of each foodstuff consumed is a $30 \times 1$ column vector.
  \item The matrix with information on the nutritional contents of the
    foodstuffs is a $30 \times 400$ matrix, and the vector of amounts
    of each foodstuff consumed is a $400 \times 1$ column vector.
  \item The matrix with information on the nutritional contents of the
    foodstuffs is a $400 \times 400$ matrix, and the vector of amounts
    of each foodstuff consumed is a $30 \times 1$ column vector.
  \end{enumerate}

  {\em Answer}: Option (C)

  {\em Explanation}: The input food vector is $30 \times 1$, the
  output nutrient vector is $400 \times 1$, so the matrix must be $400
  \times 30$.

  {\em Performance review}: 14 out of 27 got this. 12 chose (D), 1
  left the question blank.

  {\em Historical note (last time)}: $20$ out of $25$ got this. $2$
  each chose (A) and (B), $1$ chose (E).
\end{enumerate}
\end{document}
