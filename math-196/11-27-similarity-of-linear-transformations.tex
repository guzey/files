\documentclass[10pt]{amsart}

%Packages in use
\usepackage{fullpage, hyperref, vipul, enumerate}

%Title details
\title{Take-home class quiz: due Wednesday November 27: Similarity of linear transformations}
\author{Math 196, Section 57 (Vipul Naik)}
%List of new commands

\begin{document}
\maketitle

Your name (print clearly in capital letters): $\underline{\qquad\qquad\qquad\qquad\qquad\qquad\qquad\qquad\qquad\qquad}$

{\bf PLEASE FEEL FREE TO DISCUSS {\em ALL} QUESTIONS.}

This quiz corresponds to material discussed in the lecture notes
titled {\tt Coordinates}. It also corresponds to Section 3.4 of the
text.

Recall that $n \times n$ matrices $A$ and $B$ are termed {\em similar}
if there exists an invertible $n \times n$ matrix $S$ such that $A =
SBS^{-1}$. The relation of matrices being similar is an {\em
  equivalence relation} (please refer to the notes for an explanation
of the terminology).

For these questions, assume $n > 1$, because a lot of phenomena are
much simpler in the case $n = 1$ and you may be misled if you look
only at that case. In other words, just because an equality is true
for $1 \times 1$ matrices, do not assume it is always true. On the
other hand, if you can find {\em counterexamples} to a statement for
$1 \times 1$ matrices, you can probably use that to construct
counterexamples for all sizes of matrices by using scalar matrices.

\begin{enumerate}
\item Which of the following can we say about two (possibly equal,
  possibly distinct) similar $n \times n$ matrices $A$ and $B$? Please
  see Options (D) and (E) before answering.

  \begin{enumerate}[(A)]
  \item $A$ is invertible if and only if $B$ is invertible.
  \item $A$ is nilpotent if and only if $B$ is nilpotent.
  \item $A$ is idempotent if and only if $B$ is idempotent.
  \item All of the above.
  \item None of the above.
  \end{enumerate}

  \vspace{0.1in}
  Your answer: $\underline{\qquad\qquad\qquad\qquad\qquad\qquad\qquad}$
  \vspace{0.1in}

\item Which of the following can we say about two (possibly equal,
  possibly distinct) similar $n \times n$ matrices $A$ and $B$? Please
  see Options (D) and (E) before answering.

  \begin{enumerate}[(A)]
  \item $A$ is scalar if and only if $B$ is scalar.
  \item $A$ is diagonal if and only if $B$ is diagonal.
  \item $A$ is upper triangular if and only if $B$ is upper triangular.
  \item All of the above.
  \item None of the above.
  \end{enumerate}

  \vspace{0.1in}
  Your answer: $\underline{\qquad\qquad\qquad\qquad\qquad\qquad\qquad}$
  \vspace{0.1in}

\item Suppose $A_1,A_2,B_1,B_2$ are $n \times n$ matrices such that
  $A_1$ is similar to $B_1$ and $A_2$ is similar to $B_2$. Which of
  the following is {\em definitely} true? Please see Options (D) and
  (E) before answering.

  \begin{enumerate}[(A)]
  \item $A_1 + A_2$ is similar to $B_1 + B_2$.
  \item $A_1 - A_2$ is similar to $B_1 - B_2$.
  \item $A_1A_2$ is similar to $B_1B_2$.
  \item All of the above.
  \item None of the above.
  \end{enumerate}

  \vspace{0.1in}
  Your answer: $\underline{\qquad\qquad\qquad\qquad\qquad\qquad\qquad}$
  \vspace{0.1in}

\item Suppose $A_1,A_2,B_1,B_2$ are $n \times n$ matrices such that
  $A_1$ is similar to $B_1$ and $A_2$ is similar to $B_2$. Which of
  the following is {\em definitely} true? Please see Options (D) and
  (E) before answering.

  \begin{enumerate}[(A)]
  \item $A_1 + B_1$ is similar to $A_2 + B_2$.
  \item $A_1 - B_1$ is similar to $A_2 - B_2$.
  \item $A_1B_1$ is similar to $A_2B_2$.
  \item All of the above.
  \item None of the above.
  \end{enumerate}

  \vspace{0.1in}
  Your answer: $\underline{\qquad\qquad\qquad\qquad\qquad\qquad\qquad}$
  \vspace{0.1in}

\item Suppose $A$ and $B$ are both $n \times n$ matrices
  (but they are not given to be similar). Which of the following
  holds?

  \begin{enumerate}[(A)]
  \item $A$ is similar to $B$ if and only if $-A$ is similar to $-B$.
  \item If $A$ is similar to $B$, then $-A$ is similar to
    $-B$. However, $-A$ being similar to $-B$ does not
    imply that $A$ is similar to $B$.
  \item If $-A$ is similar to $-B$, then $A$ is similar to
    $B$. However, $A$ being similar to $B$ does not imply that
    $-A$ is similar to $-B$.
  \item $A$ being similar to $B$ does not imply that $-A$ is
    similar to $-B$. Also, $-A$ being similar to $-B$ does
    not imply that $A$ is similar to $B$.
  \end{enumerate}

  \vspace{0.1in}
  Your answer: $\underline{\qquad\qquad\qquad\qquad\qquad\qquad\qquad}$
  \vspace{0.1in}
 
\item Suppose $A$ and $B$ are both $n \times n$ matrices (but they are
  not given to be similar). Which of the following holds?

  \begin{enumerate}[(A)]
  \item $A$ is similar to $B$ if and only if $2A$ is similar to $2B$.
  \item If $A$ is similar to $B$, then $2A$ is similar to
    $2B$. However, $2A$ being similar to $2B$ does not imply that
    $A$ is similar to $B$.
  \item If $2A$ is similar to $2B$, then $A$ is similar to
    $B$. However, $A$ being similar to $B$ does not imply that $2A$
    is similar to $2B$.
  \item $A$ being similar to $B$ does not imply that $2A$ is similar
    to $2B$. Also, $2A$ being similar to $2B$ does not imply that
    $A$ is similar to $B$.
  \end{enumerate}

  \vspace{0.1in}
  Your answer: $\underline{\qquad\qquad\qquad\qquad\qquad\qquad\qquad}$
  \vspace{0.1in}

\item Suppose $A$ and $B$ are both invertible $n \times n$ matrices
  (but they are not given to be similar). Which of the following
  holds?

  \begin{enumerate}[(A)]
  \item $A$ is similar to $B$ if and only if $A^{-1}$ is similar to $B^{-1}$.
  \item If $A$ is similar to $B$, then $A^{-1}$ is similar to
    $B^{-1}$. However, $A^{-1}$ being similar to $B^{-1}$ does not
    imply that $A$ is similar to $B$.
  \item If $A^{-1}$ is similar to $B^{-1}$, then $A$ is similar to
    $B$. However, $A$ being similar to $B$ does not imply that
    $A^{-1}$ is similar to $B^{-1}$
  \item $A$ being similar to $B$ does not imply that $A^{-1}$ is
    similar to $B^{-1}$. Also, $A^{-1}$ being similar to $B^{-1}$ does
    not imply that $A$ is similar to $B$.
  \end{enumerate}

  \vspace{0.1in}
  Your answer: $\underline{\qquad\qquad\qquad\qquad\qquad\qquad\qquad}$
  \vspace{0.1in}

\item Suppose $A$ and $B$ are both $n \times n$ matrices (but they are
  not given to be similar). Which of the following holds?

  \begin{enumerate}[(A)]
  \item $A$ is similar to $B$ if and only if $A^2$ is similar to $B^2$.
  \item If $A$ is similar to $B$, then $A^2$ is similar to
    $B^2$. However, $A^2$ being similar to $B^2$ does not imply that
    $A$ is similar to $B$.
  \item If $A^2$ is similar to $B^2$, then $A$ is similar to
    $B$. However, $A$ being similar to $B$ does not imply that $A^2$
    is similar to $B^2$.
  \item $A$ being similar to $B$ does not imply that $A^2$ is similar
    to $B^2$. Also, $A^2$ being similar to $B^2$ does not imply that
    $A$ is similar to $B$.
  \end{enumerate}

  \vspace{0.1in}
  Your answer: $\underline{\qquad\qquad\qquad\qquad\qquad\qquad\qquad}$
  \vspace{0.1in}

\item Suppose $A$ and $B$ are $n \times n$ matrices (but they are not
  given to be similar and they are not given to be invertible). We say
  that $A$ and $B$ are {\em quasi-similar} (not a standard term!) if
  there exist $n \times n$ matrices $C$ and $D$ such that $A = CD$ and
  $B = DC$. What can we say is the relation between being similar and
  being quasi-similar?

  \begin{enumerate}[(A)]
  \item $A$ and $B$ are similar if and only if they are quasi-similar.
  \item If $A$ and $B$ are similar, they are quasi-similar. However,
    the converse is not necessarily true: $A$ and $B$ may be
    quasi-similar but not similar.
  \item If $A$ and $B$ are quasi-similar, they are similar. However,
    the converse is not necessarily true: $A$ and $B$ may be similar
    but not quasi-similar.
  \item Neither implies the other. $A$ and $B$ may be similar but not
    quasi-similar. Also, $A$ and $B$ may be quasi-similar but not
    similar.
  \end{enumerate}

  \vspace{0.1in}
  Your answer: $\underline{\qquad\qquad\qquad\qquad\qquad\qquad\qquad}$
  \vspace{0.1in}

\item With the notion of quasi-similar as defined in the preceding
  question, what can we say about the relation between being similar
  and being quasi-similar for $n \times n$ matrices $A$ and $B$ that
  are both given to be {\em invertible}?

  \begin{enumerate}[(A)]
  \item $A$ and $B$ are similar if and only if they are quasi-similar.
  \item If $A$ and $B$ are similar, they are quasi-similar. However,
    the converse is not necessarily true: $A$ and $B$ may be
    quasi-similar but not similar.
  \item If $A$ and $B$ are quasi-similar, they are similar. However,
    the converse is not necessarily true: $A$ and $B$ may be similar
    but not quasi-similar.
  \item Neither implies the other. $A$ and $B$ may be similar but not
    quasi-similar. Also, $A$ and $B$ may be quasi-similar but not
    similar.
  \end{enumerate}

  \vspace{0.1in}
  Your answer: $\underline{\qquad\qquad\qquad\qquad\qquad\qquad\qquad}$
  \vspace{0.1in}

\item Suppose $A$ and $B$ are two $n \times n$ matrices. Which of the
  following best describes the relation between similarity and having
  the same rank?

  \begin{enumerate}[(A)]
  \item $A$ and $B$ are similar if and only if they have the same rank.
  \item If $A$ and $B$ are similar, then they have the same
    rank. However, it is possible for $A$ and $B$ to have the same
    rank but not be similar.
  \item If $A$ and $B$ have the same rank, then they are
    similar. However, it is possible for $A$ and $B$ to be similar but
    not have the same rank.
  \item $A$ and $B$ may be similar but have different ranks. Also, $A$
    and $B$ may have the same rank but not be similar.
  \end{enumerate}

  \vspace{0.1in}
  Your answer: $\underline{\qquad\qquad\qquad\qquad\qquad\qquad\qquad}$
  \vspace{0.1in}

\item Suppose $A$ and $B$ are two $n \times n$ matrices. Which of the
  following best describes the relation between quasi-similarity and having
  the same rank?

  \begin{enumerate}[(A)]
  \item $A$ and $B$ are quasi-similar if and only if they have the same rank.
  \item If $A$ and $B$ are quasi-similar, then they have the same
    rank. However, it is possible for $A$ and $B$ to have the same
    rank but not be quasi-similar.
  \item If $A$ and $B$ have the same rank, then they are
    quasi-similar. However, it is possible for $A$ and $B$ to be
    quasi-similar but not have the same rank.
  \item $A$ and $B$ may be quasi-similar but have different ranks. Also, $A$
    and $B$ may have the same rank but not be quasi-similar.
  \end{enumerate}

  \vspace{0.1in}
  Your answer: $\underline{\qquad\qquad\qquad\qquad\qquad\qquad\qquad}$
  \vspace{0.1in}

\end{enumerate}
\end{document}
