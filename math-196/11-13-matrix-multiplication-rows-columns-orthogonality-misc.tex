\documentclass[10pt]{amsart}

%Packages in use
\usepackage{fullpage, hyperref, vipul, enumerate}

%Title details
\title{Take-home class quiz: due Wednesday November 13: Matrix multiplication: rows, columns, orthogonality, and other miscellanea}
\author{Math 196, Section 57 (Vipul Naik)}
%List of new commands

\begin{document}
\maketitle

Your name (print clearly in capital letters): $\underline{\qquad\qquad\qquad\qquad\qquad\qquad\qquad\qquad\qquad\qquad}$

{\bf PLEASE FEEL FREE TO DISCUSS {\em ALL} QUESTIONS.}

The purpose of this quiz is two-fold. First, many of the ideas related
to matrix multiplication are at the stage where a bit of review will
help prevent their fading out. Drawing from the best research on {\em
  spaced repetition} (see for instance
\url{http://en.wikipedia.org/wiki/Spaced_repetition}) we will try to
recall some of the stuff. But with a twist, because we consider it
from a somewhat different angle.

Second, the new angle will also turn out to be useful for later
material.

For Questions 1-5: Given a $n$-dimensional vector $\langle
a_1,a_2,\dots,a_n \rangle \in \R^n$, the vector can be interpreted as
a $n \times 1$ matrix (a column vector). This is the default
interpretation. But there are also two other interpretations: as a $1
\times n$ matrix (a row vector) and as a diagonal $n \times n$ matrix.

Also note that for Questions 1-5, all the three ways of representing
vectors coincide with each other for $n = 1$, so the questions are
uninteresting for $n = 1$ because all answer options are
equivalent. You may therefore assume that $n > 1$ for these questions,
though obviously the correct answers are correct for $n = 1$ as well.

\begin{enumerate}
\item Suppose I want to add two vectors $\vec{a} = \langle a_1, a_2,\dots,a_n
  \rangle$ and $\vec{b} = \langle b_1,b_2,\dots,b_n \rangle$ to obtain the
  output vector $\langle a_1 + b_1, a_2 + b_2, \dots, a_n + b_n
  \rangle$ using matrix addition. What format (row vector, column
  vector, or diagonal matrix) should I use? Please see Option (D)
  before answering and select the option that best describes your view.

  \begin{enumerate}[(A)]
  \item Represent both $\vec{a}$ and $\vec{b}$ as row vectors and
    interpret the sum as a row vector.
  \item Represent both $\vec{a}$ and $\vec{b}$ as column vectors and
    interpret the sum as a column vector.
  \item Represent both $\vec{a}$ and $\vec{b}$ as diagonal matrices and
    interpret the sum as a diagonal matrix.
  \item We can use any of the above.
  \end{enumerate}

  \vspace{0.1in}
  Your answer: $\underline{\qquad\qquad\qquad\qquad\qquad\qquad\qquad}$
  \vspace{0.1in}

\item Suppose I want to perform coordinate-wise multiplication on two
  vectors. Explicitly, I have two vectors $\vec{a} = \langle a_1,
  a_2,\dots,a_n \rangle$ and $\vec{b} = \langle b_1,b_2,\dots,b_n
  \rangle$ and I want to obtain the output vector $\langle a_1b_1,
  a_2b_2, \dots, a_nb_n \rangle$ using matrix multiplication (with the
  matrix for $\vec{a}$ written on the left and the matrix for
  $\vec{b}$ written on the right). What format (row vector, column
  vector, or diagonal matrix) should I use?  Please see Option (D)
  before answering and select the option that best describes your
  view.

  \begin{enumerate}[(A)]
  \item Represent both $\vec{a}$ and $\vec{b}$ as row vectors and
    interpret the matrix product as a row vector.
  \item Represent both $\vec{a}$ and $\vec{b}$ as column vectors and
    interpret the matrix product as a column vector.
  \item Represent both $\vec{a}$ and $\vec{b}$ as diagonal matrices and
    interpret the matrix product as a diagonal matrix.
  \item We can use any of the above.
  \end{enumerate}

  \vspace{0.1in}
  Your answer: $\underline{\qquad\qquad\qquad\qquad\qquad\qquad\qquad}$
  \vspace{0.1in}

\item Suppose I am given two vectors $\vec{a} = \langle a_1,
  a_2,\dots,a_n \rangle$ and $\vec{b} = \langle b_1,b_2,\dots,b_n
  \rangle$ and I want to obtain a $1 \times 1$ matrix with entry
  $\sum_{i=1}^n a_ib_i$ using matrix multiplication (with the matrix
  for $\vec{a}$ written on the left and the matrix for $\vec{b}$
  written on the right). What format (row vector, column vector, or
  diagonal matrix) should I use?
  \begin{enumerate}[(A)]
  \item Represent both $\vec{a}$ and $\vec{b}$ as row vectors.
  \item Represent both $\vec{a}$ and $\vec{b}$ as column vectors.
  \item Represent both $\vec{a}$ and $\vec{b}$ as diagonal matrices.
  \item Represent $\vec{a}$ as a row vector and $\vec{b}$ as a column
    vector.
  \item Represent $\vec{a}$ as a column vector and $\vec{b}$ as a row vector.
  \end{enumerate}

  \vspace{0.1in}
  Your answer: $\underline{\qquad\qquad\qquad\qquad\qquad\qquad\qquad}$
  \vspace{0.1in}

\item Suppose I am given three vectors $\vec{a} = \langle
  a_1,a_2,\dots,a_n \rangle$, $\vec{b} = \langle b_1,b_2,\dots,b_n
  \rangle$, and $\vec{c} = \langle c_1,c_2,\dots,c_n \rangle$. I want
  to obtain a $1 \times 1$ matrix with entry $\sum_{i=1}^n
  (a_ib_ic_i)$ using matrix multiplication (with the matrix for
  $\vec{a}$ written on the left, the matrix for $\vec{b}$ written in
  the middle, and the matrix for $\vec{c}$ written on the right). What
  format should I use?

  \begin{enumerate}[(A)]
  \item $\vec{a}$ as a row vector, $\vec{b}$ as a column vector,
    $\vec{c}$ as a diagonal matrix.
  \item $\vec{a}$ as a column vector, $\vec{b}$ as a row vector,
    $\vec{c}$ as a diagonal matrix.
  \item $\vec{a}$ as a diagonal matrix, $\vec{b}$ as a row vector,
    $\vec{c}$ as a column vector.
  \item $\vec{a}$ as a column vector, $\vec{b}$ as a diagonal matrix,
    $\vec{c}$ as a row vector.
  \item $\vec{a}$ as a row vector, $\vec{b}$ as a diagonal matrix,
    $\vec{c}$ as a column vector.
  \end{enumerate}

  \vspace{0.1in}
  Your answer: $\underline{\qquad\qquad\qquad\qquad\qquad\qquad\qquad}$
  \vspace{0.5in}

  The next few questions rely on the concept of orthogonality ({\em
    orthogonal} is a synonym for {\em perpendicular} or {\em at right
    angles}). We say that two vectors (of the same dimension) are
  orthogonal if their dot product is zero. By this definition, the
  zero vector of a given dimension is orthogonal to every vector of
  that dimension. Note that it does not make sense to talk of
  orthogonality for vectors with different dimensions, i.e., with
  different numbers of coordinates.

\item Suppose $A$ is a $n \times m$ matrix. We can think of solving
  the system $A\vec{x} = \vec{0}$ (where $\vec{x}$ is a $m \times 1$
  column vector of unknowns) as trying to find all the vectors
  orthogonal to all the vectors in a given set of vectors. What set of
  vectors is that?

  \begin{enumerate}[(A)]
  \item The set of row vectors of $A$, i.e., the rows of $A$, viewed
    as $m$-dimensional vectors.
  \item The set of column vectors of $A$, i.e., the columns of $A$,
    viewed as $n$-dimensional vectors.
  \end{enumerate}

  \vspace{0.1in}
  Your answer: $\underline{\qquad\qquad\qquad\qquad\qquad\qquad\qquad}$
  \vspace{0.1in}

\item Suppose $A$ is a $p \times q$ matrix and $B$ is a $q \times r$
  matrix where $p$, $q$, and $r$ are positive integers. The matrix
  product $AB$ is a $p \times r$ matrix. What orthogonality condition
  corresponds to the condition that the matrix product $AB$ is a zero
  matrix (i.e., all its entries are zero)?

  \begin{enumerate}[(A)]
  \item Every row of $A$ is orthogonal to every row of $B$.
  \item Every row of $A$ is orthogonal to every column of $B$.
  \item Every column of $A$ is orthogonal to every row of $B$.
  \item Every column of $A$ is orthogonal to every column of $B$.
  \end{enumerate}

  \vspace{0.1in}
  Your answer: $\underline{\qquad\qquad\qquad\qquad\qquad\qquad\qquad}$
  \vspace{0.1in}

\item Suppose $A$ is an invertible $n \times n$ square matrix. Which
  of the following correctly characterizes the $n \times n$ matrix
  $A^{-1}$ using orthogonality? Recall that $AA^{-1}$ and $A^{-1}A$
  are both equal to the $n \times n$ identity matrix.

  \begin{enumerate}[(A)]
  \item For every $i$ in $\{ 1,2,\dots,n\}$, the $i^{th}$ row of $A$ is
    orthogonal to the $i^{th}$ row of $A^{-1}$. The dot product of the
    $i^{th}$ row of $A$ and the $j^{th}$ row of $A^{-1}$ for distinct
    $i,j$ in $\{1,2,\dots,n\}$ equals $1$.
  \item For every $i$ in $\{ 1,2,\dots,n\}$, the $i^{th}$ column of $A$ is
    orthogonal to the $i^{th}$ column of $A^{-1}$. The dot product of the
    $i^{th}$ row of $A$ and the $j^{th}$ column of $A^{-1}$ for distinct
    $i,j$ in $\{1,2,\dots,n\}$ equals $1$.
 \item For every distinct $i,j$ in $\{1,2,\dots,n\}$, the $i^{th}$ row
   of $A$ is orthogonal to the $j^{th}$ row of $A^{-1}$. The dot
   product of the $i^{th}$ row of $A$ with the $i^{th}$ row of
   $A^{-1}$ equals $1$.
   \item For every $i$ in $\{ 1,2,\dots,n\}$, the $i^{th}$ row of $A$ is
    orthogonal to the $i^{th}$ column of $A^{-1}$. The dot product of the
    $i^{th}$ row of $A$ and the $j^{th}$ column of $A^{-1}$ for distinct
    $i,j$ in $\{1,2,\dots,n\}$ equals $1$.
  \item For every distinct $i,j$ in $\{1,2,\dots,n\}$, the $i^{th}$
    row of $A$ is orthogonal to the $j^{th}$ column of $A^{-1}$. The
    dot product of the $i^{th}$ row of $A$ and the $i^{th}$ column of
    $A^{-1}$ equals $1$.
  \end{enumerate}

  \vspace{0.1in}
  Your answer: $\underline{\qquad\qquad\qquad\qquad\qquad\qquad\qquad}$
  \vspace{0.5in}

  The remaining questions review your skills at abstract behavior
  prediction.

\item Suppose $n$ is a positive integer greater than $1$. Which of the
  following is always true for two invertible $n \times n$ matrices
  $A$ and $B$?

  \begin{enumerate}[(A)]
  \item $A + B$ is invertible, and $(A + B)^{-1} = A^{-1} + B^{-1}$
  \item $A + B$ is invertible, and $(A + B)^{-1} = B^{-1} + A^{-1}$
  \item $A + B$ is invertible, though neither of the formulas of the
    preceding two options is correct
  \item $AB$ is invertible, and $(AB)^{-1} = A^{-1}B^{-1}$
  \item $AB$ is invertible, and $(AB)^{-1} = B^{-1}A^{-1}$
  \end{enumerate}

  \vspace{0.1in}
  Your answer: $\underline{\qquad\qquad\qquad\qquad\qquad\qquad\qquad}$
  \vspace{0.1in}

\item Suppose $n$ is a positive integer greater than $1$. For a
  nilpotent $n \times n$ matrix $C$, define the {\em nilpotency} of
  $C$ as the smallest positive integer $r$ such that $C^r = 0$. Note
  that the nilpotency is not defined for a non-nilpotent matrix. Given
  two $n \times n$ matrices $A$ and $B$, what is the relation between
  the nilpotencies of $AB$ and $BA$?

  \begin{enumerate}[(A)]
  \item $AB$ is nilpotent if and only if $BA$ is nilpotent, and if
    so, their nilpotencies must be equal.
  \item $AB$ is nilpotent if and only if $BA$ is nilpotent, and if so,
    their nilpotencies must differ by $1$.
  \item $AB$ is nilpotent if and only if $BA$ is nilpotent, and if so,
    their nilpotencies must either be equal or differ by $1$.
  \item It is possible for $AB$ to be nilpotent and $BA$ to be
    non-nilpotent; however, {\em if} both are nilpotent, then their
    nilpotencies must be equal.
  \item It is possible for $AB$ to be nilpotent and $BA$ to be
    non-nilpotent, however {\em if} both are nilpotent, then their
    nilpotencies must differ by $1$.
  \end{enumerate}

  \vspace{0.1in}
  Your answer: $\underline{\qquad\qquad\qquad\qquad\qquad\qquad\qquad}$
  \vspace{0.1in}

\item What is the smallest $n$ for which there exist examples of
  invertible $n \times n$ matrices $A$ and $B$ such that $A \ne B$ but
  $A^2 = B^2$?

  \begin{enumerate}[(A)]
  \item $1$
  \item $2$
  \item $3$
  \item $4$
  \item This is not possible for any $n$.
  \end{enumerate}

  \vspace{0.1in}
  Your answer: $\underline{\qquad\qquad\qquad\qquad\qquad\qquad\qquad}$
  \vspace{0.1in}

\item What is the smallest $n$ for which there exist examples of
  invertible $n \times n$ matrices $A$ and $B$ such that $A \ne B$ but
  $A^3 = B^3$?

  \begin{enumerate}[(A)]
  \item $1$
  \item $2$
  \item $3$
  \item $4$
  \item This is not possible for any $n$.
  \end{enumerate}

  \vspace{0.1in}
  Your answer: $\underline{\qquad\qquad\qquad\qquad\qquad\qquad\qquad}$
  \vspace{0.1in}

\item What is the smallest $n$ for which there exist examples of
  invertible $n \times n$ matrices $A$ and $B$ such that $A \ne B$ but
  $A^2 = B^2$ and $A^3 = B^3$?

  \begin{enumerate}[(A)]
  \item $1$
  \item $2$
  \item $3$
  \item $4$
  \item This is not possible for any $n$.
  \end{enumerate}

  \vspace{0.1in}
  Your answer: $\underline{\qquad\qquad\qquad\qquad\qquad\qquad\qquad}$
  \vspace{0.1in}

\item What is the smallest $n$ for which there exist examples of (not
  necessarily invertible) $n \times n$ matrices $A$ and $B$ such that
  $A \ne B$ but $A^2 = B^2$ and $A^3 = B^3$?

  \begin{enumerate}[(A)]
  \item $1$
  \item $2$
  \item $3$
  \item $4$
  \item This is not possible for any $n$.
  \end{enumerate}

  \vspace{0.1in}
  Your answer: $\underline{\qquad\qquad\qquad\qquad\qquad\qquad\qquad}$
  \vspace{0.1in}

\end{enumerate}
\end{document}
