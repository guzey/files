\documentclass[10pt]{amsart}
\usepackage{fullpage,hyperref,vipul,graphicx}
\title{Image and kernel of a linear transformation}
\author{Math 196, Section 57 (Vipul Naik)}

\begin{document}
\maketitle

{\bf Corresponding material in the book}: Section 3.1.

\section*{Executive summary}

\begin{enumerate}
\item For a function $f:A \to B$, we call $A$ the domain, $B$ the
  co-domain, $f(A)$ the range, and $f^{-1}(b)$, for any $b \in B$, the
  fiber (or inverse image or pre-image) of $b$. For a subset $S$ of
  $B$, $f^{-1}(B) = \bigcup_{b \in S} f^{-1}(b)$.
\item The sizes of fibers can be used to characterize injectivity
  (each fiber has size at most one), surjectivity (each fiber is
  non-empty), and bijectivity (each fiber has size exactly one).
\item Composition rules: composite of injective is injective,
  composite of surjective is surjective, composite of bijective is
  bijective.
\item If $g \circ f$ is injective, then $f$ must be injective.
\item If $g \circ f$ is surjective, then $g$ must be surjective.
\item If $g \circ f$ is bijective, then $f$ must be injective and $g$
  must be surjective.
\item Finding the fibers for a function of one variable can be
  interpreted geometrically (intersect graph with a horizontal line)
  or algebraically (solve an equation).
\item For continuous functions of one variable defined on all of $\R$,
  being injective is equivalent to being increasing throughout or
  decreasing throughout. More in the lecture notes, sections 2.2-2.5.
\item A vector $\vec{v}$ is termed a {\em linear combination} of the
  vectors $\vec{v_1}, \vec{v_2}, \dots, \vec{v_r}$ if there exist real
  numbers $a_1,a_2,\dots,a_r \in \R$ such that $\vec{v} = a_1\vec{v_1}
  + a_2\vec{v_2} + \dots + a_r\vec{v_r}$. We use the term {\em
    nontrivial} if the coefficients are not all zero.
\item A subspace of $\R^n$ is a subset that contains the zero vector
  and is closed under addition and scalar multiplication.
\item The span of a set of vectors is defined as the set of all
  vectors that can be written as linear combinations of vectors in
  that set. The span of any set of vectors is a subspace.
\item A spanning set for a subspace is defined as a subset of the
  subspace whose span is the subspace.
\item Adding more vectors either preserves or increases the span. If
  the new vectors are in the span of the previous vectors, it
  preserves the span, otherwise, it increases it.
\item The kernel and image (i.e., range) of a linear transformation
  are respectively subspaces of the domain and co-domain. The kernel
  is defined as the inverse image of the zero vector.
\item The column vectors of the matrix of a linear transformation form
  a spanning set for the image of that linear transformation.
\item To find a spanning set for the kernel, we convert to rref, then
  find the solutions parametrically (with zero as the augmenting
  column) then determine the vectors whose linear combinations are
  being discussed. The parameters serve as the coefficients for the
  linear combination. There is a shortening of this method. (See the
  lecture notes, Section 4.4, for a simple example done the long way
  and the short way).
\item The fibers for a linear transformation are translates of the
  kernel. Explicitly, the inverse image of a vector is either empty or
  is of the form (particular vector) + (arbitrary element of the
  kernel).
\item The dimension of a subspace of $\R^n$ is defined as the minimum
  possible size of a spanning set for that subspace.
\item For a linear transformation $T:\R^m \to \R^n$ with $n \times m$
  matrix having rank $r$, the dimension of the kernel is $m - r$ and
  the dimension of the image is $r$. Full row rank $r = n$ means
  surjective (image is all of $\R^n$) and full column rank $r = m$
  means injective (kernel is zero subspace).
\item We can define the intersection and sum of subspaces of $\R^n$.
\item The kernel of $T_1 + T_2$ contains the intersection of the
  kernels of $T_1$ and $T_2$. More is true (see the lecture notes).
\item The image of $T_1 + T_2$ is contained in the sum of the images
  of $T_1$ and $T_2$. More is true (see the lecture notes).
\item The dimension of the inverse image $T^{-1}(X)$ of any subspace $X$
  of $\R^n$ under a linear transformation $T:\R^m \to \R^n$ satisfies:

$$\operatorname{dim}(\operatorname{Ker}(T)) \le \operatorname{dim}(T^{-1}(X)) \le \operatorname{dim}(\operatorname{Ker}(T)) + \operatorname{dim}(X)$$

  The upper bound holds if $X$ lies inside the image of $T$.
\item Please read through the lecture notes thoroughly, since the
  summary here is very brief and inadequate.
\end{enumerate}


\section{Image (range) and inverse images (fibers) for a function}

\subsection{Domain, range (image), and co-domain}

Suppose $f:A \to B$ is a function (of any sort). We call $A$ the {\em
  domain} of $f$ and we call $B$ the {\em co-domain} of $f$. Note that
it is not necessary that every element of $B$ occur as the image of an
element of $A$ under $f$. The subset of $B$ that comprises the
elements of the form $f(x), x \in A$ is termed the {\em range} of $f$
or the {\em image} of $f$. Note that if the range is all of $B$, we
say that $f$ is {\em surjective}.

\subsection{Fibers of a function}

As before, $f: A \to B$ is a function. For any $b \in B$, define
$f^{-1}(b)$ as the set $\{ a \in A \mid f(a) = b \}$. We call
$f^{-1}(b)$ the {\em fiber} of $f$ corresponding to $b$. Other words
used are {\em pre-image} and {\em inverse image}. Note the following:

\begin{itemize}
\item $f^{-1}(b)$ is non-empty if and only if $b$ is in the range of
  $f$. In particular, $f^{-1}(b)$ is non-empty for {\em every} $b \in
  B$ if and only if $f$ is surjective.
\item $f^{-1}(b)$ has size {\em at most one} for every $b \in B$ if
  and only if $f$ is injective.
\item $f^{-1}(b)$ has size {\em exactly one} for every $b \in B$ if
  and only if $f$ is bijective, i.e., invertible.
\end{itemize}

We sometimes say that a set map is {\em uniform} (not a standard term)
if all its non-empty fibers have the same size. Note that injective
set maps are uniform. However, there may exist uniform non-injective
set maps. For instance, a set map where all the fibers have size two
is uniform. The idea of uniformity based on cardinality works well for
maps of finite sets. For infinite sets, we may use structurally
dependent concepts of uniformity since cardinality (set size) is too
crude a size measure in the infinite context. We will revisit this
idea in the context of linear transformations.

\subsection{Composition, injectivity, surjectivity, and one-sided inverses}

We've discussed in the past that a function has a two-sided inverse
(i.e., is invertible) if and only if it is bijective. We now discuss
some material that will help us understand one-sided inverses.

First, a slight extension of the terminology introduced
previously. Consider a function $f:A \to B$ For a subset $S$ of $B$,
define $f^{-1}(S) = \{ a \in A \mid f(a) \in S \}$. It is clear that
$f^{-1}(S) = \bigcup_{b \in S} f^{-1}(b)$. Pictorially, it is the
union of the fibers over all the points in $S$.

Now, consider a situation where we are composing two functions:

$$f: A \to B, g: B \to C, \text{composite } g \circ f: A \to C$$

Given $c \in C$, let's try to figure out what $(g \circ f)^{-1}(c)$
would look like. We must find all $a \in A$ such that $g(f(a)) =
c$. In order to do that, we first find the $b \in B$ such that $g(b) =
c$. In other words, we first find $g^{-1}(c)$. Then, we find, for each
$b \in g^{-1}(c)$, the $a \in A$ such that $f(a) = b$. The upshot is
that we are computing $f^{-1}(g^{-1}(c))$, where the inner computation
yields a subset to which we apply the outer inverse function.

Pictorially, we first take the fiber over $c$ for $g$. This is a
subset of $B$. For each element here, we take the fiber in $A$. Then,
we take the union of the fibers to get $(g \circ f)^{-1}(a)$.

We note the following facts regarding composition, injectivity and
surjectivity:

\begin{enumerate}
\item Suppose $f:A \to B$ and $g:B \to C$ are both injective
  functions. Then, the composite function $g \circ f:A \to C$ is also
  an injective function. Here's the proof:

  {\em Want to show}: If $a_1,a_2 \in A$ satisfy $g(f(a_1)) =
  g(f(a_2))$, then $a_1 = a_2$.

  {\em Proof}: We ``peel off the layers'' one by one. Explicitly, we
  begin with:

  $$g(f(a_1)) = g(f(a_2))$$

  We use that $g$ is injective as a function from $B$ to $C$ to
  conclude that we can cancel the $g$, so we obtain that:

  $$f(a_1) = f(a_2)$$

  We now use the injectivity of $f$ as a function from $A$ to $B$ and
  obtain that:

  $$a_1 = a_2$$

  as desired.
\item Suppose $f:A \to B$ and $g:B \to C$ are both surjective
  functions. Then, the composite function $g \circ f:A \to C$ is also
  a surjective function. The proof follows:

  {\em Want to show}: For any $c \in C$, there exists $a \in A$ such
  that $g(f(a)) = c$.

  {\em Proof}:

  \begin{itemize}
  \item Since $g$ is surjective, the set $g^{-1}(c)$ is non-empty,
    i.e., there exists $b \in B$ such that $g(b) = c$.
  \item Since $f$ is surjective, the set $f^{-1}(b)$ is non-empty,
    i.e., there exists $a \in A$ such that $f(a) = b$.
  \item The element $a$ obtained this way is the one we desire:
    $g(f(a)) = g(b) = c$.
  \end{itemize}

  In both the injectivity and surjectivity cases, we use the
  information about the individual functions one step at a time.
\item Suppose $f:A \to B$ and $g:B \to C$ are both bijective
  functions. Then, the composite function $g \circ f:A \to C$ is also
  a bijective function.

  This follows by combining the corresponding statements for
  injectivity and surjectivity, i.e., points (1) and (2) above.
\item Suppose $f:A \to B$ and $g:B \to C$ are functions such that the
  composite $g \circ f: A \to C$ is injective. Then, $f$ must be
  injective. This is easiest to prove by contradiction: we can show
  that if $f$ is not injective, then $g \circ f$ is not
  injective. Here's how: if $f$ has a collision, i.e., two different
  elements $a_1, a_2 \in A$ satisfying $f(a_1) = f(a_2)$, then $g
  \circ f$ also has a collision for the same pair of inputs, i.e., we
  also have $g(f(a_1)) = g(f(a_2))$.

  In fact, the proof shows something stronger: any collision for $f$
  remains a collision for $g \circ f$.
\item Suppose $f:A \to B$ and $g:B \to C$ are functions such that the
  composite $g \circ f:A \to C$ is surjective. Then, $g$ is
  surjective. 

  To see this, note that if everything in $C$ is hit by the composite,
  it must be hit by $g$. Explicitly, if every $c \in C$ is of the form
  $g(f(a))$ for $a \in A$, then in particular it is $g(b)$ where $b =
  f(a)$.

  In fact, the proof shows something slightly stronger: the range of
  $g \circ f$ is contained within the range of $g$.
\item Suppose $f:A \to B$ and $g:B \to C$ are functions such that the
  composite function $g \circ f:A \to C$ is bijective. Then, $g$ is surjective
  and $f$ is injective. This follows from the preceding two
  observations ((4) and (5) in the list).

\item Suppose $f:A \to B$ and $g:B \to A$ are functions such that the
  composite function $g \circ f: A \to A$ is the identity map on
  $A$. In other words, $g$ is a left inverse to $f$, and $f$ is a
  right inverse to $g$. Then, $f$ is injective and $g$ is
  surjective. This is a direct consequence of (6), along with the
  observation that the identity map from any set to itself is
  bijective.
\end{enumerate}

{\em There was a lot of confusion about points (4) and (5) of the list
  in class, suggesting that it is pretty hard to ``get'' these by
  yourself. So please review the reasoning carefully and convince
  yourself of it}.

Point (7) raises two interesting converse questions: 

\begin{itemize}
\item Suppose $f: A \to B$ is injective. Does there exist $g: B \to A$
  such that $g \circ f$ is the identity map on $A$? The answer turns
  out to be {\em yes} for set maps. The idea is to construct $g$ as
  essentially the inverse function to $f$ on the range of $f$, and
  define it whatever way we please on the rest of $B$. Note that if
  $f$ is injective but not surjective (so that the range of $f$ is not
  all of $B$) then this one-sided inverse to $f$ is non-unique,
  because we have flexibility regarding where we send all the elements
  that are in $B$ but outside $f(A)$.
\item Suppose $g:B \to A$ is surjective. Does there exist $f:A \to B$
  such that $g \circ f$ is the identity map on $A$? The answer turns
  out to be {\em yes} for set maps. The idea is to construct $f$ as
  follows: for every element $a \in A$, let $f(a)$ be any element $b$
  satisfying $g(b) = a$. We always {\em can} pick such an element,
  because $g$ is surjective.
\end{itemize}

\section{Some warm-up examples involving functions of one variable}

\subsection{Real-valued functions of one variable}

Before we proceed to apply the ideas above to linear transformations,
let us consider what the ideas tell us in the context of continuous
functions from $\R$ to $\R$. Since you are quite familiar with
single-variable precalculus and calculus, this will be helpful.

For a function $f$ with domain a subset of $\R$ and co-domain $\R$ (so
$f$ is a real-valued function of one variable), we can find the fiber
$f^{-1}(y_0)$ for a particular $y_0 \in \R$ in either of these ways:

\begin{itemize}
\item {\em Graphically}: We draw the line $y = y_0$, find all its
  points of intersection with the graph of $f$, and then look at the
  $x$-coordinates of those points of intersection. The set of all the
  $x$-coordinates is the inverse image (also called the pre-image or
  fiber) of $y_0$.
\item {\em Algebraically}: We set up the equation $f(x) = y_0$ and try
  to solve for $x$.
\end{itemize}

The following turn out to be true:

\begin{itemize}
\item A function is injective if and only if every horizontal line
  intersects its graph at at most one point.
\item The range of a function is the set of $y$-values for which the
  corresponding horizontal line intersects the graph.
\item A function is surjective to $\R$ if and only if every horizontal
  line intersects its graph at at least one point.
\item A function is bijective to $\R$ if and only if every horizontal
  line intersects its graph at {\em exactly} one point.
\end{itemize}

\subsection{The continuous case}

Suppose $f:\R \to \R$ is a continuous function. Note that some of the
remarks we make have analogues for functions defined on intervals in
$\R$ instead of on all of $\R$, but we will for simplicity consider
only the case of $f$ defined on all of $\R$. The following are true:

\begin{enumerate}
\item $f$ is injective if and only if it is increasing throughout $\R$
  or decreasing throughout $\R$.
\item $f$ is surjective if and only if one of these cases is satisfied
  ({\em Note: this point was edited to include the oscillatory cases}):

  \begin{enumerate}
  \item $\lim_{x \to -\infty} f(x) = -\infty$ and $\lim_{x \to \infty}
    f(x) = \infty$. For instance, $f$ could be any polynomial with odd
    degree and positive leading coefficient.
  \item $\lim_{x \to-\infty} f(x) = \infty$ and $\lim_{x \to \infty}
    f(x) = -\infty$. For instance, $f$ could be any polynomial with
    odd degree and negative leading coefficient.
  \item As $x \to \infty$, $f(x)$ is oscillatory between $-\infty$ and
    $\infty$. For instance, $f(x) := e^x\sin x$.
  \item As $x \to -\infty$, $f(x)$ is oscillatory between $-\infty$
    and $\infty$. For instance, $f(x) := e^{-x}\sin x$.
  \item As $x \to \infty$, $f(x)$ is oscillatory with the upper end of
    oscillation going to $\infty$, and as $x \to -\infty$, $f(x)$ is
    oscillatory with the lower end of oscillation going to
    $-\infty$. For instance, $f(x) := x \sin^2x$.
  \item As $x \to \infty$, $f(x)$ is oscillatory with the lower end of
    oscillation going to $-\infty$, and as $x \to -\infty$, $f(x)$ is
    oscillatory with the upper end of oscillation going to
    $+\infty$. For instance, $f(x) := -x\sin^2x$.
  \end{enumerate}

  Note that the cases (c)-(f) are not mutually exclusive. For
  instance, $f(x) := x \sin x$ satisfies the conditions for cases (c),
  (d), (e), and (f). Note also that all the cases (c)-(f) are
  incompatible with injectivity, and also that these cases do not
  occur for polynomial functions.

\item $f$ is bijective if and only if one of these cases is satisfied:

  \begin{enumerate}
  \item $f$ is increasing, $\lim_{x \to -\infty} f(x) = -\infty$, and
    $\lim_{x \to \infty} f(x) = \infty$
  \item $f$ is decreasing, $\lim_{x \to-\infty} f(x) = \infty$, and
    $\lim_{x \to \infty} f(x) = -\infty$
  \end{enumerate}
\end{enumerate}

\subsection{Addition, injectivity, and surjectivity}

We can see from these facts that the following is true:

\begin{itemize}
\item The sum of two continuous injective (respectively, surjective or
  bijective) functions need not be injective (respectively, surjective
  or bijective). The problem arises if the two functions are of
  opposite subtypes, for instance, one is increasing and the other is
  decreasing. If both functions are of a similar type, then the sum is
  also of that type. Note that for surjective functions,
  the types (c)-(f) are in general problematic.
\item If we take three continuous injective (respectively, bijective)
  functions, then at least one pair of them has the same subtype, so
  that sum is also of the same subtype, hence is continuous injective
  (respectively, bijective). In general, we cannot say more. Note that
  we cannot make any assertion for surjectivity per se, because of the
  weird types (c)-(f) for continuous surjective functions. However, if
  we restrict attention to continuous surjective functions of the
  subtypes (a) and (b), then adding two functions of the same subtype
  gives a function of the same subtype.
\end{itemize}

\subsection{Fibers for polynomial functions}

If $f:\R \to \R$ is a polynomial function of degree $n \ge 1$, finding
the inverse image under $f$ of any particular real number amounts to
solving a particular polynomial equation of degree $n$. Explicitly,
for an output $y_0$, finding $f^{-1}(y_0)$ is tantamount to solving
the polynomial equation $f(x) - y_0 = 0$ which is a degree $n$
equation in $x$. The polynomial equations for different points in the
co-domain differ only in their constant term. The following are true:

\begin{itemize}
\item In the case $n = 1$ (i.e., $f$ is a linear polynomial), every
  fiber has size exactly $1$, since $f$ is bijective. In fact, the
  inverse function to $f$ is also a linear polynomial.
\item In the case $n = 2$ (i.e., $f$ is a quadratic polynomial), every
  fiber has size $0$, $1$, or $2$. The maximum of the fiber sizes is
  always $2$. If the leading coefficient of $f$ is positive, then $f$
  has a unique absolute minimum value. The fiber size for that is $1$,
  the fiber size for bigger values is $2$, and the fiber size for
  smaller values is $0$. If the leading coefficient of $f$ is
  negative, then $f$ has a unique absolute maximum value. The fiber
  size for that is $1$, the fiber size for bigger values is $0$, and
  the fiber size for smaller values is $2$.
\item In the case $n \ge 3$, the maximum of the fiber sizes is at most
  $n$. However, it is {\em not} necessary that there exist fibers of
  size $n$ for a particular $f$ (by Rolle's theorem, we know that for
  this to happen it must be the case that $f'$ has $n -1$ distinct
  roots, and this does not happen for all $f$). An extreme case where
  we have small fiber sizes is $x^n$. For $n$ odd, the fiber sizes are
  all $1$ (the function is bijective). For $n$ even, the fiber sizes
  are all $0$, $1$, or $2$.
\end{itemize}

\subsection{Fibers for periodic functions, specifically trigonometric functions}

For a periodic function with period $h$, the inverse image of any
point is invariant under translation by $h$ and therefore also by all
integer multiples of $h$. In particular, the inverse image of any
point, if non-empty, must be infinite.

For instance, consider the $\sin$ function from $\R$ to $\R$. The
range is $[-1,1]$. For any point in this set, the inverse image is
invariant under translation by $2\pi$ and therefore also by all
integer multiples of $2\pi$.

To determine the inverse image for such a function, it makes sense to
first determine the inverse image over an interval of length equal to
the period, and then note that the inverse image over $\R$ is obtained
from this set by translating by multiples of $2\pi$. For instance, to
find the inverse image of $1/2$ under $\sin$, we note that on
$[0,2\pi]$, the only solutions to $\sin x = 1/2$ are $x = \pi/6$ and
$x = 5\pi/6$. Thus, the fiber is the set:

$$\{ 2n\pi + \pi/6 \mid n \in \mathbb{Z} \} \cup \{ 2n\pi + 5\pi/6 \mid n \in \mathbb{Z}\}$$

\section{Linear combinations and spans}

\subsection{Linear combination}

There is a very important concept that undergirds linear algebra: {\em
  linear combination}.

Suppose $\vec{v_1},\vec{v_2},\dots,\vec{v_r}$ are vectors in $\R^n$. A
vector $\vec{v}$ is termed a {\em linear combination} of the vectors
$\vec{v_1}$, $\vec{v_2}$, $\dots$, $\vec{v_r}$ if there exist real
numbers $a_1$, $a_2$, $\dots$, $a_r$ such that:

$$\vec{v} = a_1\vec{v_1} + a_2\vec{v_2} + \dots + a_r\vec{v_r}$$

The real numbers $a_1,a_2,\dots,a_r$ are allowed to be zero and
repetitions in values are allowed.

The term ``linear combination'' indicates the idea of ``combining''
the vectors in a {\em linear} fashion. Recall that linearity has to do
with addition and scalar multiplication. The values
$a_1,a_2,\dots,a_r$ are termed the {\em coefficients} used in the
linear combination.

Geometrically, linear combinations include scaling and addition. If we
think of vectors as physical translation, then addition is how we
compose the translation operations. Thus, the linear combinations of
vectors are the vectors we can obtain by moving along the directions
of the individual vectors one after the other. The coefficients
$a_1,a_2,\dots,a_r$ describe how much we move along the direction of
each vector.

We use the following terms:

\begin{itemize}
\item The {\em trivial} linear combination refers to the situation
  where all the coefficients are zero. The only vector you can write
  as a trivial linear combination is the zero vector. Thus, given any
  collection of vectors, the zero vector is the trivial linear
  combination of these.
\item Any other linear combination is termed a {\em nontrivial} linear
  combination. However, a nontrivial linear combination may still have
  {\em some} coefficients equal to zero. We may use informal jargon
  such as ``uses a vector nontrivially'' to indicate that the linear
  combination has a nonzero coefficient for that particular vector.
\end{itemize}

Here are a couple of extreme cases:

\begin{itemize}
\item For the empty collection of vectors, we can still define the
  ``trivial linear combination'' to be the zero vector. This seems
  somewhat silly, but there's a deep logic.
\item For a set of vectors of size one, the ``linear combinations''
  possible are precisely the scalar multiples of that vector.
\end{itemize}

Here are a few easy observations:

\begin{itemize}
\item If a vector is a linear combination of a certain set of vectors,
  the given vector is also a linear combination of any bigger set of
  vectors: just set the coefficients on all the other vectors to be
  $0$. For instance, if $\vec{v} = 2\vec{v_1} + 5\vec{v_2}$, then we
  also have $\vec{v} = 2\vec{v_1} + 5\vec{v_2} + 0\vec{v_3}$. We don't
  end up ``using'' the extra vectors, but it is still a legitimate
  linear combination.
\item For a (possibly infinite) set $S$ of vectors in $\R^n$, writing
  a vector $\vec{v}$ as a linear combination of $S$ means writing
  $\vec{v}$ as a linear combination of a finite subset of $S$. We can
  think of that linear combination as attaching a coefficient of zero
  to all but finitely many vectors in $\R^n$.
\end{itemize}

{\em A priori}, there may be more than one way that a given
vector can be expressed as a linear combination of a particular
collection of vectors. We will later on discuss a concept called
{\em linear independence} to describe a situation where no vector
can be written as a linear combination in multiple ways.

\subsection{Definition of subspace}

We will study the concept of subspace more later, but a very brief
definition is worthwhile right now. Consider a subset $V$ of
$\R^n$. We call $V$ a {\em subspace} of $\R^n$ if it saitsfies the
following three conditions:

\begin{itemize}
\item The zero vector is in $V$.
\item $V$ is closed under addition, i.e., for any vectors $\vec{v},
  \vec{w}$ (possibly equal, possibly distinct) in $V$, the sum
  $\vec{v} + \vec{w}$ is also a vector in $V$. Note that the choice of
  leters for the vectors is not important. What matters is that
  regardless of the vectors chosen, the sum remains in the subspace.
\item $V$ is closed under scalar multiplication, i.e., if $\vec{v}$ is
  a vector in $V$ and $a$ is a real number, then $a\vec{v}$ is also a
  vector in $V$.
\end{itemize}

Two extreme examples of subspaces are:

\begin{itemize}
\item The zero subspace, i.e., the subspace comprising only the zero vector.
\item The whole space, in this case $\R^n$.
\end{itemize}

Pictorially, a subspace is something flat, closed both under scaling
and addition. Lines through the origin are subspaces. Planes through
the origin are subspaces. There are other higher-dimensional concepts
of subspaces. I'd love to say more, but that would be jumping the gun.

\subsection{Spanning sets and spans}

Consider a subset $S$ of $\R^n$. For convenience, you can imagine $S$
to be finite, though this is not necessary for the definition. The
{\em span} of $S$ is defined as the set of all vectors that arise as
linear combinations of vectors in $S$. The span of $S$ is a {\em
  subspace} of $\R^n$. Explicitly:

\begin{itemize}
\item The zero vector is in the span of any set, on account of being
  the trivial linear combination.
\item {\em Addition}: If $\vec{v}$ and $\vec{w}$ are vectors in the
  span of $S$, then $\vec{v} + \vec{w}$ is also in the span of $S$: To
  see this, note that we can add up the coefficients for each
  vector. For instance, if we are looking at the span of three vectors
  $\vec{u_1}$, $\vec{u_2}$, $\vec{u_3}$, and we have:

  \begin{eqnarray*}
    \vec{v} & = & 4\vec{u_1} - 7\vec{u_2} + 6\vec{u_3}\\
    \vec{w} & = & 11\vec{u_1} + 9\vec{u_2} + 3\vec{u_3}\\
  \end{eqnarray*}
  
  Then we get:

  $$\vec{v} + \vec{w} = (4 + 11)\vec{u_1} + (-7 + 9)\vec{u_2} + (6 + 3)\vec{u_3}$$

  The same idea applies in general.

  Note that when we add, some of the coefficients may become zero, so
  the sum may end up ``not using'' all the vectors used in the
  descriptions of the individual vectors. However, {\em a priori}, we
  do not know whether any vectors will become unused after adding.
\item {\em Sclaar multiplication}: If $\vec{v}$ is in the span of $S$,
  and $a \in \R$, then $a\vec{v}$ is also in the span of $S$: Whatever
  linear combination we use to describe $\vec{v}$, we can multiply
  each of the coefficients there with $a$ to describe $a\vec{v}$ as a
  linear combination of the same set.
\end{itemize}

Suppose $W$ is a sub{\em space} of $\R^n$, i.e., it is a nonempty
subset of $\R^n$ closed under addition and scalar
multiplication. Then, a subset $S$ of $W$ is termed a {\em spanning
  set} for $W$ if $W$ is the span of $S$. Note that if $S_1 \subseteq S_2
\subseteq W$ with $W$ a subspace, and $S_1$ is a spanning set for $W$,
then $S_2$ is also a spanning set for $W$.

\subsection{Why do we care about spanning sets and spans?}

Vector spaces such as $\R^n$, and their subspaces, are {\em infinite}
(with the exception of the zero space, which has only one element,
namely the zero vector). Describing a subspace of $\R^n$ by listing
out all the elements will take infinite time. Thus, it is useful to
have a way to describe subspaces of $\R^n$ using finite data. Finite
spanning sets offer a convenient method.

Note, however, that there can be many different candidates for a
spanning set for a vector space. We will later on consider a concept
of {\em basis}, which can be defined as a {\em minimal} spanning
set. Even if we use basis instead of spanning set, however, we do not
have any form of uniqueness. While this is a drawback, we will later
see ways to figure out whether two sets have the same span.

\section{Image and kernel}

\subsection{Image of a linear transformation is a subspace}

Suppose $T:\R^m \to \R^n$ is a linear transformation and $A$ is the
matrix of $T$. The {\em image} of $T$ (also called the {\em range} of
$T$) is the subset of $\R^n$ comprising those vectors that can be
written in the form $T(\vec{v})$ where $\vec{v} \in \R^m$. The
question is: {\em what does the image of $T$ look like}?

First, it is easy to see from abstract considerations that the image
of $T$ is a subspace of $\R^n$. We show this in three parts:

\begin{itemize}
\item The zero vector is in the image of any linear transformation,
  since it is the image of the zero vector.
\item {\em Addition}: Suppose $\vec{u_1}$ and $\vec{u_2}$ are vectors
  in the image (range) of $T$. Our goal is to show that $\vec{u_1} +
  \vec{u_2}$ is in the image of $T$.

  {\em Proof sketch}: By the definition of what it means to be in the
  image, there exist vectors $\vec{v_1}, \vec{v_2} \in \R^m$ such that
  $\vec{u_1} = T(\vec{v_1})$ and $\vec{u_2} = T(\vec{v_2})$. Thus,
  $\vec{u_1} + \vec{u_2} = T(\vec{v_1}) + T(\vec{v_2}) = T(\vec{v_1} +
  \vec{v_2})$ (using the linearity of $T$). Thus, $\vec{u_1} +
  \vec{u_2}$ is in the image of $T$. Hence, the image of $T$ is closed
  under addition.
\item {\em Scalar multiplication}: Suppose $\vec{u}$ is a vector in
  the image of $T$ and $a$ is a real number. Our goal is to show
  that the vector $a\vec{u}$ is in the image of $T$.

  {\em Proof sketch}: By the definition of what it means to be in the
  image, there exists a vector $\vec{v} \in \R^m$ such that
  $T(\vec{v}) = \vec{u}$. Then, $a\vec{u} = a T(\vec{v})
  = T(a \vec{v})$. In particular, $a\vec{u}$ is in the image of $T$.
\end{itemize}

Note that both these proofs rely on the {\em abstract} conception of
linearity. In particular, the proofs above also work for
``infinite-dimensional vector spaces'' -- something we shall return to
after a while.

\subsection{Column vectors span the image}

The goal now is to {\em explicitly describe} the image as a vector
space. The explicit description is as the span of a finite set of
vectors, namely, the column vectors of the matrix. This explicit
description crucially relies on the concrete description of the linear
transformation.

Recall that the matrix $A$ for $T:\R^m \to \R^n$ is a $n \times m$
matrix. The columns of $A$ describe the values $T(\vec{e_1})$,
$T(\vec{e_2})$, and so on, right till $T(\vec{e_m})$ (the last
column). Thus, each column of $A$, viewed as a column vector, is in
the image of $T$.

There are other vectors in the image of $T$, but these are essentially
described as {\em linear combinations} of these column vectors. By
{\em linear combination}, I mean a sum of scalar multiples of the
given vectors (as described a little while back). Explicitly, consider
a generic vector of the domain $\R^m$:

$$\left[\begin{matrix} x_1  \\ x_2 \\ \cdot \\ \cdot \\ \cdot \\ x_m \end{matrix}\right] = x_1 \vec{e_1} + x_2 \vec{e_2} + \dots + x_m \vec{e_m}$$

The image of this is:

$$T\left(\left[\begin{matrix} x_1  \\ x_2 \\ \cdot \\ \cdot \\ \cdot \\ x_m \end{matrix}\right]\right) = T(x_1 \vec{e_1} + x_2 \vec{e_2} + \dots + x_m \vec{e_m}) = x_1T(\vec{e_1}) + x_2T(\vec{e_2}) + \dots + x_mT(\vec{e_m})$$

In particular, the image is a linear combination of $T(\vec{e_1})$,
$T(\vec{e_2})$, $\dots$, $T(\vec{e_m})$. Conversely, every linear
combination of these vectors is in the image. In other words, the
image of the linear transformation $T$ is {\em precisely} the set of
linear combinations of $T(\vec{e_1})$, $T(\vec{e_2})$, $\dots$,
$T(\vec{e_m})$.

As per the terminology introduced in a previous section, a set of
vectors such that every vector in the image can be expressed as a
linear combination of vectors in that set is said to be a {\em
  spanning set} for the image or is said to {\em span} the image. The
upshot is that the image of a linear transformation $T:\R^m \to \R^n$
is a subspace of $\R^n$ and that the column vectors of the matrix form
a spanning set for the image. Note that its being a subspace follows
both directly and from the fact that the span of any set of vectors is
a subspace.

For instance, consider the linear transformation $T:\R^2 \to \R^3$
with matrix:

$$\left[\begin{matrix} 3 & 4 \\ 2 & 6\\ 1 & 5 \\\end{matrix}\right]$$

Note that this is a $3 \times 2$ matrix, as expected, since the linear
transformation goes from $\R^2$ to $\R^3$. The image of $\vec{e_1}$
under the linear transformation is the column vector
$\left[\begin{matrix} 3 \\ 2 \\ 1 \\\end{matrix}\right]$ and the image
of $\vec{e_2}$ under the linear transformation is the column vector
$\left[ \begin{matrix} 4 \\ 6 \\ 5 \\\end{matrix}\right]$.

The image is thus the set of all vectors that can be expressed as
linear combinations of these two vectors. Explicitly, it is the set of
all vectors of the form (secretly, the input vector is
$\left[ \begin{matrix} x_1 \\ x_2 \\\end{matrix}\right]$):

$$x_1 \left[\begin{matrix} 3 \\ 2 \\ 1 \\\end{matrix}\right] + x_2 \left[ \begin{matrix} 4 \\ 6 \\ 5 \\\end{matrix}\right], x_1, x_2 \in \R$$

The generic vector in the image could alternatively be written in the form:

$$\left[ \begin{matrix} 3x_1 + 4x_2 \\ 2x_1 + 6x_2 \\ x_1 + 5x_2 \\\end{matrix}\right], x_1, x_2 \in \R$$

We can therefore use the two column vectors as our spanning set:

$$ \left[\begin{matrix} 3 \\ 2 \\ 1 \\\end{matrix}\right], \left[ \begin{matrix} 4 \\ 6 \\ 5 \\\end{matrix}\right]$$
\subsection{Kernel of a linear transformation is a subspace}

The kernel of a linear transformation $T: \R^m \to \R^n$ is the set of
those vectors in the domain that get mapped to the zero vector. In
other words, the kernel is the fiber (inverse image) corresponding to
the zero vector.

It turns out that the kernel of a linear transformation is also a
subspace:

\begin{itemize}
\item The zero vector is in the kernel of any linear transformation,
  because its image is the zero vector.
\item {\em Addition}: We want to show that if $\vec{v}$ and $\vec{w}$
  are both vectors in the kernel of $T$, then $\vec{v} + \vec{w}$ is
  also in the kernel of $T$. This follows from the additivity of $T$:

  $$T(\vec{v} + \vec{w}) = T(\vec{v}) + T(\vec{w}) = 0 + 0 = 0$$
\item {\em Scalar multiplication}: If $\vec{v}$ is in the kernel of
  $T$ and $a$ is a real number, then $a\vec{v}$ is also in the kernel
  of $T$. This follows from the scalar multiples aspect of $T$:

  $$T(a\vec{v}) = aT(\vec{v}) = a(0) = 0$$
\end{itemize}

Thus, the kernel is a subspace. Note that the argument used relied
only on the {\em abstract} definition of linearity and therefore
applies to vector spaces in the abstract setting, and also to
infinte-dimensional spaces.

\subsection{Finding a spanning set for the kernel}

Finding the kernel is effectively the same as solving the system with
the matrix as coefficient matrix and the augmenting column the zero
column. We solve this the usual way: first, convert the matrix to
reduced row echelon form. We can now describe the kernel the way we
have done in the past: parametrically, as a solution
set. Alternatively, we can try to describe the kernel using a spanning
set. To describe it in the latter way, do the following: for each
non-leading variable, take the case where that variable takes the
value $1$ and all the other non-leading variables take the value
$0$. Find the values of the leading variables, and construct a vector
from these. Thus, for each non-leading variable, we get a
corresponding vector. This set of vectors spans the kernel.

We will illustrate both the longer and shorter approach using a worked
example in the next section.

\subsubsection{A worked-out example, the long way}

Suppose that the reduced row-echelon form of a matrix turns out to be:

$$\left[\begin{matrix} 1 & 3 & 8 & 0 & 10 \\ 0 & 0 & 0 & 1 & 7 \\\end{matrix}\right]$$

There are three non-leading variables: the second, third, and fifth
variable. Let's first use the parametric solution description for the
kernel. Suppose $t$ is the fifth variable, $u$ is the third variable,
and $v$ is the second variable. Then, the general solution can be written as:

$$\left[\begin{matrix} -3v - 8u - 10t \\ v \\ u \\ -7t \\ t \\\end{matrix}\right]$$

Let us separate out the multiples of $t$, $u$, and $v$ here. We can write this as:

$$\left[\begin{matrix} -3v \\ v \\ 0 \\ 0 \\ 0 \\\end{matrix}\right] + \left[\begin{matrix} -8u \\ 0 \\ u \\ 0 \\ 0 \\\end{matrix}\right] + \left[\begin{matrix} -10t \\ 0 \\ 0 \\ -7t \\ t \\\end{matrix}\right]$$

If we now ``pull out'' the variables from their respective
expressions, we obtain:

$$v \left[\begin{matrix} -3 \\ 1 \\ 0 \\ 0 \\ 0 \\\end{matrix}\right] + u \left[\begin{matrix} -8 \\ 0 \\ 1 \\ 0 \\ 0 \\\end{matrix}\right] + t \left[\begin{matrix} -10 \\ 0 \\ 0 \\ -7 \\ 1 \\\end{matrix}\right]$$

Note that $t,u,v$ all vary freely over all of $\R$. Thus, the kernel
is spanned by the vectors:

$$\left[\begin{matrix} -3 \\ 1 \\ 0 \\ 0 \\ 0 \\\end{matrix}\right], \left[\begin{matrix} -8 \\ 0 \\ 1 \\ 0 \\ 0 \\\end{matrix}\right], \left[\begin{matrix} -10 \\ 0 \\ 0 \\ -7 \\ 1 \\\end{matrix}\right]$$

Setting up names for the parameters, writing the general expression,
and then pulling the variables out is a somewhat time-consuming
process. Let's now consider a shorter way of doing the same example.

\subsubsection{The same worked example, a shorter way}

\begin{itemize}
\item Our spanning set will include one vector corresponding to each
  non-leading variable, i.e., each column without a pivotal $1$.
\item The vector corresponding to a particular non-leading variable is
  obtained as follows: set the value of that variable to equal $1$,
  and set the values of all the other non-leading variables to equal
  $0$. Now, calculate the values of the {\em leading} variables based
  on the linear system. The value of a particular leading variable is
  simply the negative of the entry for the pivotal row of the leading
  variable in the column of the non-leading variable.
\end{itemize}

Let's first take the case where the second variable is $1$ while the
third and fifth variable are $0$. In this case, the first variable is
$-3$ and the fourth variable is $0$, so we get the vector:

$$\left[\begin{matrix} -3 \\ 1 \\ 0 \\ 0 \\ 0 \\\end{matrix}\right]$$

The $-3$ out there comes from the fact that the entry in the first row
and second column is $3$. The fourth variable is $0$ because it comes
after the second variable.

The vector corresponding to the third variable being $1$ while the
second and fifth are $0$ is:

$$\left[\begin{matrix} -8 \\ 0 \\ 1 \\ 0 \\ 0 \\\end{matrix}\right]$$

The $-8$ out there comes from the fact that the entry in the first row
and third column is $8$. The fourth variable is $0$ because it comes
after the third variable.

The vector corresponding to the fifth variable being $1$ while the
second and third are $0$ is:

$$\left[\begin{matrix} -10 \\ 0 \\ 0 \\ -7 \\ 1 \\\end{matrix}\right]$$

The first variable is $-10$ because the entry in the first row and
fifth column is $10$. The fourth variable is $-7$ because the entry in
the {\em second} row (the row where the pivotal variable is the fourth
variable) and fifth column is $7$.

If you compare this shorter approach with the longer approach that
involves writing the parameters explicitly, you should after a while
be able to see just how the shorter approach is a shortening of the
longer approach.

\section{Fibers of a linear transformation}

\subsection{Fibers are translates of the kernel}

The fibers of a linear transformation are the inverse images of
specific points in the co-domain. To find the fiber (inverse image)
for a particular point, we need to solve the linear system with that
as the augmenting column. The solution set is either empty or is
described parametrically.

Thus, the fibers over points not in the image are empty. For points
that are in the image, the corresponding fiber looks like a translate
of the kernel. Explicitly, any nonempty fiber looks like:

(particular solution vector) + (arbitrary element of the kernel)

This is extremely crucial to understand, so let's go over it
explicitly.

Suppose $T: \R^m \to \R^n$ is a linear transformation. We have a
vector $\vec{v} \in \R^n$. We will show the following two things:

\begin{itemize}
\item If $\vec{u}$ is a vector in $\R^m$ such that $T(\vec{u}) =
  \vec{v}$, and $\vec{w}$ is in the kernel of $T$, then $T(\vec{u} +
  \vec{w})= \vec{v}$. In other words, adding any element of the
  kernel to an element in the fiber keeps one within the fiber.

  {\em Proof}: We have that:

  $$T(\vec{u} + \vec{w}) = T(\vec{u}) + T(\vec{w}) = \vec{v} + 0 = \vec{v}$$

\item If there are two elements in the same fiber, they differ by an
  element in the kernel. Explicitly, if $\vec{u_1}, \vec{u_2} \in
  \R^m$ are such that $T(\vec{u_1}) = T(\vec{u_2}) = \vec{v}$, then
  $\vec{u_1} - \vec{u_2}$ is in the kernel of $T$. In other words,
  each fiber must involve only a {\em single} translate of the kernel.

  {\em Proof}: We have:

  $$T(\vec{u_1} - \vec{u_2}) = T(\vec{u_1}) - T(\vec{u_2}) = \vec{v} - \vec{v} = 0$$
\end{itemize}

In other words, each non-empty fiber is a {\em single} translate of
the kernel. For instance, imagine the linear transformation $T:\R^2
\to \R^2$ with matrix:

$$\left[\begin{matrix} 0 & 0  \\ 1 & 0 \\\end{matrix}\right]$$

The kernel of this linear transformation is the subspace generated by
the vector $\vec{e_2}$. The image of this linear transformation is the
space generated by the vector $\vec{e_2}$.

If we use $x$ to denote the first coordinate and $y$ to denote the
second coordinate, the map can be written as:

$$\left[\begin{matrix} x \\ y \\ \end{matrix}\right] \mapsto \left[\begin{matrix} 0 \\ x \\\end{matrix}\right]$$

The kernel is the $y$-axis and the image is also the $y$-axis.

Now, consider a vector, say:

$$\vec{v} = \left[ \begin{matrix} 0 \\ 5 \\ \end{matrix}\right]$$

We know that the vector is in the image of the linear
transformation. What's an example of a vector that maps to it?
Clearly, since this transformation sends $\vec{e_1}$ to $\vec{e_2}$,
the following vector $\vec{u}$ gets mapped to $\vec{v}$ under the
transformation:

$$\vec{u} = \left[ \begin{matrix} 5 \\ 0 \\\end{matrix}\right]$$

Now, however, note that since the kernel comprises the scalar
multiples of the vector $\vec{e_2}$, we can add any scalar multiple of
$\vec{e_2}$ to the vector $\vec{u}$ and get a new vector that maps to
$\vec{v}$. Explicitly, the fiber over $\vec{v}$ is the set:

$$\left\{ \left[\begin{matrix} 5 \\ y \\\end{matrix}\right] \mid y \in \R \right \}$$

Thus, the fiber over the point $(0,5)$ in the image is the line $x =
5$. Notice that this is a line parallel to the kernel of the linear
transformation. In fact, all the fibers are lines parallel to the
kernel of the linear transformation. In this case, they are all
vertical lines.

Let's look at another, somewhat more complicated example:

$$\left[\begin{matrix} 1 & 1 \\ 0 & 0 \\\end{matrix}\right]$$

Explicitly, the map is:

$$\left[\begin{matrix} x \\ y \\\end{matrix}\right] \mapsto \left[\begin{matrix} x + y \\ 0 \\\end{matrix}\right]$$

The map sends both $\vec{e_1}$ and $\vec{e_2}$ to $\vec{e_1}$. Hence,
the matrix is idempotent.

The kernel is the line $y = -x$, and it is spanned by the vector:

$$\left[\begin{matrix} -1 \\ 1 \\\end{matrix}\right]$$

The image is the line $y = 0$, i.e., the $x$-axis, so it is spanned by
the vector:

$$\left[\begin{matrix} 1 \\ 0 \\\end{matrix}\right]$$

The inverse image of the vector:

$$\vec{v} = \left[ \begin{matrix} 2 \\ 3 \\\end{matrix}\right]$$

is the empty set, because this vector is not in the image of the
linear transformation. On the other hand, the inverse image of the vector:

$$\vec{w} = \left[ \begin{matrix} 2 \\ 0 \\\end{matrix}\right]$$

is the set of all vectors on the line $x + y = 2$. Note that this is a
parallel line to the kernel $x + y = 0$. Also note that the linear
transformation is idempotent, and this explains the fact that every
point in the image is in its own fiber.

\subsection{In terms of solving linear systems}

Let's think a bit more about this in terms of solving linear
systems. If the linear system has full row rank, then row reducing
gives a consistent system and we can read off the solutions after row
reducing. Thus, {\em the linear transformation is surjective}. Each
fiber looks like a translate of the kernel.

If the system does not have full row rank, there are rows that, after
converting to rref, become zero in the coefficient matrix. Start with
an augmenting column describing a vector in the co-domain. If the
corresponding augmenting entry after doing the row operations is
nonzero, the system is inconsistent, so the augmenting column is not
in the image, or equivalently, the fiber is empty. If, however, after
row reduction, all zero rows in the coefficient matrix have zero as
the augmenting entry, then the system is consistent. We can take the
particular solution vector as a vector where each leading variable is
the corresponding augmenting column value and the non-leading
variables are all zero. This is clearly a solution. We can add
anything from the kernel to it and still get a solution.

\section{Dimension computation}

\subsection{Dimension of subspace}

The {\em dimension} of a subspace is defined as the minimum possible
size of a spanning set. The dimension of the zero subspace is zero,
because by convention we consider the span of the empty set to be zero
(this may be unclear right now but will hopefully become clearer as we
proceed). The dimension of $\R^p$ itself is $p$.

\subsection{Dimension of kernel and image}

Suppose $T: \R^m \to \R^n$ is a linear transformation. The matrix $A$
of $T$ is a $n \times m$ matrix. We have seen the following in the past:

\begin{itemize}
\item $T$ is injective if and only if $A$ has full column rank, i.e.,
  rank $m$. This implies $m \le n$.

  {\em Explanation}: For $T$ to be injective, we need that the linear
  system $A\vec{x} = \vec{y}$ always has at most one solution. This
  requires that there be no non-leading variables, i.e., the rref of
  $A$ should have all its columns as pivotal columns. This forces full
  column rank.
\item $T$ is surjective if and only if $A$ has full row rank, i.e.,
  rank $n$. This implies $n \le m$.

  {\em Explanation}: For $T$ to be surjective, we need that the linear
  system $A \vec{x} = \vec{y}$ always has at least one solution,
  regardless of $\vec{y}$. In particular, this means that we cannot
  have any zero rows in the rref of $A$, because a zero row would open
  up the possibility that the corresponding augmenting entry is
  nonzero and the system could therefore be inconsistent for some
  $\vec{y}$. This possibility needs to be avoided, so we need that all
  rows of $A$ be nonzero, i.e., all of them contain pivotal $1$s. This
  forces $A$ to have full row rank.
\end{itemize}

Also note that:

\begin{itemize}
\item {\em Injectivity} is the same as the kernel being the zero
  subspace: This was noted at the end of our discussion on fibers.
\item {\em Surjectivity} is the same as the image being the whole
  target space $\R^n$.
\end{itemize}

Thus, we get that:

\begin{itemize}
\item The kernel of $T$ is the zero subspace if and only if $T$ has
  full column rank, i.e., rank $m$. This implies $m \le n$.
\item The image of $T$ is all of $\R^n$ if and only if $T$ has full
  row rank, i.e., rank $n$. This implies $n \le m$.
\end{itemize}

We can generalize these to give formulas for the dimension of the
kernel and the image:

\begin{itemize}
\item The dimension of the kernel is (number of columns) - (rank). In
  terms of systems of linear equations, it is the number of
  non-leading variables. Intuitively, this is because the non-leading
  variables are the parameters in an equational description. Our
  explicit construction of a spanning set for the kernel gave one
  vector for each non-leading variable. In fact, a spanning set
  constructed this way is minimal.
\item The dimension of the image is the rank: Although our original
  choice of spanning set for the image was the set of {\em all} column
  vectors, we may not need all column vectors. If the map is not
  injective, then some of the column vectors are {\em redundant} (we
  will discusss this term later). It will turn out that it suffices to
  use the column vectors (in the original matrix) corresponding to the
  leading variables. Note that determining which variables are leading
  and which variables are non-leading requires us to convert the
  matrix to rref, but we need to use the columns of the original
  matrix, {\em not} those of the rref. We will discuss this more after
  introducing the concept of basis.
\end{itemize}

\subsection{Facts about dimension we will prove later}

We will show later that if $A$ is a vector subspace of a vector space
$B$, then the dimension of $A$ is less than or equal to the dimension
of $B$. In fact, we will show that any minimal spanning set of $A$
(also called a {\em basis} for $A$) can be extended (non-uniquely) to
a minimal spanning set of $B$.


\section{Kernel and image for some easy and useful types of linear transformations}

\subsection{Definition of projection}

A linear transformation $T:\R^n \to \R^n$ is termed a {\em projection}
if $T^2 = T$. Here, $T^2$ denotes the composite $T \circ T$ of $T$
with itself. Another term used to describe a projection is {\em
  idempotent}, and the term {\em idempotent} is typically used for the
matrix corresponding to a projection (the term ``idempotent'' means
``equal to its own square'' and is used in all algebraic structures,
whereas projection is a term specific to thinking of transformations).

If $T$ is a projection, then the intersection of the kernel of $T$ and
the image of $T$ comprises only the zero vector. However, this is not
a sufficient condition to describe projections.

Here is an example of a projection (discussed earlier in the notes,
when talking about fibers):

$$\left[\begin{matrix} 1 & 1 \\ 0 & 0 \\\end{matrix}\right]$$

This linear transformation sends $\vec{e_1}$ to itself and sends the
vector $\vec{e_2}$ to the vector $\vec{e_1}$. Doing it twice is the
same as doing it once.

Note that the matrix is already in rref, so we can read off the kernel
as well as the image:

\begin{itemize}
\item The kernel is the one-dimensional space spanned by the vector
  $\left[\begin{matrix} -1 \\ 1 \\\end{matrix}\right]$. To see this,
  note that the second variable is non-leading, and setting that to
  $1$ gives the first variable as $-1$.
\item The image is the one-dimensional space spanned by
  $\vec{e_1}$. We can see this from the fact that the image is spanned
  by the column vectors, and that in this case, both the column
  vectors are the same.
\end{itemize}

Pictorially, both the kernel and the image are lines. If we call the
coordinate directions $x$ and $y$, then the kernel is the line $y =
-x$ and the image is the line $y = 0$ (i.e., the $x$-axis). These
lines intersect at a unique point (the origin).

Note that it is possible to have a linear transformation whose kernel
and image intersect only in the zero vector, but such that the linear
transformation is not a projection. In fact, any linear automorphism
other than the identity automorphism is of this form: the kernel is
the zero subspace, and the image is the whole space, so the
intersection is the zero subspace, so the kernel and image intersect
in the zero subspace, but the map is not a projection. As a concrete
example, consider the linear transformation with matrix:

$$\left[ \begin{matrix} 2 \\\end{matrix}\right]$$

The kernel of this is the zero subspace, and the image is all of
$\R$. It is not idempotent. To see this, note that $2^2 = 4$ which
differs from $2$.

\subsection{Orthogonal projection}

We define $T:\R^n \to \R^n$ to be an {\em orthogonal projection} if it
is a projection and every vector in the kernel of $T$ is orthogonal to
every vector in the image of $T$. Note that the projection described
above, namely:

$$\left[\begin{matrix} 1 & 1 \\ 0 & 0 \\\end{matrix}\right]$$

is {\em not} an orthogonal projection, i.e., it does {\em not} have
orthogonal kernel and image.

Suppose an orthogonal projection $T:\R^n \to \R^n$ has rank $m$. That
means that the image is $m$-dimensional and the kernel is $(n -
m)$-dimensional. The kernel and image are orthogonal subspaces.

For now, we consider a special case of orthogonal projection: {\em the
  orthogonal projection to the subspace of the first few
  coordinates}. Explicitly, consider a vector space $\R^n$ and
consider the projection map that preserves the first $m$ coordinates,
with $0 < m < n$, and sends the remaining coordinates to zero. The
matrix for such a linear transformation has the following block form:

$$\left[\begin{matrix} I_m & 0 \\ 0 & 0 \\\end{matrix}\right]$$

where $I_m$ is the $m \times m$ identity matrix. Explicitly, the first
$m$ diagonal entries are $1$. The remaining $n - m$ entries on the
diagonal are $0$. All the entries that are not on the diagonal are
also zero.

In the case $n = 2$, $m = 1$, the projection is given by the matrix:

$$\left[ \begin{matrix} 1 & 0 \\ 0  & 0 \\\end{matrix}\right]$$

Physically, this is the orthogonal projection onto the $x$-axis.

If $P$ is an orthogonal projection matrix, then $P^2 = P$. This makes
sense, because projecting something and then projecting again should
have the same effect as doing one projection.

\subsection{Linear transformations that project, permute, and pad}

A linear transformation $T:\R^m \to \R^n$ could be constructed that
combines aspects from many different types of linear transformations:

\begin{itemize}
\item It could look at only $r$ of the $m$ coordinates and forget the
  remaining $m - r$.
\item It could then map these $r$ coordinates into separate
  coordinates of $\R^n$.
\item It could then fill in zeros for the remaining coordinates.
\end{itemize}

The matrix for this linear transformation has rank $r$. There are $r$
columns with one $1$ and the other ($n - 1$) entries $0$. The
remaining $m - r$ columns are all zeros. Similarly, there are $r$ rows
with one $1$ and the other ($m - 1$) entries zero. The remaining $n -
r$ rows are all zeros.

Here's what we can say about the kernel and image:

\begin{itemize}
\item The kernel comprises those vectors that have zeros in all the
  $r$ coordinates that matter, and can have arbitrary entries in the
  other coordinates. It thus has dimension $m - r$, and has a spanning
  set comprising those $m - r$ standard basis vectors.
\item The image has dimension $r$, and has a spanning set comprising
  the standard basis vectors for the coordinates that get hit.
\end{itemize}

Here's an example of a linear transformation of this sort from $\R^3$
to $\R^4$:

$$\left[\begin{matrix} x_1 \\ x_2 \\ x_3 \\\end{matrix}\right] \mapsto \left[ \begin{matrix} x_3 \\ 0 \\ x_2 \\ 0 \\\end{matrix}\right]$$

The kernel here is spanned by $\vec{e_1}$ in $\R^3$ and the image is
spanned by $\vec{e_1}$ and $\vec{e_3}$ in $\R^4$. The kernel is
one-dimensional and the image is two-dimensional.

\section{Intersection and sum of vector spaces}

\subsection{Intersection of vector subspaces}

Suppose $U$ and $V$ are vector subspaces of $\R^n$. The intersection
$U \cap V$ of $U$ and $V$ as sets is then also a vector subspace of
$\R^n$. This is fairly easy to see: if both $U$ and $V$ individually
contain the zero vector, so does the intersection. If both $U$ and $V$
are closed under addition, so is the intersection. Finally, if both
$U$ and $V$ are closed under scalar multiplication, so is the
intersection.
\subsection{Sum of vector subspaces}

Suppose $U$ and $V$ are vector subspaces of $\R^n$. We define $U + V$ as the
subset of $\R^n$ comprising vectors expressible as the sum of a vector
in $U$ and a vector in $V$. The sum $U + V$ is a vector space: clearly
it contains the $0$ vector (since $0 = 0 + 0$). In addition, it is
closed under addition and scalar multiplication:

\begin{itemize}
\item {\em Addition}: Suppose $\vec{w_1}$, $\vec{w_2}$ are vectors in
  $U + V$. Our goal is to show that $\vec{w_1} + \vec{w_2}$ is also in
  $U + V$. We know that for $\vec{w_1}$, we can find vectors
  $\vec{u_1} \in U$ and $\vec{v_1} \in V$ such that $\vec{w_1} =
  \vec{u_1} + \vec{v_1}$. Similarly, for $\vec{w_2}$, we can find
  vectors $\vec{u_2} \in U$ and $\vec{v_2} \in V$ such that $\vec{w_2}
  = \vec{u_2} + \vec{v_2}$. Thus, we have that:

  $$\vec{w_1} + \vec{w_2} = (\vec{u_1} + \vec{v_1}) + (\vec{u_2} + \vec{v_2}) = (\vec{u_1} + \vec{u_2}) + (\vec{v_1} + \vec{v_2})$$

  Since $U$ is a vector space, we know that $\vec{u_1} + \vec{u_2} \in
  U$. And since $V$ is a vector space, we know that $\vec{v_1} +
  \vec{v_2} \in V$. Thus, the vector $\vec{w_1} + \vec{w_2}$ is in $U
  + V = W$.  

\item {\em Scalar multiplication}: Suppose $\vec{w} \in W$ and $a \in
  \R$. We want to show that $a\vec{w} \in W$. Note that by the
  definition of $W$ as $U + V$, there exists a vector $\vec{u} \in U$
  and $vec{v} \in V$ such that $\vec{w} = \vec{u} + \vec{v}$. Then:

  $$a\vec{w} = a(\vec{u} + \vec{v}) = a\vec{u} + a\vec{v}$$
\end{itemize}


\section{Sum of linear transformations, kernel, and image}

In general, adding two linear transformations can play havoc with the
structure, particularly if the linear transformations look very
different from each other. We explore the kernel and image separately.


\subsection{Effect of adding linear transformations on the kernel}

Suppose $T_1,T_2:\R^m \to \R^n$ are linear transformations. Note that
although for convenience we describe these as maps from $\R^m$ to
$\R^n$, whatever we say generalizes to maps between arbitrary vector
spaces, including infinite-dimensional ones (with the caveat that
statements about dimension would have to accommodate the possibility
of the dimension being infinite). The sum $T_1 + T_2$ is also a linear
transformation. What is the relation between the kernels of $T_1$,
$T_2$, and $T_1 + T_2$? The kernels satisfy a ``flower arrangement''
which means that the following are true:

\begin{itemize}
\item If a vector $\vec{u} \in \R^m$ is in the kernels of both $T_1$
  and $T_2$, then $\vec{u}$ is also in the kernel of $T_1 + T_2$. To
  see this, note that:

  $$(T_1 + T_2)(\vec{u}) = T_1(\vec{u}) + T_2(\vec{u}) = 0 + 0 = 0$$
\item If a vector $\vec{u} \in \R^m$ is in the kernels of both $T_1$
  and $T_1 + T_2$, then $\vec{u}$ is also in the kernel of $T_2$. To
  see this, note that:

  $$T_2(\vec{u}) = (T_1 + T_2)(\vec{u}) - T_1(\vec{u}) = 0 - 0 = 0$$
\item If a vector $\vec{u} \in \R^m$ is in the kernels of both $T_2$
  and $T_1 + T_2$, then $\vec{u}$ is also in the kernel of $T_1$. To
  see this, note that:

  $$T_1(\vec{u}) = (T_1 + T_2)(\vec{u}) - T_2(\vec{u}) = 0 - 0 = 0$$
\item In particular, this means that the intersection of the kernels
  of $T_1$, $T_2$, and $T_1 + T_2$ equals the intersection of any two
  of the three.
\end{itemize}

Note that the kernel of $T_1 + T_2$ could be a lot {\em bigger} than
the kernels of $T_1$ and $T_2$. For instance, if $T_1$ is the identity
transformation and $T_2 = -T_1$, then both $T_1$ and $T_2$ have zero
kernel but the kernel of $T_1 + T_2$ is the entire domain.

\subsection{Effect of adding linear transformations on the image}

Using the same notation as for the preceding section, the following
results hold regarding the relationship between the image of $T_1$,
the image of $T_2$, and the image of $T_1 + T_2$:

\begin{itemize}
\item The image of $T_1 + T_2$ is contained in the sum of the image of
  $T_1$ and the image of $T_2$.
\item The image of $T_1$ is contained in the sum of the image of $T_2$
  and the image of $T_1 + T_2$.
\item The image of $T_2$ is contained in the sum of the image of $T_1$
  and the image of $T_1 + T_2$.
\item Combining, we get that the sum of the images of all three of
  $T_1$, $T_2$, and $T_1 + T_2$ equals the sum of the images of any
  two of the three.
\end{itemize}

\section{Composition, kernel, and image}

\subsection{Dimensions of images of subspaces}

Suppose $T:U \to V$ is a linear transformation and $W$ is a subspace
of $U$. Then, $T(W)$ is a subspace of $V$, and the dimension of $T(W)$
is less than or equal to the dimension of $W$. It's fairly easy to see
this: if $S$ is a spanning set for $W$, then $T(S)$ is a spanning set
for $T(W) = \operatorname{Im}(T)$.

In fact, there is a stronger result, which we will return to later:

$$\operatorname{dim}(W) = \operatorname{dim}(\operatorname{Im}(T)) + \operatorname{dim}(\operatorname{ker}(T))$$

\subsection{Dimensions of inverse images of subspaces}

Suppose $T:U \to V$ is a linear transformation and $X$ is a linear
subspace of $V$. Then, $T^{-1}(X)$, defined as the set of vectors in
$U$ whose image is in $X$, is a subspace of $U$. We have the following
lower and upper bounds for the dimension of $T^{-1}(X)$:

$$\operatorname{dim}(\operatorname{Ker}(T)) \le \operatorname{dim}(T^{-1}(X)) \le \operatorname{dim}(\operatorname{Ker}(T)) + \operatorname{dim}(X)$$

The second inequality becomes an equality if and only if $X \subseteq
\operatorname{Im}(T)$.

The first inequality is easy to justify: since $X$ is a vector space,
the zero vector is in $X$. Thus, everything in $\operatorname{Ker}(T)$
maps inside $X$, so that:

$$\operatorname{Ker}(T) \subseteq T^{-1}(X)$$

Based on the observation made above about dimensions, we thus get the
first inequality:

$$\operatorname{dim}(\operatorname{Ker}(T)) \le \operatorname{dim}(T^{-1}(X))$$

The second inequality is a little harder to show directly. Suppose $Y
= T^{-1}(X)$. Now, you may be tempted to say that $X = T(Y)$. That is
not quite true. What is true is that $T(Y) = X \cap
\operatorname{Im}(T)$. Based on the result of the preceding section:

$$\operatorname{dim}(Y) = \operatorname{dim}(\operatorname{Ker}(T)) + \operatorname{dim}(T(Y))$$

Using the fact that $T(Y) \subseteq X$ we get the second inequality.

These generic statements are easiest to justify using the case of
projections.

\subsection{Composition, injectivity, and surjectivity in the abstract linear context}

The goal here is to adapt the statements we made earlier about
injectivity, surjectivity, and composition to the context of linear
transformations.

Suppose $T_1: U \to V$ and $T_2:V \to W$ are linear
transformations. The composite $T_2 \circ T_1: U \to W$ is also a
linear transformation. The following are true:

\begin{enumerate}
\item If $T_1$ and $T_2$ are both injective, then $T_2 \circ T_1$ is
  also injective. This follows from the corresponding statement for
  arbitrary functions.

  In linear transformation terms, if the kernel of $T_1$ is the zero
  subspace of $U$ and the kernel of $T_2$ is the zero subspace of $V$,
  then the kernel of $T_2 \circ T_1$ is the zero subspace of $U$.

\item If $T_1$ and $T_2$ are both surjective, then $T_2 \circ T_1$ is
  also surjective. This follows from the corresponding statement for
  arbitrary functions.

  In linear transformation terms, if the image of $T_1$ is all of $V$
  and the image of $T_2$ is all of $W$, then the image of $T_2 \circ
  T_1$ is all of $W$.

\item If $T_1$ and $T_2$ are both bijective, then $T_2 \circ T_1$ is
  also bijective. This follows from the corresponding statement for
  arbitrary functions. Bijective linear transformations between vector
  spaces are called {\em linear isomorphisms}, a topic we shall return
  to shortly.
\item If $T_2 \circ T_1$ is injective, then $T_1$ is injective. This
  follows from the corresponding statement for functions.

  In linear transformation terms, if the kernel of $T_2 \circ T_1$ is
  the zero subspace of $U$, the kernel of $T_1$ is also the zero
  subspace of $U$.
\item If $T_2 \circ T_1$ is surjective, then $T_2$ is surjective. This
  follows from the corresponding statement for functions.

  In linear transformation terms, if $T_2 \circ T_1$ has image equal
  to all of $W$, then the image of $T_2$ is also all of $W$.
\end{enumerate}

\subsection{Composition, kernel and image: more granular formulations}

The above statements about injectivity and surjectivity can be
modified somewhat to keep better track of the {\em lack} of
injectivity and surjectivity by using the dimension of the kernel and
image. Explicitly, suppose $T_1:U \to V$ and $T_2:V \to W$ are linear
transformations, so that the composite $T_2 \circ T_1: U \to W$ is
also a linear transformation. We then have:

\begin{enumerate}
\item The dimension of the kernel of the composite $T_2 \circ T_1$ is
  at {\em least} equal to the dimension of the kernel of $T_1$ and at
  {\em most} equal to the sum of the dimensions of the kernels of
  $T_1$ and $T_2$ respectively. This is because the kernel of $T_2
  \circ T_1$ can be expressed as $T_1^{-1}(T_2^{-1}(0))$, and we can
  apply the upper and lower bounds discussed earlier. Note that the
  lower bound is realized through an actual subspace containment: the
  kernel of $T_1$ is contained in the kernel of $T_2 \circ T_1$.
\item The dimension of the image of the composite $T_2 \circ T_1$ is
  at {\em most} equal to the minimum of the dimensions of the images
  of $T_1$ and $T_2$. The bound in terms of the image of $T_2$ is
  realized interms of an explicit subspace containment: the image of
  $T_2 \circ T_1$ is a subspace of the image of $T_2$. The bound in
  terms of the dimension of $T_1$ is more subtle: the image of $T_2
  \circ T_1$ is an image under $T_2$ of the image of $T_1$, therefore
  its dimension is at most equal to the dimension of the image of
  $T_1$.
\end{enumerate}

\subsection{Consequent facts about matrix multiplication and rank}

Suppose $A$ is a $m \times n$ matrix and $B$ is a $n \times p$ matrix,
where $m$, $n$, $p$ are (possibly equal, possibly distinct) positive
integers. Suppose the rank of $A$ is $r$ and the rank of $B$ is
$s$. Recall that the rank equals the dimension of the image of the
associated linear transformation. The difference between the dimension
of the domain and the rank equals the dimension of the kernel of the
associated linear transformation. We will later on use the term {\em
  nullity} to describe the dimension of the kernel. The nullity of $A$
is $n - r$ and the nullity of $B$ is $p - s$.

The matrix $AB$ is a $m \times p$ matrix. We can say the following:

\begin{itemize}
\item The rank of $AB$ is at {\em most} equal to the minimum of the
  ranks of $A$ and $B$. In this case, it is at most equal to $\min \{
  r,s \}$. This follows from the observation about the dimension of
  the image of a composite, made in the preceding section.
\item The nullity of $AB$ is at {\em least} equal to the nullity of
  $B$. In particular, in this case, it is at least equal to $p -
  s$. Note that the information about the maximum value of the rank
  actually allows us to improve this lower bound to $p - \min \{ r,
  s\}$, though that is not directly obvious at all.
\item In particular, if the product $AB$ has full column rank $p$,
  then $B$ has full column rank $p$, $s = p$, and $r \ge p$ (so in
  particular $m \ge p$ and $n \ge p$).
 
  Note that this is the linear algebra version of the statement that
  if the composite $g \circ f$ is injective, then $f$ must be
  injective.
\item If the product $AB$ has full row rank $m$, then $A$ has full row
  rank $m$, $r = m$, and $s \ge m$ (so in particular $n \ge m$ and $p
  \ge m$).

  Note that this is the linear algebra version of the statement that
  if $g \circ f$ is surjective, then $g$ is surjective.
\end{itemize}


\end{document}
