\documentclass[10pt]{amsart}

%Packages in use
\usepackage{fullpage, hyperref, vipul, enumerate}

%Title details
\title{Take-home class quiz solutions: due Wednesday December 4: Matrix transpose: Preliminaries}
\author{Math 196, Section 57 (Vipul Naik)}

%List of new commands

\begin{document}
\maketitle

\section{Performance review}

25 people took this 7-question quiz. The score distribution was as follows:

\begin{itemize}
\item Score of 4: 4 people
\item Score of 5: 5 people
\item Score of 6: 4 people
\item Score of 7: 12 people
\end{itemize}

The mean score was 5.96.

The question-wise answers and performance review were as follows:

\begin{enumerate}
\item Option (A): 25 people (everybody)
\item Option (D): 25 people (everybody)
\item Option (A): 23 people
\item Option (C): 22 people
\item Option (A): 18 people
\item Option (D): 18 people
\item Option (A): 18 people
\end{enumerate}

\section{Solutions}

{\bf PLEASE FEEL FREE TO DISCUSS {\em ALL} QUESTIONS.}

The following questions are related to material from parts of Chapter
5 that we are glossing over. You do not need to read that chapter,
because we are using a very limited part of it in a very limited
fashion and we've included all relevant definitions in the
quiz. However, if you want to understand some of the constructs in
more detail, please do read the chapter.

{\em Note}: Due to limited class time, I'm making this a take-home
class quiz, but in an ideal world, this would have been a diagnostic
in-class quiz.

For a $n \times m$ matrix $A$, denote by $A^T$ (spoken as {\em
  $A$-transposed} and called the {\em transpose of $A$}) the $m \times
n$ matrix whose $(ij)^{th}$ entry is defined as the $(ji)^{th}$ entry
of $A$. In other words, the roles of rows and columns are interchanged
when we transition from $A$ to $A^T$. The ${}^T$ should not be
interpreted as an exponent letter. Note that whereas $A$ describes a
linear transformation from $\R^m$ to $\R^n$, $A^T$ describes a linear
transformation from $\R^n$ to $\R^m$. Note, however, that although the
domain and co-domain for $A$ and $A^T$ are interchanged with each
other, $A$ and $A^T$ are not in general inverses of each other.


\begin{enumerate}
\item Suppose $A$ is a $n \times m$ matrix and $A^T$ is the tranpose
  of $A$. Under what conditions does the sum $A + A^T$ make sense
  (i.e., exist as a matrix)?

  \begin{enumerate}[(A)]
  \item $A + A^T$ makes sense if and only if $m = n$.
  \item $A + A^T$ makes sense if and only if $m < n$.
  \item $A + A^T$ makes sense if and only if $m > n$.
  \item $A + A^T$ makes sense regardless of whether $m = n$, $m < n$,
    or $m > n$.
  \end{enumerate} 

  {\em Answer}: Option (A)

  {\em Explanation}: $A$ is a $n \times m$ matrix whereas $A^T$ is a
  $m \times n$ matrix. We know that $A + A^T$ makes sense if and only
  if $A$ and $A^T$ have the same number of rows and also have the same
  number of columns. Both of these conditions are equivalent to
  requiring that $m = n$.

  {\em Performance review}: All 25 got this correct.

\item Suppose $A$ is a $n \times m$ matrix and $A^T$ is the transpose
  of $A$. Under what conditions does the product $AA^T$ make sense
  (i.e., exist as a matrix)?

  \begin{enumerate}[(A)]
  \item $AA^T$ makes sense if and only if $m = n$.
  \item $AA^T$ makes sense if and only if $m < n$.
  \item $AA^T$ makes sense if and only if $m > n$.
  \item $AA^T$ makes sense regardless of whether $m = n$, $m < n$,
    or $m > n$.
  \end{enumerate} 

  {\em Answer}: Option (D)

  {\em Explanation}: $A$ is a $n \times m$ matrix and $A^T$ is a $m
  \times n$ matrix. The product $AA^T$ makes sense because the number
  of columns in $A$ equals the number of rows in $A^T$, so the product
  $AA^T$ is a $n \times n$ matrix.

  {\em Performance review}: All 25 got this correct.

\item Suppose $A$ is a $n \times m$ matrix and $A^T$ is the transpose
  of $A$. Under what conditions do both $AA^T$ and $A^TA$ exist {\em
    and} have the same number of rows as each other and the same
  number of columns as each other (note that they still need not be
  equal)?

  \begin{enumerate}[(A)]
  \item This happens if and only if $m = n$.
  \item This happens if and only if $m < n$.
  \item This happens if and only if $m > n$.
  \item This happens always, regardless of whether $m = n$, $m < n$,
    or $m > n$.
  \end{enumerate} 

  {\em Answer}: Option (A)

  {\em Explanation}: The product $AA^T$ is a $n \times n$ matrix (see
  the explanation for Question 2). For similar reasons, the product
  $A^TA$ is a $m \times m$ matrix. These have the same size if and
  only if $m = n$.

  {\em Performance review}: 23 out of 25 got this. 2 chose (D).

\item Suppose $A$ is a $n \times n$ matrix such that $A^T =
  A^{-1}$. We describe this condition by saying that $A$ is an {\em
    orthogonal} $n \times n$ matrix. Which of the following is a
  correct characterization of a matrix being orthogonal? Please see
  Option (C) before answering, and select the option that best
  reflects your view.

  \begin{enumerate}[(A)]
  \item Every row vector of $A$ is a unit vector, and any two distinct
    rows of $A$ are orthogonal.
  \item Every column vector of $A$ is a unit vector, and any two
    distinct columns of $A$ are orthogonal.
  \item Both of the above work, i.e., they are equivalent to each
    other and to the condition that $A^T = A^{-1}$.
  \end{enumerate}

  {\em Answer}: Option (C)

  {\em Explanation}: The condition that $A^T = A^{-1}$ can be
  interpreted in two equivalent ways: $AA^T = I_n$ and $A^TA = I_n$.

  With the $AA^T = I_n$ interpretation, we see that the dot product of
  each row of $A$ with the corresponding column of $A^T$ is $1$, and
  the dot product of each row of $A$ with a different column of $A^T$
  is $0$. Since the ``corresponding column of $A^T$'' agrees with the
  original row of $A$, we obtain that the dot product of each row of
  $A$ with itself is $1$ (i.e., each row of $A$ is a unit vector) and
  the dot product of any two distinct rows of $A$ is $0$, i.e., any
  two distinct rows of $A$ are orthogonal.

  With the $A^TA = I_n$ interpretation, we obtain the analogous result
  for columns, because we are now dealing with dot products between
  rows of $A^T$ and columns of $A$.

  {\em Performance review}: 22 out of 25 got this. 2 chose (B), 1 chose (A).

  \vspace{0.5in}

  A square matrix $A$ is termed {\em symmetric} if $A = A^T$ and {\em
    skew-symmetric} if $A = -A^T$. 

  The following facts are true and can be easily verified:

  \begin{itemize}
  \item Suppose $A$ and $B$ are matrices such that $A + B$ makes
    sense. Then, $(A + B)^T = A^T + B^T$.
  \item Suppose $A$ and $B$ are matrices such that $AB$ makes
    sense. Then, $(AB)^T = B^TA^T$. Note that the order of
    multiplication flips over. The rule is similar to the rule for
    inverses, even though the transpose is {\em not} the same as the
    inverse.
  \item For any matrix $A$, $(A^T)^T = A$.
  \end{itemize}

\item Suppose $A$ is a matrix. What can we say that the nature of the
  matrices $A + A^T$ and $AA^T$?

  \begin{enumerate}[(A)]
  \item $A + A^T$ is symmetric if it makes sense. $AA^T$ is symmetric
    if it makes sense.
  \item $A + A^T$ is symmetric if it makes sense. $AA^T$ is
    skew-symmetric if it makes sense.
  \item $A + A^T$ is skew-symmetric if it makes sense. $AA^T$ is
    symmetric if it makes sense.
  \item $A + A^T$ is skew-symmetric if it makes sense. $AA^T$ is
    skew-symmetric if it makes sense.
  \end{enumerate}

  {\em Answer}: Option (A)

  {\em Explanation}: For the sum: $(A + A^T)^T = A^T + (A^T)^T = A^T +
  A = A + A^T$. We use that $(A^T)^T = A$ while simplifying.

  For the product: $(AA^T)^T = (A^T)^TA^T$ (we use the rule for
  transpose of a product). This simplifies to $AA^T$ using the fact
  that $(A^T)^T = A$.

  {\em Performance review}: 18 out of 25 got this. 5 chose (B), 2 chose (C).

\item Suppose $n$ is a positive integer. Consider the vector space
  $\R^{n \times n}$ of $n \times n$ matrices. The subset comprising
  symmetric matrices is a linear subspace and the subset comprising
  skew-symmetric matrices is also a linear subspace. The subset
  comprising diagonal matrices is also a linear subspace. Which of the
  following best describes the containment relation between the
  subspaces of diagonal matrices, symmetric matrices, and
  skew-symmetric matrices?

  \begin{enumerate}[(A)]
  \item The subspace comprising all diagonal matrices is contained in
    the subspace comprising all skew-symmetric matrices, which in turn
    is contained in the subspace comprising all symmetric matrices.
  \item The subspace comprising all diagonal matrices is contained
    both in the subspace comprising all skew-symmetric matrices and in
    the subspace comprising all symmetric matrices. However, neither
    of the two latter subspaces is contained in the other.
  \item The subspace comprising all diagonal matrices is contained in
    the subspace comprising all skew-symmetric matrices, but neither
    of these subspaces is contained in the subspace comprising all
    symmetric matrices.
  \item The subspace comprising all diagonal matrices is contained in
    the subspace comprising all symmetric matrices, but neither
    of these subspaces is contained in the subspace comprising all
    skew-symmetric matrices.
  \item None of the three subspaces is fully contained in any of the
    others.
  \end{enumerate}

  {\em Answer}: Option (D)

  {\em Explanation}: The subspace of diagonal matrices is contained in
  the subspace of skew-symmetric matrices, because any diagonal matrix
  is symmetric. This can easily be seen by looking at such a
  matrix. In the $2 \times 2$ case, a diagonal matrix is a matrix of
  the form:

  $$\left[\begin{matrix} a_{11} & 0 \\ 0 & a_{22} \\\end{matrix}\right]$$

  Clearly, such a matrix equals its own transpose.

  On the other hand, the {\em only} symmetric matrix that is also
  skew-symmetric is the zero matrix. Explicitly, if $A$ is a matrix
  satisfying both the conditions $A = A^T$ and $A + A^T = 0$, then we
  get $2A = 0$, hence $A = 0$. In particular, this also means that the
  only {\em diagonal} matrix that is skew-symmetric is the zero
  matrix. Thus, the space of diagonal matrices is not contained in the
  space of skew-symmetric matrices.

  {\em Performance review}: 18 out of 25 got this. 2 each chose (B),
  (C), and (E), 1 chose (A).

\item Suppose $n$ is a positive integer. Consider the vector space
  $\R^{n \times n}$ of $n \times n$ matrices. This vector space has
  dimension $n \times n = n^2$. What are the respective dimensions of
  the subspaces comprising symmetric and skew-symmetric matrices? {\em
    Hint}: Try the case $n = 1$ and then the case $n = 2$. In both
  cases, try to write down an explicit basis for each of the
  subspaces. You might want to revisit the preceding question in light
  of your improved understanding after solving this question.

  \begin{enumerate}[(A)]
  \item The subspace comprising all symmetric matrices has dimension
    $n(n + 1)/2$ and the subspace comprising all skew-symmetric
    matrices has dimension $n(n - 1)/2$.
  \item The subspace comprising all symmetric matrices has dimension
    $n(n - 1)/2$ and the subspace comprising all skew-symmetric
    matrices has dimension $n(n + 1)/2$.
  \item Both subspaces have dimension $n^2/2$.
  \end{enumerate}

  {\em Answer}: Option (A)

  {\em Explanation}: For a symmetric matrix $A$, we can freely choose
  all the diagonal entries (this gives $n$ dimensions) and for each
  pair $i \ne j$ of distinct elements of $\{ 1,2,\dots,n \}$, we can
  choose the entry $a_{ij} = a_{ji}$. There are $(n^2 - n)/2$
  dimensions arising from the latter choices. The total number of
  choices we have is therefore $n + (n^2 - n)/2 = n(n+1)/2$.

  In the skew-symmetric case, the diagonal entries all need to be
  zero. For each pair $i \ne j$ of distinct elements of $\{
  1,2,\dots,n\}$, we can choose the entry $a_{ij} = -a_{ji}$. There
  are $(n^2 - n)/2 = n(n - 1)/2$ such choices.

  In the $1 \times 1$ case, all matrices are symmetric and only the
  zero matrix is skew-symmetric. This agrees with our counts: the
  dimension of the subspace of symmetric matrices is $n(n+1)/2 = 1(1 +
  1)/2 = 1$ (as it should be, because that's the dimension of the
  whole space) and the dimension of the subspace of skew-symmetric
  matrices is $n(n-1)/2 = 1(1-1)/2 = 0$ (as it should be, because
  that's the dimension of the zero subspace).

  In the $2 \times 2$ case, the following matrices form a basis for
  the subspace of symmetric matrices (the basis has size $2(2+1)/2 =
  3$):

  $$\left[ \begin{matrix} 1 & 0 \\ 0 & 0 \\\end{matrix}\right], \left[\begin{matrix} 0 & 0 \\ 0 & 1 \\\end{matrix}\right], \left[\begin{matrix} 0 & 1 \\ 1 & 0 \\\end{matrix}\right]$$

  The subspace of skew-symmetric matrices has dimension $2(2-1)/2 = 1$
  and has a basis given by the following matrix:

  $$\left[ \begin{matrix} 0 & 1 \\ -1 & 0 \\\end{matrix}\right]$$

  In the $3 \times 3$ case, the following matrices form a basis for
  the subspace of symmetric matrices (there are $3(3 + 1)/2 = 6$ of them):

  $$\left[ \begin{matrix} 1 & 0 & 0 \\ 0 & 0 & 0 \\ 0 & 0 & 0 \\\end{matrix}\right], \left[ \begin{matrix} 0 & 0 & 0 \\ 0 & 1 & 0 \\ 0 & 0 & 0 \\\end{matrix}\right], \left[ \begin{matrix} 0 & 0 & 0 \\ 0 & 0 & 0 \\ 0 & 0 & 1 \\\end{matrix}\right]$$

  $$\left[\begin{matrix} 0 & 1 & 0 \\ 1 & 0 & 0 \\ 0 & 0 & 0 \\\end{matrix}\right], \left[\begin{matrix} 0 & 0 & 1 \\ 0 & 0 & 0 \\ 1 & 0 & 0 \\\end{matrix}\right], \left[\begin{matrix} 0 & 0 & 0 \\ 0 & 0 & 1 \\ 0 & 1 & 0 \\\end{matrix}\right]$$

  The following matrices form a basis for the subspace of
  skew-symmetric matrices (there are $3(3 - 1)/2 = 3$ of them):

  $$\left[ \begin{matrix} 0 & 1 & 0 \\ -1 & 0 & 0 \\ 0 & 0 & 0 \\\end{matrix}\right], \left[\begin{matrix} 0 & 0 & 1 \\ 0 & 0 & 0 \\ -1 & 0 & 0 \\\end{matrix}\right], \left[\begin{matrix} 0 & 0 & 0 \\ 0 & 0 & 1 \\ 0 & -1 & 0 \\\end{matrix}\right]$$

  {\em Performance review}: 18 out of 25 got this. 5 chose (B), 2 chose (C).

\end{enumerate}
\end{document}
