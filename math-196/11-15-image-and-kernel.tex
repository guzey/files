\documentclass[10pt]{amsart}

%Packages in use
\usepackage{fullpage, hyperref, vipul, enumerate}

%Title details
\title{Take-home class quiz: due Friday November 15: Image and kernel}
\author{Math 196, Section 57 (Vipul Naik)}
%List of new commands

\begin{document}
\maketitle

Your name (print clearly in capital letters): $\underline{\qquad\qquad\qquad\qquad\qquad\qquad\qquad\qquad\qquad\qquad}$

{\bf PLEASE FEEL FREE TO DISCUSS {\em ALL} QUESTIONS.}

The purpose of this quiz is to review in greater depth the ideas
behind image and kernel. The goal of the first seven questions is to
review the ideas of injectivity, surjectivity, and bijectivity in the
context of arbitrary functions between sets. The purpose is two-fold:
(i) to give a functions-based approach to justifying, intuitively and
formally, facts about the effect of matrix multiplication on rank, and
(ii) to hint at ways in which linear transformations behave better
than other types of functions.

The corresponding lecture notes are titled {\tt Image and kernel of a
  linear transformation} and the corresponding section of the text is
Section 3.1.

Just as a reminder, a function $f:A \to B$ between sets $A$ and $B$ is
said to be:

\begin{itemize}
\item {\em injective} if for every $b \in B$, there is {\em at most}
  one value of $a$ such that $f(a) = b$. In other words, if we denote
  by $f^{-1}(b)$ the set $\{ a \in A \mid f(a) = b \}$, then
  $|f^{-1}(b)| \le 1$ for all $b \in B$ (here $|f^{-1}(b)|$ denotes
  the size of the set $f^{-1}(b)$).
\item {\em surjective} if for every $b \in B$, there is {\em at least}
  one value of $a$ such that $f(a) = b$. In other words, if we denote
  by $f^{-1}(b)$ the set $\{ a \in A \mid f(a) = b \}$, then
  $|f^{-1}(b)| \ge 1$ for all $b \in B$.
\item {\em bijective} if for every $b \in B$, there is {\em exactly}
  one value of $a$ such that $f(a) = b$. In other words, if we denote
  by $f^{-1}(b)$ the set $\{ a \in A \mid f(a) = b \}$, then
  $|f^{-1}(b)| = 1$ for all $b \in B$.
\end{itemize}

\begin{enumerate}
\item Suppose $g:A \to B$ and $f:B \to C$ are functions. The composite
  $f \circ g$ is a function from $A$ to $C$. What can we say the
  relationship between the injectivity of $f \circ g$, the injectivity
  of $f$, and the injectivity of $g$?

  \begin{enumerate}[(A)]
  \item $f \circ g$ is injective if and only if $f$ and $g$ are both injective.
  \item If $f$ and $g$ are both injective, then $f \circ g$ is
    injective. However, $f \circ g$ being injective does not imply
    anything about the injectivity of either $f$ or $g$.
  \item If $f$ and $g$ are both injective, then $f \circ g$ is
    injective. If $f \circ g$ is injective, then at least one of $f$
    and $g$ is injective, but we cannot conclusively say for any
    specific one of the two that it must be injective.
  \item If $f$ and $g$ are both injective, then $f \circ g$ is
    injective. If $f \circ g$ is injective, then $f$ is injective, but
    we do not have enough information to deduce whether $g$ is
    injective.
  \item If $f$ and $g$ are both injective, then $f \circ g$ is
    injective. If $f \circ g$ is injective, then $g$ is injective, but
    we do not have enough information to deduce whether $f$ is
    injective.
  \end{enumerate}

  \vspace{0.1in}
  Your answer: $\underline{\qquad\qquad\qquad\qquad\qquad\qquad\qquad}$
  \vspace{0.1in}

\item Suppose $g:A \to B$ and $f:B \to C$ are functions. The composite
  $f \circ g$ is a function from $A$ to $C$. What can we say the
  relationship between the surjectivity of $f \circ g$, the surjectivity
  of $f$, and the surjectivity of $g$?

  \begin{enumerate}[(A)]
  \item $f \circ g$ is surjective if and only if $f$ and $g$ are both
    surjective.
  \item If $f$ and $g$ are both surjective, then $f \circ g$ is
    surjective. However, $f \circ g$ being surjective does not imply
    anything about the surjectivity of either $f$ or $g$.
  \item If $f$ and $g$ are both surjective, then $f \circ g$ is
    surjective. If $f \circ g$ is surjective, then at least one of $f$
    and $g$ is surjective, but we cannot conclusively say for any
    specific one of the two that it must be surjective.
  \item If $f$ and $g$ are both surjective, then $f \circ g$ is
    surjective. If $f \circ g$ is surjective, then $f$ is surjective,
    but we do not have enough information to deduce whether $g$ is
    surjective.
  \item If $f$ and $g$ are both surjective, then $f \circ g$ is
    surjective. If $f \circ g$ is surjective, then $g$ is surjective,
    but we do not have enough information to deduce whether $f$ is surjective.
  \end{enumerate}

  \vspace{0.1in}
  Your answer: $\underline{\qquad\qquad\qquad\qquad\qquad\qquad\qquad}$
  \vspace{0.1in}

\item Suppose $g:A \to B$ and $f:B \to C$ are functions. The composite
  $f \circ g$ is a function from $A$ to $C$. Suppose $f \circ g$ is
  bijective. What can we say about $f$ and $g$ individually?

  \begin{enumerate}[(A)]
  \item Both $f$ and $g$ must be bijective.
  \item Both $f$ and $g$ must be injective, but neither of them need be surjective.
  \item Both $f$ and $g$ must be surjective, but neither of them need be injective.
  \item $f$ must be injective but need not be surjective. $g$ must be
    surjective but need not be injective.
  \item $f$ must be surjective but need not be injective. $g$ must be
    injective but need not be surjective.
  \end{enumerate}

  \vspace{0.1in}
  Your answer: $\underline{\qquad\qquad\qquad\qquad\qquad\qquad\qquad}$
  \vspace{0.1in}

\item $g:A \to B$ and $f:B \to C$ are functions. The composite $f
  \circ g$ is a function from $A$ to $C$. Suppose both $f$ and $g$ are
  surjective. Further, suppose that for every $b \in B$, $g^{-1}(b)$
  has size $m$ (for a fixed positive integer $m$) and for every $c \in
  C$, $f^{-1}(c)$ has size $n$ (for a fixed positive integer
  $n$). Then, what can we say about the sizes of the fibers (i.e., the
  inverse images of points in $C$) under the composite $f \circ g$?

  \begin{enumerate}[(A)]
  \item The size is $\min \{ m,n \}$
  \item The size is $\max \{ m,n \}$
  \item The size is $m + n$
  \item The size is $mn$
  \item The size is $m^n$
  \end{enumerate}

  \vspace{0.1in}
  Your answer: $\underline{\qquad\qquad\qquad\qquad\qquad\qquad\qquad}$
  \vspace{0.1in}

\item {\bf PLEASE READ THIS VERY CAREFULLY AND CONSIDER A WIDE VARIETY
  OF POLYNOMIAL EXAMPLES}: Suppose $f$ is a polynomial function of
  degree $n > 2$ from $\R$ to $\R$. What can we say about the fibers
  of $f$, i.e., the sets of the form $f^{-1}(x)$, $x \in \R$?

  {\em Hint}: At the one extreme, consider a polynomial of the form
  $x^n$. Consider the sizes of the fibers $f^{-1}(0)$ and $f^{-1}(x)$
  for a positive value of $x$ (the fiber size for the latter will
  depend on whether $n$ is even or odd). Alternatively, consider a
  polynomial of the form $(x - 1)(x - 2) \dots (x - n)$. Consider the
  size of the fiber $f^{-1}(0)$.
  \begin{enumerate}[(A)]
  \item Every fiber has size $n$.
  \item The minimum of the sizes of fibers is exactly $n$, but every
    fiber need not have size $n$.
  \item The maximum of the sizes of fibers is exactly $n$, but every
    fiber need not have size $n$.
  \item The minimum of the sizes of fibers is at least $n$, but need
    not be exactly $n$.
  \item The maximum of the sizes of fibers is at most $n$, but need not
    be exactly $n$.
  \end{enumerate}

  \vspace{0.1in}
  Your answer: $\underline{\qquad\qquad\qquad\qquad\qquad\qquad\qquad}$
  \vspace{0.1in}

\item Suppose $f$ is a continuous injective function from $\R$ to
  $\R$. What can we say about the nature of $f$?

  \begin{enumerate}[(A)]
  \item $f$ must be an increasing function on all of $\R$.
  \item $f$ must be a decreasing function on all of $\R$.
  \item $f$ must be a constant function on all of $\R$.
  \item $f$ must be either an increasing function on all of $\R$ or a
    decreasing function on all of $\R$, but the information presented
    is insufficient to decide which case occurs.
  \item $f$ must be either an increasing function or a decreasing
    function or a constant function on all of $\R$, but the
    information presented is insufficient for deciding anything
    stronger.
  \end{enumerate}

  \vspace{0.1in}
  Your answer: $\underline{\qquad\qquad\qquad\qquad\qquad\qquad\qquad}$
  \vspace{0.1in}

\item {\bf PLEASE READ THIS CAREFULLY, MAKE CASES, AND CHECK YOUR
  REASONING}: Suppose $f$, $g$, and $h$ are continuous bijective
  functions from $\R$ to $\R$. What can we say about the functions $f
  + g$, $f + h$, and $g + h$?

  {\em Hint}: Based on the preceding question, you know something
  about the nature of $f$, $g$, and $h$ individually as functions, but
  there is some degree of ambiguity in your knowledge. Make cases
  based on the possibilities and see what you can deduce in the best
  and worst case.

  \begin{enumerate}[(A)]
  \item They are all continuous bijective functions from $\R$ to $\R$.
  \item At least two of them are continuous bijective functions from
    $\R$ to $\R$. However, we cannot say more.
  \item At least one of them is a continuous bijective function from
    $\R$ to $\R$. However, we cannot say more.
  \item Either all three sums are continuous bijective functions from
    $\R$ to $\R$, or none is.
  \item It is possible that none of the sums is a continuous
    bijective functions from $\R$ to $\R$; it is
    also possible that one, two, or all the sums are continuous
    bijective functions from $\R$ to $\R$.
  \end{enumerate}

  \vspace{0.1in}
  Your answer: $\underline{\qquad\qquad\qquad\qquad\qquad\qquad\qquad}$
  \vspace{0.5in}

  The questions that follow tripped up students quite a bit last time,
  so I urge you to proceed with caution. You can do each of these
  questions in either of two ways:

  \begin{itemize}
  \item Using abstract, general reasoning.
  \item Constructing concrete examples.
  \end{itemize}

  While the former approach is one you should eventually be able to
  embrace without trepidation, feel free to rely on the latter
  approach for now. For this, consider matrices describing the linear
  transformations and use matrix multiplication to compute the
  composite where needed. Compute the kernel, image, and rank using
  the methods known to you. Take matrices such as those arising from
  finite state automata (as described in the ``linear transformations
  and finite state automata'' quiz) or their generalizations to
  rectangular matrices.

  For instance, you might try taking a matrix such as
  $\left[ \begin{matrix} 1 & 0 & 0 & 0 & 0 \\ 0 & 1 & 0 & 0 & 0\\ 0 &
      0 & 1 & 0 & 0 \\ 0 & 0 & 0 & 0 & 0 \\\end{matrix}\right]$. This
  describes a linear transformation $\R^5 \to \R^4$ and has rank
  three. The dimension of the kernel (inside $\R^5$) is 2 (explicitly,
  the kernel is precisely the set of vectors in $\R^5$ whose first
  three coordinates are zero) and the dimension of the image (inside
  $\R^4$) is 3 (explicitly, the image is precisely the set of vectors
  in $\R^4$ whose fourth coordinate is $0$).

\item {\em This is the analogue for linear transformations of Question
  1}: Suppose $m,n,p$ are positive integers. Suppose $A$ is a $m
  \times n$ matrix and $B$ is a $n \times p$ matrix. The product $AB$
  is a $m \times p$ matrix. Denote by $T_A$, $T_B$, and $T_{AB}$
  respectively the linear transformations corresponding to $A$, $B$,
  and $AB$. We have $T_A:\R^n \to \R^m$, $T_B: \R^p \to \R^n$, and
  $T_{AB}: \R^p \to \R^m$. Note that $T_{AB} = T_A \circ T_B$.

  Recall that a matrix has full column rank if and only if the
  corresponding linear transformation is injective.

  Which of the following describes correctly the relationship between
  $A$ having full column rank (i.e., rank $n$), $B$ having full column
  rank (i.e., rank $p$), and $AB$ having full column rank (i.e., rank
  $p$)?

  \begin{enumerate}[(A)]
  \item $AB$ has full column rank (i.e., rank $p$) if and only if $A$
    and $B$ both have full column rank (ranks $n$ and $p$
    respectively).
  \item If $A$ and $B$ both have full column rank, then $AB$ has full
    column rank. However, $AB$ having full column rank does not imply
    anything (separately or jointly) regarding whether $A$ or $B$ has
    full column rank.
  \item If $A$ and $B$ both have full column rank, then $AB$ has full
    column rank. If $AB$ has full column rank, then at least one of
    $A$ and $B$ has full column rank, but we cannot definitively say
    for any particular one of $A$ and $B$ that it must have full column
    rank.
  \item If $A$ and $B$ both have full column rank, then $AB$ has full
    column rank. $AB$ having full column rank implies that $A$ has
    full column rank, but it does not tell us for sure that $B$ has
    full column rank.
  \item If $A$ and $B$ both have full column rank, then $AB$ has full
    column rank. $AB$ having full column rank implies that $B$ has
    full column rank, but it does not tell us for sure that $A$ has
    full column rank.
  \end{enumerate}

  \vspace{0.1in}
  Your answer: $\underline{\qquad\qquad\qquad\qquad\qquad\qquad\qquad}$
  \vspace{0.1in}

\item {\em This is the analogue for linear transformations of Question
  2}: Suppose $m,n,p$ are positive integers. Suppose $A$ is a $m
  \times n$ matrix and $B$ is a $n \times p$ matrix. The product $AB$
  is a $m \times p$ matrix. Denote by $T_A$, $T_B$, and $T_{AB}$
  respectively the linear transformations corresponding to $A$, $B$,
  and $AB$. We have $T_A:\R^n \to \R^m$, $T_B: \R^p \to \R^n$, and
  $T_{AB}: \R^p \to \R^m$. Note that $T_{AB} = T_A \circ T_B$.

  Recall that a matrix has full row rank if and only if the
  corresponding linear transformation is surjective.

  Which of the following describes correctly the relationship between
  $A$ having full row rank (i.e., rank $m$), $B$ having full row rank
  (i.e., rank $n$), and $AB$ having full row rank (i.e., rank $m$)?

  \begin{enumerate}[(A)]
  \item $AB$ has full row rank if and only if $A$ and $B$ both have
    full row rank.
  \item If $A$ and $B$ both have full row rank, then $AB$ has full row
    rank. However, $AB$ having full row rank does not imply anything
    (separately or jointly) regarding whether $A$ or $B$ has full row
    rank.
  \item If $A$ and $B$ both have full row rank, then $AB$ has full
    row rank. If $AB$ has full row rank, then at least one of
    $A$ and $B$ has full row rank, but we cannot definitively say
    for any particular one of $A$ and $B$ that it must have full row
    rank.
  \item If $A$ and $B$ both have full row rank, then $AB$ has full
    row rank. $AB$ having full row rank implies that $A$ has
    full row rank, but it does not tell us for sure that $B$ has
    full row rank.
  \item If $A$ and $B$ both have full row rank, then $AB$ has full
    row rank. $AB$ having full row rank implies that $B$ has
    full row rank, but it does not tell us for sure that $A$ has
    full row rank.
  \end{enumerate}

  \vspace{0.1in}
  Your answer: $\underline{\qquad\qquad\qquad\qquad\qquad\qquad\qquad}$
  \vspace{0.1in}

\item {\em This is the analogue for linear transformations of Question
  3}: Suppose $m$ and $n$ are positive integers. Suppose $A$ is a $m
  \times n$ matrix and $B$ is a $n \times m$ matrix. The product $AB$
  is a $m \times m$ matrix. The corresponding linear transformations
  are $T_A: \R^n \to \R^m$, $T_B: \R^m \to \R^n$, and $T_{AB}: \R^m
  \to \R^m$. 

  Suppose the square matrix $AB$ has full rank $m$. What can we deduce
  about the ranks of $A$ and $B$?

  \begin{enumerate}[(A)]
  \item Both $A$ and $B$ have full row rank, {\em and} both $A$ and
    $B$ have full column rank.
  \item Both $A$ and $B$ have full column rank, but neither of them
    need have full row rank.
  \item Both $A$ and $B$ have full row rank, but neither of them need
    have full column rank.
  \item $A$ must have full column rank but need not have full row
    rank. $B$ must have full row rank but need not have full column
    rank.
  \item $A$ must have full row rank but need not have full column
    rank. $B$ must have full column rank but need not have full row
    rank.
  \end{enumerate}

  \vspace{0.1in}
  Your answer: $\underline{\qquad\qquad\qquad\qquad\qquad\qquad\qquad}$
  \vspace{0.5in}

  For the coming questions, we will denote vector spaces by letters
  such as $U$, $V$, and $W$. You can, however, consider them to be
  finite-dimensional vector spaces of the form $\R^n$. However, you
  should take care not to use a letter for the dimension of a vector
  space if the letter is already in use elsewhere in the
  question. Also, you should take care to use different letters for
  the dimensions of different vector spaces, unless it is given to you
  that the vector spaces have the same dimension. The results also
  hold for infinite-dimensional vector spaces, but you can work on all
  the problems assuming you are working in the finite-dimensional
  setting.

\item {\em This is an analogue for linear transformations of Question
  4}: Suppose $T_1: U \to V$ and $T_2:V \to W$ are linear
  transformations. The composite $T_2 \circ T_1$ is also a linear
  transformation, this time from $U$ to $W$. Suppose the kernel of
  $T_1$ has dimension $m$ and the kernel of $T_2$ has dimension
  $n$. Suppose both $T_1$ and $T_2$ are surjective. What can you say
  about the dimension of the kernel of $T_2 \circ T_1$?

  {\em Please note this carefully}: Although this question is
  analogous to Question 4, the correct answer options differ for the
  two questions. Here is an intuitive explanation for the relationship
  between the questions. Question 4 asked abot the {\em sizes} of the
  fibers. This question asks about the dimensions of the kernels. The
  fibers do correspond to the kernels. But the relationship between
  dimension and size is of a {\em logarithmic nature}. What we mean is
  that the dimension can be thought of as the logarithm of the
  size. This isn't literally true, because the size is infinite. But
  metaphorically, it makes sense, because, for instance, the dimension
  of $\R^p$ is the exponent $p$, and that comports with the laws of
  logarithms (similar to how the $\log_2(2^p) = p$).
  
  \begin{enumerate}[(A)]
  \item The dimension is $\min \{ m,n \}$.
  \item The dimension is $\max \{ m,n \}$.
  \item The dimension is $m + n$.
  \item The dimension is $mn$.
  \item The dimension is $m^n$.
  \end{enumerate}

  \vspace{0.1in}
  Your answer: $\underline{\qquad\qquad\qquad\qquad\qquad\qquad\qquad}$
  \vspace{0.1in}

\item Suppose $T_1: U \to V$ and $T_2:V \to W$ are linear
  transformations. The composite $T_2 \circ T_1$ is also a linear
  transformation, this time from $U$ to $W$. Suppose the kernel of
  $T_1$ has dimension $m$ and the kernel of $T_2$ has dimension
  $n$. However, unlike the preceding question, we are not given any
  information about the surjectivity of either $T_1$ or $T_2$. The
  answer to the preceding question gives an (inclusive) {\em upper}
  bound on the dimension of the kernel of $T_2 \circ T_1$. Which of
  the following is the best {\em lower} bound we can manage in
  general?

  \begin{enumerate}[(A)]
  \item $|m - n|$
  \item $m$
  \item $n$
  \item $m + n$
  \end{enumerate}

  \vspace{0.1in}
  Your answer: $\underline{\qquad\qquad\qquad\qquad\qquad\qquad\qquad}$
  \vspace{0.1in}

\item Suppose $T_1,T_2:U \to V$ are linear transformations. Which of
  the following is true? Please see Options (D) and (E) before
  answering and select the single option that best reflects your view.

  \begin{enumerate}[(A)]
  \item If both $T_1$ and $T_2$ are injective, then $T_1 + T_2$ is injective.
  \item If both $T_1$ and $T_2$ are surjective, then $T_1 + T_2$ is surjective.
  \item If both $T_1$ and $T_2$ are bijective, then $T_1 + T_2$ is bijective.
  \item All of the above
  \item None of the above
  \end{enumerate}

  \vspace{0.1in}
  Your answer: $\underline{\qquad\qquad\qquad\qquad\qquad\qquad\qquad}$
  \vspace{0.1in}

\item Suppose $T_1, T_2: U \to V$ are linear transformations. Which of
  the following best describes the relation between the kernels of
  $T_1$, $T_2$, and $T_1 + T_2$?

  \begin{enumerate}[(A)]
  \item The kernel of $T_1 + T_2$ equals the intersection of the
    kernel of $T_1$ and the kernel of $T_2$.
  \item The kernel of $T_1 + T_2$ is contained inside the intersection
    of the kernel of $T_1$ and the kernel of $T_2$, but need not be
    equal to the intersection.
  \item The kernel of $T_1 + T_2$ contains the intersection of the
    kernel of $T_1$ and the kernel of $T_2$, but need not be equal to
    the intersection.
  \item The kernel of $T_1 + T_2$ is contained inside the sum of the
    kernel of $T_1$ and the kernel of $T_2$, but need not be equal to
    the sum.
  \item The kernel of $T_1 + T_2$ contains the sum of the kernel of
    $T_1$ and the kernel of $T_2$, but need not be equal to the sum.
  \end{enumerate}

  \vspace{0.1in}
  Your answer: $\underline{\qquad\qquad\qquad\qquad\qquad\qquad\qquad}$
  \vspace{0.1in}

\item Suppose $T_1, T_2: U \to V$ are linear transformations. Which of
  the following best describes the relation between the images of
  $T_1$, $T_2$, and $T_1 + T_2$?

  \begin{enumerate}[(A)]
  \item The image of $T_1 + T_2$ equals the intersection of the
    image of $T_1$ and the image of $T_2$.
  \item The image of $T_1 + T_2$ is contained inside the intersection
    of the image of $T_1$ and the image of $T_2$, but need not be
    equal to the intersection.
  \item The image of $T_1 + T_2$ contains the intersection of the
    image of $T_1$ and the image of $T_2$, but need not be equal to
    the intersection.
  \item The image of $T_1 + T_2$ is contained inside the sum of the
    image of $T_1$ and the image of $T_2$, but need not be equal to
    the sum.
  \item The image of $T_1 + T_2$ contains the sum of the image of
    $T_1$ and the image of $T_2$, but need not be equal to the sum.
  \end{enumerate}

  \vspace{0.1in}
  Your answer: $\underline{\qquad\qquad\qquad\qquad\qquad\qquad\qquad}$
  \vspace{0.1in}

\item Suppose $T$ is a linear transformation from a vector space $V$
  to itself. Note that $V$ may be an infinite-dimensional space, such
  as $C^\infty(\R)$ (with $T$ being differentiation), but for
  convenience, you can imagine $V$ to be finite-dimensional (we will
  not reference the dimension of $V$ in this question,
  however). Suppose the kernel of $T$ has dimension $n$. What can you
  say from this information about the dimension of the kernel of $T^r$
  for a positive integer $r$?

  \begin{enumerate}[(A)]
  \item It is at least $n$ and at most $n + r$.
  \item It is at least $n$ and at most $nr$.
  \item It is at least $n + r$ and at most $nr$.
  \item It is at least $n + r$ and at most $n^r$.
  \end{enumerate}

  \vspace{0.1in}
  Your answer: $\underline{\qquad\qquad\qquad\qquad\qquad\qquad\qquad}$
  \vspace{0.5in}

  The next few questions deal with the relationship between the rows
  and columns of the matrix on the one hand, and the image and kernel
  of the linear transformation on the other hand.

\item Suppose $A$ is a $n \times m$ matrix and $T_A: \R^m \to \R^n$ is
  the corresponding linear transformation. Which of the following
  correctly describes the relationship between the rows and columns of
  $A$ and the image and kernel of $T_A$?

  \begin{enumerate}[(A)]
  \item The kernel of $T_A$ is precisely the subspace of $\R^m$
    spanned by the rows of $A$. The image of $T_A$ is precisely the
    subspace of $\R^n$ spanned by the columns of $A$.
  \item The kernel of $T_A$ is precisely the subspace of $\R^m$
    spanned by the columns of $A$. The image of $T_A$ is precisely the
    subspace of $\R^n$ spanned by the rows of $A$.
  \item The kernel of $T_A$ is precisely the subspace of $\R^m$
    comprising the vectors that are {\em orthogonal} to the rows of
    $A$. The image of $T_A$ is precisely the subspace of $\R^n$
    comprising the vectors that are {\em orthogonal} to the columns of
    $A$.
  \item The kernel of $T_A$ is precisely the subspace of $\R^m$
    comprising the vectors that are {\em orthogonal} to the rows of
    $A$. The image of $T_A$ is the subspace of $\R^n$ spanned by the
    columns of $A$.
  \item The kernel of $T_A$ is precisely the subspace of $\R^m$
    spanned by the rows of $A$. The image of $T_A$ is precisely the
    subspace of $\R^n$ comprising the vectors that are {\em
      orthogonal} to the columns of $A$.
  \end{enumerate}

  \vspace{0.1in}
  Your answer: $\underline{\qquad\qquad\qquad\qquad\qquad\qquad\qquad}$
  \vspace{0.1in}

\item Suppose $A$ and $B$ are $n \times m$ matrices, $T_A:\R^m \to
  \R^n$ is the linear transformation corresponding to $A$, and
  $T_B:\R^m \to \R^n$ is the linear transformation corresponding to
  $B$. Which of the following correctly describes the relation between
  the rows, columns, image and kernel? Please see Option (E) before
  answering.

  \begin{enumerate}[(A)]
  \item If $B$ can be obtained from $A$ by a sequence of row
    interchange operations, then $T_A$ and $T_B$ have the same kernel
    as each other and also the same image as each other.
  \item If $B$ can be obtained from $A$ by a sequence of column
    interchange operations, then $T_A$ and $T_B$ have the same kernel
    as each other and also the same image as each other.
  \item If $B$ can be obtained from $A$ by a sequence of row
    interchange operations, then $T_A$ and $T_B$ have the same kernel
    as each other. If $B$ can be obtained from $A$ by a sequence of
    column interchange operations, then $T_A$ and $T_B$ have the same
    image as each other.
  \item If $B$ can be obtained from $A$ by a sequence of row
    interchange operations, then $T_A$ and $T_B$ have the same image
    as each other. If $B$ can be obtained from $A$ by a sequence of
    column interchange operations, then $T_A$ and $T_B$ have the same
    kernel as each other.
  \item None of the above.
  \end{enumerate}

  \vspace{0.1in}
  Your answer: $\underline{\qquad\qquad\qquad\qquad\qquad\qquad\qquad}$
  \vspace{0.1in}

\end{enumerate}
\end{document}
