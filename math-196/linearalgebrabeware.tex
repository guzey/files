\documentclass[10pt]{amsart}
\usepackage{fullpage,hyperref,vipul,graphicx}
\title{Linear algebra: beware!}
\author{Math 196, Section 57 (Vipul Naik)}

\begin{document}
\maketitle

You might be expecting linear algebra to be a lot like your calculus
classes at the University. This is probably true in terms of the
course structure and format. But it's not true at the level of subject
matter. Some important differences are below.

\begin{itemize}
\item Superficially, linear algebra is a lot easier, since it relies
  mostly on arithmetic rather than algebra. The computational
  procedures involve the systematic and correct application of
  processes for adding, subtracting, multiplying, and dividing
  numbers. But the key word here is {\em superficially}. 
\item Even for the apparently straightforward computational exercises,
  it turns out that people are able to do them a lot better if they
  understand what's going on. In fact, in past test questions, people
  have often made fewer errors when doing the problem using full-scale
  algebraic symbol manipulation rather than the synthetic arithmetic
  method.
\item One important difference between linear algebra and calculus is
  that with calculus, it's relatively easy to understand ideas {\em
    partially}. One can obtain much of the basic intuition of calculus
  by understanding graphs of functions. In fact, limit, continuity,
  differentaaition, and integration all have basic descriptions in
  terms of the graph. Note: these aren't fully rigorous, which is why
  you had to take a year's worth of calculus class to cement your
  understanding of the ideas. But it's a {\em start}. With linear
  algebra, there is no single compelling visual tool that connects all
  the ideas, and conscious effort is needed even for a partial
  understanding.
\item While linear algebra lacks any {\em single} compelling visual
  tool, it requires {\em either} considerable visuo-spatial skill {\em
    or} considerable abstract symbolic and verbal skill (or a suitable
  linear combination thereof). Note the gap here: the standard
  computational procedures require only arithmetic. But getting an
  understanding requires formidable visuo-spatial and/or symbolic
  manipulation skill. So one can become a maestro at manipulating
  matrices without understanding anything about the meaning or purpose
  thereof.
\item Finally, even if you master linear algebra, the connection of
  linear algebra to its applications is relatively harder to grasp
  than the connection of calculus to its applications. Applying
  multivariable calculus is easy: {\em marginal rates of change equal
    (partial) derivatives}. Summing up over a continuously variable
  parameter equals integration. One can get quite far with just these
  two ideas. With linear algebra, on the other hand, there is no
  punchline. There's no easy way of spotting when and how a given
  situation in economics or computer science or any other branch of
  mathematics requires the use of linear algebra.
\end{itemize}

Keeping all the above in mind, you can treat this course in either of
two ways. You can look at it as a bunch of arithmetic procedures (with
the occasional algebraic insight behind a procedure). From that
perspective, it's a ``grind''-course: relatively low-skilled, but
requires a lot of gruntwork and concentration. Or, you can look at it
as a bunch of difficult ideas to be mastered, with a few arithmetic
procedures to codify the execution of these ideas. From that
perspective, it's an intellectually challenging course, and if you
succeed, a satisfying one.

In practice, I would recommend seeing the course as a mix. Make sure
you master the basic computational procedures. I'll try to keep them
to the bare minimum you need to attain familiarity with the structures
involved. The homeworks (particularly the routine homeworks) will
serve that role. And if I fall short on the explanations in class (I
hope I don't), the book will fill the gap.

And give a decent shot to trying to understand the concepts. I can't
guarantee success. In fact, as mentioned above, you're much more
likely to take away zero conceptual knowledge from a linear algebra
course than you are from a calculus course. But at least if you try,
you have {\em some} chance. Give it a shot. The quizzes will help you
with that.


\end{document}
