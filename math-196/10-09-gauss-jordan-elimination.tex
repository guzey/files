\documentclass[10pt]{amsart}

%Packages in use
\usepackage{fullpage, hyperref, vipul, enumerate}

%Title details
\title{Diagnostic in-class quiz: due Friday October 11: Gauss-Jordan elimination (originally due Wednesday October 9, but postponed)}
\author{Math 196, Section 57 (Vipul Naik)}
%List of new commands

\begin{document}
\maketitle

Your name (print clearly in capital letters): $\underline{\qquad\qquad\qquad\qquad\qquad\qquad\qquad\qquad\qquad\qquad}$

{\bf PLEASE DO {\em NOT} DISCUSS ANY QUESTIONS}

The quiz covers basics related to Gauss-Jordan elimination (notes
titled {\tt Gauss-Jordan elimination}, corresponding section in the
book Section 1.2). Explicitly, the quiz covers:

\begin{itemize}
\item Setting up linear systems and interpreting the coefficient
  matrix in terms of the setup.
\item Knowledge of the permissible rules for manipulating linear
  systems.
\item Metacognition of the process of Gauss-Jordan elimination and its
  eventual result, the reduced row-echelon form, as well as the
  interpretation in terms of the solution set.
\end{itemize}

The questions are fairly easy questions if you understand the
material. But it's important that you be able to answer them,
otherwise what we study later will not make much sense.

\begin{enumerate}
\item {\em Do not discuss this!}: The row operations that we can
  perform on the augmented matrix of a linear system include adding or
  subtracting another row. However, they do not include multiplying
  another row. In other words, suppose we start with:

  $$\left[\begin{matrix} 1 & 2 & | & 5\\ 2 & 7 & \mid & 6 \\\end{matrix}\right]$$

  What we're not allowed to do is multiply row 2 by row 1 and obtain:

  $$\left[\begin{matrix} 1 & 2 & | & 5\\ 2 & 14 & \mid & 30 \\\end{matrix}\right]$$

  What's the most compelling reason for our not being allowed to
  perform this operation?
  \begin{enumerate}[(A)]
  \item The row operations arise from the corresponding operations on
    equations. For the ``multiplication of rows'' operation to be
    legitimate, it must correspond to multiplication of the
    corresponding equations, and multiplying equations is not a
    legitimate operation.
  \item The row operations arise from the corresponding operations on
    equations. However, the ``multiplication of rows'' operation does
    not correspond to any legitimate operation on equations. Note that
    it does not correspond to multiplying the equations, because that
    is not how multiplication of linear polynomials work (in fact, if
    we multiplied the equations, we would end up with an equation that
    is not linear).
  \end{enumerate}

  \vspace{0.1in}
  Your answer: $\underline{\qquad\qquad\qquad\qquad\qquad\qquad\qquad}$
  \vspace{0.1in}

\item {\em Do not discuss this!}: Consider a model where the
  functional form is linear in the parameters (though not necessarily
  in the inputs). We can use (input, output) pairs to set up a system
  of linear equations in the parameters. Given enough such equations,
  we can determine the values of the parameters.

  What is the relation between the coefficient matrix and the
  parameters and (input, output) pairs?

  \begin{enumerate}[(A)]
  \item The columns of the coefficient matrix correspond to the
    (input, output) pairs and the rows correspond to the parameters.
  \item The rows of the coefficient matrix correspond to the
    (input, output) pairs and the columns correspond to the parameters.
  \end{enumerate}

  \vspace{0.1in}
  Your answer: $\underline{\qquad\qquad\qquad\qquad\qquad\qquad\qquad}$
  \vspace{0.1in}

\item {\em Do not discuss this!}: Consider a model where the
  functional form is linear in the parameters (though not necessarily
  in the inputs). We can use (input, output) pairs to set up a system
  of linear equations in the parameters. Given enough such equations,
  we can determine the values of the parameters.

  What is the relation between the inputs, the outputs, the
  coefficient matrix, and the augmenting column?

  \begin{enumerate}[(A)]
  \item The inputs correspond to the coefficient matrix and the
    outputs correspond to the augmenting column. In other words,
    knowing the values of the inputs allows us to write down the
    coefficient matrix. Knowing the values of the outputs allows us to
    write down the augmenting column.
  \item The outputs correspond to the coefficient matrix and the
    inputs correspond to the augment column. In other words, knowing
    the values of the outputs allows us to write down the coefficient
    matrix. Knowing the values of the inputs allows us to write down
    the augmenting column.
  \end{enumerate}

  \vspace{0.1in}
  Your answer: $\underline{\qquad\qquad\qquad\qquad\qquad\qquad\qquad}$
  \vspace{0.1in}

\item {\em Do not discuss this!}: Consider the following rule to check
  for consistency using the augmented matrix: the system is
  inconsistent if and only if there is a zero row of the coefficient
  matrix with a nonzero value for that row in the augmenting
  column. In what sense does this rule work?

  \begin{enumerate}[(A)]
  \item The rule can be applied to the augmented matrix directly in
    both the {\em if} and the {\em only if} direction.
  \item The rule can be applied to the augmented matrix only in the
    {\em if} direction in general. In the {\em only if} direction, the
    rule can be applied to the augmented matrix {\em after} we have
    reduced the system to a situation where the coefficient matrix is
    in row-echelon form (note: it's not necessary to reach reduced
    row-echelon form).
  \item The rule can be applied to the augmented matrix only in the
    {\em only if} direction in general. In the {\em if} direction, the
    rule can be applied to the augmented matrix {\em after} we have
    reduced the system to a situation where the coefficient matrix is
    in row-echelon form (note: it's not necessary to reach reduced
    row-echelon form).
  \item The rule can be applied in either direction only {\em after}
    we have reduced the system to a situation where the coefficient
    matrix is in row-echelon form (note: it's not necessary to reach
    reduced row-echelon form).
  \end{enumerate}

  \vspace{0.1in}
  Your answer: $\underline{\qquad\qquad\qquad\qquad\qquad\qquad\qquad}$
  \vspace{0.1in}

\item {\em Do not discuss this!}: Which of the following is {\em not} a
  possibility for the number of solutions to a system of simultaneous
  linear equations? Please see Options (D) and (E) before answering.

  \begin{enumerate}[(A)]
  \item $0$
  \item $1$
  \item $2$
  \item All of the above, i.e., none of them is a possibility
  \item None of the above, i.e., they are all possibilities
  \end{enumerate}

  \vspace{0.1in}
  Your answer: $\underline{\qquad\qquad\qquad\qquad\qquad\qquad\qquad}$
  \vspace{0.1in}

\item {\em Do not discuss this!}: Which of the following describes the
  situation for a consistent system of simultaneous linear equations?

  \begin{enumerate}[(A)]
  \item The leading variables are the parameters used to describe the
    general solution, and the number of leading variables equals the
    number of nonzero equations in the reduced row-echelon form (here
    nonzero equation makes an equation that does not have a zero row
    in the augmented matrix).
  \item The non-leading variables are the parameters used to describe the
    general solution, and the number of non-leading variables equals the
    number of nonzero equations in the reduced row-echelon form (here
    nonzero equation makes an equation that does not have a zero row
    in the augmented matrix).
  \item The leading variables are the parameters used to describe the
    general solution, and the number of leading variables equals the
    value (number of variables) - (number of nonzero equations in the
    reduced row-echelon form) (here nonzero equation makes an equation
    that does not have a zero row in the augmented matrix).
  \item The non-leading variables are the parameters used to describe
    the general solution, and the number of non-leading variables
    equals the value (number of variables) - (number of nonzero equations
    in the reduced row-echelon form) (here nonzero equation makes an
    equation that does not have a zero row in the augmented matrix).
  \end{enumerate}

  \vspace{0.1in}
  Your answer: $\underline{\qquad\qquad\qquad\qquad\qquad\qquad\qquad}$
  \vspace{0.1in}

\end{enumerate}

\end{document}
