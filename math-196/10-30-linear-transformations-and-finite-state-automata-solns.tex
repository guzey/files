\documentclass[10pt]{amsart}

%Packages in use
\usepackage{fullpage, hyperref, vipul, enumerate}

%Title details
\title{Take-home class quiz solutions: due Wednesday October 30: Linear transformations and finite state automata}
\author{Math 196, Section 57 (Vipul Naik)}
%List of new commands

\begin{document}
\maketitle

\section{Performance review}

28 people took this 7-question quiz. The score distribution was as follows:

\begin{itemize}
\item Score of 0: 2 people
\item Score of 1: 1 person
\item Score of 2: 1 person
\item Score of 3: 5 people
\item Score of 4: 3 people
\item Score of 5: 3 people
\item Score of 6: 8 people
\item Score of 7: 5 people
\end{itemize}

The mean score was 4.57.

The question-wise answers and performance review are below:

\begin{enumerate}
\item Option (E): 19 people
\item Option (B): 23 people
\item Option (A): 21 people
\item Option (C): 10 people
\item Option (D): 17 people
\item Option (B): 18 people
\item Option (C): 20 people
\end{enumerate}
\section{Solutions}

{\bf PLEASE FEEL FREE TO DISCUSS {\em ALL} QUESTIONS.}

The purpose of this quiz is to explore in greater depth particular
types of matrices, the corresponding linear transformations, and the
relationship between operations on sets and similar operations on
vector spaces. The material covered in the quiz will also prove to be
a fertile source of {\em examples} and {\em counterexamples} for later
content: in the future, when you are asked to come up with matrices
that satisfy some very loosely stated conditions, the matrices of the
type described here can be a place to begin your search.

\vspace{0.2in}

Let $n$ be a natural number greater than $1$. Suppose $f: \{
0,1,2,\dots,n\} \to \{ 0,1,2,\dots,n\}$ is a function satisfying $f(0)
= 0$. Let $T_f$ denote the linear transformation from $\R^n$ to $\R^n$
satisfying the following for all $i \in \{ 1,2,\dots,n\}$:

$$T_f(\vec{e}_i) = \left \lbrace \begin{array}{rl} \vec{e}_{f(i)}, & f(i) \ne 0\\ \vec{0}, & f(i) = 0\\\end{array}\right.$$

Let $M_f$ denote the matrix for the linear transformation $T_f$. $M_f$
can be described explicitly as follows: the $i^{th}$ column of $M_f$
is $\vec{0}$ if $f(i) = 0$ and is $\vec{e}_{f(i)}$ if $f(i) \ne 0$.

Note that if $f,g: \{ 0,1,2,\dots,n \} \to \{ 0,1,2,\dots,n\}$ are
functions with $f(0)= g(0) = 0$, then $M_{f \circ g} = M_fM_g$ and
$T_{f \circ g} = T_f \circ T_g$.

We will also use the following terminology:

\begin{itemize}
\item A $n \times n$ matrix $A$ is termed {\em idempotent} if $A^2 = A$.
\item A $n \times n$ matrix $A$ is termed {\em nilpotent} if there
  exists a positive integer $r$ such that $A^r = 0$.
\item A $n \times n$ matrix $A$ is termed a {\em permutation matrix}
  if every row contains one $1$ and all other entries $0$, {\em and}
  every column contains one $1$ and all other entries $0$.
\end{itemize}

\begin{enumerate}
\item What condition on a function $f: \{ 0,1,2,\dots, n\} \to \{
  0,1,2,\dots,n\}$ (satisfying $f(0) = 0$) is equivalent to requiring
  $M_f$ to be idempotent?

  \begin{enumerate}[(A)]
  \item $(f(x))^2 = x$ for all $x \in \{ 0,1,2,\dots,n\}$
  \item $f(x^2) = x$ for all $x \in \{ 0,1,2,\dots,n \}$
  \item $(f(x))^2 = f(x)$ for all $x \in \{ 0,1,2,\dots,n\}$
  \item $f(f(x)) = x$ for all $x \in \{ 0,1,2,\dots,n\}$
  \item $f(f(x)) = f(x)$ for all $x \in \{0,1,2,\dots,n \}$
  \end{enumerate}

  {\em Answer}: Option (E)

  {\em Explanation}: As noted above, $M_f^2 = M_fM_f = M_{f \circ
    f}$. In order to have $M_f^2 = M_f$, we need $M_{f \circ f} =
  M_f$. Since the matrix determines the function, we get $f \circ f =
  f$, so $f(f(x)) = f(x)$ for all $x \in \{ 0,1,2,\dots,n \}$.

  {\em Performance review}: 19 out of 28 got this. 5 chose (C), 3
  chose (D), 1 chose (B).

  {\em Historical note (last time)}: $12$ out
  of $26$ got this. $9$ chose (C), $5$ chose (D).

\item What condition on a function $f: \{ 0,1,2,\dots, n\} \to \{
  0,1,2,\dots,n\}$ (satisfying $f(0) = 0$) is equivalent to requiring
  $M_f$ to be nilpotent?

  \begin{enumerate}[(A)]
  \item Composing $f$ enough times with itself gives the identity
    function (i.e., the function that sends everything to itself).
  \item Composing $f$ enough times with itself gives the function that
    sends everything to $0$.
  \item Composing $f$ enough times with itself gives the function that
    sends everything to $1$.
  \item Multiplying $f$ enough times with itself gives the identity
    function (i.e., the function that sends everything to itself).
  \item Multiplying $f$ enough times with itself gives the function
    that sends everything to $0$.
  \end{enumerate}

  {\em Answer}: Option (B)

  {\em Explanation}: The matrix $M_f^r$ is the same as $M_{f \circ
    \dots \circ f}$ where $f$ is composed with itself $r$ times. Thus,
  $M_f^r = 0$ if and only if $M_{f \circ \dots f} = 0$, which means
  that the $r$-fold composite of $f$ is the function that sends
  everything to zero.

  {\em Performance review}: 23 out of 28 got this. 3 chose (A), 2
  chose (E).

  {\em Historical note (last time)}: $23$ out of $26$ got this. $2$
  chose (C), $1$ chose (E).
\item What condition on a function $f: \{ 0,1,2,\dots, n\} \to \{
  0,1,2,\dots,n\}$ (satisfying $f(0) = 0$) is equivalent to requiring
  $M_f$ to be a permutation matrix?

  \begin{enumerate}[(A)]
  \item Composing $f$ enough times with itself gives the identity
    function (i.e., the function that sends everything to itself).
  \item Composing $f$ enough times with itself gives the function that
    sends everything to $0$.
  \item Composing $f$ enough times with itself gives the function that
    sends everything to $1$.
  \item Multiplying $f$ enough times with itself gives the identity
    function (i.e., the function that sends everything to itself).
  \item Multiplying $f$ enough times with itself gives the function
    that sends everything to $0$.
  \end{enumerate}

  {\em Answer}: Option (A)

  {\em Explanation}: $f$ itself is bijective, which means it cycles
  around the coordinates. Repeating $f$ enough times should bring
  everything back where it was.

  {\em Performance review}: 21 out of 28 got this. 7 chose (D).

  {\em Historical note (last time)}: $23$ out of $26$ got this. $2$ chose (C),
  $1$ chose (B).
\item Consider a function $f: \{ 0,1,2,\dots,n \} \to
  \{0,1,2,\dots,n\}$ with the property that $f(0) = 0$ and, for each
  $i \in \{ 1,2,\dots,n\}$, $f(i)$ is either $i$ or $0$. Note that the
  behavior may be different for different values of $i$ (so some of
  them may go to themselves, and others may go to $0$). What can we say
  $M_f$ must be?

  \begin{enumerate}[(A)]
  \item $M_f$ must be the identity matrix.
  \item $M_f$ must be the zero matrix.
  \item $M_f$ must be an idempotent matrix.
  \item $M_f$ must be a nilpotent matrix.
  \item $M_f$ must be a permutation matrix.
  \end{enumerate}

  {\em Answer}: Option (C)

  {\em Explanation}: The function $f$ given here satisfies the
  condition that $f(f(i)) = f(i)$ for all $i$. To see this, note that:

  \begin{itemize}
  \item if $f(i) = i$, then $f(f(i)) = f(i) = i$, i.e., we stay looped at $i$.
  \item if $f(i) = 0$, then $f(f(i)) = f(0) = 0$, i.e., once we reach
    $0$, we stay there.
  \end{itemize}

  {\em Performance review}: 10 out of 28 got this. 9 chose (D), 7
  chose (E), 2 chose (A).

  {\em Historical note (last time)}: $12$ out of $26$ got this. $6$ chose (D),
  $5$ chose (E), $3$ chose (A).
\item Which of the following pairs of candidates for $f,g: \{ 0,1,2 \}
  \to \{ 0,1,2 \}$ satisfies the condition that $M_fM_g = 0$ but
  $M_gM_f \ne 0$? 

  \begin{enumerate}[(A)]
  \item $f(0) = 0$, $f(1) = 1$, $f(2) = 2$, whereas $g(0) = 0$, $g(1)
    = 2$, $g(2) = 1$
  \item $f(0) = 0$, $f(1) = 0$, $f(2) = 1$, whereas $g(0) = 0$, $g(1)
    = 2$, $g(2) = 0$
  \item $f(0) = 0$, $f(1) = 1$, $f(2) = 0$, whereas $g(0) = 0$, $g(1)
    = 0$, $g(2) = 2$
  \item $f(0) = 0$, $f(1) = 0$, $f(2) = 1$, whereas $g(0) = 0$, $g(1)
    = 1$, $g(2) = 0$
  \item $f(0) = 0$, $f(1) = 1$, $f(2) = 0$, whereas $g(0) = 0$, $g(1)
    = 0$, $g(2) = 1$
  \end{enumerate}

  {\em Answer}: Option (D)

  {\em Explanation}: For this option, $f \circ g$ sends everything to
  $0$. $g \circ f$ sends $0$ and $1$ to $0$ and sends $2$ to $1$.

  {\em Performance review}: 17 out of 28 got this. 5 chose (E), 2 each
  chose (A), (B), and (C).

  {\em Historical note (last time)}: $20$ out of $26$ got this. $2$
  each chose (A), (B), and (E).
\item Which of the following pairs of candidates for $f,g: \{ 0,1,2 \}
  \to \{ 0,1,2 \}$ satisfies the condition that $M_f$ and $M_g$ are
  both nilpotent but $M_fM_g$ is not nilpotent?

  \begin{enumerate}[(A)]
  \item $f(0) = 0$, $f(1) = 1$, $f(2) = 2$, whereas $g(0) = 0$, $g(1)
    = 2$, $g(2) = 1$
  \item $f(0) = 0$, $f(1) = 0$, $f(2) = 1$, whereas $g(0) = 0$, $g(1)
    = 2$, $g(2) = 0$
  \item $f(0) = 0$, $f(1) = 1$, $f(2) = 0$, whereas $g(0) = 0$, $g(1)
    = 0$, $g(2) = 2$
  \item $f(0) = 0$, $f(1) = 0$, $f(2) = 1$, whereas $g(0) = 0$, $g(1)
    = 1$, $g(2) = 0$
  \item $f(0) = 0$, $f(1) = 1$, $f(2) = 0$, whereas $g(0) = 0$, $g(1)
    = 0$, $g(2) = 1$
  \end{enumerate}

  {\em Answer}: Option (B)

  {\em Explanation}: $f \circ f$ is the function sending everything to
  $0$. $g \circ g$ is also the function sending everything to $0$. On
  the other hand, $f \circ g$ is the function that sends $1$ to $1$
  and sends $0$ and $2$ to $0$.

  {\em Performance review}: 18 out of 28 got this. 3 each chose (A)
  and (E), 2 chose (D), 1 chose (C), and 1 left the question blank.

  {\em Historical note (last time)}: $22$ out of $26$ got this. $3$
  chose (A), $1$ chose (C).
\item Which of the following pairs of candidates for $f,g: \{ 0,1,2 \}
  \to \{ 0,1,2 \}$ satisfies the condition that neither $M_f$ and
  $M_g$ is nilpotent but $M_fM_g$ is nilpotent?

  \begin{enumerate}[(A)]
  \item $f(0) = 0$, $f(1) = 1$, $f(2) = 2$, whereas $g(0) = 0$, $g(1)
    = 2$, $g(2) = 1$
  \item $f(0) = 0$, $f(1) = 0$, $f(2) = 1$, whereas $g(0) = 0$, $g(1)
    = 2$, $g(2) = 0$
  \item $f(0) = 0$, $f(1) = 1$, $f(2) = 0$, whereas $g(0) = 0$, $g(1)
    = 0$, $g(2) = 2$
  \item $f(0) = 0$, $f(1) = 0$, $f(2) = 1$, whereas $g(0) = 0$, $g(1)
    = 1$, $g(2) = 0$
  \item $f(0) = 0$, $f(1) = 1$, $f(2) = 0$, whereas $g(0) = 0$, $g(1)
    = 0$, $g(2) = 1$
  \end{enumerate}

  {\em Answer}: Option (C)

  {\em Explanation}: $M_f$ and $M_g$ are both idempotent, since $f
  \circ f = f$ and $g \circ g = g$. However, $f \circ g$ is the
  function sending everything to $0$, so $M_fM_g = 0$.

  {\em Performance review}: 20 out of 28 got this. 3 chose (A), 2
  chose (B), 1 each chose (D) and (E), 1 left the question blank.

  {\em Historical note (last time)}: $20$ out of $26$ got this. $3$
  chose (B), $1$ each chose (A), (D), and (E).

\end{enumerate}
\end{document}
