\documentclass[10pt]{amsart}

%Packages in use
\usepackage{fullpage, hyperref, vipul, enumerate}

%Title details
\title{Take-home class quiz: due Monday November 25: Stochastic matrices}
\author{Math 196, Section 57 (Vipul Naik)}
%List of new commands

\begin{document}
\maketitle

Your name (print clearly in capital letters): $\underline{\qquad\qquad\qquad\qquad\qquad\qquad\qquad\qquad\qquad\qquad}$

{\bf PLEASE FEEL FREE TO DISCUSS {\em ALL} QUESTIONS.}

This quiz can be viewed as a continuation of the quiz on linear
dynamical systems. The book defines column-stochastic matrices using
the jargon ``transition matrix'' on Page 53 (Definition 2.1.4) and
uses them throughout the text when describing (a simplified version
of) Google's PageRank algorithm. The quiz questions are self-contained
and do not require you to read the book, but you may benefit from
skimming through the book's discussion of PageRank to complement these
questions. Note that the 4th Edition does not include the discussion
of transition matrices and PageRank.

In this quiz, we discuss the dynamics of a very special type of linear
transformation. A $n \times n$ matrix $A$ is termed a {\em
  row-stochastic matrix} if all its entries are in the interval
$[0,1]$ and all the row sums are equal to $1$. A $n \times n$ matrix
is termed a {\em column-stochastic matrix} if all its entries are in
the interval $[0,1]$ and all the column sums are equal to $1$. A $n
\times n$ matrix $A$ is termed a {\em doubly stochastic matrix} if it
is both row-stochastic and column-stochastic, i.e., all the entries
are in the interval $[0,1]$, all the row sums are equal to $1$, and
all the column sums are equal to $1$.

\begin{enumerate}

\item Suppose $A$ and $B$ are two $n \times n$ row-stochastic
  matrices. Which of the following is {\em guaranteed} to be
  row-stochastic? Please see Options (D) and (E) before answering.

  \begin{enumerate}[(A)]
  \item $A + B$
  \item $A - B$
  \item $AB$
  \item All of the above
  \item None of the above
  \end{enumerate}

  \vspace{0.1in}
  Your answer: $\underline{\qquad\qquad\qquad\qquad\qquad\qquad\qquad}$
  \vspace{0.1in}

\item Suppose $A$ and $B$ are two $n \times n$ column-stochastic
  matrices. Which of the following is {\em guaranteed} to be
  column-stochastic? Please see Options (D) and (E) before answering.

  \begin{enumerate}[(A)]
  \item $A + B$
  \item $A - B$
  \item $AB$
  \item All of the above
  \item None of the above
  \end{enumerate}

  \vspace{0.1in}
  Your answer: $\underline{\qquad\qquad\qquad\qquad\qquad\qquad\qquad}$
  \vspace{0.1in}
\item Suppose $A$ and $B$ are two $n \times n$ doubly stochastic
  matrices. Which of the following is {\em guaranteed} to be
  doubly stochastic? Please see Options (D) and (E) before answering.

  \begin{enumerate}[(A)]
  \item $A + B$
  \item $A - B$
  \item $AB$
  \item All of the above
  \item None of the above
  \end{enumerate}

  \vspace{0.1in}
  Your answer: $\underline{\qquad\qquad\qquad\qquad\qquad\qquad\qquad}$
  \vspace{0.6in}

  We now consider the case $n = 2$. In this case, the doubly
  stochastic matrices have the form:

  $$\left[ \begin{matrix} a & 1 - a \\ 1 - a & a \\\end{matrix}\right]$$

  where $a \in [0,1]$. Denote this matrix by $D(a)$ for short.

\item Suppose $a,b \in [0,1]$ (they are allowed to be equal). The
  product $D(a)D(b)$ equals $D(c)$ for some $c \in [0,1]$. What is
  that value of $c$?

  \begin{enumerate}[(A)]
  \item $a + b$
  \item $ab$
  \item $2ab + a + b$
  \item $(1 - a)(1 - b)$
  \item $1 - a - b + 2ab$
  \end{enumerate}

  \vspace{0.1in}
  Your answer: $\underline{\qquad\qquad\qquad\qquad\qquad\qquad\qquad}$
  \vspace{0.1in}

\item For what value(s) of $a$ is the matrix $D(a)$ non-invertible?
  Note that when judging invertibility, we do not insist that the
  inverse matrix also be doubly stochastic.

  \begin{enumerate}[(A)]
  \item $a = 0$ only
  \item $a = 1/2$ only
  \item $a = 1$ only
  \item $0 < a < 1$ (i.e., $D(a)$ is invertible only at $a = 0$ and $a
    = 1$)
  \item $a \ne 1/2$
  \end{enumerate}

  \vspace{0.1in}
  Your answer: $\underline{\qquad\qquad\qquad\qquad\qquad\qquad\qquad}$
  \vspace{0.1in}

\item For what value(s) of $a$ is it true that the matrix $D(a)$ does
  not have an inverse that is a doubly stochastic matrix? In other
  words, either $D(a)$ should be non-invertible or it should be
  invertible but the inverse is not a doubly stochastic matrix.

  \begin{enumerate}[(A)]
  \item $a = 0$ only
  \item $a = 1/2$ only
  \item $a = 1$ only
  \item $0 < a < 1$ (i.e., $D(a)$ has an inverse that is also doubly
    stochastic only if $a = 0$ or $a = 1$)
  \item $a \ne 1/2$
  \end{enumerate}

  \vspace{0.1in}
  Your answer: $\underline{\qquad\qquad\qquad\qquad\qquad\qquad\qquad}$
  \vspace{0.1in}

  For the next few questions, denote by $T_a$ the linear
  transformation whose matrix is $D(a)$. For any vector $\vec{x} \in
  \R^2$, we can consider the sequence:

  $$\vec{x}, T_a(\vec{x}), T_a^2(\vec{x}), \dots$$

  Note that if we were to start with a vector $\vec{x} \in \R^2$ with
  both coordinates equal, it would be invariant under $T_a$.

  Thus, for the questions below, assume that we start with a nonzero
  vector $\vec{x} \in \R^2$ for which the two coordinates are not
  equal to each other.

\item For what value of $a$ is it the case that $\lim_{r \to \infty}
  T_a^r(\vec{x})$ does {\em not} exist?

  \begin{enumerate}[(A)]
  \item $a = 0$ only
  \item $a = 1/2$ only
  \item $a = 1$ only
  \item $0 < a < 1$
  \item $a \ne 1/2$
  \end{enumerate}

  \vspace{0.1in}
  Your answer: $\underline{\qquad\qquad\qquad\qquad\qquad\qquad\qquad}$
  \vspace{0.1in}

\item For what value of $a$ is it the case that the sequence

  $$\vec{x}, T_a(\vec{x}), T_a^2(\vec{x}), \dots$$

  is a constant sequence?

  \begin{enumerate}[(A)]
  \item $a = 0$ only
  \item $a = 1/2$ only
  \item $a = 1$ only
  \item $0 < a < 1$
  \item $a \ne 1/2$
  \end{enumerate}

  \vspace{0.1in}
  Your answer: $\underline{\qquad\qquad\qquad\qquad\qquad\qquad\qquad}$
  \vspace{0.1in}

\item For what value of $a$ is it the case that the sequence

  $$\vec{x}, T_a(\vec{x}), T_a^2(\vec{x}), \dots$$

  is not a constant sequence but becomes constant from $T_a(\vec{x})$
  onward?

  \begin{enumerate}[(A)]
  \item $a = 0$ only
  \item $a = 1/2$ only
  \item $a = 1$ only
  \item $0 < a < 1$ 
  \item $a \ne 1/2$
  \end{enumerate}

  \vspace{0.1in}
  Your answer: $\underline{\qquad\qquad\qquad\qquad\qquad\qquad\qquad}$
  \vspace{0.1in}

\item For $a$ other than $0$, $1/2$, or $1$, what is the limit
  $\lim_{r \to \infty} (D(a))^r$? Here, when we talk of taking the
  limit of a sequence of matrices, we are taking the limit entry-wise.

  \begin{enumerate}[(A)]
  \item The matrix $D(0)$
  \item The matrix $D(1/2)$
  \item The matrix $D(1)$
  \item The matrix $D(a)$
  \item The matrix $D(1 - a)$
  \end{enumerate}

  \vspace{0.1in}
  Your answer: $\underline{\qquad\qquad\qquad\qquad\qquad\qquad\qquad}$
  \vspace{0.1in}
\end{enumerate}
\end{document}
