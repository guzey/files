\documentclass[10pt]{amsart}

%Packages in use
\usepackage{fullpage, hyperref, vipul, enumerate}

%Title details
\title{Take-home class quiz: due Friday November 22: Linear dynamical systems}
\author{Math 196, Section 57 (Vipul Naik)}
%List of new commands

\begin{document}
\maketitle

Your name (print clearly in capital letters): $\underline{\qquad\qquad\qquad\qquad\qquad\qquad\qquad\qquad\qquad\qquad}$

{\bf PLEASE FEEL FREE TO DISCUSS {\em ALL} QUESTIONS.}

This quiz covers a topic that we will not be able to get to formally
in the course due to time constraints. The corresponding section of
the book is Section 7.1, and there is more relevant material discussed
in the later sections of Chapter 7. However, you do not need to read
those sections in order to attempt this quiz. Also, simply mastering
the computational techniques in those sections of the book will not
help you much with the quiz questions.

The questions here consider a linear dynamical system. Consider a
linear transformation $T:\R^n \to \R^n$. Let $A$ be the matrix of $T$,
so that $A$ is a $n \times n$ matrix. For any positive integer $r$,
the matrix $A^r$ is the matrix for the linear transformation $T^r$
(note here that $T^r$ refers to the $r$-fold {\em composite} of
$T$). The goal is to determine, starting off with an arbitrary vector
$\vec{x} \in \R^n$, how the following sequence behaves:

$$\vec{x}, T(\vec{x}), T^2(\vec{x}), T^3(\vec{x}), \dots$$

More explicitly, each term of the sequence is obtained by applying $T$
to the preceding term. In other words, the sequence is:

$$\vec{x}, T(\vec{x}), T(T(\vec{x})), T(T(T(\vec{x}))), \dots$$

\begin{enumerate}
\item What is the necessary and sufficient condition on $A$ such that
  for {\em every} choice of $\vec{x} \in \R^n$, the sequence described
  above eventually reaches, and stays at, the zero vector? Note that
  if it reaches the zero vector, it must do so in at most $n$
  steps. Please see Option (E) before answering.

  \begin{enumerate}[(A)]
  \item $A$ is a nilpotent matrix.
  \item $A$ is an idempotent matrix.
  \item $A$ is an invertible matrix.
  \item $A$ is a non-invertible matrix.
  \item None of the above.
  \end{enumerate}

  \vspace{0.1in}
  Your answer: $\underline{\qquad\qquad\qquad\qquad\qquad\qquad\qquad}$
  \vspace{0.1in}

\item What is the necessary and sufficient condition on $A$ such that
  there {\em exists} a nonzero vector $\vec{x} \in \R^n$ for which the
  sequence described above eventually reaches, and stays at, the zero
  vector?  Note that if it reaches the zero vector, it must do so in
  at most $n$ steps. Please see Option (E) before answering.

  \begin{enumerate}[(A)]
  \item $A$ is a nilpotent matrix.
  \item $A$ is an idempotent matrix.
  \item $A$ is an invertible matrix.
  \item $A$ is a non-invertible matrix.
  \item None of the above.
  \end{enumerate}

  \vspace{0.1in}
  Your answer: $\underline{\qquad\qquad\qquad\qquad\qquad\qquad\qquad}$
  \vspace{0.1in}

\item What is the necessary and sufficient condition on $A$ such that
  for {\em every} choice of $\vec{x} \in \R^n$, the sequence described
  above returns to $\vec{x}$ after a finite and positive number of
  steps? Please see Option (E) before answering.

  \begin{enumerate}[(A)]
  \item $A$ is a nilpotent matrix.
  \item $A$ is an idempotent matrix.
  \item $A$ is an invertible matrix.
  \item $A$ is a non-invertible matrix.
  \item None of the above.
  \end{enumerate}

  \vspace{0.1in}
  Your answer: $\underline{\qquad\qquad\qquad\qquad\qquad\qquad\qquad}$
  \vspace{0.1in}

\item What is the necessary and sufficient condition on $A$ such that
  there {\em exists} a nonzero vector $\vec{x} \in \R^n$ for which the
  sequence described above returns to $\vec{x}$ after a finite and
  positive number of steps? Please see Option (E) before answering.

  \begin{enumerate}[(A)]
  \item $A$ is a nilpotent matrix.
  \item $A$ is an idempotent matrix.
  \item $A$ is an invertible matrix.
  \item $A$ is a non-invertible matrix.
  \item None of the above.
  \end{enumerate}

  \vspace{0.1in}
  Your answer: $\underline{\qquad\qquad\qquad\qquad\qquad\qquad\qquad}$
  \vspace{0.1in}

\item Suppose $n = 2$ and $T$ is a rotation by an angle that is a
  rational multiple of $\pi$. What can we say about the range of
  the sequence

  $$\vec{x}, T(\vec{x}), T^2(\vec{x}), T^3(\vec{x}), \dots$$

  starting from a nonzero vector $\vec{x}$?

  \begin{enumerate}[(A)]
  \item The range is finite, i.e., there are only finitely many
    distinct vectors in the sequence.
  \item The range is infinite and forms a dense subset of the circle
    centered at the origin and with radius equal to the length of the
    vector $\vec{x}$. However, it is not the entire circle.
  \item The range is infinite and is the entire circle centered at the
    origin and with radius equal to the length of the vector
    $\vec{x}$.
  \item The range is infinite and forms a dense subset of the line of
    the vector $\vec{x}$ (excluding the origin), but is not the entire
    line (excluding the origin).
  \item The range is infinite and is the entire line of the vector
    $\vec{x}$, excluding the origin.
  \end{enumerate}

  \vspace{0.1in}
  Your answer: $\underline{\qquad\qquad\qquad\qquad\qquad\qquad\qquad}$
  \vspace{0.1in}

\item Suppose $n = 2$ and $T$ is a rotation by an angle that is a
  irrational multiple of $\pi$. What can we say about the range of
  the sequence

  $$\vec{x}, T(\vec{x}), T^2(\vec{x}), T^3(\vec{x}), \dots$$

  starting from a nonzero vector $\vec{x}$?

  \begin{enumerate}[(A)]
  \item The range is finite, i.e., there are only finitely many
    distinct vectors in the sequence.
  \item The range is infinite and forms a dense subset of the circle
    centered at the origin and with radius equal to the length of the
    vector $\vec{x}$. However, it is not the entire circle.
  \item The range is infinite and is the entire circle centered at the
    origin and with radius equal to the length of the vector
    $\vec{x}$.
  \item The range is infinite and forms a dense subset of the line of
    the vector $\vec{x}$ (excluding the origin), but is not the entire
    line (excluding the origin).
  \item The range is infinite and is the entire line of the vector
    $\vec{x}$, excluding the origin.
  \end{enumerate}

  \vspace{0.1in}
  Your answer: $\underline{\qquad\qquad\qquad\qquad\qquad\qquad\qquad}$
  \vspace{0.6in}

  We return to generic $n$ now.

\item A nonzero vector $\vec{x}$ is termed an {\em eigenvector} for a
  linear transformation $T: \R^n \to \R^n$ with {\em eigenvalue} a
  real number $\lambda \in \R$ if $T(\vec{x}) = \lambda\vec{x}$. Note
  that $\lambda$ is allowed to be $0$. We sometimes conflate the roles
  of $T$ and its matrix $A$, so that we call $\vec{x}$ an eigenvector
  for $A$ and $\lambda$ an eigenvalue for $A$.

  If $\vec{x}$ is an eigenvector of $T$ (or equivalently, of $A$) with
  eigenvalue $\lambda$, which of the following is true? We denote by
  $I_n$ the identity transformation from $\R^n$ to itself.

  \begin{enumerate}[(A)]
  \item $\vec{x}$ must be in the kernel of the linear transformation
    $T + \lambda I_n$
  \item $\vec{x}$ must be in the image of the linear transformation $T
    + \lambda I_n$
  \item $\vec{x}$ must be in the kernel of the linear transformation
    $T - \lambda I_n$
  \item $\vec{x}$ must be in the image of the linear transformation $T
    - \lambda I_n$
  \item $\vec{x}$ must be in the kernel of the linear transformation
    $\lambda T$
  \end{enumerate}

  \vspace{0.1in}
  Your answer: $\underline{\qquad\qquad\qquad\qquad\qquad\qquad\qquad}$
  \vspace{0.1in}

\item As above, let $T: \R^n \to \R^n$ be a linear transformation with
  matrix $A$. Use the terminology of eigenvector and eigenvalue from
  the preceding question. Which of the following is a characterization
  of the situation that $A$ is a diagonal matrix?

  \begin{enumerate}[(A)]
  \item Every nonzero vector in $\R^n$ is an eigenvector for $T$.
  \item Every standard basis vector in $\R^n$ is an eigenvector for $T$.
  \item Every vector with at least one zero coordinate in $\R^n$ is an
    eigenvector for $T$.
  \item $T$ has a unique eigenvector (up to scalar multiples, i.e.,
    all eigenvectors of $T$ are scalar multiples of each other).
  \item $T$ has no eigenvector.
  \end{enumerate}

  \vspace{0.1in}
  Your answer: $\underline{\qquad\qquad\qquad\qquad\qquad\qquad\qquad}$
  \vspace{0.1in}

\item As above, let $T: \R^n \to \R^n$ be a linear transformation with
  matrix $A$. Use the terminology of eigenvector and eigenvalue from
  the preceding question. Which of the following is a characterization
  of the situation that $A$ is a scalar matrix (i.e., a diagonal matrix
  with all diagonal entries equal)?

  \begin{enumerate}[(A)]
  \item Every nonzero vector in $\R^n$ is an eigenvector for $T$.
  \item Every standard basis vector in $\R^n$ is an eigenvector for $T$.
  \item Every vector with at least one zero coordinate in $\R^n$ is an
    eigenvector for $T$.
  \item $T$ has a unique eigenvector (up to scalar multiples, i.e.,
    all eigenvectors of $T$ are scalar multiples of each other).
  \item $T$ has no eigenvector.
  \end{enumerate}

  \vspace{0.1in}
  Your answer: $\underline{\qquad\qquad\qquad\qquad\qquad\qquad\qquad}$
  \vspace{0.1in}

\item Suppose $A$ is a strictly upper-triangular $n \times n$ matrix,
  i.e., all entries of $A$ that are on or below the main diagonal are
  zero. $T$ is the linear transformation corresponding to $A$. It will
  turn out that the only eigenvalue for $T$ is $0$. What can we say
  about the eigenvectors for $T$ for this eigenvalue?

  \begin{enumerate}[(A)]
  \item All nonzero vectors in $\R^n$ are eigenvectors for $T$ with
    eigenvalue $0$.
  \item All standard basis vectors in $\R^n$ are eigenvectors for $T$
    with eigenvalue $0$.
  \item The vector $\vec{e}_1$ is an eigenvector for $T$ with
    eigenvalue $0$. The information presented is not sufficient to
    determine whether any of the other standard basis vectors is an
    eigenvector.
  \item The vector $\vec{e}_n$ is an eigenvector for $T$ with
    eigenvalue $0$. The information presented is not sufficient to
    determine whether any of the other standard basis vectors is an
    eigenvector.
  \item At least one of the standard basis vectors is an eigenvector
    for $T$ with eigenvalue $0$. However, the information presented is
    not sufficient to say definitively for any particular standard
    basis vector that it is an eigenvector.
  \end{enumerate}

  \vspace{0.1in}
  Your answer: $\underline{\qquad\qquad\qquad\qquad\qquad\qquad\qquad}$
  \vspace{0.1in}

\item Suppose $A$ is a strictly upper-triangular $n \times n$ matrix,
  i.e., all entries of $A$ that are on or below the main diagonal are
  zero. $T$ is the linear transformation corresponding to $A$. Which
  of the following is $A$ guaranteed to be? Please see Options (D) and
  (E) before answering.

  \begin{enumerate}[(A)]
  \item Nilpotent
  \item Idempotent
  \item Invertible
  \item All of the above
  \item None of the above
  \end{enumerate}

  \vspace{0.1in}
  Your answer: $\underline{\qquad\qquad\qquad\qquad\qquad\qquad\qquad}$
  \vspace{0.1in}

\item Consider the case $n = 2$ and let $T:\R^2 \to \R^2$ be a
  rotation by an angle that is {\em not} an {\em integer} multiple of
  $\pi$. What can we say about the set of eigenvectors and eigenvalues
  for $T$?

  \begin{enumerate}[(A)]
  \item $T$ has no eigenvectors
  \item $T$ has one eigenvector (up to scalar multiples) with
    eigenvalue $1$
  \item $T$ has one eigenvector (up to scalar multiples) and the
    eigenvalue depends on the angle of rotation
  \item $T$ has two linearly independent eigenvectors (so that the set
    of all eigenvectors is obtained as the set of scalar multiples of
    either one of these vectors) with the same eigenvalue
  \item $T$ has two linearly independent eigenvectors (so that the set
    of all eigenvectors is obtained as the set of scalar multiples of
    either one of these vectors) with distinct eigenvalues
  \end{enumerate}

  \vspace{0.1in}
  Your answer: $\underline{\qquad\qquad\qquad\qquad\qquad\qquad\qquad}$
  \vspace{0.1in}
\end{enumerate}
\end{document}
