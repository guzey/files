\documentclass[10pt]{amsart}

%Packages in use
\usepackage{fullpage, hyperref, vipul, enumerate}

%Title details
\title{Take-home class quiz: due Friday October 11: Linear systems}
\author{Math 196, Section 57 (Vipul Naik)}
%List of new commands

\begin{document}
\maketitle

Your name (print clearly in capital letters): $\underline{\qquad\qquad\qquad\qquad\qquad\qquad\qquad\qquad\qquad\qquad}$

{\bf YOU MAY DISCUSS ALL QUESTIONS, BUT PLEASE ENTER FINAL ANSWER
  OPTIONS THAT YOU ARE PERSONALLY MOST CONVINCED OF. BEWARE OF GROUPTHINK!}

The quiz questions here, although not hard {\em per se}, are
conceptually demanding: answering them requires a clear
understanding of multiple concepts and an ability to execute them
conjunctively. Even if you feel that you've understood the material as
presented in class, you will need to think through each question
carefully. Some of the questions are related to similar homework
problems (Homeworks 1 and 2), and they test a conceptual understanding
of the solutions to these problems. You might want to view them in
conjunction with the homework problems. Other questions sow the seeds
of ideas we will explore later. The quiz should seem relatively easier
when you review it later, assuming that you work hard on attempting
the questions right now and read the solutions once they're put up.

\begin{enumerate}
\item (*) Rashid and Riena are trying to study a function $f$ of two
  variables $x$ and $y$. Rashid is convinced that the function is
  linear (i.e., it is of the form $f(x,y) := ax + by + c$) but has no
  idea what $a$, $b$, and $c$ are. Riena thinks a linear model is
  completely out-of-place. Rashid is eager to find $a$, $b$, and $c$,
  whereas Riena is eager to disprove Rashid's linear
  model. Unfortunately, all they have is a black box that will output
  the value of the function for a given input pair $(x,y)$, and that
  black box can only be called three times. What should Rashid and
  Riena try for?

  \begin{enumerate}[(A)]
  \item Rashid and Riena would both like to provide three input pairs that
    are non-collinear as points in the $xy$-plane
  \item Rashid would like to provide three input pairs that are
    non-collinear, while Riena would like to provide three input pairs
    that are collinear as points in the $xy$-plane.
  \item Rashid and Riena would both like to provide three input pairs
    that are collinear as points in the $xy$-plane.
  \item Rashid would like to provide three input pairs that are
    collinear, while Riena would like to provide three input pairs
    that are non-collinear as points in the $xy$-plane.
  \item Both Rashid and Riena are indifferent regarding how the three
    input pairs are picked.
  \end{enumerate}

  \vspace{0.1in}
  Your answer: $\underline{\qquad\qquad\qquad\qquad\qquad\qquad\qquad}$
  \vspace{0.1in}

\item (*) Let $m$ and $n$ be natural numbers with $m \ge 3$. We are
  given a bunch of numbers $x_1< x_2< \dots<x_m$ and another bunch of
  numbers $y_1,y_2,\dots,y_m$. We want to find a continuous function
  $f$ on $[x_1,x_m]$, such that $f(x_i) = y_i$ for all $1 \le i \le
  m$, and such that the restriction of $f$ to any interval of the form
  $[x_i,x_{i+1}]$ (for $1 \le i \le m - 1$) is a polynomial of degree
  $\le n$. What is the smallest value of $n$ for which we are
  guaranteed to be able to find such a function $f$?

  \begin{enumerate}[(A)]
  \item $1$
  \item $2$
  \item $3$
  \item $4$
  \item $5$
  \end{enumerate}

  \vspace{0.1in}
  Your answer: $\underline{\qquad\qquad\qquad\qquad\qquad\qquad\qquad}$
  \vspace{0.1in}

\item (*) Let $m$ and $n$ be natural numbers with $m \ge 3$. We are
  given a bunch of numbers $x_1< x_2< \dots<x_m$ and another bunch of
  numbers $y_1,y_2,\dots,y_m$. We want to find a continuous function
  $f$ on $[x_1,x_m]$, such that $f(x_i) = y_i$ for all $1 \le i \le
  m$, and such that the restriction of $f$ to any interval of the form
  $[x_i,x_{i+1}]$ (for $1 \le i \le m - 1$) is a polynomial of degree
  $\le n$. In addition, we want to make sure that $f$ is
  differentiable on the open interval $(x_1,x_m)$. What is the
  smallest value of $n$ for which we are guaranteed to be able to find
  such a function $f$?

  \begin{enumerate}[(A)]
  \item $1$
  \item $2$
  \item $3$
  \item $4$
  \item $5$
  \end{enumerate}

  \vspace{0.1in}
  Your answer: $\underline{\qquad\qquad\qquad\qquad\qquad\qquad\qquad}$
  \vspace{0.1in}

\item (*) Let $m$ and $n$ be natural numbers with $m \ge 3$. We are given
  a bunch of numbers $x_1< x_2< \dots<x_m$ and another bunch of
  numbers $y_1,y_2,\dots,y_m$. We want to find a continuous function
  $f$ on $[x_1,x_m]$, such that $f(x_i) = y_i$ for all $1 \le i \le
  m$, and such that the restriction of $f$ to any interval of the form
  $[x_i,x_{i+1}]$ (for $1 \le i \le m - 1$) is a polynomial of degree
  $\le n$. In addition, we want to make sure that $f$ is
  differentiable on the open interval $(x_1,x_m)$. In addition, we are
  told the value of the right hand derivative of $f$ at $x_1$ and the
  left hand derivative of $f$ at $x_m$. What is the smallest value of
  $n$ for which we are guaranteed to be able to find such a function
  $f$?

  \begin{enumerate}[(A)]
  \item $1$
  \item $2$
  \item $3$
  \item $4$
  \item $5$
  \end{enumerate}

  \vspace{0.1in}
  Your answer: $\underline{\qquad\qquad\qquad\qquad\qquad\qquad\qquad}$
  \vspace{0.1in}

\item (*) Let $k$, $m$, and $n$ be natural numbers with $m \ge 3$. We are
  given a bunch of numbers $x_1< x_2< \dots<x_m$ and another bunch of
  numbers $y_1,y_2,\dots,y_m$. We want to find a continuous function
  $f$ on $[x_1,x_m]$, such that $f(x_i) = y_i$ for all $1 \le i \le
  m$, and such that the restriction of $f$ to any interval of the form
  $[x_i,x_{i+1}]$ (for $1 \le i \le m - 1$) is a polynomial of degree
  $\le n$. In addition, we want to make sure that $f$ is at least $k$
  times differentiable on the open interval $(x_1,x_m)$. What is the
  smallest value of $n$ for which we are guaranteed to be able to find
  such a function $f$?

  \begin{enumerate}[(A)]
  \item $k - 2$
  \item $k - 1$
  \item $k$
  \item $k + 1$
  \item $k + 2$
  \end{enumerate}

  \vspace{0.1in}
  Your answer: $\underline{\qquad\qquad\qquad\qquad\qquad\qquad\qquad}$
  \vspace{0.1in}

  The next few questions are framed deterministically, though similar
  real-world applications would be probabilistic, with some square
  roots floating around. Unfortunately, we do not have the tools yet
  to deal with the probabilistic versions of the questions.

\item (*) A function $f$ of one variable is known to be linear. We know
  that $f(1) = 1.5 \pm 0.5$ and $f(2) = 2.5 \pm 0.5$. Assume these are
  the full ranges, without any probability distribution known. Assuming
  nothing is known about how the measurement errors for $f$ at
  different points are related, what can we say about $f(3)$?

  \begin{enumerate}[(A)]
  \item $f(3) = 3.5$ (exactly)
  \item $f(3) = 3.5 \pm 0.5$
  \item $f(3) = 3.5 \pm 1$
  \item $f(3) = 3.5 \pm 1.5$
  \item $f(3) = 3.5 \pm 2.5$
  \end{enumerate}

  \vspace{0.1in}
  Your answer: $\underline{\qquad\qquad\qquad\qquad\qquad\qquad\qquad}$
  \vspace{0.1in}

\item (*) A function $f$ of one variable is known to be linear. We know
  that $f(1) = 1.5 \pm 0.5$ and $f(2) = 2.5 \pm 0.5$. Assume these are
  the full ranges, without any probability distribution known. Assume
  also that the measurement error for $f$ at all points is the same in
  magnitude and sign. What can we say about $f(3)$?

  \begin{enumerate}[(A)]
  \item $f(3) = 3.5$ (exactly)
  \item $f(3) = 3.5 \pm 0.5$
  \item $f(3) = 3.5 \pm 1$
  \item $f(3) = 3.5 \pm 1.5$
  \item $f(3) = 3.5 \pm 2.5$
  \end{enumerate}

  \vspace{0.1in}
  Your answer: $\underline{\qquad\qquad\qquad\qquad\qquad\qquad\qquad}$
  \vspace{0.1in}

\item (*) Suppose $f$ is a linear function on a bounded interval $[a,b]$
  but our measurement of outputs for given inputs has some measurement
  error (with the range of measurement error the same regardless of
  the input, and no known correlation between the magnitude of
  measurement error at different points). Assume we can get the
  outputs for any two specified inputs we desire, and we will then fit
  a line through the (input,output) pairs to get the graph of $f$. How
  should we choose our inputs?

  \begin{enumerate}[(A)]
  \item Choose the inputs as far as possible from each other, i.e.,
    choose them as $a$ and $b$.
  \item Choose the inputs to be as close to each other as possible,
    i.e., choose them to be nearby points but not equal to each other.
  \item It does not matter. Any choice of two distinct inputs is good
    enough.
  \end{enumerate}

  \vspace{0.1in}
  Your answer: $\underline{\qquad\qquad\qquad\qquad\qquad\qquad\qquad}$
  \vspace{0.1in}

\item (*) $f$ is a function of one variable defined on an interval
  $[a,b]$. You are trying to find an explicit function that fits
  $f$ well. You initially try a straight line fit that works at the points
  $a$ and $b$. It turns out that this fit systematically overestimates
  $f$ for points in between (i.e., the actual function $f$ is below
  the linear function) with the maximum magnitude of discrepancy
  occurring at the midpoint $(a + b)/2$. Based on this information,
  what kind of fit should you try to look for?

  \begin{enumerate}[(A)]
  \item Try to fit $f$ using a logarithmic function
  \item Try to fit $f$ using an exponential function
  \item Try to fit $f$ using a quadratic function
  \item Try to fit $f$ using a polynomial of degree at most $3$
  \item Try to fit $f$ using the reciprocal of a linear function
  \end{enumerate}

  \vspace{0.1in}
  Your answer: $\underline{\qquad\qquad\qquad\qquad\qquad\qquad\qquad}$
  \vspace{0.1in}

\item (*) Recall the Leontief input-output model. Recall that the GDP is
  defined as the total money value of all the {\em final} goods and
  services produced in the economy, which in this case means only
  those that go into meeting consumer demand, not interindustry demand
  (note that we are assuming away the existence of investment and
  government spending, which complicate the GDP calculation). Assuming
  that the unit prices of the goods are constant (a very unrealistic
  assumption given that price itself responds to supply and demand,
  but fortunately it does not affect the conclusion we draw here) what
  might be a way of increasing GDP while keeping the magnitude of
  output of each industry the same?

  \begin{enumerate}[(A)]
  \item Increase interindustry dependence, i.e., increase the amount
    needed from each industry that is necessary to produce a given
    amount in another industry.
  \item Reduce interindustry dependence, i.e., reduce the amount
    needed from each industry that is necessary to produce a given
    amount in another industry.
  \item Changes in interindustry dependence have no effect.
  \end{enumerate}

  \vspace{0.1in}
  Your answer: $\underline{\qquad\qquad\qquad\qquad\qquad\qquad\qquad}$
  \vspace{0.1in}

\end{enumerate}

\end{document}
