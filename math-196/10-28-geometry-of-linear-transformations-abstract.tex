\documentclass[10pt]{amsart}

%Packages in use
\usepackage{fullpage, hyperref, vipul, enumerate}

%Title details
\title{Take-home class quiz: due Monday October 28: Geometry of linear transformations (abstract)}
\author{Math 196, Section 59 (Vipul Naik)}
%List of new commands

\begin{document}
\maketitle

Your name (print clearly in capital letters): $\underline{\qquad\qquad\qquad\qquad\qquad\qquad\qquad\qquad\qquad\qquad}$

{\bf PLEASE FEEL FREE TO DISCUSS {\em ALL} QUESTIONS.}

For the questions here, please use the following terminology.

Suppose $n$ is a fixed natural number greater than $1$. For ease of
geometric visualization, you can take $n = 2$ for the discussion.

\begin{itemize}
\item A {\em linear automorphism} of $\R^n$ is defined as
  a bijective linear transformation from $\R^n$ to $\R^n$.
\item An {\em affine linear automorphism} of $\R^n$ is defined as a
  bijective function from $\R^n$ to itself that preserves
  collinearity, i.e., it sends lines to lines. In addition, it
  preserves the ratios of lengths within each line. This can be
  included as part of the definition or deduced from the fact that
  collinearity is preserved for $n > 1$.
\item A {\em self-isometry} of $\R^n$ is defined as a bijective
  function from $\R^n$ to itself that preserves Euclidean distance:
  for all pairs of points $\vec{x},\vec{y} \in \R^n$, the Euclidean
  distance between $\vec{x}$ and $\vec{y}$ equals the Euclidean
  distance between $T(\vec{x})$ and $T(\vec{y})$.
\item A {\em self-homothety} (or {\em similitude transformation} or
  {\em similarity transformation}) of $\R^n$ is defined as a bijective
  function from $\R^n$ to itself that multiplies all distances by a
  fixed number called the {\em factor of similitude} (dependent on the
  transformation): if the factor of similitude is $\lambda$, then for
  all pairs of points $\vec{x},\vec{y} \in \R^n$, the distance between
  $T(\vec{x})$ and $T(\vec{y})$ equals $\lambda$ times the distance
  between $\vec{x}$ and $\vec{y}$.
\end{itemize}

\begin{enumerate}
\item What is the relationship between linear automorphisms and affine
  linear automorphisms of $\R^n$?

  \begin{enumerate}[(A)]
  \item Being a linear automorphism is precisely the same as being an
    affine linear automorphism.
  \item Every linear automorphism is an affine linear automorphism,
    but not every affine linear automorphism is a linear automorphism.
  \item Every affine linear automorphism is a linear automorphism, but
    not every linear automorphism is an affine linear automorphism.
  \item A linear automorphism need not be affine linear, and an affine
    linear automorphism need not be linear.
  \end{enumerate}

  \vspace{0.1in}
  Your answer: $\underline{\qquad\qquad\qquad\qquad\qquad\qquad\qquad}$
  \vspace{0.1in}


\item What is the relationship between self-homotheties and
  self-isometries of $\R^n$?

  \begin{enumerate}[(A)]
  \item Being a self-homothety is precisely the same as being a
    self-isometry.
  \item Every self-homothety is a self-isometry, but not every
    self-isometry is a self-homothety.
  \item Every self-isometry is a self-homothety, but not every
    self-homothety is a self-isometry.
  \item A self-homothety need not be a self-isometry, and a
    self-isometry need not be a self-homothety.
  \end{enumerate}

  \vspace{0.1in}
  Your answer: $\underline{\qquad\qquad\qquad\qquad\qquad\qquad\qquad}$
  \vspace{0.1in}

\item What is the relationship between self-homotheties and affine
  linear automorphisms of $\R^n$?

  \begin{enumerate}[(A)]
  \item Being an affine linear automorphism is precisely the same as
    being a self-homothety.
  \item Every affine linear automorphism is a self-homothety, but not
    every self-homothety is an affine linear automorphism.
  \item Every self-homothety is an affine linear automorphism, but not
    every affine linear automorphism is a self-homothety.
  \item An affine linear automorphism need not be a self-homothety,
    and a self-homothety need not be affine linear.
  \end{enumerate}

  \vspace{0.1in}
  Your answer: $\underline{\qquad\qquad\qquad\qquad\qquad\qquad\qquad}$
  \vspace{0.1in}

\item What is the relationship between affine linear automorphisms and
  self-isometries of $\R^n$?

  \begin{enumerate}[(A)]
  \item Being an affine linear automorphism is precisely the same as
    being a self-isometry.
  \item Every affine linear automorphism is a self-isometry, but not
    every self-isometry is an affine linear automorphism.
  \item Every self-isometry is an affine linear automorphism, but not
    every affine linear automorphism is a self-isometry.
  \item An affine linear automorphism need not be a self-isometry, and
    a self-isometry need not be affine linear.
  \end{enumerate}

  \vspace{0.1in}
  Your answer: $\underline{\qquad\qquad\qquad\qquad\qquad\qquad\qquad}$
  \vspace{0.1in}

\item There is a special kind of bijection from $\R^n$ to $\R^n$
  called a {\em translation}. A translation with translation vector
  $\vec{v}$ is defined as the bijection $\vec{x} \mapsto \vec{x} +
  \vec{v}$. A {\em nontrivial} translation is a translation whose
  translation vector is not the zero vector. Which of the following is
  an automorphism type that nontrivial translations are {\em not}?
  Please see Option (E) before answering.

  \begin{enumerate}[(A)]
  \item Linear automorphism
  \item Affine linear automorphism
  \item Self-isometry
  \item Self-homothety
  \item None of the above, i.e., nontrivial translations are of all
    these types
  \end{enumerate}

  \vspace{0.1in}
  Your answer: $\underline{\qquad\qquad\qquad\qquad\qquad\qquad\qquad}$
  \vspace{0.1in}

\item A collection of bijections from $\R^n$ to itself is said to form
  a {\em group} if it satisfies all these three conditions:

  \begin{itemize}
  \item The composite of any two (possibly equal, possibly distinct)
    bijections in the collection is also in the collection.
  \item The identity bijection (i.e., the map sending every vector to
    itself) is in the collection.
  \item For every bijection in the collection, the inverse bijection is
    also in the collection.
  \end{itemize}

  For fixed $n$, which of the following collections of bijections from
  $\R^n$ to itself does {\em not} form a group? Please see Option (E)
  before answering.

  \begin{enumerate}[(A)]
  \item The collection of all linear automorphisms of $\R^n$
  \item The collection of all affine linear automorphisms of $\R^n$
  \item The collection of all self-isometries of $\R^n$
  \item The collection of all self-homotheties of $\R^n$
  \item None of the above, i.e., each of them is a group
  \end{enumerate}

  \vspace{0.1in}
  Your answer: $\underline{\qquad\qquad\qquad\qquad\qquad\qquad\qquad}$
  \vspace{0.1in}


  For the remaining questions, we deal with the case $n = 2$.

  We consider two special types of bijections from $\R^2$ to $\R^2$:
  {\em rotations} (a rotation is specified by the center of rotation
  and the angle of rotation) and {\em reflections} (a reflection is
  specified by the line of reflection).
  
  The identity map (i.e., the map sending every point to itself) is
  considered both a translation and a rotation. It is the translation
  by the zero vector. It can be viewed as a rotation about any point
  by the zero angle.

  Note that for a rotation, the angle of rotation is determined
  uniquely up to additive multiples of $2\pi$. The center of rotation
  is determined uniquely for all nontrivial rotations.

 \item What is the composite of two rotations centered at the same
   point in $\R^2$? Assume for simplicity that the composite is not the
   identity, i.e., the two rotations do not cancel each other. Note
   that the rotations must commute, so the order of operation does
   not matter.

   \begin{enumerate}[(A)]
   \item It must be a reflection about a line passing through that center point.
   \item It must be a reflection about a line {\em not} passing through
     that center point.
   \item It must be a rotation centered at the same point
   \item It must be a rotation but it need not be centered at the same point.
   \item It must be a translation.
   \end{enumerate}

   \vspace{0.1in}
   Your answer: $\underline{\qquad\qquad\qquad\qquad\qquad\qquad\qquad}$
   \vspace{0.1in}

 \item What is the composite of two reflections about lines in $\R^2$,
   if the two lines of reflection are known to be parallel but
   distinct? Although the two reflections do not commute, the {\em
     type} of their composite does not depend upon the order in which
   we compose them.

   \begin{enumerate}[(A)]
   \item It must be a reflection about a third line which is parallel
     to both the lines and is equidistant from them.
   \item It must be a reflection about a third line which is
     perpendicular to both the lines.
   \item It must be a rotation about a point that is equidistant from
     both lines.
   \item It must be a translation by a vector parallel to the lines
     about which we are reflecting.
   \item It must be a translation by a vector perpendicular to the
     lines about which we are reflecting.
   \end{enumerate}

   \vspace{0.1in}
   Your answer: $\underline{\qquad\qquad\qquad\qquad\qquad\qquad\qquad}$
   \vspace{0.1in}

 \item What is the composite of two reflections about lines in $\R^2$,
   if the two lines of reflection are distinct and intersect? Once
   again, the reflections do not in general commute, but the {\em type}
   of the composite does not depend on the order of composition.

   \begin{enumerate}[(A)]
   \item It must be a reflection about a third line which passes
     through the point of intersection of the two lines of reflection.
   \item It must be a reflection about a third line which does not pass
     through the point of intersection of the two lines of reflection.
   \item It must be a rotation about the point of intersection.
   \item It must be a translation by a vector that makes equal angles
     with both the lines.
   \item It need not be a translation, rotation, or reflection.
   \end{enumerate}

   \vspace{0.1in}
   Your answer: $\underline{\qquad\qquad\qquad\qquad\qquad\qquad\qquad}$
   \vspace{0.1in}

 \item What is the composite of a nontrivial rotation in $\R^2$ (i.e.,
   the angle of rotation is not a multiple of $2\pi$) and a nontrivial
   translation?

   \begin{enumerate}[(A)]
   \item It must be a rotation with the same center of rotation but
     with a different angle of rotation.
   \item It must be a rotation with the same angle of rotation but with a
     different center of rotation.
  \item It must be a reflection about a line passing through the
    center of rotation.
  \item It must be a reflection about a line {\em not} passing through
    the center of rotation.
  \item It must be a translation.
  \end{enumerate}

  \vspace{0.1in}
  Your answer: $\underline{\qquad\qquad\qquad\qquad\qquad\qquad\qquad}$
  \vspace{0.1in}

\item An affine linear automorphism of $\R^2$ is termed {\em
  area-preserving} if it preserves areas, i.e., the area of the image
  of any triangle under the automorphism is the same as the area of
  the original triangle. What is the relation between being a
  self-isometry and being an area-preserving affine linear
  automorphism of $\R^2$?

  \begin{enumerate}[(A)]
  \item Being a self-isometry is precisely the same as being an
    area-preserving affine linear automorphism.
  \item Every self-isometry is area-preserving, but not every
    area-preserving affine linear automorphism is a self-isometry.
  \item Every area-preserving affine linear automorphism is a
    self-isometry, but not every self-isometry is area-preserving.
  \item A self-isometry need not be an area-preserving affine linear
    automorphism, and an area-preserving affine linear automorphism
    need not be a self-isometry.
  \end{enumerate}

  \vspace{0.1in}
  Your answer: $\underline{\qquad\qquad\qquad\qquad\qquad\qquad\qquad}$
  \vspace{0.1in}

\item An affine linear automorphism of $\R^2$ is termed {\em
  orientation-preserving} if it preserves orientation, i.e., it does
  not interchange left with right. An affine linear automorphism of
  $\R^2$ is termed {\em orientation-reversing} if it reverses
  orientation, i.e., it interchanges the roles of left and
  right. Obviously, the composite of two orientation-preserving affine
  linear automorphisms is orientation-preserving. What can we say
  about the composite of two orientation-reversing affine linear
  automorphisms?

  \begin{enumerate}[(A)]
  \item It must be orientation-preserving
  \item It must be orientation-reversing
  \item It may be orientation-preserving or orientation-reversing
  \end{enumerate}

  \vspace{0.1in}
  Your answer: $\underline{\qquad\qquad\qquad\qquad\qquad\qquad\qquad}$
  \vspace{0.1in}

\item The linear automorphism of $\R^2$ with matrix:

  $$\left[\begin{matrix} 1 & 1 \\ 0 & 1 \\\end{matrix}\right]$$

  is an example of a {\em shear automorphism}. Which of the following
  is this automorphism {\em not}? Please see options (D) and (E)
  before answering.

  \begin{enumerate}[(A)]
  \item Area-preserving
  \item Orientation-preserving
  \item Self-isometry
  \item None of the above, i.e., it is area-preserving,
    orientation-preserving, and a self-isometry
  \item All of the above, i.e., it is not area-preserving, not
    orientation-preserving, and not a self-isometry of $\R^2$.
  \end{enumerate}

  \vspace{0.1in}
  Your answer: $\underline{\qquad\qquad\qquad\qquad\qquad\qquad\qquad}$
  \vspace{0.1in}

\item Which of the following is guaranteed to send any triangle in
  $\R^2$ to a similar triangle? Please see Options (D) and (E) before
  answering.

  \begin{enumerate}[(A)]
  \item Linear automorphism
  \item Affine linear automorphism
  \item Self-homothety
  \item All of the above
  \item None of the above
  \end{enumerate}

  \vspace{0.1in}
  Your answer: $\underline{\qquad\qquad\qquad\qquad\qquad\qquad\qquad}$
  \vspace{0.1in}

\end{enumerate}
\end{document}
