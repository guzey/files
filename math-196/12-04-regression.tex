\documentclass[10pt]{amsart}

%Packages in use
\usepackage{fullpage, hyperref, vipul, enumerate}

%Title details
\title{Take-home class quiz: due Wednesday December 4: Ordinary least squares regression}
\author{Math 196, Section 57 (Vipul Naik)}
%List of new commands

\begin{document}
\maketitle

Your name (print clearly in capital letters): $\underline{\qquad\qquad\qquad\qquad\qquad\qquad\qquad\qquad\qquad\qquad}$

{\bf PLEASE FEEL FREE TO DISCUSS {\em ALL} QUESTIONS.}

\begin{enumerate}

\item Assume no measurement error. Consider the situation where we
  have a function $f$ of the form $f(x) =a_0 + a_1x$ with unknown
  values of the parameters $a_0$ and $a_1$. We collect $n$ distinct
  input-output pairs, i.e., we collect $n$ distonct inputs and compute
  the outputs for them. The coefficient matrix for the system is a $n
  \times 2$ matrix (the rows correspond to the input values, and the
  columns correspond to the unknown parameters). What is the rank of
  this matrix?

  \begin{enumerate}[(A)]
  \item It is always $2$
  \item It is always $n$
  \item It is always $\min \{ 2, n \}$
  \item It is always $\max \{ 2, n \}$
  \end{enumerate}

  \vspace{0.1in}
  Your answer: $\underline{\qquad\qquad\qquad\qquad\qquad\qquad\qquad}$
  \vspace{0.1in}

\item Assume no measurement error. Consider the situation where we
  have a function $f$ of the form $f(x) =a_0 + a_1x + a_2x^2 + \dots +
  a_mx^m$ with unknown values of the parameters $a_0$, $a_1$, $\dots$,
  $a_m$. We collect $n$ distinct input-output pairs, i.e., we collect
  $n$ distinct inputs and compute the outputs for them. The
  coefficient matrix for the system is a $n \times (m + 1)$ matrix
  (the rows correspond to the input values, and the columns correspond
  to the unknown parameters). What is the rank of this matrix?

  \begin{enumerate}[(A)]
  \item It is always $m + 1$
  \item It is always $n$
  \item It is always $\min \{ m + 1, n \}$
  \item It is always $\max \{ m + 1, n \}$
  \end{enumerate}

  \vspace{0.1in}
  Your answer: $\underline{\qquad\qquad\qquad\qquad\qquad\qquad\qquad}$
  \vspace{0.1in}

\item Assume no measurement error. Consider the situation where we
  have a function $f$ of the form $f(x,y) = a_0 + a_1x + a_2y$ with
  unknown values of the parameters $a_0$, $a_1$, and $a_2$. We collect
  $n$ distinct input-output pairs, i.e., we collect $n$ distinct
  inputs (here an input specification involves specifying both the
  $x$-value and the $y$-value) and compute the outputs for them. The
  coefficient matrix for the system is a $n \times 3$ matrix (the rows
  correspond to the input values, and the columns correspond to the
  unknown parameters). What is the rank of this matrix?

  \begin{enumerate}[(A)]
  \item It is always $\min \{ 3,n \}$
  \item It is always $\max \{ 3,n \}$
  \item For $n = 1$, it is $1$. For $n \ge 2$, it is $2$ if the input
    points are all collinear in the $xy$-plane. Otherwise, it is $3$.
  \item For $n = 1$, it is $1$. For $n \ge 2$, it is $3$ if the input
    points are all collinear in the $xy$-plane. Otherwise, it is $2$.
  \end{enumerate}

  \vspace{0.1in}
  Your answer: $\underline{\qquad\qquad\qquad\qquad\qquad\qquad\qquad}$
  \vspace{0.1in}

\item Which of the following is closest to correct in the setting
  where we use a linear system to find the parameters using
  input-output pairs given a functional form that is linear in the
  parameters? Assume for simplicity that we are dealing with a
  functional form $y = f(x)$ with one input and one output, but
  possibly multiple parameters in the general description.

  \begin{enumerate}[(A)]
  \item The solutions to the linear system that we set up correspond
    to possibilities for the inputs to the function, and geometrically
    correspond to choices of points $x$ for the graph $y = f(x)$.
  \item The solutions to the linear system that we set up correspond
    to possibilities for the inputs to the function, and geometrically
    correspond to different possible choices for the line or curve
    that is the graph $y = f(x)$.
  \item The solutions to the linear system that we set up correspond
    to possibilities for the parameters, and geometrically correspond
    to choices of points $x$ for the graph $y = f(x)$.
  \item The solutions to the linear system that we set up correspond
    to possibilities for the parameters, and geometrically correspond
    to different possible choices for the line or curve that is the
    graph $y = f(x)$.
  \end{enumerate}

  \vspace{0.1in}
  Your answer: $\underline{\qquad\qquad\qquad\qquad\qquad\qquad\qquad}$
  \vspace{0.1in}

\item Continuing with the notation and setup of the preceding
  question, consider the coefficient matrix of the linear system. This
  matrix defines a linear transformation from the vector space of
  possible parameter values to the vector space of the outputs of the
  function. What is the image of this linear transformation?

  \begin{enumerate}[(A)]
  \item The image is the set of possible output values for which the
    linear system is consistent, i.e., we can find {\em at least one}
    function $f$ of the required functional form that fits all the
    input-output pairs with {\em no measurement error}.
  \item The image is the set of possible output values for which the
    linear system has {\em at most one solution}, i.e., the set of
    output valus for which we can find {\em at most one} function $f$
    of the required functional form that fits all the input-output
    pairs with {\em no measurement error}.
  \end{enumerate}
  \vspace{0.1in}
  Your answer: $\underline{\qquad\qquad\qquad\qquad\qquad\qquad\qquad}$
  \vspace{0.1in}

\item Consider the case of polynomial regression for a polynomial
  function of one variable, allowing for measurement error. We believe
  that a function has the form of a polynomial. We can tentatively
  choose a degree $m$ for the polynomial we are trying to fit, and a
  value $n$ for the number of distinct inputs for which we compute the
  corresponding outputs to obtain input-output pairs (i.e., data
  points). We will get a $n \times (m + 1)$ coefficient matrix. Which
  of the following correctly describes what we should try for?

  \begin{enumerate}[(A)]
  \item We should choose $n$ and $m + 1$ to be exactly equal, so that
    we get a unique polynomial.
  \item We should choose $n$ to be greater than $m + 1$, so that the
    system is guaranteed to be consistent and we can find the
    polynomial.
  \item We should choose $n$ to be less than $m + 1$, so that the
    system is guaranteed to be consistent and we can find the polynomial.
  \item We should choose $n$ to be greater than $m + 1$, so that the
    system is {\em not} guaranteed to be consistent, but we do have a
    unique solution after we project the output vector to a vector for
    which the system is consistent.
  \item We should choose $n$ to be less than $m + 1$, so that the
    system is {\em not} guaranteed to be consistent, but we do have a
    unique solution after we project the output vector to a vector for
    which the system is consistent.
  \end{enumerate}

  \vspace{0.1in}
  Your answer: $\underline{\qquad\qquad\qquad\qquad\qquad\qquad\qquad}$
  \vspace{0.1in}

\item Consider the general situation of linear regression. Denote by
  $X$ the coefficient matrix for the linear system (also called the
  design matrix). Denote by $\vec{\beta}$ the parameter vector that we
  are trying to solve for. Denote by $\vec{y}$ an observed output
  vector. The idea in ordinary least squares regression is to choose a
  suitable vector $\vec{\varepsilon}$ such that the linear system
  $X\vec{\beta} = \vec{y} - \vec{\varepsilon}$ can be solved for
  $\vec{\beta}$. Among the many possibilities that we can choose for
  $\vec{\varepsilon}$, what criterion do we use to select the
  appropriate choice? Recall that the {\em length} of a vector is the
  square root of the sum of squares of its coordinates.

  \begin{enumerate}[(A)]
  \item We choose $\vec{\varepsilon}$ to have the minimum length
    possible subject to the constraint that $X\vec{\beta} = \vec{y} -
    \vec{\varepsilon}$ has a solution.
  \item We choose $\vec{\varepsilon}$ such that the system
    $X\vec{\beta} = \vec{y} - \vec{\varepsilon}$ can be solved and
    such that the solution vector $\vec{\beta}$ has the minimum
    possible length (among all such choices of $\vec{\varepsilon}$).
  \item We choose $\vec{\varepsilon}$ such that the system
    $X\vec{\beta} = \vec{y} - \vec{\varepsilon}$ can be solved and
    such that the difference vector $\vec{y} - \vec{\varepsilon}$ has
    the minimum possible length (among all such choices of
    $\vec{\varepsilon}$).
  \end{enumerate}

  \vspace{0.1in}
  Your answer: $\underline{\qquad\qquad\qquad\qquad\qquad\qquad\qquad}$
  \vspace{0.1in}

\item We have data for the logarithm of annual per capita GDP for a
  country for the last 100 years. We want to see if this fits a
  polynomial model. The idea is to try to first fit a polynomial of
  degree $0$ (i.e., per capita GDP remains constant), then fit a
  polynomial of degree $\le 1$ (i.e., per capita GDP grows or decays
  exponentially), then fit a polynomial of degree $\le 2$ (i.e., per
  capita GDP grows or decays as the exponential of a quadratic
  function), and so on.

  What happens to the length of the error vector $\vec{\varepsilon}$ as
  we increase the degree of the polynomial that we are trying to fit?

  \begin{enumerate}[(A)]
  \item The error vector $\vec{\varepsilon}$ keeps getting smaller and
    smaller in length, with a probability of almost $1$ that it keeps
    {\em strictly} decreasing in length at each step, until the error
    vector becomes $\vec{0}$ (which we expect will happen when we get
    to the stage of trying to fit the function using a polynomial of
    degree 99).
  \item The error vector $\vec{\varepsilon}$ keeps getting larger and
    larger in length, with a probability of almost $1$ that it keeps
    {\em strictly} increasing in length at each step, until the error
    vector becomes $\vec{y}$ (which we expect will happen when we get
    to the stage of trying to fit the function using a polynomial of
    degree 99).
  \end{enumerate}

  \vspace{0.1in}
  Your answer: $\underline{\qquad\qquad\qquad\qquad\qquad\qquad\qquad}$
  \vspace{0.1in}
\end{enumerate}
\end{document}
