\documentclass[10pt]{amsart}

%Packages in use
\usepackage{fullpage, hyperref, vipul, enumerate}

%Title details
\title{Take-home class quiz: due Friday November 8: Linear transformations: seeds for reaping later}
\author{Math 196, Section 57 (Vipul Naik)}
%List of new commands

\begin{document}
\maketitle

Your name (print clearly in capital letters): $\underline{\qquad\qquad\qquad\qquad\qquad\qquad\qquad\qquad\qquad\qquad}$

{\bf PLEASE FEEL FREE TO DISCUSS {\em ALL} QUESTIONS.}

In this quiz, we will sow the seeds of ideas that we will reap
later. There are two broad classes of ideas that we touch upon here:

\begin{itemize}
\item Conjugation, similarity transformations, and products of
  matrices: This will be of relevance later when we discuss change of
  coordinates. We cover change of coordinates in more detail in
  Section 3.4 of the text.
\item Kernel and image for linear transformations arising from
  calculus, typically for infinite-dimensional spaces: This will be
  helpful in understanding linear transformations in an {\em abstract}
  sense, a topic that we cover in more detail in Chapter 4 of the text.
\end{itemize}

\begin{enumerate}

\item Suppose $A$ and $B$ are (possibly equal, possibly distinct) $n
  \times n$ matrices for some $n > 1$. Recall that the {\em trace} of
  a matrix is defined as the sum of its diagonal entries. Suppose $C =
  AB$ and $D = BA$. Which of the following is true?

  \begin{enumerate}[(A)]
  \item It must be the case that $C = D$
  \item The {\em set} of entry values in $C$ is the same as the set of
    entry values in $D$, but they may appear in a different order.
  \item $C$ and $D$ need not be equal, but the sum of all the matrix
    entries of $C$ must equal the sum of all the matrix entries of
    $D$.
  \item $C$ and $D$ need not be equal, but they have the same
    diagonal, i.e., every diagonal entry of $C$ equals the
    corresponding diagonal entry of $D$.
  \item $C$ and $D$ need not be equal and they need not even have the
    same diagonal. However, they must have the same trace, i.e., the
    sum of the diagonal entries of $C$ equals the sum of the diagonal
    entries of $D$.
  \end{enumerate}

  \vspace{0.1in}
  Your answer: $\underline{\qquad\qquad\qquad\qquad\qquad\qquad\qquad}$
  \vspace{0.1in}

  Suppose $A$ is an invertible $n \times n$ matrix. The {\em
    conjugation operation} corresponding to $A$ is the map that sends
  any $n \times n$ matrix $X$ to $AXA^{-1}$. We can verify that the
  following hold for any two (possibly equal, possibly distinct) $n
  \times n$ matrices $X$ and $Y$:

  \begin{eqnarray*}
    A(X + Y)A^{-1} & = &  AXA^{-1} + AYA^{-1}\\
    A(XY)A^{-1} & = & (AXA^{-1})(AYA^{-1})\\
    AX^rA^{-1} & = & (AXA^{-1})^r\\
  \end{eqnarray*}

  The conceptual significance of this will (hopefully!) become clearer
  as we proceed. 

\item Which of the following is guaranteed to be the same for $X$ and
  $AXA^{-1}$?

  \begin{enumerate}[(A)]
  \item The sum of all entries
  \item The sum of squares of all entries
  \item The product of all entries
  \item The sum of all diagonal entries (i.e., the trace)
  \item The sum of squares of all diagonal entries
  \end{enumerate}

  \vspace{0.1in}
  Your answer: $\underline{\qquad\qquad\qquad\qquad\qquad\qquad\qquad}$
  \vspace{0.1in}

\item $A$ and $X$ are $n \times n$ matrices, with $A$
  invertible. Which of the following is/are true? Please see Options
  (D) and (E) before answering, and select a single option that best
  reflects your view.

  \begin{enumerate}[(A)]
  \item $X$ is invertible if and only if $AXA^{-1}$ is invertible.
  \item $X$ is nilpotent if and only if $AXA^{-1}$ is nilpotent.
  \item $X$ is idempotent if and only if $AXA^{-1}$ is idempotent.
  \item All of the above.
  \item None of the above.
  \end{enumerate}

  \vspace{0.1in}
  Your answer: $\underline{\qquad\qquad\qquad\qquad\qquad\qquad\qquad}$
  \vspace{0.1in}

\item $A$ and $X$ are $n \times n$ matrices, with $A$
  invertible. Which of the following is equivalent to the condition
  that $AXA^{-1} = X$?

  \begin{enumerate}[(A)]
  \item $A + X = X + A$
  \item $A - X = X - A$
  \item $AX = XA$
  \item $XA^{-1} = AX^{-1}$
  \item None of the above
  \end{enumerate}

  \vspace{0.1in}
  Your answer: $\underline{\qquad\qquad\qquad\qquad\qquad\qquad\qquad}$
  \vspace{0.1in}

  Let's look at a computational application of matrix conjugation.

  One computational application is power computation. Suppose we have
  a $n \times n$ matrix $B$ and we need to compute $B^r$ for a very
  large $r$. This requires $O(\log_2r)$ multiplications, but note that
  each multiplication, if done naively, takes time $O(n^3)$ for a
  generic matrix. Suppose, however, that there exists a matrix $A$
  such that the matrix $C = ABA^{-1}$ is diagonal. If we can find $A$
  (and hence $C$) efficiently, then we can compute $C^r = (ABA^{-1})^r
  = AB^rA^{-1}$, and therefore $B^r = A^{-1}C^rA$. Note that each
  multiplication of diagonal matrices takes $O(n)$ multiplications, so
  this reduces the overall arithmetic complexity from $O(n^3\log_2r)$
  to $O(n\log_2r)$. Note, however, that this is contingent on our
  being able to find the matrices $A$ and $C$ first. We will later see
  a method for finding $A$ and $C$. Unfortunately, this method relies
  on finding the set of solutions to a polynomial equation of degree
  $n$, which requires operations that go beyond ordinary arithmetic
  operations of addition, subtraction, multiplication, and
  division. Even in the case $n = 2$, it requires solving a quadratic
  equation. We do have the formula for that.

\item Consider the following example of the above general setup with $n = 2$:

  $$B = \left[ \begin{matrix} 1 & -1 \\ 0 & 2 \\\end{matrix}\right]$$

  We can choose:

  $$A = \left[ \begin{matrix} 1 & 1 \\ 0 & 1 \\\end{matrix}\right]$$

  The matrix $C = ABA^{-1}$ is a diagonal matrix. What diagonal matrix
  is it?

  \begin{enumerate}[(A)]
  \item $\left[\begin{matrix} 1 & 0 \\ 0 & -2 \\\end{matrix}\right]$
  \item $\left[\begin{matrix} -1 & 0 \\ 0 & 2 \\\end{matrix}\right]$
  \item $\left[\begin{matrix} 2 & 0 \\ 0 & -1 \\\end{matrix}\right]$
  \item $\left[\begin{matrix} 1 & 0 \\ 0 & 2 \\\end{matrix}\right]$
  \item $\left[\begin{matrix} -1 & 0 \\ 0 & -2 \\\end{matrix}\right]$
  \end{enumerate}

  \vspace{0.1in}
  Your answer: $\underline{\qquad\qquad\qquad\qquad\qquad\qquad\qquad}$
  \vspace{0.1in}

\item With $A$, $B$, and $C$ as in the preceding question, what is the
  value of $B^8$? Use that $2^8 = 256$.

  \begin{enumerate}[(A)]
  \item $\left[ \begin{matrix} 1 & -1 \\ 0 & 256 \\\end{matrix}\right]$
  \item $\left[ \begin{matrix} 1 & -255 \\ 0 & 256 \\\end{matrix}\right]$
  \item $\left[ \begin{matrix} 1 & 253 \\ 0 & 256 \\\end{matrix}\right]$
  \item $\left[ \begin{matrix} 1 & 253 \\ 254 & 256 \\\end{matrix}\right]$
  \item $\left[ \begin{matrix} 16 & -8 \\ 0 & 256 \\\end{matrix}\right]$
  \end{enumerate}

  \vspace{0.1in}
  Your answer: $\underline{\qquad\qquad\qquad\qquad\qquad\qquad\qquad}$
  \vspace{0.1in}

\item Suppose $n > 1$. Let $A$ be a $n \times n$ matrix such that the
  linear transformation corresponding to $A$ is a self-isometry of
  $\R^n$, i.e., it preserves distances. Which of the following must
  necessarily be true? You can use the case $n = 2$ and the example of
  rotations to guide your thinking.

  \begin{enumerate}[(A)]
  \item The trace of $A$ (i.e., the sum of the diagonal entries of
    $A$) must be equal to $0$
  \item The trace of $A$ (i.e., the sum of the diagonal entries of
    $A$) must be equal to $1$
  \item The sum of the entries in each column of $A$ must be equal to
    $1$
  \item The sum of the absolute values of the entries in each column
    of $A$ must be equal to $1$
  \item The sum of the squares of the entries in each column of $A$
    must be equal to $1$
  \end{enumerate}

  \vspace{0.1in}
  Your answer: $\underline{\qquad\qquad\qquad\qquad\qquad\qquad\qquad}$
  \vspace{0.1in}

  A {\em real vector space} (just called {\em vector space} for short)
  is a set $V$ equipped with the following structures:

  \begin{itemize}
  \item A binary operation $+$ on $V$ called addition that is
    commutative and associative.
  \item A special element $0 \in V$ that is an identity for addition.
  \item A scalar multiplication operation $\R \times V \to V$ denoted
    by concatenation such that:
    \begin{itemize}
    \item $0\vec{v} = 0$ (the $0$ on the right side being the vector
      $0$) for all $\vec{v} \in V$.
    \item $1\vec{v} = \vec{v}$ for all $\vec{v} \in V$.
    \item $a(b\vec{v}) = (ab)\vec{v}$ for all $a,b \in \R$ and
      $\vec{v} \in V$.
    \item $a(\vec{v} + \vec{w}) = a\vec{v} + a\vec{w}$ for all $a \in
      \R$ and $\vec{v},\vec{w} \in V$.
    \item $(a + b)\vec{v} = a\vec{v} + b\vec{v}$ for all $a,b \in \R$,
      $\vec{v} \in V$.
    \end{itemize}
  \end{itemize}

  A {\em subspace} of a vector space is defined as a nonempty subset
  that is closed under addition and scalar multiplication. In
  particular, any subspace must contain the zero vector. A subspace of
  a vector space can be viewed as being a vector space in its own
  right.

  Suppose $V$ and $W$ are vector spaces. A function $T: V \to W$ is
  termed a {\em linear transformation} if $T$ preserves addition and
  scalar multiplication, i.e., we have the following two conditions:

  \begin{itemize}
  \item $T(\vec{v_1} + \vec{v_2}) = T(\vec{v_1}) + T(\vec{v_2})$ for
    all vectors $\vec{v_1},\vec{v_2} \in V$.
  \item $T(a\vec{v}) = aT(\vec{v})$ for all $a \in \R$, $\vec{v} \in
    V$.
  \end{itemize}

  The {\em kernel} of a linear transformation $T$ is defined as the
  set of all vectors $\vec{v}$ such that $T(\vec{v})$ is the zero
  vector. The {\em image} of a linear transformation $T$ is defined as
  its range as a set map.

  Denote by $C(\R)$ (or alternatively by $C^0(\R)$) the vector space
  of all continuous functions from $\R$ to $\R$, with pointwise
  addition and scalar multiplication. Note that the elements of this
  vector space, which we would ordinarily call ``vectors'', are now
  {\em functions}.

  For $k$ a positive integer, denote by $C^k(\R)$ the subspace of
  $C(\R)$ comprising those continuous functions that are at least $k$
  times {\em continuously} differentiable. Note that $C^{k+1}(\R)$ is
  a subspace of $C^k(\R)$, so we have a descending chain of subspaces:

  $$C(\R) = C^0(\R) \supseteq C^1(\R) \supseteq C^2(\R) \supseteq \dots $$

  The intersection of these spaces is the vector space $C^\infty(\R)$,
  defined as the subspace of $C(\R)$ comprising those functions that
  are {\em infinitely} differentiable.

\item We can think of differentiation as a linear transformation. Of
  the following options, which is the broadest way of viewing
  differentiation as a linear transformation? By ``broadest'' we mean
  ``with the largest domain that makes sense among the given options.''

  \begin{enumerate}[(A)]
  \item From $C^\infty(\R)$ to $C^\infty(\R)$
  \item From $C^0(\R)$ to $C^1(\R)$
  \item From $C^1(\R)$ to $C^0(\R)$
  \item From $C^1(\R)$ to $C^2(\R)$
  \item From $C^2(\R)$ to $C^1(\R)$
  \end{enumerate}

  \vspace{0.1in}
  Your answer: $\underline{\qquad\qquad\qquad\qquad\qquad\qquad\qquad}$
  \vspace{0.1in}

\item Under the differentiation linear transformation, what is the
  image of $C^k(\R)$ for a positive integer $k$?

  \begin{enumerate}[(A)]
  \item $C^{k-1}(\R)$
  \item $C^k(\R)$
  \item $C^{k+1}(\R)$
  \item $C^1(\R)$
  \item $C^\infty(\R)$
  \end{enumerate}

  \vspace{0.1in}
  Your answer: $\underline{\qquad\qquad\qquad\qquad\qquad\qquad\qquad}$
  \vspace{0.1in}

\item What is the kernel of differentiation?

  \begin{enumerate}[(A)]
  \item The vector space of all constant functions
  \item The vector space of all linear functions (i.e., functions of
    the form $x \mapsto mx + c$ with $m,c \in \R$)
  \item The vector space of all polynomial functions
  \item $C^\infty(\R)$
  \item $C^1(\R)$
  \end{enumerate}

  \vspace{0.1in}
  Your answer: $\underline{\qquad\qquad\qquad\qquad\qquad\qquad\qquad}$
  \vspace{0.1in}

\item Suppose $k$ is a positive integer greater than $2$. Consider the
  operation of ``differentiating $k$ times.'' This is a linear
  transformation that can be defined as the $k$-fold composite of
  differentiation with itself. Viewed most generally, this is a linear
  transformation from $C^k(\R)$ to $C(\R)$. What is the kernel of this
  linear transformation?

  \begin{enumerate}[(A)]
  \item The set of all constant functions
  \item The set of all polynomial functions of degree at most $k - 1$
  \item The set of all polynomial functions of degree at most $k$
  \item The set of all polynomial functions of degree at most $k + 1$
  \item The set of all polynomial functions
  \end{enumerate}

  \vspace{0.1in}
  Your answer: $\underline{\qquad\qquad\qquad\qquad\qquad\qquad\qquad}$
  \vspace{0.1in}

\item Suppose $k$ is a positive integer greater than $2$. Consider the
  set $P_k$ of all polynomial functions of degree at most $k$. This
  set is a vector subspace of $C(\R)$. Of the following subspaces of
  $C(\R)$, which is the {\em smallest} subspace of which $P_k$ is a
  subspace?

  \begin{enumerate}[(A)]
  \item $C^1(\R)$
  \item $C^{k-1}(\R)$
  \item $C^k(\R)$
  \item $C^{k+1}(\R)$
  \item $C^\infty(\R)$
  \end{enumerate}

  \vspace{0.1in}
  Your answer: $\underline{\qquad\qquad\qquad\qquad\qquad\qquad\qquad}$
  \vspace{0.1in}

  Two more definitions of use. A {\em linear functional} on a vector
  space $V$ is a linear transformation from $V$ to $\R$, where $\R$ is
  viewed as a one-dimensional vector space over itself in the obvious way.

  We define $C([0,1])$ as the set of all continuous functions from
  $[0,1]$ to $\R$ with pointwise addition and scalar multiplication.

\item Which of the following is {\em not} a linear functional on
  $C([0,1])$?

  \begin{enumerate}[(A)]
  \item $f \mapsto f(0)$
  \item $f \mapsto f(1)$
  \item $f \mapsto \int_0^1 f(x) \, dx$
  \item $f \mapsto \int_0^1 f(x^2) \, dx$
  \item $f \mapsto \int_0^1 (f(x))^2 \, dx$
  \end{enumerate}

   \vspace{0.1in}
   Your answer: $\underline{\qquad\qquad\qquad\qquad\qquad\qquad\qquad}$
   \vspace{0.1in}


\end{enumerate}
\end{document}
