\documentclass[10pt]{amsart}

%Packages in use
\usepackage{fullpage, hyperref, vipul, enumerate}

%Title details
\title{Take-home class quiz solutions: due Monday October 7: Linear functions and equation-solving (part 2)}
\author{Math 196, Section 57 (Vipul Naik)}
%List of new commands

\begin{document}
\maketitle

\section{Performance review}

28 people took this $5$-question quiz. The score distribution was as
follows:

\begin{itemize}
\item Score of 0: 3 people
\item Score of 1: 3 people
\item Score of 2: 9 people
\item Score of 3: 5 people
\item Score of 4: 6 people
\item Score of 5: 2 people
\end{itemize}

The question-wise answers and performance summary are given below.

\begin{enumerate}
\item Option (C): 25 people
\item Option (D): 12 people
\item Option (E): 12 people
\item Option (C): 15 people
\item Option (D): 6 people
\end{enumerate}

{\em On comparison with last time}: Last year, I had set the due date
for the take-home quiz as Wednesday rather than Monday, so that might
partly explain the better performance of students on the quiz last
time. The main question where performance diverged considerable from
last time was Q5, and this is the question where the sophistication of
one extra class can go a long way.

\section{Solutions}

This quiz covers some basics involving linear functions and
equation-solving (notes at {\tt Linear functions: a primer} and {\tt
  Equation-solving with a special focus on the linear case}). The quiz
tests for the following:

\begin{itemize}
\item The distinction between behavior relative to the variables (the
  inputs) and behavior relative to the parameters.
\item Counting the number of parameters by creating the explicit
  general functional form from a verbal description (with a special
  focus on polynomial functional forms).
\item Figuring out how to ``ask the right questions'' with respect to
  input choices, so that the answers provide meaningful
  information. This builds towards the ideas of hypothesis testing,
  rank, and overdetermination that we will see in the future.
\end{itemize}

\begin{enumerate}
\item Suppose $f$ is a polynomial function of $x$ of degree at most a
  {\em known number} $n$. What is the minimum number of (input,output)
  pairs that we need in order to determine $f$ uniquely? {\em Extra
    information: Somewhat surprisingly, in this case, we do not need
    to be judicious about our input choices. Any set of distinct
    inputs of the required number will do. This has something to do
    with the ``Vandermonde matrix'' and ``Vandermonde determinant''
    and is also related to the Lagrange interpolation formula.}

  \begin{enumerate}[(A)]
  \item $n - 1$
  \item $n$
  \item $n + 1$
  \item $2n$
  \item $n^2$
  \end{enumerate}

  {\em Answer}: Option (C)

  {\em Explanation}: There are $n + 1$ unknown coefficients, namely,
  the coefficients of $x^i$ for $0 \le i \le n$. In general, a
  polynomial of degree $n$ is of the form:

  $$a_0 + a_1x + a_2x^2 + \dots + a_nx^n$$

  Every input-output pair gives rise to a linear equation in terms of
  the coefficients. With $n + 1$ input-output pairs, we get $n + 1$
  equations in $n + 1$ variables where the ``variables'' here are the
  parameters). Due to linearity in the parameters, we actually get a
  system of $n + 1$ {\em linear} equations in the $n + 1$
  variables. We would expect the system to have a unique solution if
  the inputs are chosen judiciously.

  In fact, it turns out that the system always has a unique
  solution. The existence of a solution is given by the Lagrange
  interpolation formula, and its uniqueness follows from the fact that
  any polynomial of degree $\le n$ that has $n + 1$ distinct roots
  must be the zero polynomial. This is also related to the idea of the
  Vandermonde determinant, but that is beyond the scope of the current
  discussion.

  {\em Update for after understanding coefficient matrices and ranks}:
  Choosing the inputs judiciously basically amounts to choosing the
  inputs in a manner that the coefficient matrix (a $(n + 1) \times (n
  + 1)$ square matrix) has full rank $n + 1$. The statement above, that
  distinct inputs always work, is the statement that the coefficient
  matrix always has full rank $n + 1$ as long as the inputs are
  distinct. This type of matrix is called a Vandermonde matrix and
  there is a considerable theory related to these in algebra.

  {\em Performance review}: 25 out of 28 got this. 3 chose (B).

  {\em Historical note (last time)}: $27$ out of $28$ got this. $1$ chose (B).

\item $f$ is a polynomial function of two variables $x$ and $y$ of
  total degree at most $2$. In other words, for each monomial
  occurring in $f$, the total of the degrees of $x$ and $y$ in that
  monomial is at most $2$. No other information is given about
  $f$. What is the minimum number of judiciously chosen (input,output)
  pairs we need in order to determine $f$ uniquely?

  \begin{enumerate}[(A)]
  \item $2$
  \item $3$
  \item $4$
  \item $6$
  \item $7$
  \end{enumerate}

  {\em Answer}: Option (D)

  {\em Explanation}: First, a little bit to clarify the question. The
  total degree of a monomial $x^iy^j$ is $i + j$. For instance, the
  degree of $x^3y^2$ is $3 + 2 = 5$.

  The total degree of a polynomial that involves addition and scalar
  multiplication of monomials is the maximum of the degrees of the
  individual monomials. This is very similar to the situation with
  polynomials of one variable: the degree of the polynomial is the
  maximum of the degrees of the constituent monomials. Thus, for
  instance, $x - y + 17xy - x^2y$ has total degree $3$ because the
  constituent monomials have degree $1$, $1$, $2$, and $3$, and the
  maximum of these is $3$.

  In order to determine how many input-output pairs we would need, we
  need to count the number of parameters in the generic functional
  form. So, the first step is to figure out a {\em general} expression
  for all the possibilities we can have such polynomials of total
  degree at most $2$. Since all such polynomials are obtained by
  (linearly) combining monomials of the sort, our first step is to
  list the monomials that are allowed.

  The possible monomials involved in $f$ are $1$,
  $x$, $y$, $x^2$, $xy$, and $y^2$. There are six of them, hence there
  are $6$ coefficients to be determined by setting up and solving the
  linear system. The generic form is:

  $$a_1 + a_2x + a_3y + a_4x^2 + a_5xy + a_6y^2$$ 

  Each (input,output) pair gives a {\em linear} equation in terms of
  $a_1,a_2,\dots,a_6$. Six suitably chosen inputs will give six linear
  equations that we can then solve to uniquely determine the
  inputs. Note that unlike the functions of one variable case, it is
  possible to choose the inputs ``badly'', i.e., in a way that does
  not reveal information.

  {\em Update for after understanding coefficient matrices and ranks}:
  Choosing the inputs judiciously basically amounts to choosing the
  inputs in a manner that the coefficient matrix (a $6 \times 6$
  square matrix) has full rank $6$. In this case, a bad choice of
  inputs could lead to a coefficient matrix of lower rank, leading to
  redundant equations. Unfortunately, it is beyond the current scope
  to describe the geometry of sets of inputs for which the system does
  not have full rank (though one example would be where all inputs
  have the same $y$-value), but anyway, ``most'' random choices of
  inputs will give systems with full rank.

  {\em Performance review}: 12 out of 28 got this. 6 chose (B), 5
  chose (C), 4 chose (A), 1 chose (E).

  {\em Historical note (last time)}: $15$ out of $28$ got this. $7$
  chose (B), $3$ chose (A), $2$ chose (C), $1$ chose (E).

\item $f$ is a polynomial function of two variables $x$ and $y$ of
  total degree at most $3$. In other words, for each monomial
  occurring in $f$, the total of the degrees of $x$ and $y$ in that
  monomial is at most $3$. No other information is given about
  $f$. What is the minimum number of judiciously chosen (input,output)
  pairs we need in order to determine $f$ uniquely?

  \begin{enumerate}[(A)]
  \item $3$
  \item $6$
  \item $8$
  \item $9$
  \item $10$
  \end{enumerate}

  {\em Answer}: Option (E)

  {\em Explanation}: In addition to the $6$ monomials for the
  preceding question, there are $4$ monomials of degree three, namely
  $x^3$, $x^2y$, $xy^2$, and $y^3$. Thus, there is a total of $10$
  monomials with unknown coefficients, so we need $10$ data points to
  pin down the values of the coefficients. Explicitly, the general form looks like:

  $$a_1 + a_2x + a_3y + a_4x^2 + a_5xy + a_6y^2 + a_7x^3 + a_8x^2y + a_9xy^2 + a_{10}y^3$$   

  Each (input,output) pair gives a {\em linear} equation in terms of
  $a_1,a_2,\dots,a_{10}$. We want 10 such equations, so we want 10
  judiciously chosen input-output pairs.

  {\em Update for after understanding coefficient matrices and ranks}:
  Choosing the inputs judiciously basically amounts to choosing the
  inputs in a manner that the coefficient matrix (a $10 \times 10$
  square matrix) has full rank $10$. In this case, a bad choice of
  inputs could lead to a coefficient matrix of lower rank, leading to
  redundant equations. Unfortunately, it is beyond the current scope
  to describe the geometry of sets of inputs for which the system does
  not have full rank (though one example would be where all inputs
  have the same $y$-value), but anyway, ``most'' random choices of
  inputs will give systems with full rank.

  {\em Performance review}: 12 out of 28 got this. 7 chose (A), 5
  chose (B), 2 each chose (C) and (D).

  {\em Historical note (last time)}: $14$ out of $28$ got this. $8$ chose (A),
  $2$ each chose (B), (C), and (D).

\item (*) {\em The perils of overfitting; see also Occam's Razor}:
  Suppose we are trying to model a function that we expect to behave
  in a polynomial-like manner, though we don't really have a good
  reason to believe this. Additionally, there is a possibility for
  measurement error in our observations. Our goal is to find the
  parameters so that we can both predict unmeasured values and do a
  qualitative analysis of the nature of the function and its
  derivatives and integrals.

  We have a large number of observations (say, several thousands). We
  could attempt to ``fit'' the function using a polynomial of degree
  $n$ for some fixed $n$ using all those data points, and we will get
  a certain ``best fit'' that minimizes the deviation between the
  curve used for fitting and the function being fit. For instance, for
  $n = 1$, we are trying to find the best fit by a straight line
  function. For $n = 2$, we are trying to find the best fit by a
  polynomial of degree at most $2$. We could try fitting using
  different values of $n$. Which of the following is true?

  {\em If you are interested in more on this, look up
    ``overfitting''. A revealing quote is by mathematician and
    computer scientist John von Neumann: ``With four parameters I can
    fit an elephant. And with five I can make him wiggle his trunk.''
    Another is by prediction guru Nate Silver: ``The wide array of
    statistical methods available to researchers enables them to be no
    less fanciful and no more scientific than a child finding animal
    patterns in clouds.''}

  \begin{enumerate}[(A)]
  \item Larger values of $n$ give better fits, therefore the larger
    the value of $n$ we use, the better.
  \item Smaller values of $n$ give better fits, therefore the smaller
    the value of $n$ we use, the better.
  \item Larger values of $n$ give better fits, therefore the larger
    the value of $n$ we use, the less impressive a good fit (i.e., low
    deviation between the polynomial and the actual set of
    observations) should be.
  \item Smaller values of $n$ give better fits, therefore the smaller
    the value of $n$ we use, the less impressive a good fit (i.e., low
    deviation between the polynomial and the actual set of
    observations) should be.
  \item The value of $n$ we use for trying to get a good fit is
    irrelevant. A good fit is a good fit, regardless of the type of
    function used.
  \end{enumerate}

  {\em Answer}: Option (C)

  {\em Explanation}: Larger values of $n$ mean more parameters, and we
  can use more parameters to get a better fit, generally
  speaking. However, that better fit may well be {\em fitting the
    noise in the measurements rather than the signal}. Even without
  measurement error, if we do not have {\em a priori} theoretical
  reasons to be sure of the model, it may just be fitting
  idiosyncracies of the observed values that do not extend to other
  values. For both these reasons, using too many parameters gives a
  misleading sense of complacency based on what appears like a good
  fit, but is a result of sheer chance.

  For instance, {\em any} collection of $n$ data points can be fitted
  (without need for error tolerance!) on a polynomial of degree at
  most $n - 1$. But what if there were actually measurement error?
  Then that polynomial of degree $n - 1$ would be a fake good
  fit. Imagine that the actual function is $f(x) = x$, but we are
  measuring values near $0$ and the best fit for the measured values
  turns out to be $f(x) = x + x^3$. This may very well be a much
  better fit for values close to the origin because of biased
  measurement errors, but extrapolating it to a larger domain could go
  really awry.

  {\em Performance review}: 15 out of 28 got this. 9 chose (E), 2
  chose (B), 1 chose (A), and 1 did not attempt the question.

  {\em Historical note (last time)}: $14$ out of $28$ people got this. $9$
  chose (D), $4$ chose (E), and $1$ chose (A).

\item (*) $F$ is an affine linear function of two variables $x$ and
  $y$, i.e., it has the form $F(x,y) := ax + by + c$ with $a$, $b$,
  and $c$ real numbers. We want to determine the values of the
  parameters $a$, $b$, and $c$ by using input-output pairs. It is,
  however, costly to find input-output pairs. We have already found
  $F(1,3)$ and $F(3,7)$. We want to find $F$ for one other pair of
  inputs to determine $a$, $b$, and $c$. Which of these will {\em not}
  be a good choice?

  \begin{enumerate}[(A)]
  \item $F(2,2)$, i.e., the input $x = 2$, $y = 2$
  \item $F(2,3)$, i.e., the input $x = 2$, $y = 3$
  \item $F(2,4)$, i.e., the input $x = 2$, $y = 4$
  \item $F(2,5)$, i.e., the input $x = 2$, $y = 5$
  \item $F(2,6)$, i.e., the input $x = 2$, $y = 6$
  \end{enumerate}

  {\em Answer}: Option (D)

  {\em Explanation}: The point $(2,5)$ is collinear with the points
  $(1,3)$ and $(3,7)$ (in fact, it is their midpoint) so the value at
  that point can be predicted based on the values at $(1,3)$ and
  $(3,7)$ as being the arithmetic mean between these values. Thus, it
  does not provide new information. Mathematically, if we use this
  point to get the third equation, that equation will be redundant
  with the existing equations. In equational form:

  $$F(2,5) = \frac{F(1,3) + F(3,7)}{2}$$

  {\em Purely arithmetic version of observation}: We can obtain this
  observation computationally even without explicitly noting the
  observation about the midpoint.  Explicitly, we have:

  \begin{eqnarray*}
    F(1,3) & = & a + 3b + c \\
    F(3,7) & = & 3a + 7b + c\\
  \end{eqnarray*}

  Adding, we get:

  $$F(1,3) + F(3,7) = 4a + 10b + 2c = 2(2a + 5b + c) = 2F(2,5)$$

  Thus, we get:

  $$F(2,5) = \frac{F(1,3) + F(3,7)}{2}$$

  {\em Alternative geometric explanation}: Think of the inputs as
  living in the $xy$-plane, and the output axis as the $z$-axis. The
  graph $z = F(x,y) = ax + by + c$ gives a plane in the
  three-dimensional space with coordinates $x,y,z$. We know that a
  plane is determined by knowing three non-collinear points on it. The
  points are of the form $(x,y,F(x,y))$ where $x$ and $y$ vary
  freely. The graph is a plane {\em because} $F$ is a linear
  function. In general, the graph would be a surface.

  The inputs $(1,3)$, $(2,5)$, and $(3,7)$ being collinear, along with
  the fact that $F$ is affine linear, tells us that the triples
  $(1,3,F(1,3))$, $(2,5,F(2,5))$, and $(3,7,F(3,7))$ are collinear in
  three-dimensional space. In fact, they lie on the line obtained by
  intersecting the plane that is the graph of $F$ with the plane
  parallel to the z-axis whose intersection with the xy-plane passes
  through the points $(1,3)$ and $(3,7)$ (explicitly, this is the
  plane with equation $y = 2x + 1$). In this particular case, since
  $(2,5)$ is the midpoint between the points $(1,3)$ and $(3,7)$, the
  point $(2,5,F(2,5))$ is the midpoint between the points
  $(1,3,F(1,3))$ and $F(3,7,F(3,7))$.

  Thus, if we use these three input-output pairs, then we get three
  {\em collinear} points in the plane we are trying to find, and we
  cannot determine the plane uniquely (any plane through the line
  joining the points works). If we chose an input $(x_0,y_0)$ that was
  not collinear with the points $(1,3)$ and $(3,7)$, we would get a
  point $(x_0,y_0,F(x_0,y_0))$ that was not collinear with the points
  $(1,3,F(1,3))$ and $(3,7,F(3,7))$, and therefore, the plane would be
  determined uniquely.

  {\em Update for after understanding coefficient matrices and ranks}:
  When trying to find the parameters $a$, $b$, and $c$, we need to set
  up a system of simultaneous linear equations. Each input-output pair
  gives one equation, and therefore, one row of the augmenting
  matrix. The augmenting column corresponds to the outputs, and the
  coefficient matrix part corresponds to the inputs. In other words,
  each input determines a row of the coefficient matrix.

  For an input $(x_i,y_i)$, the corresponding equation is:

  $$ax_i + by_i + c = F(x_i,y_i)$$

  We are now viewing this as an equation in the variables $a$, $b$,
  and $c$. The row of the coefficient matrix when we set up the linear
  system in terms of the parameters is:

  $$\left[\begin{matrix} x_i & y_i & 1 \\\end{matrix}\right]$$

  For our system with three input-output pairs, the coefficient matrix
  becomes:

  $$\left[\begin{matrix} 1 & 3 & 1 \\ 3 & 7 & 1 \\ x & y & 1 \\\end{matrix} \right]$$
  
  where $(x,y)$ is the third input that we choose. The input that
  would be bad to choose is the input for which the coefficient matrix
  has rank two rather than the expected rank of three. We can see that
  $x = 2$, $y = 5$ is the only such input. In other words, the matrix:

  $$\left[\begin{matrix} 1 & 3 & 1 \\ 3 & 7 & 1 \\ 2 & 5 & 1 \\\end{matrix} \right]$$

  has rank two. We can verify this by carrying out Gauss-Jordan
  elimination (the row reduction process) and obtaining that the last
  column is all zeros. On the other hand, the coefficient matrices for
  all the other choices of the third input have full rank three.

  {\em Performance review}: 6 out of
  28 got this. 10 chose (B), 7 chose (E), 3 chose (A), 2 chose (C).

  {\em Historical note (last time)}: $16$ out of $28$ got this. $9$ chose (E),
  $2$ chose (B), $1$ chose (C).
\end{enumerate}

\end{document}
