\documentclass[10pt]{amsart}

%Packages in use
\usepackage{fullpage, hyperref, vipul, enumerate}

%Title details
\title{Take-home class quiz solutions: due Monday December 2: Similarity of linear transformations (applied)}
\author{Math 196, Section 57 (Vipul Naik)}
%List of new commands

\begin{document}
\maketitle

\section{Performance review}

25 people took this 16-question quiz. The score distribution was as follows:

\begin{itemize}
\item Score of 4: 3 people
\item Score of 5: 4 people
\item Score of 6: 3 people
\item Score of 8: 4 people
\item Score of 9: 4 people
\item Score of 10: 4 people
\item Score of 12: 3 people
\end{itemize}

The mean score was 7.76.

The question-wise answers and performance review were as follows:

\begin{enumerate}
\item Option (A): 15 people
\item Option (B): 16 people
\item Option (A): 15 people
\item Option (B): 20 people
\item Option (D): 10 people
\item Option (B): 12 people%$9$ people
\item Option (A): 3 people%$2$ people
\item Option (B): 19 people%$9$ people
\item Option (B): 15 people%$4$ people
\item Option (B): 6 people%$4$ people
\item Option (C): 10 people%$6$ people
\item Option (E): 8 people%$4$ people
\item Option (D): 6 people%$4$ people
\item Option (C): 11 people%$8$ people
\item Option (D): 16 people%$8$ people
\item Option (A): 12 people%$7$ people
\end{enumerate}

\section{Solutions}

{\bf PLEASE FEEL FREE TO DISCUSS {\em ALL} QUESTIONS.}

This quiz corresponds to material discussed in the lecture notes
titled {\tt Coordinates}. It also corresponds to Section 3.4 of the
text.

Recall that $n \times n$ matrices $A$ and $B$ are termed {\em similar}
if there exists an invertible $n \times n$ matrix $S$ such that $A =
SBS^{-1}$. The relation of matrices being similar is an {\em
  equivalence relation}.

Recall that $n \times n$ matrices $A$ and $B$ are termed {\em
  quasi-similar} if there exist $n \times n$ matrices $C$ and $D$ such
that $A = CD$ and $B = DC$. Recall that similar matrices are always
quasi-similar, but quasi-similar matrices need not be
similar. However, for {\em invertible} matrices, similarity and
quasi-similarity are equivalent.

Also, note that if $A$ and $B$ are quasi-similar matrices, then $A$
and $B$ have the same trace. However, the converse is not true: it is
possible to have two matrices $A$ and $B$ that have the same trace but
are not quasi-similar.

For these questions, assume $n > 1$, because a lot of phenomena are
much simpler in the case $n = 1$ and you may be misled if you look
only at that case.

Note also that the trace of a square matrix is defined as the sum of
its diagonal entries.

The {\em determinant} of a $2 \times 2$ matrix, denoted $\det$, is
defined as follows:

$$\det \left[\begin{matrix} a & b \\ c & d \\\end{matrix}\right] = ad - bc$$

The following are some important facts about the determinant:

\begin{itemize}
\item The determinant of a $2 \times 2$ diagonal matrix is the product
  of the diagonal entries.
\item The determinant of a $2 \times 2$ matrix is zero if and only
  if the matrix is non-invertible.
\item The determinant of the product of two $2 \times 2$ matrices is
  the product of the determinants.
\item The determinant of the inverse of an invertible $2 \times 2$
  matrix is the reciprocal of the determinant.
\item If $A$ and $B$ are similar $2 \times 2$ matrices, they have
  the same determinant.
\item If $A$ and $B$ are quasi-similar $2 \times 2$ matrices, they
  have the same determinant.
\item If the determinant of $A$ is positive, then the linear
  transformation given by $A$ is an orientation-preserving linear
  automorphism of $\R^2$.
\item If the determinant of $A$ is negative, then the linear
  transformation given by $A$ is an orientation-reversing linear
  automorphism of $\R^2$.
\end{itemize}


\begin{enumerate}

\item Suppose $A$ and $B$ are both $n \times n$ matrices (but they are
  not given to be similar). Denote by $I_n$ the $n \times n$ identity
  matrix. Which of the following holds?

  \begin{enumerate}[(A)]
  \item $A$ is similar to $B$ if and only if $A - I_n$ is similar to
    $B - I_n$.
  \item If $A$ is similar to $B$, then $A - I_n$ is similar to $B -
    I_n$. However, $A - I_n$ being similar to $B - I_n$ does not imply
    that $A$ is similar to $B$.
  \item If $A - I_n$ is similar to $B - I_n$, then $A$ is similar to
    $B$. However, $A$ being similar to $B$ does not imply that $A -
    I_n$ is similar to $B - I_n$.
  \item $A$ being similar to $B$ does not imply that $A - I_n$ is
    similar to $B - I_n$. Also, $A - I_n$ being similar to $B - I_n$
    does not imply that $A$ is similar to $B$.
  \end{enumerate}

  {\em Answer}: Option (A)

  {\em Explanation}: We have that for any invertible matrix $S$, $S(B
  - I_n)S^{-1} = (SBS^{-1} - SI_nS^{-1} = SBS^{-1} - I_n$. In other
  words, if $A = SBS^{-1}$, then $A - I_n = S(B -
  I_n)S^{-1}$. Conversely, if $A - I_n = S(B - I_n)S^{-1}$, then $A =
  SBS^{-1}$. Thus, $A$ is similar to $B$ if and only if $A - I_n$ is
  similar to $B - I_n$, and the matrix used for similarity is the same in
  both cases.

  {\em Performance review}: 15 out of 25 got this. 7 chose (D), 2
  chose (C), 1 chose (B).

  \vspace{0.6in}

  Suppose $f$ is a polynomial of degree $r$ in one variable with real
  coefficients. For a $n \times n$ matrix $X$, we denote by $f(X)$ we
  mean the matrix we get by applying the polynomial to $f$, where
  constant terms are interpreted as scalar matrices. For instance, if
  $f(x) = x^2 + 3x + 5$, then $f(X) = X^2 + 3X + 5I_n$.

\item Suppose $A$ and $B$ are both $n \times n$ matrices (but they are
  not given to be similar). Suppose $f$ is a polynomial of degree $r$
  in one variable, where $r \ge 2$. Which of the following holds?

  \begin{enumerate}[(A)]
  \item $A$ is similar to $B$ if and only if $f(A)$ is similar to
    $f(B)$.
  \item If $A$ is similar to $B$, then $f(A)$ is similar to
    $f(B)$. However, $f(A)$ being similar to $f(B)$ does not
    imply that $A$ is similar to $B$.
  \item If $f(A)$ is similar to $f(B)$, then $A$ is similar to
    $B$. However, $A$ being similar to $B$ does not imply that $f(A)$
    is similar to $f(B)$.
  \item $A$ being similar to $B$ does not imply that $f(A)$ is
    similar to $f(B)$. Also, $f(A)$ being similar to $f(B)$
    does not imply that $A$ is similar to $B$.
  \end{enumerate}

  {\em Answer}: Option (B)

  {\em Explanation}: For the forward direction, note that $A =
  SBS^{-1}$ implies that $f(A) = Sf(B)S^{-1}$. For the breakdown of
  the reverse direction, see the explanation for Q8 of the November 27
  quiz. This covers the $f(x) = x^2$ case. Similar examples can be
  constructed for other polynomials.

  {\em Performance review}: 16 out of 25 got this. 4 chose (D), 3
  chose (A), 2 chose (C).
\item Suppose $A$ and $B$ are both $n \times n$ matrices (but they are
  not given to be similar). Suppose $f$ is a polynomial of degree $r$
  in one variable, where $r = 1$. Which of the following holds?

  \begin{enumerate}[(A)]
  \item $A$ is similar to $B$ if and only if $f(A)$ is similar to
    $f(B)$.
  \item If $A$ is similar to $B$, then $f(A)$ is similar to
    $f(B)$. However, $f(A)$ being similar to $f(B)$ does not
    imply that $A$ is similar to $B$.
  \item If $f(A)$ is similar to $f(B)$, then $A$ is similar to
    $B$. However, $A$ being similar to $B$ does not imply that $f(A)$
    is similar to $f(B)$.
  \item $A$ being similar to $B$ does not imply that $f(A)$ is
    similar to $f(B)$. Also, $f(A)$ being similar to $f(B)$
    does not imply that $A$ is similar to $B$.
  \end{enumerate}

  {\em Answer}: Option (A)

  {\em Explanation}: Degree one polynomials differ from higher degree
  polynomials in that we can recover the original matrix from a degree
  one polynomial of it. Thus, we can deduce that $A = SBS^{-1}$ if and
  only if $f(A) = Sf(B)S^{-1}$. Special cases of this were covered in
  Question 1 of this quiz and Questions 5 and 6 of the November 27
  quiz.

  {\em Performance review}: 15 out of 25 got this. 4 chose (D), 3 each
  chose (B) and (C).

  \vspace{0.6in}

Suppose $p$ and $q$ are real numbers (possibly equal, possibly distinct). The diagonal matrices:

$$A = \left[\begin{matrix} p & 0 \\ 0 & q \\\end{matrix}\right]$$

and

$$B = \left[\begin{matrix} q & 0 \\ 0 & p \\\end{matrix}\right]$$

are similar. Explicitly, the two matrices are similar under the
change-of-basis transformation that interchanges the coordinates,
i.e., if we set:

$$S = \left[\begin{matrix} 0 & 1 \\ 1 & 0 \\\end{matrix}\right]$$

then:

$$S = S^{-1}$$

and we have:

$$B = S^{-1}AS$$

Moreover, the only diagonal matrices similar to $A$ are $A$ and $B$
(in the special case that $p = q$, we get $A = B$ is a scalar matrix,
so $A$ is the only diagonal matrix similar to $A$).

\item What is the necessary and sufficient condition on $p$ and $q$
  such that the matrix $A = \left[\begin{matrix} p & 0 \\ 0 & q
      \\\end{matrix}\right]$ is similar to $-A$?

  \begin{enumerate}[(A)]
  \item $p = q$
  \item $p = -q$
  \item $p = 1/q$
  \item $p = -1/q$
  \item $p + q = 1$
  \end{enumerate}

  {\em Answer}: Option (B)

  {\em Explanation}: We have:

  $$-A = \left[\begin{matrix} -p & 0 \\ 0 & -q \\\end{matrix}\right]$$

  Now, for $A$ to be similar to $-A$, we have one of these two conditions:

  $$-A = A \text{ or } -A = \left[\begin{matrix} q & 0 \\ 0 & p \\\end{matrix}\right]$$

  The first case ($-A = A$) gives us that $p = q = 0$, so that $A$ is
  the zero matrix. The second case gives us that $-p = q$ and $-q =
  p$. Both of these are equivalent to $p = -q$. We now notice that the
  first case $p = q = 0$ is subsumed within the second case, so that
  $p = -q$ describes the necessary and sufficient condition.

  {\em Performance review}: 20 out of 25 got this. 2 each chose (A)
  and (D), 1 chose (C).

\item Which of the following is a necessary and sufficient condition
  on $p$ and $q$ so that the matrix $A = \left[\begin{matrix} p & 0
      \\ 0 & q \\\end{matrix}\right]$ is invertible and similar to
  $-A^{-1}$?

  \begin{enumerate}[(A)]
  \item $p = q$
  \item $p = -q$
  \item $p = 1/q$
  \item $p = -1/q$
  \item $p + q = 1$
  \end{enumerate}

  {\em Answer}: Option (D)

  {\em Explanation}: We have:

  $$-A^{-1} = \left[\begin{matrix} -1/p & 0 \\ 0 & -1/q \\\end{matrix}\right]$$

  For this to be similar to $A$, we must have:

  $$-A^{-1} = A \text{ or } -A^{-1} = \left[\begin{matrix} q & 0 \\ 0 & p \\\end{matrix}\right]$$

  The first case gives $-1/p = p$ and $-1/q = q$, solving to $p^2 =
  -1$ and $q^2 = - 1$, which is not possible. Thus, the first case is
  ruled out.

  This brings us to the second case. In this case, $-1/p = q$ and
  $-1/q = p$. Both of these are equivalent to $p = -1/q$. This is the
  correct answer.

  {\em Performance review}: 10 out of 25 got this. 9 chose (B), 5
  chose (C), 1 chose (E).

  \vspace{0.6in}
  Consider the matrix:

  $$S = \left[\begin{matrix} 0 & 1 \\ 1 & 0 \\\end{matrix}\right]$$

  used above. We have $S = S^{-1}$. For a general matrix:

  $$A = \left[\begin{matrix} a & b \\ c & d \\\end{matrix}\right]$$

  we have:

  $$S^{-1}AS = \left[\begin{matrix} d & c \\ b & a \\\end{matrix}\right]$$

  In other words, it swaps the rows {\em and} swaps the columns. This
  observation may be useful for some of the following questions.

\item For an angle $\theta$ with $-\pi \le \theta \le \pi$, the rotation
  matrix for $\theta$ is given as:

  $$R(\theta) = \left[\begin{matrix} \cos \theta & -\sin \theta \\ \sin \theta & \cos \theta \\\end{matrix}\right]$$

  Note that $R(-\pi) = R(\pi)$, but other than that equality, all the
  $R(\theta)$s are distinct.

  Which of these describes the relation between the rotation matrices
  for different values of $\theta$?

  \begin{enumerate}[(A)]
  \item All the rotation matrices $R(\theta)$, $-\pi < \theta \le
    \pi$, are similar to each other.
  \item The rotation matrix $R(\theta)$ is similar to itself and to
    the rotation matrix $R(-\theta)$. However, it is not in general
    similar to any other rotation matrix.
  \item No two different rotation matrices are similar.
  \item The rotation matrix $R(\theta)$ is similar to itself and to
    the rotation matrix $R(\pi -\theta)$ (or $R(-\pi - \theta)$,
    depending on which of the two angles lies within the specified
    range). However, it is not in general similar to any other
    rotation matrix.
  \end{enumerate}

  {\em Answer}: Option (B)

  {\em Explanation}: The matrix we can use for the similarity
  transformation is the following self-inverse matrix:

  $$\left[\begin{matrix} 1 & 0 \\ 0 & -1 \\\end{matrix}\right]$$

  This matrix essentially performs a reflection about the $x$-axis,
  and the net effect is to negate the angle of rotation. Explicitly:

  $$\left[\begin{matrix} 1 & 0 \\ 0 & -1 \\\end{matrix}\right]\left[\begin{matrix} \cos \theta & -\sin \theta \\ \sin \theta & \cos \theta \\\end{matrix}\right]\left[\begin{matrix} 1 & 0 \\ 0 & -1 \\\end{matrix}\right] = \left[\begin{matrix} \cos \theta & \sin \theta \\ -\sin \theta & \cos \theta \\\end{matrix}\right]$$

  {\em Performance review}: 12 out of 25 got this. 9 chose (D), 4 chose (A).

  {\em Historical note (last time)}: $9$ out of $19$ got this. $4$ each chose
  (A) and (D), $2$ chose (C).

\item Consider the linear automorphisms of $\R^2$ that are given as
  {\em reflections} about lines in $\R^2$ through the origin. (Note
  that we need the line of reflection to pass through the origin for
  the automorphism to be {\em linear} rather than merely being {\em
    affine linear}). Which of these describes the relation between
  reflection matrices for different possible lines of reflection
  through the origin?

  \begin{enumerate}[(A)]
  \item All the reflection matrices are similar to each other.
  \item No two reflection matrices for different lines of reflection are similar.
  \item The reflection matrices for two different lines of reflection
    are similar if and only if the lines of reflection are
    perpendicular.
  \item The reflection matrices for two different lines of reflection
    are similar if and only if the lines are reflection make an angle
    that is a rational multiple of $\pi$.
  \end{enumerate}

  {\em Answer}: Option (A)

  {\em Explanation}: The rotation matrix that rotates one line to the
  other can be used as the matrix for similarity.

  {\em Performance review}: 3 out of 25 got this. 12 chose (C), 6
  chose (B), 4 chose (D).

  {\em Historical note (last time)}: $2$ out of $19$ people got this. $12$
  chose (C), $4$ chose (D), $1$ chose (B).

\item Suppose $m$ and $n$ are positive integers with $m < n$. Denote
  by $P_m$ the ``orthogonal projection onto the first $m$
  coordinates'' linear transformation from $\R^n$ to $\R^n$, defined
  as follows. This takes as input a $n$-dimensional vector, sends
  each of the first $m$ coordinates to itself, and sends the remaining
  coordinates to zero. What is the trace of the matrix of $P_m$?

  \begin{enumerate}[(A)]
  \item $1$
  \item $m$
  \item $n$
  \item $n - m$
  \item $m - n$
  \end{enumerate}

  {\em Answer}: Option (B)

  {\em Explanation}: The matrix is diagonal with the first $m$
  diagonal entries equal to $1$ and the remaining $n - m$ diagonal
  entries equal to $0$. The trace is thus $m$.

  {\em Performance review}: 19 out of 25 got this. 3 each chose (D)
  and (E).

  {\em Historical note (last time)}: $9$ out of $19$ got this. $5$ chose (D),
  $4$ chose (E), $1$ chose (C).

\item It is a fact that if $A,B$ are $n \times n$ matrices that
  describe orthogonal projections onto (possibly different)
  $m$-dimensional subspaces of $\R^n$, then $A$ and $B$ are
  similar. What can we say must be the trace of an orthogonal
  projection onto any $m$-dimensional subspace of $\R^n$?

  \begin{enumerate}[(A)]
  \item $1$
  \item $m$
  \item $n$
  \item $n - m$
  \item $m - n$
  \end{enumerate}

  {\em Answer}: Option (B)

  {\em Explanation}: This follows from the preceding question.

  {\em Performance review}: 15 out of 25 got this. 4 each chose (A)
  and (D), 2 chose (C).

  {\em Historical note (last time)}: $4$ out of $19$ got this. $6$ chose (C),
  $5$ chose (D), $4$ chose (E).

\item Suppose $A$, $B$ and $C$ are $n \times n$ matrices. Which of the
  following matrices is {\em not} guaranteed (based on the given
  information) to have the same trace as the product $ABC$? Please see
  (and read carefully) Options (D) and (E) before answering.

  \begin{enumerate}[(A)]
  \item $BCA$
  \item $CBA$
  \item $CAB$
  \item None the above, i.e., they are all guaranteed to have the same
    trace as $ABC$.
  \item All of the above, i.e., none of them is guaranteed to have the
    same trace as $ABC$.
  \end{enumerate}

  {\em Answer}: Option (B)

  {\em Explanation}: First, let's note that the other two options
  don't work:

  \begin{itemize}
  \item Option (A): $ABC = A(BC)$ whereas $BCA = (BC)A$, so $ABC$
    and $BCA$ are quasi-similar matrices. Therefore, they have the same trace.
  \item Option (C): $ABC = (AB)C$ and $CAB = C(AB)$, so $ABC$ and
    $CAB$ are quasi-similar matrices. Therefore, they have the same
    trace.
  \end{itemize}

  Consider the example:

  $$A = \left[\begin{matrix} 0 & 0 \\ 1 & 0 \\\end{matrix}\right], B = \left[\begin{matrix} 0 & 0 \\ 0 & 1 \\\end{matrix}\right], C = \left[\begin{matrix} 0 & 1 \\ 1 & 0 \\\end{matrix}\right]$$

  The product $ABC$ is the zero matrix, because $AB$ is the zero
  matrix. Thus, $ABC$ has trace zero. The product $CBA$, on the other
  hand, is the matrix:

  $$\left[\begin{matrix} 1 & 0 \\ 0 & 0 \\\end{matrix}\right]$$

  {\em Performance review}: 6 out of 25 got this. 11 chose (D), 7
  chose (E), 1 chose (C).

  {\em Historical note (last time)}: $4$ out of $19$ got this. $8$ chose (D),
  $5$ chose (E), $1$ each chose (A) and (C).

\item Which of the following gives a pair of matrices $A$ and $B$ that
  have the same trace as each other {\em and} the same determinant as
  each other, but that are {\em not} similar to each other?

  \begin{enumerate}[(A)]
  \item $A = \left[\begin{matrix} 1 & 0 \\ 0 & 2 \\\end{matrix}\right], B = \left[\begin{matrix} 2 & 0 \\ 0 & 1 \\\end{matrix}\right]$
  \item $A = \left[\begin{matrix} 1 & 0 \\ 0 & -1 \\\end{matrix}\right], B = \left[\begin{matrix} 0 & 1 \\ 1 & 0 \\\end{matrix}\right]$
  \item $A = \left[\begin{matrix} 1 & 0 \\ 0 & 1 \\\end{matrix}\right], B = \left[\begin{matrix} 1 & 1 \\ 0 & 1 \\\end{matrix}\right]$
  \item $A = \left[\begin{matrix} 1 & 1 \\ 0 & 1 \\\end{matrix}\right], B = \left[\begin{matrix} 1 & 0 \\ 1 & 1 \\\end{matrix}\right]$
  \item $A = \left[\begin{matrix} 1 & 1 \\ 0 & 1 \\\end{matrix}\right], B = \left[\begin{matrix} 1 & 2 \\ 0 & 1 \\\end{matrix}\right]$
  \end{enumerate}

  {\em Answer}: Option (C)

  {\em Explanation}: The identity matrix is not similar to any
  non-identity matrix, because it is scalar, so conjugating it by
  anything leaves it as it is.

  All other pairs of matrices are in fact similar:

  \begin{itemize}
  \item Option (A): We can use the coordinate interchange matrix
    $\left[\begin{matrix} 0 & 1 \\ 1 & 0 \\\end{matrix}\right]$.
  \item Option (B): We can use the conjugating matrix $\left[\begin{matrix} 1 & 1 \\ 1 & -1 \\\end{matrix}\right]$.
  \item Option (D) We can use the coordinate interchange matrix $\left[\begin{matrix} 0 & 1 \\ 1 & 0 \\\end{matrix}\right]$.
  \item Option (E): Use $\left[\begin{matrix} 1 & 0 \\ 0 & 2
      \\\end{matrix}\right]$.
  \end{itemize}

  {\em Performance review}: 10 out of 25 got this. 7 chose (B), 5
  chose (E), 2 chose (D), 1 left the question blank.

  {\em Historical note (last time)}: $6$ out of $19$ got this. $5$ chose (B),
  $3$ each chose (D) and (E), and $2$ chose (A).

\item Suppose $A$ and $B$ are $2 \times 2$ matrices. Which of the
  following correctly describes the relation between $\det A$, $\det
  B$, and $\det (A + B)$? Please see Option (E) before answering.

  \begin{enumerate}[(A)]
  \item $\det(A + B) = \det A + \det B$
  \item $\det(A + B) \le \det A + \det B$, but equality need not
    necessarily hold.
  \item $\det(A + B) \ge \det A + \det B$, but equality need not
    necessarily hold.
  \item $|\det(A + B)| \le |\det A| + |\det B|$, but equality need not
    necessarily hold.
  \item None of the above.
  \end{enumerate}

  {\em Answer}: Option (E)

  {\em Explanation}: The determinant does not interact in any
  meaningful manner with addition. In fact, for any (possibly equal,
  possibly distinct) real numbers $p$, $q$, and $r$, we can construct
  matrices $A$ and $B$ such that $\det A = p$, $\det B = q$, $\det (A
  + B) = r$. We can in fact just use $2 \times 2$ matrices to achieve
  this. The general description is tricky, so let's just give
  counter-examples for each part:

  \begin{itemize}
  \item Options (A), (B) and (D): Consider $A = \left[\begin{matrix} 1 & 0
      \\ 0 & 0 \\\end{matrix}\right], B = \left[\begin{matrix} 0 & 0
      \\ 0 & 1 \\\end{matrix}\right]$. Note that $\det A = \det B =
    0$, but $\det(A + B) = 1$.
  \item Option (C): Consider $A = \left[\begin{matrix} 1 & 0 \\ 0 & 1
      \\\end{matrix}\right]$ and $B = \left[\begin{matrix} -1 & 0 \\ 0
      & -1 \\\end{matrix}\right]$. Then, $\det A = \det B = 1$, but
    $\det(A + B) = 0$.
  \end{itemize}

  {\em Performance review}: 8 out of 25 got this. 7 chose (C), 5 chose
  (B), 3 chose (D), 2 chose (A).

  {\em Historical note (last time)}: $4$ out of $19$ got this. $5$ chose (B),
  $4$ chose (C), $3$ each chose (A) and (D).

  \vspace{0.6in}

  Let $n$ be a natural number greater than $1$. Suppose $f: \{
  0,1,2,\dots,n\} \to \{ 0,1,2,\dots,n\}$ is a function satisfying $f(0)
  = 0$. Let $T_f$ denote the linear transformation from $\R^n$ to $\R^n$
  satisfying the following for all $i \in \{ 1,2,\dots,n\}$:
  
  $$T_f(\vec{e}_i) = \left \lbrace \begin{array}{rl} \vec{e}_{f(i)}, & f(i) \ne 0\\ \vec{0}, & f(i) = 0\\\end{array}\right.$$
    
    Let $M_f$ denote the matrix for the linear transformation $T_f$. $M_f$
    can be described explicitly as follows: the $i^{th}$ column of $M_f$
    is $\vec{0}$ if $f(i) = 0$ and is $\vec{e}_{f(i)}$ if $f(i) \ne 0$.
    
    Note that if $f,g: \{ 0,1,2,\dots,n \} \to \{ 0,1,2,\dots,n\}$ are
    functions with $f(0)= g(0) = 0$, then $M_{f \circ g} = M_fM_g$ and
    $T_{f \circ g} = T_f \circ T_g$.

    For the following questions, the discussion prior to Question 3
    might be helpful. Note, however, that while that discussion gives
    one possible candidate for the matrix $S$ of the similarity
    transformation, it is not the only possible candidate. For some
    but not all of the following questions, in the case that two
    matrices are similar, the matrix $S$ described there works. In the
    case that they are not similar, the lack of similarity can be
    inferred from the traces not being equal, or from the determinants
    not being equal.
    
\item $n = 2$ for this question. For the following three functions
  $f$, $g$, and $h$, consider the corresponding matrices
  $M_f,M_g,M_h$. Either two of them are similar and the third is not
  similar to either (in which case, select the matrix that is not
  similar to the other two), or all three are similar (if so, select
  Option (D)), or no two are similar (if so, select Option (E)).

  \begin{enumerate}[(A)]
  \item $f(0) = 0, f(1) = 1, f(2) = 0$
  \item $g(0) = 0, g(1) = 0, g(2) = 2$
  \item $h(0) = 0, h(1) = 1, h(2) = 1$
  \item All the above give similar matrices.
  \item No two of the corresponding matrices are similar.
  \end{enumerate}

  {\em Answer}: Option (D)

  {\em Explanation}: All three matrices are similar. Here is what the
  matrices look like:

  \begin{itemize}
  \item Option (A): $M_f = \left[\begin{matrix} 1 & 0 \\ 0 & 0 \\\end{matrix}\right]$
  \item Option (B): $M_g = \left[\begin{matrix} 0 & 0 \\ 0 & 1 \\\end{matrix}\right]$
  \item Option (C): $M_h =  \left[\begin{matrix} 1 & 1 \\ 0 & 0 \\\end{matrix}\right]$
  \end{itemize}

  Clearly, they all have trace $1$, rank $1$, and determinant
  $0$. Thus, we cannot {\em prima facie} rule out the possibility of
  their being similar. But to actually confirm that they are similar,
  it would help to demonstrate a matrix that accomplishes the
  similarity transformation.

  The similarity of Options (A) and (B) is relatively easy: the two
  options are related in that they have interchanged roles of the
  first and second vector relative to each other. Thus, the following
  matrix works well for accomplishing similarlity:

  $$S = \left[\begin{matrix} 0 & 1 \\ 1 & 0 \\\end{matrix}\right]$$

  Explicitly, $M_f = SM_gS^{-1}$. Note that $S^{-1} = S$.

  The similarity between $M_f$ and $M_h$ is trickier. If we set:

  $$X = \left[\begin{matrix} 1 & 1 \\ 0 & 1 \\\end{matrix}\right]$$

  Then:

  $$X^{-1} = \left[\begin{matrix} 1 & -1 \\ 0 & 1 \\\end{matrix}\right]$$

  Then, $M_f = XM_hX^{-1}$.

  Now that we have shown the similarity of $M_f$ with $M_g$ and the
  similarity of $M_g$ with $M_h$, the fact that similarity is an
  equivalence relation tells us that all three matrices are similar.

  {\em Performance review}: 6 out of 25 got this. 10 chose (C), 4
  chose (E), 3 chose (B), 2 chose (A).

  {\em Historical note (last time)}: $4$ out of $19$ got this. $10$ chose (C),
  $3$ chose (A), $1$ each chose (A) and (E).

\item $n = 2$ for this question. For the following three functions
  $f$, $g$, and $h$, consider the corresponding matrices
  $M_f,M_g,M_h$. Either two of them are similar and the third is not
  similar to either (in which case, select the matrix that is not
  similar to the other two), or all three are similar (if so, select
  Option (D)), or no two are similar (if so, select Option (E)).

  \begin{enumerate}[(A)]
  \item $f(0) = 0, f(1) = 0, f(2) = 1$
  \item $g(0) = 0, g(1) = 2, g(2) = 0$
  \item $h(0) = 0, h(1) = 2, h(2) = 1$
  \item All the above give similar matrices.
  \item No two of the corresponding matrices are similar.
  \end{enumerate}

  {\em Answer}: Option (C)

  {\em Explanation}: The three matrices are:

  \begin{itemize}
  \item Option (A): $M_f = \left[\begin{matrix} 0 & 1 \\ 0 & 0
      \\\end{matrix}\right]$
  \item Option (B): $M_g = \left[\begin{matrix} 0 & 0 \\ 1 & 0
      \\\end{matrix} \right]$
  \item Option (C): $M_h = \left[\begin{matrix} 0 & 1 \\ 1 & 0
      \\\end{matrix}\right]$
  \end{itemize}

  $M_f$ and $M_g$ are similar. In fact, we can use the matrix $M_h$ itself:

  $$M_f = M_hM_gM_h^{-1}$$

  $M_h$ is not similar to either $M_f$ or $M_g$. We can see this, for
  instance, by noting that $M_h$ has full rank, but neither $M_f$ nor
  $M_g$ do. Alternatively, note that $M_h$ has determinant $-1$,
  unlike both $M_f$ and $M_g$, that have determinant $0$.

  {\em Performance review}: 11 out of 25 got this. 5 chose (D), 4
  chose (E), 3 chose (B), 2 chose (A).

  {\em Historical note (last time)}: $8$ out of $19$ got this. $4$ each chose
  (B) and (D), $3$ chose (E).


\item $n = 3$ for this question. For the following three functions
  $f$, $g$, and $h$, consider the corresponding matrices
  $M_f,M_g,M_h$. Either two of them are similar and the third is not
  similar to either (in which case, select the matrix that is not
  similar to the other two), or all three are similar (if so, select
  Option (D)), or no two are similar (if so, select Option (E)).

  \begin{enumerate}[(A)]
  \item $f(0) = 0, f(1) = 2, f(2) = 1, f(3) = 3$
  \item $g(0) = 0, g(1) = 1, g(2) = 3, g(3) = 2$
  \item $h(0) = 0, h(1) = 3, h(2) = 2, h(3) = 1$
  \item All the above give similar matrices.
  \item No two of the corresponding matrices are similar.
  \end{enumerate}

  {\em Answer}: Option (D)

  {\em Explanation}: In each case, the matrix interchanges two
  coordinates and leaves the third coordinate as is. {\em Which} two
  coordinates get interchanged just depends on how we label, and
  therefore, all the matrices are similar. The explicit descriptions
  are below:

  \begin{itemize}
  \item Option (A): $M_f = \left[\begin{matrix} 0 & 1 & 0 \\ 1  & 0 & 0\\ 0 & 0 & 1 \\\end{matrix}\right]$
  \item Option (B): $M_g = \left[\begin{matrix} 1 & 0 & 0 \\ 0 & 0 & 1 \\ 0 & 1 & 0 \\\end{matrix}\right]$
  \item Option (C): $M_h = \left[\begin{matrix} 0 & 0 & 1 \\ 0 & 1 & 0 \\ 1 & 0 & 0 \\\end{matrix}\right]$
  \end{itemize}

  {\em Performance review}: 16 out of 25 got this. 4 chose (E), 3
  chose (C), 2 chose (B).

  {\em Historical note (last time)}: $8$ out of $19$ got this. $4$ chose (B),
  $3$ chose (E), $2$ each chose (A) and (C).

\item $n = 3$ for this question. For the following three functions
  $f$, $g$, and $h$, consider the corresponding matrices
  $M_f,M_g,M_h$. Either two of them are similar and the third is not
  similar to either (in which case, select the matrix that is not
  similar to the other two), or all three are similar (if so, select
  Option (D)), or no two are similar (if so, select Option (E)).

  \begin{enumerate}[(A)]
  \item $f(0) = 0, f(1) = 1, f(2) = 2, f(3) = 3$
  \item $g(0) = 0, g(1) = 2, g(2) = 3, g(3) = 1$
  \item $h(0) = 0, h(1) = 3, h(2) = 1, h(3) = 2$
  \item All the above give similar matrices.
  \item No two of the corresponding matrices are similar.
  \end{enumerate}

  {\em Answer}: Option (A)

  {\em Explanation}: Let's first write out the matrices:

  \begin{itemize}
  \item Option (A): $M_f = \left[\begin{matrix} 1 & 0 & 0 \\ 0 & 1 & 0 \\ 0 & 0 & 1 \\\end{matrix}\right]$
  \item Option (B): $M_g = \left[\begin{matrix} 0 & 0 & 1 \\ 1 & 0 & 0 \\ 0 & 1 & 0 \\\end{matrix}\right]$
  \item Option (C): $M_h = \left[\begin{matrix} 0 & 1 & 0 \\ 0 & 0 & 1 \\ 1 & 0 & 0 \\\end{matrix}\right]$
  \end{itemize}

  $M_f$ is the identity matrix, and therefore, cannot be similar to
  anything else. $M_g$ and $M_h$ both describe matrices that cycle the
  three coordinates, albeit in opposite orders. A re-labeling can
  change $M_g$ to $M_h$. Explicitly, if:

  $$S = \left[\begin{matrix} 0 & 1 & 0 \\ 1 & 0 & 0 \\ 0 & 0 & 1 \\\end{matrix}\right]$$

  Then $SM_gS^{-1} = M_h$.

  {\em Performance review}: 12 out of 25 got this. 6 chose (D), 4
  chose (E), 2 chose (C), 1 chose (B).

  {\em Historical note (last time)}: $7$ out of $19$ got this. $3$ each chose
  (B), (C), (D), and (E).

\end{enumerate}
\end{document}
