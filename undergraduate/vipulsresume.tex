\documentclass[12pt,a4paper,oneside]{amsart}

%Title details
\usepackage{hyperref, fancyhdr}
\oddsidemargin -0.5in
\evensidemargin -0.5in
\topmargin -0.5in
\textwidth 7.0in
\textheight 9.5in
\pagestyle{fancy}
\title{Resume}
\author{\small{Vipul Naik, date of birth April 23, 1986}}

\begin{document}
\maketitle
%\tableofcontents
\fancyhead{}
\fancyfoot{}

\rhead{\tiny{Resume of: Vipul Naik, dob 23rd April 1986}}
\cfoot{\thepage}
\section{General information}
\vspace{0.25in}

\begin{tabular}{|l|l|}
  \hline
  Name & Vipul Naik\\
  Date of birth & April 23, 1986\\
  Sex & Male\\
  Current occupation & Student (B.Sc. Math 3rd Year) \\
  Course & B.Sc. (Hons) Mathematics\\
  Expected date of completion & July 2007 \\
  Institution & Chennai Mathematical Institute\\
  & Chennai, India\\
  \hline
\end{tabular}

\section{Contact information}

Address for correspondence:

\begin{quote}
  606, RMV Clusters Phase 2 Block 4,\\
  Devi Nagar, Lottegollahalli\\
  Bangalore, India\\
  PIN: 560 094\\
  Telephone number: +91 80 65676775\\
\end{quote}

Email ID: {\tt vipul@cmi.ac.in}

\section{Academic history}

\subsection{Overview}

I am currently in the third year of a three-year programme of B.Sc.
(Hons) in Mathematics at the Chennai Mathematical Institute. Given below
are my aggregate scores at important turning points:

\vspace{0.25in}
\begin{tabular}{|l|l|l|}
  \hline
  Level & Year of completion & Score\\
  \hline
  Class 10 (secondary) & 2002 &  89\% (100\% in mathematics)\\
  Class 12 (senior secondary) & 2004  & 91.4\% (100\% in mathematics)\\
  B.Sc. (first five semesters) & 2007 & 9.64/10 (CGPA)\\
  \hline
\end{tabular}
\vspace{0.25in}

\newpage
\vspace{0.25in}
\subsection{Undergraduate course details}

In Chennai Mathematical Institute, a grade point is awarded in each
subject out of $10$. A grade of $A$ corresponds to $10$ out of $10$, a
grade of $AB$ corresponds to $9$ out of $10$, while a grade of $B$
corresponds to $8$ out of $10$.

Information about the evaluation and grading system in CMI is available at:

\url{http://www.cmi.ac.in//locallinks/evaluation.php}

List of courses, with instructor name and grades:

\vspace{0.25in}
\begin{tabular}{|l|l|l|}
  \hline
  Course & Instructor & Grade \\
  \hline
  \multicolumn{3}{|c|}{First semester}\\
  \hline
  Algebra I & K.R. Nagarajan & A \\
  Calculus I & D.S. Nagaraj & A \\
  Classical Mechanics & P.P. Divakaran & B\\
  English & Shreekumar Varma & A\\
  Programming I (Haskell) & Madhavan Mukund & A \\
  \hline
  \multicolumn{3}{|c|}{Second semester}\\
  \hline
  Algebra II & S. Ramanan & A \\
  Calculus II & Guest faculty & B \\
  Discrete Mathematics & Bharat Adsul & A\\
  Economics & Lakshmi Kumar & B\\
  Programming II (C) & S.P. Suresh & A \\
  \hline
  \multicolumn{3}{|c|}{Third semester}\\
  \hline
  Algebra III & K.R. Nagarajan & A\\
  Analysis I & Amritanshu Prasad & A\\
  Calculus III& Suresh Nayak & A\\
  Design and Analysis of Algorithms & K.V. Subramanyam & B\\
  Global Calculus$^*$ & S Ramanan & A\\
  Theory of Computation$^+$ & Narayan Kumar & A\\
  \hline
  \multicolumn{3}{|c|}{Fourth semester}\\
  \hline
  Analysis II & Suresh Nayak & A\\
  Computer Organization & S.P.Suresh & A\\
  Electromagnetism I & K.S. Balaji & A\\
  Game Theory$^*$ & T. Parthasarathy & A\\
  Topology & V. Balaji & A\\
  Programming Language Concepts$^+$ & Madhavan Mukund & AB\\
  Analytic Number Theory & R. Balasubramanian & audit\\
  Computational Complexity & V. Arvind & audit\\
  Automata, Logic, Games and Algebra & K. Narayan Kumar & audit\\
  \hline
  \multicolumn{3}{|c|}{Fifth semester}\\
  \hline
  Algebra IV & R Sridharan & A\\
  Elementary Differential Geometry$^*$ & C.S.Aravinda &A\\
  Intro to Abelian varieties $^*$ & S.Ramanan & A\\
  Ordinary Differential EquationsDE & R Srinivasan & A\\
  Rep Finite Groups & Kannan & A\\
  \hline
\end{tabular}

\vspace{0.25in}

$^*$ : optional course

$^+$ : fast-forwarded course (intended for a later semester)

\newpage
\vspace{0.25in}

\subsection{Courses being studied in the current semester}

This semester, I have 1 compulsory course and 4 optional courses.

\vspace{0.25in}

\begin{tabular}{|l|l|l|}
  \hline
  Course & Instructor & Compulsory/Optional \\
  \hline
  Algebra and Computation & V. Arvind & optional\\
  Lie-theoretic methods & Alladi Sitaram, & optional\\
  & Amritanshu Prasad & \\
  Optimization Techniques & T. Parthasarathi & optional\\
  Probability & P. Vanchinathan & compulsory\\
  Riemannian Geometry & M.K. Vemuri & optional\\
  \hline
\end{tabular}

\vspace{0.25in}

\subsection{Summer camps}

\begin{enumerate}

\item Summer camp at the {\bf Institute of Mathematical Sciences},
  Chennai from May
  9th to June 17th, 2005. The topic was ``Groups, Representations and
  Algebras''. The instructors were Professor V.S. Sunder, Dr. K.N.
  Raghavan, and Dr. Amritanshu Prasad.

\item {\bf Microsoft Research Summer School on Algorithms, Complexity
    and Cryptography} from May 22nd to June 10th, 2006, at the Indian
  Institute of Science. The co-ordinators were Ramaratnam Venkatesan
  (Microsoft Research) and Professor Pandu Rangan (IIT Chennai). The
  webpage is:

  \url{http://math.iisc.ernet.in/~imi/sacc.htm}

  The list of selected candidates is available at:

  \url{http://math.iisc.ernet.in/~imi/downloads/weblist.pdf}

\item {\bf Visiting Students Research Programme} at the 
  {\bf Tata Institute of Fundamental Research}, 
  from June 15th to July 14th, 2006. Professor
  Dipendra Prasad was the co-ordinator and he was also my guide. I
  studied the paper ``Lie Group Representations of Polynomial Rings''
  by Bertram Kostant.

  The list of selected students is available at:

  \url{http://www.math.tifr.res.in/~vsrp/selected.html}

\item I am among three students from CMI selected for the {\bf ENS-CMI
    Exchange Programme} to be held from May 2, 2007 to June 29, 2007
  at Ecole Normale Superieure, Paris.

\end{enumerate}

\section{Other important achievements/activities/awards}

\subsection{Olympiads}

\begin{enumerate}

\item I represented India at the {\bf International Mathematical
  Olympiad} in 2003 held in Tokyo, Japan. I scored 23 points out of 42,
  the highest in the Indian team, and secured a silver medal.

\item I represented India at the {\bf International Mathematical
    Olympiad} in 2004 held in Athens, Greece. I scored 30 points out
  of 42, the highest in the Indian team, and secured a silver medal.

  My scores in both Olympiads can be checked by searching for ``Vipul Naik''
  (case sensitive, enter without quotes) on the IMO Compendium search page:

  \url{http://www.imo.org.yu/index.php?options=gl|imotres&p=39v_31}

\item I qualified the {\bf Zonal Informatics Olympiad} 2004, a
  national inter-school examination where approximately 5000 students
  participate and 200 are selected.

\item I was placed in the top 1\% of approximately
  31,000 participants for the {\bf National Standard Examination in Physics}.
\end{enumerate}

\subsection{Scholarships}

I have won the following scholarships:

\begin{itemize}

\item The {\bf National Talent Search Examination} (NTSE) scholarship.
  I won this scholarship based on an entrance-test-cum-interview selection.

\item The {\bf Kishore Vaigyanik Protsahan Yojana} (KVPY) scholarship.
  This scholarship was instituted by the Department of Science and
  Technology, Government of India, to promote excellence in pure
  science. The scholarship covered all my under-graduate study
  expenses.
\end{itemize}

\subsection{Other school-level competitions}

\begin{itemize}

\item I secured an All India Rank 10 in the Screening Test and 
  an All India Rank 158 in the Mains of the {\bf Joint Entrance
  Examination} (JEE) for the {\bf Indian Institutes of Technology} (IITs).

\item I was placed in the top ten for my class for the four
  consecutive years (Classes 8, 9, 10, 11) that I participated in the
  {\bf National Science Olympiad}, an annual competition in which over
  1500 schools participate. I secured the first rank twice.

\item I represented and won prizes for my school at numerous
  inter-school mathematics and programming competitions.

\end{itemize}

\section{Teaching experience/writing articles}

\subsection{Olympiad-related training}

At the request of regional and national co-ordinators, I have taken classes
training high school students for Olympiad-related mathematics.

\begin{enumerate}

\item The {\bf Association of Mathematics Teachers of India} conducted
  a ten-week training camp on problem-solving for the Olympiads, from
  August to October 2005.  I gave eight lectures in the camp to
  students from standards $9$ to $12$. I also conducted lectures/problem
  sessions in later camps organized by AMTI.

\item I gave {\bf Medalist's sessions} at the International Mathematical
  Olympiad Training Camp, in years 2004 and 2005.

\item I spent one day training the outgoing International Mathematical
  Olympiad team for 2006, at the {\bf Pre-departure camp}. I conducted
  three problem sessions.

\end{enumerate}

\subsection{Articles and problems}

I wrote an article on combinatorial identities that was published in 
{\em Samasya}, a mathematical problems journal.

A geometry problem I created was sent by India's Mathematical Olympiad
Cell as a proposal for International Mathematical Olympiad (IMO) 2006.

\section{Examinations for Graduate School application}

\subsection{General GRE}

I sat for the General GRE on September 14, 2006. My registration
number is 3034524. My scores are:

\begin{tabular}{|l|r|r|r|}
  \hline
  Section & Marks obtained & Total marks & Percentile \\
  \hline
  Quantitative & 800 & 800 & 94\\
  Verbal & 690 & 800 & 96\\
  Analytical writing & 6.0 & 6.0 & 96\\
  \hline
\end{tabular}

\subsection{TOEFL iBT}

I gave the TOEFL iBT on October 8, 2006. My scores are:

\begin{tabular}{|l|r|r|}
  \hline
  Section & Marks obtained & Total marks \\
  \hline
  Speaking & 29 & 30\\
  Listening & 29 & 30\\
  Reading & 29 & 30\\
  Writing & 29 & 30\\
  \hline
\end{tabular}

\enlargethispage{5mm}
\subsection{Subject GRE}

I gave the Subject GRE on November 4, 2006. My scaled score is 880/990
and my percentile is 97.

\end{document}

