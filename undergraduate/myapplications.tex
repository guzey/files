\documentclass[a4paper]{amsart}

%Packages in use
\usepackage{fullpage, hyperref, vipul}

%Title details
\title{The story of my applications}
\author{Vipul Naik}
\thanks{\copyright Vipul Naik, B.Sc. (Hons) Math and C.S., Chennai Mathematical Institute}

%List of new commands

\makeindex

\begin{document}
\maketitle
%\tableofcontents

\begin{abstract}
  This write-up describes the story of my applying to universities in
  the United States for a Ph.D. in Mathematics. It is concentrated in
  the period of May 2006 to January 2007, viz the first semester of my
  last year of undergraduate studies at the Chennai Mathematical
  Institute.
\end{abstract}

\section{Getting started}

\subsection{Should I apply?}

{\em Time period: April - May 2006}

The first natural questions before I could embark on my applications were:

\begin{enumerate}

\item Do I want to pursue Ph.D. after completing my B.Sc. or do I
  want to first try for an M.Sc.? Should I think only of continuing at
  CMI or should I explore other options?

\item Do I want to do Ph.D. in Mathematics or should I also consider
  other options like Applied Mathematics, Computer Science and
  Operations Research?

\item In which countries should I apply? India, United States, France,
  Germany, the United Kingdom?

\end{enumerate}

While answers to some of these questions were already well-formulated
in my mind, I had to do some hard thinking for the other questions:

\begin{enumerate}

\item Do I want to pursue Ph.D. after completing my B.Sc. or do I
  want to first try for an M.Sc.?

  I was reasonably sure by this time that I would like to continue
  along the line of research, and that my B.Sc. had provided me an
  adequate background. Hence, I was confident of being able to pursue
  a Ph.D. directly.

  However, I felt that applying to some universities (particularly
  those in the U.S.) with only a three-year undergraduate background
  might be a disadvantage and they may not admit me. So I had in mind
  the possibility of spending one more year at CMI. There is a scheme
  wherein a student can complete M.Sc. in a single year at CMI, and I
  felt that if I do that, my application for Ph.D. will be
  much stronger.

  After some thought, though, I realized that if I continue at CMI for
  another year as a M.Sc. or Ph.D. student, then there might be
  expectations within CMI that I will stay there for my later studies,
  and leaving at that stage might be more difficult. The admissions procedure
  would be more streamlined to people just completing B.Sc.
  and hence I decided to apply after my B.Sc.

  It took me a long time to reach this final decision, and it was
  around the middle of May that I decided to take the plunge and apply
  this year.

\item Do I want to do Ph.D. in Mathematics or should I also consider
  other options like Applied Mathematics, Computer Science and
  Operations Research?

  When I had first joined CMI, (pure) mathematics was my main area of
  interest.  However, after coming and interacting with the computer
  science faculty, as well as taking optional courses in computer
  science, I began to get interested in theoretical computer science
  as well.  Moreover, the links between complexity and algorithmic
  problems and certain areas of algebra made these subjects dearer to
  me than many parts of mathematics (such as analysis) where my exposure
  was more limited.

  At one point of time, I was considering whether to also apply to
  departments of Computer Science or Applied Mathematics. However, I
  decided that my credentials and interest were more towards
  mathematics. Nonetheless, I decided that one of my criteria for
  selection of a good university would be a high level of interaction
  between the faculty in mathematics and computer science, and the
  presence of faculty in those areas of computer science that I was
  interested in.

\item In which countries should I apply? India, United States, France,
  Germany, the United Kingdom?

  Within India, the only two places that I felt would be well-above
  CMI in terms of quality were Tata Institute of Fundamental Research
  (TIFR) and Institute of Mathematical Sciences (IMSc). Thanks to the
  close collaboration between CMI and IMSc in academic programmes, and
  due to a summer camp I had attended at IMSc, I had a fair idea about
  IMSc's Ph.D. programme. I planned to visit TIFR for its Visiting
  Students' Research Programme in the summer of 2006, and I hoped to
  use this visit to learn about TIFR's Ph.D. programme. Both these
  places seemed attractive.  Unfortunately, neither had faculty in the
  areas I was most keen on (group theory).

  The United States was definitely an attractive option -- there were
  a few universities there which had faculty in the areas I wanted to
  study. More generally, I had got feedback that the top American
  universities have an environment highly conducive to research.
  There were three major issues regarding applying to the U.S.: the
  {\em financial cost}, the {\em time/effort required}, and the {\em
    four-year rule} in some American universities (which said that a
  person must have a four-year undergraduate degree to apply to the
  U.S.). Another factor was {\em adjustment within the United States}.

  The financial cost for application (more details are available
  later), while quite a huge amount in terms relative to costs of
  living in India, seemed insignificant relative to what I would get
  if I got admission on full scholarship (which is the norm for
  mathematics programmes in leading American universities). In fact,
  even the {\em delta} between the estimated expense and the actual
  expense of one year could cover the entire application costs.
  Keeping all these factors in mind, I felt that financial
  considerations should not come in my way if I feel I have a
  reasonable chance and am keen on going abroad.

  The time/effort factor certainly seemed daunting, but again, when
  viewed from the perspective of {\em investment}, seemed far less so.
  The time I would ``waste'' in applying would probably be far less, I
  computed, than the time I would ``waste'' if I got admitted at a
  place that was ill-suited to me or where I would not get an
  environment for research.

  Regarding the four-year rule, I decided to collect data and try, as
  far as possible, to apply to U.S. universities that had admitted students
  completing a 3-year B.Sc. in the past. 

  Regarding adjustment within the United States, I felt that while
  this may require some effort, it should not act as a deterrent to
  applying. I was also aware that universities in the United States are used
  to coping with international students.

  I was not clear, at the time, of whether I should apply to places in Europe,
  in particular France, Germany and the United Kingdom.

\end{enumerate}

\subsection{Gearing for the general GRE}

{\em Time period: May - June 2006}

After I decided, some time in the middle of May, that I would apply
this year, I made a tentative list of universities which included
TIFR, IMSc, and a few names in the U.S. Since all U.S. universities
required scores for the general GRE, subject GRE and TOEFL, I decided
to look at the registration procedure for the general GRE. I visited
the GRE website:

\url{http://www.ets.org/gre}

I figured out the following basic facts about the general GRE.

\begin{itemize}

\item The general GRE has three components: {\bf Verbal}, which
  comprises 30 questions and is out of 800, {\bf Quantitative}, which
  comprises 28 questions and is out of 800, and {\bf Analytical
    Writing}, which comprises an Essay question and an Argument
  question, and is out of 6 points.

\item The general GRE is a Computer-Based Test (CBT). That is, the
  questions are asked on a computer. The testing service
  for GRE in India is Thomson Prometric, and they have centers
  in major Indian centers, including Delhi, Bangalore, Chennai, Mumbai etc.

\item The general GRE is held several times a month at each center,
  at timings 9 a.m. to 1 p.m. and 1 p.m. to 5 p.m.

\item One can register for the general GRE (this include selecting a
  center, test date, and paying the money by credit card) online.
  The center and test date cannot be changed after registration.
  The registration fee is 140 dollars, and if one cancels after registration,
  100 dollars are refunded.

\end{itemize}

I booked my general GRE for 14th September. I did this some time
in the end of May.

\subsection{The passport hassle}

While registering for the general GRE, I discovered that in order to
write the examination, I need to carry with me my original passport.
No other substitute document can be used. At the time, I did have a
passport, which would not expire by the date of the general GRE.
However, I had long planned to renew my passport because I had taken
it as a minor and it was due to expire in 2008. I also needed to change
my residential address.

In the first week of June, I was attending a Microsoft Research Summer
School, and on June 14th, I was scheduled to leave for Mumbai to
attend the Visiting Students' Research Programme at TIFR. So I gave my
application towards the fag end of the Microsoft Research Summer
School. On the Monday after the summer school ended, I was asked to
come up to Sanjaynagar police station with my documents and originals.
The officer indicated that my passport would reach me within a few
days' time.

My passport reached some time in the beginning of July, while I was in
TIFR. Thus, I was able to take it along with me when I went to Chennai.

{\em Note: It is not necessary to specify the passport number when
  registering; however one should make sure one has the passport in
  hand when going for the actual examination.}

\subsection{Initial preparation for the general GRE}

From what I had heard, the Quantitative section of the general GRE was
a cakewalk, and CMI students had consistently got full scores
on it. 

For the Verbal section, I had heard that there was a GRE-specific
vocabulary that I needed to master to do well on the examination. The
general opinion was that if one knew the meanings of all the words, it
was easy to get a near-perfect score on the verbal component.

Regarding the essay part, I had heard that the essay topics were fairly
hard, but getting something like 4-4.5 was not too tough.

While I was going for the Microsoft Research Summer School (which was
being conducted at IISc), I visited the Tata Book House at IISc and bought
two books by Kaplan, one for the Verbal GRE and the other for the Analytical
Writing section of the GRE.

I also downloaded the list of topics for Analytical Writing from the
ETS website and started looking at them. Most of the topics seemed
very hard and I had very little idea of how to write them. The grading
policy which the ETS had outlined was also not too clear to me, and my
own evaluation of the pieces I had written ranged from 2 to 4
depending on how I interpreted their guidelines. Thus, I decided to
spend some more time in preparing for the essays to ensure that my essays
were, in all respects, at least as good as 4.

I had already started doing vocabulary practice for the GRE from the website:

\url{http://www.number2.com}

I planned to use this website and Kaplan's book to prepare for the
Verbal section.

I also downloaded the GRE online practice material (a software called
POWERPREP) and had a look at it. This contained two sample tests and
problems in the various problem types.

\section{Deciding my universities}

\subsection{Preliminary web-surfing}

{\em Time period: July 2006}

I had the following criteria in mind:

\begin{enumerate}

\item The mathematics department of the university must either be strong
  in the overall sense or have faculty in my particular area of interest,
  or both

\item The university should (preferably) have a good computer science
  department as well, and have strong interactions between
  the mathematics and computer science department

\item The university should, in general, offer full scholarship to students
  it admits for its Ph.D. programmes in mathematics.

\item There should be a reasonable chance of the university giving a fair
  chance to students from India who have only a three-year bachelors' degree.

\end{enumerate}

I visited the websites of all the universities to determine the application
procedure, and also to find out which of them required GRE, TOEFL scores,
and also to find out their deadlines. I started compiling information
about the deadlines of the universities and the requirements for each.
I found the following common features:

\begin{itemize}

\item All the universities required general and subject GRE scores
  as well as TOEFL scores. These scores must be {\em sent} by the ETS
  to the universities. Each university has a university code
  and the scores must be sent to the university according
  to the university code.

\item The universities seemed to ask for information on financial resources

\item The deadlines ranged from December 1st (Princeton) to January
  15th (Caltech)

\item Almost all the application material (including recommendations)
  was online. The only thing that had to be sent by snail mail was the
  academic transcript.

\end{itemize}

\subsection{Collecting information over email}

{\em Time period: July - August 2006}

Based on doubts I had regarding the form, I consulted some of my seniors
who were passing out of CMI or had already left CMI. They gave me the
following information/suggestions:

\begin{enumerate}

\item For the academic transcript, only the first four semesters need
  to be specified.

\item For financial resources, if one is applying for financial aid,
  the sections on one's own financial resources are irrelevant. Those
  sections are meant for people who are offering to pay their fee in full
  or part.

\item For recommendations, it is best that I ask well in advance, so
  that the recommender has enough time to prepare a thoughtful
  recommendation.

\item In general, when applying to a university, it is good to ensure
  that there is some chance that the university will admit one.

\item It is good to avoid clashing with other students who also
  intend to apply to the same place, specially if recommendations
  are being taken from the same person.

\end{enumerate}

\subsection{The final list}

The universities I decided to apply to were:

\begin{enumerate}

\item {\bf University of Chicago}: This was my favourite because it had
  all the features: a strong mathematics department on the whole,
  faculty in my main area of interest (group theory), a few people
  working in computational group theory (particular Babai), as well
  as a history of many CMI students having gone there in the past.

\item {\bf Princeton University}: The main attractor was the good
  overall strength of the university. Princeton University did not
  have faculty in group theory, but on the other hand, it is renowned
  for its mathematics department as well as its departments in related
  areas.

\item {\bf Massachussetts Institute of Technology}: This had faculty
  in representation theory, which is close to my primary interests.
  MIT also has an overall strong department in mathematics, as well as good
  departments in applied mathematics and computer science. Again, there
  was a history of CMI students having got admission in MIT.

\item {\bf Harvard University}: Although Harvard official policy has a
  four-year rule, I knew of one person (Samik Basu) who had gone to
  Harvard after three years at ISI, Bangalore doing B. Math. So I felt
  that Harvard was worth a try. Although there were no faculty
  in Harvard working directly in group theory, Mazur and others
  were working in automorphic function theory and number theory,
  which were also areas of interest for me.

\item {\bf University of Pennsylvania}: This was famous for its Graduate
  Program and also had a strong faculty group on the inverse Galois problem,
  which was closely linked with my areas of interest. Further, I knew
  that Penn would not be hard getting admitted to; there was a precedent
  of many CMI students having joined Penn in the past.

\item {\bf California Institute of Technology}: Professor Ramanan
  told me that admission in Caltech would not be very hard. Caltech
  is a smaller place but has good faculty, and I had got good feedback
  about it from some others who have joined there.

\item {\bf University of Wisconsin-Madison}: I selected this University
  because I had had extensive email correspondence with Professor
  Martin Isaacs of this university on an unsolved problem I had proposed,
  and I wanted to work under him. Also, Wisc-Mad's math department was
  strong on the whole. I decided to apply to Wisc-Mad some time in August,
  after I had decided on my first six universities.

\item {\bf Rutgers University}: I began considering this University
  when Professor Balaji told me that Rutgers was strong in group theory.
  I confirmed this by going through the Rutgers webpage. I also knew
  of CMI students who had been admitted to Rutgers after a three-year B.Sc.
  programme.

\end{enumerate}

\section{Bookings}

\subsection{Subject GRE booking}

{\em Time period: July 2006}

In the end of July, just before leaving for the CMI term, I decided to
book my Subject GRE.  Based on my tentative investigations, I was sure
I would apply to the following places: Chicago, Princeton, MIT,
Harvard, Caltech and Penn. All these places had been known to admit
students after a three-year B.Sc. programme in the past (either from
CMI or from ISI, Bangalore).

For the subject GRE, at the time of booking, one can specify
up to four universities to which the subject GRE scores
are sent for free. For the remaining universities, the score sending
needs to be booked over phone, later, and the booking charges
are 15 dollars per university. Moreover, for universities booked later,
the scores would take later to reach. So I had to decide what
my top four were.

Among the universities I had decided to apply to, I used a combination
of my relative preference and the deadline information to come up with
the following four names: MIT, Chicago, Princeton, and Harvard. I thus booked
my subject GRE for November 4th with these four universities.

\subsection{TOEFL iBT booking}

{\em Time period: August 2006}

The TOEFL or Test of English as a Foreign Language, had recently shifted
from the old Computer-Based Test format to a new Internet-based Test
format. Dates for the new iBT had not been scheduled when I left home
for my semester at CMI. Around a week later, I saw that the list of dates
was up, and I selected October 8th, on account of its being a Sunday
as well as on account of its being sufficiently separated both from
my general GRE (September 14th) and my subject GRE (November 4th).

I booked for the TOEFL some time on August 13th. While booking for my
TOEFL, I had to again specify four universities to which the scores
can be sent for free, and I specified the same four: MIT, Chicago,
Princeton, and Harvard.

\section{The exams begin}

\subsection{Preparation for the general GRE}

{\em Time period: August-September 2006}

In the middle of August, I started getting geared up for the general
GRE.  The argument topics were not hard and I practised enough of them
till I got the hang of doing a good Argument in about 25-35 minutes
(the time limit for an Argument is 30 minutes). For the essay, I
practised an average of one essay a day for around 2 weeks and also
collected general facts related with labels like ``education'',
``society'' and ``politics'' which I could throw into any essay on the
subject. Kaplan's book was a great help for essay writing. Some time towards
the end of August, I felt I had enough of a grip on essay preparation
to take it easy.

Preparation for the verbal part involved combining Kaplan's verbal workbook
with practice from Number2. Some of my friends were using Barrons'
wordlist to prepare for the General GRE, and I had a look at that wordlist
also. Barrons' book had more words than Kaplan but it was not as 
well-organized.

\subsection{The general GRE examination}

{\em Time: September 14, 2006}

A week before the general GRE, I checked out the center. The Thomson
Prometric center is located in Nungambakkam. My exam was scheduled for
9 a.m., and the security persons told me to come at 8:30 a.m. with my
passport and, as a precautionary measure, with a printout of my
registration email and a college ID card. He said, however, that
neither of these documents would be necessary.

I reached the place at 8:20 a.m. I was made to sign a non-disclosure
agreement, which basically said that I was not supposed to reveal
anything about the paper to anybody and that the questions were to be
used by me only for the purpose of answering. After signing this
agreement, I was photographed and my passport was checked. I then
began on the general GRE examination.

The essay and argument seemed to proceed well, but I felt that I had
fumbled at many places in the verbal section. The quantitative section
was smooth: I finished with around 5 minutes to spare, after having
double-checked every answer. At the end, I was feeling somewhat disheartened
because the Verbal section had not gone as well as I had expected.

After the examination was over, there was an experimental writing
question, on which I wrote for fifteen minutes and then quit. I then
selected my top four universities (for the general GRE, the selection
is after the examination). I was finally asked whether I want
this test to be put in my record or whether I want the test scores
to be cancelled. I asked for the test to be put in my record and then
I was told my scores. I had got a full 800 in the quantitative section
and a 690 in the verbal section.

On the way back I was feeling a little disheartened about my performance
in the verbal section, and even considered writing the examination
again. When I returned, I looked again at the POWERPREP CD for
information on how good a given score is. It turned out that a score
of 690 in the Verbal section placed me at the 97th percentile among
mathematics applicants sitting for the GRE, while a {\em full} score
in the Quantitative section placed me at the 89th percentile among
mathematics applicants (percentile, the way ETS uses it, is the
percentage of people scoring strictly less marks). This heartened me
considerably. After collecting inputs from some more people, I came to
the following conclusions:

\begin{itemize}

\item The general GRE serves more as a cutoff than as a determinant of one's
  ability, especially for people applying to mathematics. Scores
  of 690 and 800 are good enough with respect to the cutoff

\item Good performance in the subject GRE is far more important

\item Since (as I believed) I had performed well in the essays also,
  there was no point in giving the examination again for just one section
  (the Verbal section).

\end{itemize}

I received my score report by post around 3 weeks later. It turned out
that I had got a full score (6.0) in the Analytical Writing section.
My percentile (this is among all applicants) in Verbal and Analytical
Writing was 96th, that in Quantitative was 94th.

\subsection{The TOEFL iBT: preparation and examination}

Earlier on, before the semester had started, I had bought
Barrons' book on the TOEFL. The book at the time was not exclusively
geared for the iBT; it still called the iBT the ``next generation TOEFL''.
The main thing I practised from the book was Speaking; I did that
for around an hour each around 2-3 times in the week just before
the examination.

My TOEFL iBT examination was on October 8th, at 1:00 p.m. I left for
lunch at 11:15, hoping to be back at 12:00 so that I could come and
leave for the examination. When I returned, I found the room locked.
It turned out that there was some confusion. Luckily, the confusion
was sorted just in time for me to reach the examination center, and I
was able to write the TOEFL.

All my sections went smoothly, except one speech topic, where I finished
my speech around ten seconds before the allotted time had ended,
and so I made a fumbled-up last sentence.

Later, around October 27th, I checked my score online and discovered
that I had got 29/30 in each of the four sections: Reading, Writing,
Listening and Speaking.

{\em Note: TOEFL also had a confidentiality agreement}

\subsection{Subject GRE: Preparation and examination}

{\em Time period: September - November 2006}

My subject GRE was scheduled for November 4th. Around July, I had
started hunting for books to prepare for the subject GRE. Some of my seniors
told me about the book on subject GRE by The Princeton Review. I checked
in bookstores for this book but was unable to locate it. I tried
to procure the copy that earlier belonged to the seniors but this
copy had been taken back home by one of them. So, around the first week
of August, I ordered the book from Amazon along with some other books
my mother was ordering from Amazon.

Around mid-September, immediately after the general GRE, I ``sized''
this book, as in I determined the amount of effort needed for the
subject GRE. I then picked it up again around two weeks before the
subject GRE. I worked in around 2-3 sessions of 4-5 hours,
and followed this up with trying a practice test of the subject GRE
(two days before the actually subject GRE).

The subject GRE was on November 4th at 9 a.m. I managed to reach the
examination center (Stella Maris college) around 8:15 a.m. and everything
proceeded smoothly.

{\em Note: Subject GRE also had a confidentiality agreement}

\subsection{Additional universities for general GRE}

{\em Time: 1st November, 2006}

I decided to book my additional universities for the general GRE. I
made an international phone call to the number given by the ETS on the
website, and I booked the four extra universities that I had decided
to apply to: Caltech, Penn, Rutgers and Wisconsin-Madison.

ETS charged 15 dollars per university, and 6 dollars service charge.
In addition, there were phone charges (although the number
given by the ETS is toll-free within the U.S., international callers
have to pay at the standard international calling rate).

\subsection{Score report and additional universities for TOEFL}

Although I got to know my TOEFL score online, I did not receive my
paper score report for the whole of the month of November. In a follow-up
email with the ETS, I was told that the report has been mailed
on October 27th. I checked up with some people  and they were
of the opinion that since all universities anyway require the official scores
directly from the ETS, my not having a paper score report should not
be a problem.

I finally got my TOEFL score report in January.

For TOEFL, I booked my additional universities online (unfortunately,
the option is not available for the general and subject GRE). The charge
was 17 dollars per university.

\subsection{Score report and additional universities for subject GRE}

I had found out from some seniors that the paper score report for the
subject GRE typically reaches some time towards the fag end of
December. I felt that waiting for the score report to reach me before
booking my additional universities would be cutting it too thin. So, on
December 8th, I called the phone number for the GRE (the same one I
had used for the general GRE) to find my score and book additional
universities.

Finding my score cost 10 dollars and booking the additional universities
cost 66 dollars (15 dollars per university and 6 dollars extra).

Later, I realized that when I book for a university to receive my
scores, I can request for {\em both} the general and subject test
scores to be sent to the university (with the total cost still being
15 dollars). Hence, if I had waited for the subject GRE scores and
then booked my additional universities together, I would have saved 66
dollars. However, that would also have meant delaying the reaching of
the General GRE scores to the extra universities. I think this point
should have been clearer to me earlier so that I could have taken a
more informed decision (I may still have chosen the way I did).

I finally got my paper score report around December 21st. My raw score
was 62/66 (63 correct and 3 wrong out of 66). My scaled score was 880/990
and my percentile was 97th.

\section{Recommendations}

\subsection{Number of recommendations}

All the universities I was applying to accepted online recommendations
(two of them, Chicago and Rutgers, started accepting online
recommendations only this year). For all of them, a minimum of three
recommendations was needed. There were some universities that allowed
me to get more than three recommendations. However, I did not use
more than three recommenders for any of the three universities -- probably
because following up for three recommenders itself was enough work!
This is something I could probably have done differently.

\subsection{Whom should I take recommendations from?}

I was reasonably sure that none of the professors I would approach for
recommendations would write negatively about me; however, I wondered
whether they would have had enough interaction with me to write
something that would bring out my strengths and initiatives properly.
After some thought, I selected on three people who had taught me and
interacted with me throughout my two years at CMI: Professor Ramanan,
Dr. Amritanshu Prasad, and Professor Balaji.

\subsection{Professor Ramanan}

I had attended two courses under Professor Ramanan and had also had
many personal sessions with him where I head learnt topics of algebraic
geometry, differential geometry and complex analysis. In addition,
I had shared and discussed some of my original work with him.

Professor Ramanan is a well-known person and is also knowledgeable
about a number of universities, so I sought both his advice on where
to apply and his recommendations to the places I decided on.

During the holidays itself, I asked Professor Ramanan for advice on
which places to apply to. Later, in August, I had a long discussion with
him on the factors to consider while applying. This helped me decide
how to go about selecting my universities and projecting myself
to those universities.

After this discussion, I requested Professor Ramanan for recommendations
to the universities I was applying to. He agreed to write recommendations
for me. Till the end of September, I had decided on seven universities
(all excluding Rutgers) and I had registered him for all of them.

\subsection{Dr. Amritanshu Prasad}

I had interacted with Dr. Amritanshu Prasad over the past two years.
He taught a few classes in the Algebra II course, then taught the Analysis
I course. I also attended a summer camp where he gave some lectures.
I had had many fruitful personal sessions with Dr. Prasad, and I felt
I should take a recommendation from him.

I approached Dr. Prasad sometime in September, and he agreed. I filled his
name for all the universities I was applying to.
\subsection{Professor Balaji}

Unfortunately, Professor Balaji had a hectic travel schedule in August
and September and I got around to meeting him only in the middle of
October.  I told him that my main area of interest was group theory
and algebra.  He suggested Rutgers as a good place for pursuing group
theory, and also gave the names of faculty members at some other areas
who were doing cutting-edge work in group theory. Based on his
suggestion, I did some research on Rutgers and appended it to the list
of universities I was applying to. This gave the grand total of eight.

I also had a look at the websites of some of the other universities
that Professor Balaji had suggested, but felt that they are not worth
applying to.

\subsection{Following up for recommendations}

There was a slight glitch in the MIT recommendation process. MIT had
an evaluation request form, and the rule was that I could submit the
evaluation request form only once. I first filled in Professor
Ramanan's name and submitted the form (at the time, I had not
confirmed with the other recommenders).

After realizing my folly, I scrapped that online application form and
started another one. Since Professor Ramanan had not send his
recommendation, I requested him to use the {\em new} application form
and ignore the earlier request.
 
Professor Ramanan and Dr. Amritanshu Prasad did all the universities some time
in the third week of November, with the following exceptions:

\begin{itemize}

\item They could not do Rutgers, because they had not received any online
  request for Rutgers

\item Professor Ramanan was unable to locate the request email from Wisc-Mad

\end{itemize}

I promptly resent the request email from Wisc-Mad, and then I sent an email
to Rutgers' admissions office asking them why request forms had not
reached the professors.

It turned out that that week was Thanksgiving, so Rutgers responded
about a week later, saying that the recommendation requests would be sent
{\em after} I submitted the online application form. This created
an added complication.

The Princeton deadline was approaching (it was November 24th and the
deadline was December 1st). Professor Balaji had planned to do my
recommendations the previous weekend, but had been unable to complete
his drafts. He was also in a hurry as he had to go out of station.
So he finished off Princeton and said he would do the remaining as and 
when he got free time.

Over the next three weeks, he gradually completed the remaining universities,
and by December 23rd, he had done all the universities.

Professor Ramanan did Wisc-Mad and Rutgers some time in the second
week of December.

\section{Statement of purpose and resume}

\subsection{The resume}

At one point in time, I was considering working at Microsoft Research
over the winter, and I prepared a resume for this purpose. Though the plans
didn't work out, the plus point was that I had a resume to start with.
I spruced this up for my university applications and used it for Princeton.
With some minor modifications, I used the resume for all the other
universities as well.

\subsection{Statement of purpose}

Preparing my statement of purpose was probably the most daunting and
intellectually draining of all tasks. While preparing my statement of purpose,
I wanted to keep all the following things in mind:

\begin{enumerate}

\item It should bring out my talent, initiative and interest

\item It should bring out my areas of interest

\item It should bring out why I am choosing to apply to the particular
  university

\item It should have good language, good sentence structure, and good
  overall organization.

\end{enumerate}

\subsection{Some preliminary write-ups}

In the period between June and October 2006, I started working 
on preparing stuff related to my competencies, my areas of interest,
and my past activities. Some of these are put up at:

\begin{itemize}

\item Competencies are at the competencies part of my homepage:

  \url{http://www.cmi.ac.in/~vipul/competencies}

\item Areas of interest (group theory, differential geometry, algebraic
  geometry, number theory) are at:

  \url{http://www.cmi.ac.in/~vipul/areasofinterest}

\item Activities and achievements which area t:

  \url{http://www.cmi.ac.in/~vipul/activities}

\end{itemize}

\subsection{The first draft}

The weekend after my subject GRE, I started on the first draft of my
Statement of Purpose. I had been putting this off for quite a long time,
hoping to collect enough things to write about before I began.
But I finally realized that instead of keeping on trying
to collect points, it may make more sense to work on a draft.

The first draft had a lot of good content but sounded pretty
mechanical (lore like a shopping list). I made a few generic
iterations (that is, iterations that were not university-specific) until
I realized that the Princeton deadline was approaching, and I had better
start gearing towards that.

In my general draft, I tried to do the following:

\begin{itemize}

\item I started off by stating where I am applying to, and what
  my areas of interest are.

\item I then proceeded to describe a bit of my Olympiad background and
  why I came to choose mathematics as a career option (thus bringing
  out both my achievements and my initiative/desire to pursue mathematics)

\item I then discussed what I had learn in the various areas of mathematics,
  giving special emphasis to summer camps, extra courses I had credited,
  and other ways that I had demonstrated initiative.

\item I then proceeded to describe areas where I had done some
  original thinking, research and exploration.

\item At the fag end, I mentioned my interest in teaching and sharing
  mathematics.

\item The final paragraph was intended to be a university-specific one,
  a paragraph where I would describe why I wanted to be selected
  to that particular university.

\end{itemize}

\subsection{Working on the piece for Princeton}

During my earlier exploration of the universities, I had picked on
Princeton as a place to apply to largely because of the overall
reputation of its mathematics department. This was despite the fact
that Princeton didn't seem to have too many faculty in my areas of
interest. However, as I started trying to jot my last paragraph, I
realized I was pretty much in a fix about what to write about Princeton.

I went through the webpage again, and made a last paragraph that again
looked like a shopping list of people and things there. Then I
realized that this is not the kind of final paragraph I should
make. After some serious thinking, and consulting with others, I made
a paragraph that reflected why I would like to join the Princeton mathematics
department.

Princeton also had a serious words constraint: an upper limit of 1000
words.  Since I had a lot to write in the beginning, working under
this word constraint was quite restrictive, and I had to end up
chopping a lot of words and using many awkward constructs (to be
miserly on words).

I finally had a reasonably final SOP by around 23rd of November.

\subsection{The next batch of 3}

After Princeton, I took a break of a few days, and then began work on
the next batch of universities. These included Rutgers, Harvard and
Pennsylvania. They had the following constraints:

\begin{itemize}

\item Rutgers had no constraints, except the fact that it had to be given
  in a text area

\item Harvard had a 1000-word limit (just like Princeton)

\item Pennsylvania had a 1000-word limit, and a 6000-character limit

\end{itemize}

I first started work on the Rutgers SOP because I wanted to write
freely without a word constraint. While working on the Rutgers SOP,
with Princeton's SOP as a starting point, I discovered that I had made
a few minor factual errors in the Princeton SOP (discernible only to
the trained mathematical eye). This was a slight shock to me (I had
gone through the SOP many times for language and style but had
forgotten to check the math of it). So this time I did a far more
careful check.

Writing the Rutgers SOP felt better than writing the Princeton SOP,
primarily because there was more elbow room. Also, there were more
people at Rutgers who seemed potentially interesting for me,
and I was able to sprinkle mentions of all of them.

In parallel, I started working on the SOPs for Harvard and for
Pennsylvania.  These again proved somewhat challenging, but now that I
had done the groundwork, the incremental work for each SOP was less.

I had finished all these by December 12th or so.

\subsection{The next batch of 3}

Of the four universities now remaining, Caltech had a deadline as late
as January 15th, so I decided to defer work on the Caltech application
for later.

I thus had the following places to work on:

\begin{itemize}

\item Chicago, which was at the time my top choice, and where the word
  limit was 2500 words, which gave me ample space. I also had a lot of things
  to say for Chicago. The material had to be entered in a text area

\item Wisconsin-Madison, where the SOP was expected to be two and a
  half pages long (when viewed in their PDF preview). This was not as long
  as I would have liked, but it was still more than a thousand words.
  Again, the material had to be entered in a text area

\item MIT, where there was no explicit limit for the SOP, and where the SOP
  had to be written in a text area.

\end{itemize}

I spent the maximum time on my Chicago SOP, for two reasons: the word
count for Chicago was the highest, giving me maximum elbow room, and
I had the most to write for Chicago (considering it seemed my top choice
at the time).

In parallel I was working on the SOP for Wisconsin. I was very pleased
with the way the Wisconsin SOP turned out, particularly considering that it
was only two and a half pages.

I also finished the MIT SOP.

\subsection{The last one}

By December 20th, I had closed my applications to all the other universities,
and Caltech was the only place left. By this time, I was pretty much fatigued
with university applications.

Caltech didn't have any word limit, but their instructions urged for a
{\em short} piece, and I wasn't in the mood to write anything much. So I kept
it roughly at the same length as the Wisconsin piece, with a few minor edits
here and there.

\section{Making the final submissions}

\subsection{Getting sealed copies of the transcript}

Around the middle of November, five of us (who were applying abroad)
requested the office for sealed copies of our CMI internal transcripts
for the first four semesters. As far as I expected, we were supposed
to send one transcript to each university. Since I was interested
in applying to eight universities at the time, I took eight sealed copies
of the transcript. It later turned out that this was a mistake.

\subsection{Princeton submission}

The Princeton deadline was 1st December. I had to send the following things
to Princeton:

\begin{itemize}

\item The online application included Statement of Purpose, 3 letters
  of recommendations, resume, and various form details

\item The postal application had to include a sealed copy of my transcript.
  I decided to also send photocopies of the various other certificates
  (for Olympiads and other achievements) -- not that it mattered, but just
  as proof of the various things I had said in my resume.

  I also included photocopies of my GRE and TOEFL scores; these photocopies
  could not serve as substitutes for the official scores from ETS but
  at least they could serve a temporary purpose in case the official scores
  were delayed in coming.
\end{itemize}

To prepare the postal application, I did the following:

\begin{itemize}

\item I went through all the instructions on the website and figured out
  that Princeton did not want anything other than the transcript.

\item I found out the rates and schemes for international courier.
  It turned out that DHL charged Rs. 983 for students applying abroad
  (this is called University Express scheme). I found out the 24-hour
  center in Chennai (it is located in Guindy).

\item I confirmed a few things with the Princeton Graduate Admissions
  department before sending (namely, whether I can send photocopies of
  other certificates, and whether I need to attach any covering letter
  with identifying information). They replied positively about sending
  other certificates and said I don't need to put any identifying
  information apart from my name and the department to which I am
  applying (to be very safe, I also added my date of birth)

\item On the morning of Friday, 24th November, I sent off my packet to
  the Princeton address via DHL from the center at Guindy. The packet
  comprised a sealed copy of the transcript, copies of my general GRE
  and TOEFL scores, and photocopies of some of my certificates on
  Olympiads.
\end{itemize}

After that, I had a look at my online application again.
Some final things with my online application:

\begin{itemize}

\item I confirmed all the details I had put in, and my parents
  cross-checked the biographical information and information i had
  entered about them. I also deliberately left the financial resources
  section blank (rather, filled in zeroes everywhere).

\item I was in a bit of a quandary on whether to wait for all the
  recommendations to be received before i submitted the online
  application. I was really keen to submit the online application by
  the afternoon of Friday, but one recommendation had still not come
  through.  Luckily, the recommender, Professor Balaji, finished the
  recommendation some time on Friday afternoon. I was able to submit
  the recommendation and thus my first university application was completed.

\end{itemize}

After I had submitted:

\begin{itemize}

\item As soon as I submitted my online application, I received an
  automated confirmation for Princeton. This said that a personally
  written confirmation about my online application having been received
  would be sent soon.

\item On Monday, I got a letter saying that my online application was
  securely received, and that I could check my further status
  on the Status Tracking page.

\item Princeton had a Status Tracking page, and the DHL also allowed
  Status tracking. Through Status Tracking, I found out that the
  Princeton application reached on 27th morning, however, the packet
  reaching was not immediately updated by Princeton. Princeton did the
  update a couple of days later. It not only showed my transcript
  as having been received, it also put in separate entries to indicate
  the additional certificate photocopies that had been received.

\item The Status Tracking page on Princeton also showed,
  in a little time, that my GRE General score had been received.
  A day or so later, I saw that my TOEFL score had also been received.
  It didn't show my subject GRE score as being received -- but at that time,
  I myself hadn't received my subject GRE score.

\end{itemize}

Thus, by 1st December (the official deadline) Princeton had received
my online application, my letters of recommendation, my transcript, my
general GRE score, and my TOEFL score. It had {\em not} received my
subject GRE score -- the only thing missing for a complete application.

\subsection{Rutgers}

When I came to the Rutgers application and went through the details again,
I came across two pieces of news:

\begin{itemize}

\item The application was due to be submitted on 1st January, as
  opposed to 15th December (which was what I had initially noted).

\item The Rutgers department required {\em two} transcripts, as opposed
  to just one.

\end{itemize}

Since I had just as many transcripts as universities, and since I was now at
home (Bengaluru) rather than in my institute, I was a in bit of a quandary
about what to do. I considered two options:

\begin{itemize}

\item I could postpone sending the Caltech application to once I was
  back in Chennai. This was feasible since the Caltech
  deadline was January 15th.

\item I could request CMI to send some more sealed transcripts to me.

\end{itemize}

At the time, I decided to exercise the first option, or at least keep to
the first option unless I came across yet another university that required
two copies of the transcript.

I was keen to submit the online application for Rutgers as soon as
possible because the recommendations could be sent for Rutgers only
{\em after} I had sent the online application. I submitted the online
application at around 6th December.

As soon as I submitted the application, one of my recommenders,
Dr. Amritanshu Prasad, submitted his recommendation for
Rutgers. Professors Balaji and Ramanan submitted their recommendations
over the coming two weeks. Thus, by the time the final Rutgers deadline
was reached (January 1st) all the recommendations had been sent.

I sent my Rutgers transcripts around December 20th, along with the applications
for other universities (MIT, Wisconsin, Chicago). More on this is discussed
in the later section on those universities.

\subsection{Pennsylvania}

Some time in the end of November, my University of Pennsylvania
application was shifted from the old system (ExpressApp) to the new
system (ApplyYourself). By that time, two of my three recommenders had
already submitted their recommendations on the old system.

I was in a fix as to whether to ask them to resubmit recommendations
on the new system, or whether to ask the Penn admin people to shift
the recommendations. Finally, I decided to write to the Penn admin people
asking them whether they could shift the recommendations automatically.
They asked me to mark those who had already submitted recommendations
as ``offline'' providers and that the Penn admin people would attach a printout
of the recommendations to my application.

I finally submitted my Pennsylvania postal application on December
8th, and on December 12th, I submitted my online application. After I
submitted my online application, I received an email from them. The
email had two bombshells:

\begin{itemize}

\item The address which the email gave for sending supplementary
  materials was the {\em departmental} address but the address which
  had been given on the website was the general Graduate Admission Office
  address.

\item The email asked for {\em two} copies of the undergraduate transcript.

\end{itemize}

I at once sent an email to the administrative person at Pennsylvania
explaining both the problems to her. She did not reply immediately.
The deadline was mounting and I was wondering whether I should send
another set to the correct address and with the two transcripts. I wrote
another email, this time to the mathematics coordinator
for admissions, and she replied immediately saying that neither matter
was a problem because:

\begin{itemize}

\item The mathematics department worked in coordination with the
  Graduate Admissions Office for admissions and material sent to either place
  would be accessible to the other

\item They would make an extra copy of the transcript themselves

\end{itemize}

This was a relief to me. Nonetheless, I decided that:

\begin{itemize}

\item I would henceforth send the postal thing only {\em after}
  submitting the online application

\item I would ask CMI for some extra sealed copies of my transcript

\end{itemize}

\subsection{Harvard}

Harvard was by and large smooth, though the application had a few
tricky components, including the one on finances (where I had to do a
careful totalling from their estimated living expenses on various
fronts).

The day on which I decided to submit my Harvard application, I found
that the application system was misbehaving. In particular, my
recommendations seemed to not be there, and more alarmingly, the
mathematics department seemed to have disappeared from the list of
departments to apply to. I sent an email to Harvard, and got an
automated response saying that they got hundred of emails per day and
I should expect a reply in a week's time.

I instead located the mathematics department person responsible for
admission and emailed her about the problem. In parallel, my father
called up the address at Harvard (once it was working hours for them)
asking them about the problem. The person replied saying that he had
already received hundreds of calls on the matter, but that the
administrative people were in California and their office hours hadn't
yet begun\footnote{U.S. has different time zones, with Harvard in the east zone and California in the West zone}.

Also, in some time the mathematics department person replied saying that the problem had been fixed, and that the problem had been from the administrative end.

After this, I managed to submit my Harvard application.

\subsection{Wisconsin-Madison, MIT, Chicago}

All these applications proceeded smoothly. I submitted the online applications,
confirmed that there were no ``double transcript'' requirements nor was 
there a different address suggested. I then sent the transcripts
to each university.

The problem though was with their receiving the postal applications.
In all cases, the courier company indicated that my transcript had
been received. However:

\begin{itemize}

\item All the universities were on holiday in the Christmas week. Wisconsin
  updated a couple of days after it returned from holiday, so around
  January 2nd, I came to see that it had received my transcript.

\item Chicago and Rutgers updated some time around January 11th (they were
  both catching up with the huge backlog of applications).

\item On January 11th, I wrote to MIT asking them whether they have received
  my transcript. The Graduate Admissions Office replied saying that nobody
  from the department had received my transcript. I wrote to them
  giving the name of the person who had picked the courier, but they said that
  it was not anybody from the department. So I sent my transcript
  to MIT yet again (I did this on January 15th, again via DHL).

\end{itemize}

\subsection{Caltech}

Caltech had the option of sending the transcript electronically. I scanned
my original transcript and sent the scanned transcript over to Caltech.
However, since I was not sure about whether there would be any problems with
the scanned transcript, I anyway sent the full packet by courier.

Caltech indicated in its online tracking that it had received my
transcript electronically (unfortunately that was {\em after} I had already
sent the postal version).

\section{Summary}

\subsection{Expenses}

The procedure for applying abroad has costs of varying natures. Here
is a summary of the various cost headers:

\begin{enumerate}

\item Examinations: the general GRE, subject GRE and TOEFL cost 140,
  140 and 150 dollars respectively (totalling to 430 dollars -- the rates
  increase every year).

  Further, for the TOEFL, there is a 17 dollar additional change for
  every extra university booked after the first four, and for the
  general/subject GRE, there is a 15 dollar charge plus a service fee
  of 6 dollars (plus phone charges).

  If one uses a single phone call for the general and subject GRE,
  and one applies to eight universities, this comes to a total of roughly
  \$ 570. For me, the total came to this plus another \$ 66
  because I made the phone call for additional universities twice -- once
  for the general GRE and once for the subject GRE.

  Of course, if one chooses to cancel or reschedule an examination
  (due to unavailability of passport or some other reason) the cost
  goes up by the cancellation charges for that examination.

\item University application costs: The university application costs
  range from \$ 45 to \$ 90 (the lowest was Wisconsin-Madison and the
  highest was Harvard). Most are in the range of \$ 50 to \$ 70. The
  application fee has to be paid at the end of online submission -- so
  you can start filling out a form, get recommendations and then decide
  not to apply at the last minute. For me, the total across eight
  universities was around \$ 460.

\item Cost of sending the postal materials: DHL university express
  costs Rs. 975 per application, so the total for eight applications
  is Rs. 8000 (if you also throw in charges for commuting to the DHL
  center etc.) If you use registered Government post, the charges are
  somewhat less (something like Rs. 700 per application). DHL offers
  better online tracking facilities and is a bit faster.

\item Cost of preparatory materials for the GRE/TOEFL: Good preparatory
  books for the general GRE typically come in the range of Rs. 400,
  good preparatory books for the TOEFL are also in a similar range.
  In my case, I bought one book on the verbal component and one on the
  essay component, as well as one book on TOEFL.

  Procuring the subject GRE preparatory book is a more tricky task.
  This book has very few copies available in India, so if you see a copy,
  {\em buy it at once}. I had to order the book from Amazon,
  which levies an additional charge of 5 dollars for the book
  (the base cost being \$ 9. 50).

  The total cost of books, even at its worst, is unlikely to be more than 
  the equivalent of \$ 40.
\end{enumerate}


The total application costs (if applying to eight universities) thus
come to somewhere in the range of \$ 1100 (this is an upper limit).

\subsection{Timelines}

Applying to universities in the United States requires a lot of time,
patience and effort spread over a long duration, and in that sense, it
is painful and taxing. Roughly, these are what I would call ideal timelines
(of course, many of these things might change with the new pattern
being introduced for general GRE):

\begin{itemize}

\item As early as possible: Try to procure a copy of the book for
  subject GRE
\item April -- May: decide to apply; arrange for the passport if you
  don't have one

\item May: Make a tentative list of possible places and areas to apply
  for, start shooting off emails with specific queries and talking
  both to students in the universities you want to go to and
  professors in your own place for advice

\item May: have a look at the general GRE syllabus and assess your
  preparedness and the amount of time you need to be well-prepared. Book
  the GRE for some time in August-September.

\item June: Finalize a list of the top four among the universities
  you want to apply to.

\item July: Register for the subject GRE in November as soon as registration 
  opens up. Also register for the TOEFL after assessing preparedness
  (hopefully for some time in September)

\item June-July: Keep working on general GRE preparation (specially verbal,
  also a but of essays)

\item July-August: Start working on asking people if they are
  interested in giving recommendations in principles, also ask their
  help in getting a list of universities ; start filling out
  application forms and clarify doubts from seniors as soon as
  possible

\item August: Gear for and give the general GRE

\item September: Start working on speech for TOEFL and give the TOEFL
  examination

\item September: Fill in the names of recommenders in the online application
  forms and request them for specific recomendations

\item October: Work for the subject GRE. This is probably the most important
  of the examinations, so don't underestimate its value.

\item October: Register for extra universities in TOEFL

\item October: Start working on the Statement of Purpose and Resume;
  collect all necessary inputs for it

\item October: Finalize (and close) the list of universities to which
  you are applying

\item October: Get sealed transcripts (you may want to do this earlier
  if it takes time)

\item November: Give the subject GRE

\item November: Complete work on the Statement of Purpose and Resume;
  follow up and have the letters of recommendation completed

\item Submit the online applications for universities with early deadlines,
  then send the applications by post

\item December: Find your subject GRE score (by phone, costs \$10) so
  that you can fill it in in application forms; book extra
  universities.  If you want to optimize the phone call, you should
  make sure that all the early deadline universities were booked among
  the top four, so that you can defer this additional call to combine
  the General and Subject GRE.

\item December: Submit applications for the remaining universities.

  As for sending postal applications, there is a trade-off. If you
  want to attach a photocopy of the subject GRE score sheet to your
  application, wait till December 20th or so. However, if you delay
  this long, your application may get lost or misplaced in the
  Christmas week. So in that case, be prepared to have to send an
  application again.

\end{itemize}





\printindex

\end{document}
