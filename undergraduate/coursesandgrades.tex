\documentclass[a4paper]{amsart}

%Packages in use
\usepackage{fullpage}
%\usepackage{setspace} %To provide double spacing
%\usepackage{float,graphicx} %To allow images
\usepackage{hyperref} %To make links clickable -- only for PDF
%\usepackage{showkeys} %To show the list of labels
%\usepackage{index} %To construct an advanced index
\usepackage{complexity, vipul}

%Title details
\title{Courses attended and grades obtained}
\author{Vipul Naik}
\thanks{\copyright Vipul Naik, B.Sc. (Hons) Math and C.S., Chennai Mathematical Institute}

%List of new commands
\newcommand{\coursename}[1]{{\bf #1}{\small{(course name)}}}
\newcommand{\placename}[1]{{\bf #1}{\small{(place name)}}\index{#1}}
\newcommand{\bookname}[1]{{\em #1}{\small{(book name)}}}
\newcommand{\campname}[1]{{\bf #1}{\small{(camp name)}}\index{#1}}

\makeindex

\begin{document}
\maketitle

\begin{abstract}
  Here, I describe the courses I attended in each semester, and the
  grades obtained in each course. I give explanations for the grades
  obtained. For each course, I also describe my personal experience
  and the value I gained from the course. For optional or
  fast-forwarded courses, I give reasons for opting for or
  fast-forwarding the course.
\end{abstract}

\section{Course and grading system}

\subsection{Course structure}

\placename{Chennai Mathematical Institute} (CMI)
\footnote{\url{http://www.cmi.ac.in}} follows a semester system. Each
course is fitted into a semester.  Each semester is four months long.
The {\em odd semesters} are from August to November (or first week of
December) and the {\em even semesters} are from January to April.

CMI has courses in mathematics, physics and computer science. It does
not offer any humanities courses other than the compulsory courses for
first-year students. Students in CMI can also attend and credit
courses offered at the \placename{Institute of Mathematical Sciences} (IMSc)
subject to agreement with the student's advisor and with the
instructor offering the course. Some of the courses offered at CMI are
also by IMSc instructors.

Classes are held on weekdays (Monday to Friday). On average, a course
is allotted two lectures per week, with the running time being 16
weeks and a lecture duration being 75 minutes. Lectures missed due to
festivals or unavailability of the lecturer are compensated for in
free slots or on weekends. The total lecture time for a course is thus
approximately 35-40 hours.

\subsection{Degree requirement}

I am enrolled for the B.Sc. (Hons) Mathematics and Computer Science
course. The course duration is three years.  The completion
requirement is 26 credited courses, and all of these are used in the
Grade Point Average.  The student can additionally audit any number of
extra courses and can also receive a grade not counted towards the
grade point average. Further credited courses are not shown in the
official transcript given by the Madhya Pradesh Bhoj Open University,
but they are shown in the institute's internal transcript.
\footnote{CMI has recently been approved for the status of a
  degree-granting institution. Thus, we are likely to receive our
  degrees from CMI itself} There are 22 compulsory courses, one
additional required course in physics and 3 optional courses that can
be picked from any course in CMI or IMSc. The compulsory course
distribution is as follows:

\begin{enumerate}

\item First semester: 5 compulsory courses. Algebra I, Calculus I,
  Classical Mechanics I, English, Introduction to Programming I

\item Second semester: 5 compulsory courses. Algebra II, Calculus II,
  Discrete Mathematics, Economics, Introduction to Programming II

\item Third semester: 4 compulsory courses. Algebra III, Analysis I,
  Calculus III, Design and Analysis of Algorithms.

\item Fourth Semester: 3 compulsory courses. Analysis II, Computer
  Organization, Topology. One physics course also needs to be taken,
  the options being Statistical Mechanics, Electromagnetism, and
  Classical Mechanics II

\item Fifth semester: 3 compulsory courses. Algebra IV, Ordinary
  Differential Equations and Theory of Computation.  At least one
  optional course (other than the physics optional) must be completed
  in or before the fifth semester.

\item Sixth semester: 2 compulsory courses. Probability and
  Programming Language Concepts.

\end{enumerate}

A student can thus meet the degree requirements by taking four courses
per semester.

There are three kinds of courses a student may take:

\begin{enumerate}

\item {\bf Compulsory course}: A course that the student must take in
  the semester towards fulfilment of the degree requirement.

\item {\bf Fast-forward course}: A course the student needs to take in a later semester, but is crediting earlier.

\item {\bf Optional course}: A course the student does not need to
  take. A student may {\em credit} or {\em audit} an optional course.
  If a student credits a course, he/she has the option of including it
  in the CGPA.

\end{enumerate}

\subsection{Grading and evaluation}

The course is evaluated by the instructor, based on the following:

\begin{itemize}

\item Regularity of attendance, and submission of assignments (if set
  by the instructor)

\item Interest and responsiveness in class

\item Performance in the mid-semester and end-semester examination.
  The mid-semester examination is held in end September for the odd
  semesters and in the end of February for the even sesmters. The
  end-semester examination is held in the end of November to beginning
  of December for the odd semesters, and in the end of April for the
  even semesters.

\end{itemize}

The instructor gives a grade on a scale of 2 to 10, with the
correspondnce as follows:

\begin{tabular}{|l|r|l|}
  \hline
  Letter & number & interpretation\\
  \hline
  A & 10 & \\
  AB & 9 & \\
  B & 8 & \\
  BC & 7 & \\
  C & 6 & \\
  CD & 5 & \\
  D & 4 & \\
  E & 2 & Fail\\
  \hline
\end{tabular}

At the end of each semester, students receive a cumulative grade point
average of the credited courses taken so far, rounded off to the
second decimal place.

Further information is available at:

\url{http://www.cmi.ac.in//locallinks/evaluation.php}

\subsection{Question papers}

Question papers for some of the courses I attended are available at:

\url{http://www.cmi.ac.in/~ramprasad/qpapers/}

These are actually question papers for courses that my friend
Ramprasad attended. Since we are in the same batch, most of our
courses are common.

\section{First semester: August to December 2004}\label{sem1}

\subsection{Grade point summary}\label{sem1gps}

%no honorifics to instructors in the table :P
\begin{tabular}{|l|l|l|}
  \hline
  Course & Instructor & Grade \\
  \hline
  Algebra I & K.R. Nagarajan & A \\
  Calculus I & D.S. Nagaraj & A \\
  Classical Mechanics & P.P. Divakaran & B\\
  English & Shreekumar Varma & A\\
  Programming I (Haskell) & Madhavan Mukund & A \\
  \hline
\end{tabular}
\subsection{Algebra I}\label{algebra1}

\subsubsection{Instructor}

The course was taught by Professor K. R. Nagarajan. He is a mathematics
professor at CMI. He has just retired.

Email ID: \email{nagaraj@cmi.ac.in}

\subsubsection{Course contents}

The course focussed on Linear Algebra and Group Theory.  The textbooks
suggested and recommended in the course were:

\begin{enumerate}

\item \bookname{Algebra} by Michael Artin. This was the official textbook.
  The course targetted at covering Chapters 1-4 of this book.

\item \bookname{Topics in Algebra} by I.N. Herstein.

\item \bookname{Linear Algebra} by Hoffmann and Kunze.

\item \bookname{Abstract Algebra} by Dummit and Foote.  This was not recommended
  by the instructor but was used extensively by students.

\end{enumerate}

\subsubsection{My personal experience}

I had already read more {\bf group theory} than was needed for the course 
before coming to CMI. My reading had been from the book by Dummit and Foote
and from another old book.

I had picked up a little {\bf linear algebra} earlier
and, during the first month of the course, I
familiarized myself with 
the linear algebra that was to be covered in the course.

I spent most of my energy learning further group theory and abstract algebra.

I secured a grade of A.

\subsubsection{Learning and value gained}

Discussions inside and outside the classroom raised interesting questions
in group theory. In attempting to solve these questions, I exposed myself
to a lot of further group theory.

\subsection{Calculus I}\label{calculus1}

\subsubsection{Instructor}

The course was taught by Professor D.S. Nagaraj.

Email ID: \email{dsn@imsc.res.in}

\subsubsection{Course contents}

The course included the following:

\begin{enumerate}

\item Construction of the real number system.

\item Differentation and its applications. Taylor expansions was the only subtopic 
  which we had not covered in school

\item Basic definition of integration, and applications of integration.

\item The Fundamental Theorem of Calculus.

\end{enumerate}

\subsubsection{My personal experience}

I was regular with the course and got full scores in the examinations. I had no difficulties. I secured a grade of A.

\subsubsection{Learning and value gained}

During the course, I learnt about famous ``counterexamples'' in calculus,
and got a clearer picture about the difference between differentiable, continuously differentiable,
infinitely differentiable and real analytic functions.


\subsection{Classical mechanics}\label{classmech}

\subsubsection{Instructor}

The coruse was taught by Professor P.P. Divakaran. He is a visiting  
professor at CMI and at IMSc.

Email ID: \email{ppd@cmi.ac.in}

\subsubsection{Course contents}

The first half of the course contained basic vector algebra and laws of 
mechanics. The second half had a few challenging topics, such as:

\begin{itemize}
\item The equation for planetary motion, with full derivation.
\item The proof of Euler's theorem about the presence 
  of a fixed axis for every rotation in three 
  dimensions. I enjoyed the linking up with linear algebra.
\end{itemize}

The instructor did not follow any textbook. He warned us against using
Feynmann's books at this stage.
\subsubsection{My personal experience}

The course content was easy. However, the material was covered and presented in
a different style. Misled by the beginning of the course, I neglected the course requirements.
I failed to submit one assignment and 
messed up 2-3 questions on the final examination
paper. I believe these were responsible for a grade of B. 

\subsubsection{Learning and value gained}

I gained value from the course content as follows:

\begin{itemize}

\item Towards the end of the course, I began understanding how  Lie groups and 
  linear groups are used in physics.

\item The instructor presented vector cross 
  products using the tensor formalism. At the time I was reading tensor
  products in ring theory. The link between the mathematical notion
  of tensors and the physical notion was fascinating.

\end{itemize}

I also gained value from my experience with the course as follows:

\begin{itemize}

\item I resolved to be regular about class assignments. 

\item I decided to avoid the trap of judging the level of difficulty of a course based on the first few classes.

\end{itemize}

\subsection{English}\label{english}

\subsubsection{Instructor}

The course was taught by Mr. Shreekumar Varma. 

Email ID: \email{varma@shreevarma.com}

\subsubsection{Course contents}

Interactive English was a light course. Every class began with a talk by a student, followed by a question answer
session and an evaluation session where all other students graded the speaker and gave their  opinions.

The remaining time was devoted to literary reading and discussion. The literary texts we discussed were:

\begin{enumerate}

\item A short story: ``An occurrence at Owl Creek Bridge'' by Ambrose Bierce.

\item A part of a play: ``Hamlet'' by Shakespeare.

\item A poem ``Death be not proud'' by John Donne.

\item A poem ``Ode to a Nightingale''.

\end{enumerate}

\subsubsection{My personal experience}

I enjoyed the course, especially giving my talk and evaluating other people's talks. Through the talks, I also got
to know the other students better. The assignments and examinations were also a fun experience. I secured an A.

\subsubsection{Learning and value gained}

The course improved my ability to express myself, using the written and the oral media.

\subsection{Introduction to Programming I}\label{prog1}

\subsubsection{Instructor}

The course was taught by Professor Madhavan Mukund, who is on the permanent faculty of
CMI.

Email ID: \email{madhavan@cmi.ac.in}

\subsubsection{Course contents}

The course included:

\begin{enumerate}

\item Functional programming via Haskell

\item Algorithms (and basic O notation), 
  heuristics (in particular, search and traversal heuristics), data structures, dynamic programming,
  memoization and garbage collection

\item Programming assignments in Haskell that tested familiarity with the language and also the ability to develop
  algorithms and express them in a functional programming setup

\end{enumerate}

The course assignments are available at:

\url{http://www.cmi.ac.in/~madhavan/courses/programming04/}

The course has also been taught in later years, and the course assignments for this year are available at:

\url{http://www.cmi.ac.in/~madhavan/courses/programming06/}

\subsubsection{My personal experience}

I was regular with the course, in terms of attendance as well as assignments. I did well on the mid term
and final examinations and secured an A grade.

\subsubsection{Learning and value gained}

Before coming to CMI, I was interested mainly in mathematics. This first course helped me widen my interests
to computer science as well, specifically to the area of algorithms and programming languages. I read a lot on my
own about programming paradigms.

\section{Second semester}\label{sem2}

\subsection{Grade point summary}\label{sem2gps}

\begin{tabular}{|l|l|l|}
  \hline
  Course & Instructor & Grade \\
  \hline

  Algebra II & S. Ramanan & A \\
  Calculus II & -- & B \\
  Discrete Mathematics & Bharat Adsul & A\\
  Economics & Lakshmi Kumar & B\\
  Programming II (C) & S.P. Suresh & A \\
  \hline
\end{tabular}

This semester was fairly hectic for me because I also got involved with organizing the annual college fest
and the annual online programming contest. So, I did not do justice to some courses, particularly \coursename{Economics} and \coursename{Calculus II}.

\subsection{Algebra II}\label{algebra2}

\subsubsection{Instructors}

The bulk of the course was by Professor S. Ramanan. He went abroad during the last month of the course
and our classes at that time were taken by Dr. K.N. Raghavan and Dr. Amritanshu Prasad.

Email IDs:

\begin{itemize}

\item Professor S Ramanan: \email{sramanan@cmi.ac.in}

\item Dr. K.N. Raghavan: \email{knr@imsc.res.in}

\item Dr. Amritanshu Prasad: \email{amri@imsc.res.in}

\end{itemize}

\subsubsection{Course contents}

The course covered group theory, basic facts abour matrix groups, bilinear forms, and the representation
theory of finite groups. Textbooks used:

\begin{enumerate}

\item \bookname{Algebra} by Michael Artin. This was the recommended textbook.
  We covered Chapters 5,6,7, and 9 of this book.

\item \bookname{Topics in Algebra} by Herstein.

\item \bookname{Abstract Algebra} by Dummit and Foote.

\end{enumerate}

\subsubsection{My personal experience}

I enjoyed this course the most. I was familiar with many of the topics covered in the course, but the clear perspective
with which Professor Ramanan presented the topics showed me the relevance and significance of groups. 

Dr. Raghavan's clear and definite emphasis on the theorem and his style of sketching the proof was spellbinding.
I enjoyed his way of formulating the orthogonality relations.

I also enjoyed the last lecture by Dr. Amritanshu Prasad where he concretely calculated the character table of the
alternating group by proving it to be isomorphic to the icosahedral group.

I was regular in the course, in terms of attendance as well as assignments. I got full marks in both the examinations.

\subsubsection{Learning and value gained}

I learnt a lot, not only in the classroom, but also through
discussions with Professor Ramanan after the class.  I also got to
know Dr. Raghavan and Dr. Amritanshu Prasad with whom I had extensive
mathematical discussions later on.

\subsection{Calculus II}\label{calculus2}

\subsubsection{Instructors}

The course was taken by four people, two for the first half and two
for the second half. All of them were Ph.D. students from the
\placename{Ecole Normale Superieure}, and were in CMI as part of an
exchange programme.

\subsubsection{Course contents}

\subsubsection{My personal experience}

This course was a disaster for me. The course was taken by four people
in total, two covering the first half of the semester and two covering
the second half. I did well on the first half and was the highest
scorer on the mid semester examination. However, due to poor planning,
bad organization, and inability to communicate with the Fernch
teachers, I messed up the second ahlf of the course. Many others in
our batch suffered a similar problem. My final examination performance
was satisfactory but my poor attendance in between may have been a
factor explaining my poor grade.

\subsubsection{Learning and value gained}

I resolved to not let my equation with the instructors slip, and also to pay attention to a course, specially if there was a communication
problem with the instructors.

\subsection{Discrete Mathematics}\label{dmath}

\subsubsection{Instructor}

The course was taken by Dr. Bharat Adsul, who is a Research Fellow at CMI.

Email ID: \email{abharat@cmi.ac.in}

Dr. Bharat Adsul took this course. 

\subsubsection{Course contents}

The first half of the course focussed on basic combinatorics, and the second half focussed on graph theory.

Reference books:

\begin{itemize}

\item \bookname{Discrete Mathematics} by Norman Biggs

\item \bookname{Graph Theory} by Douglas B. West

\item \bookname{Graph Theory} by Harary

\end{itemize}

\subsubsection{My personal experience}

I was familiar with most of its contents due to my preparation for the
Olympiads.  However, I learnt a lot of new graph theory and became
interested in the subject. The examinations went well and I secured an
A grade.

\subsubsection{Learning and value gained}

I learned a lot of new graph theory and also became more interested in graph theory. I became keen to study graph algorithms.

\subsection{Economics}\label{economics}

\subsubsection{Instructor}

The course was taught by Dr. Lakshmi Kumar. She works at the \placename{Institute of Financial Management and Research} (IFMR).

Email ID: \email{lakshmik@ifmr.ac.in}

\subsubsection{Course contents}

Textbooks/sources used for the course:

\begin{enumerate}

\item \bookname{Introduction to Economics} by Samuelson and Nordhaus. This was the recommended textbook.

\item Lecture notes by George Mankiw. Most of our lectures were based on these lecture notes.

\end{enumerate}

The course covered parts of microeconomics as well as macroeconomics.

\subsubsection{My personal experience}

I have been interested in economics. However, I did not enjoy the
course much, due to the instructor's approach.  Towards the middle of
the course, I missed a few classes because of the college festival and
because I was keen not to compromise on my other subjects of study.  I
messed up many questions on the mid semester examination.

I became serious towards the end of the course and did well on the 
final examination. 

My overall grade point average was a B.

\subsubsection{Learning and value gained}

Learnings that I should have incorporated after the Physics (Classical
Mechanics) course were reiterated here. I realized that even if the
style od presentation of the course is not attractive at the time, I
should give my best to the course, because the opportunity comes only
once. I had lost not only in terms of the grade point but also in
terms of the learning value I could have gained from the course.

Nonetheless, the course furthered my interest in economics.  I later
contacted the instructor again, and talked to some students in IFMR
from CMI.  I also talked to a researcher at IFMR and got a broad
understanding of the research areas there.

\subsection{Introduction to Programming II}\label{prog2}

\subsubsection{Instructor}

The course was taken by Dr. S.P. Suresh, who later also took the Computer Organization course.

Email ID: \email{spsuresh@cmi.ac.in}

\subsubsection{Course contents}

This course focussed on programming in C.

The course was relatively light and covered applications of 
programming concepts learnt in the Haskell 
course (section \ref{prog1}) in a more conventional programming environment (C). 

\subsubsection{My personal experience}

I did the assignments regularly and secured an A grade.

\subsubsection{Learning and  value gained}

I became better at converting an algorithm into a working program. I
also learnt more about memory, space and time constraints, which had
been ignored in the Haskell course.


\section{Third semester}

I took the following courses:

\begin{tabular}{|l|l|l|}
  \hline
  Course & Instructor & Grade \\
  \hline
  Algebra III & K.R. Nagarajan & A\\
  Analysis I & Amritanshu Prasad & A\\
  Calculus III& Suresh Nayak & A\\
  Design and Analysis of Algorithms & K.V. Subramanyam & B\\
  Global Calculus$^*$ & S Ramanan & A\\
  Theory of Computation$^+$ & Narayan Kumar & A\\
  \hline
\end{tabular}

Star marked courses are optional, and plus marked courses are fast forward courses.

\subsection{Algebra III}\label{algebra3}

\subsubsection{Instructor}

The course was taught by Professor K. R. Nagarajan. He is a mathematics
professor at CMI. He has just retired. He is the same person who taught 
\coursename{Algebra I} (Section \ref{algebra1}).
Email ID: \email{nagaraj@cmi.ac.in}

\subsubsection{Course contents}

The course had a {\em standard part} and an {\em extra part},. The standard part was based on Artin's chapters
on Ring Theory.

The extra part included the following:

\begin{enumerate}

\item The theory of projective, injcetive and free modules

\item Basic algebraic number theory

\end{enumerate}

The books/references were same as for the previous algebra courses.
However, no specific references were given for the algebraic number
theory part. The book \bookname{Abstract Algebra} by Dummit and Foote
covers both these extra topics.

\subsubsection{My personal experience}

I found the course reasonably simple as I was already familiar with
bulk of the course content.  Some material such as algebraic number
theory, was new. But I was also attending a lecture series by
Professor Balasubramanian where these topics were covered in more
detail. I also had a few private discussions with Professor Nagarajan
on the Hilbert's nullstellensatz.

\subsubsection{Learning and value gained}

The course contents helped me brush up and organize my understanding of ring theory.

\subsection{Analysis I}\label{analysis1}

\subsubsection{Instructor}

The course was taken by Dr. Amritanshu Prasad. He is a faculty member at IMSc.

Email ID: \email{amri@imsc.res.in}

\subsubsection{Course contents}

The course contained the following:

Full course details are available at:
\url{http://www.imsc.res.in/~amri/analysis/}

\subsubsection{My personal experience}

This was an interesting and slightly challenging course. 
I submitted the weekly assignments regularly.
Although the subject matter was not tough,
the assignments contained thought provoking questions that helped me think
further. 

\subsubsection{Learning and value gained}


\subsection{Calculus III}\label{calculus3}

\subsubsection{Instructor}

The course was taken by Dr. Suresh Nayak. He is a faculty member at CMI.

Email ID: \email{snayak@cmi.ac.in}

\subsubsection{Course contents}

The course covered basic multivariable calculus. The main reference was \bookname{Calculus II} by Tom Apostol.

\subsubsection{My personal experience} 

The course required only a little effort and I managed an A by performing well
on the mid semester and end semester examinations.

\subsubsection{Learning and value gained}

The course exposed me formally to differentation in higher dimensions.
It helped me follow some parts of the \bookname{Global Calculus} course (section
\ref{globalcalculus}) by Professor S Ramanan.

\subsection{Design and Analysis of Algorithms}\label{algorithms}

\subsubsection{Instructor}

The course was taken by Professor K.V. Subramanhyam, who is a member of the permanent faculty at CMI.

Email ID : \email{kv@cmi.ac.in}

\subsubsection{Course contents}

The course covered a broad range of topics including: %but not limited to :)

\begin{enumerate}

\item Basic definition of algorithm and complexity measures for algorithms

\item Matroids and greedy algorithms

\item Data structures, specifically heaps and balanced binary search trees 

\item NP completeness

\item The AKS primality test

\end{enumerate}

The textbooks followed/recommended were:

\begin{enumerate}

\item \bookname{Introduction to Algorithms} by Cormen, Leiserston, Rivest and Stein. 

\item \bookname{Algorithms} by Kozen

\item \bookname{Algorithms} by Aho, Hocroft, and Ullmann

\end{enumerate}

The final examination was an {\em open book} examination.

\subsubsection{My personal experience}

In the beginning, I was overwhelmed with the course and also with the
other subjects and did not give the course my best.  However, towards
the end of the course, my interest in the subject of computer science
had increased and I regretted having neglected the course in the
beginning. By then, it was too late to rectify my poor course
performance, and I was stuck with a course grade of B.

During the end of the course, I also attended seminars conducted by the 
instructor and by others on recent breakthrough results in computer science.

\subsubsection{Learning and value gained}

The course sparked off my interests in complexity and led me to audit
a course in \coursename{Computational Complexity} the next semester.

It was also the last time I would neglect a course even for part of
the term. I resolved to now be very regular with all my course work
and to try to get benefit from the course while it was gonig on rather
than trying to learn the material later.

\subsection{Global calculus (optional)}\label{globalcalculus}

\subsubsection{Instructor}

The course was taken by Professor S Ramanan. He had also taken the Algebra II course (section \ref{algebra2}).

\subsubsection{Course contents}

The course text was the book \bookname{Global Calculus} by Professor Ramanan himself. In the first half of the course,
Professor Ramanan covered Chapters 1 and 2 of his book, with some deviations. In the second half, he covered parts
from the next four chapters. He also gave background material. During the course, he also introduced:

\begin{itemize}

\item Simplicial and singular cohomology at a a rudimentary level

\item The concept of gradation, filtration, and tensor product

\item The concept of Lie algebra and universal enveloping algebra

\end{itemize}

\subsubsection{Why I took the course}

The global calculus course was my fisrt credited optional course. The
course structure is designed for optionals to be taken in the third
year, but students have the flexibility to take optionals in the
second year itself.  I had first come across the global calculus book
by Professor Ramanan during the \coursename{Algebra II} course. On
hearing that the course was being offered this year and may not be
offered in future, I enrolled for it.

\subsubsection{My personal experience}


I attended the classes regularly. There were certain places where I could not
understand the course material, but by and large, Professor Ramanan assumed
no prerequisites and I followed the gist of the lessons. This course was 
helpful to me in two later courses: \coursename{Elementary Differential Geometry} and \coursename{Abelian varieties}.

I secured an A in the course.

\subsection{Theory of Computation}\label{toc}

\subsubsection{Instructor}

The course was taught by Dr. K. Narayan Kumar, a Fellow at CMI.

Email ID: \email{kumar@cmi.ac.in}

\subsubsection{Course contents}

The main course text is \bookname{Automata and Computability} by
Dexter C. Kozen. The course covers:

\begin{enumerate}

\item Regular languages and finite-state automata

\item Context-free languages and pushdown automata

\item Turing machines, recursive languages and recursively enumerable languages

\item Notions of reduction; proofs of undecidability

\end{enumerate}

\subsubsection{Why I fast-forwarded the course}

This is a compulsory subject in the fifth semester but I decided to
fast-forward it to the third semester itself to satisfy my curiosity
about regular languages and context-free languages. This course, along
with the Algorithms course, was crucial in building my interest in the
subject of Computer Science.

\subsubsection{My personal experience}

I enjoyed the course. I was regular with the classes and also
routinely discussed problems from the text with friends.

I secured an A in the course.

\subsubsection{Learning and value gained}

At the end of the course, I achieved what I had set out to achieve:
get a better idea of regular and context-free languages. In addition,
I also understood basic premises of computational complexity such as
the notion of resource-bounded complexity classes (space and time).
This motivated me to take up a \coursename{Computational Complexity} course
next semester.

I also furthered my understanding of automata and their relation to
algebra/logic by auditing an informal course in \coursename{Automata, Logic, Games and Algebra} next semester.

\section{Fourth semester}

\subsection{Courses and grades}

\begin{tabular}{|l|l|l|}
  \hline
  Course & Instructor & Grade\\
  \hline
  Analysis II & Suresh Nayak & A\\
  Computer Organization & S.P.Suresh & A\\
  Electromagnetism I & K.S. Balaji & A\\
  Game Theory$^*$ & T. Parthasarathy & A\\
  Topology & V. Balaji & A\\
  Programming Language Concepts$^+$ & Madhavan Mukund & AB\\
  Analytic Number Theory & R. Balasubramanian & audit\\
  Computational Complexity & V. Arvind & audit\\
  Automata, Logic, Games and Algebra & K. Narayan Kumar & audit\\
  \hline
\end{tabular}

Star marked courses are optional, and plus marked courses are fast forward.
\subsection{Analysis II}

\subsubsection{Instructor}

The course was taught by Dr. Suresh Nayak. He was the same person who
taught \coursename{Calculus III} (section \ref{calculus3}).

Email ID: \email{snayak@cmi.ac.in}

\subsubsection{Course contents}

The course covered complex analysis. Recommonded textbooks were:

\begin{enumerate}

\item \email{Complex Analysis} by John Conway. This was the main text.

\item \email{Complex Analysis} by Lars Ahlfors.

\end{enumerate}

%give a list of topics covered

\subsubsection{My personal experience}

The first half of the course was easy, and I did well on the mid semester examination. 
However, the second half contained a lot of new material for me. 
Cautioned by my previous mistakes, I buckled up in the second half and studied the course material thoroughly.
I did well on the final paper.

I secured an A.

\subsubsection{Learning and value gained}

A poor knowledge of complex analysis had been a handicap for men in
many areas of mathematics that I was trying to read and explore. After
completing the course, I took private classes from Professor Ramanan
in some particular topics of complex analysis (such as Mittag-Lefler
theorem, divisors, riemann-Roch theorem). These classes also served as
foundation for the \coursename{Abelian varieties} course that I took
in my fifth semester.

Techniques of integration to count the number of zeroes and poles
were also used in the \coursename{Elliptic Curves and Modular Forms}
course that I was simultaneously auditing.

\subsection{Computer organization}

\subsubsection{Instructor}

The course was taken by Dr. S. P. Suresh. He is the same person who took the Introduction to Programming II course
(section \ref{prog2}).

Email ID: \email{spsuresh@cmi.ac.in}

\subsubsection{Course contents}

The two main textbooks for the course were:

\begin{enumerate}

\item \bookname{Digital Design} by Morris M. Mano. The first 8 chapters were in the course.

\item \bookname{Computer Organization and Design} by Patterson and Hennessy. We did three chapters of this text. %fillin

\end{enumerate}

From Digital Design, the main topics were:

\begin{enumerate}

\item Combinational logic

\item Design of synchronous circuits using latches and flipflops

\end{enumerate}


The main course topics from Patterson and Hennessy were:

\begin{enumerate}

\item MIPS assembly language programming (Chapter 2/3 of Patterson and Hennessy).

\item The processor: datapath and control (Chapter 4/5 of Patterson and Hennessy).

\item Pipelining (Chapter 6 of Patterson and Hennessy).

\end{enumerate}


\subsubsection{My personal experience}

The course was light. I did not have to study for the first part of the course, as i had covered Boolean circuits in school.
The second part required some new stuff that I read.

I secured an A.

\subsection{Electromagnetism I}

\subsubsection{Instructor}

The course was taught by Dr. K. S. Balaji %honorific

%email %fillin

\subsubsection{Course contents}

The main text was \bookname{Introduction to Electrodynamics} by
Griffiths. We covered the pre relativity part of the text omitting a
few calculations in spherical and cylindrical coordinates. The
instructor also taught relativity in a style somewhat different from
Griffiths.

\subsubsection{My personal experience}

Course was a little tough and demanding, particularly since it is
outside my direct line of interest.  However, I gave it reasonable
effort and submitted almost all my assignments. I also did well on
the mid-semester and end-semester examinations.

I secured an A in the course.

\subsubsection{Learning and value gained}

The course exposed me to the beauty of relativity and its physical
importance.  Also, the computational focus of the course gave me
hands-on practice in multivariable calculus, that I had already
studied in \coursename{Calculus III}.

\subsection{Game theory (optional)}

\subsubsection{Instructor}

The course was taught by Professor T. Parthasarathy. His email ID is \email{pacha@cmi.ac.in}.

\subsubsection{Course contents}

The instructor covered the following:

\begin{enumerate}

\item Zero and nonzero sum two player noncooperative games with solution
  concepts

\item Multi player nonzero sum cooperative games and concept of equilibrium

\item Linear programming, the LCP and their use in determining solutinos and equilibria.

\item Stochastic games and solution concepts for these

\end{enumerate}

\subsubsection{Why I took this course}

During the \coursename{economics} course (section \ref{economics}) in
the second semester, I learnt about the use of game theory in solving
economic problems. I visited the \placename{Institute of Financial Management and
Research} and talked to Dr. Somdeb Lahiri working there. He explained
basic concepts of optimization and mathematical programming. Based on
these, I read up a book on game thory in economic strategy.

On hearing that a game theory was being offered this semester, I decided to
take the course.

\subsubsection{My personal experience}

I enjoyed the course. I also read up a lot of other stuff during the course.
I secured an A.

\subsubsection{Learning and value gained}

The course gave me a clear udnerstnading of formulating problems in
the language of games. It also gave me practice in the use of linear
equations and inequations to model problems. I began to see the
correlation between the properties of matrices and the properties of
games.

Towards the end of the course, I became particularly interested in evolutionaril stable strategies. I read a number of books in related areas, including
popular books such as \bookname{Selfish Gene}.

\subsection{Topology}\label{topology}

\subsubsection{Instructor}

The course was taught by Professor V. Balaji. He is a permanenta faculty member of CMI.

Email ID: \email{balaji@cmi.ac.in}

\subsubsection{Course contents}

The first half of the course was on point set topology. The instructor did not follow a course book
but recommended two texts:

\begin{enumerate}

\item \bookname{Topology} by James Munkres.

\item \bookname{Lecture notes of elementary topology or differential geometry} 
  by I.M.Singer and J.A. Thorpe.

\end{enumerate}

The lecture notes of Balaji's course were in greater proximity to
Singer and Thorpe.

The second half covered fundamental groups and covering spaces from an
algebraic geoemtry perspective.  This was fairly different from the
textbook perspective and followed ideas of Bourbaki and Grothendieck.

\subsubsection{My personal experience}

The first half was plain sailing. In the second half, I found a few
difficulties, but resolved them.

I secured an A in this course.

\subsubsection{Learning and value gained}

I had already studied fundamental groups and covering spaces from
Munkres.  The instructor's style of defining covering spaces and
fundamental groups was striking and helped me understand the notions
from a new perspective. I hope it will help me when I undertake a
detailed study of algebraic geometry.

\subsection{Programming Language concepts}

\subsubsection{Instructor}

The course was taken by Professor Madhavan Mukund. He is a permanent faculty member of Computer Science at CMI.
He also taught Programming I (section \ref{prog1}).

\subsubsection{Course contents}

Professor Madhavan Mukund follows his own lecture notes %fillin website link

The course covered:

\begin{enumerate}

\item The programming language Java and concepts elucidated in Java.

\item The lambda calculus and functional programming. An overview of Haskell

\item Relational programming and a quick look at Prolog

\item Scripting languages and a quick look at Perl

\end{enumerate}

\subsubsection{Why I fast forwarded the course}

This is a sixth semester compulsory course. I wanted to do it in the fourth semester because I was keen on learning
Java as well as getting familiar with the lambda calculus. 

\subsubsection{My personal experience}

The course was fun and challenging and met my objectives.

I secured an AB.

\subsubsection{Learning and value gained}

I met my learning objectives of getting a hang of Java. I also got to learn Perl and Prolog.

\subsection{Analytic number theory: Elliptic curves and modular forms}

\subsubsection{Instructor}

The course was taught by Professor R. Balasubramanian. He is the director of IMSc and is known as an eminent
number theorist.

\subsubsection{Course contents}

The course concentrated on elliptic curves and modular forms. The contents were:

\begin{enumerate}

\item Basic results on elliptic curves: Weierstrass normal form, group structure on elliptic curves, Lutz Nagell theorem,
  Mordell Weil theorem. Mazur's theorem.

\item Modular forms, cusp forms, modular functions

\item Characters, stroke operators, and twisted modular forms

\end{enumerate}

\subsubsection{Why I took the course}

I have been fascinated with elliptic curves since I read about them during my Olympiad preparation. I took the course to learn
more about them
\subsubsection{My personal experience}

Since I was only auditing the course, I did not give it top priority.
The course covered elliptic curves, and I was able to follow it in the
class.  It also went into modular forms, some of which I did not
follow at the time but understood later. I did not give the final
examination but had a look at the examination paper.

\subsubsection{Learning and value gained}

The course made me appreciate the importance of complex analysis in number
theory, and also introduced me to congruence subgroups.

Whatever I learnt about elliptic curves and modular forms in this
course helped me greatly in the coming \campname{Microsoft Research
  Summer School on Algorithms, Complexity and Cryptography} and in the
coming \coursename{Abelian varieties} course. In fact, the Abelian varieties
course was a natural continuation to this course.
\subsection{Computational Complexity}

\subsubsection{Instructor}

The course was taught by Professor V. Arvind, head of the Theoretical Computer
Science (TCS) department at IMSc. It was held in IMSc.

Email ID: \email{arvind@imsc.res.in}

\subsubsection{Course contents}

The course covered the following:

\begin{itemize}

\item Basic familiarity with important complexity classes: $\L$, $\NL$, $\P$, $NP$, $\PSPACE$, $\EXP$. Space and time hierarchy theorems, for deterministic and nondeterministic
  classes. The Immermann-Szelepscenyi theorem which states that $\NSPACE = coNSPACE$.

\item Polynomial hierarchy in detail. The $\Sigma$, $\Pi$, and $\Delta$ hierarchies.

\item Randomization in polynomial time algorithms. Complexity classes: $\RP$, $\coRP$, $\BPP$, $\ZPP$. 

\item Circuit complexity. The $\AC$ and $\NC$ complexity classes. The proof 
  that parity is not in $\AC^0$.

\item Proof of $IP = PSPACE$.

\end{itemize}

\subsection{Why I took the course}

In the algorithms course (Section \ref{algorithms}) I enjoyed learnign about NP completeness. The instructor also gave exercises
that led to Savitch's Theorem. Further, some seminars about Logspace further fuelled my interest in space and time complexity.

These factors led me to audit Professor Arvind's foundational course in complexity course.

\subsection{My personal experience}

Initially, I was irrregular with the classes, but as the material started getting more interesting, I became more regular. I learnt
a lot of stuff in the course. The final examination was a {\em take home} examination. Although I did not give the examination, I attempted
the questions and discussed the solutions with others after the examination.
 
\subsection{Learning and value gained}

I got a clearer idea of computational complexity. I hope this will help me when I undertake a deep study of algebraic computational complexity.

My learning in this course helped me in the Microsoft Research Summer School that I attended after finishing my second year.

\subsection{Automata, logic, games and algebra}

This informal course covered many interesting results and proofs related to automata. 

\section{Fifth semester}

\subsection{Courses (without grades)}

\begin{tabular}{|l|l|l|}
  \hline
  Course & Instructor & Grade\\
  \hline
  Algebra IV & R. Sridharan & A\\
  Intro to Abelian varieties & S Ramanan & A\\
  Ordinary differential equations & R Srinivasan & A\\
  Rep Theory of Finite Groups & S Kannan & A\\
  Elementary Diff Geom & C.S. Aravinda & A\\
  Complexity II & V. Arvind & Audit (optional) \\
  \hline
\end{tabular}

\subsection{Algebra IV}

\subsubsection{Instructor}

The course was taught by Professor R Sridharan. His email ID is \email{rsridhar@cmi.ac.in}

Professor Sridharan went for the International Congress of
Mathematicians from August 20th -- September 20th.  Professor K. R.
Nagarajan taught us for that duration, and was also responsible for
administering and evaluating the mid-semester examination paper.

\subsubsection{Course contents}

The course covered the following:

\begin{itemize}

\item {\bf Theory of transcendental extensions}: Introduction to function fields and transcendental extensions;
  Steinitz' theorem and Luroth's theorem.

\item {\bf Theory of finite Galois extensions}: Concept of separable extension, normal separable extension,
  Galois group of an extension, existence of normal closures, fundamental theorem of Galois theory.

\item {\bf Theory of infinite Galois extensions}: Concept of profinite groups and the Krull topology, fundamental
  theorem for infinite Galois theory.

\item {\bf Kummer extensions and radical extensions}: Kummer theory, the proof that Kummer extensions are precisely
  the Abelian extensions, the proof that radical extensions are precisely the solvable extensions.
  Generalization to the characteristic $p$ case by including Artin-Schreier extensions.

\item {\bf Constructibility}: Characterization of constructible numbers, constructibility of $n$-gons, and cutting
  the Bernoulli lemniscate into equal parts.

\end{itemize}

\subsubsection{My personal experience}

I was sincere in attending the lectures, though I did not get as much
time as I would have liked to study the material at home. I plan to
read more on topics covered in the course in the months of January and
February.

I secured an A in the course.

\subsubsection{Learning and value gained}

I was already familiar with the Fundamental Theorem of Galois theory
and infinite Galois extensions. The proof of Luroth's theorem and the
connections with the study of curves was new and exciting stuff for
me. Although I had earlier seen Kummer theory, the course provided me a rigourous
understanding of the subject and of all the proofs.

Towards the end of the course, the instructor mentioned the {\em inverse Galois problem}.
He also explained how the work we had done so far had settled the inverse Galois problem for Abelian groups.
The course has stimulated me to learn more about inverse Galois theory
and I hope to pursue these interests in the holidays or next semester.

\subsection{Introduction to Abelian varieties}

\subsubsection{Instructor}

The course was taught by Professor S Ramanan. His email ID is \email{sramanan@cmi.ac.in}. Professor Ramanan had also
taught me Algebra II(section \ref{algebra2}) and Global Calculus (section \ref{globalcalculus}).

\subsubsection{Why I took this course}

I had enjoyed both of Professor Ramanan's earlier courses and was fond of his teaching style.
I was also keen to learn algebraic geometry and Abelian varieties, particularly after the flavour
given by Professor Balaji in his topology course (section \ref{topology}). Further, the course in elliptic curves
and modular forms that I had audited last semester had made me eager to learn more about Abelian varieties,
which generalized elliptic curves.

\subsubsection{Course contents}

Professor Ramanan covered the following in his course:

\begin{enumerate}

\item Lattices in complex vector spaces, quotient of a vector space by
  a lattice

\item Correspondence between elliptic curves and lattices; explicit
  description of the fundamental domain and use fo the $j$-invariant
  to correspond with $\C$.

\item Definition of cocycle, coboundary, and factor of automorphy; study
  of modular forms.

\item Hochschild-Serre spectral cohomology (a special case)

\item Study of $\theta$ functions; the idea of generalizing meromorphic
  functions to holomorphic sections of line bundles.

\item A little flavour of {\em Geometrie Algebraique Geometrie Analytique} or
  GAGA.

\item Notion of Riemann form; its natural occurrence as alternatization
  of Chern class of a factor of automorphy.

\item Study of symmetric and alternating bilinear forms over arbitrary
  integral domains.

\item Very ample and ample line bundles; effective divisors, and
  Lefschetz's theorem.

\item Riemann-Roch theorem for complex curves and some consequences
\end{enumerate}

I plan to put up more information on the course at:

\url{http://www.cmi.ac.in/~vipul/courses/abelianvarieties/}

\subsubsection{My personal experience}

I was sincere about attending the lectures, and tried to read as much
more as I could. I went through William Stein's lecture notes and
Samuel Grushevsky's page. Abelian varieties seems to be an area where
I might eventually do research.

Both the mid-semester and the end-semester examinations were take-home,
and I did well on them. I secured an A.

\subsubsection{Learning and value gained}

\subsection{Ordinary differential equations}

\subsubsection{Instructor}

The course was taught by Dr. R. Srinivasan. His email ID is \email{vasanth@cmi.ac.in}.

\subsubsection{Course contents}

The main textbook for the course was \bookname{Differential Equations With Applications and Historical Notes} by George F Simmons.
Topics for the course were:

\begin{enumerate}

\item First Order Equations (Chapter 2 of Simmons)

\item Second Order Linear Equations (Chapter 3 of Simmons)

\item Qualitative Properties of Solutions (Chapter 4 of Simmons)

\item Power Series Solutions and Special Functions (Chapter 5 of Simmons)

\item Some Special Functions of Mathematical Physics (Chapter 8 of Simmons)

\item Laplace transforms (Chapter 9 of Simmons)

\item Systems of First-order linear equations (Chapter 10 of Simmons)

\item Nonlinear equations (Chapter 11 of Simmons)

\item Calculus of variations (Chapter 12 of Simmons)

\end{enumerate}

\subsubsection{My personal experience}

I was regular with classes, and did reasonably on  the examinations.

I secured an A.

\subsubsection{Learning and value gained}

Material done in the first half of the course was largely a repetition
of what I had learned as part of the high school syllabus and in extra
reading at the time. The second half of the course covered interesting
topics such as the Lapalce transform, nonlinear equations, and the
calculus of variations, that I enjoyed studying.

The second half of the course covered proofs of existence and
uniqueness theorems for differential equations. I liked the way
constructive approximations techniques could be used to provide
abstract existence and uniqueness proofs. I believe these proof
techniques will be important in the study of partial differential
equations and of differential operators, that I plan to undertake
soon.

\subsection{Representation theory of finite groups}

\subsubsection{Instructor}

The course was taught by Professor Senthamarai Kannan. His email ID is \email{kannan@cmi.ac.in}.

\subsubsection{Course contents}

The course followed \bookname{Linear Representations of Finite Groups} by
J. P. Serre. The topics covered include:

\begin{enumerate}

\item Orthogonality results

\item Tensor products of representations

\item Artin-Wedderburn Theorem and its consequences for group representations

\item Frobenius reciprocity

\item Mackey's lemma and Mackey's irreducibility criterion

\item Brauer's induction theorem

\item Rationality questions and Schur indices

\end{enumerate}

I plan to put up more information on the course at:

\url{http://www.cmi.ac.in/~vipul/courses/reptheoryoffinitegroups/}
\subsubsection{My personal experience}

I was sincere with the coursework and assignments. The mid-semester
examination was hard but I did reasonably well on it. I solved all
questions in the end-semester examination.

I also gave a much-appreciated Student Seminar on the Artin Induction Theorem.

\subsubsection{Learning and value gained}

I enjoyed the classes and particularly the mid-semester and end-semester
examination papers, which led me to explore the subject further.

The course has given me the foundation in rationality questions
as well as in induction theorems, and I hope to build further on it
by reading/learning more in representation theory.

\subsection{Elementary Differential Geometry}

\subsubsection{Instructor}

The course was taught by Dr. C.S.Aravinda (Email ID:
\email{aravinda@cmi.ac.in}), a faculty member of CMI.

\subsubsection{Course contents}

The course covered the following:

\begin{enumerate}

\item Curves in $\R^3$, Frenet-Serret equations 

\item Cartan structural equations in $\R^3$
\item Classification of surfaces, beginning with the study of regular
  surfaces in $\R^3$, and leading up to ``Theorema Egregium'' and the
  ``Gauss-Bonnet theorem''.

\end{enumerate}

There were also a number of student seminars. I myself gave a seminar
on the Whitney Embedding Theorem.

Course details are available at:

\url{http://www.cmi.ac.in/courses/Aug2006/EDG/}

\subsubsection{Why I audited the course}

I had been interested in differential geometry for a long time, particularly
after Ramanans' \coursename{Global Calculus} course (section \ref{globalcalculus}) and after a lecture series on Differential Geometry that I had attended
in \placename{Tata Institute of Fundamental Research}. I was keen
to formally study the subject. Aravinda's course was a natural choice.

More about my own interest in differential geometry is available at:

\url{http://www.cmi.ac.in/~vipul/areasofinterest/differentialgeometry.pdf}

\subsubsection{My personal experience}

I was regular with the coursework and scored well on the mid-semester
and end-semester examinations. I also enjoyed giving my Student Seminar
on  the Whitney Embedding Theorem, and my seminar was much appreciate
by the instructor and by other students.

Although I only audited the course, I gave all the examinations and
was assigned a final grade of A.

\subsubsection{Learning and value gained}

I got a thorough understanding of differential geometry, its computational
flavour, and the way it ties in with other subjects. I also attended
a seminar by Dr. Aravinda on Perelman's recent proof of the long-outstanding
Poincare conjecture.

I have become interested in studying the classification of manifolds
of higher dimensions. I intend to take a course in
\coursename{Riemannian geometry} by Dr. Aravinda next semester, and
also study papers on Ricci flows and Hamilton's theorem.

\subsection{Computational complexity II}

\subsubsection{Instructor}

The course was taught by Professor V. Arvind (email ID:
\email{arvind@imsc.res.in}) of IMSc. He had also taught the first
course on \coursename{Computational Complexity}.

\subsubsection{Course contents}

The following topics were covered:

\begin{enumerate}

\item Pseudorandom gemerators

\item Error-correcting codes and extractors

\item Expander graphs

\item PCP theorem

\item Expander codes

\item Monotone circuit lower bounds for clique
 
\end{enumerate}

Course notes are available at:

\url{http://www.cmi.ac.in/~ramprasad/lecturenotes/}

\subsubsection{Why I audited the course}

I had enjoyed the first \coursename{Computational Complexity} course,
and wanted to learn more about complexty classes. The
\campname{Microsoft Research Summer School on Algorithms, Complexity
  and Cryptography} had also  introduced me to error-correcting codes
and their importance in cryptography. Further, the course coverage
included expander graphs, which is related to ideas of generating
sets for groups and Cayley graphs of groups.

\subsubsection{My personal experience}

I attended the classes regularly, barring a few lectures I missed on
the PCP theorem. I submitted scribed notes for a part of the course, even
though I was only auditing the course. I also gave a seminar
on a paper on hard-core measures. This paper also used some ideas
that I had encountered earlier in a \coursename{Game Theory} course.

I have not yet seen the end-semester examination paper but I intend to
attempt it after reviewing the lecture notes.

\subsubsection{Learning and value gained}

I gained an understanding of expander graphs, that I think will
prove useful in a later study of generating sets of groups.

I also got a good feel for probability, including the application
of Chernoff bounds. This is likely to help me in my
\coursename{Probability} course next semester.

The course also gave me a clearer idea of notions of ``randomness''
and ``derandomization''.

\section{Sixth semester}

I intend to take the following courses:

\begin{tabular}{|l|l|l|}
  \hline
  Course & Instructor & Plan\\
  \hline
  Probability & P. Vanchinathan & Compulsory\\
  Optimization Techniques & T. Parthasarathi & Credit\\
  Lie-theoretic methods & Alladi Sitaram and & Credit\\
  in Analysis & Amritanshu Prasad & \\
  Riemannian geometry & Murali K. Vemuri & Credit\\
  Algebra and Computation & V. Arvind & Credit\\
  \hline
\end{tabular}

\subsection{Probability}

\subsubsection{Instructor}

The instructor is Dr. P Vanchinathan. His email ID is \email{vanchi@cmi.ac.in}

\subsubsection{Course contents}

\subsection{Optimization techniques}

\subsubsection{Instructor}

The instructor is Professor T Parthasarathy. His email ID is \email{pacha@cmi.ac.in}.
He also taught the \coursename{Game Theory} course.

\subsubsection{Course contents}

\subsubsection{Why I plan to take this course}

I have been interested in problems of optimization for a long time. I
have done self-study of books on Linear and Nonlinear programming, and
also encountered them in Parthasarathy's \coursename{Game Theory}
course. I also came across optimization problems in the context of economics.

I enjoy Parthasarathy's teaching style and hope to learn a lot in this
course.

\subsection{Lie-theoretic methods in analysis}

\subsubsection{Instructor}

There are two instructors for the course: Professor Alladi Sitaram and Dr. Amritanshu Prasad
(\email{amri@imsc.res.in}).

\subsubsection{Course contents}

Sitaram plans to use the book \bookname{A First Course in
  Representation Theory and Linear Lie Groups} by Bagchi, Madan,
Sitaram and Tiwari. The prerequisites are elementary real analysis
(including some Fourier analysis) and functional analysis.

\subsubsection{Why I plan to take this course}

I have studied and been interested in the representation theory of Lie groups for a long time.
Also, I have not taken any courses with a strongly analytic flavour to them, and I hope that 
this course helps me understand Lie representation theory from an analytic viewpoint.

\subsection{Riemannian geometry}

\subsubsection{Instructor}

The instructor for the course is Dr. Murali K. Vemuri. His email ID is
\email{mkvemuri@cmi.ac.in}.

\subsubsection{Course contents}

\subsubsection{Why I plan to take this course}

In my fifth semester, I had audited Dr. Aravinda's
\coursename{Elementary Differential Geometry} course.  Though I was
only auditing the course, I gave all the examinations and got an
unofficial grade of A.  Dr. Aravinda encouraged me to take a
Riemannian geometry course the next semester, which he had initially
been planning to take himself.

The elementary differential geometry course ended with the general
notion of ``geometric surface'' and the properties of geometric
surfaces. Riemannian manifolds are higher-dimensional generalizations of geometric
surfaces, and I am keen to learn about them.

\subsection{Algebra and computation}

\subsubsection{Instructor}

The instructor for this course is Professor V. Arvind. His email ID is
\email{arvind@imsc.res.in}. 

\subsubsection{Course contents}

I am not aware of the global plan for Professor Arvind's course. He is
currently studying algorithms for problems in permutation groups,
and the relation with important complexity classes like Graph Isomorphism.

\subsubsection{Why I plan to take this  course}

Having audited two courses on computational complexity under Professor
V. Arvind, I was keen to attend the new course being offered by him.
In the first class, Professor Arvind covered many group-theoretic
questions from an algorithmic viewpoint. I have been interested in
these questions for a long time and look forward to understanding them
properly in this course.



\printindex

\end{document}
