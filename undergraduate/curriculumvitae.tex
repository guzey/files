\documentclass[a4paper]{amsart}

%Title details
\usepackage{fullpage, hyperref}
\title{Curriculum Vitae: Vipul Naik}
\author{Vipul Naik}

\begin{document}
\maketitle
%\tableofcontents

\section{General information}

\subsection{Basic information}

My name, date of birth and other information:

\begin{tabular}{|l|l|}
  \hline
  Name & Vipul Naik\\
  Date of birth & April 23, 1986\\
  Current occupation & Student (B.Sc. Math 3rd Year) \\
  Course & B.Sc. (Hons) Mathematics\\
  Expected date of completion & July 2007 \\
  Institution & Chennai Mathematical Institute\\
  \hline
\end{tabular}

\subsection{Contact information}

Address for correspodence:

\begin{quote}
  606, RMV Clusters Phase 2 Block 4,\\
  Devi Nagar, Lottegollahalli\\
  Bangalore, India\\
  PIN: 560 094\\
  Telephone number: +91 80 65676775\\
\end{quote}

Email ID: {\tt vipul@cmi.ac.in}
\section{Academic history}

\subsection{Quick overview}

I am currently in the third year of a three-year programme of B.Sc.
(Hons) in Mathematics at the Chennai Mathematical Institute. Given below
are my aggregate scores at important turning points:

\begin{tabular}{|l|l|l|}
  \hline
  Level & Year  & Score\\
  \hline
  Class 10 (secondary) & 2002 & 89\% (100\% in mathematics)\\
  Class 12 (senior secondary) & 2004 & 91.4\% (100\% in mathematics)\\
  B.Sc. (first four semesters) & 2007 & 9.59/10 (CGPA)\\
  \hline
\end{tabular}

\subsection{Undergraduate course details}

In Chennai Mathematical Institute, a grade point is awarded in each
subject out of $10$. A grade of $A$ corresponds to $10$ out of $10$, a
grade of $AB$ corresponds to $9$ out of $10$, while a grade of $B$
corresponds to $8$ out of $10$.

List of courses, with instructor name and grades:

\begin{tabular}{|l|l|l|}
  \hline
  Course & Instructor & Grade \\
  \hline
  \multicolumn{3}{|c|}{First semester}\\
  \hline
  Algebra I & K.R. Nagarajan & A \\
  Calculus I & D.S. Nagaraj & A \\
  Classical Mechanics & P.P. Divakaran & B\\
  English & Shreekumar Varma & A\\
  Programming I (Haskell) & Madhavan Mukund & A \\
  \hline
  \multicolumn{3}{|c|}{Second semester}\\
  \hline
  Algebra II & S. Ramanan & A \\
  Calculus II & -- & B \\
  Discrete Mathematics & Bharat Adsul & A\\
  Economics & Lakshmi Kumar & B\\
  Programming II (C) & S.P. Suresh & A \\
  \hline
  \multicolumn{3}{|c|}{Third semester}\\
  \hline
  Algebra III & K.R. Nagarajan & A\\
  Analysis I & Amritanshu Prasad & A\\
  Calculus III& Suresh Nayak & A\\
  Design and Analysis of Algorithms & K.V. Subramanyam & B\\
  Global Calculus$^*$ & S Ramanan & A\\
  Theory of Computation$^+$ & Narayan Kumar & A\\
  \hline
  \multicolumn{3}{|c|}{Fourth semester}\\
  \hline
  Course & Instructor & Grade\\
  \hline
  Analysis II & Suresh Nayak & A\\
  Computer Organization & S.P.Suresh & A\\
  Electromagnetism I & K.S. Balaji & A\\
  Game Theory$^*$ & T. Parthasarathy & A\\
  Topology & V. Balaji & A\\
  Programming Language Concepts$^+$ & Madhavan Mukund & AB\\
  Analytic Number Theory & R. Balasubramanian & audit\\
  Complexity Theory & V. Arvind & audit\\
  Automata, Logic, Games and Algebra & K. Narayan Kumar & audit\\
  \hline
\end{tabular}

Star-marked courses are optional, courses marked with a $+$ are
fast-forwarded courses intended for a later semester.

Information about the evaluation and grading system in CMI is available at:

\url{http://www.cmi.ac.in//locallinks/evaluation.php}

\subsection{Summer camps}

\begin{enumerate}

\item Summer camp at the {\bf Institute of Mathematical Sciences} from May
  9th to June 17th, 2005. The topic was ``Groups, Representations and
  Algebras''. The instrcutors weer Proefssor V.S. Sunder, Dr. K.N.
  Raghavan, and Dr. Amritanshu Prasad.

\item {\bf Microsoft Research Summer School on Algorithms, Complexity
    and Cryptography} from May 22nd to June 10th, 2006, at the Indian
  Institute of Science. The co-ordinators were Ramaratnam Venkatesan
  (Microsoft Research) and Professor Pandu Rangan (IIT Chennai). The
  webpage is:

  \url{http://math.iisc.ernet.in/~imi/sacc}

  The list of selected candidates is availabe at:

  \url{http://math.iisc.ernet.in/~imi/downloads/weblist.pdf}

\item {\bf Visiting Students Research Programme} at the 
  {\bf Tata Institute of Fundamental Research}, 
  from June 15th to July 14th, 2006. Professor
  Dipendra Prasad was the co-ordinator and he was also my guide. I
  studied the paper ``Lie Group Representations of Polynomial Rings''
  by Bertram Kostant.

  The list of selected students is available at:

  \url{http://www.math.tifr.res.in/~vsrp/selected.html}

\item I am among three students from CMI selected for the {\bf ENS-CMI
    Exchange Programme} to be held from May 2, 2007 to June 29, 2007.

\end{enumerate}

\section{Other important achievements/activities/awards}

\subsection{Olympiads}

My scores for the two Olympiads can be viewed at:


\begin{enumerate}

\item I represented India at the {\bf International Mathematical
  Olympiad} in 2003 held in Tokyo, Japan. I scored 23 points out of 42,
  the highest in the Indian team, and secured a silver medal.

\item I represented India at the {\bf International Mathematical
    Olympiad} in 2004 held in Athens, Greece. I scored 30 points out
  of 42, the highest in the Indian team, and secured a silver medal.

  My scores in both Olympiads can be checked by searching for ``Vipul Naik''
  on the IMP Compendium search page:

  \url{http://www.imo.org.yu/index.php?options=gl|imotres&p=39v_31}

\item I qualified the {\bf Zonal Informatics Olympiad} 2004(where approximately
  5000 people participate and 200 are selected).

\item I qualified the {\bf National Standard Examination in Physics}
  2004, coming in the top 1\% out of approximately 31,000
  participants. Hence, I appeared for the Indian National Physics
  Olympiad, which is a qualifying round for selection to the Indian
  team for the International Physics Olympiad.
\end{enumerate}

\subsection{Scholarships}

I have won the following scholarships:

\begin{itemize}

\item The {\bf National Talent Search Examination} (NTSE) scholarship.
  I won this scholarship based on an entrance-test-cum-interview selection.

\item The {\bf Kishore Vaigyanik Protsahan Yojana} (KVPY) scholarship.
  This scholarship was instituted by the Department of Science and
  Technology to promote excellence in pure science. The scholarship
  covered all my under-graduate study expenses.
\end{itemize}

\subsection{Trivia: School-level competitions}

\begin{itemize}

\item I secured an All India Rank 10 in the Screening Test and 
  an All India Rank 158 in the Mains of the Joint Entrance
  Examination (JEE) for the Indian Institutes of Technology (IITs).

\item I secured first rank in the National Science Olympiad in Standard
  8.

\end{itemize}


\end{document}
