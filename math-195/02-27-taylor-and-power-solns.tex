\documentclass[10pt]{amsart}

%Packages in use
\usepackage{fullpage, hyperref, vipul, enumerate}

%Title details
\title{Take-home class quiz solutions: due Wednesday February 27: Taylor series and power series}
\author{Math 195, Section 59 (Vipul Naik)}
%List of new commands

\begin{document}
\maketitle

\section{Performance review}

$24$ people took this $17$-question quiz. The score distribution was
as follows:

\begin{itemize}
\item Score of $3$: $1$ person
\item Score of $5$: $1$ person
\item Score of $6$: $2$ people
\item Score of $8$: $1$ person
\item Score of $13$: $1$ person
\item Score of $15$: $1$ person
\item Score of $16$: $5$ people
\item Score of $17$: $12$ people
\end{itemize}

\begin{enumerate}
\item Option (C): $18$ people
\item Option (C): $21$ people
\item Option (C): $20$ people
\item Option (D): $22$ people
\item Option (E): $22$ people
\item Option (E): $18$ people
\item Option (C): $17$ people
\item Option (B): $18$ people
\item Option (D): $20$ people
\item Option (E): $19$ people
\item Option (C): $20$ people
\item Option (A): $22$ people
\item Option (D): $19$ people
\item Option (D): $22$ people
\item Option (A): $21$ people
\item Option (D): $19$ people
\item Option (D): $22$ people
\end{enumerate}

\section{Solutions}
For these questions, we denote by $C^\infty(\R)$ the space of
functions from $\R$ to $\R$ that are {\em infinitely} differentiable
{\em everywhere} in $\R$.

We denote by $C^k(\R)$ the space of functions from $\R$ to $\R$ that
are at least $k$ times continuously differentiable on all of
$\R$. Note that for $k \ge l$, $C^k(\R)$ is a subspace of
$C^l(\R)$. Further, $C^\infty(\R)$ is the intersection of $C^k(\R)$
for all $k$.

We say that a function $f$ is analytic about $c$ if the Taylor series
of $f$ about $c$ converges to $f$ on some open interval about $c$. We
say that $f$ is {\em globally analytic} if the Taylor series of $f$
about $0$ converges to $f$ everywhere on $\R$.

It turns out that if a function is globally analytic, then it is
analytic not only about $0$ but about any other point. In particular,
globally analytic functions are in $C^\infty(\R)$.

\begin{enumerate}

\item Recall that if $f$ is a function defined and continuous around
  $c$ with the property that $f(c) = 0$, the order of the zero of $f$
  at $c$ is defined as the least upper bound of the set of real $\beta$ for
  which $\lim_{x \to c} |f(x)|/|x - c|^\beta = 0$. If $f$ is in
  $C^{\infty}(\R)$, what can we conclude about the orders of zeros of
  $f$? {\em Two years ago: $11/26$ correct}

  \begin{enumerate}[(A)]
  \item The order of any zero of $f$ must be between $0$ and $1$.
  \item The order of any zero of $f$ must be between $1$ and $2$.
  \item The order of any zero of $f$, if finite, must be a positive
  integer.
  \item The order of any zero of $f$ must be exactly $1$.
  \item The order of any zero of $f$ must be $\infty$.
  \end{enumerate}

  {\em Answer}: Option (C)

  {\em Explanation}: We consider two cases. First, that for every
  positive integer $k$, we have $f^{(k)}(c) = 0$. In that case, we can
  verify using the LH rule that $f(x)/(x - c)^k \to 0$ for every
  positive integer $k$, and hence, there is no finite least upper
  bound and hence no finite order.

  Next, suppose there is a smallest $k$ such that $f^{(k)}(c) \ne
  0$. Suppose $f^{(k)}(c) = \lambda$. This $k$ must be greater than
  $0$, because we are given that $f^{(0)}(c) = f(c) = 0$. We can show
  by a $k$-fold application of the LH rule that $\lim_{x \to c}
  f(x)/(x - c)^k = \lambda/k!$ which is a finite nonzero number. By
  suitable chaining, we can therefore show that $\lim_{x \to c}
  |f(x)|/|x - c|^\beta = 0$ for all $\beta \in (0,k)$. Thus, the order
  of the zero at $f$ is precisely $k$, whuch is a positive integer.

  {\em Performance review}: $18$ out of $24$ people got this. $3$
  chose (E), $2$ chose (A), $1$ chose (D).

  {\em Historical note (Math 153)}: $32$ out of $40$ got this. $7$ chose (E),
  $1$ chose (B).

  {\em Historical note (last year)}: $3$ out of $11$ people got this. $3$ chose
  (B), $4$ chose (E), and $1$ left the question blank.

  {\em Historical note (two years ago)}: $11$ out of $26$ people
  got this correct. $7$ chose (A), $3$ each chose (B) and (E), $1$
  chose (D), and $1$ left the question blank.

\item For the function $f(x) := x^2 + x^{4/3} + x + 1$ defined on
  $\R$, what can we say about the Taylor polynomials about $0$? {\em
  Two years ago: $8/26$ correct}

  \begin{enumerate}[(A)]
  \item No Taylor polynomial is defined for $f$.
  \item $P_0(f)(x) = 1$, $P_n(f)$ is not defined for $n > 0$.
  \item $P_0(f)(x) = 1$, $P_1(f)(x) = 1 + x$, $P_n(f)$ is not defined
    for $n > 1$.
  \item $P_0(f)(x) = 1$, $P_1(f)(x) = 1 + x$, $P_2(f) = f$, and
    $P_n(f)$ is not defined for $n > 2$.
  \item $P_0(f)(x) = 1$, $P_1(f)(x) = 1 + x$, $P_2(f) = f$, and
    $P_n(f) = f$ for all $n > 2$.
  \end{enumerate}

  {\em Answer}: Option (C)

  {\em Explanation}: The fraction power $x^{4/3}$ can be
  differentiated once but not twice about $0$. The rest of the
  expression for $f$ is polynomial. Thus, $f$ is once differentiable
  but not twice differentiable at $0$. Hence, we cannot define
  $P_2(f)$. $P_0(f)$ is just $f(0)$, which is $1$, and $P_1(f)(x) =
  f(0) + f'(0)x = 1 + x$. Alternatively, $P_1(f)(x)$ is simply the
  truncation of $f$ to the parts of degree at most $1$.

  {\em Performance review}: $21$ out of $24$ people got this. $2$
  chose (E), $1$ chose (D).

  {\em Historical note (Math 153)}: $34$ out of $40$ got this. $3$ chose (D),
  $2$ chose (E), and $1$ chose (B).

  {\em Historical note (last year)}: $2$ out of $11$ got this correct. $7$
  chose (E), $1$ each chose (B) and (D).

  {\em Historical note (two years ago)}: $8$ out of $26$ people got this
  correct. $5$ each chose (A) and (B), $4$ each chose (D) and (E).

\item Consider the function $F(x,p) = \sum_{n=1}^\infty x^n/n^p$. For
  fixed $p$, this is a power series in $x$. What can we say about the
  interval of convergence of this power series about $x = 0$, in terms
  of $p$ for $p \in (0,\infty)$? {\em Two years ago: $4/26$ correct}
  \begin{enumerate}[(A)]
  \item The interval of convergence is $(-1,1)$ for $0 < p \le 1$ and
    $[-1,1]$ for $p > 1$.
  \item The interval of convergence is $(-1,1)$ for $0 < p < 1$ and
    $[-1,1]$ for $p \ge 1$.
  \item The interval of convergence is $[-1,1)$ for $0 < p \le 1$ and
    $[-1,1]$ for $p > 1$.
  \item The interval of convergence is $(-1,1]$ for $0 < p < 1$ and
    $[-1,1]$ for $p \ge 1$.
  \item The interval of convergence is $(-1,1)$ for $0 < p \le 1$ and
    $[-1,1)$ for $p > 1$.
  \end{enumerate}

  {\em Answer}: Option (C)

  {\em Explanation}: The radius of convergence is $1$ for obvious
  reasons. Convergence at the boundary point $-1$ follows from the
  alternating series theorem. At the boundary point $1$, we get a
  $p$-series, which converges if and only if $p > 1$.

  {\em Performance review}: $20$ out of $24$ got this. $3$ chose (A),
  $1$ chose (B).

  {\em Historical note (Math 153)}: $28$ out of $40$ got this. $5$ each chose
  (A) and (D), $1$ chose (B), and $1$ left the question blank.

  {\em Historical note (last year)}: $5$ out of $11$ got this correct. $3$
  chose (A), $2$ chose (D), $1$ chose (B).

  {\em Historical note (two years ago)}: $4$ out of $26$ people got this
  correct. $10$ chose (B), $7$ chose (A), $3$ chose (E), $2$ chose
  (D).

\item Which of the following functions of $x$ has a power series
  $\sum_{k=0}^\infty x^{4k}/(4k)!$? {\em Two years ago: $9/26$ correct}
  \begin{enumerate}[(A)]
  \item $(\sin x + \sinh x)/2$
  \item $(\sin x - \sinh x)/2$
  \item $(\sinh x - \sin x)/2$
  \item $(\cosh x + \cos x)/2$
  \item $(\cosh x - \cos x)/2$
  \end{enumerate}

  {\em Answer}: Option (D)

  {\em Explanation}: Take the Taylor series and add. Also, use that
  both $\cosh$ and $\cos$ are globally analytic.

  {\em Performance review}: $22$ out of $24$ got this. $1$ each chose
  (B) and (C).

  {\em Historical note (Math 153)}: $35$ out of $40$ got this. $3$ chose (A),
  $1$ each chose (B) and (C).

  {\em Historical note (last year)}: $5$ out of $11$ got this correct. $3$
  chose (C), $2$ chose (A), $1$ chose (B).

  {\em Historical note (two years ago)}: $9$ out of $26$ people got this
  correct. $5$ chose (B), $4$ each chose (A), (C), and (E).

\item What is the sum $\sum_{k=0}^\infty (-1)^kx^{2k}/k!$? Note that
  the denominator is $k!$ and {\em not} $(2k)!$. {\em Two years ago:
  $12/26$ correct}

  \begin{enumerate}[(A)]
  \item $\cos x$
  \item $\sin x$
  \item $\cos(x^2)$
  \item $\cosh(x^2)$
  \item $\exp(-x^2)$
  \end{enumerate}

  {\em Answer}: Option (E)

  {\em Explanation}: Put $u = -x^2$, and we get $\sum_{k=0}^\infty
  u^k/k!$.

  {\em Performance review}: $22$ out of $24$ got this. $1$ each chose
  (A) and (B).

  {\em Historical note (Math 153)}: $35$ out of $40$ got this. $5$ chose (C).

  {\em Historical note (last year)}: $1$ out of $11$ got this correct. $7$
  chose (C), $2$ chose (D), $1$ chose (B).

  {\em Historical note (two years ago)}: $12$ out of $26$ people got this
  correct. $7$ chose (C), $3$ each chose (A) and (D), $1$ chose (B).

\item Define an operator $R$ from the set of power series about $0$
  to the set $[0,\infty]$ (nonnegative real numbers along with
  $+\infty$) that sends a power series $a = \sum a_kx^k$ to the radius
  of convergence of the power series about $0$. For two power series
  $a$ and $b$, $a + b$ is the sum of the power series. What can we say
  about $R(a + b)$ given $R(a)$ and $R(b)$?

  \begin{enumerate}[(A)]
  \item $R(a + b) = \max \{ R(a), R(b) \}$ in all cases.
  \item $R(a + b) = \min \{ R(a), R(b) \}$ in all cases.
  \item $R(a + b) = \max \{ R(a), R(b) \}$ if $R(a) \ne R(b)$. If
    $R(a) = R(b)$, then $R(a + b)$ could be any number greater than or
    equal to $\max \{ R(a), R(b) \}$.
  \item $R(a + b) = \max \{ R(a), R(b) \}$ if $R(a) \ne R(b)$. If
    $R(a) = R(b)$, then $R(a + b)$ could be any number less than or
    equal to $\max \{ R(a), R(b) \}$.
  \item $R(a + b) = \min \{ R(a), R(b) \}$ if $R(a) \ne R(b)$. If
    $R(a) = R(b)$, then $R(a + b)$ could be any number greater than or
    equal to $\min \{ R(a), R(b) \}$.
  \end{enumerate}

  {\em Answer}: Option (E)

  {\em Explanation}: $R(a + b) \ge \min \{ R(a), R(b) \}$ because if
  both $a$ and $b$ converge, so does $a + b$. Let $c = a + b$. We also then get
  that $R(a) \ge \min \{ R(b), R(c) \}$ and $R(b) \ge \min \{ R(a),
  R(c) \}$ because $a = c - b$ and $b = c - a$.

  Juggling these possibilities, we find that of the three numbers
  $R(a)$, $R(b)$, and $R(a + b)$, the smaller two of the three numbers
  must be equal. This forces option (E).

  {\em This is a type of hyperbolic geometry -- all ``triangles'' must
    be isosceles.}

  {\em Performance review}: $18$ out of $24$ got this. $3$ chose (C),
  $2$ chose (A), $1$ chose (B).

  {\em Historical note (Math 153)}: $29$ out of $40$ got this. $5$ chose (D),
  $3$ each chose (B) and (C).

  {\em Historical note (last year)}: $3$ out of $11$ got this correct. $4$
  chose (B) $3$ chose (D), $1$ chose (C).

  {\em Historical note (two years ago)}: $3$ out of $26$ people got this
  correct. $11$ chose (C), $7$ chose (D), $3$ chose (A), $1$ chose (B).

\item Which of the following is/are true? {\em Two years ago: $5/26$ correct}

  \begin{enumerate}[(A)]
  \item If we start with any function in $C^\infty(\R)$ and take the
    Taylor series about $0$, the Taylor series converges everywhere on
    $\R$.
  \item If we start with any function in $C^\infty(\R)$ and take the
    Taylor series about $0$, the Taylor series converges to the
    original function on its interval of convergence (which may not be
    all of $\R$).
  \item If we start with a power series about $0$ that converges
    everywhere in $\R$, then the function it converges to is in
    $C^\infty(\R)$ and its Taylor series about $0$ equals the original
    power series.
  \item All of the above.
  \item None of the above.
  \end{enumerate}

  {\em Answer}: Option (C)

  {\em Explanation}: See the lecture notes. A counterexample to (A) is
  $\arctan$, and a counterexample to (B) is $e^{-1/x^2}$.

  {\em Performance review}: $17$ out of $24$ got this. $4$ chose (D),
  $2$ chose (B), $1$ chose (A).

  {\em Historical note (Math 153)}: $29$ out of $40$ got this. $6$ chose (D),
  $4$ chose (B), $1$ chose (A).

  {\em Historical note (last year)}: $9$ out of $11$ got this
  correct. $1$ chose (B) and $1$ chose (E).

  {\em Historical note (two years ago)}: $5$ out of $26$ people got this
  correct. $10$ chose (B), $5$ each chose (D) and (E), and $1$ chose
  (A).

\item Consider the function $f(x) := \sum_{k=0}^\infty
  x^k/2^{k^2}$. The power series converges everywhere, so $f$ is a
  globally analytic function. What is the best description of the
  manner in which $f$ grows as $x \to \infty$? {\em Two years ago: $12/26$
  correct}

  \begin{enumerate}[(A)]
  \item $f$ grows polynomially in $x$.
  \item $f$ grows faster than any polynomial function but slower than
    any exponential function of $x$ (i.e., any function of the form $x
    \mapsto e^{mx}, m >0$).
  \item $f$ grows like an exponential function of $x$, i.e., it can be
    sandwiched between two exponentially growing functions of $x$.
  \item $f$ grows faster than any exponential function but slower than
    any doubly exponential function of $x$. Here, doubly exponential
    means something of the form $e^{ae^{bx}}$ where $a$ and $b$ are
    both positive.
  \item $f$ grows like a doubly exponential function of $x$. Here,
    doubly exponential means something of the form $e^{ae^{bx}}$ where
    $a$ and $b$ are both positive.
  \end{enumerate}

  {\em Answer}: Option (B)

  {\em Explanation}: Note that {\em any} power series with infinitely
  many positive coefficients (and no negative coefficients) must grow
  faster than a polynomial, which, after all, has finite degree. Note
  that this logic does not work for power series that have a mix of
  positive and negative coefficients, such as the power series for the
  $\sin$ and $\cos$ functions.

  The rough reason that growth is strictly slower than an exponential
  function is that the denominators are growing much faster than
  $k!$. Recall that if the denominators are $k!$, we get precisely the
  exponential function. This can be made more precise.

  {\em Performance review}: $18$ out of $24$ got this. $4$ chose (C),
  $1$ each chose (A) and (E).

  {\em Historical note (Math 153)}: $30$ out of $40$ got this. $5$ chose (A),
  $3$ chose (C), $1$ each chose (D) and (E).

  {\em Historical note (last year)}: $5$ out of $11$ got this correct. $4$
  chose (D), $1$ each chose (A) and (C).

  {\em Historical note (two years ago)}: $12$ out of $26$ people got this
  correct. $6$ chose (D), $4$ chose (C), $2$ each chose (A) and (E).

\item Consider the function $f(x) := \sum_{k=0}^\infty
  x^k/(k!)^2$. The power series converges everywhere, so the function
  is globally analytic. What pair of functions bounds $f$ from above
  and below for $x > 0$? {\em Two years ago: $12/26$ correct}

  \begin{enumerate}[(A)]
  \item $\exp(x)$ from below and $\cosh(2x)$ from above.
  \item $\exp(x)$ from below and $\cosh(x^2)$ from above.
  \item $\exp(x/2)$ from below and $\exp(x)$ from above.
  \item $\cosh(\sqrt{x})$ from below and $\exp(x)$ from above.
  \item $\cosh(2x)$ from below and $\cosh(x^2)$ from above.
  \end{enumerate}


  {\em Answer}: Option (D)

  {\em Explanation}: We use the fact that:

  $$k! \le (k!)^2 \le (2k)!$$

  for all $k \ge 0$, with both inequalities strict if $k \ge 2$.

  We thus get:

  $$\frac{x^k}{k!} \ge \frac{x^k}{(k!)^2} \ge \frac{x^k}{(2k)!}$$

  for $x > 0$, with both inequalities strict if $k \ge 2$. Summing up, we get:
  
  $$\sum_{k=0}^\infty \frac{x^k}{k!} > \sum_{k=0}^\infty \frac{x^k}{(k!)^2} > \sum_{k=0}^\infty \frac{x^k}{(2k)!}$$

  The left most expression is $e^x$. For the right most expression,
  put $u = \sqrt{x}$, and we get $\cosh u$, so $\cosh \sqrt{x}$. Thus
  option (D) is the right choice.

  {\em Performance review}: $20$ out of $24$ got this. $2$ chose (C),
  $1$ each chose (A) and (B).

  {\em Historical note (Math 153)}: $28$ out of $40$ got this. $7$ chose (C),
  $4$ chose (B), $1$ chose (A).

  {\em Historical note (last year)}: $1$ out of $11$ got this correct. $6$
  chose (C) and $4$ chose (B).

  {\em Historical note (two years ago)}: $12$ out of $26$ people got this
  correct. $5$ each chose (A) and (C), $3$ chose (B), and $1$ chose
  (E).

\item Consider the function $f(x) := \max \{ 0, x\}$. What can we say
  about the Taylor series of $f$ centered at various points?

  \begin{enumerate}[(A)]
  \item The Taylor series of $f$ centered at any point is the zero
    series.
  \item The Taylor series of $f$ centered at any point simplifies to
  $x$.
  \item The Taylor series of $f$ centered at any point other than zero
    converges to $f$ globally. However, the Taylor series centered at
    $0$ is not defined.
  \item The Taylor series of $f$ centered at any point is either the
    zero series or simplifies to $x$.
  \item The Taylor series of $f$ centered at any point other than the
    point $0$ is either the zero series or simplifies to $x$. However,
    the Taylor series is not defined at $0$.
  \end{enumerate}

  {\em Answer}: Option (E)

  {\em Explanation}: A piecewise description of $f$ is:

  $$f(x) := \left \lbrace \begin{array}{rr} x, & x > 0 \\ 0, & x \le 0 \\\end{array}\right.$$

  The Taylor series is not defined at $0$. The
  reason is that the function is not differentiable at $0$, because
  the left hand derivative $f'_-(0)$ is $0$ and the right hand
  derivative $f'_+(0)$ is $1$.

  At any other point, the Taylor series corresponds globally to the
  piece function for that point. So, the Taylor series at any positive
  real number is just the $x$ function and the Taylor series at any
  negative real number is just the $0$ function. This ties in with the
  idea that the Taylor series is completely determined by the local
  behavior of the function and cannot see changes in the function
  definition far from the point.

  {\em Performance review}: $19$ out of $24$ got this. $2$ each chose
  (B) and (C), $1$ chose (A).

  {\em Historical note (Math 153)}: $26$ out of $40$ got this. $6$ chose (C),
  $4$ chose (D), $3$ chose (A), $1$ chose (B).

\item Which of the following functions is in $C^\infty(\R)$ but is
  {\em not} analytic about $0$? {\em Two years ago: $3/26$ correct}

  \begin{enumerate}[(A)]
  \item $f_1(x) := \left \lbrace \begin{array}{rl} (\sin x)/x, & x \ne 0\\ 1, & x = 0 \\\end{array} \right.$
  \item $f_2(x) := \left \lbrace \begin{array}{rl} e^{-1/x}, & x \ne 0\\ 0, & x = 0 \\\end{array} \right.$
  \item $f_3(x) := \left \lbrace \begin{array}{rl} e^{-1/x^2}, & x \ne 0\\ 0, & x = 0 \\\end{array} \right.$
  \item $f_4(x) := \left \lbrace \begin{array}{rl} \sin(1/x), & x \ne 0 \\ 0, & x = 0 \\\end{array} \right.$
  \item All of the above.
  \end{enumerate}

  {\em Answer}: Option (C)

  {\em Explanation}: This answer is explained more in the lecture
  notes.

  {\em Why the other options are wrong}:

  Option (A): This is in fact globlally analytic, and is given by the
  power series $1 - x^2/3! + x^4/5! - \dots$

  Option (B): This is not continuous at $0$. The left hand limit at
  $0$ is $+\infty$.

  Option (D): This is not continuous at $0$. The limit at $0$ is not
  defined.

  {\em Performance review}: $20$ out of $24$ got this. $2$ each chose
  (A) and (B).

  {\em Historical note (Math 153)}: $30$ out of $42$ got this. $5$ chose (D),
  $4$ chose (E), $2$ chose (B), $1$ chose (A).

  {\em Historical note (last year)}: $1$ out of $11$ got this correct. $5$
  chose (D), $2$ chose (B), $1$ each chose (A) and (E).

  {\em Historical note (two years ago)}: $3$ out of $26$ people got this
  correct. $11$ chose (D), $6$ each chose (A) and (E).

\item Which of the following functions is in $C^\infty(\R)$ and is
  analytic about $0$ but is not globally analytic? {\em Two years ago:
  $7/26$ correct}
  \begin{enumerate}[(A)]
  \item $x \mapsto \ln(1 + x^2)$
  \item $x \mapsto \ln(1 + x)$
  \item $x \mapsto \ln(1 - x)$
  \item $x \mapsto \exp(1 + x)$
  \item $x \mapsto \exp(1 - x)$
  \end{enumerate}

  {\em Answer}: Option (A)

  {\em Explanation}: The function is in $C^\infty(\R)$ because it can
  be differentiated infinitely often: the first derivative is $2x/(1 +
  x^2)$, and each subsequent derivative is a rational function whose
  denominator is a power of $1 + x^2$. Since $1 + x^2$ does not vanish
  anywhere on $\R$, each derivative is defined and continuous on all
  of $\R$.

  The radius of convergence of the power series is $1$, basically
  because it is a power series where the coefficients are rational
  functions, and any such power series has radius of convergence $1$
  by the root test or ratio test. Thus, the function is not globally
  analytic.

  {\em Why the other options are wrong}:

  Option (B) is not in $C^\infty(\R)$ because the function is not
  defined for $x \le -1$.

  Option (C) is not in $C^\infty(\R)$ because the function is not
  defined for $x \ge 1$.

  Options (D) and (E) are globally analytic because $\exp$ is globally
  analytic.

  {\em Performance review}: $22$ out of $24$ got this. $1$ each chose
  (C) and (E).

  {\em Historical note (Math 153)}: $28$ out of $42$ got this. $8$ chose (B),
  $6$ chose (C).

  {\em Historical note (last year)}: Nobody got this correct! $8$ chose (B),
  $3$ chose (C).

  {\em Historical note (two years ago)}: $7$ out of $26$ people got this
  correct. $6$ chose (C), $5$ chose (B), $4$ chose (E), $3$ chose (D),
  and $1$ left the question blank.

\item Suppose $f$ and $g$ are globally analytic functions and $g$ is
  nowhere zero. Which of the following is {\em not necessarily}
  globally analytic?

  \begin{enumerate}[(A)]
  \item $f + g$, i.e., the function $x \mapsto f(x) + g(x)$
  \item $f - g$, i.e., the function $x \mapsto f(x) - g(x)$
  \item $fg$, i.e., the function $x \mapsto f(x)g(x)$
  \item $f/g$, i.e., the function $x \mapsto f(x)/g(x)$
  \item $f \circ g$, i.e., the function $x \mapsto f(g(x))$
  \end{enumerate}

  {\em Answer}: Option (D)

  {\em Explanation}: See the example for the next question.

  {\em Performance review}: $19$ out of $24$ got this. $4$ chose (E),
  $1$ chose (C).

  {\em Historical note (Math 153)}: $37$ out of $42$ got this. $4$ chose (E),
  $1$ chose (A).

\item Which of the following is an example of a globally analytic
  function whose reciprocal is in $C^\infty(\R)$ but is not globally
  analytic? {\em Two years ago: $10/26$ correct}

  \begin{enumerate}[(A)]
  \item $x$
  \item $x^2$
  \item $x + 1$
  \item $x^2 + 1$
  \item $e^x$
  \end{enumerate}

  {\em Answer}: Option (D)

  {\em Explanation}: The reciprocal $1/(x^2 + 1)$ is a rational
  function all of whose derivatives are rational functions with
  denominator a power of $x^2 + 1$, hence defined and continuous
  derivatives. Hence it is is $C^\infty(\R)$. Further, the power
  series expansion for it is like a geometric series, which has radius
  of convergence $1$, hence it is not globally analytic.

  {\em Why the other options are wrong}:

  Options (A) and (B): The reciprocals are not in $C^\infty(\R)$
  because the functions $1/x$ and $1/x^2$ are not defined or
  continuous at $0$.

  Option (C): The reciprocal is not in $C^\infty(\R)$ because the
  function $1/(x + 1)$ is not defined or continuous at $-1$.

  Option (E): The reciprocal, which is $\exp(-x)$, is globally
  analytic.

  {\em Performance review}: $22$ out of $24$ got this. $2$ chose (E).

  {\em Historical note (Math 153)}: $28$ out of $42$ got this. $8$ chose (C),
  $4$ chose (E), $2$ chose (A).

  {\em Historical note (last year)}: $9$ out of $11$ got this correct. $1$ each
  chose (A) and (E).

  {\em Historical note (two years ago)}: $10$ out of $26$ people got this
  correct. $5$ chose (C), $4$ each chose (A) and (E), $2$ chose (B),
  $1$ left the question blank.

\item Consider the rational function $1/\prod_{i=1}^n (x - \alpha_i)$,
  where the $\alpha_i$ are all distinct real numbers. This rational
  function is analytic about any point other than the $\alpha_i$s, and
  in particular its Taylor series converges to it on the interval of
  convergence. What is the radius of convergence for the Taylor series
  of the rational function about a point $c$ not equal to any of the
  $\alpha_i$s? {\em Two years ago: $10/26$ correct}

  \begin{enumerate}[(A)]
  \item It is the minimum of the distances from $c$ to the $\alpha_i$s.
  \item It is the second smallest of the distances from $c$ to the
    $\alpha_i$s.
  \item It is the arithmetic mean of the distances from $c$ to the $\alpha_i$s.
  \item It is the second largest of the distances from $c$ to the
    $\alpha_i$s.
  \item It is the maximum of the distances from $c$ to the $\alpha_i$s.
  \end{enumerate}

  {\em Answer}: Option (A)

  {\em Explanation}: Since the Taylor series converges to the function
  on its interval of convergence, the interval of convergence must be
  contained in the domain of definition. In particular, it must
  exclude all the $\alpha_i$s. Hence, the radius of convergence cannot
  be more than the minimum of the distances from $c$ to the
  $\alpha_i$s.

  That it is exactly equal to the minimum can be shown by using the
  fact that we get a product of geometric series.

  {\em Performance review}: $21$ out of $24$ got this. $2$ chose (E),
  $1$ chose (C).

  {\em Historical note (Math 153)}: $34$ out of $42$ got this. $6$ chose (C),
  $2$ chose (D).

  {\em Historical note (last year)}: $3$ out of $11$ got this correct. $4$
  chose (E), $2$ chose (C), $1$ chose (B).

  {\em Historical note (two years ago)}: $10$ out of $26$ people got this
  correct. $6$ people chose (C), $5$ chose (E), $2$ each chose (B) and
  (D), $1$ left the question blank.

\item What is the interval of convergence of the Taylor series for
  $\arctan$ about $0$? {\em Two years ago: $11/26$ correct}
  
  \begin{enumerate}[(A)]
  \item $(-1,1)$
  \item $[-1,1)$
  \item $(-1,1]$
  \item $[-1,1]$
  \item All of $\R$
  \end{enumerate}

  {\em Answer}: Option (D)

  {\em Explanation}: For the boundary points, we use the alternating
  series theorem. See a more detailed discussion in the lecture notes.

  {\em Performance review}: $19$ out of $24$ got this. $3$ chose (A),
  $2$ chose (E).

  {\em Historical note (Math 153)}: $40$ out of $42$ got this. $1$ each chose
  (A) and (E).

  {\em Historical note (last year)}: $10$ out of $11$ got this
  correct. $1$ chose (A).

  {\em Historical note (two years ago)}: $11$ out of $26$ people got this
  correct. $8$ chose (A), $4$ chose (C), $2$ chose (B), $1$ chose (E).

\item What is the radius of convergence of the power series
  $\sum_{k=0}^\infty 2^{\sqrt{k}} x^k$? Please keep in mind the square
  root in the exponent.

  \begin{enumerate}[(A)]
  \item $0$
  \item $1/2$
  \item $1/\sqrt{2}$
  \item $1$
  \item infinite
  \end{enumerate}

  {\em Answer}: Option (D)

  {\em Explanation}: The coefficients have subexponential growth, so
  the radius of convergence is $1$.

  {\em Performance review}: $22$ out of $24$ got this. $1$ each chose
  (A) and (C).

  {\em Historical note (Math 153)}: $39$ out of $42$ got this. $3$
  chose (B).

\end{enumerate}

\end{document}
