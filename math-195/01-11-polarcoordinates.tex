\documentclass[10pt]{amsart}

%Packages in use
\usepackage{fullpage, hyperref, vipul, enumerate}

%Title details
\title{Class quiz: Friday January 11: Polar coordinates}
\author{Math 195, Section 59 (Vipul Naik)}
%List of new commands

\begin{document}
\maketitle

Your name (print clearly in capital letters): $\underline{\qquad\qquad\qquad\qquad\qquad\qquad\qquad\qquad\qquad\qquad}$

\vspace{0.1in}

{\bf YOU ARE ALLOWED TO DISCUSS ONLY QUESTIONS THAT BEGIN WITH A (*)
  OR (**). PLEASE ATTEMPT ALL OTHER QUESTIONS BY YOURSELF. EVEN FOR THE QUESTIONS YOU DISCUSS, PLEASE FINALLY ENTER ONLY THE ANSWER OPTION YOU ARE PERSONALLY MOST CONVINCED ABOUT -- DON'T ENGAGE IN GROUPTHINK.}

\begin{enumerate}

\item (*) Consider a straight line that does not pass through the pole in
  a polar coordinate system. The equation of such a line in the polar
  coordinate system can be expressed as $r = F(\theta)$. What kind of
  function is $F$? {\em Last time: $8/21$ correct}

  \begin{enumerate}[(A)]

  \item $F(\theta)$ is a linear combination of $\sin \theta$ and $\cos \theta$
  \item $F(\theta)$ is the reciprocal of a linear combination of $\sin
    \theta$ and $\cos \theta$.
  \item $F(\theta)$ is a linear combination of $\tan \theta$ and $\cot
    \theta$.
  \item $F(\theta)$ is the reciprocal of a linear combination of $\tan
    \theta$ and $\cot \theta$.
  \item $F(\theta)$ is a linear combination of $\sec \theta$ and $\csc
    \theta$.
  \end{enumerate}

  \vspace{0.1in}
  Your answer: $\underline{\qquad\qquad\qquad\qquad\qquad\qquad\qquad}$
  \vspace{0.6in}

\item Consider the curve $r = \sin^2\theta$. Which of the following
  symmetries does the curve enjoy? Please see options (D) and (E)
  before answering. {\em Last time: $10/21$ correct}

  \begin{enumerate}[(A)]
  \item Mirror symmetry about the polar axis
  \item Mirror symmetry about an axis perpendicular to the polar axis
    (what would be the $y$-axis if the polar axis is the $x$-axis)
  \item Half turn symmetry about the pole
  \item All of the above
  \item None of the above
  \end{enumerate}

  \vspace{0.1in}
  Your answer: $\underline{\qquad\qquad\qquad\qquad\qquad\qquad\qquad}$
  \vspace{0.6in}

\item Which of the following is the correct expression for the length
  of the part of the curve $r = F(\theta)$ from $\theta = \alpha$ to
  $\theta = \beta$, with $\alpha < \beta$? {\em Last time: $14/21$
    correct}

  \begin{enumerate}[(A)]
  \item $\int_\alpha^\beta \sqrt{(F(\theta))^2 + (F'(\theta))^2} \, d\theta$
  \item $\int_\alpha^\beta |F(\theta) + F'(\theta)| \, d\theta$
  \item $\int_\alpha^\beta |F(\theta) - F'(\theta)| \, d\theta$
  \item $\int_\alpha^\beta \sqrt{(F(\theta))^2 + (F'(\theta))^2 + 4F(\theta)F'(\theta)} \, d\theta$
  \item $\int_\alpha^\beta \sqrt{(F(\theta))^2 + (F'(\theta))^2 - 4F(\theta)F'(\theta)} \, d\theta$
  \end{enumerate}

  \vspace{0.1in}
  Your answer: $\underline{\qquad\qquad\qquad\qquad\qquad\qquad\qquad}$
  \vspace{0.6in}

\end{enumerate}
\end{document}