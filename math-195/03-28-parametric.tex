\documentclass[10pt]{amsart}

%Packages in use
\usepackage{fullpage, hyperref, vipul, enumerate}

%Title details
\title{Class quiz: March 28: Parametric stuff}
\author{Math 195, Section 59 (Vipul Naik)}
%List of new commands

\begin{document}
\maketitle

Your name (print clearly in capital letters): $\underline{\qquad\qquad\qquad\qquad\qquad\qquad\qquad\qquad\qquad\qquad}$

\begin{enumerate}
\item Consider the curve given by the parametric description $x = \cos
  t$, $y = \sin t$, where $t$ varies over the interval $[a,b]$ with $a <
  b$. What is a necessary and sufficient condition on $a$ and $b$ for
  this curve to be the circle $x^2 + y^2 = 1$?

  \begin{enumerate}[(A)]
  \item $b - a =\pi$
  \item $b - a > \pi$
  \item $b - a = 2\pi$
  \item $b - a > 2\pi$
  \item $b - a \ge 2\pi$
  \end{enumerate}

  \vspace{0.1in}
  Your answer: $\underline{\qquad\qquad\qquad\qquad\qquad\qquad\qquad}$
  \vspace{0.6in}

\item Consider the curve given by the parametric description $x =
  \arctan t$ and $y = \arctan t$ for $t \in \R$. Which of the
  following is the best description of this curve?

  \begin{enumerate}[(A)]
  \item It is the graph of the function $\arctan$
  \item It is the line $y = x$
  \item It is a line segment (without endpoints) that is part of the
    line $y = x$
  \item It is a half-line (with endpoint) that is part of the line $y
    = x$
  \item It is a disjoint union of two half-lines that are both part of
    the line $y = x$
  \end{enumerate}

  \vspace{0.1in}
  Your answer: $\underline{\qquad\qquad\qquad\qquad\qquad\qquad\qquad}$
  \vspace{0.6in}

\item Consider the curve given by the parametric description $x =
  \sin^2t$ and $y = \cos^2t$ for $t \in \R$. Which of the
  following is the best description of this curve?

  \begin{enumerate}[(A)]
  \item It is the arc of the circle $x^2 + y^2 = 1$ comprising the
    first quadrant, i.e., when $x \ge 0$ and $y \ge 0$.
  \item It is the entire circle $x^2 + y^2 = 1$
  \item It is the line segment joining the points $(0,1)$ and $(1,0)$
  \item It is the line $y = 1 - x$
  \item It is a portion of the parabola $y = x^2$
  \end{enumerate}

  \vspace{0.1in}
  Your answer: $\underline{\qquad\qquad\qquad\qquad\qquad\qquad\qquad}$
  \vspace{0.6in}

\item Identify the parametric description which {\em does not}
  correspond to the set of points $(x,y)$ satisfying $x^3 = y^5$.

  \begin{enumerate}[(A)]
  \item $x = t^3$, $y = t^5$, for $t \in \R$
  \item $x = t^5$, $y = t^3$, for $t \in \R$
  \item $x = t$, $y = t^{3/5}$, for $t \in \R$
  \item $x = t^{5/3}$, $y = t$, for $t \in \R$
  \item All of the above parametric descriptions work
  \end{enumerate}

  \vspace{0.1in}
  Your answer: $\underline{\qquad\qquad\qquad\qquad\qquad\qquad\qquad}$
  \vspace{0.6in}

\item Consider the parametric description $x = f(t)$, $y = g(t)$ where
  $t$ varies over all of $\R$. What is the necessary and sufficient
  condition for the curve given by this to be the graph of a function,
  i.e., to satisfy the vertical line test?

  \begin{enumerate}[(A)]
  \item For any $t_1$ and $t_2$ satisfying $f(t_1) = f(t_2)$, we must
    have $g(t_1) = g(t_2)$.
  \item For any $t_1$ and $t_2$ satisfying $g(t_1) = g(t_2)$, we must
    have $f(t_1) = f(t_2)$.
  \item Both $f$ and $g$ are one-to-one functions.
  \item For any $t_1$ and $t_2$, we must have $f(t_1) = f(t_2)$.
  \item For any $t_1$ and $t_2$, we must have $g(t_1) = g(t_2)$.
  \end{enumerate}

  \vspace{0.1in}
  Your answer: $\underline{\qquad\qquad\qquad\qquad\qquad\qquad\qquad}$
  \vspace{0.6in}

\end{enumerate}
\end{document}