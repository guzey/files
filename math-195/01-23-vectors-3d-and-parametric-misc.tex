\documentclass[10pt]{amsart}

%Packages in use
\usepackage{fullpage, hyperref, vipul, enumerate}

%Title details
\title{Take-home class quiz: due Wednesday January 23: Vectors, 3D, and parametric stuff -- miscellanea}
\author{Math 195, Section 59 (Vipul Naik)}
%List of new commands

\begin{document}
\maketitle

Your name (print clearly in capital letters): $\underline{\qquad\qquad\qquad\qquad\qquad\qquad\qquad\qquad\qquad\qquad}$

\vspace{0.1in}

{\bf THIS IS A TAKE-HOME CLASS QUIZ, BUT I WILL GIVE YOU ABOUT 5
  MINUTES TO REVIEW YOUR ANSWERS IN CLASS AND DISCUSS WITH OTHER
  STUDENTS.}

\vspace{0.1in}

{\bf YOU ARE FREE TO DISCUSS {\em ALL} QUESTIONS, BUT PLEASE FINALLY
  ENTER ONLY THE ANSWER OPTION YOU ARE PERSONALLY MOST CONVINCED ABOUT
  -- DON'T ENGAGE IN GROUPTHINK.}

Yes, you are free to discuss {\em all} questions for this quiz.
\begin{enumerate}

\item Suppose we are given three subsets $\Gamma_1$, $\Gamma_2$, and
  $\Gamma_3$ of $\R^3$ where $\Gamma_1$ is the set of solutions to
  $F_1(x,y,z) = 0$, $\Gamma_2$ is the set of solutions to $F_2(x,y,z)
  = 0$, and $\Gamma_3$ is the set of solutions to $F_3(x,y,z) =
  0$. Which of the following equations gives precisely the set of
  points that lie in {\em at least two} of the subsets $\Gamma_1$,
  $\Gamma_2$, and $\Gamma_3$?

  \begin{enumerate}[(A)]
  \item $F_1(x,y,z)F_2(x,y,z)F_3(x,y,z) = 0$
  \item $(F_1(x,y,z))^2 + (F_2(x,y,z))^2 + (F_3(x,y,z))^2 = 0$
  \item $(F_1(x,y,z) + F_2(x,y,z) + F_3(x,y,z))^2 = 0$
  \item $(F_1(x,y,z)F_2(x,y,z)) + (F_2(x,y,z)F_3(x,y,z)) +
    (F_3(x,y,z)F_1(x,y,z)) = 0$
  \item $(F_1(x,y,z)F_2(x,y,z))^2 + (F_2(x,y,z)F_3(x,y,z))^2 +
    (F_3(x,y,z)F_1(x,y,z))^2 = 0$
  \end{enumerate}

  \vspace{0.1in}
  Your answer: $\underline{\qquad\qquad\qquad\qquad\qquad\qquad\qquad}$
  \vspace{0.1in}

\item Suppose we are given three subsets $\Gamma_1$, $\Gamma_2$, and
  $\Gamma_3$ of $\R^3$ where $\Gamma_1$ is the set of solutions to
  $F_1(x,y,z) = 0$, $\Gamma_2$ is the set of solutions to $F_2(x,y,z)
  = 0$, and $\Gamma_3$ is the set of solutions to $F_3(x,y,z) =
  0$. Which of the following equations gives precisely the set
  $\Gamma_1 \cap (\Gamma_2 \cup \Gamma_3)$?

  \begin{enumerate}[(A)]
  \item $F_1(x,y,z) + F_2(x,y,z)F_3(x,y,z) = 0$
  \item $(F_1(x,y,z))^2 + (F_2(x,y,z)F_3(x,y,z))^2 = 0$
  \item $(F_1(x,y,z) + F_2(x,y,z)F_3(x,y,z))^2 = 0$
  \item $F_1(x,y,z)(F_2(x,y,z) + F_3(x,y,z))^2 = 0$
  \item $F_1(x,y,z)((F_2(x,y,z))^2 + (F_3(x,y,z))^2) = 0$
  \end{enumerate}

  \vspace{0.1in}
  Your answer: $\underline{\qquad\qquad\qquad\qquad\qquad\qquad\qquad}$
  \vspace{0.1in}

\item Suppose we are given three subsets $\Gamma_1$, $\Gamma_2$, and
  $\Gamma_3$ of $\R^3$ where $\Gamma_1$ is the set of solutions to
  $F_1(x,y,z) = 0$, $\Gamma_2$ is the set of solutions to $F_2(x,y,z)
  = 0$, and $\Gamma_3$ is the set of solutions to $F_3(x,y,z) =
  0$. Which of the following equations gives precisely the set
  $\Gamma_1 \cup (\Gamma_2 \cap \Gamma_3)$?

  \begin{enumerate}[(A)]
  \item $F_1(x,y,z) + F_2(x,y,z)F_3(x,y,z) = 0$
  \item $(F_1(x,y,z))^2 + (F_2(x,y,z)F_3(x,y,z))^2 = 0$
  \item $(F_1(x,y,z) + F_2(x,y,z)F_3(x,y,z))^2 = 0$
  \item $F_1(x,y,z)(F_2(x,y,z) + F_3(x,y,z))^2 = 0$
  \item $F_1(x,y,z)((F_2(x,y,z))^2 + (F_3(x,y,z))^2) = 0$
  \end{enumerate}

  \vspace{0.1in}
  Your answer: $\underline{\qquad\qquad\qquad\qquad\qquad\qquad\qquad}$
  \vspace{0.1in}

\item Start with two vectors $a$ and $b$ in $\R^3$ such that $a \times
  b \ne 0$. Consider a sequence of vectors $c_1, c_2, \dots, c_n,
  \dots$ in $\R^3$ (note: each $c_n$ is a three-dimensional vector)
  defined as follows: $c_1 = a \times b$ and $c_{n+1} = a \times c_n$
  for $n \ge 1$. Which {\em one} of the following statements is {\bf
  false} about the $c_n$s? (5 points)

  \begin{enumerate}[(A)]
  \item All the vectors $c_n$ are nonzero vectors.
  \item $c_n$ and $c_{n+1}$ are orthogonal for every $n$.
  \item $c_n$ and $c_{n+2}$ are parallel for every $n$.
  \item $c_n$ and $a$ are orthogonal for every $n$.
  \item $c_n$ and $b$ are orthogonal for every $n$.
  \end{enumerate}

  \vspace{0.1in}
  Your answer: $\underline{\qquad\qquad\qquad\qquad\qquad\qquad\qquad}$
  \vspace{0.1in}

\item As a general rule, what would you expect should be the
  dimensionality of the set of solutions to $m$ independent and
  consistent equations in $n$ variables? By solution, we mean here
  that a $n$-tuple with coordinates in $\R$ (in other words, a point
  in $\R^n$) that satisfy all the $m$ equations. Assume $n \ge m \ge
  1$.

  \begin{enumerate}[(A)]
  \item $n$
  \item $m$
  \item $n - 1$
  \item $n - m$
  \item $1$
  \end{enumerate} 

  \vspace{0.1in}
  Your answer: $\underline{\qquad\qquad\qquad\qquad\qquad\qquad\qquad}$
  \vspace{0.1in}

\item As a general rule, what would you expect should be the
  dimensionality of the set of points in $\R^n$ that satisfy at least
  one of $m$ independent and consistent equations in $n$ variables?
  Assume $n \ge m \ge 1$.

  \begin{enumerate}[(A)]
  \item $n$
  \item $m$
  \item $n - 1$
  \item $n - m$
  \item $1$ 
  \end{enumerate} 

  \vspace{0.1in}
  Your answer: $\underline{\qquad\qquad\qquad\qquad\qquad\qquad\qquad}$
  \vspace{0.1in}


\item Measuring time $t$ in seconds since the beginning of the year
  2013, and stock prices on a $24 \times 7$ stock exchange in
  predetermined units, the stock prices of companies $A$, $B$, and $C$
  were found to be given by $30 + t/5000000 - \sin(t/10000)$, $16 +
  7t/3000000$, and $40 + t/1000000 - 5\sin(t/10000)$. To what extent
  can we deduce the stock prices of the companies from each other at a
  given point in time, without knowing what the time is?

  \begin{enumerate}[(A)]
  \item The stock price of any of the three companies can be used to
    deduce the other stock prices.
  \item The stock price of company $A$ can be used to deduce the stock
    prices of companies $B$ and $C$, but no other deductions are possible.
  \item The stock price of company $A$ can be used to deduce the stock
    prices of companies $B$ and $C$, and the stock price of company
    $C$ can be used to deduce the stock prices of companies $A$ and
    $B$.
  \item The stock price of company $B$ can be used to determine the
    stock prices of companies $A$ and $C$, and no other deductions are
    possible.
  \item The stock price of company $B$ can be used to determine the
    stock prices of companies $A$ and $C$, and the stock prices of
    companies $A$ and $C$ can be used to deduce each other but cannot
    be used to uniquely deduce the stock price of company $B$.
  \end{enumerate}

  \vspace{0.1in}
  Your answer: $\underline{\qquad\qquad\qquad\qquad\qquad\qquad\qquad}$
  \vspace{0.1in}

\item Lushanna is coaching $30$ young athletes for a 100 meter
  sprint. Every day, at the beginning of the day, she asks the athlete
  to run 100 meters as fast as they can and notes the time taken. She
  thus gets a vector with $30$ coordinates (measuring the time taken
  by all the athletes) everyday. Lushanna then plots a graph in
  thirty-dimensional space that includes all the points for her daily
  measurements. Each of the following is a sign that Lushanna's young
  athletes are improving. Which of these signs is {\bf strongest}, in
  the sense that it would imply all the others?

  \begin{enumerate}[(A)]
  \item The norm (length) of the vector every day (after the first) is
    less than the norm of the vector the previous day.
  \item The sum of the coordinates of the vector every day (after the
    first) is less than the sum of the coordinates of the vector the
    previous day.
  \item The minimum of the coordinates of the vector every day (after
    the first) is less than the minimum of the coordinates of the
    vector the previous day.
  \item The maximum of the coordinates of the vector every day (after
    the first) is less than the maximum of the coordinates of the
    vector the previous day.
  \item Each of the coordinates of the vector every day (after the
    first) is less than the corresponding coordinate of the vector the
    previous day.
  \end{enumerate}

  \vspace{0.1in}
  Your answer: $\underline{\qquad\qquad\qquad\qquad\qquad\qquad\qquad}$
  \vspace{0.1in}

\item In a closed system (no mass exchanged with the surroundings) a
  reversible chemical reaction $A + B \to C + D$, and its reverse, are
  proceeding. There are no other chemicals in the system, and no other
  reactions are proceeding in the system. A chemist studying the
  reaction decides to track the masses of $A$, $B$, $C$, and $D$ in
  the system as a function of time, and plots a parametric curve in
  four-dimensional space. What can we say about the nature of the
  curve, ignoring the parametrization (i.e., just looking at the set
  of points covered)?


  \begin{enumerate}[(A)]
  \item It is a part of a straight line.
  \item It is a part of a circle.
  \item It is a part of a parabola.
  \item It is a part of an astroid.
  \item It is a part of a cissoid.
  \end{enumerate}

  \vspace{0.1in}
  Your answer: $\underline{\qquad\qquad\qquad\qquad\qquad\qquad\qquad}$
  \vspace{0.1in}

  {\em Lobbying special}: Casa is a lobbyist for a special interest
  group. There are three politicians $P_1,P_2,P_3$ competing for a
  general election. Casa has computed that the probabilities of the
  politicians winning are $p_1$ for $P_1$, $p_2$ for $P_2$, and $p_3$
  for $P_3$, with $p_1,p_2,p_3 \in [0,1]$ and $p_1 + p_2 + p_3 =
  1$. Casa estimates a payoff of $m_1$ money units to her special
  interest group if $P_1$ wins, $m_2$ money units if $P_2$ wins, and
  $m_3$ money units if $P_3$ wins. (These payoffs may be in terms of
  passage of favorable laws, repeal of unfavorable laws, or
  enforcement of laws unfavorable to competitors).

\item What is the expected payoff to the special interest group that
  Casa represents?

  \begin{enumerate}[(A)]
  \item $m_1 + m_2 + m_3$
  \item $(m_1 + m_2 + m_3)/3$
  \item $(p_1 + p_2 + p_3)(m_1 + m_2 + m_3)$
  \item $p_1m_1 + p_2m_2 + p_3m_3$
  \item $\sqrt{m_1^2 + m_2^2 + m_3^2}$
  \end{enumerate}

  \vspace{0.1in}
  Your answer: $\underline{\qquad\qquad\qquad\qquad\qquad\qquad\qquad}$
  \vspace{0.1in}

\item Casa has discovered that the bribe multipliers of the
  politicians are the positive reals $b_1$, $b_2$, and $b_3$
  respectively. In other words, if Casa donates $u_i$ money units to
  $P_i$, then the expected payoff from politician $P_i$ winning is now
  $m_i + b_iu_i$. Consider the vectors $p = \langle p_1,p_2,p_3
  \rangle$, $m = \langle m_1,m_2,m_3 \rangle$, $c = \langle p_1b_1,
  p_2b_2, p_3b_3 \rangle$, and $f = \langle p_1/b_1, p_2/b_2, p_3/b_3
  \rangle$ and let $u = \langle u_1,u_2,u_3\rangle$ be the vector of
  the bribe quantities Casa gives to the politicians
  respectively. Assume that bribing politicians does not affect the
  relative probabilities of winning the election. Which of the
  following describes Casa's expected payoff from the election, once
  the bribe is made (if you want to include the cost of bribes, you'd
  need to subtract $u_1 + u_2 + u_3$ from this answer, but we're not
  doing that. {\em Note: Some of the answer options may not make sense
  from a dimensions/units viewpoint, but the correct answer does make
  sense}.

  \begin{enumerate}[(A)]
  \item $p \cdot (m + u)$
  \item $p \cdot (m + (b \cdot u))$
  \item $p \cdot (m + (f \cdot u))$
  \item $(p \cdot m) + (c \cdot u)$
  \item $p \cdot (f \cdot m + u)$
  \end{enumerate}

  \vspace{0.1in}
  Your answer: $\underline{\qquad\qquad\qquad\qquad\qquad\qquad\qquad}$
  \vspace{0.1in}

\item Continuing with the full setup of the preceing question, what is
  Casa's optimal bribing strategy on a fixed budget of money to be
  used for bribes?

  \begin{enumerate}[(A)]
  \item Donate all the money to the politician with the maximum value
    of $p_ib_i$, i.e., to the politician corresponding to the largest
    coordinate of the vector $c$.
  \item Donate all the money to the politician with the minimum value
    of $p_ib_i$, i.e., to the politician corresponding to the smallest
    coordinate of the vector $c$.
  \item Donate all the money to the politician with the maximum value
    of $p_i/b_i$, i.e., to the politician corresponding to the largest
    coordinate of the vector $f$.
  \item Donate all the money to the politician with the minimum value
    of $p_i/b_i$, i.e., to the politician corresponding to the smallest
    coordinate of the vector $f$.
  \item Split the bribery budget between the politicians in the ratio
    $p_1b_1:p_2b_2:p_3b_3$.
  \end{enumerate}

  \vspace{0.1in}
  Your answer: $\underline{\qquad\qquad\qquad\qquad\qquad\qquad\qquad}$
  \vspace{0.1in}

\end{enumerate}
\end{document}
