\documentclass[10pt]{amsart}
\usepackage{fullpage,hyperref,vipul,graphicx}
\title{Three-dimensional geometry}
\author{Math 195, Section 59 (Vipul Naik)}

\begin{document}
\maketitle

{\bf Corresponding material in the book}: Section 12.1

{\bf What students should definitely get}: The use of three
coordinates to describe points in space. Right-hand rule and
orientation issues. Octants and coordinate planes. Distance
formula. Equation of sphere.

{\bf What students should hopefully get}:
Three-dimensionality. Relation between describing equations and
dimensionality (dimensionality of surfaces, curves). The special case
of the linear situation.

\section*{Executive summary}

Words ...

\begin{enumerate}
\item Three-dimensional space is coordinatized using a Cartesian
  coordinate system by selecting three mutually perpendicular axes
  passing through a point called the origin: the $x$-axis, $y$-axis,
  and $z$-axis. These satisfy the right-hand rule. The coordinates of
  a point are written as a $3$-tuple $(x,y,z)$.
\item There are $2^3 = 8$ octants based on the signs of each of the
  coordinates. There are three coordinate planes, each corresponding
  to the remaining coordinate being zero (the $xy$-plane corresponds
  to $z = 0$, etc.). There are three axes, each corresponding to the
  other two coordinates being zero (e.g., the $x$-axis corresponds to
  $y = z = 0$).
\item The distance formula between points with coordinates
  $(x_1,y_1,z_1)$ and $(x_2,y_2,z_2)$ is: $$\sqrt{(x_2 - x_1)^2 + (y_2
  - y_1)^2 + (z_2 - z_1)^2}$$ This is similar to the formula in two
  dimensions and the squares and square root arises from the
  Pythagorean theorem.
\item The equation of a sphere with center having coordinates
  $(h,k,l)$ and radius $r$ is $(x - h)^2 + (y - k)^2 + (z - l)^2 =
  r^2$. Given an equation, we can try completing the square to see if
  it fits the model for the equation of a sphere.
\end{enumerate}

\section{Quick introduction}

\subsection{What's up with the number three?}

Why is three-dimensional space considered {\em three}-dimensional?
Because, in order to describe a point in this space, it requires three
coordinates, or three independent pieces of information. Remember,
dimensionality is the {\em number of degrees of freedom}, or the
{\em number of freely varying parameters}.

For now, the focus is on describing the behavior of a {\em Cartesian}
three-dimensional coordinate system. However, it is important to note
that {\em any} successful coordinate system to describe
three-dimensional space must use three freely varying
parameters. There does exist (though we will not be talking about it
right now) a {\em polar} coordinate system in three dimensions.

\subsection{Three coordinates: $x$, $y$, and $z$}

The three-dimensional Cartesian coordinate system involves the
following: choice of a point called the {\em origin}, and three
mutually perpendicular directed lines called the $x$-axis, $y$-axis,
and $z$-axis respectively. To specify a point with three coordinates,
we move parallel to the $x$-axis by the value of the $x$-coordinate,
then parallel to the $y$-axis by the value of the $y$-coordinate, then
parallel to the $z$-axis by the value of the $z$-coordinate.

The way to denote/describe a point with $x$-coordate $x_0$,
$y$-coordinate $y_0$, and $z$-coordinate $z_0$ is as
$(x_0,y_0,z_0)$. Or, more generally, we just write $(x,y,z)$.
\subsection{Orientation issues}

Recall that, for the two-dimensional plane, we adopt an orientation
convention, namely, the convention that the shorter angle from the
positive $x$-axis to the positive $y$-axis (measured $\pi/2$) be
counter-clockwise. In a three-dimesional system, we must adopt a
similar orientation convention. This convention is somewhat artificial
and in fact if you were doing mathematics properly then such a
convention would be meaningless. But since we're trying to do
mathematics the shortcut way, we will need to introduce this
convention.

The convention is called the {\em right-hand rule}. This says that if
you grip the $z$-axis with your right hand and curl the fingers of
your right hand so as to move from the $x$-axis to the $y$-axis, and
make your thumb point along the $z$-axis away from the fingers, then
it points along the positive $z$-axis. The book has a nice picture of
this, which I will not reproduce here.

We will thus deal with {\em right-handed coordinate systems}. The
mirror reflection of a right-hand coordinate system is a left-handed
coordinate system. There isn't really any difference between
left-handed and right-handed coordinate systems as far as their
geometry is concerned. However, they are different in the sense that
no amount of rotation can turn a right-handed coordinate system into a
left-handed coordinate system.

\subsection{Octants, axes, planes}

Since there are three coordinates, there are a total of $2^3 = 8$
possible sign combinations for the coordinates. These eight sign
combination give rise to eight chambers of three-dimensional space,
and these chambers are called {\em octants}. They are the
three-dimensional analogues of the {\em quadrants} in two-dimensional
space, of which there are $2^2 = 4$.

The eight octants are:

\begin{itemize}
\item Positive $x$, positive $y$, positive $z$
\item Positive $x$, positive $y$, negative $z$
\item Positive $x$, negative $y$, positive $z$
\item Positive $x$, negative $y$, negative $z$
\item Negative $x$, positive $y$, positive $z$
\item Negative $x$, positive $y$, negative $z$
\item Negative $x$, negative $y$, positive $z$
\item Negative $x$, negative $y$, negative $z$
\end{itemize}

There are also cases where one or more of the coordinates takes the
value $0$. These are discussed below:

\begin{itemize}
\item The $x$-coordinate takes the value $0$: The $yz$-plane.
\item The $y$-coordinate takes the value $0$: The $xz$-plane.
\item The $z$-coordinate takes the value $0$: The $xy$-plane.
\item Both the $y$ and $z$ coordinates take the value $0$: The
  $x$-axis.
\item Both the $x$ and $z$ coordinates take the value $0$: The
  $y$-axis.
\item Both the $y$ and $z$ coordinates take the value $0$: The
  $z$-axis.
\item All three coordinates are zero: We get the origin.
\end{itemize} 

\subsection{The distance formula}

The distance between the points $(x_1,y_1,z_1)$ and $(x_2,y_2,z_2)$ is
given by the formula:

$$\sqrt{(x_2 - x_1)^2 + (y_2 - y_1)^2 + (z_2 - z_1)^2}$$

Note that this is very similar to the two-dimensional distance
formula, and has a similar justification. Please note that the {\em
appearance of squares and square roots is independent of
dimensionality}. It has to do with the Pythagorean theorem. So when we
are working in three dimensions, we still deal with squares and square
roots, not with cubes and cube roots.

\section{Curves and surfaces}

\subsection{The arithmetic of dimensionality of curves and surfaces}

A surface is something two-dimensional, a curve is something
one-dimensional, and a point is something zero-dimensional.

Every condition, relation, or equation creates a constraint and
reduces the dimensionality of the set of possibilities by one. In
particular:

\begin{itemize}
\item A single equation reduces the dimensionality of the solution
  space by one. Since $3 - 1 = 2$, the solution set to a single
  equation in three dimensions is something two-dimensional, such as a
  plane or surface, or a union of finitely many planes and surfaces.
\item A pair of two equations reduces the dimensionality of the
  solution space by two. Since $3 - 2 = 1$, the solution set to a pair
  of such equations is something one-dimensional, such as a line or
  curve, or a union of finitely many lines and curves.
\item A triple of equations reduces the dimensionality of the solution
  space by three. Since $3 - 3 = 0$, the solution set to a tripe of
  such equations is something zero-dimensional, such as a point or a
  finite collection of points.
\end{itemize}

\subsection{Linear equations and systems}

We will talk about these in considerably more detail later on, but
here we simply note how linear equations fit into the model above.

\begin{itemize}
\item The solution set to a {\em linear equation} in $x$, $y$, and $z$
  is a plane, which is a flat example of a surface.

  In particular, the solution set to an equation of the form $x = x_0$
  is a plane parallel to the $yz$-plane. Similarly, $y = y_0$ gives a
  plane parallel to the $xz$-plane, and $z = z_0$ gives a plane
  parallel to the $xy$-plane.
\item The solution set to a {\em consistent but inequivalent pair of
  linear equations} in $x$, $y$, and $z$ is a line which is a flat
  example of a curve.

  In particular, the solution set to a pair of equations of the form
  $x = x_0$, $y = y_0$ is a line parallel to the $z$-axis. Similarly
  with the coordinate roles interchanged.
\item The solution set to a system of three consistent but independent
  linear equations in $x$, $y$ and $z$ is a single point.
\end{itemize}

\subsection{Ignoring one coordinate}

Suppose we have a relation $R(x,y) = 0$. The solution space to this in
{\em three} dimensions is the surface obtained by taking the solution
set in the $xy$-plane and then translating it along the $z$-direction
to cover every possible $z$-coordinate. We can think of it as
something cylinder-like.

For instance, the relation $x^2 + y^2 = 25$ gives a circle of radius
$5$ centered at the origin when viewed purely in the $xy$-plane. The
solution set overall is a cylinder stretching infinitely in both
directions, whose axis is the $z$-axis and whose cross sections are
these circles.

Similar remarks hold for a relation purely in terms of $y$ and $z$ or
a relation purely in terms of $x$ and $z$.

\subsection{Equation of a sphere}

The equation of a sphere whose center has coordinates $(h,k,l)$ and
whose radius is $r$ is given by:

$$(x - h)^2 + (y - k)^2 + (z - l)^2 = r^2$$

Given an equation, we can try doing things like completing the square
and see if, after we do that, we get the equation of a sphere.
\end{document}
