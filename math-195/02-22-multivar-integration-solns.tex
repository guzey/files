\documentclass[10pt]{amsart}

%Packages in use
\usepackage{fullpage, hyperref, vipul, enumerate}

%Title details
\title{Take-home class quiz solutions: due Friday February 22: Multi-variable integration}
\author{Math 195, Section 59 (Vipul Naik)}
%List of new commands

\begin{document}
\maketitle

\section{Performance review}

$24$ people took this quiz. The score distribution was as follows:

\begin{itemize}
\item Score of $2$: $1$ person
\item Score of $3$: $5$ people
\item Score of $4$: $7$ people
\item Score of $5$: $7$ people
\item Score of $6$: $2$ people
\item Score of $7$: $2$ people
\end{itemize}

The question wise answers and performance review are as follows:

\begin{enumerate}
\item Option (C): $14$ people
\item Option (A): $19$ people
\item Option (C): $17$ people
\item Option (D): $14$ people
\item Option (A): $7$ people
\item Option (C): $14$ people
\item Option (C): $20$ people
\item Option (D): $1$ person
\end{enumerate}

\section{Solutions}

The following setup is for the first five questions only.

Suppose $F$ is a function of two real variables, say $x$
and $t$, so $F(x,t)$ is a real number for $x$ and $t$ restricted to
suitable open intervals in the real number. Suppose, further, that $F$
is jointly continuous (whatever that means) in $x$ and $t$.

Define $f(t) := \int_0^\infty F(x,t) \, dx$. Here, while doing the
integration, $t$ is treated as a constant. $x$, the variable of
integration, is being integrated on $[0,\infty)$.
  
Suppose further that $f$ is defined and continuous for $t$ in
$(0,\infty)$.

In the next few questions, you are asked to compute the function $f$
explicitly given the function $F$, for $t \in (0,\infty)$.

\begin{enumerate}

\item {\em Do not discuss!} $F(x,t) := e^{-tx}$. Find $f$. {\em Last
  time: $15/19$ correct}

  \begin{enumerate}[(A)]
  \item $f(t) = e^{-t}/t$
  \item $f(t) = e^t/t$
  \item $f(t) = 1/t$
  \item $f(t) = -1/t$
  \item $f(t) = -t$
  \end{enumerate}

  {\em Answer}: Option (C)

  {\em Explanation}: The integral becomes
  $[-e^{-tx}/t]_0^\infty$. Plugging in at $\infty$ gives $0$ and
  plugging in at $0$ gives $-1/t$. Since the value at $0$ is being
  subtracted, we eventually get $1/t$.

  Note that the answer must be positive for the simple reason that we
  are integrating a positive function from left to right across an
  interval.

  {\em Performance review}: $14$ out of $24$ people got this. $9$
  chose (D), $1$ chose (E).

  {\em Historical note (last time)}: $15$ out of $19$ people got this
  correct. $3$ people chose (D) and $1$ person chose (B).

\item {\em Do not discuss!} $F(x,t) := 1/(t^2 + x^2)$. Find $f$. {\em
  Last time: $13/19$ correct}

  \begin{enumerate}[(A)]
  \item $f(t) = \pi/(2t)$
  \item $f(t) = \pi/t$
  \item $f(t) = 2\pi/t$
  \item $f(t) = \pi t$
  \item $f(t) = 2\pi t$
  \end{enumerate}

  {\em Answer}: Option (A)

  {\em Explanation}: We get $[(1/t)\arctan(x/t)]_0^\infty$. The
  evaluation at $\infty$ gives $\pi/(2t)$ and the evaluation at $0$
  gives $0$. Subtracting, we get $\pi/(2t)$.

  {\em Performance review}: $19$ out of $24$ people got this. $3$
  chose (B) and $2$ chose (C).

  {\em Historical note (last time)}: $13$ out of $19$ people got this
  correct. $2$ people each chose (B), (C), and (D).

\item {\em Do not discuss!} $F(x,t) := 1/(t^2 + x^2)^2$. Find $f$. {\em Last time: $13/19$ correct}

  \begin{enumerate}[(A)]
  \item $f(t) = \pi/t^3$
  \item $f(t) = \pi/(2t^3)$
  \item $f(t) = \pi/(4t^3)$
  \item $f(t) = \pi/(8t^3)$
  \item $f(t) = 3\pi/(8t^3)$
  \end{enumerate}

  {\em Answer}: Option (C)

  {\em Explanation}: Put in $\theta = \arctan(x/t)$. Substitute, and
  we get $(1/t^3) \int_0^{\pi/2} \cos^2\theta \,
  d\theta$. Integrating, we get $[\theta/2t^3 +
  \sin(2\theta)/4t^3]_0^{\pi/2}$. The trigonometric part vanishes
  between limits, and we are left with $\pi/(4t^3)$

  {\em Performance review}: $17$ out of $24$ people got this. $6$
  chose (B), $1$ chose (D).

  {\em Historical note (last time)}: $13$ out of $19$ people got this
  correct. $3$ people chose (D) and $1$ each chose (A), (B), and (E).

\item (*) {\em You can discuss this!} $F(x,t) = \exp(-(tx)^2)$. Use that
  $\int_0^\infty \exp(-x^2) \, dx= \sqrt{\pi}/2$. Find $f$. {\em Last
  time: $7/19$ correct}

  \begin{enumerate}[(A)]
  \item $f(t) = t^2\sqrt{\pi}/2$
  \item $f(t) = t\sqrt{\pi}/2$
  \item $f(t) = \sqrt{\pi}/2$
  \item $f(t) = \sqrt{\pi}/(2t)$
  \item $f(t) = \sqrt{\pi}/(2t^2)$
  \end{enumerate}

  {\em Answer}: Option (D)

  {\em Explanation}: Put $u = tx$, get a $1/t$ on the outside, giving
  $(1/t) \int_0^\infty \exp(-u^2) \, du$.

  {\em Performance review}: $14$ out of $24$ got this, $7$ chose (B),
  $2$ chose (E), $1$ chose (C).

  {\em Historical note (last time)}: $7$ out of $19$ people got this
  correct. $7$ people chose (E), $2$ each chose (A) and (B), and $1$
  chose (C).

\item (**) {\em You can discuss this!} (could confuse you if you don't
  understand it): In the same general setup as above (but with none of
  these specific $F$s), which of the following is a {\em sufficient}
  condition for $f$ to be an increasing function of $t$? {\em Last
  time: $3/19$ correct}

  \begin{enumerate}[(A)]
  \item $t \mapsto F(x_0,t)$ is an increasing function of $t$ for
    every choice of $x_0 \ge 0$.
  \item $x \mapsto F(x,t_0)$ is an increasing function of $x$ for
    every choice of $t_0 \in (0,\infty)$.
  \item $t \mapsto F(x_0,t)$ is a decreasing function of $t$ for
    every choice of $x_0 \ge 0$.
  \item $x \mapsto F(x,t_0)$ is a decreasing function of $x$ for
  every choice of $t_0 \in (0,\infty)$.
  \item None of the above.
  \end{enumerate}

  {\em Answer}: Option (A)

  {\em Explanation}: If $F$ is increasing in $t$ for every value of
  $x_0$, then that means that as $t$ gets bigger, the function $F$
  being integrated gets bigger everywhere in $x$, i.e., if $t_1 <
  t_2$, then $F(t_1,x_0) < F(t_2,x_0)$ for every $x_0 \ge 0$. The
  integral for the larger value $t_2$ must therefore also be
  bigger. (We looked at this stuff in Section 5.8 of the book).

  {\em Performance review}: $7$ out of $24$ got this. $9$ chose (C),
  $5$ chose (B), $2$ chose (D), $1$ chose (E).

  {\em Historical note (last time)}: $3$ out of $19$ people got this
  correct. $9$ chose (B), $4$ chose (E), $2$ chose (C), $1$ chose (D).

  (end of the setup)

  \vspace{0.3in}

\item {\em Do not discuss!} Suppose $f$ is a homogeneous polynomial of
  degree $d > 0$. Define $g$ as the following function on positive
  reals: $g(a)$ is the double integral of $f$ on the square $[0,a]
  \times [0,a]$. Assuming that $g(a)$ is not identically the zero
  function, which of these best describes the nature of $g(a)$? {\em
  Last time: $12/19$ correct}.

  \begin{enumerate}[(A)]
  \item A constant times $a^d$
  \item A constant times $a^{d + 1}$
  \item A constant times $a^{d + 2}$
  \item A constant times $a^{2d + 1}$
  \item A constant times $a^{2d + 2}$
  \end{enumerate}

  {\em Answer}: Option (C)

  {\em Explanation}: Each monomial is of the form a constant times
  $x^py^q$ where $p + q = d$. Integrating this as a multiplicatively
  separable function gives $x^{p+1}y^{q+1}$ times a
  constant. Evaluating between limits gives $a^{d + 2}$ times a
  constant. This is the form of the double integral of each monomial,
  and hence the double integral of the sum is also of the same form.

  {\em Performance review}: $14$ out of $24$ people got this. $5$ each
  chose (B) and (E).

  {\em Historical note (last time)}: $12$ out of $19$ people got this
  correct. $3$ people chose (D), $2$ each chose (A) and (E).

\item (*) {\em You can discuss this!} Suppose $g(x,y)$ and $G(x,y)$
  are continuous functions of two variables and $G_{xy} = g$. How can
  the double integral $\int_s^t \int_u^v g(x,y) \, dy \, dx$ be
  described in terms of the values of $G$? {\em Last time: $8/19$
  correct}

  \begin{enumerate}[(A)]
  \item $G(v,t) + G(u,s) - G(u,t) - G(v,s)$
  \item $G(v,t) - G(v,s) + G(u,t) - G(u,s)$
  \item $G(t,v) + G(s,u) - G(t,u) - G(s,v)$
  \item $G(t,v) - G(s,v) + G(t,u) - G(s,u)$
  \item $G(t,v) + G(v,t) - G(s,u) - G(u,s)$
  \end{enumerate}

  {\em Answer}: Option (C)

  {\em Explanation}: Note that the integration is over $[s,t] \times
  [u,v]$, i.e., the rectangular region with corner points $(s,u)$,
  $(s,v)$, $(t,u)$, and $(t,v)$. Recall that for the double integral,
  we put positive signs on the two extreme points (top right and
  bottom left) and negative signs on the other two points (top left
  and bottom right). See more in the lecture notes.

  {\em Performance review}: $20$ out of $24$ people got this. $2$
  chose (D), $1$ each chose (A) and (B).

  {\em Historical note (last time)}: $8$ out of $19$ people got this
  correct. $6$ people chose (D), $2$ each chose (A) and (E), $1$ chose
  (B).

\item (**) {\em You can discuss this!} Suppose $f$ is an elementarily
  integrable function, but $f(x^k)$ is not elementarily integrable for
  any integer $k > 1$ (examples are $\sin$, $\exp$, $\cos$). For which
  of the following types of regions $D$ are we {\em guaranteed to be
  able} to compute, in elementary function terms, the double integral
  $\int_D \int f(x^2) \, dA$ over the region (note that $f$ is just a
  function of $x$, but we treat it as a function of two variables)?
  Please see Option (E) before answering and select that if
  applicable.  {\em Last time: $1/19$ correct}.

  \begin{enumerate}[(A)]
  \item A rectangle with vertices $(0,0)$, $(0,b)$, $(a,0)$, and
    $(a,b)$, with $a,b > 0$.
  \item A triangle with vertices $(0,0)$, $(0,b)$, $(a,0)$, with $a, b
    > 0$.
  \item A triangle with vertices $(0,0)$, $(0,b)$, $(a,b)$, with $a, b
    > 0$.
  \item A triangle with vertices $(0,0)$, $(a,0)$, $(a,b)$, with $a, b
    > 0$.
  \item All of the above
  \end{enumerate}

  {\em Answer}: Option (D)

  {\em Explanation}: For such a triangle, we integrate on $y$ inner
  and $x$ outer. For a fixed value of $x$, the $y$-value ranges from
  $0$ to $bx/a$, so the integral becomes:

  $$\int_0^a \int_0^{bx/a} f(x^2) \, dy \, dx$$

  $f(x^2)$ pulls out of the inner integral and the inner integral
  just gives $(bx/a)$, so we get:

  $$\int_0^a \frac{b}{a} xf(x^2) \, dx$$

  This can be done by the substitution $u = x^2$ and the knowledge
  that $f$ is elementarily integrable.

  All the other integrations stumble because they require knowledge of
  an antiderivative of $f(x^2)$.

  {\em Performance review}: $1$ out of $24$ people got this. $7$ chose
  (A), $6$ each chose (B) and (E), and $4$ chose (C).

  {\em Historical note (last time)}: $1$ person got this correct. $8$
  chose (E), $5$ chose (A), $3$ chose (B), $2$ chose (C).

\end{enumerate}

\end{document}
