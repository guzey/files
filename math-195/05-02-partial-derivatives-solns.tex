\documentclass[10pt]{amsart}

%Packages in use
\usepackage{fullpage, hyperref, vipul, enumerate}

%Title details
\title{Class quiz solutions: May 2: Partial derivatives}
\author{Math 195, Section 59 (Vipul Naik)}
%List of new commands

\begin{document}
\maketitle

{\em Note}: The quiz was actually conducted on May 6.

\section{Performance review}

$20$ people took this $5$-question quiz. The score distribution was as
follows:

\begin{itemize}
\item Score of $0$: $3$ people
\item Score of $1$: $2$ people
\item Score of $2$: $6$ people
\item Score of $3$: $4$ people
\item Score of $4$: $5$ people
\end{itemize}
 
The mean score was $2.3$.
 
The question wise answers and performance were:

\begin{enumerate}
\item Option (B): $12$ people
\item Option (D): $14$ people
\item Option (B): $2$ people. {\em Really? This was an algebra
  question, not a calculus question!}
\item Option (D): $8$ people
\item Option (E): $10$ people
\end{enumerate}

\section{Solutions}
\begin{enumerate}

\item Consider a production function $f(L,K,T)$ of three inputs $L$
  (labor expenditure), $K$ (capital expenditure), and $T$ (technology
  expenditure). Suppose all mixed partials of $f$ with respect to $L$,
  $K$, and $T$ are continuous. Suppose we have the following signs of
  partial derivatives: $\partial f/\partial L > 0$, $\partial
  f/\partial K > 0$, $\partial^2f/(\partial L \partial K) < 0$, and
  $\partial^3f/(\partial L\partial K \partial T) > 0$. What does this
  mean?

  \begin{enumerate}[(A)]
  \item Increasing labor increases production, increasing capital
    increases production, and labor and capital substitute for each
    other to some extent. Increasing the expenditure on technology
    increases the degree to which labor and capital substitute for
    each other.
  \item Increasing labor increases production, increasing capital
    increases production, and labor and capital substitute for each
    other to some extent. Increasing the expenditure on technology
    decreases the degree to which labor and capital substitute for
    each other, i.e., with more technology investment, labor and
    capital become more complementary.
  \item Increasing labor increases production, increasing capital
    increases production, and labor and capital complement each other
    to some extent. Increasing the expenditure on technology increases
    the degree to which labor and capital complement for each other.
  \item Increasing labor increases production, increasing capital
    increases production, and labor and capital complement each other
    to some extent. Increasing the expenditure on technology decreases
    the degree to which labor and capital complement for each other.
  \item Increasing labor or capital decreases production.
  \end{enumerate}

  {\em Answer}: Option (B)

  {\em Explanation}: $\partial f/\partial L > 0$ shows that increasing
  labor increases production. $\partial f/\partial K > 0$ shows that
  increasing capital increases production. $\partial^2f/\partial L
  \partial K < 0$ indicates that labor and capital substitute for each
  other, i.e., a small increase in capital reduces the marginal
  product of labor. Finally, $\partial^3f/\partial L\partial K
  \partial T > 0$ indicates that $\partial^2f/\partial L \partial K$
  is increasing with $T$, i.e., getting less negative. So, although
  labor and capital substitute for each other, the degree to which
  they do so reduces as $T$ increases. Roughly speaking, more
  technology reduces the antagonism between labor and capital.

  {\em Performance review}: $12$ out of $20$ people got this
  correct. $7$ chose (A) and $1$ chose (D). The people who chose (A)
  probably didn't note that an increase in the degree of substitution
  would mean a decrease in the derivative, rather than an increasing.
\item Analysis of usage of an online social network finds that the
  total time spent by people on the social network is $P^{1.3}L^{0.5}$
  where $P$ is the total number of people on the network and $L$ is a
  number of processors used at the social network's server
  facility. Which of these is true?

  \begin{enumerate}[(A)]
  \item Increasing returns both on persons and on processors: every
    new person joining the network increases the average time spent
    {\em per person} (and not just the total time), and every new
    processor added to the server facility increases the average time
    spent per processor.
  \item Constant returns on persons, increasing returns on processors
  \item Constant returns on persons, decreasing returns on processors
  \item Increasing returns on persons, decreasing returns on processors
  \item Decreasing returns on persons, increasing returns on processors
  \end{enumerate}

  {\em Answer}: Option (D)

  {\em Explanation}: The short explanation is that the exponent on $P$
  is greater than $1$, so the second partial derivative is positive,
  and the exponent of $L$ is between $0$ and $1$, so the second
  partial derivative is negative.

  The long explanation is just working it out.

  {\em Performance review}: $14$ out of $20$ got this correct. $5$
  chose (A), $1$ chose (B).

\item {\em Not a calculus question, but has deep calculus
  interpretations -- it is basically measuring the derivative of the
  $1/x$ function with respect to $x$}: A person travels fifty miles
  every day by car and the travel distance is fixed. The price of
  gasoline, which she uses to fuel her car, is also fixed. Which of
  the following increases in fuel efficiency result in the maximum
  amount of savings for her?

  \begin{enumerate}[(A)]
  \item From $11$ to $12$ miles per gallon
  \item From $12$ to $14$ miles per gallon
  \item From $20$ to $25$ miles per gallon
  \item From $36$ to $54$ miles per gallon
  \item From $50$ to $100$ miles per gallon
  \end{enumerate}

  {\em Answer}: Option (B)

  {\em Explanation}: The gain achieved by upgrading from $a$ miles per
  gallon to $b$ miles per gallon is $(50/a - 50/b)$ times the cost of
  a gallon. In particular, it is proportional to $1/a - 1/b$. It
  remains to compute the case where this difference is largest.

  The values of $1/a - 1/b$ are: Option (A):/ $1/132$, Option (B):
  $1/84$, Option (C): $1/100$, Option (D): $1/108$, Option (E):
  $1/100$. Of these, the largest is the one with smallest denominator,
  i.e., $1/84$. In other words, the maximum gain happens in going from
  $12$ to $14$.

  This seems a little counter-intuitive at first. Looked at in terms
  of ratios, the gain from $50$ to $100$ is most impressive. Looked at
  in terms of differences in MPG values, again the gain from $50$ to
  $100$ is more impressive. However, these gains are not what we are
  measuring, because in the question, it is specified that the
  distance of travel is {\em fixed} and hence what matters is the
  absolute savings in cost.

  Intuitively, what's happening is that while a gain from $50$ to
  $100$ halves the cost, that halving is occurring from an already
  fairly small cost base, so the quantitative savings are little. On
  the other hand, a jump from $12$ to $14$ is small in proportion but
  large in absolute terms because the base from which the savings are
  occurring is much larger. In fact, even a gain from $100$ miles per
  gallon to infinite miles per gallon produces less in cost savings
  assuming fixed distance and fixed cost per gallon than a gain from
  $12$ to $14$.

  Another way of thinking of this is in terms of the derivative of
  $1/x$. We know that as $x$ increases, $1/x$ decreases. However, the
  derivative is not constant. When $x$ is small, the derivative
  $-1/x^2$ is huge in magnitude, which means that small changes in $x$
  lead to large changes in $1/x$. When $x$ is large, the derivative
  $-1/x^2$ is small in magnitude, which means that large changes in
  $x$ lead to only small changes in $1/x$.

  Thus, we can get three fairly different pictures depending on
  whether we measure things using $x$, $1/x$, or $\ln x$.

  {\em Performance review}: $2$ out of $20$ people got this
  correct. $7$ chose (C), $6$ chose (E), $4$ chose (D), and $1$ chose
  (A).
\item For which of the following production functions $f(L,K)$ of
  labor and capital is it true that labor and capital can be
  complementary for some choices of $(L,K)$, and substitutes for
  others? In other words, for which of these are labor and capital
  neither globally complements nor globally substitutes? Assume the
  domain $L > 0, K > 0$.

  \begin{enumerate}[(A)]
  \item $L^2 + LK + K^2$
  \item $L^2 - LK + K^2$
  \item $L^3 + L^2K + LK^2 + K^3$
  \item $L^3 + L^2K - LK^2 + K^3$
  \item $L^3 - L^2K - LK^2 + K^3$
  \end{enumerate}

  {\em Answer}: Option (D)

  {\em Explanation}: For option (D), the second mixed partial is $2(L
  - K)$ which is positive if $L > K$ and negative if $L < K$. For
  options (A) and (C), the second mixed partial is always positive,
  while for options (B) and (E), the second mixed partial is always
  negative.

  {\em Performance review}: $8$ out of $20$ people got this
  correct. $4$ each chose (B), (C) and (E).
\item Consider the following Leontief-like production function $f(L,K)
  = (\min \{ L, K \})^2$. What is the nature of returns and
  complementarity here?

  \begin{enumerate}[(A)]
  \item Positive increasing returns on the smaller of the inputs,
    positive constant returns on the larger of the inputs
  \item Positive constant returns of the smaller of the inputs,
    positive increasing returns on the larger of the inputs
  \item Zero returns on the smaller of the inputs, positive constant
    returns on the larger of the inputs
  \item Positive decreasing returns on the smaller of the inputs, zero
    returns on the larger of the inputs
  \item Positive increasing returns on the smaller of the inputs, zero
    returns on the larger of the inputs
  \end{enumerate}

  {\em Answer}: Option (E)

  {\em Explanation}: This is a ``weak link'' type of production
  function in the sense that the weakest link in the labor-capital
  nexus determines output. If $L < K$, then output is $L^2$, and if $K
  < L$, then output is $K^2$. This means that, at the margin,
  increasing the one which is already larger produces no gain in
  output. However, increasing the one which is smaller increases
  output as the square thereof. Since $2 > 1$, there are positive
  increasing returns on the smaller input.

  A practical example of this is where ``it takes two to tango'' --
  for instance, if each unit of labor is a person and each unit if
  capital is a machine, and if there are more machines than people or
  vice versa, the extra machines/people are completely unused.

  {\em Performance review}: $10$ out of $20$ people got this
  correct. $3$ each chose (A) and (D), $2$ each chose (B) and (C).
\end{enumerate}
\end{document}