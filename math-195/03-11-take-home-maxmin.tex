\documentclass[10pt]{amsart}

%Packages in use
\usepackage{fullpage, hyperref, vipul, enumerate}

%Title details
\title{Take-home class quiz: due Monday March 11: Max-min values: two-variable version}
\author{Math 195, Section 59 (Vipul Naik)}
%List of new commands

\begin{document}
\maketitle

Your name (print clearly in capital letters): $\underline{\qquad\qquad\qquad\qquad\qquad\qquad\qquad\qquad\qquad\qquad}$

{\bf YOU ARE FREE TO DISCUSS ALL QUESTIONS, BUT PLEASE ONLY ENTER FINAL ANSWER OPTIONS THAT YOU PERSONALLY ENDORSE. PLEASE DO {\em NOT} ENGAGE IN  GROUPTHINK.}

\begin{enumerate}
\item Suppose $F(x,y) := f(x) + g(y)$, i.e., $F$ is additively
  separable. Suppose $f$ and $g$ are differentiable functions of one
  variable, defined for all real numbers. What can we say about the
  critical points of $F$ in its domain $\R^2$?

  \begin{enumerate}[(A)]
  \item $F$ has a critical point at $(x_0,y_0)$ iff $x_0$ is a
    critical point for $f$ {\em or} $y_0$ is a critical point for $g$.
  \item $F$ has a critical point at $(x_0,y_0)$ iff $x_0$ is a
    critical point for $f$ {\em and} $y_0$ is a critical point for
    $g$.
  \item $F$ has a critical point at $(x_0,y_0)$ iff $x_0 + y_0$ is a
    critical point for $f + g$, i.e., the function $x \mapsto f(x) +
    g(x)$.
  \item $F$ has a critical point at $(x_0,y_0)$ iff $x_0y_0$ is a
    critical point for $fg$, i.e., the function $x \mapsto f(x)g(x)$.
  \item None of the above.
  \end{enumerate}
  
  \vspace{0.1in}
  Your answer: $\underline{\qquad\qquad\qquad\qquad\qquad\qquad\qquad}$
  \vspace{0.15in}

\item Suppose $F(x,y) := f(x)g(y)$ is a multiplicatively separable
  function. Suppose $f$ and $g$ are both differentiable functions of
  one variable defined for all real inputs. Consider a point
  $(x_0,y_0)$ in the domain of $F$, which is $\R^2$. Which of the
  following is true?

  \begin{enumerate}[(A)]
  \item $F$ has a critical point at $(x_0,y_0)$ if and only if $x_0$
    is a critical point for $f$ and $y_0$ is a critical point for $g$.
  \item If $x_0$ is a critical point for $f$ and $y_0$ is a critical
    point for $g$, then $(x_0,y_0)$ is a critical point for
    $F$. However, the converse is not necessarily true, i.e.,
    $(x_0,y_0)$ may be a critical point for $F$ even without $x_0$
    being a critical point for $f$ and $y_0$ being a critical point
    for $g$.
  \item If $(x_0,y_0)$ is a critical point for $F$, then $x_0$ must be
    a critical point for $f$ and $y_0$ must be a critical point for
    $g$. However, the converse is not necessarily true.
  \item $(x_0,y_0)$ is a critical point for $F$ if and only if {\em at
    least} one of these is true: $x_0$ is a critical point for $f$ and
    $y_0$ is a critical point for $g$.
  \item None of the above.
  \end{enumerate}

  \vspace{0.1in}
  Your answer: $\underline{\qquad\qquad\qquad\qquad\qquad\qquad\qquad}$
  \vspace{0.15in}

\item Consider a homogeneous polynomial $ax^2 + bxy + cy^2$ of degree
  two in two variables $x$ and $y$. Assume that at least one of the
  numbers $a$, $b$, and $c$ is nonzero. What can we say about the local
  extreme values of this polynomial on $\R^2$?

  \begin{enumerate}[(A)]
  \item If $b^2 - 4ac < 0$, then the function has no local extreme
    values and its value is unbounded from both above and below. If
    $b^2 - 4ac = 0$, the function has local extreme value $0$ and this
    is attained on a line through the origin. If $b^2 - 4ac > 0$, the
    function has local extreme value $0$ and this is attained only at
    the origin.
  \item If $b^2 - 4ac < 0$, then the function has no local extreme
    values and its value is unbounded from both above and below. If
    $b^2 - 4ac > 0$, the function has local extreme value $0$ and this
    is attained on a line through the origin. If $b^2 - 4ac = 0$, the
    function has local extreme value $0$ and this is attained only at
    the origin.
   \item If $b^2 - 4ac > 0$, then the function has no local extreme
    values and its value is unbounded from both above and below. If
    $b^2 - 4ac = 0$, the function has local extreme value $0$ and this
    is attained on a line through the origin. If $b^2 - 4ac < 0$, the
    function has local extreme value $0$ and this is attained only at
    the origin.
  \item If $b^2 - 4ac > 0$, then the function has no local extreme
    values and its value is unbounded from both above and below. If
    $b^2 - 4ac < 0$, the function has local extreme value $0$ and this
    is attained on a line through the origin. If $b^2 - 4ac = 0$, the
    function has local extreme value $0$ and this is attained only at
    the origin.
  \item If $b^2 - 4ac = 0$, then the function has no local extreme
    values and its value is unbounded from both above and below. If
    $b^2 - 4ac < 0$, the function has local extreme value $0$ and this
    is attained on a line through the origin. If $b^2 - 4ac > 0$, the
    function has local extreme value $0$ and this is attained only at
    the origin.
  \end{enumerate}

  \vspace{0.1in}
  Your answer: $\underline{\qquad\qquad\qquad\qquad\qquad\qquad\qquad}$
  \vspace{0.15in}

  A subset of $\R^n$ is termed {\em convex} if the line segment
  joining any two points in the subset is completely within the
  subset. A function $f$ of two variables defined on a closed convex
  domain is termed {\em quasiconvex} if given any two points $P$ and
  $Q$ in the domain, the maximum of $f$ restricted to the line segment
  joining $P$ and $Q$ is attained at one (possibly both) of the
  endpoints $P$ or $Q$.

  There are many examples of quasiconvex functions, including linear
  functions (which are quasiconvex but not strictly quasiconvex)
  and all convex functions.

\item What can we say about the maximum of a continuous quasiconvex
  function defined on the circular disk $x^2 + y^2 \le 1$?

  \begin{enumerate}[(A)]
  \item It must be attained at the center of the disk, i.e., the
    origin $(0,0)$.
  \item It must be attained somewhere in the interior of the disk, but
    we cannot be more specific with the given information.
  \item It must be attained somewhere on the boundary circle $x^2 +
    y^2 = 1$. However, we cannot be more specific than that with the
    given information.
  \item It must be attained at one of the four points $(1,0)$,
    $(0,1)$, $(-1,0)$, and $(0,-1)$. 
  \item It could be attained at any point. We cannot be specific at all.
  \end{enumerate}

  \vspace{0.1in}
  Your answer: $\underline{\qquad\qquad\qquad\qquad\qquad\qquad\qquad}$
  \vspace{0.15in}


\item What can we say about the maximum of a continuous quasiconvex
  function defined on the square region $|x| + |y|\le 1$? This is the
  region bounded by the square with vertices $(1,0)$, $(0,1)$,
  $(-1,0)$, and $(0,-1)$.

  \begin{enumerate}[(A)]
  \item It must be attained at the center of the square, i.e., the
    origin $(0,0)$.
  \item It must be attained somewhere in the interior of the square, but
    we cannot be more specific with the given information.
  \item It must be attained somewhere on the boundary square $|x| +
    |y| \le 1$. However, we cannot be more specific than that with the
    given information.
  \item It must be attained at one of the four points $(1,0)$,
    $(0,1)$, $(-1,0)$, and $(0,-1)$. 
  \item It could be attained at any point. We cannot be specific at all.
  \end{enumerate}

  \vspace{0.1in}
  Your answer: $\underline{\qquad\qquad\qquad\qquad\qquad\qquad\qquad}$
  \vspace{0.15in}

\item Suppose $F(x,y) := f(x) + g(y)$, i.e., $F$ is additively
  separable. Suppose $f$ and $g$ are continuous functions of one
  variable, defined for all real numbers. Which of the following
  statements about local extrema of $F$ is {\bf false}?

  \begin{enumerate}[(A)]
  \item If $f$ has a local minimum at $x_0$ and $g$ has a local
    minimum at $y_0$, then $F$ has a local minimum at $(x_0,y_0)$.
  \item If $f$ has a local minimum at $x_0$ and $g$ has a local
    maximum at $y_0$, then $F$ has a saddle point at $(x_0,y_0)$.
  \item If $f$ has a local maximum at $x_0$ and $g$ has a local
    minimum at $y_0$, then $F$ has a saddle point at $(x_0,y_0)$.
  \item If $f$ has a local maximum at $x_0$ and $g$ has a local
    maximum at $y_0$, then $F$ has a local maximum at $(x_0,y_0)$.
  \item None of the above, i.e., they are all true.
  \end{enumerate}
  
  \vspace{0.1in}
  Your answer: $\underline{\qquad\qquad\qquad\qquad\qquad\qquad\qquad}$
  \vspace{0.15in}

\item Suppose $F(x,y) := f(x)g(y)$ is a multiplicatively separable
  function. Suppose $f$ and $g$ are both continuous functions of
  one variable defined for all real inputs. Consider a point
  $(x_0,y_0)$ in the domain of $F$, which is $\R^2$. Which of the
  following statements about local extrema is {\bf true}?

  \begin{enumerate}[(A)]
  \item If $f$ has a local minimum at $x_0$ and $g$ has a local
    minimum at $y_0$, then $F$ has a local minimum at $(x_0,y_0)$.
  \item If $f$ has a local minimum at $x_0$ and $g$ has a local
    maximum at $y_0$, then $F$ has a saddle point at $(x_0,y_0)$.
  \item If $f$ has a local maximum at $x_0$ and $g$ has a local
    minimum at $y_0$, then $F$ has a saddle point at $(x_0,y_0)$.
  \item If $f$ has a local maximum at $x_0$ and $g$ has a local
    maximum at $y_0$, then $F$ has a local maximum at $(x_0,y_0)$.
  \item None of the above, i.e., they are all false.
  \end{enumerate}

  \vspace{0.1in}
  Your answer: $\underline{\qquad\qquad\qquad\qquad\qquad\qquad\qquad}$
  \vspace{0.15in}

\end{enumerate}

\end{document}