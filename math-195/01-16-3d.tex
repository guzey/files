\documentclass[10pt]{amsart}

%Packages in use
\usepackage{fullpage, hyperref, vipul, enumerate}

%Title details
\title{Take-home class quiz: due Wednesday January 16: Three dimensions}
\author{Math 195, Section 59 (Vipul Naik)}
%List of new commands

\begin{document}
\maketitle

Your name (print clearly in capital letters): $\underline{\qquad\qquad\qquad\qquad\qquad\qquad\qquad\qquad\qquad\qquad}$

\vspace{0.1in}

{\bf THIS IS A TAKE-HOME CLASS QUIZ, BUT I WILL GIVE YOU ABOUT 5
  MINUTES TO REVIEW YOUR ANSWERS IN CLASS AND DISCUSS WITH OTHER
  STUDENTS.}

\vspace{0.1in}

{\bf YOU ARE ALLOWED TO DISCUSS ONLY QUESTIONS THAT BEGIN WITH A (*)
  OR (**). PLEASE ATTEMPT ALL OTHER QUESTIONS BY YOURSELF. EVEN FOR
  THE QUESTIONS YOU DISCUSS, PLEASE FINALLY ENTER ONLY THE ANSWER
  OPTION YOU ARE PERSONALLY MOST CONVINCED ABOUT -- DON'T ENGAGE IN
  GROUPTHINK.}

\begin{enumerate}

\item (*) Consider the subset of $\R^3$ given by the condition $(x^2 +
  y^2 - 1)(y^2 + z^2 - 1)(x^2 + z^2 - 1) = 0$. What kind of subset is
  this? {\em Last time: $12/24$ correct}

  \begin{enumerate}[(A)]
  \item It is a sphere centered at the origin and of radius $1$.
  \item It is the union of three circles, each centered at the origin
    and of radius $1$, and lying in the $xy$-plane, $yz$-plane, and
    $xz$-plane respectively.
  \item It is the union of three cylinders, each of radius $1$, about
    the $x$-axis, $y$-axis, and $z$-axis respectively.
  \item It is the intersection of three circles, each centered at the origin
    and of radius $1$, and lying in the $xy$-plane, $yz$-plane, and
    $xz$-plane respectively.
  \item It is the intersection of three cylinders, each of radius $1$, about
    the $x$-axis, $y$-axis, and $z$-axis respectively.
  \end{enumerate}

  \vspace{0.1in}
  Your answer: $\underline{\qquad\qquad\qquad\qquad\qquad\qquad\qquad}$
  \vspace{0.1in}

\item Given two distinct points $A$ and $B$ in three-dimensional
  space, what is the nature of the set of possibilities for a third
  point $C$ such that $AC$ and $BC$ have equal length (i.e., $C$ is
  equidistant from $A$ and $B$)? {\em Didn't appear last time}

  \begin{enumerate}[(A)]
  \item Sphere
  \item Plane
  \item Circle
  \item Line
  \item Two points
  \end{enumerate}

  \vspace{0.1in}
  Your answer: $\underline{\qquad\qquad\qquad\qquad\qquad\qquad\qquad}$
  \vspace{0.1in}

\item Given two distinct points $A$ and $B$ in three-dimensional
  space, what is the nature of the set of possibilities for a third
  point $C$ such that the triangle $ABC$ is a right triangle with $AB$
  as its hypotenuse? {\em Last time: $15/24$ correct}

  \begin{enumerate}[(A)]
  \item Sphere (minus two points)
  \item Plane
  \item Circle (minus two points)
  \item Line
  \item Square
  \end{enumerate}

  \vspace{0.1in}
  Your answer: $\underline{\qquad\qquad\qquad\qquad\qquad\qquad\qquad}$
  \vspace{0.1in}

\item Given two distinct points $A$ and $B$ in three-dimensional
  space, what is the nature of the set of possibilities for a third
  point $C$ such that the triangle $ABC$ is a right isosceles triangle
  with $AB$ as its hypotenuse? {\em Didn't appear last time}.

  \begin{enumerate}[(A)]
  \item Sphere
  \item Plane
  \item Circle
  \item Line
  \item Square
  \end{enumerate}

  \vspace{0.1in}
  Your answer: $\underline{\qquad\qquad\qquad\qquad\qquad\qquad\qquad}$
  \vspace{0.1in}

\item Given two distinct points $A$ and $B$ in three-dimensional
  space, what is the nature of the set of possibilities for a third
  point $C$ such that the triangle $ABC$ is equilateral? {\em Last
  year: $23/24$ correct}.

  \begin{enumerate}[(A)]
  \item Sphere
  \item Plane
  \item Circle
  \item Line
  \item Two points
  \end{enumerate}

  \vspace{0.1in}
  Your answer: $\underline{\qquad\qquad\qquad\qquad\qquad\qquad\qquad}$
  \vspace{0.1in}

\item Given two distinct points $A$ and $B$ in three-dimensional
  space, what is the nature of the set of possibilities for a third
  point $C$ such that $|AC|/|BC| = \lambda$ for $\lambda$ a fixed
  positive real number not equal to $1$? {\em Didn't appear last time}.

  \begin{enumerate}[(A)]
  \item Sphere
  \item Plane
  \item Circle
  \item Line
  \item Square
  \end{enumerate}

  \vspace{0.1in}
  Your answer: $\underline{\qquad\qquad\qquad\qquad\qquad\qquad\qquad}$
  \vspace{0.1in}

\item Consider the parametric curve in three dimensions given by the
  coordinate description $t \mapsto (\cos t, \sin t, \cos(2t))$, with
  $t \in \R$. We can consider the {\em projections} of this curve onto
  the $xy$-plane, $yz$-plane, and $xz$-plane, which are basically what
  we get by dropping perpendiculars from the curve to these
  planes. What is the correct description of the curves obtained by
  doing the three projections? {\em Last time: $17/24$ correct}

  \begin{enumerate}[(A)]
  \item The projections on the $xy$-plane and $yz$-plane are both parts
    of parabolas, and the projection on the $xz$-plane is a circle.
  \item The projections on the $xy$-plane and $yz$-plane are both
    circles, and the projection on the $xz$-plane is a part of a
    parabola.
  \item The projection on the $xy$-plane is a circle, and the
    projections on the $yz$-plane and $xz$-plane are both parts of
    parabolas.
  \item The projection on the $xy$-plane is a part of a parabola, the
    projection on the $xz$-plane and $yz$-plane are both circles.
  \item All the three projections are circles.
  \end{enumerate}

  \vspace{0.1in}
  Your answer: $\underline{\qquad\qquad\qquad\qquad\qquad\qquad\qquad}$
  \vspace{0.1in}

\end{enumerate}
\end{document}