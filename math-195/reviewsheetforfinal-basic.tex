\documentclass[10pt]{amsart}
\usepackage{fullpage,hyperref,vipul, graphicx}
\title{Review sheet for final: basic}
\author{Math 195, Section 59 (Vipul Naik)}

\begin{document}
\maketitle

{\em The document does not include material that was part of the
midterm 1 and midterm 2 review sessions. Please also bring copies of
these review sheets to the review session on Monday}.

You are expected to review this on your own time. We will concentrate
on the {\em advanced} review sheet during problem session.

\section{Directional derivatives and gradient vectors}

Words ...

\begin{enumerate}
\item The directional derivative of a scalar function
  $f$ of two variables along a {\em unit} vector $\mathbf{u} =
  a\mathbf{i} + b\mathbf{j}$ at a point $(x_0,y_0)$ is defined as the
  following limit of difference quotient, if the limit exists:

  $$\lim_{h \to 0} \frac{f(x_0 + ah, y_0 + bh) - f(x_0,y_0)}{h}$$

\item The directional derivative of a differentiable scalar function
  $f$ of two variables along a {\em unit} vector $\mathbf{u} =
  a\mathbf{i} + b\mathbf{j}$ at a point $(x_0,y_0)$ is
  $D_{\mathbf{u}}(f) = af_x(x_0,y_0) + bf_y(x_0,y_0)$.
\item The gradient vector for a {\em differentiable} scalar function
  $f$ of two variables at a point $(x_0,y_0)$ is $\nabla f(x_0,y_0) =
  f_x(x_0,y_0)\mathbf{i} + f_y(x_0,y_0)\mathbf{j}$.
\item The directional derivative of $f$ is the dot product of the
  gradient vector of $\nabla f$ and the unit vector $\mathbf{u}$. 
\item Suppose $\nabla f$ is nonzero. Then, if $\mathbf{u}$ makes an
  angle $\theta$ with $\nabla f$, then $D_{\mathbf{u}}(f)$ is
  $|\nabla_f|\cos \theta$. The directional derivative is maximum in
  the direction of the gradient vector, zero in directions orthogonal
  to the gradient vector, and minimum in the direction opposite to the
  gradient vector.
\item The level curves are orthogonal to the gradient vector.
\item Similar formulas for gradient vector and directional derivative
  work in three dimensions.
\item The level surfaces are orthogonal to the gradient vector for a
  function of three variables.
\item For a surface given by $F(x,y,z) = 0$, if $(x_0,y_0,z_0)$ is a
  point on the surface, and $F_x(x_0,y_0,z_0)$, $F_y(x_0,y_0,z_0)$,
  and $F_z(x_0,y_0,z_0)$ all exist and are nonzero, then the normal
  line is given by:

  $$\frac{x - x_0}{F_x(x_0,y_0,z_0)} = \frac{y - y_0}{F_y(x_0,y_0,z_0)} = \frac{z - z_0}{F_z(x_0,y_0,z_0)}$$

  The tangent plane is given by:

  $$F_x(x_0,y_0,z_0)(x - x_0) + F_y(x_0,y_0,z_0)(y - y_0) + F_z(x_0,y_0,z_0)(z - z_0) = 0$$
\end{enumerate}

\section{Max-min values}

Words ...

\begin{enumerate}
\item For a directional local minimum, the directional derivative (in
  the outward direction from the point) is greater than or equal to
  zero. For a directional local maximum, the directional derivative
  (in the outward direction from the point) is less than or equal to
  zero.

  Note that even for {\em strict} directional local maximum or
  minimum, the possibility of the directional derivative being zero
  cannot be ruled out.
\item If a point is a point of directional local minimum from two
  opposite directions (i.e., it is a local minimum along a line
  through the point, from both directions on the line) then the
  directional derivative along the line, if it exists, must equal
  zero.
\item If a function of two variables is differentiable at a point of
  local minimum or local maximum, then the directional derivative of
  the function is zero at the point in every direction. Equivalently,
  the gradient vector of the function at the point is the zero
  vector. Equivalently, both the first partial derivatives at the
  point are zero.
  
  Points where the gradient vector is zero are termed {\em critical
  points}. 
\item If the directional derivatives along some directions are
  positive and the directional derivatives along other directions are
  negative, the point is likely to be a {\em saddle point}. A saddle
  point is a point for which the tangent plane to the surface that's
  the graph of the function slides through the graph, i.e., it is not
  completely on one side.
\item For a function $f$ of two variables with continuous second
  partials, and a critical point $(a,b)$ in the domain (so $f_x(a,b) =
  f_y(a,b) = 0$) we compute the Hessian determinant:

  $$D(a,b) = f_{xx}(a,b)f_{yy}(a,b) - [f_{xy}(a,b)]^2$$

  If $D(a,b) > 0$ and $f_{xx}(a,b) > 0$, the function has a local {\em
  minimum} at the point $(a,b)$. If $D(a,b) > 0$ and $f_{xx}(a,b) <
  0$, the function has a local {\em maximum} at the point $(a,b)$. If
  $D(a,b) < 0$, we get a saddle point at the point. If $D(a,b) = 0$,
  the situation is inconclusive, i.e., the test is indecisive.
\item For a closed bounded subset of $\R^n$ (and specifically $\R^2$)
  any continuous function with domain that subset attains its absolute
  maximum and minimum values. These values are attained either at
  interior points (in which case they are local extreme values and
  must be attained at critical points) or at boundary points.
\item {\em Relation with level curves}: {\em Typically}, local extreme
  values correspond to isolated single point level curves. However,
  this is not always the case, and there are some counterexamples. To
  be more precise, any {\em isolated} or {\em strict} local extreme
  value corresponds to a (locally) single point level curve.
\end{enumerate}

Actions ...

\begin{enumerate}
\item Strategy for finding local extreme values: First, find all the
  critical points by solving $f_x(a,b) = 0$ and $f_y(a,b) = 0$ as a
  pair of simultaneous equations. Next, use the second derivative test
  for each critical point, and if feasible, try to figure out if this
  is a point of local maximum, or local minimum, or a saddle point.
\item To find absolute extreme values of a function on a closed
  bounded subset of $\R^2$, first find critical points, then find
  critical points for a parameterization of the boundary, and then
  compute values at all of these and see which is largest and
  smallest. {\em If the list of critical points is finite, and we need
  to find absolute maximum and minimum, it is not necessary to do the
  second derivative test to figure out which points give local
  maximum, local minimum, or neither, we just need to evaluate at all
  points and find the maximum/minimum}.
\item When the domain of the function is bounded but not closed, we
  must consider the possibility of extreme values occurring as we
  approach boundary points not in the domain. If the domain is not
  bounded, we must consider directions of approach to infinity.
\end{enumerate}

\section{Lagrange multipliers}

Words ...

\begin{enumerate}
\item Two of the reasons why the derivative of a function may be zero:
  the function is constant around the point, or the function has a local
  extreme value at the point.

  Version for many variables: two of the reasons why the gradient
  vector of a function of many variables may be zero: the function is
  constant around the point, or the function has a local extreme value
  at the point.

  Version for function restricted to a subset smooth around a point:
  two of the reasons why the gradient vector may be {\em orthogonal}
  to the subset at the point: the function is constant on the subset
  around the point, or the function has a local extreme value
  (relative to the subset) at the point.
\item For a function $f$ defined on a subset smooth around a point
  (i.e., with a well defined tangent and normal space), if $f$ has a
  local extreme value at the point when restricted to the subset, then
  $\nabla f$ lives in the normal direction to the subset (this
  includes the possibility of it being zero).
\item For a codimension one subset of $\R^n$ defined by a condition
  $g(x_1,x_2,\dots,x_n) = k$, if a point $(a_1,a_2,\dots,a_n)$ gives a
  local extreme value for a function $f$ of $n$ variables, and if
  $\nabla g$ is well defined and nonzero at the point, then there
  exists a real number $\lambda$ such that $\nabla
  f(a_1,a_2,\dots,a_n) = \lambda \nabla g(a_1,a_2,\dots,a_n)$. Note
  that $\lambda$ may be zero.
\item Suppose a codimension $r$ subset of $\R^n$ is given by $r$
  independent constraints $g_1(x_1,x_2,\dots,x_n) = k_1$,
  $g_2(x_1,x_2,\dots,x_n) = k_2$, and so on till
  $g_r(x_1,x_2,\dots,x_n) = k_r$. Suppose $\nabla g_i$ are nonzero for
  all $i$ at a point $(a_1,a_2,\dots,a_n)$ of local extreme value for
  a function $f$ relative to this subset. Suppose further that all the
  $\nabla g_i$ are linearly independent. Then $\nabla
  f(a_1,a_2,\dots,a_n)$ is a linear combination of the vectors $\nabla
  g_1(a_1,a_2,\dots,a_n)$, $\nabla g_2(a_1,a_2,\dots,a_n)$, $\dots$,
  $\nabla g_r(a_1,a_2,\dots,a_n)$. In other words, there exist real
  numbers $\lambda_1$, $\lambda_2$, $\dots$, $\lambda_r$ such that:

  $$\nabla f(a_1,a_2,\dots,a_n) = \lambda_1\nabla g_1(a_1,a_2,\dots,a_n) + \lambda_2\nabla g_2(a_1,a_2,\dots,a_n) + \dots + \lambda_r\nabla g_r(a_1,a_2,\dots,a_n)$$
\item The Lagrange condition may be violated at points of local
  extremum where $\nabla g$ is zero, or more generally, where the
  $\nabla g_i$ fail to be linearly independent. This may occur either
  because the tangent and normal space are not well defined or because
  the functions fail to capture it well.
\end{enumerate}

Actions ...

\begin{enumerate}
\item Suppose we want to maximize and minimize $f$ on the set
  $g(x_1,x_2,\dots,x_n) = k$. Assume $\nabla g(x_1,x_2,\dots,x_n)$ is
  defined everywhere on the set and never zero. Suppose $\nabla f$ is
  also defined. Then, all local maxima and local minima are attained
  at points where $\nabla f = \lambda \nabla g$ for some real number
  $\lambda$. To find these, we solve the system of $n + 1$ equations
  in the $n+1$ variables $x_1$, $x_2$, $\dots$, $x_n$, namely the $n$
  scalar equations from the Lagrange condition and the equation
  $g(x_1,x_2,\dots,x_n) = k$.

  To find the actual extreme values, once we've collected all
  candidate points from the above procedure, we evaluate the function
  at all these and find the largest and smallest value to find the
  absolute maximum and minimum.

\item If there are points in the domain where $\nabla g$ takes the
  value $0$, these may also be candidates for local extreme values,
  and the function should additionally be evaluated at these as well
  to find the absolute maximum and minimum.
\item A similar procedure works for a subset given by $r$
  constraints. In this case, we have the equation:

  $$\nabla f(a_1,a_2,\dots,a_n) = \lambda_1\nabla g_1(a_1,a_2,\dots,a_n) + \lambda_2\nabla g_2(a_1,a_2,\dots,a_n) + \dots + \lambda_r\nabla g_r(a_1,a_2,\dots,a_n)$$

  as well as the $r$ equations $g_1(x_1,x_2,\dots,x_n) = k_1$,
  $g_2(x_1,x_2,\dots,x_n) = k_2$, and so on. In total, we have
  $n + r$ equations in $n + r$ variables: the $x_1$, $x_2$, $\dots$,
  $x_n$ and the $\lambda_1$, $\lambda_2$, $\dots$, $\lambda_r$.
\end{enumerate}

\section{Max-min values: examples}

\begin{enumerate}
\item {\em Additively separable, critical points}: For an additively
  separable function $F(x,y) := f(x) + g(y)$, the critical points of
  $F$ are the points whose $x$-coordinate gives a critical point for
  $f$ and $y$-coordinate gives a critical point for $g$.
\item {\em Additively separable, local extreme values}: The local
  maxima occur at points whose $x$-coordinate gives a local maximum
  for $f$ and $y$-coordinates gives a local maximum for $g$. Similarly
  for local minima. If one coordinate gives a local maximum and the
  other coordinate gives a local minimum, we get a saddle point.
\item {\em Additively separable, absolute extreme values}: If the
  domain is a rectangular region, rectangular strip, or the whole
  plane, then the absolute maximum occurs at the point for which each
  coordinate gives the absolute maximum for that coordinate, and
  analogously for absolute minimum. This does {\em not} work for
  non-rectangular regions in general.
\item {\em Multiplicatively separable, critical points}: For a
  multiplicatively separable function $F(x,y) := f(x)g(y)$ with $f$,
  $g$, differentiable, there are four kinds of critical points
  $(x_0,y_0)$: (1) $f'(x_0) = g'(y_0) = 0$, (2) $f(x_0) = f'(x_0) =
  0$, (3) $g(y_0) = g'(y_0) = 0$, (4) $f(x_0) = g(y_0) = 0$.
\item {\em Multiplicatively separable, local extreme values}: At a
  critical point of Type (1), the nature of local extreme value for
  $F$ depends on the signs of $f$ and $g$ {\em and} on the nature of
  local extreme values for each. See the table. Critical points of
  Type (4) alone do not give local extreme values. The situation with
  critical points of Types (2) and (3) is more ambiguous and too
  complicated for discussion.
\item {\em Multiplicatively separable, absolute extreme values}:
  Often, these don't exist, if one function takes arbitrarily large
  magnitude values and the other one takes nonzero values (details
  based on sign). If both functions are everywhere positive, and we
  are on a rectangular region, then the absolute maximum/minimum for
  the product occur at points whose coordinates give respective
  absolute maximum/minimum for $f$ and $g$. (See notes)
\item For a continuous quasiconvex function on a convex domain, the
  maximum must occur at one of the extreme points, in
  particular on the boundary. If the function is strictly quasiconvex,
  the maximum can occur only at a boundary point.
\item For a continuous quasiconvex function on a convex domain, the
  minimum must occur on a convex subset. If the function is strictly
  quasiconvex, it must occur at a unique point.
\item Linear functions are quasiconvex but not strictly so. The
  negative of a linear function is also quasiconvex. The maximum and
  minimum for linear functions on convex domains must occur at extreme
  points.
\item To find maxima/minima on the boundary, we can use the method of
  Lagrange multipliers.
\end{enumerate}

See also: tables, discussion for linear, quadratic, and homogeneous
functions (hard to summarize). Below is a copy of the table for the
multiplicatively separable case.

The setup here is that we have a function $F(x,y) := f(x)g(y)$ and a
point $(x_0,y_0)$ in the domain such $x_0$ is a critical point for $f$
and $y_0$ is a critical point for $g$. Visit the lecture notes for
more detailed context.

\begin{tabular}{|l|l|l|l|l|}
  \hline
  $f(x_0)$ sign & $g(y_0)$ sign & $f(x_0)$ (local max/min) & $g(y_0)$ (local max/min) & $F(x_0,y_0)$ (local max/min/saddle)\\
  \hline
  positive & positive & local max & local max & local max\\
  \hline
  positive & positive & local max & local min & saddle point \\
  \hline
  positive & positive & local min & local max & saddle point \\
  \hline
  positive & positive & local min & local min & local min \\
  \hline
  positive & negative & local max & local max & saddle point\\
  \hline
  positive & negative & local max & local min & local min\\
  \hline
  positive & negative & local min & local max & local max\\
  \hline
  positive & negative & local min & local min & saddle point\\
  \hline
  negative & positive & local max & local max & saddle point\\
  \hline
  negative & positive & local max & local min & local max \\
  \hline
  negative & positive & local min & local max & local min \\
  \hline
  negative & positive & local min & local min & saddle point \\
  \hline
  negative & negative & local max & local max & local min \\
  \hline
  negative & negative & local max & local min & saddle point \\
  \hline
  negative & negative & local min & local max & saddle point \\
  \hline
  negative & negative & local min & local min & local max \\
  \hline
\end{tabular}


\end{document}