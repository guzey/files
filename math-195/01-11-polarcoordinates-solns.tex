\documentclass[10pt]{amsart}

%Packages in use
\usepackage{fullpage, hyperref, vipul, enumerate}

%Title details
\title{Class quiz solutions: Friday January 11: Polar coordinates}
\author{Math 195, Section 59 (Vipul Naik)}
%List of new commands

\begin{document}
\maketitle

\section{Performance review}

$27$ people took this $3$-question quiz. The score distribution was as
follows:

\begin{itemize}
\item Score of $1$: $1$ person
\item Score of $2$: $8$ people
\item Score of $3$: $18$ people
\end{itemize}

The answers and performance review for individual questions are:

\begin{enumerate}
\item Option (B): $26$ people.
\item Option (D): $22$ people.
\item Option (A): $23$ people.
\end{enumerate}

\section{Solutions}

\begin{enumerate}

\item Consider a straight line that does not pass through the pole in
  a polar coordinate system. The equation of such a line in the polar
  coordinate system can be expressed as $r = F(\theta)$. What kind of
  function is $F$?

  \begin{enumerate}[(A)]

  \item $F(\theta)$ is a linear combination of $\sin \theta$ and $\cos \theta$
  \item $F(\theta)$ is the reciprocal of a linear combination of $\sin
    \theta$ and $\cos \theta$.
  \item $F(\theta)$ is a linear combination of $\tan \theta$ and $\cot
    \theta$.
  \item $F(\theta)$ is the reciprocal of a linear combination of $\tan
    \theta$ and $\cot \theta$.
  \item $F(\theta)$ is a linear combination of $\sec \theta$ and $\csc
    \theta$.
  \end{enumerate}

  {\em Answer}: Option (B)
  
  {\em Explanation}: Consider the corresponding Cartesian coordinate
  system. The Cartesian equation of a straight line is of the form $ax
  + by = c$. Since the line does not pass through the origin/pole, $c
  \ne 0$. Set $x = r \cos \theta$ and $y = r \sin \theta$, and we get
  $r(a \cos \theta + b \sin \theta) = c$. Rearranging, we obtain that
  $r = c/(a \cos \theta + b \sin \theta)$. We could rewrite as $r =
  1/((a/c)\cos \theta + (b/c)\sin \theta)$.

  {\em Performance review}: $26$ out of $27$ got this. $1$ chose (C).

  {\em Historical note (last time)}: $8$ out of $21$ people got this
  correct. $6$ chose (C), $5$ chose (A), $1$ each chose (D) and (E).
\item Consider the curve $r = \sin^2\theta$. Which of the following
  symmetries does the curve enjoy? Please see Options (D) and (E)
  before answering.

  \begin{enumerate}[(A)]
  \item Mirror symmetry about the polar axis
  \item Mirror symmetry about an axis perpendicular to the polar axis
    (what would be the $y$-axis if the polar axis is the $x$-axis)
  \item Half turn symmetry about the pole
  \item All of the above
  \item None of the above
  \end{enumerate}

  {\em Answer}: Option (D)

  {\em Explanation}: We use the fact that $\sin^2\theta =
  \sin^2(-\theta)$ to deduce mirror symmetry about the polar axis. We
  use that $\sin^2\theta = \sin^2(\pi - \theta)$ to deduce mirror
  symmetry about the $y$-axis. Finally, we use that $\sin^2\theta =
  \sin^2(\pi + \theta)$ to deduce half turn symmetry about the pole.

  {\em Performance review}: $22$ out of $27$ got this. $5$ chose (B).

  {\em Historical note (last time)}: $10$ out of $21$ people got this
  correct. $6$ chose (B), $5$ chose (A).
\item Which of the following is the correct expression for the length
  of the part of the curve $r = F(\theta)$ from $\theta = \alpha$ to
  $\theta = \beta$, with $\alpha < \beta$?

  \begin{enumerate}[(A)]
  \item $\int_\alpha^\beta \sqrt{(F(\theta))^2 + (F'(\theta))^2} \, d\theta$
  \item $\int_\alpha^\beta |F(\theta) + F'(\theta)| \, d\theta$
  \item $\int_\alpha^\beta |F(\theta) - F'(\theta)| \, d\theta$
  \item $\int_\alpha^\beta \sqrt{(F(\theta))^2 + (F'(\theta))^2 + 4F(\theta)F'(\theta)} \, d\theta$
  \item $\int_\alpha^\beta \sqrt{(F(\theta))^2 + (F'(\theta))^2 - 4F(\theta)F'(\theta)} \, d\theta$
  \end{enumerate}

  {\em Answer}: Option (A)

  {\em Explanation}: There are many ways of seeing this, including a
  direct justification in polar coordinates, but we provide an easy
  explanation using the Cartesian coordinates. We know that:

  $$x = F(\theta) \cos \theta, \qquad y= F(\theta)\sin \theta$$

  We get:

  $$\frac{dx}{d\theta} = F'(\theta)\cos \theta - F(\theta)\sin \theta, \qquad \frac{dy}{d\theta} = F'(\theta)\sin \theta + F(\theta)\cos \theta$$

  Squaring and adding, we know that the $2F(\theta)F'(\theta)\cos
  \theta \sin \theta$ term cancels between the two expressions, and we
  are left with:

  $$\left(\frac{dx}{d\theta}\right)^2 + \left(\frac{dy}{d\theta}\right)^2 = (F'(\theta))^2 \cos^2\theta + (F(\theta))^2\sin^2\theta + (F'(\theta))^2\sin^2\theta + (F(\theta))^2\cos^2\theta = [(F'(\theta))^2 + (F(\theta))^2](\cos^2\theta + \sin^2\theta)$$

  Using $\cos^2 \theta + \sin^2\theta = 1$, we get the desired result.

  {\em Performance review}: $23$ out of $27$ got this. $2$ chose (D),
  $1$ each chose (B) and (E).

  {\em Historical note (last time)}: $14$ out of $21$ people got this
  correct. $4$ chose (D), $3$ chose (E).
\end{enumerate}
\end{document}