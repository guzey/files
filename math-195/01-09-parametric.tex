\documentclass[10pt]{amsart}

%Packages in use
\usepackage{fullpage, hyperref, vipul, enumerate}

%Title details
\title{Take-home class quiz: due Wednesday January 9: Parametric stuff}
\author{Math 195, Section 59 (Vipul Naik)}
%List of new commands

\begin{document}
\maketitle

Your name (print clearly in capital letters): $\underline{\qquad\qquad\qquad\qquad\qquad\qquad\qquad\qquad\qquad\qquad}$

\vspace{0.1in}

{\bf THIS IS A TAKE-HOME CLASS QUIZ, BUT I WILL GIVE YOU ABOUT 5
  MINUTES TO REVIEW YOUR ANSWERS IN CLASS AND DISCUSS WITH OTHER
  STUDENTS.}

\vspace{0.1in}

{\bf YOU ARE ALLOWED TO DISCUSS ONLY QUESTIONS THAT BEGIN WITH A (*)
  OR (**). PLEASE ATTEMPT ALL OTHER QUESTIONS BY YOURSELF. EVEN FOR THE QUESTIONS YOU DISCUSS, PLEASE FINALLY ENTER ONLY THE ANSWER OPTION YOU ARE PERSONALLY MOST CONVINCED ABOUT -- DON'T ENGAGE IN GROUPTHINK.}

\begin{enumerate}
\item Consider the curve given by the parametric description $x = \cos
  t$, $y = \sin t$, where $t$ varies over the interval $[a,b]$ with $a
  < b$. What is a necessary and sufficient condition on $a$ and $b$
  for this curve to be the circle $x^2 + y^2 = 1$? {\em Last time:
    $11/24$ correct}

  \begin{enumerate}[(A)]
  \item $b - a =\pi$
  \item $b - a > \pi$
  \item $b - a = 2\pi$
  \item $b - a > 2\pi$
  \item $b - a \ge 2\pi$
  \end{enumerate}

  \vspace{0.1in}
  Your answer: $\underline{\qquad\qquad\qquad\qquad\qquad\qquad\qquad}$
  \vspace{0.6in}

\item (**) Consider the curve given by the parametric description $x =
  \arctan t$ and $y = \arctan t$ for $t \in \R$. Which of the
  following is the best description of this curve? {\em Last time:
    $8/24$ correct}

  \begin{enumerate}[(A)]
  \item It is the graph of the function $\arctan$
  \item It is the line $y = x$
  \item It is a line segment (without endpoints) that is part of the
    line $y = x$
  \item It is a half-line (with endpoint) that is part of the line $y
    = x$
  \item It is a disjoint union of two half-lines that are both part of
    the line $y = x$
  \end{enumerate}

  \vspace{0.1in}
  Your answer: $\underline{\qquad\qquad\qquad\qquad\qquad\qquad\qquad}$
  \vspace{0.6in}

\item (**) Consider the curve given by the parametric description $x =
  \sin^2t$ and $y = \cos^2t$ for $t \in \R$. Which of the following is
  the best description of this curve? {\em Last time: $5/24$ correct}

  \begin{enumerate}[(A)]
  \item It is the arc of the circle $x^2 + y^2 = 1$ comprising the
    first quadrant, i.e., when $x \ge 0$ and $y \ge 0$.
  \item It is the entire circle $x^2 + y^2 = 1$
  \item It is the line segment joining the points $(0,1)$ and $(1,0)$
  \item It is the line $y = 1 - x$
  \item It is a portion of the parabola $y = x^2$
  \end{enumerate}

  \vspace{0.1in}
  Your answer: $\underline{\qquad\qquad\qquad\qquad\qquad\qquad\qquad}$
  \vspace{0.6in}

\item Identify the parametric description which {\em does not}
  correspond to the set of points $(x,y)$ satisfying $x^3 = y^5$. {\em
    Last time: $16/24$ correct}

  \begin{enumerate}[(A)]
  \item $x = t^3$, $y = t^5$, for $t \in \R$
  \item $x = t^5$, $y = t^3$, for $t \in \R$
  \item $x = t$, $y = t^{3/5}$, for $t \in \R$
  \item $x = t^{5/3}$, $y = t$, for $t \in \R$
  \item All of the above parametric descriptions work
  \end{enumerate}

  \vspace{0.1in}
  Your answer: $\underline{\qquad\qquad\qquad\qquad\qquad\qquad\qquad}$
  \vspace{0.6in}

\item (**) Consider the parametric description $x = f(t)$, $y = g(t)$
  where $t$ varies over all of $\R$. What is the necessary and
  sufficient condition for the curve given by this to be the graph of
  a function, i.e., to satisfy the vertical line test? {\em Last time:
    $10/24$ correct}

  \begin{enumerate}[(A)]
  \item For any $t_1$ and $t_2$ satisfying $f(t_1) = f(t_2)$, we must
    have $g(t_1) = g(t_2)$.
  \item For any $t_1$ and $t_2$ satisfying $g(t_1) = g(t_2)$, we must
    have $f(t_1) = f(t_2)$.
  \item Both $f$ and $g$ are one-to-one functions.
  \item For any $t_1$ and $t_2$, we must have $f(t_1) = f(t_2)$.
  \item For any $t_1$ and $t_2$, we must have $g(t_1) = g(t_2)$.
  \end{enumerate}

  \vspace{0.1in}
  Your answer: $\underline{\qquad\qquad\qquad\qquad\qquad\qquad\qquad}$
  \vspace{0.6in}

\item Suppose $f$ and $g$ are both twice differentiable functions
  everywhere on $\R$. Which of the following is the correct formula
  for $(f \circ g)''$? {\em Last time: $20/21$ correct}

  \begin{enumerate}[(A)]

  \item $(f'' \circ g) \cdot g''$
  \item $(f'' \circ g) \cdot (f' \circ g') \cdot g''$
  \item $(f'' \circ g) \cdot (f' \circ g') \cdot (f \circ g'')$
  \item $(f'' \circ g) \cdot (g')^2 + (f' \circ g) \cdot g''$
  \item $(f' \circ g') \cdot (f \circ g) + (f'' \circ g'')$
  \end{enumerate}

  \vspace{0.1in}
  Your answer: $\underline{\qquad\qquad\qquad\qquad\qquad\qquad\qquad}$
  \vspace{0.6in}

\item Suppose $x = f(t)$ and $y = g(t)$ where $f$ and $g$ are both
  twice differentiable functions. What is $d^2y/dx^2$ in terms of $f$
  and $g$ and their derivatives evaluated at $t$? {\em Last time:
    $20/21$ correct}

  \begin{enumerate}[(A)]
  \item $(f'(t)g''(t) - g'(t)f''(t))/(f'(t))^3$
  \item $(f'(t)g''(t) - g'(t)f''(t))/(g'(t))^3$
  \item $(g'(t)f''(t) - f'(t)g''(t))/(f'(t))^3$
  \item $(g'(t)f''(t) - f'(t)g''(t))/(g'(t))^3$
  \item None of the above
  \end{enumerate}

  \vspace{0.1in}
  Your answer: $\underline{\qquad\qquad\qquad\qquad\qquad\qquad\qquad}$
  \vspace{0.6in}

\item Which of the following pair of bounds works for the arc length
  for the portion of the graph of the sine function between $(a,\sin
  a)$ and $(b, \sin b)$ where $a < b$? {\em Last time: $15/21$
    correct}

  \begin{enumerate}[(A)]
  \item Between $(b - a)/\sqrt{3}$ and $(b - a)/\sqrt{2}$
  \item Between $(b - a)/\sqrt{2}$ and $b - a$
  \item Between $(b - a)$ and $\sqrt{2}(b - a)$
  \item Between $\sqrt{2}(b - a)$ and $\sqrt{3}(b - a)$
  \item Between $\sqrt{3}(b - a)$ and $2(b - a)$
  \end{enumerate}

  \vspace{0.1in}
  Your answer: $\underline{\qquad\qquad\qquad\qquad\qquad\qquad\qquad}$
  \vspace{0.6in}

\item (*) Consider the parametric curve $x = e^t$, $y = e^{t^2}$. How
  does $y$ grow in terms of $x$ as $x \to \infty$? {\em Last time:
    $7/21$ correct}

  \begin{enumerate}[(A)]
  \item $y$ grows like a polynomial in $x$.
  \item $y$ grows faster than any polynomial in $x$ but slower than an
    exponential function of $x$.
  \item $y$ grows exponentially in $x$.
  \item $y$ grows super-exponentially in $x$ but slower than a double
    exponential in $x$.
  \item $y$ grows like a double exponential in $x$.
  \end{enumerate}

  \vspace{0.1in}
  Your answer: $\underline{\qquad\qquad\qquad\qquad\qquad\qquad\qquad}$
  \vspace{0.6in}

\item We say that a curve is {\em algebraic} if it admits a
  parameterization of the form $x = f(t)$, $y = g(t)$, where $f$ and
  $g$ are rational functions and $t$ varies over some subset of the
  real numbers. Which of the following curves is {\em not} algebraic?
  {\em Last time: $11/21$ correct}

  \begin{enumerate}[(A)]
  \item $x = \cos t$, $y = \sin t$, $t \in \R$
  \item $x = \cos t$, $y = \cos(3t)$, $t \in \R$
  \item $x = \cos t$, $y = \cos^2t$, $t \in \R$
  \item $x = \cos t$, $y = \cos(t^2)$, $t \in \R$
  \item None of the above, i.e., they are all algebraic
  \end{enumerate}

  \vspace{0.1in}
  Your answer: $\underline{\qquad\qquad\qquad\qquad\qquad\qquad\qquad}$
  \vspace{0.6in}

\item (**) Suppose $x = f(t)$, $y = g(t)$, $t \in \R$ is a parametric
  description of a curve $\Gamma$ and both $f$ and $g$ are continuous
  on all of $\R$. If both $f$ and $g$ are even, what can we conclude
  about $\Gamma$ and its parameterization? {\em Last time: $5/21$
    correct}

  \begin{enumerate}[(A)]
  \item $\Gamma$ is symmetric about the $y$-axis
  \item $\Gamma$ is symmetric about the $x$-axis
  \item $\Gamma$ is symmetric about the line $y = x$
  \item $\Gamma$ has half turn symmetry about the origin
  \item The parameterizations of $\Gamma$ for $t \le 0$ and for $t \ge
    0$ both cover all of $\Gamma$, and in directions mutually reverse
    to each other.
  \end{enumerate}

  \vspace{0.1in}
  Your answer: $\underline{\qquad\qquad\qquad\qquad\qquad\qquad\qquad}$
  \vspace{0.6in}

\end{enumerate}
\end{document}