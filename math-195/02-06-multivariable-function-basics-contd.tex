\documentclass[10pt]{amsart}

%Packages in use
\usepackage{fullpage, hyperref, vipul, enumerate}

%Title details
\title{Take-home class quiz: due Wednesday February 6: Multivariable function basics continued}
\author{Math 195, Section 59 (Vipul Naik)}
%List of new commands

\begin{document}
\maketitle

Your name (print clearly in capital letters): $\underline{\qquad\qquad\qquad\qquad\qquad\qquad\qquad\qquad\qquad\qquad}$

{\bf PLEASE FEEL FREE TO DISCUSS ALL QUESTIONS, BUT PLEASE ONLY ENTER
FINAL ANSWER OPTIONS THAT YOU PERSONALLY CONSIDER MOST LIKELY TO BE
CORRECT. DO NOT ENGAGE IN GROUPTHINK.}

\begin{enumerate}
\item Suppose $F$ is an additively separable function of two variables
  $x$ and $y$ that is defined everywhere, i.e., there exist functions
  $f$ and $g$ of one variable, both defined on all of $\R$, such that
  $F(x,y) = f(x) + g(y)$ for all $x,y \in \R$.
 
  We call two curves {\em parallel} if there is a vector by which we
  can translate all the points in one curve to get precisely the other
  curve.

  Consider the following three statements:

  (i) All curves obtained as the intersections of the graph of $F$
  with planes parallel to the $xy$-plane are parallel to each other.

  (ii) All curves obtained as the intersections of the graph of $F$
  with planes parallel to the $xz$-plane are parallel to each other.

  (iii) All curves obtained as the intersections of the graph of $F$
  with planes parallel to the $yz$-plane are parallel to each other.

  Which of the statements (i)-(iii) is/are necessarily true?

  \begin{enumerate}[(A)]
  \item All of (i), (ii), and (iii) are true.
  \item Both (i) and (ii) are true but (iii) need not be true.
  \item Both (ii) and (iii) are true but (i) need not be true.
  \item Both (i) and (iii) are true but (ii) need not be true.
  \item (i) is true but (ii) and (iii) need not be true.
  \end{enumerate}

  \vspace{0.1in}
  Your answer: $\underline{\qquad\qquad\qquad\qquad\qquad\qquad\qquad}$
  \vspace{0.1in}

\item Suppose $f$ is a continuous function of two variables $x$ and
  $y$, defined on the entire $xy$-plane. Suppose further that $f$ is
  increasing in $x$ for each fixed value of $y$, and that $f$ is
  increasing in $y$ for every fixed value of $x$. Which of the
  following is the most plausible description of the level curves of
  $f$ in the $xy$-plane? {\em Note: You might wish to take an
  extremely simple example, e.g., an additively separable function
  where each of the pieces is the simplest possible increasing
  function you can think of}.

  \begin{enumerate}[(A)]
  \item They are all upward-sloping, i.e., they are of the form $y =
    g(x)$ with $g$ an increasing function.
    \item They are all downward-sloping, i.e., they are of the form $y
      = g(x)$ with $g$ a decreasing function.
    \item They look like closed loops (e.g., circles).
    \item They look like graphs of functions with a unique local and
      absolute minimum (such as the parabola $y = x^2$, though the
      actual picture may be different).
    \item They look like graphs of functions with a unique local and
      absolute maximum (such as the parabola $y = -x^2$, though the
      actual function may be different).
  \end{enumerate}

  \vspace{0.1in}
  Your answer: $\underline{\qquad\qquad\qquad\qquad\qquad\qquad\qquad}$
  \vspace{0.1in}

\item What do the level curves of the function $f(x,y) := \sin(x + y)$
  look like for output value in $[-1,1]$? Note that all these level
  curves are being considered as curves in the $xy$-plane. {\em Note:
  This builds upon the idea of Question 3 of the previous quiz.}

  \begin{enumerate}[(A)]
  \item Each level curve is a single line.
  \item Each level curve is a union of two intersecting lines.
  \item Each level curve is a union of two distinct parallel lines.
  \item Each level curve is a union of infinitely many concurrent
    lines (i.e., infinitely many lines, all passing through the same
    point).
  \item Each level curve is a union of infinitely many distinct
    parallel lines (i.e., infinitely many lines, all parallel to each
    other).
  \end{enumerate}

  \vspace{0.1in}
  Your answer: $\underline{\qquad\qquad\qquad\qquad\qquad\qquad\qquad}$
  \vspace{0.1in}

\item Suppose $f$ and $g$ are both continuous functions of two
  variables $x$ and $y$, both defined on all of $\R^2$, and such that
  $f(x,y) + g(x,y)$ is a constant $C$. What is the relation between
  the level curves of $f$ and the level curves of $g$, all drawn in
  the $xy$-plane?

  \begin{enumerate}[(A)]
  \item Every level curve of $f$ is a level curve of $g$ and vice
    versa, with the same level value for both functions.
  \item Every level curve of $f$ is a level curve of $g$ and vice
    versa, but the value for which it is a level curve may be different
    for the two functions.
  \item The level curves of $f$ need not be precisely the same as the
    level curves of $g$, but we can go from one set of level curves to
    the other via a parallel translation.
  \item Each level curve of $f$ can be obtained by reflecting a
    suitable level curve of $g$ about a suitable line in the
    $xy$-plane.
  \item Each level curve of $f$ can be obtained by reflecting a
    suitable level curves of $g$ about a suitable line in the
    $xy$-plane and then performing a suitable translation.
  \end{enumerate}

  \vspace{0.1in}
  Your answer: $\underline{\qquad\qquad\qquad\qquad\qquad\qquad\qquad}$
  \vspace{0.1in}

\end{enumerate}

\end{document}