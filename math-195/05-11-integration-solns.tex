\documentclass[10pt]{amsart}

%Packages in use
\usepackage{fullpage, hyperref, vipul, enumerate}

%Title details
\title{Class quiz solutions: May 11: Integration techniques (one variable)}
\author{Math 195, Section 59 (Vipul Naik)}
%List of new commands

\begin{document}
\maketitle

\section{Performance review}

$18$ people took this quiz. The score distribution is as follows:

\begin{itemize}
\item Score of $1$: $1$ person
\item Score of $2$: $2$ people
\item Score of $3$: $2$ people
\item Score of $4$: $4$ people
\item Score of $5$: $5$ people
\item Score of $6$: $3$ people
\item Score of $9$: $1$ person
\end{itemize}

The mean score was about $4.4$. The question wise answers and
performance are as follows:

\begin{enumerate}
\item Option (B): $5$ people
\item Option (C): $8$ people
\item Option (C): $8$ people
\item Option (D): $5$ people
\item Option (A): $6$ people
\item Option (E): $3$ people
\item Option (D): $9$ people
\item Option (B): $6$ people
\item Option (D): $13$ people
\item Option (B): $8$ people
\item Option (E): $3$ people
\item Option (D): $5$ people
\end{enumerate}

\section{Solutions}

In the questions below, we say that a function is {\em expressible in
terms of elementary functions} or {\em elementarily expressible} if it
can be expressed in terms of polynomial functions, rational functions,
radicals, exponents, logarithms, trigonometric functions and inverse
trigonometric functions using pointwise combinations, compositions,
and piecewise definitions. We say that a function is {\em elementarily
integrable} if it has an elementarily expressible antiderivative.

Note that if a function is elementarily expressible, so is its
derivative on the domain of definition.

We say that a function $f$ is $k$ times elementarily integrable if
there is an elementarily expressible function $g$ such that $f$ is the
$k^{th}$ derivative of $g$.

We say that the integrals of two functions are {\em equivalent up to
elementary functions} if an antiderivative for one function can be
expressed using an antiderivative for the other function and
elementary function, again piecing them together using pointwise
combination, composition, and piecewise definitions.

\begin{enumerate}

\item Suppose $f$ is a continuous function on all of $\R$ and is the
  third derivative of an elementarily expressible function, but is not
  the fourth derivative of any elementarily expressible function. In
  other words, $f$ can be integrated three times but not four times
  within the collection of elementarily expressible functions. What is
  the {\bf largest positive integer} $k$ such that $x \mapsto x^kf(x)$
  is {\em guaranteed to be} {\bf elementarily integrable}?

  \begin{enumerate}[(A)]
  \item $1$
  \item $2$
  \item $3$
  \item $4$
  \item $5$
  \end{enumerate}

  {\em Answer}: Option (B)

  {\em Explanation}: Via integration by parts, integrating $f$ $m$ times is
  equivalent to finding antiderivatives for $f(x)$, $xf(x)$, and so on
  till $x^{m-1}f(x)$. In our case, $f$ can be integrated $3$ times, so
  the largest $k$ is $3 - 1 = 2$.

  {\em Performance review}: $5$ out of $18$ people got this
  correct. $6$ chose (D), $5$ chose (C), and $2$ chose (E).

  {\em Historical note}: This question appeared in a 153 quiz. At the
  time, $23$ out of $27$ people got this correct. $2$ people each chose
  (C) and (D).

\item Suppose $f$ is a continuous function on $(0,\infty)$ and is the
  third derivative of an elementarily expressible function, but is not
  the fourth derivative of any elementarily expressible function. In
  other words, $f$ can be integrated three times but not four times
  within the collection of elementarily expressible functions. What is
  the {\bf largest positive integer} $k$ such that the function $x
  \mapsto f(x^{1/k})$ with domain $(0,\infty)$ is {\em guaranteed to
  be} {\bf elementarily integrable}?

  \begin{enumerate}[(A)]
  \item $1$
  \item $2$
  \item $3$
  \item $4$
  \item $5$
  \end{enumerate}

  {\em Answer}: Option (C)

  {\em Explanation}: Via the $u$-substitution $u = x^{1/k}$, we get
  $\int ku^{k-1}f(u) \, du$. Now using the previous question, the
  maximum value of $k - 1$ possible is $2$, so the maximum possible
  value is $3$.

  We can also do a direct integration by parts taking $1$ as the
  second part.

  {\em Performance review}: $8$ out of $18$ people got this
  correct. $5$ chose (A), $3$ chose (D), and $2$ chose (B).

  {\em Historical note}: This question appeared in a 153 quiz. At the
  time, $14$ out of $27$ people got this correct. $4$ people chose (D),
  $4$ people chose (B), $3$ people chose (A), $1$ person chose (E),
  and $1$ person left the question blank.
  
\item Of these five functions, four of the functions are elementarily
  integrable and can be integrated using integration by parts. The
  other one function is {\bf not elementarily integrable}. Identify
  this function.

  \begin{enumerate}[(A)]
  \item $x \mapsto x \sin x$
  \item $x \mapsto x \cos x$
  \item $x \mapsto x \tan x$
  \item $x \mapsto x \sin^2x$
  \item $x \mapsto x \tan^2x$
  \end{enumerate}
  {\em Answer}: Option (C)

  {\em Explanation}: If $f$ is elementarily integrable, then $xf(x)$
  is elementarily integrable iff $f$ is twice elementarily integrable;
  this is easily seen using integration by parts. Of the function
  options given here, $\tan$ is the only function that is not twice
  elementarily integrable, because the first integration gives
  $-\ln|\cos x|$ which cannot be integrated. Of the others, note that
  $\sin$, $\cos$, and $\sin^2$ can be integrated using elementary
  functions infinitely many times. $\tan^2$ is twice elementarily
  integrable but no further: integrates the first time to $\tan x -
  x$, which integrates one more time to $-\ln|cos x| - x^2/2$, which
  cannot be integrated further.

  {\em Performance review}: $8$ out of $18$ people got this
  correct. $8$ people chose (E) and $2$ people chose (D).

  {\em Historical note}: This question appeared in a 153 quiz. At
  the time, $22$ out of $27$ people got this correct. $4$ people chose
  (E), and $1$ person chose (D).

\item Consider the four functions $f_1(x) = \sqrt{\sin x}$, $f_2(x) =
  \sin \sqrt{x}$, $f_3(x) = \sin^2 x$ and $f_4(x) = \sin(x^2)$, all
  viewed as functions on the interval $[0,1]$ (so they are all well
  defined). Two of these functions are elementarily integrable; the
  other two are not. Which are {\bf the two elementarily integrable
  functions}?

  \begin{enumerate}[(A)]
  \item $f_3$ and $f_4$.
  \item $f_1$ and $f_3$.
  \item $f_1$ and $f_4$. 
  \item $f_2$ and $f_3$.
  \item $f_2$ and $f_4$.
  \end{enumerate}

  {\em Answer}: Option (D)

  {\em Explanation}: Integration of $f_3$ is a standard procedure, so
  we say nothing about that. As for $f_2$, recall that integrating
  $f(x^{1/k})$ is equivalent to integrating $u^{k-1}f(u)$ where $u =
  x^{1/k}$, which in turn is equivalent to integrating $f$ $k$
  times. Since $\sin$ can be integrated as many times as we wish,
  $f_2$ can be integrated.

  The reason why $f_1$ and $f_4$ are not elementarily integrable is
  subtler but it's clear that none of the obvious methods work.

  {\em Performance review}: $5$ out of $18$ people got this
  correct. $5$ people each chose (B) and (E), $2$ chose (C), $1$ chose
  (A).

  {\em Historical note}: This question appeared in a 153 quiz. At the
  time, $17$ out of $27$ people got this correct. $5$ people chose (B),
  $3$ people chose (A), and $2$ people chose (C).

\item Which of the following functions has an antiderivative that is
  {\bf not equivalent} up to elementary functions to the antiderivative of
  $x \mapsto e^{-x^2}$?

  \begin{enumerate}[(A)]
  \item $x \mapsto e^{-x^4}$
  \item $x \mapsto e^{-x^{2/3}}$
  \item $x \mapsto e^{-x^{2/5}}$
  \item $x \mapsto x^2e^{-x^2}$
  \item $x \mapsto x^4e^{-x^2}$
  \end{enumerate}

  {\em Answer}: Option (A)

  {\em Explanation}: We show the equivalence with the others.

  Option (D): We use integration by parts, writing
  $x^2e^{-x^2}$ as $x \cdot (xe^{-x^2})$ and taking $xe^{-x^2}$ as the
  part to integrate, so that $x$ is the part to differentiate. An
  antiderivative for $xe^{-x^2}$ is $(-1/2)e^{-x^2}$, so we get:

  $$\frac{-x}{2}e^{-x^2} - \int \frac{-1}{2}e^{-x^2} \, dx$$

  We thus see that it reduces to $\int e^{-x^2} \, dx$.

  Option (E), via reduction to option (D): We use integration by
  parts, taking $x^3$ as the part to differentiate and $xe^{-x^2}$ as
  the part to integrate. One application of integration by parts
  reduces this to $\int x^2e^{-x^2}$, which is option (D).

  Option (B), via reduction to option (D): Start with $\int
  e^{-x^{2/3}} \, dx$. Put $u = x^{1/3}$. The substitution gives (up to
  scalars) $\int u^2e^{-u^2} \, du$, which is option (D).

  Option (C), via reduction to option (D): Start with $\int
  e^{-x^{2/5}} \, dx$. Put $u = x^{1/5}$. The substitution gives (up to
  scalars) $\int u^4e^{-u^2} \, du$, which is option (E).

  {\em Performance review}: $6$ out of $18$ people got this
  correct. $6$ chose (C), $2$ each chose (B), (D), (E).

  {\em Historical note}: This question appeared in a 153 quiz. At the
  time, $10$ out of $27$ people got this correct. $7$ people chose (D),
  $4$ people chose (C), $4$ people chose (E), and $2$ people chose
  (B).
  
\item Consider the statements $P$ and $Q$, where $P$ states that every
  rational function is elementarily integrable, and $Q$ states that
  any rational function is $k$ times elementarily integrable for all
  positive integers $k$.

  Which of the following additional observations is {\bf correct} and
  {\bf allows us to deduce} $Q$ given $P$?

  \begin{enumerate}[(A)]
  \item There is no way of deducing $Q$ from $P$ because $P$ is true
    and $Q$ is false.
  \item The antiderivative of a rational function can always be chosen
    to be a rational function, hence $Q$ follows from a repeated
    application of $P$.
  \item Using integration by parts, we see that repeated integration
    of a function $f$ is equivalent to integrating $f$, $f^2$, $f^3$,
    and higher powers of $f$ (the powers here are pointwise products,
    not compositions). If $f$ is a rational function, each of these is
    also a rational function. Applying $P$, each of these is
    elementarily integrable, hence $f$ is $k$ times elementarily
    integrable for all $k$.
  \item Using integration by parts, we see that repeated integration
    of a function $f$ is equivalent to integrating $f$, $f'$, $f''$,
    and higher derivatives of $f$. If $f$ is a rational function, each
    of these is also a rational function. Applying $P$, each of these
    is elementarily integrable, hence $f$ is $k$ times elementarily
    integrable for all $k$.
  \item Using integration by parts, we see that repeated integration
    of a function $f$ is equivalent to integrating each of the
    functions $f(x)$, $xf(x)$, $\dots$. If $f$ is a rational function,
    each of these is also a rational function. Applying $P$, each of
    these is elementarily integrable, hence $f$ is $k$ times
    elementarily integrable for all $k$.
  \end{enumerate}

  {\em Answer}: Option (E)

  {\em Explanation}: Missing!

  {\em Performance review}: $3$ out of $18$ people got this
  correct. $7$ chose (C), $6$ chose (D), $1$ each chose (A) and (B).

  {\em Historical note}: This question appeared
  in a 153 quiz. At the time, $18$ out of $27$ people got this
  correct. $4$ people chose (D), $2$ people chose (C), $2$ people
  chose (B), and $1$ person chose (A).

\item Which of these functions of $x$ is {\em not} elementarily
  integrable?

  \begin{enumerate}[(A)]
  \item $x\sqrt{1 + x^2}$
  \item $x^2\sqrt{1 + x^2}$
  \item $x(1 + x^2)^{1/3}$
  \item $x\sqrt{1 + x^3}$
  \item $x^2\sqrt{1 + x^3}$
  \end{enumerate}

  {\em Answer}: Option (D)

  {|em Explanation}: For options (A) and (C), the substitution $u = 1
  + x^2$ works fine. For option (E), the substitution $u = 1 + x^3$
  works fine. For option (B), we can solve the problem using a
  trigonometric substitution. This leaves option (D) (which,
  incidentally, requires the use of elliptic integrals).

  {\em Performance review}: $9$ out of $18$ people got this
  correct. $4$ chose (C), $4$ chose (B), $1$ chose (A).

  {\em Historical note}: This question appeared
  in a 153 quiz. At the time, $22$ out of $27$ people got this
  correct. $4$ people chose (B) and $1$ person chose (C).

\item Consider the function $f(k) := \int_1^2 \frac{dx}{\sqrt{x^2 +
  k}}$. $f$ is defined for $k \in (-1,\infty)$. What can we say about
  the nature of $f$ within this interval?

  \begin{enumerate}[(A)]
  \item $f$ is increasing on the interval $(-1,\infty)$.
  \item $f$ is decreasing on the interval $(-1,\infty)$.
  \item $f$ is increasing on $(-1,0)$ and decreasing on $(0,\infty)$.
  \item $f$ is decreasing on $(-1,0)$ and increasing on $(0,\infty)$.
  \item $f$ is increasing on $(-1,0)$, decreasing on $(0,2)$, and
    increasing again on $(2,\infty)$.
  \end{enumerate}

  {\em Answer}: Option (B)

  {\em Explanation}: For any fixed value of $x \in [1,2]$, the
  integrand $1/\sqrt{x^2 + k}$ is a {\em decreasing} function of $k$
  for $k \in (-1,\infty)$. Hence, the value we get upon integrating it
  for $x \in [1,2]$ should also be a decreasing function of $k$.

  {\em Performance review}: $6$ out of $18$ people got this
  correct. $3$ chose (A), $5$ chose (C), $2$ chose (E), $1$ chose (D),
  $1$ left the question blank.

  {\em Historical note}: This question appeared in a 153 quiz. At the
  time, $4$ out of $27$ people got this correct. $11$ people chose
  (A), $7$ people chose (C), $3$ people chose (D), $1$ person chose
  (E), and $1$ person left the question blank.

  Mainly, people confused the roles of the dummy variable $x$ (which
  gets integrated away) and the variable $k$.

\item For which of these functions of $x$ does the antiderivative
  necessarily involve {\em both} $\arctan$ {\em and} $\ln$?

  \begin{enumerate}[(A)]
  \item $1/(x + 1)$
  \item $1/(x^2 + 1)$
  \item $x/(x^2 + 1)$
  \item $x/(x^3 + 1)$
  \item $x^2/(x^3 + 1)$
  \end{enumerate}

  {\em Answer}: Option (D)

  {\em Explanation}: Option (A) integrates to $\ln|x+1|$, option (B)
  integrates to $\arctan x$, option (C) integrates to $(1/2) \ln(x^2 +
  1)$, and option (E) integrates to $(1/3) \ln|x^3 + 1|$. For option
  (D), we need to use partial fractions with denominators $x + 1$ and
  $x^2 - x + 1$, and we end up getting nonzero coefficients on terms
  that integrate to $\ln$ and to $\arctan$.

  {\em Performance review}: $13$ out of $18$ people got this
  correct. $3$ chose (C), $1$ each chose (B) and (E).

  {\em Historical note}: This question appeared in a 153 quiz. At the
  time, $21$ out of $27$ people got this correct. $2$ people chose (E),
  and $1$ person each chose (A), (B), and (C). $1$ person left the
  question blank.

\item Suppose $F$ is a (not known) function defined on $\R \setminus
  \{ -1,0,1\}$, differentiable everywhere on its domain, such that
  $F'(x) = 1/(x^3 - x)$ everywhere on $\R \setminus \{-1,0,1\}$. For
  which of the following sets of points is it true that knowing the
  value of $F$ at these points {\bf uniquely} determines $F$?

  \begin{enumerate}[(A)]
  \item $\{ -\pi, -e, 1/e,1/\pi \}$
  \item $\{ -\pi/2, -\sqrt{3}/2, 11/17,\pi^2/6 \}$
  \item $\{ -\pi^3/7,-\pi^2/6,\sqrt{13},11/2 \}$
  \item Knowing $F$ at any of the above determines the value of $F$
    uniquely.
  \item None of the above works to uniquely determine the value of
    $F$.
  \end{enumerate}

  {\em Answer}: Option (B)

  {\em Explanation}: The domain of $F$ has four connected components:
  the open intervals $(-\infty,-1)$, $(-1,0)$, $(0,1)$, and
  $(1,\infty)$. We need to know the value of $F$ at one point in each
  of these intervals. By computing values, we see that the set of
  points in option (B) has the property that it contains one point in
  each of these intervals, and those in options (A) and (C) do not.

  {\em Performance review}: $8$ out of $18$ people got this
  correct. $5$ chose (D), $3$ chose (A), $2$ left the question blank.

  {\em Historical note}: This question appeared in a 153 quiz. At the
  time, $14$ out of $27$ people got this correct. $6$ people chose (D),
  $4$ people chose (C), and $3$ people chose (A).

  Many people spent time trying to determine the coefficients of the
  partial fraction decomposition. This is not relevant to what we need
  to do in the question.

\item Suppose $F$ is a continuously differentiable function whose
  domain contains $(a,\infty)$ for some $a \in \R$, and $F'(x)$ is a
  rational function $p(x)/q(x)$ on the domain of $F$. Further, suppose
  that $p$ and $q$ are nonzero polynomials. Denote by $d_p$ the degree
  of $p$ and by $d_q$ the degree of $q$. Which of the following is a
  {\bf necessary and sufficient condition} to ensure that $\lim_{x \to
  \infty} F(x)$ is finite?

  \begin{enumerate}[(A)]
  \item $d_p - d_q \ge 2$
  \item $d_p - d_q \ge 1$
  \item $d_p = d_q$
  \item $d_q - d_p \ge 1$
  \item $d_q - d_p \ge 2$
  \end{enumerate}

  {\em Answer}: Option (E)

  {\em Explanation}: This can be justified in terms of partial
  fractions. Th ecase where $q$ is a product of linear factors can be
  justified using the previous question. But that is not the most
  elegant justification. When we cover sequences and series, we will
  see some comparison tests that make it clear why this holds. The
  basic example you can keep in mind is that the antiderivative of
  $1/x^2$ is $-1/x$, which has a finite limit as $x \to \infty$.

  {\em Performance review}: $3$ out of $18$ people got this
  correct. $7$ chose (D), $5$ chose (B), $2$ chose (C), $1$ chose (A).

  Those who chose (D) had the right idea but failed to account for the
  extra margin that needs to be maintained because an integration is
  being performed.

  {\em Historical note}: This question appeared
  in a 153 quiz. At the time, $3$ out of $27$ people got this
  correct. $10$ people chose (D), $7$ people chose (C), $6$ people
  chose (B), and $1$ person chose (A).



  For the final question, build on the observation: For any
  nonconstant monic polynomial $q(x)$, there exists a finite
  collection of transcendental functions $f_1, f_2, \dots, f_r$ such
  that the antiderivative of any rational function $p(x)/q(x)$, on an
  open interval where it is defined and continuous, can be expressed
  as $g_0 + f_1g_1 + f_2g_2 + \dots + f_rg_r$ where $g_0, g_1, \dots,
  g_r$ are rational functions.

\item For the polynomial $q(x) = 1 + x^2$, what collection of $f_i$s
  works (all are written as functions of $x$)?

  \begin{enumerate}[(A)]
  \item $\arctan x$ and $\ln|x|$
  \item $\arctan x$ and $\arctan(1 + x^2)$
  \item $\ln|x|$ and $\ln(1 + x^2)$ 
  \item $\arctan x$ and $\ln(1 + x^2)$
  \item $\ln|x|$ and $\arctan(1 + x^2)$
  \end{enumerate}

  {\em Answer}: Option (D)

  {\em Explanation}: Follows from the standard partial fraction
  decomposition. $2x/(1 + x^2)$ gives the $\ln$ integration and $1/(1
  + x^2)$ gives the $\arctan$ integration.

  {\em Performance review}: $5$ out of $18$ people got this
  correct. $4$ each chose (C) and (E), $3$ chose (A), $2$ chose (B).

  {\em Historical note}: This question appeared in a 153 quiz. At the
  time, $15$ out of $27$ people got this correct. $7$ people chose
  (A), $2$ people chose (C), $1$ person each chose (A) and (E), and
  $1$ person left the question blank.

\end{enumerate}

\end{document}
