\documentclass[10pt]{amsart}

%Packages in use
\usepackage{fullpage, hyperref, vipul, enumerate}

%Title details
\title{Take-home class quiz solutions: due Wednesday January 16: Three
dimensions} 

\author{Math 195, Section 59 (Vipul Naik)}
%List of new commands

\begin{document}
\maketitle

\section{Performance review}

$25$ people took this quiz. The score distribution is as follows:

\begin{itemize}
\item Score of $2$: $2$ people
\item Score of $3$: $2$ people
\item Score of $4$: $2$ people
\item Score of $5$: $6$ people
\item Score of $6$: $11$ people
\item Score of $7$: $2$ people
\end{itemize}

The mean score was about $5.12$.

Here are the question-wise answers and performance summary (more
details in the next section):

\begin{enumerate}
\item Option (C): $16$ people
\item Option (B): $20$ people
\item Option (A): $19$ people
\item Option (C): $20$ people
\item Option (C): $24$ people
\item Option (A): $5$ people
\item Option (C): $24$ people
\end{enumerate}

\section{Solutions}

\begin{enumerate}

\item (*) Consider the subset of $\R^3$ given by the condition $(x^2 +
  y^2 - 1)(y^2 + z^2 - 1)(x^2 + z^2 - 1) = 0$. What kind of subset is
  this?

  \begin{enumerate}[(A)]
  \item It is a sphere centered at the origin and of radius $1$.
  \item It is the union of three circles, each centered at the origin
    and of radius $1$, and lying in the $xy$-plane, $yz$-plane, and
    $xz$-plane respectively.
  \item It is the union of three cylinders, each of radius $1$, about
    the $x$-axis, $y$-axis, and $z$-axis respectively.
  \item It is the intersection of three circles, each centered at the origin
    and of radius $1$, and lying in the $xy$-plane, $yz$-plane, and
    $xz$-plane respectively.
  \item It is the intersection of three cylinders, each of radius $1$, about
    the $x$-axis, $y$-axis, and $z$-axis respectively.
  \end{enumerate}

  {\em Answer}: Option (C)

  {\em Explanation}: Each of the individual conditions gives a
  cylinder about one of the coordinate axes of radius $1$. For
  instance, $x^2 + y^2 - 1 = 0$ gives the cylinder of radius $1$ about
  the $z$-axis. For the product to be zero, one or more of the
  conditions must be satisfied, so we get the union.

  {\em Performance review}: $16$ out of $25$ got this. $7$ chose (B),
  $2$ chose (D).

  {\em Historical note (last time)}: $12$ out of $24$ people got this
  correct. $3$ chose (B), $4$ chose (D), $3$ chose (E), $2$ chose (A).

\item Given two distinct points $A$ and $B$ in three-dimensional
  space, what is the nature of the set of possibilities for a third
  point $C$ such that $AC$ and $BC$ have equal length (i.e., $C$ is
  equidistant from $A$ and $B$)? {\em Didn't appear last time}

  \begin{enumerate}[(A)]
  \item Sphere
  \item Plane
  \item Circle
  \item Line
  \item Two points
  \end{enumerate}

  {\em Answer}: Option (B)

  {\em Explanation}: This is the plane perpendicular to the line
  segment $AB$ and intersecting the line segment at its midpoint. It
  is the analogue in three dimensions of the perpendicular bisector in
  two dimensions.

  {\em Performance review}: $20$ out of $25$ got this. $3$ chose (A),
  $2$ chose (C).
\item Given two distinct points $A$ and $B$ in three-dimensional
  space, what is the nature of the set of possibilities for a third
  point $C$ such that the triangle $ABC$ is a right triangle with $AB$
  as its hypotenuse?

  \begin{enumerate}[(A)]
  \item Sphere (minus two points)
  \item Plane
  \item Circle (minus two points)
  \item Line
  \item Square
  \end{enumerate}

  {\em Answer}: Option (A)

  {\em Explanation}: By some elementary geometry, we know that this is
  the sphere with diameter $AB$. However, the points $A$ and $B$
  themselves need to be excluded.

  {\em Note that if we were in a plane, we would get merely the circle
  with diameter $AB$ minus two points}. This seems to have been the
  most popular incorrect option chosen.

  {|em Performance review}: $19$ out of $25$ got this. $3$ chose (C),
  $2$ chose (E), $1$ chose (D).

  {\em Historical note (last time)}: $15$ out of $24$ people got this correct.
  $7$ chose (C), $1$ each chose (B) and (E).

\item Given two distinct points $A$ and $B$ in three-dimensional
  space, what is the nature of the set of possibilities for a third
  point $C$ such that the triangle $ABC$ is a right isosceles triangle
  with $AB$ as its hypotenuse? {\em Didn't appear last time}.

  \begin{enumerate}[(A)]
  \item Sphere
  \item Plane
  \item Circle
  \item Line
  \item Square
  \end{enumerate}

  {\em Answer}: Option (C)

  {\em Explanation}: It arises as the intersection of spheres centered
  at $A$ and $B$ of radius equal to $|AB|/\sqrt{2}$.

  {\em Performance review}: $20$ out of $25$ got this. $4$ chose (E),
  $1$ chose (B).

\item Given two distinct points $A$ and $B$ in three-dimensional
  space, what is the nature of the set of possibilities for a third
  point $C$ such that the triangle $ABC$ is equilateral?

  \begin{enumerate}[(A)]
  \item Sphere
  \item Plane
  \item Circle
  \item Line
  \item Two points
  \end{enumerate}

  {\em Answer}: Option (C)

  {\em Explanation}: It arises as an intersection of spheres centered
  at $A$ and $B$ with radius equal to $AB$. Alternatively, pick any
  one choice of $C$. The set of all choices can be obtained by
  revolving this point about the line of $AB$, and we get a circle.

  {\em Performance review}: $24$ out of $25$ got this. $1$ chose (D).

  {\em Historical note (last time)}: $23$ out of $24$ people got this
  correct (way to go, folks!). $1$ person chose (B).

\item Given two distinct points $A$ and $B$ in three-dimensional
  space, what is the nature of the set of possibilities for a third
  point $C$ such that $|AC|/|BC| = \lambda$ for $\lambda$ a fixed
  positive real number not equal to $1$? {\em Didn't appear last time}.

  \begin{enumerate}[(A)]
  \item Sphere
  \item Plane
  \item Circle
  \item Line
  \item Square
  \end{enumerate}

  {\em Answer}: Option (A)

  {\em Explanation}: Use distance formula, simplify. Similar questions
  appear on your homework.

  {\em Performance review}: $5$ out of $25$ got this. $7$ chose (C),
  $6$ each chose (B) and (D), $1$ chose (E).

\item Consider the parametric curve in three dimensions given by the
  coordinate description $t \mapsto (\cos t, \sin t, \cos(2t))$, with
  $t \in \R$. We can consider the {\em projections} of this curve onto
  the $xy$-plane, $yz$-plane, and $xz$-plane, which are basically what
  we get by dropping perpendiculars from the curve to these
  planes. What is the correct description of the curves obtained by
  doing the three projections?

  \begin{enumerate}[(A)]
  \item The projections on the $xy$-plane and $yz$-plane are both parts
    of parabolas, and the projection on the $xz$-plane is a circle.
  \item The projections on the $xy$-plane and $yz$-plane are both
    circles, and the projection on the $xz$-plane is a part of a
    parabola.
  \item The projection on the $xy$-plane is a circle, and the
    projections on the $yz$-plane and $xz$-plane are both parts of
    parabolas.
  \item The projection on the $xy$-plane is a part of a parabola, the
    projection on the $xz$-plane and $yz$-plane are both circles.
  \item All the three projections are circles.
  \end{enumerate}

  {\em Answer}: Option (C)

  {\em Explanation}: The projection on the $xy$-plane is just $t
  \mapsto (\cos t, \sin t)$, which is the unit circle. The projection
  on the $xz$-plane is $t \mapsto (\cos t, \cos(2t))$ and we have the
  quadratic relation $\cos(2t) = 2 (\cos t)^2 - 1$, subject to domain
  restriction $\cos t \in [-1,1]$. Thus, we get a part of a
  parabola. The projection on the $yz$-plane is $t \mapsto (\sin t,
  \cos(2t))$ and we have the quadratic relation $\cos(2t) = 1 - 2(\sin
  t)^2$, subject to domain restriction $\sin t \in [-1,1]$. Thus, we
  get a part of a parabola.

  {\em Performance review}: $24$ out of $25$ got this. $1$ chose (A).

  {\em Historical note (last time)}: $17$ out of $24$ people got this
  correct. $4$ chose (B) and $3$ chose (D).
\end{enumerate}
\end{document}