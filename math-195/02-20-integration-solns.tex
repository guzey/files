\documentclass[10pt]{amsart}

%Packages in use
\usepackage{fullpage, hyperref, vipul, enumerate}

%Title details
\title{Take-home class quiz solutions: due Wednesday February 20: Integration techniques (one variable)}
\author{Math 195, Section 59 (Vipul Naik)}
%List of new commands

\begin{document}
\maketitle

\section{Performance review}

$25$ people took this quiz. The score distribution was as follows:

\begin{itemize}
\item Score of $3$: $1$ person.
\item Score of $5$: $1$ person.
\item Score of $6$: $1$ person.
\item Score of $9$: $1$ person.
\item Score of $10$: $1$ person.
\item Score of $12$: $1$ person.
\item Score of $15$: $1$ person.
\item Score of $18$: $7$ people.
\item Score of $19$: $2$ people.
\item Score of $20$: $4$ people.
\item Score of $21$: $5$ people.
\end{itemize}
Here are the question wise answers and performance review:

\begin{enumerate}
\item Option (A): $20$ people.
\item Option (B): $22$ people.
\item Option (A): $17$ people.
\item Option (C): $19$ people.
\item Option (B): $19$ people.
\item Option (D): $21$ people.
\item Option (B): $20$ people.
\item Option (D): $22$ people.
\item Option (B): $21$ people.
\item Option (C): $21$ people.
\item Option (C): $20$ people.
\item Option (D): $20$ people.
\item Option (A): $22$ people.
\item Option (E): $21$ people.
\item Option (D): $22$ people.
\item Option (B): $15$ people.
\item Option (D): $21$ people.
\item Option (B): $17$ people.
\item Option (E): $9$ people.
\item Option (D): $23$ people.
\item Option (D): $17$ people.
\end{enumerate}

\section{Solutions}

In the questions below, we say that a function is {\em expressible in
terms of elementary functions} or {\em elementarily expressible} if it
can be expressed in terms of polynomial functions, rational functions,
radicals, exponents, logarithms, trigonometric functions and inverse
trigonometric functions using pointwise combinations, compositions,
and piecewise definitions. We say that a function is {\em elementarily
integrable} if it has an elementarily expressible antiderivative.

Note that if a function is elementarily expressible, so is its
derivative on the domain of definition.

We say that a function $f$ is $k$ times elementarily integrable if
there is an elementarily expressible function $g$ such that $f$ is the
$k^{th}$ derivative of $g$.

We say that the integrals of two functions are {\em equivalent up to
elementary functions} if an antiderivative for one function can be
expressed using an antiderivative for the other function and
elementary function, again piecing them together using pointwise
combination, composition, and piecewise definitions.

\begin{enumerate}

\item Suppose $F$ and $G$ are continuously differentiable functions on
  all of $\R$ (i.e., both $F'$ and $G'$ are continuous). Which of the
  following is {\bf not necessarily true}? Please see Option (E)
  before answering.

  \begin{enumerate}[(A)]
  \item If $F'(x) = G'(x)$ for all integers $x$, then $F - G$ is a
    constant function when restricted to integers, i.e., it takes the
    same value at all integers.
  \item If $F'(x) = G'(x)$ for all numbers $x$ that are not integers,
    then $F - G$ is a constant function when restricted to the set of
    numbers $x$ that are not integers.
  \item If $F'(x) = G'(x)$ for all rational numbers $x$, then $F - G$
    is a constant function when restricted to the set of rational
    numbers.
  \item If $F'(x) = G'(x)$ for all irrational numbers $x$, then $F -
    G$ is a constant function when restricted to the set of irrational
    numbers.
  \item None of the above, i.e., they are all necessarily true.
  \end{enumerate}

  {\em Answer}: Option (A).

  {\em Explanation}: The fact that the derivatives of two functions
  agree at integers says nothing about how the derivatives behave
  elsewhere -- they could differ quite a bit at other places. Hence,
  (A) is not necessarily true, and hence must be the right option. All
  the other options are correct as statements and hence cannot be the
  right option. This is because in all of them, the set of points
  where the derivatives agree is {\em dense} -- it intersects every
  open interval. So, continuity forces the functions $F'$ and $G'$ to
  be equal everywhere, forcing $F - G$ to be constant everywhere.

  {\em Performance review}: $20$ out of $25$ people got this. $3$
  chose (E), $1$ each chose (B) and (D).

  {\em Historical note (earlier appearance this quarter)}: $15$ out of
  $24$ got this. $6$ chose (E), $3$ chose (B).

\item Suppose $F$ and $G$ are two functions defined on $\R$ and $k$ is
  a natural number such that the $k^{th}$ derivatives of $F$ and $G$
  exist and are equal on all of $\R$. Then, $F - G$ must be a
  polynomial function. What is the {\bf maximum possible degree} of $F
  - G$?  (Note: Assume constant polynomials to have degree zero)

  \begin{enumerate}[(A)]
  \item $k - 2$
  \item $k - 1$
  \item $k$
  \item $k + 1$
  \item There is no bound in terms of $k$.
  \end{enumerate}

  {\em Answer}: Option (B)

  {\em Explanation}: $F$ and $G$ having the same $k^{th}$ derivative
  is equivalent to requiring that $F - G$ have $k^{th}$ derivative
  equal to zero. For $k = 1$, this gives constant functions
  (polynomials of degree $0$). Each time we increment $k$, the degree
  of the polynomial could potentially go up by $1$. Thus, the answer
  is $k - 1$.

  {\em Performance review}: $22$ out of $25$ people got this. $2$
  chose (E), $1$ chose (C).

  {\em Historical note (earlier appearance this quarter)}: $21$ out of
  $24$ got this. $2$ chose (A), $1$ chose (E).

\item Suppose $f$ is a continuous function on $\R$. Clearly, $f$ has
  antiderivatives on $\R$. For all but one of the following
  conditions, it is possible to guarantee, without any further
  information about $f$, that there exists an antiderivative $F$
  satisfying that condition. {\bf Identify the exceptional condition}
  (i.e., the condition that it may not always be possible to satisfy).

  \begin{enumerate}[(A)]
  \item $F(1) = F(0)$.
  \item $F(1) + F(0) = 0$.
  \item $F(1) + F(0) = 1$.
  \item $F(1) = 2F(0)$.
  \item $F(1)F(0) = 0$.
  \end{enumerate}

  {\em Answer}: Option (A)

  {\em Explanation}: Suppose $G$ is an antiderivative for $f$. The
  general expression for an antiderivative is $G + C$, where $C$ is
  constant. We see that for options (b), (c), and (d), it is always
  possible to solve the equation we obtain to get one or more real
  values of $C$. However, (a) simplifies to $G(1) + C = G(0) + C$,
  whereby $C$ is canceled, and we are left with the statement $G(1) =
  G(0)$. If this statement is true, then {\em all} choices of $C$
  work, and if it is false, then {\em none} works. Since we cannot
  guarantee the truth of the statement, (a) is the exceptional
  condition.

  Another way of thinking about this is that $F(1) - F(0) = \int_0^1
  f(x) \, dx$, regardless of the choice of $F$. If this integral is
  $0$, then any antiderivative works. If it is not zero, no
  antiderivative works.

  {\em Performance review}: $17$ out of $25$ people got this. $5$
  chose (D), $2$ chose (E), $1$ chose (B).

  {\em Historical note (earlier appeance this quarter)}: $18$ out of
  $24$ got this. $3$ chose (C), $2$ chose (E), $1$ chose (B).

\item Suppose $F$ is a function defined on $\R \setminus \{ 0 \}$ such
  that $F'(x) = -1/x^2$ for all $x \in \R \setminus \{ 0 \}$. Which of
  the following pieces of information is/are {\bf sufficient} to determine
  $F$ completely? Please see options (D) and (E) before answering.

  \begin{enumerate}[(A)]
  \item The value of $F$ at any two positive numbers.
  \item The value of $F$ at any two negative numbers.
  \item The value of $F$ at a positive number and a negative number.
  \item Any of the above pieces of information is sufficient, i.e., we
    need to know the value of $F$ at any two numbers.
  \item None of the above pieces of information is sufficient.
  \end{enumerate}

  {\em Answer}: Option (C)

  {\em Explanation}: There are two open intervals: $(-\infty,0)$ and
  $(0,\infty)$, on which we can look at $F$. On each of these
  intervals, $F(x) = 1/x + $ a constant, but the constant for
  $(-\infty,0)$ may differ from the constant for $(0,\infty)$. Thus,
  we need the initial value information at one positive number and one
  negative number.

  {\em Performance review}: $19$ out of $25$ people got this. $6$
  chose (D).

  {\em Historical note (earlier appearance this quarter)}: $16$ out of
  $24$ got this. $8$ chose (D).

\item Suppose $F,G$ are continuously differentiable functions defined
  on all of $\R$. Suppose $a,b$ are real numbers with $a <
  b$. Suppose, further, that $G(x)$ is identically zero everywhere
  except on the open interval $(a,b)$. Then, what can we say about the
  relationship between the numbers $P = \int_a^b F(x)G'(x) \,dx$ and
  $Q = \int_a^b F'(x)G(x) \, dx$?

  \begin{enumerate}[(A)]
  \item $P = Q$
  \item $P = -Q$
  \item $PQ = 0$
  \item $P = 1 - Q$
  \item $PQ = 1$
  \end{enumerate}

  {\em Answer}: Option (B)

  {\em Explanation}: Integration by parts gives us that:

  $$\int_a^b F(x)G'(x) \, dx = [F(x)G(x)]_a^b - \int_a^b F'(x)G(x) \, dx$$

  Since $G(x) = 0$ outside $(a,b)$, we get that $G(a) = G(b) = 0$, so
  that the evaluation of $[F(x)G(x)]_a^b$ gives $0$. We are thus left with:

  $$P = -Q$$

  {\em Performance review}: $19$ out of $25$ people got this. $3$
  chose (D), $2$ chose (A), $1$ chose (C).

  {\em Historical note (Math 153)}: $32$ out of $41$ got this. $5$ chose (A),
  $2$ chose (C), $1$ chose (D), $1$ wrote multiple options.

\item Consider the integration $\int p(x) q''(x) \,
  dx$. Apply integration by parts twice, first taking
  $p$ as the part to differentiate, and $q$ as
  the part to integrate, and then again apply integration by parts to
  avoid a circular trap. What can we conclude?

  \begin{enumerate}[(A)]
  \item $\int p(x) q''(x) \, dx = \int p''(x) q(x) \, dx$
  \item $\int p(x) q''(x) \, dx = \int p'(x) q'(x) \, dx - \int p''(x) q(x) \, dx$
  \item $\int p(x)q''(x) \,dx = p'(x)q'(x) - \int p''(x) q(x)\, dx$
  \item $\int p(x)q''(x) \,dx = p(x)q'(x) - p'(x)q(x) + \int p''(x) q(x)\, dx$
  \item $\int p(x)q''(x) \,dx = p(x)q'(x) - p'(x)q(x) - \int p''(x) q(x)\, dx$
  \end{enumerate}

  {\em Answer}: Option (D)

  {\em Explanation}: Just write it out.

  {\em Performance review}: $21$ out of $25$ people got this. $2$
  chose (C), $1$ each chose (B) and (E).

  {\em Historical note (Math 153)}: $35$ out of $41$ got this. $4$ chose (E)
  (sign error, didn't notice double negative), $2$ chose (A).

\item Suppose $p$ is a polynomial function. In order to
  find the indefinite integral for a function of the form $x
  \mapsto p(x)\exp(x)$, the general strategy, which always
  works, is to take $p(x)$ as the part to differentiate and
  $\exp(x)$ as the part to integrate, and keep repeating
  the process. Which of the following is the best explanation for why
  this strategy works?

  \begin{enumerate}[(A)]
  \item $\exp$ can be repeatedly differentiated (staying $\exp$) and
    polynomials can be repeatedly integrated (giving polynomials all
    the way).
  \item $\exp$ can be repeatedly integrated (staying $\exp$) and
    polynomials can be repeatedly differentiated, eventually becoming
    zero.
  \item $\exp$ and polynomials can both be repeatedly differentiated.
  \item $\exp$ and polynomials can both be repeatedly integrated.
  \item We need to use the recursive version of integration by parts
    whereby the original integrand reappears after a certain number of
    applications of integration by parts (i.e., the polynomial equals
    one of its higher derivatives, up to sign and scaling).
  \end{enumerate}

  {\em Answer}: Option (B)

  {\em Explanation}: This follows because the polynomial is the part
  that we are choosing to differentiate.

  {\em Performance review}: $20$ out of $25$ people got this. $2$ each
  chose (C) and (E). $1$ left the question blank.

  {\em Historical note (Math 153)}: All $41$ got this.

\item Consider the function $x \mapsto \exp(x) \sin
  x$. This function can be integrated using integration by
  parts. What can we say about how integration by parts works?

  \begin{enumerate}[(A)]
  \item We choose $\exp$ as the part to integrate and $\sin$ as the
    part to differentiate, and apply this process once to get the
    answer directly.
  \item We choose $\exp$ as the part to integrate and $\sin$ as the
    part to differentiate, and apply this process once, then use a
    {\em recursive} method (identify the integrals on the left and
    right side) to get the answer.
  \item We choose $\exp$ as the part to integrate and $\sin$ as the
    part to differentiate, and apply this process twice to get the
    answer directly.
  \item We choose $\exp$ as the part to integrate and $\sin$ as the
    part to differentiate, and apply this process twice, then use a
    {\em recursive} method (identify the integrals on the left and
    right side) to get the answer.
  \item We choose $\exp$ as the part to integrate and $\sin$ as the
    part to differentiate, and we apply integration by parts four
    times to get the answer directly.
  \end{enumerate}

  {\em Answer}: Option (D)

  {\em Explanation}: $\sin$ is the negative of its second derivative,
  $\exp$ equals its second antiderivative.

  {\em Performance review}: $22$ out of $25$ people got this. $3$ chose (C).

  {\em Historical note (Math 153)}: $38$ out of $41$ got this. $3$
  chose (C).


\item Suppose $f$ is a continuous function on all of $\R$ and is the
  third derivative of an elementarily expressible function, but is not
  the fourth derivative of any elementarily expressible function. In
  other words, $f$ can be integrated three times but not four times
  within the collection of elementarily expressible functions. What is
  the {\bf largest positive integer} $k$ such that $x \mapsto x^kf(x)$
  is {\em guaranteed to be} {\bf elementarily integrable}?

  \begin{enumerate}[(A)]
  \item $1$
  \item $2$
  \item $3$
  \item $4$
  \item $5$
  \end{enumerate}

  {\em Answer}: Option (B)

  {\em Explanation}: Via integration by parts, integrating $f$ $m$ times is
  equivalent to finding antiderivatives for $f(x)$, $xf(x)$, and so on
  till $x^{m-1}f(x)$. In our case, $f$ can be integrated $3$ times, so
  the largest $k$ is $3 - 1 = 2$.

  {\em Performance review}: $21$ out of $25$ people got this. $3$
  chose (C), $1$ chose (D).

  {\em Historical note (last time)}: $5$ out of $18$ people got this
  correct. $6$ chose (D), $5$ chose (C), and $2$ chose (E).

\item Suppose $f$ is a continuous function on $(0,\infty)$ and is the
  third derivative of an elementarily expressible function, but is not
  the fourth derivative of any elementarily expressible function. In
  other words, $f$ can be integrated three times but not four times
  within the collection of elementarily expressible functions. What is
  the {\bf largest positive integer} $k$ such that the function $x
  \mapsto f(x^{1/k})$ with domain $(0,\infty)$ is {\em guaranteed to
  be} {\bf elementarily integrable}?

  \begin{enumerate}[(A)]
  \item $1$
  \item $2$
  \item $3$
  \item $4$
  \item $5$
  \end{enumerate}

  {\em Answer}: Option (C)

  {\em Explanation}: Via the $u$-substitution $u = x^{1/k}$, we get
  $\int ku^{k-1}f(u) \, du$. Now using the previous question, the
  maximum value of $k - 1$ possible is $2$, so the maximum possible
  value is $3$.

  We can also do a direct integration by parts taking $1$ as the
  second part.

  {\em Performance review}: $21$ out of $25$ people got this. $2$ each
  chose (A) and (B).

  {\em Historical note (last time)}: $8$ out of $18$ people got this
  correct. $5$ chose (A), $3$ chose (D), and $2$ chose (B).
  
\item Of these five functions, four of the functions are elementarily
  integrable and can be integrated using integration by parts. The
  other one function is {\bf not elementarily integrable}. Identify
  this function.

  \begin{enumerate}[(A)]
  \item $x \mapsto x \sin x$
  \item $x \mapsto x \cos x$
  \item $x \mapsto x \tan x$
  \item $x \mapsto x \sin^2x$
  \item $x \mapsto x \tan^2x$
  \end{enumerate}

  {\em Answer}: Option (C)

  {\em Explanation}: If $f$ is elementarily integrable, then $xf(x)$
  is elementarily integrable iff $f$ is twice elementarily integrable;
  this is easily seen using integration by parts. Of the function
  options given here, $\tan$ is the only function that is not twice
  elementarily integrable, because the first integration gives
  $-\ln|\cos x|$ which cannot be integrated. Of the others, note that
  $\sin$, $\cos$, and $\sin^2$ can be integrated using elementary
  functions infinitely many times. $\tan^2$ is twice elementarily
  integrable but no further: integrates the first time to $\tan x -
  x$, which integrates one more time to $-\ln|cos x| - x^2/2$, which
  cannot be integrated further.

  {\em Performance review}: $20$ out of $25$ people got this. $2$ each
  chose (D) and (E), $1$ chose (B).

  {\em Historical note (last time)}: $8$ out of $18$ people got this
  correct. $8$ people chose (E) and $2$ people chose (D).

\item Consider the four functions $f_1(x) = \sqrt{\sin x}$, $f_2(x) =
  \sin \sqrt{x}$, $f_3(x) = \sin^2 x$ and $f_4(x) = \sin(x^2)$, all
  viewed as functions on the interval $[0,1]$ (so they are all well
  defined). Two of these functions are elementarily integrable; the
  other two are not. Which are {\bf the two elementarily integrable
  functions}?

  \begin{enumerate}[(A)]
  \item $f_3$ and $f_4$.
  \item $f_1$ and $f_3$.
  \item $f_1$ and $f_4$. 
  \item $f_2$ and $f_3$.
  \item $f_2$ and $f_4$.
  \end{enumerate}

  {\em Answer}: Option (D)

  {\em Explanation}: Integration of $f_3$ is a standard procedure, so
  we say nothing about that. As for $f_2$, recall that integrating
  $f(x^{1/k})$ is equivalent to integrating $u^{k-1}f(u)$ where $u =
  x^{1/k}$, which in turn is equivalent to integrating $f$ $k$
  times. Since $\sin$ can be integrated as many times as we wish,
  $f_2$ can be integrated.

  The reason why $f_1$ and $f_4$ are not elementarily integrable is
  subtler but it's clear that none of the obvious methods work.

  {\em Performance review}: $20$ out of $25$ people got this. $2$
  chose (B), $1$ each chose (A), (C), and (E).

  {\em Historical note (last time)}: $5$ out of $18$ people got this
  correct. $5$ people each chose (B) and (E), $2$ chose (C), $1$ chose
  (A).


\item Which of the following functions has an antiderivative that is
  {\bf not equivalent} up to elementary functions to the antiderivative of
  $x \mapsto e^{-x^2}$?

  \begin{enumerate}[(A)]
  \item $x \mapsto e^{-x^4}$
  \item $x \mapsto e^{-x^{2/3}}$
  \item $x \mapsto e^{-x^{2/5}}$
  \item $x \mapsto x^2e^{-x^2}$
  \item $x \mapsto x^4e^{-x^2}$
  \end{enumerate}

  {\em Answer}: Option (A)

  {\em Explanation}: We show the equivalence with the others.

  Option (D): We use integration by parts, writing
  $x^2e^{-x^2}$ as $x \cdot (xe^{-x^2})$ and taking $xe^{-x^2}$ as the
  part to integrate, so that $x$ is the part to differentiate. An
  antiderivative for $xe^{-x^2}$ is $(-1/2)e^{-x^2}$, so we get:

  $$\frac{-x}{2}e^{-x^2} - \int \frac{-1}{2}e^{-x^2} \, dx$$

  We thus see that it reduces to $\int e^{-x^2} \, dx$.

  Option (E), via reduction to option (D): We use integration by
  parts, taking $x^3$ as the part to differentiate and $xe^{-x^2}$ as
  the part to integrate. One application of integration by parts
  reduces this to $\int x^2e^{-x^2}$, which is option (D).

  Option (B), via reduction to option (D): Start with $\int
  e^{-x^{2/3}} \, dx$. Put $u = x^{1/3}$. The substitution gives (up to
  scalars) $\int u^2e^{-u^2} \, du$, which is option (D).

  Option (C), via reduction to option (D): Start with $\int
  e^{-x^{2/5}} \, dx$. Put $u = x^{1/5}$. The substitution gives (up to
  scalars) $\int u^4e^{-u^2} \, du$, which is option (E).

  {\em Performance review}: $22$ out of $25$ people got this. $1$ each
  chose (B), (D), and (E).

  {\em Historical note (last time)}: $6$ out of $18$ people got this
  correct. $6$ chose (C), $2$ each chose (B), (D), (E).


\item Consider the statements $P$ and $Q$, where $P$ states that every
  rational function is elementarily integrable, and $Q$ states that
  any rational function is $k$ times elementarily integrable for all
  positive integers $k$.

  Which of the following additional observations is {\bf correct} and
  {\bf allows us to deduce} $Q$ given $P$?

  \begin{enumerate}[(A)]
  \item There is no way of deducing $Q$ from $P$ because $P$ is true
    and $Q$ is false.
  \item The antiderivative of a rational function can always be chosen
    to be a rational function, hence $Q$ follows from a repeated
    application of $P$.
  \item Using integration by parts, we see that repeated integration
    of a function $f$ is equivalent to integrating $f$, $f^2$, $f^3$,
    and higher powers of $f$ (the powers here are pointwise products,
    not compositions). If $f$ is a rational function, each of these is
    also a rational function. Applying $P$, each of these is
    elementarily integrable, hence $f$ is $k$ times elementarily
    integrable for all $k$.
  \item Using integration by parts, we see that repeated integration
    of a function $f$ is equivalent to integrating $f$, $f'$, $f''$,
    and higher derivatives of $f$. If $f$ is a rational function, each
    of these is also a rational function. Applying $P$, each of these
    is elementarily integrable, hence $f$ is $k$ times elementarily
    integrable for all $k$.
  \item Using integration by parts, we see that repeated integration
    of a function $f$ is equivalent to integrating each of the
    functions $f(x)$, $xf(x)$, $\dots$. If $f$ is a rational function,
    each of these is also a rational function. Applying $P$, each of
    these is elementarily integrable, hence $f$ is $k$ times
    elementarily integrable for all $k$.
  \end{enumerate}

  {\em Answer}: Option (E)

  {\em Explanation}: Review the material on rational function
  integration.

  {\em Performance review}: $21$ out of $25$ got this. $2$ chose (D),
  $1$ each chose (B) and (C).

  {\em Historical note (last time)}: $3$ out of $18$ people got this
  correct. $7$ chose (C), $6$ chose (D), $1$ each chose (A) and (B).

\item Which of these functions of $x$ is {\em not} elementarily
  integrable?

  \begin{enumerate}[(A)]
  \item $x\sqrt{1 + x^2}$
  \item $x^2\sqrt{1 + x^2}$
  \item $x(1 + x^2)^{1/3}$
  \item $x\sqrt{1 + x^3}$
  \item $x^2\sqrt{1 + x^3}$
  \end{enumerate}

  {\em Answer}: Option (D)

  {|em Explanation}: For options (A) and (C), the substitution $u = 1
  + x^2$ works fine. For option (E), the substitution $u = 1 + x^3$
  works fine. For option (B), we can solve the problem using a
  trigonometric substitution. This leaves option (D) (which,
  incidentally, requires the use of elliptic integrals).

  {\em Performance review}: $22$ out of $25$ got this. $3$ chose (C).

  {\em Historical note (last time)}: $9$ out of $18$ people got this
  correct. $4$ chose (C), $4$ chose (B), $1$ chose (A).


\item Consider the function $f(k) := \int_1^2 \frac{dx}{\sqrt{x^2 +
  k}}$. $f$ is defined for $k \in (-1,\infty)$. What can we say about
  the nature of $f$ within this interval?

  \begin{enumerate}[(A)]
  \item $f$ is increasing on the interval $(-1,\infty)$.
  \item $f$ is decreasing on the interval $(-1,\infty)$.
  \item $f$ is increasing on $(-1,0)$ and decreasing on $(0,\infty)$.
  \item $f$ is decreasing on $(-1,0)$ and increasing on $(0,\infty)$.
  \item $f$ is increasing on $(-1,0)$, decreasing on $(0,2)$, and
    increasing again on $(2,\infty)$.
  \end{enumerate}

  {\em Answer}: Option (B)

  {\em Explanation}: For any fixed value of $x \in [1,2]$, the
  integrand $1/\sqrt{x^2 + k}$ is a {\em decreasing} function of $k$
  for $k \in (-1,\infty)$. Hence, the value we get upon integrating it
  for $x \in [1,2]$ should also be a decreasing function of $k$.

  {\em Performance review}: $15$ out of $25$ got this. $6$ chose (D),
  $3$ chose (C), $1$ chose (A).

  {\em Historical note (last time)}: $6$ out of $18$ people got this
  correct. $3$ chose (A), $5$ chose (C), $2$ chose (E), $1$ chose (D),
  $1$ left the question blank.

\item For which of these functions of $x$ does the antiderivative
  necessarily involve {\em both} $\arctan$ {\em and} $\ln$?

  \begin{enumerate}[(A)]
  \item $1/(x + 1)$
  \item $1/(x^2 + 1)$
  \item $x/(x^2 + 1)$
  \item $x/(x^3 + 1)$
  \item $x^2/(x^3 + 1)$
  \end{enumerate}

  {\em Answer}: Option (D)

  {\em Explanation}: Option (A) integrates to $\ln|x + 1|$, option (B)
  integrates to $\arctan x$, option (C) integrates to $(1/2) \ln(x^2 +
  1)$, and option (E) integrates to $(1/3) \ln|x^3 + 1|$. For option
  (D), we need to use partial fractions with denominators $x + 1$ and
  $x^2 - x + 1$, and we end up getting nonzero coefficients on terms
  that integrate to $\ln$ and to $\arctan$.

  {\em Performance review}: $21$ out of $25$ people got this. $2$
  chose (E), $1$ each chose (B) and (C).

  {\em Historical note (last time)}: $13$ out of $18$ people got this
  correct. $3$ chose (C), $1$ each chose (B) and (E).

\item Suppose $F$ is a (not known) function defined on $\R \setminus
  \{ -1,0,1\}$, differentiable everywhere on its domain, such that
  $F'(x) = 1/(x^3 - x)$ everywhere on $\R \setminus \{-1,0,1\}$. For
  which of the following sets of points is it true that knowing the
  value of $F$ at these points {\bf uniquely} determines $F$?

  \begin{enumerate}[(A)]
  \item $\{ -\pi, -e, 1/e,1/\pi \}$
  \item $\{ -\pi/2, -\sqrt{3}/2, 11/17,\pi^2/6 \}$
  \item $\{ -\pi^3/7,-\pi^2/6,\sqrt{13},11/2 \}$
  \item Knowing $F$ at any of the above determines the value of $F$
    uniquely.
  \item None of the above works to uniquely determine the value of
    $F$.
  \end{enumerate}

  {\em Answer}: Option (B)

  {\em Explanation}: The domain of $F$ has four connected components:
  the open intervals $(-\infty,-1)$, $(-1,0)$, $(0,1)$, and
  $(1,\infty)$. We need to know the value of $F$ at one point in each
  of these intervals. By computing values, we see that the set of
  points in option (B) has the property that it contains one point in
  each of these intervals, and those in options (A) and (C) do not.

  {\em Performance review}: $17$ out of $25$ people got this. $4$
  chose (A), $2$ chose (E), $1$ each chose (C) and (D).

  {\em Historical note (last time)}: $8$ out of $18$ people got this
  correct. $5$ chose (D), $3$ chose (A), $2$ left the question blank.

\item Suppose $F$ is a continuously differentiable function whose
  domain contains $(a,\infty)$ for some $a \in \R$, and $F'(x)$ is a
  rational function $p(x)/q(x)$ on the domain of $F$. Further, suppose
  that $p$ and $q$ are nonzero polynomials. Denote by $d_p$ the degree
  of $p$ and by $d_q$ the degree of $q$. Which of the following is a
  {\bf necessary and sufficient condition} to ensure that $\lim_{x \to
  \infty} F(x)$ is finite?

  \begin{enumerate}[(A)]
  \item $d_p - d_q \ge 2$
  \item $d_p - d_q \ge 1$
  \item $d_p = d_q$
  \item $d_q - d_p \ge 1$
  \item $d_q - d_p \ge 2$
  \end{enumerate}

  {\em Answer}: Option (E)

  {\em Explanation}: This can be justified in terms of partial
  fractions. Th ecase where $q$ is a product of linear factors can be
  justified using the previous question. But that is not the most
  elegant justification. When we cover sequences and series, we will
  see some comparison tests that make it clear why this holds. The
  basic example you can keep in mind is that the antiderivative of
  $1/x^2$ is $-1/x$, which has a finite limit as $x \to \infty$.

  {\em Performance review}: $9$ out of $25$ people got tihs. $13$
  chose (D), $2$ chose (C), $1$ chose (B).

  {\em Historical note (last time)}: $3$ out of $18$ people
  got this correct. $7$ chose (D), $5$ chose (B), $2$ chose (C), $1$
  chose (A).

  Those who chose (D) had the right idea but failed to account for the
  extra margin that needs to be maintained because an integration is
  being performed.

  For the next two questions, build on the observation: For any
  nonconstant monic polynomial $q(x)$, there exists a finite
  collection of transcendental functions $f_1, f_2, \dots, f_r$ such
  that the antiderivative of any rational function $p(x)/q(x)$, on an
  open interval where it is defined and continuous, can be expressed
  as $g_0 + f_1g_1 + f_2g_2 + \dots + f_rg_r$ where $g_0, g_1, \dots,
  g_r$ are rational functions.

\item For the polynomial $q(x) = 1 + x^2$, what collection of $f_i$s
  works (all are written as functions of $x$)?

  \begin{enumerate}[(A)]
  \item $\arctan x$ and $\ln|x|$
  \item $\arctan x$ and $\arctan(1 + x^2)$
  \item $\ln|x|$ and $\ln(1 + x^2)$ 
  \item $\arctan x$ and $\ln(1 + x^2)$
  \item $\ln|x|$ and $\arctan(1 + x^2)$
  \end{enumerate}

  {\em Answer}: Option (D)

  {\em Explanation}: Follows from the standard partial fraction
  decomposition. $2x/(1 + x^2)$ gives the $\ln$ integration and $1/(1
  + x^2)$ gives the $\arctan$ integration.

  {\em Performance review}: $23$ out of $25$ people got this. $1$ each
  chose (A) and (C).

  {\em Historical note (last time)}: $5$ out of $18$ people got this
  correct. $4$ each chose (C) and (E), $3$ chose (A), $2$ chose (B).

\item For the polynomial $q(x) := 1 + x^2 + x^4$, what is the
  size of the smallest collection of $f_i$s that works?

  \begin{enumerate}[(A)]
  \item $1$
  \item $2$
  \item $3$
  \item $4$
  \item $5$
  \end{enumerate}

  {\em Answer}: Option (D)

  {\em Explanation}: The denominator factors into $x^2 - x + 1$ and
  $x^2 + x + 1$. Each of these contributes one $\arctan$ possibility
  and one $\ln$ possibility. A total of $4$ possibilities is achieved.

  In general, if there are no repeated factors, the smallest number of
  pieces equals the degree of the polynomial.

  {\em Performance review}: $17$ out of $25$ people got this. $8$
  chose (B).

  {\em Historical note (Math 153)}: $30$ out of $44$ got this. $8$ chose (B),
  $5$ chose (C), $1$ chose (E).

\end{enumerate}

\end{document}
