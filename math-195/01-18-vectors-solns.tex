\documentclass[10pt]{amsart}

%Packages in use
\usepackage{fullpage, hyperref, vipul, enumerate}

%Title details
\title{Class quiz solutions: Friday January 18: Vectors}
\author{Math 195, Section 59 (Vipul Naik)}
%List of new commands

\begin{document}
\maketitle

\section{Performance review}

? people took this $6$-question quiz. The score distribution was as follows:

\begin{itemize}
\item Score of $3$: $3$ people.
\item Score of $4$: $3$ people.
\item Score of $5$: $17$ people.
\item Score of $6$: $1$ person.
\end{itemize}

The mean score was about $4.5$. The question wise solutions and
performance summary are below:

\begin{enumerate}
\item Option (C): $21$ people.
\item Option (E): $23$ people.
\item Option (C): $20$ people.
\item Option (E): $21$ people.
\item Option (D): $3$ people.
\item Option (C): $24$ people.
\end{enumerate}

\section{Solutions}
\begin{enumerate}

\item Suppose $S$ is a collection of {\em nonzero} vectors in $\R^3$
  with the property that the dot product of any two distinct elements
  of $S$ is zero. What is the maximum possible size of $S$?

  \begin{enumerate}[(A)]
  \item $1$
  \item $2$
  \item $3$
  \item $4$
  \item There is no finite bound on the size of $S$
  \end{enumerate}

  {\em Answer}: Option (C)

  {\em Explanation}: The given property indicates that any two
  elements of $S$ are mutually orthogonal vectors. In $\R^3$, there
  can be at most three mutually orthogonal directions, because the
  space is three-dimensional. (This can be proved formally, but we
  won't bother here).

  {\em Performance review}: $21$ out of $24$ people got this. $3$ chose (B).

  {\em Historical note (last year)}: $14$ out of $23$ people got this
  correct. $6$ chose (E), $2$ chose (B), $1$ chose (D).

\item Suppose $S$ is a collection of {\em nonzero} vectors in $\R^3$
  such that the cross product of any two distinct elements of $S$ is the
  zero vector. What is the maximum possible size of $S$?

  \begin{enumerate}[(A)]
  \item $1$
  \item $2$
  \item $3$
  \item $4$
  \item There is no finite bound on the size of $S$
  \end{enumerate}

  {\em Answer}: Option (E)

  {\em Explanation}: The given condition means that any two elements
  of $S$ are scalar multiples of each other. We can easily construct
  an infinite set with this property: consider {\em all} nonzero
  scalar multiples of a fixed nonzero vector. There are infinitely
  many such multiples, because there are infinitely many nonzero
  reals, and each such multiple is different.

  {\em Performance review}: $23$ out of $24$ people got this. $1$
  person chose (B).

  {\em Historical note (last year)}: $17$ out of $23$ people got this
  correct. $3$ chose (A), $2$ chose (B), $1$ chose (C).

\item Suppose $a$ and $b$ are vectors in $\R^3$. Which of the
  following is/are true?

  \begin{enumerate}[(A)]
  \item If both $a$ and $b$ are nonzero vectors, then $a \times b$ is
    a nonzero vector.
  \item If $a \times b$ is a nonzero vector, then $a \cdot (a \times
    b)$ is a nonzero real number.
  \item If $a \times b$ is a nonzero vector, then $a \times (a \times
    b)$ is a nonzero vector.
  \item All of the above
  \item None of the above
  \end{enumerate}

  {\em Answer}: Option (C)

  {\em Explanation}: If $a \times b$ is a nonzero vector, then it is
  in particular orthogonal to both $a$ and $b$. Further, it also means
  that neither $a$ nor $b$ is zero. Thus, $a$ and $a \times b$ are
  mutually orthogonal nonzero vectors, and in particular are not
  scalar multiples of each other. Thus, the cross product $a \times (a
  \times b)$ is nonzero.

  {\em Performance review}: $20$ out of $24$ people got this. $2$ each
  chose (D) and (E).

  {\em Historical note (last year)}: $6$ out of $23$ people got this
  correct. $11$ chose (E), $4$ chose (D), $1$ chose (B), $1$ left the
  question blank.

\item (*) Suppose $a$, $b$, $c$, and $d$ are vectors in $\R^3$, with
  $a \times b \ne 0$ and $c \times d \ne 0$. What does $(a \times b)
  \times (c \times d) = 0$ mean?

  \begin{enumerate}[(A)]
  \item Both the vectors $a$ and $b$ are perpendicular to both the
    vectors $c$ and $d$.
  \item $a$ and $b$ are perpendicular to each other and $c$ and $d$
    are perpendicular to each other.
  \item $a$ and $c$ are perpendicular to each other and $b$ and $d$
    are perpendicular to each other.
  \item The plane spanned by $a$ and $b$ is perpendicular to the plane
    spanned by $c$ and $d$.
  \item $a$, $b$, $c$, and $d$ are all coplanar.
  \end{enumerate}

  {\em Answer}: Option (E)

  {\em Explanation}: Since $a \times b$ and $c \times d$ are both
  nonzero, but their cross product is zero, we conclude that they are
  both scalar multiples of each other. In particular, they are in the
  same line. Further, we also obtain that $a$, $b$, $c$, and $d$ are
  individually nonzero.

  We know that $a$ and $b$ both lie in the plane orthogonal to $a
  \times b$. Similarly, $c$ and $d$ both lie in the plane orthogonal
  to $c \times d$. Because $a \times b$ and $c \times d$ are in the
  same line, we obtain that, in fact, the plane of $a$ and $b$ is the
  same as the plane of $c$ and $d$.

  {\em Performance review}: $21$ out of $24$ people got this. $2$
  chose (D), $1$ chose (A).

  {\em Historical note (last year)}: $9$ out of $23$ people got this
  correct. $6$ chose (D), $4$ chose (C), $3$ chose (B), $1$ chose (A).
\item (*) The {\em correlation} between two vectors in $\R^n$ is
  defined as the quotient of the dot product of the vectors by the
  product of their lengths. Suppose the correlation between vectors
  $a$ and $b$ is $x$ and the correlation between $b$ and $c$ is $y$,
  and suppose $x,y$ are both positive. What is the maximum possible
  value of the correlation between $a$ and $c$ given this information?
  {\em Hint: Geometrically if $\theta_{ab}$ is the angle between $a$
  and $b$, $\theta_{bc}$ is the angle between $b$ and $c$, and
  $\theta_{ac}$ is the angle between $a$ and $c$, then $|\theta_{ab} -
  \theta_{bc}| \le \theta_{ac} \le \theta_{ab} + \theta_{bc}$.}

  \begin{enumerate}[(A)]
  \item $xy$
  \item $\max \{ 1, xy \}$
  \item $\min \{ 1, xy \}$
  \item $xy + \sqrt{(1 - x^2)(1 - y^2)}$
  \item $xy - \sqrt{(1 - x^2)(1 - y^2)}$
  \end{enumerate}

  {\em Answer}: Option (D)

  {\em Explanation}: We have that $\theta_{ab} = \arccos x$ and
  $\theta_{bc} = \arccos y$. Thus, $x = \cos \theta_{ab}$ and $y =
  \cos \theta_{bc}$. Further, from the given data, both angles are
  acute angles.

  The maximum possible correlation between $a$ and $c$ occurs when the
  angle between these vectors is minimum, which happens when all three
  vectors are coplanar and $\theta_{ab}$ and $\theta_{bc}$ move in
  opposite directions, so $\theta_{ac} = |\theta_{ab} -
  \theta_{bc}|$. This gives:

  $$\cos \theta_{ac} = \cos|\theta_{ab} - \theta_{bc}| = \cos \theta_{ab}\cos\theta_{bc} + \sin \theta_{ab}\sin \theta_{bc}$$

  Using $\sin \theta_{ab} = \sqrt{1 - x^2}$ and $\sin \theta_{bc} =
  \sqrt{1 - y^2}$, we get the result indicated.

  {\em Further note}: The {\em expected} correlation between $a$ and
  $c$ is $xy$, and this occurs roughly if the correlation between $a$
  and $b$ is not correlated to the correlation between $b$ and $c$,
  which basically occurs when the plane of $a$ and $b$ is orthogonal
  to the plane of $b$ and $c$. The maximum correlation is in the
  situation described above. The minimum correlation is when $a$, $b$,
  and $c$ are coplanar and $\theta_{ab}$ and $\theta_{bc}$ go in the
  same direction. In this case, the correlation is $\cos(\theta_{ab} +
  \theta_{bc}) = xy - \sqrt{(1 - x^2)(1 - y^2)}$. Note that the
  minimum correlation case changes somewhat if $\theta_{ab} +
  \theta_{bc} > \pi$, because in that case, the minimum correlation is
  $-1$. But that case does not occur here because both angles are acute.
  
  {\em Performance review}: $3$ out of $24$ people got this. $14$
  chose (E), $4$ chose (B), $2$ chose (A), $1$ chose (C).

  {\em Historical note (last year)}: $5$ out of $23$ people got this
  correct. $7$ chose (E), $7$ chose (B), $2$ each chose (A) and (C).

\item If the correlation between nonzero vector $v$ and nonzero vector
  $w$ in $\R^n$ is $c$, then we say that the {\em proportion} of
  vector $w$ {\em explained by} vector $v$ is $c^2$. If $v_1, v_2,
  \dots, v_k$ are all pairwise orthogonal nonzero vectors, and $c_i$
  is the correlation between $v_i$ and $w$, then $c_1^2 + c_2^2 +
  \dots + c_k^2 \le 1$, with equality occurring if and only if $k =
  n$. (This is all a result of the Pythagorean theorem). If $k < n$,
  then $1 - (c_1^2 + c_2^2 + \dots + c_k^2)$ is the {\em unexplained
  proportion} of $w$.

  Suppose $w$ is the {\em variation of beauty} vector, $v_1$ is the
  {\em variation of genes} vector, and $v_2$ is the {\em variation of
  make-up} vector. Assume that $v_1$ and $v_2$ are orthogonal (i.e.,
  there is no correlation between genes and make-up choice). If the
  correlation between $v_1$ and $w$ is $0.6$ and the correlation
  between $v_2$ and $w$ is $0.3$, what proportion of the variation of
  beauty remains unexplained (i.e., is not explained by either genes
  or make-up)?

  \begin{enumerate}[(A)]
  \item $0.1$
  \item $0.19$
  \item $0.55$
  \item $0.74$
  \item $1$
  \end{enumerate}

  {\em Answer}: Option (C)

  {\em Explanation}: We use the formula $1 - (0.6)^2 - (0.3)^2 = 1 -
  (0.36) - (0.09) = 0.55$.

  In other words, genes explain $36\%$ of the variance in beauty,
  make-up explains $9\%$ of the variance, and the unexplained variance
  is $55\%$.

  {\em Note}: The correlation values are bogus, this isn't a real
  world problem.

  {\em Performance review}: All $24$ people got this.

  {\em Historical note (last year)}: $17$ out of $23$ people got this
  correct. $4$ chose (A), $1$ chose (B), and $1$ left the question
  blank.
\end{enumerate}
\end{document}