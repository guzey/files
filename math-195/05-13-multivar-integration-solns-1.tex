\documentclass[10pt]{amsart}

%Packages in use
\usepackage{fullpage, hyperref, vipul, enumerate}

%Title details
\title{Class quiz solutions: May 13: Multi-variable integration}
\author{Math 195, Section 59 (Vipul Naik)}
%List of new commands

\begin{document}
\maketitle

\section{Performance review}

$19$ people took this quiz. The score distribution was as follows:

\begin{itemize}
\item Score of $0$: $5$ people
\item Score of $1$: $8$ people
\item Score of $2$: $5$ people
\item Score of $3$: $1$ person
\end{itemize}

The question wise answers were as follows:

\begin{enumerate}
\item (To be done later)
\item (To be done later)
\item (To be done later)
\item (To be done later)
\item (To be done later)
\item Option (C): $12$ people. {\em Please review this solution!}
\item Option (C): $8$ people. {\em Please review this solution!}
\item Option (D): $1$ person. {\em Please review this solution!}
\end{enumerate}
\section{Solutions}
\begin{enumerate}

\item (To be done later)
\item (To be done later)
\item (To be done later)
\item (To be done later)
\item (To be done later)
\item Suppose $f$ is a homogeneous polynomial of degree $d >
  0$. Define $g$ as the following function on positive reals: $g(a)$
  is the double integral of $f$ on the square $[0,a] \times
  [0,a]$. Assuming that $g(a)$ is not identically the zero function,
  which of these best describes the nature of $g(a)$?

  \begin{enumerate}[(A)]
  \item A constant times $a^d$
  \item A constant times $a^{d + 1}$
  \item A constant times $a^{d + 2}$
  \item A constant times $a^{2d + 1}$
  \item A constant times $a^{2d + 2}$
  \end{enumerate}

  {\em Answer}: Option (C)

  {\em Explanation}: Each monomial is of the form a constant times
  $x^py^q$ where $p + q = d$. Integrating this as a multiplicatively
  separable function gives $x^{p+1}y^{q+1}$ times a
  constant. Evaluating between limits gives $a^{d + 2}$ times a
  constant. This is the form of the double integral of each monomial,
  and hence the double integral of the sum is also of the same form.

  {\em Performance review}: $12$ people got this correct. $3$ people
  chose (D), $2$ each chose (A) and (E).
\item Suppose $g(x,y)$ and $G(x,y)$ are continuous functions of two
  variables and $G_{xy} = g$. How can the double integral $\int_s^t
  \int_u^v g(x,y) \, dy \, dx$ be described in terms of the values of
  $G$?

  \begin{enumerate}[(A)]
  \item $G(v,t) + G(u,s) - G(u,t) - G(v,s)$
  \item $G(v,t) - G(v,s) + G(u,t) - G(u,s)$
  \item $G(t,v) + G(s,u) - G(t,u) - G(s,v)$
  \item $G(t,v) - G(s,v) + G(t,u) - G(s,u)$
  \item $G(t,v) + G(v,t) - G(s,u) - G(u,s)$
  \end{enumerate}

  {\em Answer}: Option (C)

  {\em Explanation}: Note that the integration is over $[s,t] \times
  [u,v]$, i.e., the rectangular region with corner points $(s,u)$,
  $(s,v)$, $(t,u)$, and $(t,v)$. Recall that for the double integral,
  we put positive signs on the two extreme points (top right and
  bottom left) and negative signs on the other two points (top left
  and bottom right). See more in the lecture notes.

  {\em Performance review}: $8$ people got this correct. $6$ people
  chose (D), $2$ each chose (A) and (E), $1$ chose (B).

\item Suppose $f$ is an elementarily integrable function, but $f(x^k)$
  is not elementarily integrable for any integer $k > 1$ (examples are
  $\sin$, $\exp$, $\cos$). For which of the following types of regions
  $D$ are we {\em guaranteed to be able} to compute, in elementary
  function terms, the double integral $\int \int_D f(x^2) \, dA$ over
  the region (note that $f$ is just a function of $x$, but we treat it
  as a function of two variables)?

  \begin{enumerate}[(A)]
  \item A rectangle with vertices $(0,0)$, $(0,b)$, $(a,0)$, and
    $(a,b)$, with $a,b > 0$.
  \item A triangle with vertices $(0,0)$, $(0,b)$, $(a,0)$, with $a, b
    > 0$.
  \item A triangle with vertices $(0,0)$, $(0,b)$, $(a,b)$, with $a, b
    > 0$.
  \item A triangle with vertices $(0,0)$, $(a,0)$, $(a,b)$, with $a, b
    > 0$.
  \item All of the above
  \end{enumerate}

  {\em Answer}: Option (D)

  {\em Explanation}: For such a triangle, we integrate on $y$ inner
  and $x$ outer. For a fixed value of $x$, the $y$-value ranges from
  $0$ to $bx/a$, so the integral becomes:

  $$\int_0^a \int_0^{bx/a} f(x^2) \, dy \, dx$$

  $\sin(x^2)$ pulls out of the inner integral and the inner integral
  just gives $(bx/a)$, so we get:

  $$\int_0^a \frac{b}{a} xf(x^2) \, dx$$

  This can be done by the substitution $u = x^2$ and the knowledge
  that $f$ is elementarily integrable.

  All the other integrations stumble because they require knowledge of
  an antiderivative of $f(x^2)$.

  {\em Please review the similar example done in class to better
    understand this issue}.

  {\em Performance review}: $1$ person got
  this correct. $8$ chose (E), $5$ chose (A), $3$ chose (B), $2$ chose
  (C).
\end{enumerate}

\end{document}
