\documentclass[10pt]{amsart}

%Packages in use
\usepackage{fullpage, hyperref, vipul, enumerate}

%Title details
\title{Take-home class quiz: due Monday February 18: Partial derivatives}
\author{Math 195, Section 59 (Vipul Naik)}
%List of new commands

\begin{document}
\maketitle

Your name (print clearly in capital letters): $\underline{\qquad\qquad\qquad\qquad\qquad\qquad\qquad\qquad\qquad\qquad}$

{\bf YOU ARE FREE TO DISCUSS ALL QUESTIONS, BUT PLEASE ONLY ENTER
FINAL ANSWER OPTIONS THAT YOU PERSONALLY ENDORSE. DO NOT ENGAGE IN
GROUPTHINK.}

\begin{enumerate}

\item For this and the next question, consider the function on $\R^2$
  given as:

  $$f(x,y) := \left\lbrace \begin{array}{rl} 1, & x \text{ rational  or } y \text { rational }\\0, & x \text{ and } y \text{ both irrational } \\\end{array}\right.$$

  What can we say about the subset $S$ of $\R^2$ defined as the set of
  points where $f_x$ is defined?

  \begin{enumerate}[(A)]
  \item $S$ is the set of points for which at least one coordinate is rational.
  \item $S$ is the set of points for which both coordinates are rational.
  \item $S$ is the set of points for which the $x$-coordinate is rational.
  \item $S$ is the set of points for which the $y$-coordinate is rational.
  \item $S$ is the set of points for which at least one coordinate is
    irrational.
  \end{enumerate}

  \vspace{0.05in}
  Your answer: $\underline{\qquad\qquad\qquad\qquad\qquad\qquad\qquad}$
  \vspace{0.05in}

\item With $f$ as in the previous question, what is the subset $T$ of
  $\R^2$ at which the second-order mixed partial derivative $f_{xy}$
  is defined?
  
  \begin{enumerate}[(A)]
  \item $T$ is the empty subset.
  \item $T$ is the set of points for which both coordinates are rational.
  \item $T$ is the set of points for which the $x$-coordinate is rational.
  \item $T$ is the set of points for which the $y$-coordinate is rational.
  \item $T$ is the set of points for which both coordinates are
    irrational.
  \end{enumerate}

  \vspace{0.05in}
  Your answer: $\underline{\qquad\qquad\qquad\qquad\qquad\qquad\qquad}$
  \vspace{0.05in}

\item For this and the next three questions, consider the function on $\R^2$
  given as:

  $$g(x,y) := \left\lbrace \begin{array}{rl} 1, & x \text{ rational}\\0, & x \text{ irrational } \\\end{array}\right.$$

  What can we say about the subset $U$ of $\R^2$ defined as the set of
  points where $g_x$ is defined?

  \begin{enumerate}[(A)]
  \item $U$ is the empty subset.
  \item $U$ is the set of points for which both coordinates are rational.
  \item $U$ is the set of points for which the $x$-coordinate is rational.
  \item $U$ is the set of points for which the $y$-coordinate is rational.
  \item $U$ is the whole plane $\R^2$.
  \end{enumerate}

  \vspace{0.05in}
  Your answer: $\underline{\qquad\qquad\qquad\qquad\qquad\qquad\qquad}$
  \vspace{0.05in}

\item With $g$ as in the preceding question, what can we say about the
  subset $V$ of $\R^2$ defined as the set of points where $g_y$ is
  defined?

  \begin{enumerate}[(A)]
  \item $V$ is the empty subset.
  \item $V$ is the set of points for which both coordinates are rational.
  \item $V$ is the set of points for which the $x$-coordinate is rational.
  \item $V$ is the set of points for which the $y$-coordinate is rational.
  \item $V$ is the whole plane $\R^2$.
  \end{enumerate}

  \vspace{0.05in}
  Your answer: $\underline{\qquad\qquad\qquad\qquad\qquad\qquad\qquad}$
  \vspace{0.05in}

\item With $g$ as in the preceding question, what can we say about the
  subset $W$ of $\R^2$ defined as the set of points where $g_{xy}$ is
  defined?

  \begin{enumerate}[(A)]
  \item $W$ is the empty subset.
  \item $W$ is the set of points for which both coordinates are rational.
  \item $W$ is the set of points for which the $x$-coordinate is rational.
  \item $W$ is the set of points for which the $y$-coordinate is rational.
  \item $W$ is the whole plane $\R^2$.
  \end{enumerate}

  \vspace{0.05in}
  Your answer: $\underline{\qquad\qquad\qquad\qquad\qquad\qquad\qquad}$
  \vspace{0.05in}

\item With $g$ as in the preceding question, what can we say about the
  subset $X$ of $\R^2$ defined as the set of points where $g_{yx}$ is
  defined?

  \begin{enumerate}[(A)]
  \item $X$ is the empty subset.
  \item $X$ is the set of points for which both coordinates are rational.
  \item $X$ is the set of points for which the $x$-coordinate is rational.
  \item $X$ is the set of points for which the $y$-coordinate is rational.
  \item $X$ is the whole plane $\R^2$.
  \end{enumerate}

  \vspace{0.05in}
  Your answer: $\underline{\qquad\qquad\qquad\qquad\qquad\qquad\qquad}$
  \vspace{0.05in}

\item For this and the next two questions, consider the function on
  $\R^2$ given as:

  $$h(x,y) := \left\lbrace \begin{array}{rl} 1, & x \text{ an integer or } y \text{ an integer }\\0, & x \text{ not an integer and } y \text{ not an integer } \\\end{array}\right.$$

  What can we say about the subset $A$ of $\R^2$ defined as the set of
  points where $h_{xy}$ is defined?

  \begin{enumerate}[(A)]
  \item $A$ is the empty set.
  \item $A$ is the set of points whose $x$-coordinate is an integer.
  \item $A$ is the set of points whose $x$-coordinate is not an integer.
  \item $A$ is the set of points whose $y$-coordinate is an integer.
  \item $A$ is the set of points whose $y$-coordinate is not an integer.
  \end{enumerate}

  \vspace{0.05in}
  Your answer: $\underline{\qquad\qquad\qquad\qquad\qquad\qquad\qquad}$
  \vspace{0.05in}

\item With $h$ as defined in the previous question, what can we say
  about the subset $B$ of $\R^2$ defined as the set of points where
  $h_x$ is defined but $h_{xy}$ is not defined?

  \begin{enumerate}[(A)]
  \item $B$ is the empty set.
  \item $B$ is the set of points for which both coordinates are integers.
  \item $B$ is the set of points for which both coordinates are non-integers.
  \item $B$ is the set of points for which at least one coordinate is an integer.
  \item $B$ is the set of points for which at least one coordinate is
    a non-integer.
  \end{enumerate}

  \vspace{0.05in}
  Your answer: $\underline{\qquad\qquad\qquad\qquad\qquad\qquad\qquad}$
  \vspace{0.05in}

\item With $h$ as defined in the previous question, what can we say
  about the subset $C$ of $\R^2$ defined as the set of points where
  both $h_{xy}$ and $h_{yx}$ are defined?

  \begin{enumerate}[(A)]
  \item $C$ is the empty set.
  \item $C$ is the set of points for which both coordinates are integers.
  \item $C$ is the set of points for which both coordinates are non-integers.
  \item $C$ is the set of points for which at least one coordinate is an integer.
  \item $C$ is the set of points for which at least one coordinate is
    a non-integer.
  \end{enumerate}

  \vspace{0.05in}
  Your answer: $\underline{\qquad\qquad\qquad\qquad\qquad\qquad\qquad}$
  \vspace{0.05in}

\item Students training for an examination can spend money either on
  purchasing textbooks or on private tuitions. A student's expected
  performance on the examination is a function of the money the
  student spends on textbooks and on tuition (viewed as separate
  variables). Two researchers want to consider the question of whether
  increased expenditure on textbooks leads to improved performance on
  the examination, and if so, by how much.

  One researcher decides to measure the increase in the examination
  score for a marginal increase in textbook expenditure {\em holding
  constant the expenditure on tuitions}, arguing that in order to
  determine the effect of changes in textbook expenditures, the other
  expenditures need to be kept constant.

  The other researcher believes that since the student has a limited
  budget, it would be more realistic to measure the increase in the
  examination score for a marginal increase in textbook expenditure
  {\em holding constant the total expenditure on both textbook and
  tuitions}. This is because the student is likely to allocate money
  away from tuition expenditures in order to spend money on textbooks.

  Which of the following best describes what's happening?

  \begin{enumerate}[(A)]
  \item Both researchers are effectively computing the same quantity.
  \item The two quantities that the researchers are computing have a
    simple linear relationship, i.e., their sum or difference is a
    constant.
  \item The two quantities that the researchers are computing are
    meaningfully different and there is a relationship between them
    but that relationship involves other partial derivatives.
  \end{enumerate}

  \vspace{0.05in}
  Your answer: $\underline{\qquad\qquad\qquad\qquad\qquad\qquad\qquad}$
  \vspace{0.05in}

\item $F$ is an everywhere twice differentiable function of two
  variables $x$ and $y$. Which of the following captures the manner in
  which the inputs $x$ and $y$ {\em interact} with each other in the
  description of $F$?

  \begin{enumerate}[(A)]
  \item The difference $F_x - F_y$
  \item The quotient $F_x/F_y$.
  \item The product $F_xF_y$.
  \item The product $F_{xx}F_{yy}$.
  \item The mixed partial $F_{xy}$
  \end{enumerate}

  \vspace{0.05in}
  Your answer: $\underline{\qquad\qquad\qquad\qquad\qquad\qquad\qquad}$
  \vspace{0.05in}

\item $F$ is a function of two variables $x$ and $y$ such that both
  $F_x$ and $F_y$ exist. Which of the following is generically true?

  \begin{enumerate}[(A)]
  \item In general, $F_x$ depends only on $x$ (i.e., it is independent
    of $y$) and $F_y$ depends only on $y$. An exception is if $F$ is
    multiplicatively separable.
  \item In general, $F_x$ depends only on $y$ (i.e., it is independent
    of $x$) and $F_y$ depends only on $x$ (i.e., it is independent of
    $y$). An exception is if $F$ is multiplicatively separable.
  \item In general, both $F_x$ and $F_y$ could each depend on both $x$
    and $y$. An exception is if $F$ is additively separable, in which
    case $F_x$ depends only on $y$ and $F_y$ depends only on $x$.
  \item In general, both $F_x$ and $F_y$ could each depend on both $x$
    and $y$. An exception is if $F$ is additively separable, in which
    case $F_x$ depends only on $x$ and $F_y$ depends only on $y$.
  \item In general, either both $F_x$ and $F_y$ depend only on $x$ or
    both $F_x$ and $F_y$ depend only on $y$.
  \end{enumerate}

  \vspace{0.05in}
  Your answer: $\underline{\qquad\qquad\qquad\qquad\qquad\qquad\qquad}$
  \vspace{0.05in}

\item Consider a production function $f(L,K,T)$ of three inputs $L$
  (labor expenditure), $K$ (capital expenditure), and $T$ (technology
  expenditure). Suppose all mixed partials of $f$ with respect to $L$,
  $K$, and $T$ are continuous. Suppose we have the following signs of
  partial derivatives: $\partial f/\partial L > 0$, $\partial
  f/\partial K > 0$, $\partial^2f/(\partial L \partial K) < 0$, and
  $\partial^3f/(\partial L\partial K \partial T) > 0$. What does this
  mean?

  \begin{enumerate}[(A)]
  \item Increasing labor increases production, increasing capital
    increases production, and labor and capital substitute for each
    other to some extent. Increasing the expenditure on technology
    increases the degree to which labor and capital substitute for
    each other.
  \item Increasing labor increases production, increasing capital
    increases production, and labor and capital substitute for each
    other to some extent. Increasing the expenditure on technology
    decreases the degree to which labor and capital substitute for
    each other, i.e., with more technology investment, labor and
    capital become more complementary.
  \item Increasing labor increases production, increasing capital
    increases production, and labor and capital complement each other
    to some extent. Increasing the expenditure on technology increases
    the degree to which labor and capital complement for each other.
  \item Increasing labor increases production, increasing capital
    increases production, and labor and capital complement each other
    to some extent. Increasing the expenditure on technology decreases
    the degree to which labor and capital complement for each other.
  \item Increasing labor or capital decreases production.
  \end{enumerate}

  \vspace{0.05in}
  Your answer: $\underline{\qquad\qquad\qquad\qquad\qquad\qquad\qquad}$
  \vspace{0.05in}

\item Analysis of usage of an online social network finds that the
  total time spent by people on the social network is $P^{1.3}L^{0.5}$
  where $P$ is the total number of people on the network and $L$ is a
  number of processors used at the social network's server
  facility. Which of these is true?

  \begin{enumerate}[(A)]
  \item Increasing returns both on persons and on processors: every
    new person joining the network increases the average time spent
    {\em per person} (and not just the total time), and every new
    processor added to the server facility increases the average time
    spent per processor.
  \item Constant returns on persons, increasing returns on processors
  \item Constant returns on persons, decreasing returns on processors
  \item Increasing returns on persons, decreasing returns on processors
  \item Decreasing returns on persons, increasing returns on processors
  \end{enumerate}

  \vspace{0.05in}
  Your answer: $\underline{\qquad\qquad\qquad\qquad\qquad\qquad\qquad}$
  \vspace{0.05in}

\item {\em Not a calculus question, but has deep calculus
  interpretations -- it is basically measuring the derivative of the
  $1/x$ function with respect to $x$}: A person travels fifty miles
  every day by car and the travel distance is fixed. The price of
  gasoline, which she uses to fuel her car, is also fixed. Which of
  the following increases in fuel efficiency result in the maximum
  amount of savings for her?

  \begin{enumerate}[(A)]
  \item From $11$ to $12$ miles per gallon
  \item From $12$ to $14$ miles per gallon
  \item From $20$ to $25$ miles per gallon
  \item From $36$ to $54$ miles per gallon
  \item From $50$ to $100$ miles per gallon
  \end{enumerate}

  \vspace{0.05in}
  Your answer: $\underline{\qquad\qquad\qquad\qquad\qquad\qquad\qquad}$
  \vspace{0.05in}

\item For which of the following production functions $f(L,K)$ of
  labor and capital is it true that labor and capital can be
  complementary for some choices of $(L,K)$, and substitutes for
  others? In other words, for which of these are labor and capital
  neither globally complements nor globally substitutes? Assume the
  domain $L > 0, K > 0$.

  \begin{enumerate}[(A)]
  \item $L^2 + LK + K^2$
  \item $L^2 - LK + K^2$
  \item $L^3 + L^2K + LK^2 + K^3$
  \item $L^3 + L^2K - LK^2 + K^3$
  \item $L^3 - L^2K - LK^2 + K^3$
  \end{enumerate}

  \vspace{0.05in}
  Your answer: $\underline{\qquad\qquad\qquad\qquad\qquad\qquad\qquad}$
  \vspace{0.05in}

\item Consider the following Leontief-like production function $f(L,K)
  = (\min \{ L, K \})^2$. Assume the domain $L > 0$, $K > 0$. What is
  the nature of returns and complementarity here?

  \begin{enumerate}[(A)]
  \item Positive increasing returns on the smaller of the inputs,
    positive constant returns on the larger of the inputs
  \item Positive constant returns of the smaller of the inputs,
    positive increasing returns on the larger of the inputs
  \item Zero returns on the smaller of the inputs, positive constant
    returns on the larger of the inputs
  \item Positive decreasing returns on the smaller of the inputs, zero
    returns on the larger of the inputs
  \item Positive increasing returns on the smaller of the inputs, zero
    returns on the larger of the inputs
  \end{enumerate}

  \vspace{0.05in}
  Your answer: $\underline{\qquad\qquad\qquad\qquad\qquad\qquad\qquad}$
  \vspace{0.05in}

\end{enumerate}
\end{document}