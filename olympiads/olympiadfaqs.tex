\documentclass[a4paper]{amsart}

%Packages in use
\usepackage{fullpage, hyperref, vipul}

%Title details
\title{Frequently Asked Questions on Olympiads}
\author{Vipul Naik}
\thanks{\copyright Vipul Naik, B.Sc. (Hons) Math and C.S., Chennai Mathematical Institute}

%List of new commands

\makeindex

\begin{document}
\maketitle
%\tableofcontents

\begin{abstract}
  Here, I list some questions that I have been asked regarding
  Mathematical Olympiads in India and preparing for them. I have tried
  to both present all available facts and given my opinions in response
  to some of the questions.
\end{abstract}

\section{The what of Olympiads}

\subsection{The standard path}

Roughly speaking, there are four stages on the math Olympiad route in India:

\begin{enumerate}

\item The {\bf Regional Mathematical Olympiad}: There are 15 regions,
  each of which conducts the Olympiad in November-December. Students
  of standards 11 and below can give the examination in open
  competition.  For students of Standard 12, there is a limit on the
  number of people who can qualify for the next stage. A total of
  about 30 students qualify from every region. Further details are
  available at:

  \url{http://en.wikipedia.org/wiki/Regional_Mathematical_Olympiad}

\item The {\bf Indian National Mathematical Olympiad}: The
  approximately 500 people who qualify the Regional Mathematical
  Olympiad from across all regions write this examination in the first
  week of February. AAround 30 are selected to the next stage. Again,
  there is bound on the number of selected students from 12th.
  Further details are available at:
  
  \url{http://en.wikipedia.org/wiki/Indian_National_Mathematical_Olympiad}

\item The {\bf International Mathematical Olympiad Training Camp}: The
  students who qualify the Indian National Mathematical Olympiad in
  the current year, as well as people who qualified in previous years,
  are invited to this camp. Here, through a series of five selection
  tests, the six-member team for the International Mathematical Olympiad is
  selected.  Further details are available at:
  
  \url{http://en.wikipedia.org/wiki/International_Mathematical_Olympiad_Training_Camp}

\item The {\bf International Mathematical Olympiad}: This is an
  international competition with over 85 countries participating, and
  each country can send at most 6 members. Contestants have to write 6
  problems on 2 days, with 3 problems in 4.5 hours each day. Further
  details are available at:
  
   \url{http://en.wikipedia.org/wiki/International_Mathematical_Olympiad}
\end{enumerate}

\subsection{Other Olympiads}

Different regions conduct their own Olympiads. Each region has the
option of either using an RMO given to it by the center, or conducting
its own examination as RMO. In addition, the regions may conduct
further Olympiads to spread awareness and given problem-solving
practice to students.

Private bodies are also often involved in conducting Olympiads. While
these Olympiads have their own value in spreading awareness and giving
students exposure to Olympiad problems, they should {\em not} be confused
as precursors to the standard path for Olympiads.

\section{The usefulness of Olympiad preparation}

\subsection{Relevance to a career in mathematics}

Mathematicians have often attributed great significance to their
Olympiad years as the time which got them started on their
mathematical lives. As such, Olympiad mathematics looks very different
from real mathematics: in Olympiad mathematics, one typically has to
solve a problem in a short time (1-2 hours) whereas real mathematics
requires days, months and years to struggle with problems. Real mathematics
involves a deep familiarity and an ability to manipulate mathematical
constructs.

Nonetheless, Olympiad mathematics is at least {\em partly} like real
mathematics -- the time, focus and attention needed for an {\em
  individual} problem in Olympiad mathematics is far more than in the
routine school exercises. Further, people sincerely preparing for
Olympiads often get to appreciate the importance of {\em abstracting
  the common principles} and {\em converting tricks to methods and
  methods to theories}.

Apart from this, many of the subjects covered in Olympiad mathematics
prepare a foundation for higher mathematics. For instance, number
theory for Olympiads helps not only in higher number theory, but also
in abstract algebra, particularly group theory. The study of
polynomials also lays the foundation for ring theory and commutative
algebra.  While Euclidean geometry does not play a direct role in
higher mathematics, it does help equip one better for studying both
algebraic and differential geometry.

\subsection{Conflict with IIT-JEE preparation}

I have most often been asked the following question (or some variant
thereof): Does Olympiad preparation help with IIT-JEE preparation?

The standard (perhaps somewhat cliched) answer is that Olympiad
problems help cultivate a kind of analytical thinking that will be
useful in any examination, be it IIT-JEE or a school examination.

While this answer is correct, I would like to point out some very
important differences between the Olympiad situation and the IIT-JEE
situation:

\begin{itemize}

\item Olympiad problems require a lot of thought {\em per problem}.
  The typical Olympiad paper comprises anywhere between 3 and 6
  problems to be solved in 3 hours or more. Clearly, each problem
  requires the development of a train of thought specific to that problem.

  IIT-JEE problems on the other hand, require {\em speed} and {\em
    efficiency}.  Even if the problems require thought, it is expected
  that the person writing the examination has put in the thought {\em already}
  before coming to the examination.

\item Olympiad problems are completely unpredictable. In general,
  those setting the Olympiad, while they do have some clue as to the
  difficulty level of the problem, do not understand clearly how each
  contestnat will participate. Ther performance of a participant is
  much more contingent on uncontrollable factors than in a standard examination
  like IIT-JEE.

  Further, those setting the Olympiad problems are not usually trying
  to be either fair or comprehensive. On the other hand, those setting
  a paper like the IIT-JEE have a responsibility to ensure a fair and
  comprehensive distribution of questions across the syllabus, across
  difficulty levels and across question types.

\item There is as yet no systematic tool for ``cracking'' the
  Olympiads, while there is a plethora of tools (with different
  degrees of effectiveness) for the IIT-JEE.

\end{itemize}

In my view, Olympiad preparation does help make one faster in thinking
and this {\em might} positively impact IIT-JEE preparation. However,
the extent of this impact may be too indirect and it is likely that
the time expended hoping for such indirect gains could have been
better utilized in direct preparation. Further, too much synchronization
with Olympiad mathematics may in fact prove detrimental to the mindset
needed for IIT-JEE, which requires quick and sharp responses.

In one sense, though, Olympiads are very important and that is that
they help one appreciate and enjoy the connections in mathematics.
This is useful in making preparation for examinations like IIT-JEE
much more pleasant, and is also helpful throughout later life in the
study of mathematics and related subjects.

\section{The usefulness of Olympiad selection}

\subsection{For undergraduate admissions (math)}

Within India, there are two leading institutes for mathematics
education:

\begin{enumerate}

\item {\bf Chennai Mathematical Institute}
  \footnote{\url{http://www.cmi.ac.in}}: This gives direct admission
  for its B.Sc. (Hons) Mathematics programme to students who
  have qualified the Indian National Mathematical Olympiad.

\item {\bf Indian Statistical Institute, Bangalore}
  \footnote{\url{http://www.isibang.ac.in}}: This does not have any special
  policy with regard to Olympiads.

\end{enumerate}

For applying to foreign universities, praticularly those in the United
States, good performance in the Olympiads at the international level
can be a major boost to one's application. This is partly because the worth
of the International Mathematical Olympiad is recognized and appreciated
even by people from other countries.

Even going for the International Mathematical Olympiad Training Camp
is a booster, though much less so. Some of my acquaintances at the IMO
Training Camp applied to and were successfully admitted to
undergraduate programmes in universities in the United States.

\subsection{For postgraduate admissions  (math/C.S.)}

The main role of good performance in Olympiads for postgraduate
admissions is that it provides an indication of talent at a young age.
However, this in itself does not guarantee admission. A Graduate School,
while considering a student's application for admission,
is looking at the student's {\em just completed} stage of life (undergraduate
studies) and the achievements, initiatives and background acquired in this
most recent stage.

Nonetheless, it is my view that {\em good performance in Olympiads} is a far
more valuable criterion for future admissions than marks in high school
(tenth standard and twelfth standard).


\section{The how of Olympiad preparation}

\subsection{Full details}

Full details are available at:

\url{http://www.cmi.ac.in/~vipul/olymp_resources/preparingforolympiads.pdf}

Here, I shall address some specific questions that I have often
been asked.

\subsection{When to begin}

When it comes to Olympiad preparation, it is {\em never too early} and it is
{\em never too late}.

It is true that starting off early (in standards 7th and 8th) on the
Olympiad route provides a distinct advantage because it is possible
to {\em systematically} cover the Olympiad texts in a relaxed manner
and to get a thorough foundation. However, people starting early
need to keep in mind the follow:

\begin{itemize}

\item They should not expect the same rate of progress as people
  starting later, because their background (particularly in degree of
  comfort with algebraic notation and symbols) is likely to be poorer.
  In some areas, they may progress just as fast as older people; in
  some areas, a younger student may progress faster as well. However,
  there may be areas where older students are progressing faster.
  This should not be a cause for discouragement -- the younger student
  has more time to grasp the same concept, so being a little slower is
  not a huge problem.

\item They should not try too hard to optimize the immediate regional
  levels at the cost of good foundations. The performance with respect
  to others at the regional level, while important insofar as it is
  necessary for selection to the next level, should not be taken
  as the sole measure of one's potential for higher levels.

  Regional coordinators typically set the regional examinations at a
  level where people with only a school background can solve the
  problems, and thus, at the regional examination, the speed at
  solving school-level problems plays a role in performance. The
  national level contest and the selection tests, on the other hand,
  are more turned to Olympiad material.

  Thus, a student who is keen for a grand success in the long-term should
  focus on developing good foundations.

\end{itemize}

\subsection{Is tutoring necessary?}

This is a difficult question. In most places, tutors for Olympiads are not
readily available, and so the question does not even arise. However,
there are some places where both organizations and individuals
are involved with training for the Olympiads.

Attending formal tutoring or coaching for the Olympiads could have the
following advantages:

\begin{itemize}

\item General guidance on the directions in which to work; an overall
  framework within which to direct one's efforts

\item Interaction with other students who are also keen on preparing
  for the Olympiads

\item Easier understanding of concepts; doubt resolution
  in specific areas

\end{itemize}

On the other hand, it needs to be remembered that Olympiad tutoring (as far
as it exists now) is very different from IIT-JEE coaching or coaching 
for school examinations. Most coaching centers for school examinations have
a fully chalked, tried and tested programme which has been running for several
years and students are encouraged to faithfully adhere to that programme
for best results. In the case of Olympiad coaching, the coaching/tutoring
only provides guidance/motivation/peer environment and does {\em not}
lay out a full-fledged programme and schedule.

Further, there is no implicit guarantee of being ``tried-and-tested''
because the vagaries in Olympiad mathematics are inherently far greater.

\subsection{Does school/IIT-JEE coaching play a role?}

School environment is helpful in spreading basic awareness about the
Olympiads. Some schools from where students have gone for Olympiads in
the past have greatere awareness about the Olympiads and this may
encourage more students. Since the schools are also more aware, they
may also be able to get the student in contact with Olympiad
instructors etc.

However, it is unlikely that the quality of teaching in school will
help in Olympiad preparation. The kind of teaching orientation needed
to equip students for Olympiad mathematics, is, as far as I am aware,
not provided by any school.

IIT-JEE coaching institutes, even if they claim to {\em also} prepare
for Olympiads, are likely to be of very little use for Olympiad
preparation.  There are two ways in which they may help: 

\begin{enumerate}
\item They may help with certain topics and certain ways of
  thinking and formulating problems. This, however, is limited to the
  narrow intersection of Olympiad mathematics and IIT-JEE mathematics.

\item They may provide a peer environment of people who have an ability
  and interest in thinking about challenging mathematical problems
\end{enumerate}

\section{Olympiads in other subjects}

\subsection{Physics, chemistry Olympiads}

The mathematics Olympiads differs in many important respects from the
Olympiads in physics and chemistry. Some of the points of difference:

\begin{enumerate}

\item The syllabus for the Physics and Chemistry Olympiad is very
  closely related to the syllabus for school and engineering entrance
  examinations. For the Mathematics Olympiad, on the other hand, there
  is a whole range of topics that is not covered at all in school
  or engineering entrance examinations.

  There is in fact a heavy intersection between people doing well
  on the IIT-JEE and people doing well in the Physics and Chemistry Olympiad.
  Most of the team members are in the top 200-300 in the IIT-JEE.

\item Most of the students who qualify the Regional and National
  levels for the Physics Olympiad are people finishing 12th class.
  Hence, most people get only one opportunity at the International
  Physics Olympiad Training Camp or International Chemistry Olympiad
  Training Camp.

  Typically the camps contain at most 2-3 people from 11th class
  and so far, nobody from 11th class has made it to the team.

\item There is no formal or informal training for the physics and chemistry
  Olympiads outside of the Training Camps. Even within the Training Camps,
  the focus is more on testing and brushing up than on teaching.

\end{enumerate}

I am not too knowledgable about the Biology Olympiad, though I believe
that it is similar to the Physics and Chemistry Olympiads

\subsection{Informatics Olympiad}

The Olympiad in Informatics is somewhat more similar to the mathematics
Olympiad, in the following respects:

\begin{enumerate}

\item In both, the syllabus for the Olympiads is very different from
  what is usually covered in school. In the case of the Informatics
  Olympiad, there is no parallel with school coverage because
  programming is not even a compulsory part of school curricula.

\item In both, students qualify at all years. There have been students
  who have qualified for the Training Camp and even for the international
  Olympiad in ninth standard.

\item In both, the intersection with students doing well in entrance
  examinations such as IIT-JEE is extremely low. 

\end{enumerate}





\printindex

\end{document}
