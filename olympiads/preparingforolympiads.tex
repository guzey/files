\documentclass[a4paper]{amsart}

%Packages in use
\usepackage{fullpage}
%\usepackage{setspace} %To provide double spacing
%\usepackage{float,graphicx} %To allow images
%\usepackage{hyperref} %To make links clickable -- only for PDF
%\usepackage{showkeys} %To show the list of labels
%\usepackage{index} %To construct an advanced index

%Title details
\title{Preparing for the Olympiads}
\author{Vipul Naik}
\thanks{\copyright Vipul Naik, B.Sc. (Hons) Math, 2nd Year, Chennai Mathematical Institute}

%List of new commands

\makeindex

\begin{document}
\maketitle
%\tableofcontents

\begin{abstract}
  Here, I discuss various possible resources students can use to prepare for
  mathematical Olympiads. I hope this helps interested Olympiad aspirants.
\end{abstract}

\section{What is this article about?}

\subsection{Overall organization}

Very few students in India today are interested in studying Olympiad related mathematics and solving Olympiad problems.
Even students interested by nature
refrain from Olympiads for fear of distracting them from their main goals -- such as getting
good results in
engineering or medical entrance examinations. 
This makes it difficult for individuals like you who are keen on Olympiad mathematics but fail to obtain a supportive environment.

If that is the case, don't despair!
Remember that, {\em across the whole world}, many people live and breathe mathematics. You have opportunities every day to network
with others interested in Olympiad mathematics. There is a huge demand for Olympiad material and a huge supply, and this article
is about {\em where to look for it}.

Three sources are:

\begin{itemize}

\item The Internet

\item Books

\item Institutes, people and personal guidance

\end{itemize}

Caution! Much of the content is based on {\em my personal opinions}. It reflects {\em what worked for me}. It also includes
things that I missed
out on, which I felt {\em would have worked for me}. 
In order to give you a fair and unbiased picture, I have included references
wherever possible. For instance, for all the books I recommend, I have provided links to the corresponding Amazon page
which contains reviews of those books.

But now, here's some good news! In this new version, I have with me Anupam Prakash, who has been to the IMO Training Camp three times
and went with me to the IMO in 2004 to Greece. Anupam has a slightly different perspective and has added his own views of the books
I originally recommended.

In a nutshell, Anupam's advice is: ``Pick on the areas that interest you, find the problem that fascinates you. Then work on it.
Devote your time and energy to it, feel deeply involved with it.'' I wish I could have put it so nicely!

\subsection{Questions regarding the Olympiad}

Below is a list of questions that people {\em deeply interested in mathematics} and {\em keen on the Olympiads}
have asked me. Some of these questions might be on {\em your} mind as well.

On the {\em how} part of Olympiad preparation:

\begin{itemize}

\item How does one prepare for the Olympiad?

\item What should be the balance between {\em theory} and {\em problems}?

\item Should we try for the {\em difficult problems} or begin with the {\em easy problems}?

\end{itemize}

On the {\em why} of Olympiad preparation:

\begin{itemize}

\item What is the {\em relevance} of Olympiad preparation to a career in mathematics?

\item Does Olympiad preparation conflict with preparation for other examinations? How does it help for students interested
  in pursuing engineering or other subjects?

\end{itemize}

I don't address all these questions here. What I try is to give you a handle to {\em get started}. As you proceed, you are likely
to discover some of the answers yourself.

If I get a good response, I might start a monthly newsletter that addresses these issues and covers strategies for mastering
Olympiad problems.

\subsection{Basic info}

For basic knowledge on the Olympiad programmes, check out the Wikipedia entries on: INMO, IMOTC, IMO. The URLs are:
\begin{verbatim}
  http://en.wikipedia.org/wiki/INMO
  http://en.wikipedia.org/wiki/IMOTC
  http://en.wikipedia.org/wiki/IMO
\end{verbatim}

\subsection{Other sources of resources}

To learn about more resources, check out:

\begin{verbatim}
  http://www.artofproblemsolving.com/Resources/AoPS_R_Websites.php
\end{verbatim}

I learnt about many of the websites I mention below from this place!

\section{The Internet}

\subsection{A powerful preparatory tool}

Today, the Internet is a cheap, fast and interactive way of learning stuff. Bulk of my
reading, writing, learning, and interacting happens via the Internet. And if you have Internet access, you can also use the Internet
to your great advantage!

If you want to use the Internet extensively, I recommend getting a Broadband connection. This turns out to be 
faster and more economical.

In my own time, I didn't use the Internet extensively as part of Olympiad preparation. I'm sure that had I used the Internet, I would not
only have learned more, but also have found Olympiad preparation more exciting.

\subsection{Pick a good browser}

I strongly recommend a browser that supports tabbed browsing. 
It will let you open lots of pages simultaneously and comfortably.
{\bf Mozilla Firefox} is a good browser for these purposes. Check out:

\begin{verbatim}
  http://getfirefox.com
\end{verbatim}

Ever since I've started using Firefox I haven't used Internet Explorer. Firefox runs on all platforms (Windows, Linux, Apple etc.).

\subsection{Create viable email IDs}

The first step towards setting up yourself on the Internet:

\begin{itemize}

\item Create a primary email ID. Make sure that you give this same email ID for all the sites you plan to enroll onto. Check this
  email ID frequently and store your password securely. Use your real name for this email ID.

\item Create email IDs on Yahoo!, Google, and MSN as secondary IDs that you can use for Instant Messaging if the need arises. If possible,
  have mail from your primary ID forwarded to one or more of these. Send all replies as if from your primary ID.

\end{itemize}

Your email ID may be your way to communicate with friends, other people on forums, professors and trainers you may contact, and so on.
So keep it respectable!

\subsection{Internet as a reference book}

The Internet is a huge repository of information. Some landmines of information:

\subsubsection{Wikipedia}

Official site: 
{\tt http://en.wikipedia.org}

Wikipedia is a website created by the people, for the people and of the people. 
It is a general repository on all subjects including mathematics.
You and I can go and edit its pages.
Thus, it is {\em fast changing}. At the same time, because {\em anybody} can edit it, articles put up very recently are likely
to contain errors because they have not yet been read and corrected by others. So, take whatever you read on Wikipedia
only as a pointer and confirm your knowledge from other sources (some of which may be referred to in the Wikipedia article itself).

Begin reading about Wikipedia on Wikipedia itself at:

\begin{verbatim}
  http://en.wikipedia.org/wiki/Wikipedia
\end{verbatim}

If you intend to use Wikipedia a lot:

\begin{itemize}

\item Create a user ID on Wikipedia. Log on through this ID frequently. 
  A user ID is not necessary for viewing the pages on Wikipedia but it is the first step towards creating your identity.
  Give your email ID as your primary email ID.

\item Learn how Wikipedia is structured, how articles are edited and organized, and how they can be searched.

\item Create your own watchlist for articles you are interested in.

\item Learn how to edit the articles on Wikipedia. Start by correcting spelling and grammar in existing articles.

\item Create your own user page on Wikipedia and interact with others through their User Talk pages.

\end{itemize}

I hadn't heard of Wikipedia till towards the end of my Olympiad days, and how I regret it! Wikipedia does not itself contain
too much information directly pertinent to Olympiads but it is definitely a good place to check up. It's grown a lot in the past
few years and is likely to grow more soon.

\subsubsection{Mathworld}

Official site: {\tt http://mathworld.wolfram.com}

This is Eric Weisstein's ``World of Mathematics''. Mathworld is very well organized, and fairly informative. It is not free to edit,
though people with new articles/edits for existing articles can send their edits by email to Mathworld. Mathworld
is fairly strong in geometry articles.

I've been using Mathworld extensively ever since my $12^{th}$ class and it's great fun looking at the sheer {\em number} of entries.
For instance, check out the {\em huge} number of triangle centers in the page:

\begin{verbatim}
  http://mathworld.wolfram.com/topics/TriangleCenters.html
\end{verbatim}

Or check out the large number of {\em properties related to primes} on the page:

\begin{verbatim}
  http://mathworld.wolfram.com/topics/Prime-RelatedNumbers.html
\end{verbatim}
\subsubsection{Planetmath}

Official site: {\tt http://planetmath.org}

This is not as organized as Mathworld. Users can edit content on Planetmath. However, unlike Wikipedia, it is exclusively for
mathematics.

\subsection{Internet as a search tool}

Google and Yahoo search are fairly indispensable to a mathematician and I believe that the time when every student uses these
for academic purposes is not far away. Learn how to use these search tools effectively:

\begin{itemize}

\item Figure out how to search for words, phrases, OR of words, NOT of words, and so on.

\item Learn how to search within a given website, for files of a given type, pages linking to a particular page, and so on.

\end{itemize}

%fill in examples here?

\subsection{Math specific fora and resources}

\subsubsection{Mathlinks}

A must-join site for serious mathematics students is:

\begin{verbatim}
  http://www.mathlinks.ro
\end{verbatim}

Its sister site is:

\begin{verbatim}
  http://www.artofproblemsolving.com
\end{verbatim}

The website is run by Valentin Vornicu of Romania. It has a lot of material for Olympiad Mathematics as well as stuff on College
mathematics.  To make good use of it:

\begin{itemize}

\item Sign up. Give your primary email ID while signing up. Log in whenever you use Mathlinks or AoPS because in many cases, only
  logged in users can post replies and queries.

\item Go through the classroom, books and other resources available. If you feel like, take online courses on Mathlinks.

\item Post problems on the forums, reply to problems posed by others, etc.

\item Go through the Olympiad problems as and when they are released.

\item Join general discussion forums, especially discussion topics relevant to India.

\end{itemize}

Mathlinks is exciting to use because all the problems are well sorted by topic and difficulty level. Anupam has some words
of advice here: {\em choose your problems} based on what interests you. The problem you pick should be of the kind that allows you to
keep thinking and pondering over it.

\subsubsection{The Mathematical Database}

This is a really nice one set up from Hong Kong. The URL is:

\begin{verbatim}
  http://eng.mathdb.org
\end{verbatim}

I've only come across it recently and am exploring it myself!

\subsection{Journals}

\subsubsection{Mathematical Reflections}

This has been started recently by Titu Andreescu, a prominent coach of the U.S. Olympiad team.
The journal is completely available online:

\begin{verbatim}
  http://reflections.awesomemath.org/
\end{verbatim}

\subsubsection{Crux Mathematicorum}

This magazine is published in Canada. Past issues are available online for free,
and new issues may be procured from good libraries such as the library at ISI Kolkatta.
Check out the webpage:

\begin{verbatim}
  http://journals.smc.math.ca/CRUX/
\end{verbatim}

\subsection{Olympiad specific repositories}

Some people who have been involved either with attending or training in the Math Olympiads have set up websites
to help future aspirants. Some of these websites  are worth visiting.
\subsubsection{John Scholes}

Official website: {\tt http://www.kalva.demon.co.uk}

The Kalva homepage contains Olympiad problems of the IMO and also of different National Olympiads. It doesn't contain recent
Olympiad problems (beyond 2003). But old Olympiad problems never lose their glamour, so this site retains its usefulness.

I first learned about Kalva's homepage at the IMOTC where a student had actually printed out John Scholes' solutions and bonud it into
a book. He used this book to practise Olympiad problems!

\subsubsection{Reid Barton}

Reid Barton has won four gold medals at the IMO, culminating in a perfect score in 2001 in the U.S.  He's also come in the top
five in the William Lowell Putnam Competition four times. He also came first in the International Olympiad in Informatics in 2001,
with 580 points out of 600. He's a top scorer at the ACM ICPC.

Reid has put up a directory of his Olympiad resources at:

\begin{verbatim}
  http://web.mit.edu/rwbarton/Public/mop/
\end{verbatim}

\subsubsection{Andrei Jorza}

His official home page is: {\tt http://www.ajorza.org}

\subsubsection{Hojoo Lee}

Official website: {\tt http://my.netian.com/$\sim$ideahitme/orange.html}

This is the homepage of Hojoo Lee. He's from Korea. He has proposed problems for the International Mathematical Olympiads.
In fact, two of his problems appeared in IMO 2004.

\subsection{Official websites}

\subsubsection{India}

The INMO coordinator as of now is Dr. R. B. Bapat, a Professor in IIT Delhi. Check out the following page he has put up:

\begin{verbatim}
  http://www.isid.ac.in/~rbb/olympiads.html
\end{verbatim}

\subsubsection{The United States}

An important source of information is the website of the
{\bf Mathematics Olympiad Summer Programme} of the USA:

\begin{verbatim}
  http://www.unl.edu/amc/a-activities/a6-mosp/mosp.html
\end{verbatim}

\subsubsection{Britain}

Information of the {\bf British Mathematical Olympiad} is available on the page:

\begin{verbatim}
  http://www.bmoc.maths.org/
\end{verbatim}

\subsubsection{Canada}

Learn about the Canadian Mathematical Olympiad on the page:

\begin{verbatim}
  http://www.math.ca/Competitions/CMO/
\end{verbatim}

\subsubsection{Asian pacific}

\begin{verbatim}
  http://www.cms.math.ca/Competitions/APMO/
\end{verbatim}

\section{Books}

How does one select the right books to read? Here's an interesting online post:

\begin{verbatim}
  http://mathforum.org/library/drmath/view/52354.html
\end{verbatim}

\subsection{Locating and purchasing the right books}

Places from where you can learn of the right books to buy:

\begin{itemize}

\item Lists on websites

\item Math forums such as Mathlinks

\item Friends, acquaintances and individuals

\item Lists in other books

\end{itemize}

When you hear of a  book, how do you decide whether it is worth reading? Here are some pointers:

\begin{itemize}

\item Try to locate the book in a library or with a friend and go through it.

\item Go through the book reviews on Amazon. Again, it may be advisable to sign up on Amazon to avail more features.

\end{itemize}

If you decide to buy or procure the book, here are some options:

\begin{itemize}

\item Locate an Indian edition in an Indian bookshop. If it is not stocked, request the bookshop to check for it and stock it.

\item Purchase the Western edition in an Indian bookshop, if the first is not possible.

\item Purchase the Western edition from Amazon or ask some friend abroad to get it. If purchasing from Amazon, it may turn out more
  economical to purchase all the books together rather than separately

\end{itemize}

If a large number of students are interested in a book and an Indian edition is not readily available, you may consider approaching the
Indian organ of the publisher and asking them to come out with the Indian edition.

\subsection{Possible bookstores}

Here's a list of bookstores (that I plan to augment with the passage of time). These bookstores are among the most likely
to contain books on mathematics that you could be looking for. However, do check in your local bookstore as well, if that's more
convenient for you!

\begin{itemize}

\item Bangalore: Tata Book House, Indian Institute of Science, Bangalore 560 012. 
  A 20\% discount on all books.
  Website {\tt http://tatabookhouse.com}

\item Chennai: Eswara Bookstores, Natesan Street, Mambalam, Chennai 600 017.
  A 10\% discount on all academic books.
\end{itemize}

\subsection{A list of Olympiad books I recommend}

\subsubsection{General reading}

\begin{enumerate}

\item {\bf Problem Solving Strategies} by Arthur Engel. The Springer Indian Edition costs Rs. 325 (marked price) and is available
  at Tata Book House as well as at Eswara. It is a comprehensive book that touches on problem solving in broadly all areas of
  Olympiad mathematics. Although it is good in covering combinatorics, its coverage of number theory and algebra is not so good.
  For mainstream geometry, it has {\em only} problems.

  Find the book on Amazon at this address:

  \begin{verbatim}
    http://www.amazon.com/gp/product/0387982191/
  \end{verbatim}

  In my time, this book was not available as an Indian edition. A relative of mine purchased it in the U.S. I enjoyed reading
  many sections of the book and it was crucial to getting me focussed on Olympiad related problem solving.

\item {\bf Mathematical Olympiad Challenges} by Titu Andreescu and Razvan Gelca. This is published by Birkhauser. The Indian edition
  costs Rs. 275 (marked price). It is available at the Tata Book House. This book focusses on developing tricks into methods and methods
  into mastery. Each section begins with a theme of problem solving, takes one or two examples
  that are easy if we apply the theme, and then gives a whole bunch of problems that need to be solved by variants of
  the basic theme. However, it does not comprehensively cover all topics. 

  Find the book on Amazon at this address:

  \begin{verbatim}
    http://www.amazon.com/gp/product/0817641556/
  \end{verbatim}

  I found this book fascinating! I tried ordering it from an Indian retailer and had almost given up hope of getting it because
  it took a lot of time. When  I finally got it, I was hooked to reading it. Section by section, I devoured the book.

\item {\bf Challenges and Thrills of Pre College Mathematics} by C.R. Pranesachar, B.J. Venkatachala, K.N. Ranganathan 
  and V. Krishnamurthy. This book is available at various book stores across India. 
  If you are not able to procure a copy, you can send a mail to 
  B.J. Venkatachala at {\tt jana@math.iisc.ernet.in} or C.R. Pranesachar at {\tt pran@math.iisc.ernet.in}.

  The style is a little dull and the book is huge, so it contains typos. The authors are planning to bring out a new
  edition soon.

  What I liked about he book was its Geometry section -- it is one of the few books that covers results like Ceva's Theorem,
  Menelaus' Theorem, the Simson Line etc. in good detail.

  Anupam, however, says that he found it a good introduction to the wonderful world of mathematics and wholeheartedly recommends it to
  you.
\end{enumerate}

\subsubsection{Number theory}

Some good books in number theory:

\begin{enumerate}

\item {\bf Elementary Number Theory} by David M. Burton. The Indian edition costs Rs. 175. It covers most of the topics needed for
  Olympiad aspirants, at least uptil the pre IMOTC stage. The first six chapters are essential reading.

  Find the book on Amazon at the address:

  \begin{verbatim}
    http://www.amazon.com/gp/product/0073051888/
  \end{verbatim}

  I was interested in numbers right from the age of eight, but it was this book, which I started reading in the Olympiad context,
  that got me into number theory. Anupam concurs with me fully.

\item {\bf Elementary Number Theory} by Gareth A. Jones and Josephine M. Jones. The Indian edition costs Rs. 225. I haven't gone through
  the book myself but I believe it to be at par with Burton. It is available at Tata Book house as well as at Eswara Book Stores.

  Find the book on Amazon at the address:

  \begin{verbatim}
    http://www.amazon.com/gp/product/3540761977/
  \end{verbatim}

  I think buying Burton as well as Jones and Jones is redundant, so you could pick one of these.

\item {\bf An Introduction to The Theory of Numbers} by Ivan Niven, Herbert S. Zuckermann, and Hugh L. Montgomery. This is a great
  book but some parts of it depend on a rudimentary 
  understanding of abstract algebra. The book serves as a useful reference and as an
  occasional source of challenging problems. It is available at Tata Book House as well as at Eswara Bookstores.
  The Indian edition marked price is \$ 6.95.

  Find the book on Amazon at the address:

  \begin{verbatim}
    http://www.amazon.com/gp/product/0471625469/
  \end{verbatim}

  This book, though ostensibly for college undergraduates, has a number of fairly accessible parts which I really enjoyed reading.
  For instance, their way of explaining elliptic curves was fairly hands on and I, with my limited knowledge of algebra, could get the hang
  of it.

\item {\bf An Introduction to The Theory of Numbers} by G.H. Hardy and E.M. Wright

  http://www.amazon.com/gp/product/0198531710

  The book touches on both elementary and advanced aspects of number theory in a well organized manner. It goes somewhat
  beyond Niven, Zuckermann and Montgomery. It is definitely a worthwhile read for number theory lovers.
\end{enumerate}

\subsubsection{Combinatorics}

Here are some books I particularly liked in combinatorics:

\begin{enumerate}

\item {\bf Schaum's Outline of Combinatorics} by V.K. Balakrishnan and V. Balakrishnan. The book is available in India, includign at the
  Tata Book House. learn about the book on Amazon at:

  \begin{verbatim}
    http://www.amazon.com/gp/product/007003575X
  \end{verbatim}
\end{enumerate}

\subsubsection{Geometry}

I think there is a dearth in the market of focussed books on geometry of the Olympiad type. However, the geometry sections in
{\em Mathematical Olympiad Challenges} and {\em Challenges and Thrills}, along with the problems in {\em Problem Solving Strategies}
give a fairly thorough introduction to geometry. Good books on geometry include:

\begin{enumerate}

\item {\bf Problems in Plane Geometry} by I.F.Sharygin. It has problems of all levels of difficulty and solving these helps
  increase thinking speed in geometry. The book is currently not in print according to Amazon and is not available in Indian bookshops. 
  Check the entry at:
  
  \begin{verbatim}
    http://www.amazon.com/gp/product/5030001808
  \end{verbatim}

  I've never really worked this book out. But Anupam
  tells us that the book has a plethora of stimulating problems in geometry and is a {\em must read} for a person
  seriously keen on geometry.

  To procure the book, try getting it secondhand (from ex Olympiad aficionados) or procuring copies from libraries.

\item {\bf Modern Geometry} by Clement Durrell. This book is given to students after they clear INMO.

  It's fascinating. It helped me achieve new insights in geometry.

\item {\bf College Geometry} by Howard Eves. Published by Narosa in India. It is available at the Tata Book House for Rs. 145.
  Check out the book's entry on Amazon:
  \begin{verbatim}
    http://www.amazon.com/gp/product/0867204753/
  \end{verbatim}

  I've not really spent much time with the book but I have found the parts that I read very illuminative. On the whole, however,
  the book is not a primary source for learning geometry.
\end{enumerate}

I plan to put in, fairly soon, references for other higher level geometry books.

\subsection{Puzzle and info books related to Olympiads}

\begin{enumerate}

\item {\bf Mathematical Gems} by Ross Honsberger. This book has multiple volumes.

\item {\bf Mathematical Circles} by Dmitri Fomin,Segey Genkin and Ilia Itenberg. This book is available in Tata Book House
  for Rs. 215. It is available at other bookstores in India as well. Learn about the book at Amazon:
  
  \begin{verbatim}
    http://www.amazon.com/gp/product/0821804308
  \end{verbatim}


\item The {\bf Penguin Dictionary of Curious and Interesting Geometry}: This book, along with Mathworld, was a source of interesting
  geometry puzzles and helped me build by intuition for geometry very effectively. Learn more at Amazon:
  
  \begin{verbatim}
    http://www.amazon.com/gp/product/0140118136/
  \end{verbatim}


\end{enumerate}

\subsection{General mathematical reading}

Olympiad books alone are not sufficient reading for good preparation. For instance, these books may not cover 
concepts such as countability versus uncountability, which students at the Olympiad level are expected to know.
Here are some I particularly enjoyed:

\begin{enumerate}
\item {\bf The Mathematical Experience} by Phillip J. Davis and Reuben Hersh
  \begin{verbatim}
    http://www.amazon.com/gp/product/0395929687/
  \end{verbatim}

  I got this book secondhand from somewhere and it influenced my thinking and ideas about mathematics.
\item {\bf What is Mathematics? An Elementary Approach to Ideas and Methods} by Richard Courant, Herbert Robbins and Ian Stewart.
  This book is in any case given at the end of the IMO Training Camp. It is extremely expensive but if you can get hold of a copy
  (in a library) do read it.
  \begin{verbatim}
    http://www.amazon.com/gp/product/0195105192/
  \end{verbatim}

  It's a serious book but all the same, it's the kind where you don't have to work things out with paper and pencil.
  Thus, it's ideal for concentrated but enjoyable reading. I {\em loved} reading this book.

\item {\bf Mathematics: The New Golden Age} by Keith Devlin.
  \begin{verbatim}
    http://www.amazon.com/gp/product/0140258655
  \end{verbatim}

  I read it a long time ago -- and enjoyed it then. Devlin discussed ten major areas of work in mathematics today.
  It makes a good first reading, at the very least.
\end{enumerate}

\subsection{Books about mathematicians}

\begin{enumerate}

\item {\bf Men of Mathematics} by Eric Temple Bell. This book is given at the end of the IMOTC. It chronicles the lives
  of mathematicians since the time of Euclid to the time of Henri Poincare. The book may or may not be worth
  purchasing but it's definitely worth a read if you encounter it in a library. Check it out at Amazon:

  \begin{verbatim}
    http://www.amazon.com/gp/product/0671628186
  \end{verbatim}
\item {\bf The Man Who Loved Only Numbers} by Paul Hoffmann, about Paul Erdos. Erdos was a somewhat eccentric mathematician who lived
  and breathed mathematics, right till his death at the age of about 96.
  \begin{verbatim}
    http://www.amazon.com/gp/product/0786884061
  \end{verbatim}

  This is inspiring! It paints a very real picture of mathematicians, and the world of mathematics. Hoffmann has done a thorough
  job!
\item {\bf A Beautiful Mind} by Sylvia Nasar. The first half of the book has a lot about the mathematical life of John Nash.
  \begin{verbatim}
    http://www.amazon.com/gp/product/0743224574
  \end{verbatim}

  I loved reading the first part of the book.
\item {\bf Fermat's Last Theorem} by Simon Singh. I haven't read this book myself but I have heard that it's very good.

  \begin{verbatim}
    http://www.amazon.com/gp/product/1841157910/
  \end{verbatim}

\end{enumerate}

\subsection{Journals published in India}

\begin{enumerate}

\item {\bf Samasya} is a journal published in India. The editors are B.J. Venkatachala and C.S. Yogananda, working in
  the Mathematical Olympiad Cell at IISc, Bangalore.

\item {\bf The Mathematics Teacher} is a journa published by the Association of Mathematics Teachers of India.

\item {\bf Sankhya} is published by people in ISI Kolkatta.
\end{enumerate}


\section{Institutes, coaches and guidance in India}

\subsection{Bhaskaracharya Pratishthana}

The {\bf Bhaskaracharya Pratishthana} in Pune has been a hub for Olympiad students for the past ten years. 
BP is primarily a research institute and its Olympiad training activities supplement its research.
Earlier, Mr. M. Prakash
was responsible for Olympiad activities at BP, and used to headhunt students from junior classes (like $7^{th}$) itself.
Prakash has now left BP but there continue to be Olympiad classes.

Check out the website: {\tt http://www.bprim.org}

\subsection{AMTI}

The {\bf Association of Mathematics Teachers of India} has its headquarters in Chennai. It is responsible for conducting Olympiad
activities in Tamil Nadu and also publishes magazines such as {\bf The Mathematics Teacher}. In 2005-06, they conducted a ten class
programme to prepare studnets for Olympiad problems, where I also went to teacha few classes.

The AMTI programme may not have shown immediate results at the INMO, but it is probable 
that if this programme is continued for a few years,
we will see a higher representation from Chennai in the IMOTC.

Check out the website: {\tt http://www.amtionline.com}

\subsection{Training in other states}

I believe that there is Olympiad related training in the state of Orissa as well, but I'm not sure. In most other states,
there is only an {\bf INMO Training Camp} for the students qualifying the RMO. This training camp may span 2-3 days
and is conducted by the local Olympiad coordinators and others they may manage to get for the job. 

\section{Some last words}

Preparing for the Olympiads is fun and it gives both immediate and long term rewards. Depending on how serious you are about it,
you may choose to invest the corresponding fraction of your energy.

My purpose here is to give you the right kind of inputs to help you explore for the Olympiads. The inputs I am providing are far from
comprehensive. You'll find many other inputs and you should weigh them accordingly. Moreover, I'll probably learn from your feedback
and augment my own inputs for future generations of students.

Please let me know how you found this writeup by sending an email to me at the email ID {\tt vipul@cmi.ac.in}

Do point out any errors, any problems you faced while locating or using the resources I have suggested, or other things you want
to know or think others may want to know. In particular, let me know of:

\begin{itemize}

\item Other bookstores which stock a wide range of Olympiad
  related books.

\item Libraries where these books can be read/borrowed and from where
  fragments of the book can be photocopied.

\end{itemize}

Thanks for reading and good luck!

\appendix

There are lots of people who've already responded to my plea for suggestions!

\begin{itemize}

\item All the people who, to begin with, helped me in my Olympiad preparation.

\item Anupam Prakash has patiently gone through the document and added his own comments and reviews at some places, which have been
  acknowledged fully.

\item Indraneel Mukherjee, an IMOTC two timer, and an
  IOI bronze medalist, and a keen proselytizer on the Olympiads, sent me a number
  of links, in particular, he sent me the link to the place to look for other links! Some of those links form part of the article.

\end{itemize}
% http://www-math.mit.edu/~kedlaya/geometryunbound/
\end{document}

%Calcutta Mathematical Society: library which has Crux Mathematicorum, American Mathematical Monthly