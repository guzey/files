\documentclass[a4paper]{amsart}

%Packages in use
\usepackage{fullpage}
%\usepackage{setspace} %To provide double spacing
%\usepackage{float,graphicx} %To allow images
%\usepackage{hyperref} %To make links clickable -- only for PDF
%\usepackage{showkeys} %To show the list of labels
%\usepackage{index} %To construct an advanced index

\input{/home/students/vipul/math/ownworks/seriousart.sty}

%Title details
\title{}
\author{Vipul Naik}
\thanks{\copyright Vipul Naik, B.Sc. (Hons) Math, 2nd Year, Chennai Mathematical Institute}

%List of new commands

\makeindex

\begin{document}
\maketitle
%\tableofcontents

\begin{abstract}
  Anupam and I are sitting together discussing how a beginner should get into Olympiad preparation.
\end{abstract}

\section{How do we begin number theory?}

\subsection{Basic theory}

Number theory is a lot of theory which is not covered in the school syllabus.

%burton and niven (paste from preparingforolympiads.tex)

Checklist:

\begin{itemize}

\item Congruences

\item Fermat's little theorem and Euler's $\phi$ function

\item Binomial theorem

\item What is a Diophantine equation and why it's difficult to solve (with some ideas on how to solve it).

\end{itemize}

\subsection{Problems in parallel}

How do we locate problems in a particular area?

Suppose VN thinks he's good at congruences, how does he prove or disprove it?

For workbook exercises, try the following thing: ``congruence'' in ``Number Theory:Solved Problems''.
``diophantine equation'' in ``Number Theory:Solved Problems''.

%then try in unsolved problems

\subsection{How to solve problems here}

%congruence chasing

\subsection{If you can't get it...}

Wait.

Take a break, but never say die. Pick another problem.

\subsection{Sample nice problems}

This is {\em not intended} to be a complete listing of good problems. Rather, I hope that this listing will tell
you how to judge problems and pick on the nice ones. 
Some things given as theorems and results in Burton serve as good problem material and can inspire other problems as well.
The ones below are in Burton as results:

\begin{enumerate}
\item A Poulet number is a number which satisfies Fermat's test for the base $2$. 
  If $n$ is a Poulet number, show that $M_n$ is a Poulet number.

\item $F_n$ is a Poulet number for all $n$.

\end{enumerate}

%why are these problems nice? How do you select a problem?

%good as a result (what is achieved is nice), element of surprise

%avoids restricting yourself to the workbook exercise (but don't develop distaste for the wb exercise)

\subsection{When you start getting the hang of things}

Anupam managed to prove that every prime of the form $4k+1$ can be written as $a^2 + b^2$. %one way is obvious, other one is pretty hard

Keep working at it.

Even tougher...

There are infinitely many primes congruent to $1$ mod $n$ for all $n$. Find out on the way about cyclotomic polynomials.

Another one, if you know Fermat's little theorem

Prove that $$(a + b\sqrt{3})^{p^2 - 1} \equiv 1 \mod p$$

Prove the sum of reciprocals of prime %wolstenholme's theorem baby version

%give pointers to my own material and to some articles that I locate specifically on number theory

%what makes these problems important and nice

\section{Geometry}

\subsection{Where does one start?}

High school geometry (upto tenth class) is reasonably good as far as the main theorem are concerned. Main further topics:

\begin{itemize}

\item Ceva's and Menelaus' Theorems and other techniques for concurrence and collinearity %pat my own back?

\item Triangle centers and triangle geometry%mathworld link

\end{itemize}

Other nice to know things:

\begin{itemize}

\item Transformation geometry

\item Coordinate geometry

\item Vector methods

\item Complex numbers

\item Circular inversion

\end{itemize}

\subsection{Testing oneself at geometry}

%seems easier to search, get problems, selecting nice problems may not be such an important criterion

\subsection{How to attempt a geometry problem}

\begin{itemize}
%point to some writeup
\item Diagram drawing : Some subtleties there

\item Diagram chasing and angle chasing

\item Reasoning should not depend on diagram assumptions, or should take all cases into account

\end{itemize}

\section{Combinatorics}

\subsection{Basics}

\begin{enumerate}

\item Existential combinatorics: PHP also called Box principle...

\item Enumerative combinatorics: Sum rule, product rule, principle of inclusion and exclusion

\item Facts in enumerative combinatorics: $n!$, Stirling numbers, binomial coefficients, rising and falling polynomials

\item Facts in existential combinatorics: $R(3,3)=6$, and related ones

\item Basic language of graph theory

\end{enumerate}

Book I've suggested is Schaum's Outline Series. Chapters in Engel are also good.

\subsection{Locating problems}

%not hard in combinatorics

%give some sample problems here also?

\subsection{How to attempt the problem}










\printindex

\end{document}
