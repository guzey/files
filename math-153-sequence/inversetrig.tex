\documentclass{amsart}
\usepackage{fullpage,hyperref,vipul,graphicx}
\title{Inverse trigonometric functions}
\author{Math 153, Section 55 (Vipul Naik)}

\newcommand{\arccot}{\operatorname{arccot}}
\begin{document}
\maketitle

{\bf Corresponding material in the book}: Section 7.7.

{\bf What students should already know}: The definitions of the
trigonometric functions and the key identities relating them.

{\bf What students should definitely get}: The definitions of the
inverse trigonometric functions for sine and cosine. The definitions
of the other inverse trigonometric functions, and the differentiation
rules for these functions. The application of these to some specific
types of indefinite integration.

{\bf What students should hopefully get}: The domain/range subtleties
for inverse trigonometric functions. The one-sided nature of inverses,
and how to compute $\arcsin \circ \sin$ and similar things in
general. The ``magic'' of finding inverse trigonometric functions as
antiderivatives of algebraic functions (rational functions and radical
functions).

{\em Note}: If you would like a more detailed review of trigonometry
concepts and ideas, please look at the notes titled {\em Trigonometry
review} back from 152. These notes were not covered in class, but were
included for reference.

\section*{Executive summary}

Words ...

\begin{enumerate}
\item The functions $\sin$, $\cos$, $\tan$, and their reciprocals are
  all periodic functions. While $\tan$ and $\cot$ have a period of
  $\pi$, the other four have a period of $2\pi$ each. While $\sin$ and
  $\cos$ are continuous and defined for all real numbers, the other
  four functions have points of discontinuity where they approach
  infinities of different signs from both sides. Also, $\tan$ and
  $\cot$ are one-to-one on a single period domain, while the other
  functions are usually two-to-one.
\item To construct inverses to these functions, we take intervals
  small enough such that the function is one-to-one restricted to that
  interval, but the range of the function restricted to that interval
  is the whole range. We also try to make our choice in such a manner
  that the {\em other function in the square sum/difference
  relationship} is nonnegative on the domain.
\item The choices are: $[-\pi/2,\pi/2]$ for $\sin$ (note that $\cos$
  is nonnegative on this), $[0,\pi]$ for $\cos$ (note that $\sin$ is
  nonnegative on this), $(-\pi/2,\pi/2)$ for $\tan$ (note that $\sec$
  is nonnegative on this), and $(0,\pi)$ for $\cot$ (note that $\csc$
  is nonnegative on this).
\item We define the corresponding inverse trigonometric functions
  $\arcsin$, $\arccos$, $\arctan$ and $\operatorname{arccot}$. The
  domains for both $\arcsin$ and $\arccos$ equal $[-1,1]$ while the
  domains for both $\arctan$ and $\operatorname{arccot}$ equal all of
  $\R$. The range of $\arcsin$ is $[-\pi/2, \pi/2]$ and the range of
  $\arccos$ is $[0,\pi]$. The range of $\arctan$ is $(-\pi/2,\pi/2)$
  and the range of arccot is $(0,\pi)$.
\item $\arcsin$ is an increasing function with vertical tangents at
  the endpoints, and $\arccos$ is a decreasing function with vertical
  tangents at the endpoints. For all $x$, we have $\arcsin x + \arccos
  x = \pi/2$. $\arctan$ is an increasing function with horizontal
  asymptotes valued at $-\pi/2$ and $\pi/2$ and arccot is a decreasing
  function with horizontal asymptotes valued at $\pi$ and $0$.
\item We define the arc secant function and the arc cosecant function
  as $\operatorname{arcsec}(x) = \arccos(1/x)$ and
  $\operatorname{arccsc}(x) = \arcsin(1/x)$. These are defined for all
  $x$ outside $(-1,1)$.
\item Using the formula for differentiating the inverse function, we
  obtain that $\arcsin'(x) = 1/\sqrt{1 - x^2}$ and $\arccos'(x) =
  -1/\sqrt{1 - x^2}$. Thus, $\int dx/\sqrt{1 - x^2} = \arcsin x +
  C$. Also, we obtain $\int dx/\sqrt{a^2 - x^2} = \arcsin(x/a) + C$.
\item Similarly, we obtain that $\arctan'(x) = 1/(1 + x^2)$ and
  $\operatorname{arcsec}'(x) = 1/(|x|\sqrt{x^2 - 1})$. Note that in
  the case of arc secant, we need the absolute value to account for
  the fact that tangent is not nonnegative on the range of arc secant.
\end{enumerate}

Actions ...

\begin{enumerate}

\item $\sin(\arcsin x) = x$ if $x$ lies in the domain of the $\arcsin$
  function. Note that otherwise $\sin(\arcsin x)$ does not make
  sense. Similar observations hold for the other trigonometric
  functions.
\item {\em Solving equations}: The solutions to $\sin x = \alpha$,
  where $\alpha \in [-1,1]$ come in two families: $\{ 2n\pi + \arcsin
  \alpha : n \in \mathbb{Z} \}$ and $\{ 2n\pi + (\pi - \arcsin \alpha)
  : n \in \mathbb{Z} \}$. Similarly, the solutions to $\cos x =
  \alpha$ where $\alpha \in [-1,1]$ come in two families: $\{ 2n \pi +
  \arccos \alpha: n \in \mathbb{Z} \}$ and $\{ 2n\pi - \arccos \alpha
  : n \in \mathbb{Z} \}$. In the special case where $\alpha = 1$
  (respectively, $\alpha = -1$), the two solution families for $\sin$
  (respectively, $\cos$) collapse into one solution family.
\item $\arcsin(\sin x)$ need not be equal to $x$ -- they are equal if
  and only if $x$ is in the range of $\arcsin$.
\item We often want to compute things like $\cos(2 \arctan x)$. To
  tackle these situations, we set $\theta = \arctan x$, so we obtain
  $\tan \theta = x$. The problem now reduces to determining
  $\cos(2\theta)$ in terms of $\tan \theta$, which is some elementary
  trigonometry. (It's elementary only if you know at least some of the
  double-angle formulas).
\item The integration formulas for $1/\sqrt{1 - x^2}$ etc. give rise
  to many slightly more general integration formulas. A short list is
  given in the book; a longer list is given on the last page of the
  the lecture notes. Rather than {\em just} memorize the long list,
  try to make sure you can {\em see} how the more complicated formulas
  arise from the basic ones, and also how you would apply them to
  actual integration problems. Internalize it to the level that you
  {\em just know it}.
\end{enumerate}

\section{A brief review of trigonometry}

In this lecture, we will discuss the inverse trigonometric
functions. But before doing that, we will quickly review our knowledge
of trigonometric functions.

\subsection{A review of sine and cosine}

So far, we have seen two broad definitions of sine and cosine. We
review those two definitions:

\begin{enumerate}
\item The definition in terms of right triangles. Here, if $\theta$ is
  one of the acute angles of a right-angled triangle, $\sin \theta$ is
  defined as the quotient of the side opposite to $\theta$ to the
  hypotenuse of the right-angled triangle. $\cos \theta$ is defined as
  the quotient of the leg adjacent to $\theta$ to the hypotenuse. This
  definition works for $\theta \in (0,\pi/2)$. That the definition
  makes sense relies on the notion of similar triangles: any two
  triangles with a common value of $\theta$ are {\em similar}, hence
  the quotients of corresponding side lengths are equal.
\item The definition in terms of the unit circle in the coordinate
  plane: Here, we start at the point $(1,0)$ in the coordinate plane,
  and move a length of $\theta$ along the circle $x^2 + y^2 = 1$ in
  the counter-clockwise direction. The $x$-coordinate of the point we
  land at is defined as $\cos \theta$ and the $y$-coordinate is
  defined as $\sin \theta$. This definition works for all $\theta \in
  \R$.
\end{enumerate}

We have seen the following key identities relating $\sin$ and $\cos$:

\begin{enumerate}
\item $\sin^2 \theta + \cos^2 \theta = 1$.
\item $\sin(\pi/2 - \theta) = \cos \theta$.
\item $\cos(\pi/2 - \theta) = \sin \theta$.
\item $\sin(-\theta) = - \sin \theta$, i.e., $\sin$ is an odd function.
\item $\cos(-\theta) = \cos \theta$, i.e., $\cos$ is an even function.
\item $\sin(\pi/2 + \theta) = \cos \theta$.
\item $\cos(\pi/2 + \theta) = -\sin \theta$.
\item $\sin(\pi - \theta) = \sin \theta$.
\item $\cos(\pi - \theta) = -\cos \theta$.
\item $\sin(\pi + \theta) = - \sin \theta$.
\item $\cos(\pi + \theta) = - \cos \theta$.
\item $\sin(2\pi + \theta) = \sin \theta$.
\end{enumerate}

Most of these can be deduced from the following four angle sum formulas:

\begin{enumerate}
\item $\sin(x + y) = \sin x \cos y + \cos x \sin y$.
\item $\cos(x + y) = \cos x \cos y - \sin x \sin y$.
\item $\sin(x - y) = \sin x \cos y - \cos x \sin y$.
\item $\cos(x - y) = \cos x \cos y + \sin x \sin y$.
\end{enumerate}

From these, we have the following double-angle formulas:

\begin{enumerate}
\item $\sin(2x) = 2 \sin x \cos x$.
\item $\cos(2x) = \cos^2 x - \sin^2 x = 2\cos^2 x - 1 = 1 - 2\sin^2
  x$. This also translates to $\cos^2 x = (1 + \cos (2x))/2$ and $\sin^2 x
  = (1 - \cos(2x))/2$.
\end{enumerate}

The integration and differentiation relationships are:

\begin{enumerate}
\item The derivative of $\sin$ is $\cos$.
\item The derivative of $\cos$ is $-\sin$.
\item An antiderivative of $\sin$ is $-\cos$.
\item An antiderivative of $\cos$ is $\sin$.
\end{enumerate}

\subsection*{Aside: The tyranny of great expectations}

Unlike the earlier, carefree days, you are now expected to be able to
do all the following:

\begin{enumerate}
\item Remember or be able to quickly reconstruct all the above formulas.
\item Relate the above formulas to facts about the graphs. For
  instance, the fact that $\sin(\pi - \theta) = \sin \theta$ indicates
  that the graph of $\sin$ exhibits mirror symmetry about the line $x
  = \pi/2$. Similarly, the fact that $\sin(2 \pi - \theta) = - \sin
  \theta$ indicates a half-turn symmetry about $\pi$. If your
  graphical intuition is strong, you may be able to reconstruct the
  algebraic identities from their self-evident graphical
  interpretations.
\item Derive the results from each other, and reconcile them. For
  instance, what is the derivative of $\sin (x + \pi)$? There are two
  ways of computing this. You should be able to do things in both ways
  and quickly reconcile your answers.
\end{enumerate}

\subsection{Finding solutions to $\sin x = a$}

What are the solutions to $\sin x = a$? Obviously, the answer depends
on $a$. We first consider four special cases:

\begin{enumerate}
\item If $|a| > 1$, there are {\em no} solutions.
\item If $a = 0$, then the solutions are the integer multiples of
  $\pi$.
\item If $a = 1$, the solutions are of the form $2n\pi + \pi/2$, $n
  \in \Z$.
\item If $a = -1$, the solutions are of the form $2n\pi - \pi/2$, $n
  \in \Z$.
\end{enumerate}

What about the other values of $a$? In these cases, we first find some
angle $\theta$ such that $\sin \theta = a$. There are then two solution families:

$$\{ 2n\pi + \theta : n \in \Z \}$$

and

$$\{ 2n\pi + (\pi - \theta) : n \in \Z \}$$

Basically, these are the $2\pi$-translates of $\theta$ and of $\pi -
\theta$. In particular, in any interval of length $2\pi$, there are
two solutions.  

Note that this general solution applies to the cases (2), (3) and (4)
listed above. However, something special happens in each of the
cases. In case (2), the two families are the even and odd multiples of
$\pi$: they can be readily collapsed into a single family, namely, all
multiples of $\pi$. In case (3), the two families coincide, because
$\pi - \pi/2 = \pi/2$. For similar reasons, the two families coincide
in case (4).

\subsection{Finding solutions to $\cos x = a$}

Here, if $|a| > 1$, there is no solution. Otherwise, we first find an
angle $\theta$ such that $\cos \theta = a$. The two families are then:

$$ \{ 2n \pi + \theta : n \in \mathbb{Z} \}$$

and

$$\{ 2n\pi - \theta : n \in \mathbb{Z} \}$$

The two families can be combined using the $\pm$ symbol:

$$\{ 2n \pi \pm \theta : n \in \mathbb{Z} \} $$

Note that in the special case where $a = 1$, the two families
coincide, and we simply obtain the multiples of $2\pi$. In the case $a
= -1$, the two families coincide, and we simply obtain the odd
multiples of $\pi$. In the case where $a = 0$, the two families can be
combined into one family: odd multiples of $\pi/2$. 

\subsection{The underclass of trigonometric functions}

We recall a bunch of trigonometric functions that are usually under
the radar, but recently surfaced in integration problems:

\begin{enumerate}

\item The $\tan$ function is defined as the quotient of the opposite
  side to the adjacent side. It is defined as $\sin/\cos$. Note that
  $\tan$ is undefined wherever $\cos$ is zero, which is to say, at odd
  multiples of $\pi/2$.
\item The $\cot$ function is defined as the quotient of the adjacent
  side to the opposite side. It is defined as $\cos/\sin$. Note that
  $\cot$ is undefined wherever $\sin$ is zero, which is to say, at all
  multiples of $\pi$.
\item The $\sec$ function is defined as the quotient of the hypotenuse
  to the adjacent side. It is defined as $1/\cos$. Note that $\sec$ is
  undefined wherever $\cos$ is zero, which is to say, at odd multiples
  of $\pi/2$.
\item The $\csc$ function is defined as the quotient of the hypotenuse
  to the opposite side. It is defined as $1/\sin$. Note that $\csc$ is
  undefined wherever $\sin$ is zero, which is to say, at all multiples
  of $\pi$.
\end{enumerate}

$\tan$ and $\cot$ have periods of $\pi$ each. $\tan$ is increasing on
each interval where defined (but more than makes up for it at the
isolated points where it is undefined) while $\cot$ is decreasing on
each interval where defined (but more than makes up for it at the
isolated points where it is undefined). You can think of these as
phoenix functions -- they start out as cute little creatures, grow to
ugly monsters, then vanish and reappear as cute little creatures. 

Here are the key relationships:

\begin{enumerate}
\item $\tan^2 x + 1 = \sec^2 x$.
\item $\cot^2 x + 1 = \csc^2 x$.
\item $\tan$ and $\cot$ are reciprocals of each other where both are defined.
\item $\sec$ and $\cos$ are reciprocals of each other where both are defined.
\item $\csc$ and $\sin$ are reciprocals of each other where both are defined.
\item $\tan(\pi/2 - x) = \cot x$.
\item $\sec(\pi/2 - x) = \csc x$.
\item The derivative of $\tan$ is $\sec^2$.
\item The derivative of $\cot$ is $-\csc^2$.
\item The derivative of $\sec$ is $\sec \cdot \tan$.
\item The derivative of $\csc$ is $-\csc \cdot \cot$.
\item The antiderivative of $\tan$ is $-\ln | \cos |$.
\item The antiderivative of $\cot$ is $\ln | \sin |$.
\item The antiderivative of $\sec$ is $\ln | \sec + \tan |$.
\item The antiderivative of $\csc$ is $\ln | \csc - \cot |$.
\end{enumerate}

\subsection{Three forms of complementarity}

There are six trigonometric functions, $\sin$, $\cos$, $\tan$, $\cot$,
$\sec$, and $\csc$, and there are three ways we can pair them up.

\begin{enumerate}
\item {\em Complementary angle pairing}: Here, we pair $f$ and $g$ if
  $f(x) = g(\pi/2 - x)$ for all $x$ where $f$ is well-defined. Under
  this pairing $\sin \leftrightarrow \cos$, $\tan \leftrightarrow
  \cot$, and $\sec \leftrightarrow \csc$. The prefix {\em co}
  basically refers to this pairing, hence {\em co}sine, {\em
  co}tangent, and {\em co}secant.
\item {\em Reciprocal pairing}: Here, we pair $f$ and $g$ if $f(x) =
  1/g(x)$ for all $x$ where both sides make sense. Under this pairing,
  $\sin \leftrightarrow \csc$, $\cos \leftrightarrow \sec$, and $\tan
  \leftrightarrow \cot$.
\item {\em Square sum/difference relation pairing}: Here, we pair $f$
  and $g$ if the sum or difference of their squares is constant. Under
  this pairing, $\sin \leftrightarrow \cos$, $\tan \leftrightarrow
  \sec$, and $\cot \leftrightarrow \csc$.
\end{enumerate}

\section{Ready to invert now}

\subsection{Looking before we leap}

With this brief primer on trigonometric functions, we are almost ready
to explore {\em inverse trigonometric functions}.

Recall that one of the conditions for inverting a function is that the
function be one-to-one. However, even if a function is not one-to-one
on its entire domain, it is possible that the function is one-to-one
on some subset of the domain. If the image of that subset equals the
range of the function on the whole domain, then we can try to get an
inverse by picking the representative in that subset.

For instance the squaring function is not one-to-one on $\R$. However,
its restriction to the nonnegative reals is one-to-one. And we can
define an inverse, otherwise known as the {\em square root} function,
from the nonnegative reals to the nonnegative reals.

The trigonometric functions are far from one-to-one when viewed on
their entire domain: after all, they're periodic. So the first step
will be to find a subset of the domain that is small enough for the
function to be one-to-one, but large enough for its image to be the
whole range of the function. We will observe a lot of interesting
things in the process. Enough looking, let's leap.

\subsection*{Aside: Photographic memories}

When you hear a person's name, the person's face probably pops in your
mind. When you hear the word ``apple'', the picture (and perhaps the
taste/smell) of an apple probably pops into your mind. In the same
way, when you hear the name of a familiar function, the picture of
that function should immediately pop into your mind. Not a very
refined picture with labels, just a crude picture. If the situation
demands it, you should be able to refine the picture.

You should also be able to recall the graphs of multiple functions
together, particularly functions that are commonly juxtaposed with
each other. For instance, you should be able to imagine $\sin$ and
$\cos$ graphed together, or $x$ and $x^2$ graphed together, etc.
\subsection{Inverting the sine function}

We see that the sine function has a period of $2\pi$, so definitely,
if we take an interval of length $2\pi$, the sine function takes all
the values in its range, which is $[-1,1]$. However, we observe that
within an interval of length $2\pi$, there are repetitions, and the
function is not one-to-one.

In fact, all we need to do is look at an interval where the $\sin$
function is increasing from $-1$ to $1$ or decreasing from $1$ to
$-1$. Several such intervals come to mind. For instance, the interval
$[-\pi/2,\pi/2]$ sees the sine function go from $-1$ to $1$, starting
off at a sluggish pace, picking pace to max right in the middle (at
$0$), and then coming to a graceful halt at $\pi/2$. There are other
such intervals, but we will stick t othe interval $[-\pi/2,\pi/2]$,
which we will call the {\em principal branch}.

\includegraphics[width=3in]{principalbranchofsine.png}

We now define the {\em inverse sine function} or the {\em arcsine
function}, as the function from $[-1,1]$ to $[-\pi/2,\pi/2]$ that
sends $a$ to the unique $\theta \in [-\pi/2,\pi/2]$ such that $\sin
\theta = a$. This function is denoted $\arcsin$, and sometimes as
$\sin^{-1}$. It is a one-to-one increasing function from $[-1,1]$ to
$[-\pi/2,\pi/2]$. Its graph can be obtained by reflecting about the $y
= x$ line the graph of the sine function on the interval
$[-\pi/2,\pi/2]$. Note that this has vertical tangents at the
endpoints, and its sense of concavity is opposite to that of $\sin$ on
the relevant intervals.

Now, consider the function $\cos$ on this interval. You see something
very interesting: on the range of the arcsine function (which is the
interval $[-\pi/2,\pi/2]$) the cosine function is $\ge 0$. This
observation is very important for some of the trigonometric
substitutions that we will see in integrations. Let's repeat: {\em
cosine is nonnegative on the range of the arc sine function}.

\subsection*{Aside: Something to feel good about next time}

In the next class, we will study integrations involving inverse
trigonometric functions. It would be great if, when we do that, one of
you can point out this observation at the particular juncture in the
reasoning where it becomes crucial.

\subsection{Inverting the cosine function}

For the cosine function, we choose the interval $[0,\pi]$, where the
function decreases from $1$ to $-1$. 

\includegraphics[width=3in]{principalbranchofcosine.png}

We define $\arccos(a)$ as the unique value $\theta \in [0,\pi]$ such
that $\cos \theta = a$. $\arccos$ is thus a decreasing function from
$[-1,1]$ to $[0,\pi]$. As in the previous example, we see that $\sin$
is nonnegative on the range of $\arccos$.

Below are the pictures of the arc sine and arc cosine functions:

\includegraphics[width=1.5in]{arcsinandarccos.png}

\subsection{An important relationship}

We have the following relationship:

$$\arcsin(x) + \arccos(x) = \pi/2$$

This basically follows from the fact that $\sin \theta = \cos(\pi/2 -
\theta)$. We also need to use the fact that $\theta \in [-\pi/2,\pi/2]
\iff \pi/2 - \theta \in [0,\pi]$.

\subsection{Solving $\sin x = a$ and $\cos x = a$, again}

Earlier, we talked about solving $\sin x = a$ (respectively, $\cos x =
a$). The first step of the solution in the general case was finding a
$\theta$ such that $\sin \theta = a$ (respectively, $\cos \theta =
a$). Now, for that first step, we can simply write $\theta =
\arcsin(a)$ (respectively, $\arccos(a)$).

\subsection*{Aside: No easy way out}

Don't use $\arcsin$ and $\arccos$ mindlessly. For instance, when
solving $\sin x = 2$, don't just write answers in terms of
$\arcsin(2)$ -- you should know that $2$ lies outside the domain of
$\arcsin$. Also, for values where you can easily find an angle
$\theta$, such as $a = 0, \pm 1, \pm 1/\sqrt{2}, \pm \sqrt{3}/2, \pm
1/2$, you must do so.

\section{Differentiating the inverse sine and cosine functions}

\subsection{A review of arc sine and arc cosine}

We recall, in point form, some key facts about the arc sine and the
arc cosine functions:

\begin{enumerate}
\item The arc sine function has domain $[-1,1]$, range
  $[-\pi/2,\pi/2]$. It is an increasing function, and has vertical
  tangents at its ends. Further, by definition, $\sin(\arcsin x) = x$
  for all $x \in [-1,1]$.
\item The arc cosine function has domain $[-1,1]$ and range
  $[0,\pi]$. It is a decreasing function, and has vertical tangents at
  both ends. Further, by definition, $\cos(\arccos x) = x$ for all $x
  \in [-1,1]$.
\item $\arcsin(x) + \arccos(x) = \pi/2$ for all $x \in [-1,1]$.
\item $\cos$ is nonnegative on the range of $\arcsin$, and $\sin$ is
  nonnegative on the range of $\arccos$.
\end{enumerate}

\subsection*{Aside: odd and even}

A little point of interest: if you take an odd one-to-one function, and
consider the inverse function, that is also odd. Since $\sin$ is odd,
and the domain $[-\pi/2,\pi/2]$ to which we restrict it is symmetric
about $0$, the restriction of $\sin$ to this domain is also an odd
function -- hence, its inverse function $\arcsin$ is an odd function.

Is the inverse function to an even function also even? It turns out
that the question does not make sense, because an even function
cannot be one-to-one if it is defined anywhere other than $0$. To get
an inverse, we first need to restrict the domain -- but the restricted
domain cannot have any values $a$ and $-a$ simultaneously. And then,
since the inverse function is again one-to-one, it cannot be even.

\subsection{Finding $\arcsin(\sin x)$}

We next turn to problems of the form: find $\arcsin(\sin x)$.

Your first temptation may be to say that the answer is $x$. This is
indeed true if $x$ is in the principal branch, i.e., $x$ is in the
range of $\arcsin$. Otherwise, the answer cannot be $x$. What then? We
know that if $t = \arcsin(\sin x)$, then $\sin t = \sin x$. Thus, $t$
is the unique value in the principal branch $[-\pi/2,\pi/2]$ such that
$\sin t = \sin x$. Finding this $t$ essentially involves using the
general solution method discussed last time or just an inspection based on the graph.

For instance, $\arcsin(\sin(7\pi/9))$ is an angle $\theta \in
[-\pi/2,\pi/2]$ such that $\sin \theta = \sin(7\pi/9)$. Now, we know
that $\sin(\pi - x) = \sin x$, so we have $\sin(\pi - 7\pi/9) =
\sin(7\pi/9)$, giving that $\sin(2\pi/9) = \sin(7\pi/9)$. And $2\pi/9$
lies in the range of $\arcsin$, so the much coveted $\theta$ is
$2\pi/9$.

The $\arcsin \circ \sin$ function is a continuous, periodic, and
piecewise linear function. It has a different linear definition on
each interval between successive odd multiples of $\pi/2$. The graph
of the function overall has a sawtooth shape with sharp turns at odd
multiples of $\pi/2$. Below is the picture:

\includegraphics[width=3in]{arcsinsin.png}

\subsection{Differentiating $\arcsin$}

Recall a tidbit about inverse functions: if $f(a) = b$, then:

$$(f^{-1})'(b) = \frac{1}{f'(a)}$$

Equivalently:

$$(f^{-1})'(x) = \frac{1}{f'(f^{-1}(x))}$$

In particular, we have that:

$$\arcsin'(x) = \frac{1}{\sin'(\arcsin(x))}$$

Since $\sin' = \cos$, we obtain that:

$$\arcsin'(x) = \frac{1}{\cos(\arcsin(x))}$$

The next key step involves trying to figure out if we can simplify
$\cos(\arcsin x)$. So let's say $\arcsin x = \theta$. Then, $\sin
\theta = x$ by definition. And we have the relationship:

$$\sin^2 \theta + \cos^2 \theta = 1$$

We need to find $\cos \theta$. The above relation gives:

$$\cos^2 \theta = 1 - \sin^2 \theta = 1 - x^2$$

Thus, either $\cos \theta = \sqrt{1 - x^2}$ or $\cos \theta = -\sqrt{1
- x^2}$. how do we figure out which one it is?

It turns out that it is always the nonnegative square root, because, by
the choice of principal branch, $\cos$ is nonnegative on the range of
$\arcsin$. Thus, we obtain that:

$$\arcsin'(x) = \frac{1}{\sqrt{1 - x^2}}$$

Here is a graphical illustration of $\arcsin$ and its derivative {\em
on the same graph}:

\includegraphics[width=1.5in]{arcsinanditsderivative.png}
\subsection{Voodoo magic}

Something amazing just happened. We defined trigonometric
functions. First we did it for acute angles. Then we extended the
definition to all real numbers. We developed the machinery of
differentiation and integration. We developed the machinery of inverse
functions. We used this machinery to define inverse trigonometric
functions. And then we differentiated an inverse trigonometric
function, and what did we get? We got something that looks {\em purely
algebraic}! We got something that involves polynomials and
squareroots, with nary a hint of trigonometry.

To see how magical this is, consider if, instead, I had presented the
indefinite integration problem:

$$\int \frac{dx}{\sqrt{1 - x^2}}$$

From what we just saw above, the answer is $\arcsin(x) + C$. Thus, the
entire machinery of trigonometry is necessary to unlock an innocuous
indefinite integration problem arising from algebra.

Note that this is similar to the way we saw the natural logarithm pop
up when integrating $(dx)/x$. However, there is a key difference. In
that example, we used the integration of $(dx)/x$ to {\em define} the
natural logarithm, and then noticed that the function relates quite a
bit with our existing notions of exponentiation. In the trigonometric
case, we obtained the $\arcsin$ function through other means, and then
noticed that it works as an antiderivative for $(1 - x^2)^{-1/2}$.

\subsection{A general observation}

In the previous quarter, we saw that if $f'(x) = g(f(x))$, then
$f^{-1}$ is an antiderivative for $1/g$. The above is a special case,
where $f = \sin$ (restricted to $[-\pi/2,\pi/2]$), and $g(t) = \sqrt{1
- t^2}$.

\subsection{The derivative of arc cosine}

By similar reasoning to the above:

$$\arccos'(x) = \frac{-1}{\sqrt{1 - x^2}}$$

Note that this makes sense because $\arcsin + \arccos$ is a constant,
hence its derivative shoud be zero, and hence, $\arccos' = -\arcsin'$.

\section{The other inverse trigonometric functions}

\subsection{The inverse of the tangent function}

Recall that $\tan$ is not defined at odd multiples of $\pi/2$, and is
increasing on each of the intervals on which it is defined. Thus, a
natural choice of principal branch for $\tan$ is
$(-\pi/2,\pi/2)$.

\includegraphics[width=1in]{principalbranchoftan.png}

The inverse function, denoted $\arctan$ or sometimes $\tan^{-1}$, and
called the arc tangent function, takes as input $x \in \R$ and outputs
the unique $\theta \in (-\pi/2,\pi/2)$ such that $\tan \theta =
x$. Thus, it is a one-to-one increasing function with domain $\R$ and
range $(-\pi/2,\pi/2)$. In particular, it has two horizontal
asymptotes: the horizontal asymptote $y = +\pi/2$ for $x \to +\infty$
and the horizontal asymptote $y = -\pi/2$ for $x \to -\infty$.

\includegraphics[width=3in]{arctan.png}

The derivative of this is given by:

$$\arctan'(x) = \frac{1}{\sec^2(\arctan x)} = \frac{1}{1 + x^2}$$

This also gives another indefinite integration formula:

$$\int \frac{dx}{1 + x^2} = \arctan x + C$$

{\em We saw a variant of this in Homework 9 of Math 152, Advanced
Problem 1.}

Below is a graph showing $\arctan$ and its derivative together:

\includegraphics[width=3in]{arctananditsderivative.png}

\subsection{The inverse of the cotangent function}

Recall that $\cot$ is not defined at multiples of $\pi$, and is
decreasing on each of the intervals where it is defined. Thus, a
natural choice of principal branch for $\cot$ is $(0,\pi)$.

The inverse function, denoted $\arccot$ or sometimes $\cot^{-1}$, and
called the arc cotangent function, is thus a one-to-one decreasing
function from $\R$ to $(0,\pi)$. Further, its derivative is:

$$\arccot'(x) = \frac{-1}{1 + x^2}$$

Also, we have that:

$$\arctan(x) + \arccot(x) = \frac{\pi}{2}$$

As was the case with arc sine and arc cosine, the sum of the
derivatives of $\arctan$ and $\arccot$ is zero, as it should be.

\subsection{The inverse of the secant and cosecant function}

We do not talk much about these, but we formally define:

$$\operatorname{arcsec}(x) = \arccos(1/x)$$

with domain $(-\infty,-1] \cup [1,\infty)$ and range $[0,\pi]
\setminus \{ \pi/2 \}$.

And we define:

$$\operatorname{arccsc}(x) = \arcsin(1/x)$$

with domain $(-\infty,-1] \cup [1,\infty)$ and range $[-\pi/2,\pi/2]
\setminus \{ 0 \}$. We have:

$$\operatorname{arcsec}(x) + \operatorname{arccsc}(x) = \frac{\pi}{2}$$

We also have the derivative formulas:

$$\operatorname{arcsec}'(x) = \frac{1}{|x|\sqrt{x^2 - 1}}$$

and:

$$\operatorname{arccsc}'(x) = \frac{-1}{|x|\sqrt{x^2 - 1}}$$

The proof of these requires multiple cases, basically because $\tan$
is not nonnegative on the entire range of arcsec, and $\cot$ is not
nonnegative on the entire range of arccsc. Look up the proof in the
book if you are interested.

\section{Note on improper integrals}

Consider the definite integral:

$$\int_0^1 \frac{dx}{\sqrt{1 - x^2}}$$

This becomes:

$$[\arcsin x]_0^1 = \arcsin 1 - \arcsin 0 = \frac{\pi}{2}$$

This integration is correct. However, the integrand itself ($1/\sqrt{1
  - x^2}$) is not defined at $1$. In fact, $\arcsin$ is defined at $1$
but has a one-sided vertical tangent, with the derivative approaching
$+\infty$.

This is an example of an {\em improper integration}. We will come back
to improper integrations later in the course.
\section{The formulas for indefinite integration}

We have the following formulas for indefinite integration:

\begin{eqnarray*}
  \int \frac{dx}{\sqrt{1 - x^2}} & = & \arcsin(x) + C \\
  \int \frac{dx}{1 + x^2} & = & \arctan(x) + C\\
  \int \frac{dx}{|x|\sqrt{x^2 - 1}} & = & \operatorname{arcsec}(x) + C
\end{eqnarray*}

We now consider slight variants of these. In all the formulas below,
$a > 0$ is a constant:

\begin{eqnarray*}
  \int \frac{dx}{\sqrt{a^2 - x^2}} & = & \arcsin\left(\frac{x}{a}\right) + C\\
  \int \frac{dx}{a^2 + x^2} & = & \frac{1}{a} \arctan \left(\frac{x}{a}\right) + C\\
  \int \frac{dx}{|x|\sqrt{x^2 - a^2}} & = & \frac{1}{a} \operatorname{arcsec}\left(\frac{x}{a}\right) +C
\end{eqnarray*}

We consider yet more variants of these. Here, $a > 0$ and $b$ is
arbitrary, but both are constants:

\begin{eqnarray*}
  \int \frac{dx}{\sqrt{a^2 - (x - b)^2}} & = & \arcsin\left(\frac{x - b}{a}\right) + C\\
  \int \frac{dx}{a^2 + (x - b)^2} & = & \frac{1}{a} \arctan \left(\frac{x - b}{a} \right) + C\\
  \int \frac{dx}{|x - b|\sqrt{(x - b)^2 - a^2}} & = & \frac{1}{a} \operatorname{arcsec}\left(\frac{x - b}{a}\right) + C
\end{eqnarray*}

Next, we see how these can be combined with other ideas. As before, $a
> 0$ is a constant:

\begin{eqnarray*}
  \int \frac{f'(x) \, dx}{\sqrt{a^2 - (f(x))^2}} & = & \arcsin\left(\frac{f(x)}{a}\right) + C\\
  \int \frac{f'(x) \, dx}{(f(x))^2 + a^2} & = & \frac{1}{a}\arctan\left(\frac{f(x)}{a}\right) + C\\
  \int \frac{f'(x) \, dx}{|f(x)|\sqrt{(f(x))^2 - a^2}} & = & \frac{1}{a} \operatorname{arcsec}\left(\frac{f(x)}{a}\right) + C\\
  \int \frac{f(\arctan (x/a))}{a^2 + x^2}  \, dx& = & \frac{1}{a} \int f(u) \, du \text{ where } u = \arctan(x/a)\\
  \int \frac{f(\arcsin(x/a))}{\sqrt{a^2 - x^2}} \, dx& = & \int f(u) \, du \text{ where } u = \arcsin(x/a)
\end{eqnarray*}

\end{document}
