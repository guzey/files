\documentclass[10pt]{amsart}

%Packages in use
\usepackage{fullpage, hyperref, vipul, enumerate}

%Title details
\title{Class quiz: January 6: Exponential growth}
\author{Math 153, Section 55 (Vipul Naik)}
%List of new commands

\begin{document}
\maketitle

Your name (print clearly in capital letters): $\underline{\qquad\qquad\qquad\qquad\qquad\qquad\qquad\qquad\qquad\qquad}$

\begin{enumerate}
\item A species of unicellular micro-organisms doubles in number every
  one hour at room temperature and remains constant when placed in a
  refrigerator. Given that the initial number of micro-organisms in a
  dish is $N_0$, and the dish is kept at room temperature for $A$ hours
  and in a refrigerator for $B$ hours, what is the {\bf total number} of
  micro-organisms at {\bf the end}? {\em Last year: $29/29$ correct}

  \begin{enumerate}[(A)]
  \item $N_0 \cdot 2^{A - B}$
  \item $N_0 \cdot 2^{A + B}$
  \item $N_0 \cdot 2^{AB}$
  \item $N_0 \cdot 2^A$
  \item $N_0 \cdot 2^B$
  \end{enumerate}

  \vspace{0.1in}
  Your answer: $\underline{\qquad\qquad\qquad\qquad\qquad\qquad\qquad}$
  \vspace{0.1in}

\item A radioactive substance has a half-life of $3$ years. {\bf
  Determine the integer} $n$ such that $90\%$ of the substance decays
  within somewhere between $n - (1/2)$ and $n + (1/2)$ years. {\em
  Last year: $23/29$ correct}

  \begin{enumerate}[(A)]
  \item $5$
  \item $10$
  \item $15$
  \item $20$
  \item $25$
  \end{enumerate}

  \vspace{0.1in}
  Your answer: $\underline{\qquad\qquad\qquad\qquad\qquad\qquad\qquad}$
  \vspace{0.1in}

\item $A$, $B$, and $C$ are three species of unicellular
  micro-organisms. Under specified conditions, species $A$ doubles in
  number every $2$ hours, species $B$ triples in number every $3$
  hours, and species $C$ quadruples (i.e., becomes $4$ times) in
  number every $4$ hours. Assume that they start off in the same
  quantities at the beginning. What can we say about their relative
  rates of growth? {\em Last year: $22/29$ correct}
  \begin{enumerate}[(A)]
  \item They are all growing at the same rate.
  \item Species $A$ is growing fastest, species $C$ is growing
    slowest, and species $B$ is growing at an intermediate rate.
  \item Species $A$ is growing slowest, species $C$ is growing
    fastest, and species $B$ is growing at an intermediate rate.
  \item Species $A$ and $C$ are both growing at the same rate, which
    is faster than the rate at which species $B$ is growing.
  \item Species $A$ and $C$ are both growing at the same rate, which
    is slower than the rate at which species $B$ is growing.
  \end{enumerate}

  \vspace{0.1in}
  Your answer: $\underline{\qquad\qquad\qquad\qquad\qquad\qquad\qquad}$
  \vspace{0.1in}

\item A species of bacteria doubles in number every hour. It takes $9$
  hours for a given initial quantity of this species to fill up a
  petri dish volume. How many hours from the start did the species
  occup half the petri dish volume (assume that the volume occupied is
  proportional to the quantity)? {\em Last year: $28/29$ correct}

  \begin{enumerate}[(A)]
  \item $1$ hour from the beginning
  \item $3$ hours from the beginning
  \item $4.5$ hours from the beginning
  \item $6$ hours from the beginning
  \item $8$ hours from the beginning
  \end{enumerate}

  \vspace{0.1in}
  Your answer: $\underline{\qquad\qquad\qquad\qquad\qquad\qquad\qquad}$
  \vspace{0.1in}
\item Suppose the populations in two countries $A$ and $B$ are growing
  exponentially at possibly different rates. Which of the following
  statements is {\bf false}? {\em Last year: $24/29$ correct}

  \begin{enumerate}[(A)]
  \item If the initial population of $A$ is more, and the exponential
    population growth rate of $A$ is greater, then the population of
    $A$ will always be greater than that of $B$.
  \item If the initial population of $A$ is more, and the exponential
    population growth rate of $B$ is greater, then the population of
    $B$ will eventually overtake the population of $A$.
  \item If the initial population of $A$ is more, and the exponential
    population growth rates of $A$ and $B$ are equal, then the
    populations of $A$ and $B$ will eventually become equal.
  \item All of the above.
  \item None of the above.
  \end{enumerate}

  \vspace{0.1in}
  Your answer: $\underline{\qquad\qquad\qquad\qquad\qquad\qquad\qquad}$
  \vspace{0.1in}

\item (**) It takes time $T$ for $1/5$ of a radioactive substance to
  decay. How much time does it take for $2/5$ of the same radioactive
  substance to decay? {\em Last year: $7/28$ correct}

  \begin{enumerate}[(A)]
  \item Precisely $T/2$
  \item Between $T/2$ and $T$
  \item Between $T$ and $2T$
  \item Precisely $2T$
  \item Between $2T$ and $3T$
  \end{enumerate}

  \vspace{0.1in}
  Your answer: $\underline{\qquad\qquad\qquad\qquad\qquad\qquad\qquad}$
  \vspace{0.1in}

\item (**) The population in the island of Andrognesia as a function
  of time is believed to be an exponential function. On January 1,
  1984, the population was measured to be $3 * 10^5$ with a
  measurement error of up to $10^5$ on either side, i.e., the
  population was measured to be between $2* 10^5$ and $4 * 10^5$. On
  January 1, 1998, the population was measured to be $1.2 * 10^6$ with
  a measurememt error of up to $4 * 10^5$ on either side, i.e., the
  population was measured to be between $8 * 10^5$ and $1.6 *
  10^6$. If the population is an exponential function of time (i.e.,
  the increment in population per year is a fixed proportion of the
  population that year), what is the {\bf range of possible values} of
  the population measured on January 1, 2012? {\em Last year: $4/29$
  correct}

  \begin{enumerate}[(A)]
  \item Between $3.2 * 10^6$ and $6.4 * 10^6$
  \item Between $3.2 * 10^6$ and $1.28 * 10^7$
  \item Between $1.6 * 10^6$ and $3.2 * 10^6$
  \item Between $1.6 * 10^6$ and $6.4 * 10^6$
  \item Between $1.6 * 10^6$ and $1.28 * 10^7$
  \end{enumerate}

  \vspace{0.1in}
  Your answer: $\underline{\qquad\qquad\qquad\qquad\qquad\qquad\qquad}$
  \vspace{0.1in}

\end{enumerate}

\end{document}
