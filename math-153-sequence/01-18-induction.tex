\documentclass[10pt]{amsart}

%Packages in use
\usepackage{fullpage, hyperref, vipul, enumerate}

%Title details
\title{Class quiz: January 18: Mathematical induction}
\author{Math 153, Section 55 (Vipul Naik)}
%List of new commands

\begin{document}
\maketitle

Your name (print clearly in capital letters): $\underline{\qquad\qquad\qquad\qquad\qquad\qquad\qquad\qquad\qquad\qquad}$

For all these questions, {\em natural number} refers to positive
integer. In particular, $0$ is not considered to be a natural number.
\begin{enumerate}
\item Suppose $S$ is a subset of the natural numbers with the property
  that $1 \in S$ and $k \in S \implies k + 2 \in S$. What can we
  conclude is {\bf definitely true} about $S$? {\em Last year: $18/29$
  correct}

  \begin{enumerate}[(A)]
  \item $S$ contains all natural numbers.
  \item $S$ contains all natural numbers other than $2$. It may or may
    not contain $2$.
  \item $S$ contains all odd natural numbers.
  \item $S$ contains all even natural numbers.
  \item $S$ does not contain any natural number other than $1$.
  \end{enumerate}

  \vspace{0.1in}
  Your answer: $\underline{\qquad\qquad\qquad\qquad\qquad\qquad\qquad}$
  \vspace{0.2in}

\item Suppose $S$ is a subset of the natural numbers with the property
  that whenever $k \in S$, we have $k + 5 \in S$. Which of these is
  the {\bf smallest subset} $T$ with the property that checking $T
  \subseteq S$ is sufficient to show that $S$ is the set of all
  natural numbers? {\em Last year: $21/29$ correct}

  \begin{enumerate}[(A)]
  \item $\{ 1,2,3 \}$
  \item $\{ 1,2,3,4 \}$
  \item $\{ 1,2,3,4,5 \}$
  \item $\{ 1,4 \}$
  \item $\{ 1,3,5 \}$
  \end{enumerate}

  \vspace{0.1in}
  Your answer: $\underline{\qquad\qquad\qquad\qquad\qquad\qquad\qquad}$
  \vspace{0.2in}

\item Consider the function $f(x) := a \sin x + b \cos x$, with $a$,
  $b$ nonzero reals. The $n^{th}$ derivative of $f$ is denote
  $f^{(n)}$. The association $n \mapsto f^{(n)}$ is periodic, i.e.,
  there is a unique smallest positive integer $h$ such that $f^{(n +
  h)} = f^{(n)}$ for all $n$. What is {\bf this value} of $h$? {\em
  Last year: $25/29$ correct}

  \begin{enumerate}[(A)]
  \item $1$
  \item $2$
  \item $3$
  \item $4$
  \item $5$
  \end{enumerate}

  \vspace{0.1in}
  Your answer: $\underline{\qquad\qquad\qquad\qquad\qquad\qquad\qquad}$
  \vspace{0.2in}

\item What is the correct {\bf general expression} for the {\bf sum}
  $\sum_{k=2}^n \frac{1}{k^2 - 1}$ for a positive integer $n \ge 2$?
  {\em Last year: $18/29$ correct}

  \begin{enumerate}[(A)]
  \item $\frac{3}{2} - \frac{2n + 3}{2(n+1)}$
  \item $\frac{3}{2} - \frac{2n + 3}{n(n+1)}$
  \item $\frac{3}{4} - \frac{2n + 1}{(n + 1)(n+2)}$
  \item $\frac{3}{4} - \frac{2n - 1}{2n(n-1)}$
  \item $\frac{3}{4} - \frac{2n + 1}{2n(n+1)}$
  \end{enumerate}

  \vspace{0.1in}
  Your answer: $\underline{\qquad\qquad\qquad\qquad\qquad\qquad\qquad}$
  \vspace{0.2in}
\end{enumerate}

\end{document}
