\documentclass[10pt]{amsart}
\usepackage{fullpage,hyperref,vipul, graphicx}
\title{Review sheet for midterm 2: advanced}
\author{Math 153, Section 55 (Vipul Naik)}

\begin{document}
\maketitle

{\bf To maximize efficiency, please bring a copy (print or readable
electronic) of both the basic and the advanced review sheet AND the
previous review sheet to the review session.}

In addition to spotting the errors in the existing reasoning, try to
solve the problems and find the {\em correct} solution for each of the
error-spotting exercises.

\section{Left-overs from integration}

\subsection{Integrating radicals}

Error-spotting exercises..

\begin{enumerate}
\item Consider the integration:

  $$\int_0^1 \sqrt{1 - x^2} \, dx$$

  To perform this integration, we put $\theta = \arcsin x$. Then $x =
  \sin \theta$ and $\sqrt{1 -x^2}= \cos \theta$, so we get:

  $$\int_0^1 \cos \theta \, d\theta$$

  This simplifies to $\sin 1 - \sin 0 = \sin 1$.

\item Consider the integral

  $$\int_0^\infty \frac{dx}{(x^2 + a^2)^{3/2}}$$

  Let $\theta = \arctan(x/a)$. Then, the above becomes:

  $$\int_0^{\pi/2} \frac{\sec^2 \theta \, d\theta}{a^3 \sec^3 \theta}$$

  This simplifies to:

  $$\frac{1}{a^3} \int_0^{\pi/2} \cos \theta \, d\theta$$

  The integral of $\cos$ gives the value $1$, so the overall answer is
  $1/a^3$.

\item Consider the integral:

  $$\int_0^\infty \frac{dx}{x^2 + 2x \cos\alpha + 1}$$

  Completing the square in the denominator, we get:

  $$\int_0^\infty \frac{dx}{(x + \cos \alpha)^2 + \sin^2\alpha}$$

  We thus get:

  $$\left[\frac{1}{\cos \alpha} \arctan\left(\frac{x + \cos \alpha}{\sin\alpha}\right)\right]_0^\infty$$

  Plugging in limits and evaluating, we get:

  $$\frac{1}{\cos \alpha} (\pi/2 - \cot \alpha)$$

  This simplifies to:

  $$\frac{\pi}{2 \cos \alpha} - \csc \alpha$$

\end{enumerate}


\subsection{Partial fractions}

Error-spotting exercises:

\begin{enumerate}
\item Suppose $a,b$ are distinct positive numbers. We can do the
  integration:

  $$\int \frac{dx}{(x^2 + a^2)(x^2 + b^2)}$$

  as follows:

  $$\int \frac{dx}{(x^2 + a^2)(x^2 + b^2)} = \frac{1}{a^2 - b^2}\int \frac{(x^2 + a^2) - (x^2 + b^2)}{(x^2 + a^2)(x^2 + b^2)}\, dx$$

  This simplifies to:

  $$\frac{1}{a^2 - b^2} \int \left[\frac{1}{x^2 + a^2} - \frac{1}{x^2 + b^2} \right] \, dx$$

  This becomes:

  $$\frac{1}{a^2 - b^2} [\ln(x^2 + a^2) - \ln(x^2 + b^2)] + C$$

\item Consider the integration:

  $$\int_{-\infty}^\infty \frac{dx}{(x^2 + a^2)^2}$$

  The integration gives:

  $$\left[\frac{-1}{x^2 + a^2}\right]_{-\infty}^\infty $$

  Both the endpoint limits are zero, so the integral is $0$.

\item Consider the integral

  $$\int \frac{dx}{(x - \alpha)(x - \beta)(x - \gamma)}$$

  where $\alpha$, $\beta$, $\gamma$ are distinct real numbers.

  Using partial fractions, we get that the integral is:

  $$\int \frac{(\alpha - \beta)(\alpha - \gamma) \, dx}{x - \alpha} + \frac{(\beta - \gamma)(\beta - \alpha) \, dx}{x - \beta} + \frac{(\gamma - \alpha)(\gamma - \beta) \, dx}{x - \gamma}$$

  This gives:

  $$(\alpha - \beta)(\alpha - \gamma)\ln|x - \alpha| + (\beta - \gamma)(\beta - \alpha) \ln|x - \beta| + (\gamma - \alpha)(\gamma - \beta) \ln|x - \gamma| + C$$
\end{enumerate}

\subsection{Improper integrals}

Error-spotting exercises:

\begin{enumerate}
\item We have:

  $$\int_{-\infty}^\infty \frac{dx}{x^3} = \left[\frac{1}{2x^2}\right]_{-\infty}^\infty = 0$$

\item We have:

  $$\int_{-\infty}^\infty \frac{x \, dx}{x^2 + 1} = \lim_{a \to \infty} \int_{-a}^a \frac{x \, dx}{x^2 + 1} = \lim_{a \to \infty} 0 = 0$$

  where the integral on any finite interval $[-a,a]$ is zero on account of the function being an odd function.

\end{enumerate}

\section{Differential equations}

\subsection{Solving differential equations at large}

Error-spotting exercises

\begin{enumerate}
\item Consider the differential equation:

  $$(y')^2 - 3yy' + 2y^2 = 0$$

  his factors as:

  $$(y' - y)(y' - 2y) = 0$$

  Thus, either $y' = y$ or $y' = 2y$ for which respective general
  solutions are $Ce^x$ and $Ce^{2x}$. The general solution is thus of
  the form $C_1e^x +C_2e^{2x}$ where $C_1,C_2$ are arbitrary real
  numbers.

\item Consider the differential equation:

  $$\frac{dy}{dt} = \sqrt{y(1 - y)}$$

  Solving, we get:

  $$y = \frac{1}{2}(1 + \sin(t + C))$$

  The solution to this differential equation thus looks like a sinusoidal oscillating curve.
\end{enumerate}

\subsection{Graphical interpretation and initial value problems}

Words ...

\begin{enumerate}
\item Consider the differential equation:

  $$\frac{dy}{dt} = y \ln y$$

  with the additional condition that $y(0) > 0$. The solution to this
  differential equation is a function $y$ of $t$ such that $\lim_{t
  \to \infty} y(t) = \infty$ and the function grows doubly
  exponentially.
\item Consider the differential equation:

  $$\frac{dy}{dt}= y^2$$

  with $y(0) = 1$. Then, solving the differential equation, we get:

  $$\frac{-1}{y} = t + C$$

  Plug in $y(0) = 1$ to get $C = -1$, so we get:

  $$y = \frac{1}{1 - t}$$

  The upshot is that $\lim_{t \to \infty} y(t) = 0$.

\end{enumerate}

\section{The least upper bound axiom}

Error-spotting exercises ...

\begin{enumerate}
\item Suppose $T$ is a nonempty subset of a nonempty set $S$. Then,
  the glb of $T$ is less than or equal to the glb of $S$. Similarly, the
  lub of $T$ is less than or equal to the lub of $S$.
\item For subsets $A$ and $B$ of $\R$, denote by $AB$ the set $\{ ab
  \mid a \in A, b \in B \}$. Then, if $A$ and $B$ are both nonempty
  and bounded, so is $AB$. Further, the glb of $AB$ is the product of
  the glb of $A$ and the glb of $B$. Similarly, the lub of $AB$ is the
  product of the lub of $A$ and the lub of $B$.
\end{enumerate}

\section{Sequences of reals}

\subsection{Sequences: basics}

No error-spotting exercises

\subsection{Continuous-discrete interplay}

No error-spotting exercises

\section{Limit computation techniques}

Error-spotting exercises ...

\begin{enumerate}
\item Consider the limit computation problem:

  $$\lim_{x \to 0} \frac{2 - \cos x}{x^2}$$

  To do this limit, we apply the LH rule once to get:

  $$\lim_{x \to 0} \frac{\sin x}{x}$$

  Apply the LH rule another time and get:

  $$\lim_{x \to 0} \frac{\cos x}{1}$$

  This evaluates to $1$.
\item Consider the limit computation problem:

  $$\lim_{x \to 0} \frac{x - \sin x}{(1 - \cos x)\ln (1 + x)}$$

  The numerator has a zero of order three, and the denominator is a
  product of terms of zeros of order two and one. Hence the denomintor
  has a zero of order three as well. Thus, the quotient has a limiting
  value of $1$.

\item Consider the limit computation problem:

  $$\lim_{x \to 1} \frac{\sin(2 \pi x)}{\sin (\pi x)}$$

  We can strip off the $\sin$ from the numerator and denominator and we get:

  $$\lim_{x \to 1} \frac{2 \pi x}{\pi x}$$

  This simplifies to $2$.

\item Consider the limit problem:

  $$\lim_{x \to \pi/2^-} \frac{\tan(5x)}{\tan x}$$

  We can strip off the $\tan$ from both the numerator and the
  denominator, and we are left with a limit of $5$.
\end{enumerate}

\section{Tricky topics}

\subsection{Important classes of functions outside of the elementary world}

{\em This is not part of the executive summary of any of the lecture
notes, but is related to some of the homework problems and the
challenge problem.}

We note here three such classes of functions:

\begin{enumerate}
\item The function arising as an integral of $\exp(-x^2)$. All the
  following functions have antiderivatives expressible in terms of
  this antiderivative and elementary functions: 

  \begin{itemize}
  \item $x^{2k}\exp(-mx^2)$ where $k$ is a positive integer and $m$ is
    a positive real: The idea here is to use integration by parts,
    taking $x^{2k - 1}$ as the part to differentiate and
    $x\exp(-mx^2)$ as the part to integrate. The problem reduces to
    $x^{2k - 2}\exp(-mx^2)$. Repeat.
  \item $x^{m + (1/2)}\exp(-x)$ where $m$ is an integer (not
    necessarily positive. Putting $u = \sqrt{x}$ allows us to go back
    and forth with $\exp(-x^2)$. Note that the $x^{m + (1/2)}\exp(-x)$
    are related to each other for different values of $m$ via
    integration by parts: take $\exp(-x)$ as the part to integrate.
  \item $\exp(x^{-2/(2m + 1)}$ where $m$ is a positive integer. Put $u
    = x^{1/(2m + 1)}$ to go back and forth.
  \end{itemize}

  In particular, in applicable cases, we can calculate the {\em
  improper definite integral} on all of $\R$ for each of these
  functions based on the fact that:

  $$\int_{-\infty}^\infty \exp(-x^2) = \sqrt{\pi}$$

  In particular, since the function is even, we have:

  $$\int_0^\infty \exp(-x^2) = \frac{\sqrt{\pi}}{2}$$

  We define:

  $$\Gamma(x) := \int_0^\infty t^{x - 1}e^{-t} \, dt$$

  for $x \in (0,\infty)$.

  Based on the fact about the integral of $\exp(-x^2)$, we can
  calculate the value of $\Gamma$ at all half-integers. Note that for
  $x$ a positive integer, $\Gamma(x) = (x - 1)!$ as seen in a homework
  exercise.

\item The function arising as an integral of $1/(\ln x)$, for $x >
  1$. All the following have antiderivatives expressible in terms of
  this antiderivatve and elementary functions:

  \begin{itemize}
  \item $\ln(\ln x)$ for $x > 1$.
  \item $e^x/x$, for $x > 0$. Also, $e^x\ln x$ and $e^x/x^r$ for all
    positive integers $r$.
  \item $e^{e^x}$ for all $x$.
  \end{itemize}

\item The function arising as an integral of $(\sin x)/x$ (we fill in
  the value of the integrand at $0$ to be $1$). All the following have
  antiderivatives expressible in terms of this: $(\cos x)(\ln x)$, $(1
  - \cos x)/x^2$, $(x - \sin x)/x^3$. The back-and-forth technique is
  integration by parts: one thing gets integrated, the other
  differentiated.

  In particular, given that:

  $$\int_0^\infty \frac{\sin x}{x} \, dx = \frac{\pi}{2}$$

  we can compute the integrals from $0$ to $\infty$ of $(1 - \cos
  x)/x^2$ and $(x - \sin x)/x^3$.
\end{enumerate}

\section{Quickly}

This section lists things you should be able to do quickly.

\subsection{Sequence pattern finding}

You should be able to do the following:

\begin{enumerate}
\item Given the first few terms of an easy sequence, predict the next
  few terms.
\item Describe a sequence using a recurrence relation.
\item Write the general expression (possibly making cases if there's
  periodic-like behavior) for the $n^{th}$ term.
\item Move back and forth between general term descriptions and
  recurrence relation descriptions of sequences.
\end{enumerate}

\subsection{Our common values}

Preferably remember these (or be capable of computing quickly) to at
least one digit. Generally, you will {\em not} be asked to do any
numerical computations using these. In practice, the main way this is
useful is to figure out whether something is positive or negative. For
instance, is $3\sqrt{2} - 4$ positive? What about $e^2 - 8$? Often,
there are other ways of answering such questions, but remembering the
numerical values is a quick and dirty approach.

\begin{enumerate}
\item Square roots of $2$, $3$, $5$, $6$, $7$, $10$.
\item Natural logarithms of $2$, $3$, $5$, $7$, and $10$.
\item Value of $\pi$, $1/\pi$, $\sqrt{\pi}$, and $\pi^2$.
\item Value of $e$, $1/e$.
\item Some relative logarithms, such as $\log_23$ or
  $\log_2(10)$. Although you don't need these values to a significant
  degree of precision, it is useful to have some idea of their
  magnitude.
\end{enumerate}

\subsection{Adding things up: arithmetic}

You should be able to:

\begin{enumerate}
\item Do quick arithmetic involving fractions.
\item Sense when an expression will simplify to $0$.
\item Compute approximate values for square roots of small numbers,
  $\pi$ and its multiples, etc., so that you are able to figure out,
  for instance, whether $\pi/4$ is smaller or bigger than $1$, or two
  integers such that $\sqrt{39}$ is between them.
\item Know or quickly compute small powers of small positive
  integers. This is particularly important for computing definite
  integrals. For instance, to compute $\int_2^3 (x + 1)^3 \, dx$, you
  need to know/compute $3^4$ and $4^4$.
\end{enumerate}

\subsection{Computational algebra}

You should be able to:

\begin{enumerate}
\item Add, subtract, and multiply polynomials.
\item Factorize quadratics or determine that the quadratic cannot be
  factorized.
\item Factorize a cubic if at least one of its factors is a small and
  easy-to-spot number such as $0$, $\pm 1$, $\pm 2$, $\pm 3$. {\em
  This could be an area for potential improvement for many people.}
\item Factorize an even polynomial of degree four. {\em This could be
  an area for potential improvement for many people.}
\item Do polynomial long division.
\item Solve simple inequalities involving polynomial and rational
  functions once you've obtained them in factored form.
\end{enumerate}

\subsection{Computational trigonometry}

You should be able to:

\begin{enumerate}
\item Determine the values of $\sin$, $\cos$, and $\tan$ at multiples
  of $\pi/2$.
\item Determine the intervals where $\sin$ and $\cos$ are positive and
  negative.
\item Remember the formulas for $\sin(\pi \pm x )$ and $\cos(\pi \pm x)$,
  as well as formulas for $\sin(-x)$ and $\cos(-x)$.
\item Recall the values of $\sin$ and $\cos$ at $\pi/6$, $\pi/4$, and
  $\pi/3$, as well as at the corresponding obtuse angles or other
  larger angles.
\item Reverse lookup for these, for instance, you should quickly
  identify the acute angle whose $\sin$ is $1/2$.
\item Formulas for double angles, half angles: $\sin(2x)$, $\cos(2x)$
  in terms of $\sin$ and $\cos$; also the reverse: $\sin^2x$ and
  $\cos^2x$ in terms of $\cos(2x)$.
\item Remember the formulas for $\sin(A + B)$, $\cos(A + B)$, $\sin(A
  - B)$, and $\cos(A - B)$.
\item Convert between products of $\sin$ and $\cos$ functions and
  their sums: for instance, the identity $2\sin A \cos B = \sin(A + B)
  + \sin (A - B)$. You don't have to remember these identities
  separately since they follow from the identities covered in the
  previous point, but you should be comfortable going back and forth.
\end{enumerate}

\subsection{Computational limits}

This includes:

\begin{enumerate}
\item Look at a rational function and determine its limit to $+\infty$
  and $-\infty$ based on heuristics, without thought.
\item Look at something involving a quotient of an exponential and a
  polynomial, or a polynomial and a logarithmic, and predict the
  behavior as we go to $0$ or $\infty$.
\item Know which functions can be stripped from composites.
\item Apply the LH rule mentally multiple times and figure out how
  many times it needs to be applied.
\end{enumerate}

\subsection{Computational differentiation}

You should be able to:

\begin{enumerate}
\item Differentiate a polynomial (written in expanded form) once or
  twice on sight, without rough work.
\item Differentiate sums of powers of $x$ on sight (without rough
  work).
\item Differentiate rational functions with a little thought.
\item Do multiple differentiations of expressions whose derivative
  cycle is periodic, e.g., $a \sin x + b \cos x$ or $a \exp(-x)$.
\item Do multiple differentiations of expressions whose derivative
  cycle is periodic up to constant factors, e.g. $a \exp(mx + b)$ or $a
  \sin(mx + \varphi)$.
\item Differentiate simple composites without rough work (e.g.,
  $\sin(x^3)$).
\item Differentiate $\ln$, $\exp$, and expressions of the form $f^g$
  and $\log_f(g)$.
\end{enumerate}

\subsection{Computational integration}

You should be able to:

\begin{enumerate}
\item Compute the indefinite integral of a polynomial (written in
  expanded form) on sight without rough work.
\item Compute the definite integral of a polynomial with very few
  terms within manageable limits quickly.
\item Compute the indefinite integral of a simple-looking rational
  function and also the definite integral, with a little time,
  including improper integrals.
\item Compute the indefinite integral of a sum of power functions
  quickly.
\item Know that the integral of sine or cosine on any quadrant is $\pm
  1$.
\item Compute the integral of $x \mapsto f(mx)$ if you know how to
  integrate $f$. In particular, integrate things like $(a + bx)^m$.
\item Integrate $\sin$, $\cos$, $\sin^2$, $\cos^2$, $\tan^2$,
  $\sec^2$, $\cot^2$, $\csc^2$, $\sin^3$, $\cos^3$, $\tan^3$,
  $\sec^3$, $\cot^3$, $\csc^3$, and other higher powers of the basic
  trigonometric functions.
\item Integrate on sight things such as $x\sin(x^2)$, getting the
  constants right without much effort.
\item {\em By parts}: Integrate $\int xf(x) \, dx$ on sight if it is
  easy to integrate $f$ twice, without having to think much (the
  answer is $x \int f - \int \int f$). Similarly, do $\int x^2f(x) \,
  dx$ if you know how to integrate $f$ twice.
\item Using the previous point, integrate $\int f(\sqrt{x}) \, dx$
  with minimal stress and effort.
\item {\em By parts}: Integrate $\int f(x) \, dx$ on sight if $xf'(x)$
  is easy to integrate, e.g., $\int \ln x \, dx$.
\item Remember the integrals for formats such as $e^x \cos x$ and $e^x
  \sin x$.
\item If there is an easy choice of $f$ such that $f + f' = g$,
  integrate $\int e^xg(x) \, dx = e^xf(x)$ on sight and similarly
  integrate $\int g(\ln x) \, dx = xf(\ln x)$ on siht.
\end{enumerate}
\subsection{Being observant}

You should be able to look at a function and:

\begin{enumerate}
\item Sense if it is odd (even if nobody pointedly asks you whether it
  is).
\item Sense if it is even (even if nobody asks you whether it is).
\item Sense if it is periodic and find the period (even if nobody asks
  you about the period).
\end{enumerate}

\subsection{Graphing}

You should be able to:

\begin{enumerate}
\item Mentally graph a linear function.
\item Mentally graph a power function $x^r$ (see the list of things to
  remember about power functions). Sample cases for $r$: $1/3$, $2/3$,
  $4/3$, $5/3$, $1/2$, $1$, $2$, $3$, $-1$, $-1/3$ $-2/3$.
\item Graph a piecewise linear function with some thought.
\item Mentally graph a quadratic function (very approximately) --
  figure out conditions under which it crosses the axis etc.
\item Graph a cubic function after ascertaining which of the cases for
  the cubic it falls under.
\item Mentally graph $\sin$ and $\cos$, as well as functions of the $A
  \sin(mx)$ and $A\cos(mx)$.
\item Graph a function of the form linear + trigonometric, after doing
  some quick checking on the derivative.
\item Graph the inverse trigonometric functions $\arctan$, $\arcsin$,
  and $\arccos$.
\end{enumerate}

\subsection{Graphing: transformations}

Given the graph of $f$, you should be able to quickly graph the following:

\begin{enumerate}
\item $f(mx)$, $f(mx + b)$: pre-composition with a linear function;
  how does $m < 0$ differ from $m > 0$?
\item $Af(x) + C$: post-composition with a linear function, how does
  $A > 0$ differ from $A < 0$?
\item $f(|x|)$, $|f(x)|$, $f(x^+)$, and $(f(x))^+$: pre- and
  post-composition with absolute value function and positive part
  fnuction.
\item More slowly: $f(1/x)$, $1/f(x)$, $\ln(|f(x)|)$, $f(\ln|x|)$,
  $\exp(f(x))$, and other popular composites.
\end{enumerate}

\end{document}