\documentclass[10pt]{amsart}

%Packages in use
\usepackage{fullpage, hyperref, vipul, enumerate}

%Title details
\title{Take-home class quiz solutions: Integration by parts: due Wednesday January 18}
\author{Math 153, Section 55 (Vipul Naik)}
%List of new commands

\begin{document}
\maketitle

\section{Performance review}

$11$ people took this $9$-question quiz. The score distribution is as
follows:

\begin{itemize}
\item Score of $4$: $2$ people.
\item Score of $5$: $2$ people.
\item Score of $6$: $2$ people.
\item Score of $7$: $2$ people.
\item Score of $8$: $3$ people.
\end{itemize}

The mean score was $6.18$ and the mean and modal score were $6$.

Here is the question-wise performance:

\begin{enumerate}
\item Option (E): $6$ people
\item Option (B): Everybody
\item Option (C): $8$ people
\item Option (C): $10$ people
\item Option (D): $7$ people
\item Option (B): $8$ people
\item Option (E): $7$ people
\item Option (A): $7$ people
\item Option (A): $4$ people
\end{enumerate}

\section{Solutions}

In the questions below, we say that a function is {\em expressible in
terms of elementary functions} or {\em elementarily expressible} if it
can be expressed in terms of polynomial functions, rational functions,
radicals, exponents, logarithms, trigonometric functions and inverse
trigonometric functions using pointwise combinations, compositions,
and piecewise definitions. We say that a function is {\em elementarily
integrable} if it has an elementarily expressible antiderivative.

Note that if a function is elementarily expressible, so is its
derivative on the domain of definition.

We say that a function $f$ is $k$ times elementarily integrable if
there is an elementarily expressible function $g$ such that $f$ is the
$k^{th}$ derivative of $g$.

We say that the integrals of two functions are {\em equivalent up to
elementary functions} if an antiderivative for one function can be
expressed using an antiderivative for the other function and
elementary function, again piecing them together using pointwise
combination, composition, and piecewise definitions.

\begin{enumerate}
\item Consider the statements $P$ and $Q$, where $P$ states that every
  rational function is elementarily integrable, and $Q$ states that
  any rational function is $k$ times elementarily integrable for all
  positive integers $k$.

  Which of the following additional observations is {\bf correct} and
  {\bf allows us to deduce} $Q$ given $P$?

  \begin{enumerate}[(A)]
  \item There is no way of deducing $Q$ from $P$ because $P$ is true
    and $Q$ is false.
  \item The antiderivative of a rational function can always be chosen
    to be a rational function, hence $Q$ follows from a repeated
    application of $P$.
  \item Using integration by parts, we see that repeated integration
    of a function $f$ is equivalent to integrating $f$, $f^2$, $f^3$,
    and higher powers of $f$ (the powers here are pointwise products,
    not compositions). If $f$ is a rational function, each of these is
    also a rational function. Applying $P$, each of these is
    elementarily integrable, hence $f$ is $k$ times elementarily
    integrable for all $k$.
  \item Using integration by parts, we see that repeated integration
    of a function $f$ is equivalent to integrating $f$, $f'$, $f''$,
    and higher derivatives of $f$. If $f$ is a rational function, each
    of these is also a rational function. Applying $P$, each of these
    is elementarily integrable, hence $f$ is $k$ times elementarily
    integrable for all $k$.
  \item Using integration by parts, we see that repeated integration
    of a function $f$ is equivalent to integrating each of the
    functions $f(x)$, $xf(x)$, $\dots$. If $f$ is a rational function,
    each of these is also a rational function. Applying $P$, each of
    these is elementarily integrable, hence $f$ is $k$ times
    elementarily integrable for all $k$.
  \end{enumerate}

  {\em Answer}: Option (E)

  {\em Explanation}: Fill in yourself, it's been said often enough.

  {\em Performance review}: $6$ out of $11$ got this. $5$ chose (B).

  {\em Historical note (last year)}: $18$ out of $27$ people got this
  correct. $4$ people chose (D), $2$ people chose (C), $2$ people
  chose (B), and $1$ person chose (A).

\item Suppose $f$ is a continuous function on all of $\R$ and is the
  third derivative of an elementarily expressible function, but is not
  the fourth derivative of any elementarily expressible function. In
  other words, $f$ can be integrated three times but not four times
  within the collection of elementarily expressible functions. What is
  the {\bf largest positive integer} $k$ such that $x \mapsto x^kf(x)$
  is {\em guaranteed to be} {\bf elementarily integrable}?

  \begin{enumerate}[(A)]
  \item $1$
  \item $2$
  \item $3$
  \item $4$
  \item $5$
  \end{enumerate}

  {\em Answer}: Option (B)

  {\em Explanation}: Via integration by parts, integrating $f$ $m$ times is
  equivalent to finding antiderivatives for $f(x)$, $xf(x)$, and so on
  till $x^{m-1}f(x)$. In our case, $f$ can be integrated $3$ times, so
  the largest $k$ is $3 - 1 = 2$.

  {\em Note}: It is true that $x^2f(x)$ can be integrated and
  $x^3f(x)$ cannot. {\em A priori}, we cannot say whether $x^4f(x)$
  and $x^5f(x)$ can or cannot be integrated. They cannot be integrated
  {\em using the integration by parts approach}, but it may happen to
  be the case that they could be integrated by other methods, though
  this is rare in practical cases (and when that does happen, it is
  obvious).

  {\em Performance review}: Everybody got this.

  {\em Historical note}: $23$ out of $27$ people got this
  correct. $2$ people each chose (C) and (D).

\item Suppose $f$ is a continuous function on $(0,\infty)$ and is the
  third derivative of an elementarily expressible function, but is not
  the fourth derivative of any elementarily expressible function. In
  other words, $f$ can be integrated three times but not four times
  within the collection of elementarily expressible functions. What is
  the {\bf largest positive integer} $k$ such that the function $x
  \mapsto f(x^{1/k})$ with domain $(0,\infty)$ is [ADDED: {\em
  guaranteed to be}] {\bf elementarily integrable}?

  \begin{enumerate}[(A)]
  \item $1$
  \item $2$
  \item $3$
  \item $4$
  \item $5$
  \end{enumerate}

  {\em Answer}: Option (C)

  {\em Explanation}: Via the $u$-substitution $u = x^{1/k}$, we get
  $\int ku^{k-1}f(u) \, du$. Now using the previous question, the
  maximum value of $k - 1$ possible is $2$, so the maximum possible
  value is $3$.

  We can also do a direct integration by parts taking $1$ as the
  second part.

  {\em Note}: It is true that $f(x^{1/3})$ can be integrated and
  $f(x^{1/4})$ cannot. {\em A priori}, we cannot say whether
  $f(x^{1/5})$ can or cannot be integrated. It cannot be integrated
  {\em using the integration by parts approach}, but it may happen to
  be the case that they could be integrated by other methods, though
  this is rare in practical cases (and when that does happen, it is
  obvious).

  {\em Performance review}: $8$ out of $11$ got this. $2$ chose (D)
  and $1$ chose (B).

  {\em Historical note}: $14$ out of $27$ people got this correct. $4$
  people chose (D), $4$ people chose (B), $3$ people chose (A), $1$
  person chose (E), and $1$ person left the question blank.


\item Of these five functions, four of the functions are elementarily
  integrable and can be integrated using integration by parts. The
  other one function is  {\bf not
  elementarily integrable}. Identify this function.

  \begin{enumerate}[(A)]
  \item $x \mapsto x \sin x$
  \item $x \mapsto x \cos x$
  \item $x \mapsto x \tan x$
  \item $x \mapsto x \sin^2x$
  \item $x \mapsto x \tan^2x$
  \end{enumerate}

  {\em Answer}: Option (C)

  {\em Explanation}: If $f$ is elementarily integrable, then $xf(x)$
  is elementarily integrable iff $f$ is twice elementarily integrable;
  this is easily seen using integration by parts. Of the function
  options given here, $\tan$ is the only function that is not twice
  elementarily integrable, because the first integration gives
  $-\ln|\cos x|$ which cannot be integrated. Of the others, note that
  $\sin$, $\cos$, and $\sin^2$ can be integrated using elementary
  functions infinitely many times. $\tan^2$ is twice elementarily
  integrable but no further: integrates the first time to $\tan x -
  x$, which integrates one more time to $-\ln|cos x| - x^2/2$, which
  cannot be integrated further.

  {\em Performance review}: $10$ out of $11$ got this. $1$ chos e(D).

  {\em Historical note}: $22$ out of $27$ people got this
  correct. $4$ people chose (E), and $1$ person chose (D).


\item Consider the four functions $f_1(x) = \sqrt{\sin x}$, $f_2(x) =
  \sin \sqrt{x}$, $f_3(x) = \sin^2 x$ and $f_4(x) = \sin(x^2)$, all
  viewed as functions on the interval $[0,1]$ (so they are all well
  defined). Two of these functions are elementarily integrable; the
  other two are not. Which are the {\bf two elementarily integrable
  functions}?

  \begin{enumerate}[(A)]
  \item $f_3$ and $f_4$.
  \item $f_1$ and $f_3$.
  \item $f_1$ and $f_4$. 
  \item $f_2$ and $f_3$.
  \item $f_2$ and $f_4$.
  \end{enumerate}

  {\em Answer}: Option (D)

  {\em Explanation}: Integration of $f_3$ is a standard procedure, so
  we say nothing about that. As for $f_2$, recall that integrating
  $f(x^{1/k})$ is equivalent to integrating $u^{k-1}f(u)$ where $u =
  x^{1/k}$, which in turn is equivalent to integrating $f$ $k$
  times. Since $\sin$ can be integrated as many times as we wish,
  $f_2$ can be integrated.

  The reason why $f_1$ and $f_4$ are not elementarily integrable is
  subtler but it's clear that none of the obvious methods work.

  {\em Performance review}: $7$ out of $11$ got this. $1$ chose (E),
  $1$ each chose (A) and (B).

  {\em Historical note}: $17$ out of $27$ people got this
  correct. $5$ people chose (B), $3$ people chose (A), and $2$ people
  chose (C).
\item Suppose $f$ is an elementarily expressible and infinitely
  differentiable function on the positive reals (so all derivatives of
  $f$ are also elementarily expressible). An antiderivative for
  $f''(x)/x$ is {\bf not equivalent} up to elementary functions to
  {\bf which one} of the following?

  \begin{enumerate}[(A)]
  \item An antiderivative for $x \mapsto f''(e^x)$, domain all of $\R$.
  \item An antiderivative for $x \mapsto f'(e^x/x)$, domain positive reals.
  \item An antiderivative for $x \mapsto f'''(x)(\ln x)$, domain positive
    reals.
  \item An antiderivative for $x \mapsto f'(1/x)$, domain positive
    reals.
  \item An antiderivative for $x \mapsto f(1/\sqrt{x})$, domain positive reals.
  \end{enumerate}

  {\em Answer}: Option (B)

  {\em Explanation}: We will show how an antiderivative for $f''(x)/x$
  is equivalent to all the antiderivatives in options (A), (C), (D),
  and (E).

  Option (A): Starting with $\int \frac{f''(x)}{x} \, dx$. Put $u =
  \ln x$. We get $\int f''(e^u) \, du$. Note that the domain now
  becomes all of $\R$. Replace the dummy variable $u$ by the dummy
  variable $x$, and we get $\int f''(e^x) \, dx$.

  Option (C): Let's start with $f'''(x)(\ln x)$. Integrate by parts
  taking $f'''(x)$ as the part to integrate. We get $\int f'''(x)(\ln
  x) \, dx = (\ln x)(f''(x)) - \int \frac{1}{x}f''(x) \, dx$. Thus, we
  see that the antiderivatives of $f'''(x)(\ln x)$ and $f''(x)/x$ add
  up to $f''(x)(\ln x)$, which is an elementarily expressible
  function, hence the antiderivatives are elementarily equivalent.

  Option (D): Start with $\int f'(1/x) \, dx$. Put $u = 1/x$ to get
  $\int \frac{-1}{u^2} f'(u)\,du$. Now integrate by parts taking
  $-1/u^2$ as the part to integrate, and we obtain a relationship with
  the integral of $f''(u)/u$.

  Option (E): Here, put $u = 1/\sqrt{x}$, so $x = 1/u^2$, giving $\int
  f(u)/u^3 \, du$. Integrate by parts twice taking the rational
  function as the part to integrate each time. We get $f''(u)/u$ (up
  to constants).

  {\em Performance review}: $8$ out of $11$ got this, $1$ each chose
  (A), (C), and (E).

  {\em Historical note}: $10$ out of $27$ people got this
  correct. $8$ people chose (C), $6$ people chose (E), and $3$ people
  chose (A).

\item Of the five functions below, four of them have antiderivatives
  that are equivalent up to elementary functions, i.e., an
  antiderivative for any one of them can be used to provide an
  antiderivative for the other three. The fifth function hais {\bf not
  equivalent} to any of these. Identify the fifth function.

  \begin{enumerate}[(A)]
  \item $x \mapsto e^{e^x}$, domain all reals
  \item $x \mapsto \ln(\ln x)$, domain $(1,\infty)$
  \item $x \mapsto e^x/x$, domain $(0,\infty)$
  \item $x \mapsto 1/(\ln x)$, domain $(1,\infty)$
  \item $x \mapsto 1/(\ln(\ln x))$, domain $(e,\infty)$
  \end{enumerate}

  {\em Answer}: Option (E)

  {\em Explanation}: We show the equivalence of all the others:

  (A) and (C): Starting with $\int e^{e^x} \, dx$, put $u = e^x$, to
  get $\int e^u/u \, du$. Note that the domain of definition transforms
  correctly.

  (C) and (D): Starting with $\int e^x/x \, dx$, put $u = e^x$, to get
  $du/(\ln u)$. Note that the domain of definition transforms
  correctly.

  (B) and (D): Start with $\int \ln(\ln x) \, dx$. Use integration by
  parts taking $1$ as the part to integrate. We get $x\ln(\ln x) -
  \int \frac{1}{\ln x} \, dx$, establishing the equivalence.

  {\em Performance review}: $7$ out of $11$ got this, $4$ chose (B).

  {\em Historical note}: $7$ out of $27$ people got this
  correct. $14$ people chose (C), $3$ people chose (B), $2$ people
  chose (D), $1$ person chose (A).

\item Which of the following functions has an antiderivative that is
  {\bf not equivalent} up to elementary functions to the
  antiderivative of $x \mapsto e^{-x^2}$?

  \begin{enumerate}[(A)]
  \item $x \mapsto e^{-x^4}$
  \item $x \mapsto e^{-x^{2/3}}$
  \item $x \mapsto e^{-x^{2/5}}$
  \item $x \mapsto x^2e^{-x^2}$
  \item $x \mapsto x^4e^{-x^2}$
  \end{enumerate}

  {\em Answer}: Option (A)

  {\em Explanation}: We show the equivalence with the others.

  Option (D): We use integration by parts, writing
  $x^2e^{-x^2}$ as $x \cdot (xe^{-x^2})$ and taking $xe^{-x^2}$ as the
  part to integrate, so that $x$ is the part to differentiate. An
  antiderivative for $xe^{-x^2}$ is $(-1/2)e^{-x^2}$, so we get:

  $$\frac{-x}{2}e^{-x^2} - \int \frac{-1}{2}e^{-x^2} \, dx$$

  We thus see that it reduces to $\int e^{-x^2} \, dx$.

  Option (E), via reduction to option (D): We use integration by
  parts, taking $x^3$ as the part to differentiate and $xe^{-x^2}$ as
  the part to integrate. One application of integration by parts
  reduces this to $\int x^2e^{-x^2}$, which is option (D).

  Option (B), via reduction to option (D): Start with $\int
  e^{-x^{2/3}} \, dx$. Put $u = x^{1/3}$. The substitution gives (up to
  scalars) $\int u^2e^{-u^2} \, du$, which is option (D).

  Option (C), via reduction to option (D): Start with $\int
  e^{-x^{2/5}} \, dx$. Put $u = x^{1/5}$. The substitution gives (up to
  scalars) $\int u^4e^{-u^2} \, du$, which is option (E).

  {\em Performance review}: $7$ out of $11$ got this, $3$ chose (E),
  $1$ chose (D).

  {\em Historical note}: $10$ out of $27$ people got this
  correct. $7$ people chose (D), $4$ people chose (C), $4$ people
  chose (E), and $2$ people chose (B).

\item Which of the following has an antiderivative that is not
  equivalent up to elementary functions to the antiderivative of the
  function $f(x) := e^x/x, x > 0$?

  \begin{enumerate}[(A)]
  \item $x \mapsto e^x/\sqrt{x}, x > 0$
  \item $x \mapsto e^x/x^2, x > 0$
  \item $x \mapsto e^x(\ln x), x > 0$
  \item $x \mapsto e^{1/\sqrt{x}}, x > 0$
  \item $x \mapsto e^{1/x}, x > 0$
  \end{enumerate}

  {\em Answer}: Option (A)

  {\em Explanation}: Option (B) is equivalent via one application of
  integration by parts. Option (C) is also equivalent via one
  application of integration by parts. Option (E) reduces to option
  (B) when we put $u = 1/\sqrt{x}$. Option (D) reduces to $e^x/x^3$,
  which is equivalent to option (B) via one application of integration
  by parts.

  {\em Performance review}: $4$ out of $11$ got this, $6$ chose (D),
  $1$ chose (C).

  {\em Historical note}: $5$ out of $27$ people got this
  correct. $9$ people chose (C), $10$ people chose (D), and $3$ people
  chose (E).
\end{enumerate}

\end{document}
