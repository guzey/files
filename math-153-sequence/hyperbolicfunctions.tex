\documentclass{amsart}
\usepackage{fullpage,hyperref,vipul, graphicx}
\title{Hyperbolic functions: an introduction}
\author{Math 153, Section 55 (Vipul Naik)}

\newcommand{\arccot}{\operatorname{arccot}}
\begin{document}
\maketitle

{\bf Corresponding material in the book}: Section 7.8.

{\bf What students should already know}: The definition of the
exponential function, the graphs of $e^x$ and $e^{-x}$, the derivative
and antiderivative of the exponential.

{\bf What students should definitely get}: The definitions of
hyperbolic sine and hyperbolic cosine, the fact that these are
derivatives of each other, the graphs of these functions, the key
identities involving hyperbolic sine and cosine and the general
procedure of proving them.

{\bf What students should hopefully get}: The analogy between
sine/cosine and hyperbolic sine/cosine, a general comprehension of
Osborne's rule.

{\bf Important note}: We are skipping, for now, the material in the
book titled ``Catenary'' and ``Relation to the hyperbola $x^2 - y^2 =
1$'' -- though we might talk about it later. You are encouraged to
read this material for your own curiosity.

\section*{Executive summary}

\begin{enumerate}
\item We define {\em hyperbolic cosine} $\cosh x := (e^x + e^{-x})/2$
  and {\em hyperbolic sine} $\sinh x := (e^x - e^{-x})/2$. $\cosh$ is
  the {\em even part} of the exponentiation function (and in
  particular, is an even function) while $\sinh$ is the {\em odd part}
  of the exponentiation function (and in particular, is an odd
  function).
\item $\cosh$ and $\sinh$ are derivatives of each other, and hence
  also antiderivatives of each other.
\item $\cosh$ is even and positive, decreasing on $(-\infty,0)$ and
  increasing on $(0,\infty)$, concave up throughout, goes to $\infty$
  as $x \to \pm \infty$, and its local and absolute minimum value of
  $1$ are attained at $0$.
\item $\sinh$ is odd, increasing on all of $\R$, negative and concave
  down on $(-\infty,0)$, and positive and concave up on
  $(0,\infty)$. It passes through $(0,0)$ where it has its unique
  point of inflection. Note that at $(0,0)$, the derivative takes its
  minimum value, which is $1$. In this important respect, the graph
  does {\em not} look like $x^3$, where we have a horizontal tangent
  at $x = 0$.
\item $\cosh^2 x - \sinh^2 x = 1$. A lot of the identities involving
  hyperbolic sine and hyperbolic cosine look very similar to the
  corresponding identities involving the trigonometric (circular) sine
  and cosine. In fact, we can move back and forth between the circular
  and the hyperbolic using the following rule: change the sign in
  front of any term that involves a product of two sine terms. This
  rule is termed {\em Osborne's rule}.
\end{enumerate}

\section{The hyperbolic sine and cosine}

\subsection{Definition and the key derivative property}

We define the following two functions from $\R$ to $\R$.

The {\em hyperbolic cosine}, denoted $\cosh$, is defined as:

$$\cosh (x) := \frac{e^x + e^{-x}}{2}$$

The {\em hyperbolic sine}, denoted $\sinh$, is defined as:

$$\sinh(x) := \frac{e^x - e^{-x}}{2}$$

The key facts about these two functions are:

\begin{enumerate}
\item $\cosh' = \sinh$. In other words, $\int \sinh(x) \, dx = \cosh x +
  C$.
\item $\sinh' = \cosh$. In other words, $\int \cosh(x) \, dx = \sinh x + C$.
\item $\cosh'' = \cosh$ and $\sinh'' = \sinh$: These follow from the
  above two.
\end{enumerate}

\subsection{Even and odd}

Hyperbolic cosine is an {\em even} function, i.e., $\cosh(-x) =
\cosh(x)$. This is because it is obtained by {\em averaging} the value
of $e^x$ and $e^{-x}$. Similarly, hyperbolic sine is an {\em odd}
function, i.e., $\sinh(-x) = -\sinh x$.

In this respect, hyperbolic cosine behaves like cosine, and hyperbolic
sine behaves like sine.

Recall, from a long time ago, that for any function $f$ defined on
$\R$, we can write $f$ as the sum of an even function $f_e$ and an odd
function $f_o$ in a {\em unique} way. Here, we define:

$$f_e(x) := \frac{f(x) + f(-x)}{2}$$

and

$$f_o(x) := \frac{f(x) - f(-x)}{2}$$

The even function $f_e$ is termed the {\em even part} of $f$ and the
function $f_o$ is termed the {\em odd part} of $f$. Further:

$$f(x) = f_e(x) + f_o(x), \qquad f(-x) = f_e(x) - f_o(x)$$

How does this relate to $\cosh$ and $\sinh$? If we take $f = \exp$,
i.e., $f(x) := e^x$, then the even part of $f$ is $\cosh$ and the odd
part of $f$ is $\sinh$. What we have done essentially is to split the
exponentiation function additively into its even and odd parts. In
particular, $\exp(x) = \cosh(x) + \sinh(x)$ and $\exp(-x) = \cosh(x) -
\sinh(x)$.

\subsection{Graphing the functions}

We make the following observations in sequence:

\begin{enumerate}
\item $\cosh$ is always positive, while $\sinh$ is positive for $x >
  0$, negative for $x < 0$, and zero at $x = 0$. Thus, while the graph
  of $\cosh$ is completely above the $x$-axis, the graph of $\sinh$ is
  above the $x$-axis for $x > 0$ and below the $x$-axis for $x <
  0$. Also, it passes through the origin.
\item The derivative of $\cosh$ is $\sinh$, which is positive for $x >
  0$ and negative for $x < 0$. Thus, the graph of $\cosh$ is
  decreasing on $(-\infty,0)$ and increasing on $(0,\infty)$. The
  derivative of $\sinh$ is $\cosh$, which is always positive. Thus the
  graph of $\sinh$ is increasing on all of $\R$.
\item The second derivative of $\cosh$ is $\cosh$, which is always
  positive. Thus, the graph of $\cosh$ is concave up on all of
  $\R$. The second derivative of $\sinh$ is $\sinh$, which is positive
  on $x > 0$ and negative on $x < 0$. Thus, the graph of $\sinh$ is
  concave down on $x < 0$ and concave up on $x > 0$.
\item The summary: $\cosh$ is an even function that approaches
  $+\infty$ as $x \to \pm \infty$, is concave up throughout, decreases
  on $(-\infty,0)$, increases on $(0,\infty)$, and has a unique local
  and absolute minimum at $0$ with value $1$. $\sinh$ is an odd
  function that approaches $-\infty$ at $-\infty$ and $+\infty$ at
  $+\infty$, increases throughout, is concave down on $(-\infty,0)$,
  is concave up on $(0,\infty)$, and has a unique point of inflection
  at $0$ with value $0$. Note that this is {\em not} a horizontal
  tangent-type point of inflection of the $x^3$ creed. Rather, the
  tangent line to this is $y = x$. The {\em minimum value} of the
  derivative of $\sinh$ is $1$, attained at $0$.
\end{enumerate}

Here are the graphs of the two functions:

\includegraphics[width=3in]{hyperbolicsineandcosine.png}

In addition to everything we've noted so far, it's also true that the
graphs of $\cosh$ and $\sinh$ are asymptotic to each other as $x \to
+\infty$ but {\em not} as $x \to -\infty$. To see this, note that
$\cosh x - \sin h = \exp(-x)$, which tends to $0$ as $x \to +\infty$
and tends to $+\infty$ as $x \to -\infty$.
\subsection{Basic identities}

\begin{eqnarray*}
  \cosh^2 x - \sinh^2 x & = & 1\\
  \cosh(x + y) & = & \cosh x \cosh y + \sinh x \sinh y\\
  \sinh(x + y) & = & \sinh x \cosh y + \cosh x \sinh y \\
  \cosh(x - y) & = & \cosh x \cosh y - \sinh x \sinh y \\
  \sinh(x - y) & = & \sinh x \cosh y - \cosh x \sinh y \\
  \sinh(2x) & = & 2 \sinh x \cosh x\\
  \cosh(2x) & = & 2\cosh^2 x - 1 = 2\sinh^2 x + 1 = \cosh^2 x + \sinh^2 x
\end{eqnarray*}

All of these identities can be proved by plugging in the definitions
of $\cosh$ and $\sinh$ in terms of $\exp$ and then using the
properties of $\exp$ to simplify the expression on both sides. For instance:

$$\cosh^2 x - \sinh^2 x = \left[\frac{e^x + e^{-x}}{2}\right]^2 - \left[\frac{e^x - e^{-x}}{2}\right]^2 = \frac{e^{2x} + 2 + e^{-2x}}{4} - \frac{e^{2x} - 2 + e^{-2x}}{4} = \frac{2 + 2}{4} = 1$$

\section{Relating the hyperbolic sine and cosine to the circular sine and cosine}

In this section, we use the term {\em circular} sine and cosine to
refer to the usual sine and cosine.

The hyperbolic sine and cosine have very similar behavior to the
circular sine and cosine in the following important respects:

\begin{enumerate}
\item {\em The mutual derivative relationship, along with the even-odd
  nexus}: The hyperbolic sine and cosine are derivatives of each
  other, with the former being even and the latter being odd. The
  circular cosine is the derivative of the circular sine and the
  circular sine is the {\em negative} derivative of the circular
  cosine.
\item {\em Very similar identities} such as $\cos^2 + \sin^2 = 1$
  being replaced by $\cosh^2 - \sinh^2 = 1$, as well as similar
  identities for sums and doubles.
\end{enumerate}

On the other hand, the graphs of the functions look very
different. $\sin$ and $\cos$ are periodic functions with a period of
$2\pi$, while $\sinh$ and $\cosh$ barely repeat ($\sinh$ is one-to-one
and $\cosh$ restricts to a one-to-one function on the nonnegative
reals). So what is really going on?

The answer to this is beyond our current scope, but it in fact has
something to do with complex numbers. It turns out that we can extend
all these definitions to complex inputs, and with that extended
definition, we have $\cos(x) = \cosh(ix)$ and $\sin(x) = \sinh(ix)/i$,
where $i$ is a non-real squareroot of $-1$. The fact that $i^2 = -1$
is the key reason for both the similarities and differences.

This gives rise to a rule called {\em Osborne's rule}: To convert an
identity involving the circular sine and cosine to a corresponding
identity involving hyperbolic sine and cosine, change the sign on all
terms that involve the product of two sines. For instance, in the
identity $\cos^2 x + \sin^2 x = 1$, the $\sin^2$ term is the product
of two $\sin$ terms, so the sign on this is changed. (The precise rule
is more sophisticated, but this is good enough). It is a good exercise
to use this rule to relate each of the hyperbolic sine/cosine
identities and the corresponding circular sine/cosine identity.

\end{document}