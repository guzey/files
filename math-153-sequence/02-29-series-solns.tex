\documentclass[10pt]{amsart}

%Packages in use
\usepackage{fullpage, hyperref, vipul, enumerate}

%Title details
\title{Class quiz solutions: February 29: Series}
\author{Math 153, Section 55 (Vipul Naik)}
%List of new commands

\begin{document}
\maketitle

\section{Performance review}

$11$ people took this $10$-question quiz. The score distribution was
as follows:

\begin{itemize}
\item Score of $3$: $2$ people
\item Score of $4$: $1$ person
\item Score of $5$: $4$ people
\item Score of $6$: $2$ people
\item Score of $7$: $2$ people
\end{itemize}


Below are the question wise answers and performance review:

\begin{enumerate}
\item Option (C): $9$ people
\item Option (B): $3$ people
\item Option (E): $5$ people
\item Option (B): $10$ people
\item Option (A): $7$ people
\item Option (D): $5$ people
\item Option (E): $3$ people
\item Option (D): $3$ people
\item Option (E): $5$ people
\item Option (C): $6$ people
\end{enumerate}

\section{Solutions}

\begin{enumerate}

\item Suppose $p$ is a polynomial that take positive values on all
  nonnegative integers. Consider the summation $\sum_{k=1}^\infty
  \frac{(k^2 + 1)^{2/3}}{p(k)}$. Under what conditions does the
  summation converge? Note that the degree of $p$ must be a
  nonnegative integer.

  \begin{enumerate}[(A)]
  \item The summation converges if and only if the degree of $p$ is
    {\em at least} one
  \item The summation converges if and only if the degree of $p$ is
    {\em at least} two
  \item The summation converges if and only if the degree of $p$ is
    {\em at least} three
  \item The summation converges if and only if the degree of $p$ is
    {\em at most} two
  \item The summation converges if and only if the degree of $p$ is
    {\em at most} one
  \end{enumerate}

  {\em Answer}: Option (C)

  {\em Explanation}: This is a straightforward application of the
  degree difference rule in its more general form. The numerator has
  degre $4/3$, so the degree difference is $\operatorname{deg}(p) -
  (4/3)$. This difference needs to be greater than $1$ for the series
  to converge, so we need that the degree of the denominator is
  strictly greater than $7/3$. The smallest integer greater than $7/3$
  is $3$, so that is the answer.

  {\em Performance review}: $9$ out of $11$ got this. $2$ chose (B).

\item Suppose $p$ is a polynomial that take positive values on all
  nonnegative integers. Consider the summation $\sum_{k=1}^\infty
  \frac{(-1)^k(k^2 + 1)^{2/3}}{p(k)}$. Under what conditions does the
  summation converge? Note that the degree of $p$ must be a
  nonnegative integer.

  \begin{enumerate}[(A)]
  \item The summation converges if and only if the degree of $p$ is
    {\em at least} one
  \item The summation converges if and only if the degree of $p$ is
    {\em at least} two
  \item The summation converges if and only if the degree of $p$ is
    {\em at least} three
  \item The summation converges if and only if the degree of $p$ is
    {\em at most} two
  \item The summation converges if and only if the degree of $p$ is
    {\em at most} one
  \end{enumerate}

  {\em Answer}: Option (B)

  {\em Explanation}: The previous question tells us that if the degree
  of $p$ is at least three, then the series of absolutely convergent.

  If the degree of $p$ is two, then the series is conditionally
  convergent. To see this, note that $p(k)$ is always positive, so the
  summation is an alternating series summation. Also, the terms are
  eventually decreasing in magnitude, and they go to zero. Thus, by
  the alternating series theorem, the series converges. On the other
  hand, the degree difference rule tells us that it does not converge
  absolutely.

  If the degree of $p$ is one or zero, then the terms of the series do
  not approach zero, so the series does not converge.

  {\em Performance review}: $3$ out of $11$ got this. $7$ chose (C),
  $1$ chose (E).

\item Which of the following series converges? Assume for all
  series that the starting point of summation is large enough that the
  terms are well defined.

  \begin{enumerate}[(A)]
  \item $\sum 1/(k \ln (\ln k))$
  \item $\sum 1/(k \ln k)$
  \item $\sum 1/(k (\ln (\ln k))^2)$
  \item $\sum 1/(k (\ln k)(\ln (\ln k)))$
  \item $\sum 1/(k (\ln k)(\ln (\ln k))^2)$
  \end{enumerate}

  {\em Answer}: Option (E)

  {\em Explanation}: Options (B) and (D) diverge by the integral
  test. As for options (A) and (C), these have smaller denominators,
  hence larger terms, than option (B), hence, by basic comparison,
  these diverge too. This leaves option (E), which converges by the
  integral test.

  {\em Performance review}: $5$ out of $11$ got this. $4$ chose (B),
  $2$ chose (D).

  {\em Historical note (last year)}: $11$ out of $25$
  people got this correct. $7$ chose (C), $3$ each chose (A) and (B),
  and $1$ left the question blank.

  The main attraction of (C) seems to have been its superficial
  resemblance to $1/(k(\ln k)^2)$ which does converge.

\item Which of the following series converges?

  \begin{enumerate}[(A)]
  \item $\sum \frac{k + \sin k}{k^2 + 1}$
  \item $\sum \frac{k + \cos k}{k^3 + 1}$
  \item $\sum \frac{k^2 - \sin k}{k + 1}$
  \item $\sum \frac{k^3 + \cos k}{k^2 + 1}$
  \item $\sum \frac{k}{\sin(k^3 + 1)}$
  \end{enumerate}

  {\em Answer}: Option (B)

  {\em Explanation}: We can use a comparison test, either rigorously
  or in the form of a heuristic of looking at degree of denominator
  minus degree of numerator. Note that for (A), the degree difference
  is $1$, so it diverges. For (C) and (D), the numerator actually has
  larger degree than the denominator, so it diverges. For (E), the
  denominator is bounded in $[-1,1]$, and the numerator goes to
  $\infty$, so it diverges. This leaves (B), which converges because
  the degree of denominator minus degree of numerator equals $2$.

  {\em Performance review}: $10$ out of $11$ got this. $1$ chose (A).

  {\em Historical note (last year)}: $23$ out of $25$ people got this
  correct. $1$ person chose (C) and $1$ person chose (D).

\item Consider the series $\sum_{k=0}^\infty \frac{1}{2^{2^k}}$. What
  can we say about the sum of this series?

  \begin{enumerate}[(A)]
  \item It is finite and strictly between $0$ and $1$.
  \item It is finite and equal to $1$.
  \item It is finite and strictly between $1$ and $2$.
  \item It is finite and equal to $2$.
  \item It is infinite.
  \end{enumerate}

  {\em Answer}: Option (A)

  {\em Explanation}: The summation goes like:

  $$\frac{1}{2} + \frac{1}{4} + \frac{1}{16} + \frac{1}{256} + \dots$$

  Notice that the series being summed is a subseries of the series:

  $$\frac{1}{2} + \frac{1}{4} + \frac{1}{8} + \frac{1}{16} + \dots$$

  In particular, the sum of the former is less than the sum of the
  latter. The latter sums up to $1$, so the sum of the former is less
  than $1$. Also, since the first term is $1/2$, it must be greater
  than $1/2$. Thus, the series sum is between $1/2$ and $1$. Option
  (A) is the best fit.

  {\em Performance review}: $7$ out of $11$ got this. $3$ chose (C),
  $1$ chose (B).

  {\em Historical note (last year)}: $14$ out of $26$ people got this
  correct. $9$ chose (C) (possibly because of getting the first term
  wrong?), $2$ chose (E), $1$ chose (B).

\item For one of the following functions $f$ on $(0,\infty)$, the
  integral $\int_0^\infty f(x) \, dx$ converges but $\int_0^\infty
  |f(x)| \, dx$ does not converge. What is that function $f$? (Note
  that this is similar to, but not quite the same as, the absolute
  versus conditional convergence notion for series).

  \begin{enumerate}[(A)]
  \item $f(x) = \sin x$
  \item $f(x) = \sin(\sin x)$
  \item $f(x) = (\sin \sqrt{x})/\sqrt{x}$
  \item $f(x) = (\sin x)/x$
  \item $f(x) = (\sin^3x)/x^3$
  \end{enumerate}

  {\em Answer}: Option (D)

  {\em Explanation}: For options (A) and (B), the integral
  $\int_0^\infty f(x) \, dx$ does not converge. The reason is simple:
  in neither case is it true that $\lim_{x \to \infty} f(x) = 0$. The
  function itself going to zero is a necessary (but not sufficient)
  condition for the integral to converge.

  For option (C), the antiderivative for $f$ is $-2\cos\sqrt{x}$,
  evaluated between limits $0$ and $\infty$. However, the limit for
  the antiderivative at $\infty$ does not exist, hence the integral
  does not converge. 

  This leaves options (D) and (E). For option (D), it is a well known (?) fact that:

  $$\int_0^\infty \frac{\sin x \, dx}{x} = \frac{\pi}{2}$$

  Hence, the integral does converge. However, if we consider the integral:

  $$\int_0^\infty \frac{|\sin x| \, dx}{|x|}$$

  This integral does not converge, a fact that we can prove by
  bounding it in terms of the summation $\sum_{n=1}^\infty 1/n$, which
  diverges.

  Finally, for option (E), both $\int_0^\infty f(x) \, dx$ and
  $\int_0^\infty |f(x)| \, dx$ converge. To see this, first split the
  integral as $\int_0^1 + \int_1^\infty$. The former integral is
  finite because it is integrating a bounded function over a bounded
  interval (note that the limit of the function at $0$ is $1$). The
  latter integral is finite because we can compare it to $1/x^3$ which
  has a finite integral. The reasoning works for both $f$ and $|f|$,
  so we are done.

  {\em Performance review}: $5$ out of $11$ got this. $4$ chose (C),
  $2$ chose (D).

\item Consider the function $F(x,p) := \sum_{n=1}^\infty
  \frac{x^n}{n^p}$ with $x$ and $p$ both real numbers. For what values
  of $x$ and what values of $p$ does this summation converge?
  \begin{enumerate}[(A)]
  \item For $|x| < 1$, it converges for all $p \in \R$. For $|x| \ge
    1$, it does not converge for any $p$.
  \item For $|x| \le 1$, it converges for all $p \in \R$. For $|x| >
    1$, it does not converge for any $p$.
  \item For $|x| < 1$, it converges for all $p \in \R$. For $|x| > 1$,
    it does not converge for any $p$. For $|x| = 1$, it converges if
    and only if $p > 1$.
  \item For $|x| < 1$, it converges for all $p \in \R$. For $|x| > 1$,
    it does not converge for any $p$. For $x = 1$, it converges
    if and only if $p > 0$. For $x = -1$, it converges if and only if
    $p > 1$.
  \item For $|x| < 1$, it converges for all $p \in \R$. For $|x| > 1$,
    it does not converge for any $p \in \R$. For $x = 1$, it converges
    if and only if $p > 1$. For $x = -1$, it converges if and only if
    $p > 0$.
  \end{enumerate}

  {\em Answer}: Option (E)

  {\em Explanation}: When $|x| < 1$, then the series is absolutely
  convergent and when $|x| > 1$ it diverges, as we can see by the root
  test or ratio test. What's happening is that $(1/n^p)^{1/n} \to 1$
  (regardless of $p$), so the radius of convergence is $1$.

  This leaves the case $|x| = 1$. If $x = 1$, then we get the usual
  $p$-series, which we know converges iff $p > 1$. If $x = -1$, then
  the terms have alternating signs. Obviously, the series cannot
  converge for $p \le 0$ because the terms do not tend to $0$. For $p
  > 0$, on the other hand, the terms are alternating in sign and
  decrease monotonically, tending to $0$. Thus, by the alternating
  series theorem, it converges for $p > 0$.

  {\em Performance review}: $3$ out of $11$ got this. $4$ chose (C),
  $2$ each chose (B) and (D).

  {\em Historical note (last year)}: $7$ out of $26$ people got this
  correct. $10$ chose (C), $4$ chose (D), $3$ chose (A), $2$ chose
  (B). The large vote for (C) indicates that many people did not
  notice the special application of the alternating series theorem to
  the case of $x = -1$.

  {\em Action point}: Please review what happens in the case $x =
  -1$. This will be covered in more detail in class when we study the
  notion of interval of convergence of a power series.

  There is a result of calculus which states that, under suitable
  conditions, if $f_1, f_2, \dots, f_n, \dots$ are all functions, and
  we define $f(x) := \sum_{n=1}^\infty f_n(x)$, then $f'(x) =
  \sum_{n=1}^\infty f_n'(x)$. In other words, under suitable
  assumptions, we can differentiate a sum of countably many functions
  by differentiating each of them and adding up the derivatives.

  We will not be going into what those assumptions are, but will
  consider some applications where you are explicitly told that these
  assumptions are satisfied.


\item Consider the summation $\zeta(p) := \sum_{n=1}^\infty
  \frac{1}{n^p}$ for $p > 1$. Assume that the required assumptions are
  valid for this summation, so that $\zeta'(p)$ is the sum of the
  derivatives of each of the terms (summands) with respect to
  $p$. What is the correct expression for $\zeta'(p)$?
  
  \begin{enumerate}[(A)]
  \item $\sum_{n=1}^\infty \frac{-p}{n^{p+1}}$
  \item $\sum_{n=1}^\infty \frac{-1}{(p+1)n^{p+1}}$
  \item $\sum_{n=1}^\infty \frac{p}{n^{p-1}}$
  \item $\sum_{n=1}^\infty \frac{-\ln n}{n^p}$
  \item $\sum_{n=1}^\infty \frac{-\ln n}{n^{p+1}}$
  \end{enumerate}

  {\em Answer}: Option (D)

  {\em Explanation}: We need to differentiate $(1/n)^p$ with respect
  to $p$. This is the same as differentiating $a^x$ with respect to
  $x$, which gives $a^x \ln a$. In our case, we get $(1/n)^p \ln(1/n)$
  which is $(-\ln n)/n^p$.

  Note that Option (A) arises if we try to differentiate formally with
  respect to $n$, which is not the correct operation at all. $n$ is a
  dummy variable and the expression should be differentiated with
  respect to $p$.

  {\em Performance review}: $3$ out of $11$ got this. $8$ chose
  (A). This indicates that many people differentiated with respect to
  the wrong variable.

  {\em Historical note (last year)}: $8$ out of $26$ people got this
  correct. $13$ chose (A), $3$ chose (B), $1$ chose (C), and $1$ left
  the question blank. The most commonly chosen wrong option, (A),
  indicates that many people differentiated with respect to the wrong
  variable.

\item Going back to question 2, recall that we defined $F(x,p) :=
  \sum_{n=1}^\infty \frac{x^n}{n^p}$ with $x$ and $p$ both real
  numbers. Assume that, for a particular fixed value of $p$, the
  summation satisfies the conditions as a function of $x$ for $|x| <
  1$. What is its derivative with respect to $x$, keeping $p$
  constant?
  \begin{enumerate}[(A)]
  \item $\sum_{n=1}^\infty \frac{x^{n+1}}{n^{p+1}}$
  \item $\sum_{n=1}^\infty \frac{x^{n+1}}{n^{p-1}}$
  \item $\sum_{n=1}^\infty \frac{x^{n-1}}{n^{p+1}}$
  \item $\sum_{n=1}^\infty \frac{x^{n-1}\ln n}{n^{p+1}}$
  \item $\sum_{n=1}^\infty \frac{x^{n-1}}{n^{p-1}}$
  \end{enumerate}

  {\em Answer}: Option (E)

  {\em Explanation}: We need to differentiate each term with respect
  to $x$. Differentiating $x^n/n^p$ with respect to $x$ gives
  $nx^{n-1}/n^p$, which, upon rearrangement, gives $x^{n-1}/n^{p-1}$.

  {\em Performance review}: $5$ out of $11$ got this correct. $3$ each
  chose (C) and (D).

 {\em Historical note (last year)}: $14$ out of $26$ got this
  correct. $4$ each chose (B), (C), (D), possibly indicating minor
  computational errors.

\item The series $\sum_{n=1}^\infty \frac{1}{n}$ diverges. Since it is
  a series of positive terms, this means that the partial sums get
  arbitrarily large. What is the approximate smallest value of $N$
  such that $\sum_{n=1}^N \frac{1}{n} > 100$?
  \begin{enumerate}[(A)]
  \item Between $90$ and $110$
  \item Between $2000$ and $3000$
  \item Between $10^{40}$ and $10^{50}$
  \item Between $10^{90}$ and $10^{110}$
  \item Between $10^{220}$ and $10^{250}$
  \end{enumerate}

  {\em Answer}: Option (C)

  {\em Explanation}: We can see that $\sum_{n=1}^N 1/n$ is
  approximately $\ln N$. More precisely, we can use the standard
  methods for comparising integrals and summations and obtain that the
  finite sum is between $\ln N$ and $1 + \ln N$. In particular, the
  $N$ that works must have $\ln N$ between $99$ and $100$. Thus,
  $\log_{10}N$ is between $99/(\ln 10)$ and $100/(\ln 10)$. Both these
  numbers are between $40$ and $50$, so Option (C) is the correct
  choice.

  {\em Performance review}: $6$ out of $11$ got this correct. $2$ each
  chose (D) and (E), $1$ chose (A).

  {\em Historical note (last year)}: $14$ out of $26$ got this correct. $6$
  chose (D), $2$ chose (B), $2$ chose (E), $1$ chose (A), and $1$ left
  the question blank.

\end{enumerate}

\end{document}
