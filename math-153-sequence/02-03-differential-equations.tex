\documentclass[10pt]{amsart}

%Packages in use
\usepackage{fullpage, hyperref, vipul, enumerate}

%Title details
\title{Class quiz: February 3: Differential equations}
\author{Math 153, Section 55 (Vipul Naik)}
%List of new commands

\begin{document}
\maketitle

Your name (print clearly in capital letters): $\underline{\qquad\qquad\qquad\qquad\qquad\qquad\qquad\qquad\qquad\qquad}$

\begin{enumerate}
\item It takes time $T$ for $1/10$ of a radioactive substance to
  decay. How much does it take for $3/10$ of the same substance to
  decay? {\em Last yer: $22/26$ correct}
  
  \begin{enumerate}[(A)]
  \item Between $T$ and $2T$
  \item Between $2T$ and $3T$
  \item Exactly $3T$
  \item Between $3T$ and $4T$
  \item Between $4T$ and $5T$
  \end{enumerate}

  \vspace{0.1in}
  Your answer: $\underline{\qquad\qquad\qquad\qquad\qquad\qquad\qquad}$
  \vspace{0.15in}

\item (*) Suppose a function $f$ satisfies the differential equation
  $f''(x) = 0$ for all $x \in \R$. Which of the following is true
  about $\lim_{x \to \infty} f(x)$ and $\lim_{x \to -\infty} f(x)$?
  {\em Last year: $13/26$ correct}

  \begin{enumerate}[(A)]
  \item If either limit is finite, then both are finite and they are
    equal. Otherwise, both the limits are infinities of opposite
    signs.
  \item If either limit is finite, then both are finite and they are
    equal. Otherwise, both the limits are infinities of the same sign.
  \item One of the limits is finite and the other is infinite.
  \item Both the limits are finite and unequal.
  \item Both the limits are infinite but they may be of the same or of
    opposite signs.
  \end{enumerate}

  \vspace{0.1in}
  Your answer: $\underline{\qquad\qquad\qquad\qquad\qquad\qquad\qquad}$
  \vspace{0.15in}

\item (*) For $y$ a function of $x$, consider the differential
  equation $(y')^2 - 3yy' + 2y^2 = 0$. What is the description of the
  {\bf general solution} to this differential equation? {\em Last
  year: $12/26$ correct}

  \begin{enumerate}[(A)]
  \item $y = C_1e^x + C_2e^{2x}$, where $C_1$ and $C_2$ are arbitrary
    real numbers.
  \item $y = C_1e^x + C_2e^{2x}$, where $C_1$ and $C_2$ are real
    numbers satisfying $C_1C_2 = 0$ (i.e., at least one of them is
    zero)
  \item $y = C_1e^x + C_2e^{2x}$, where $C_1$ and $C_2$ are real
    numbers satisfying $C_1 + C_2 = 0$.
  \item $y = C_1e^x + C_2e^{2x}$, where $C_1$ and $C_2$ are real
    numbers satisfying $C_1C_2 = 1$.
  \item $y = C_1e^x + C_2e^{2x}$, where $C_1$ and $C_2$ are real
    numbers satisfying $C_1 + C_2 = 1$.
  \end{enumerate}

  \vspace{0.1in}
  Your answer: $\underline{\qquad\qquad\qquad\qquad\qquad\qquad\qquad}$
  \vspace{0.15in}

\item (*) Suppose $F(t)$ represents the number of gigabytes of disk
  space that can be purchased with one dollar at time $t$ in
  commercially available disk drive formats (not adjusted for
  inflation). Empirical observation shows that $F(1980) \approx 5 *
  10^{-6}$, $F(1990) \approx 10^{-4}$, $F(2000) \approx 10^{-1}$, and
  $F(2010) \approx 10$. From these data, what is a good estimate for
  the ``doubling time'' of $F$, i.e., the time it takes for the number
  of gigabytes purchaseable with a dollar to double? {\em Last year:
  $10/26$ correct}
  \begin{enumerate}[(A)]
  \item Between $6$ months and $1$ year.
  \item Between $1$ year and $2$ years.
  \item Between $2$ years and $4$ years.
  \item Between $4$ years and $5$ years.
  \item Between $5$ years and $6$ years.
  \end{enumerate}

  \vspace{0.1in}
  Your answer: $\underline{\qquad\qquad\qquad\qquad\qquad\qquad\qquad}$
  \vspace{0.15in}

\item (*) The size $S$ of an online social network satisfies the
  differential equation $S'(t) = kS(t)(1 - (S(t))/(W(t)))$ where
  $W(t)$ is the world population at time $t$. Suppose $W(t)$ itself
  satisfies the differential equation $W'(t) = k_0W(t)$ where $k_0$ is
  positive but much smaller than $k$. How would we expect $S$ to
  behave, assuming that initially, $S(t)$ is positive but much smaller
  than $W(t)$? {\em Last year: $11/26$ correct}

  \begin{enumerate}[(A)]
  \item It initially appears like an exponential function with
    exponential growth rate $k$, but over time, it slows down to
    (roughly) an exponential function with exponential growth rate
    $k_0$.
  \item It initially appears like an exponential function with
    exponential growth rate $k_0$, but over time, it speeds up to
    (roughly) an exponential function with exponential growth rate $k$.
  \item It behaves roughly like an exponential function with growth
    rate $k_0$ for all time.
  \item It behaves roughly like an exponential function with growth
    rate $k$ for all time.
  \item It initially behaves like an exponential function with
    exponential growth rate $k$ but then it starts declining.
  \end{enumerate}

  \vspace{0.1in}
  Your answer: $\underline{\qquad\qquad\qquad\qquad\qquad\qquad\qquad}$
  \vspace{0.15in}

\item (**) Suppose the growth of a population $P$ with time is
  described by the equation $dP/dt = aP^{1 - \beta}$ with $a > 0$ and
  $0 < \beta < 1$. What can we say about the nature of the population
  as a function of $t$, assuming that the population at time $0$ is
  positive? {\em Last year: $8/26$ correct}

  \begin{enumerate}[(A)]
  \item The population grows as a sub-linear power function of $t$,
    i.e., roughly like $t^\gamma$ where $0 < \gamma < 1$.
  \item The population grows as a linear power function of $t$, i.e.,
    roughly like $t$.
  \item The population grows as a superlinear power function of $t$,
    i.e., roughly like $t^\gamma$ where $\gamma > 1$.
  \item The population grows like an exponential function of $t$,
    i.e., roughly like $e^{kt}$ for some $k > 0$.
  \item The population grows super-exponentially, i.e., it eventually
  surpasses any exponential function.
  \end{enumerate}

  \vspace{0.1in}
  Your answer: $\underline{\qquad\qquad\qquad\qquad\qquad\qquad\qquad}$
  \vspace{0.15in}

\item (**) Suppose the growth of a population $P$ with time is described by
  the equation $dP/dt = aP^{1 + \theta}$ with $0 < \theta$ {\em and $a
  > 0$}. What can we say about the nature of the population as a
  function of $t$, assuming that the population at time $0$ is
  positive? {\em Last year: $3/26$ correct}

  \begin{enumerate}[(A)]
  \item The population approaches infinity in finite time, and the
    differential equation makes no sense beyond that.
  \item The population increases at a decreasing rate and approaches a
    horizontal asymptote, i.e., it proceeds to a finite limit as time
    approaches infinity.
  \item The population grows linearly.
  \item The population grows super-linearly but sub-exponentially.
  \item The population grows exponentially.
  \end{enumerate}

  \vspace{0.1in}
  Your answer: $\underline{\qquad\qquad\qquad\qquad\qquad\qquad\qquad}$
  \vspace{0.15in}


\item (**) Let $r(t)$ denote the fractional growth rate per annum in
  per capita income, which we denote by $I(t)$. In other words, $r(t)
  = I'(t)/I(t)$, measured in units of (per year). It is observed that,
  over a certain time period, $r'(t) = kr(t)$ for a positive constant
  $k$. Assuming that the initial values of $I(t)$ and $r(t)$ are
  positive, what best describes the nature of the function $I(t)$?
  {\em Last year: $2/26$ correct}

  \begin{enumerate}[(A)]
  \item $I(t)$ is a linear function of $t$, i.e., per capita income is
    getting incremented by a constant {\em amount} (rather than a
    constant proportion).
  \item $I(t)$ is a super-linear but sub-exponential function of $t$,
    i.e., per capita income is rising, but less than exponentially.
  \item $I(t)$ is an exponential function of $t$, i.e., per capita
    income is rising by a constant proportion per year.
  \item $I(t)$ is a super-exponential function of $t$ but slower than
    a doubly exponential function of $t$.
  \item $I(t)$ is a doubly exponential function of $t$.
  \end{enumerate}

  \vspace{0.1in}
  Your answer: $\underline{\qquad\qquad\qquad\qquad\qquad\qquad\qquad}$
  \vspace{0.15in}

\end{enumerate}

\end{document}
