\documentclass[10pt]{amsart}
\usepackage{fullpage,hyperref,vipul}
\title{Continuous and discrete: the interplay}
\author{Math 153, Section 55 (Vipul Naik)}

\begin{document}
\maketitle

This is some additional material that does not directly correspond to
material in the book but is helpful for perspective.

\section*{Executive summary}

Words ...

\begin{enumerate}
\item Given a function on $\R$, we can restrict the function to $\N$
  and obtain a sequence. This restriction is unique.
\item Conversely, given a sequence, i.e., a function on $\N$, we can
  extend it to a continuous function on $\R$. However, the extension
  is not unique, and there are a lot of different ways of
  extending. If the sequence is described by means of a nice closed
  form functional expression, we may be able to extend it by
  considering that functional expression for all real numbers.
\item Usually, information about the function on the reals gives us
  corresponding information about the corresponding sequence, but we
  cannot get information in the reverse direction that easily. For
  instance, an increasing function gives an increasing sequence, but
  increasing sequences can arise from functions that are not
  increasing. A decreasing function gives a decreasing sequence, a
  monotonic function gives a monotonic sequence, and a bounded
  function gives a bounded sequence.
\item A function with integer period gives a periodic sequence.
\item The mean value theorem relates the derivative of a function to
  the discrete derivative (i.e., forward difference operator) of the
  corresponding sequence.
\item We can define a notion of concave up and concave down for
  sequences based on the second discrete derivative. If a function is
  concave up, so is the corresponding sequence. If the function is
  concave down, so is the corresponding sequence.
\end{enumerate}

\section{Filling the holes: continuating the discrete}

In the previous lecture or part of lecture, we developed the analogy
between the calculus apparatus that we developed for continuous
functions and the study of sequences. The analogy was illuminative,
but is it more than just an analogy? Specifically, we know that:

\begin{enumerate}
\item {\em Continuous to discrete}: Any function on the reals can be
  domain-restricted to give a function on the natural numbers, i.e., a
  sequence.
\item {\em Discrete to continuous}: If a function on the natural
  numbers has a nice enough closed form expression (such as a
  polynomial) that closed form expression can be extended to more real
  numbers.
\end{enumerate}

Some natural questions are: how does the discrete derivative (i.e.,
the forward difference operator) relate to its continuous counterpart
for the same function? How does the discrete integral (which is
basically just a summing up) relate to its continuous counterpart? The
answers to these questions, while not completely satisfying, are
useful.

\subsection{Filling the gaps}

Recall that $\N \subset \Z \subset \Q \subset \R$. Note that if a
function is defined on a {\em bigger} set among these, it can always
be restricted to a smaller set in a unique manner -- just restrict
inputs to that smaller set. {\em Extending} functions is a more
interesting question: can a function on a smaller set be extended to a
bigger set in a nice way? The answers are as follows:

\begin{enumerate}
\item If $f$ is a function on $\Q$, there is at most one way of
  extending $f$ {\em continuously} to all of $\R$. The uniqueness of
  the extension arises from the fact that $\Q$ is dense in $\R$, so
  where a particular real number goes is determined by where its
  rational neighbors go. However, it may not always be possible to
  extend. For instance, $f(x) := 1/(x^2 - 2)$ is defined and
  continuous on $\Q$, but cannot be extended to the point $\sqrt{2}
  \in \R$ continuously.
\item If $f$ is a function on $\Z$ (or on $\N$), there are many ways
  of extending it to a continuous function on $\R$. Basically, all we
  have to do is join up the function values at consecutive integers by
  continuous curves. With a few kindergarten lessons in smooth
  drawing, we can even make sure that the function on $\R$ that we
  obtain is {\em continuously differentiable}, and with more practice
  yet, we can ensure that the function that we obtain is {\em
  infinitely differentiable}. Basically, there is a lot of freedom.
\end{enumerate}

Since there are infinitely many ways of extending functions on $\N$ to
continuous and even to infinitely differentiable functions on $\R$, we
are faced with a paradox of choice. Nonetheless, in most cases, the
nicest closed form expression for the function on $\N$ suggests an
obvious extension to $\R$. Thus, for a polynomial sequence, the
extension to a {\em polynomial} function on $\R$ is unique.

In some cases, we need to be more ingenuous in finding a natural
extension to $\R$. Luckily, calculus has provided us with enough tools
to find these functions in many cases. We discuss some examples.

\subsection{Exponential and superexponential sequences}

Consider the sequence:

$$1,2,4,8,16,32,\dots$$

The recurrence relation is clear:

$$a_n = 2a_{n-1}, \qquad a_1 = 1$$

We can also see that a closed form expression is:

$$a_n = 2^{n-1}$$

This kind of sequence is termed a {\em geometric sequence} or {\em
geometric progression} because the quotient of successive terms is
constant. It is a solution to the discrete differential equation $a_n
= 2a_{n-1}$, whose continuous analogue is $y' = ky$. Both the
continuous and the discrete versions result in exponential growth.

Note that in this case, the solution can be extended to all real
numbers, via the function:

$$f(x) := 2^{x-1}$$

But we were able to do this {\em only because we had developed a prior
theory of exponentiation} with arbitrary positive bases and arbitrary
real exponents. Had we not developed such a theory, it would not have
been clear how we could extend the function.

Consider instead the function:

$$-1,1,-1,1,\dots$$

The $n^{th}$ term is given by:

$$a_n = (-1)^n$$

Now, this particular expression {\em cannot} be extended to all real
numbers, because $(-1)^x$ does not make sense for arbitrary
$x$. However, there is another way of defining the $n^{th}$ term,
namely, as:

$$a_n = \cos(n\pi)$$

With this expression, we can extend it to all real numbers, as the function:

$$f(x) := \cos(\pi x)$$

This extension is not {\em natural} in any strong sense of the
word. Other possible etensions to the reals include $\cos(3\pi x)$,
$\cos(5\pi x)$ and $\cos^3(\pi x)$.

Consider another example:

$$1,2,6,24,120,720,5040,\dots$$

The recurrence relation here is:

$$a_n = na_{n-1}, \qquad a_1 = 1$$

Note that this is {\em not} autonomous. We solve it to obtain that:

$$a_n = n \cdot (n - 1) \cdot \dots 2 \cdot 1 =: n!$$

We thus have a nice expression for $a_n$. However, unlike the previous
case, it is not clear how this can be extended to a function on all
real numbers, or even on {\em anything} that is not a positive
integer. The reason is that {\em we have not pondered this question
before}. However, a recent homework problem you did shows that:

$$\int_0^\infty x^ne^{-x} = n!$$

This actually gives a way of extending the function to all positive
real numbers, and in fact, to all real numbers greater than
$-1$. Basically, although the {\em right side} makes sense only for
positive integers, the {\em left side} makes sense in a much broader
context. This is closely related to the gamma function which is a
mainstay of analysis and statistics, defined as $\Gamma(a) := (a -
1)!$. We will not delve more into it, except to remark that
$\Gamma(1/2) = (-1/2)! = \sqrt{\pi}$, and that this can be derived
from the results about the integral of $e^{-x^2}$. 

\section{Back and forth between discrete and continuous}

\subsection{Continuous and discrete: relationship of the functions}

When we restrict the domain to the natural numbers for a function
defined on the reals, then what we are really doing is taking a very
restrictive snapshot of the function. Some observations:

\begin{enumerate}
\item If $f$ is increasing on the reals, its restriction to the
  natural numbers is increasing. However, the converse does not
  hold. In other words, a function may be increasing on $\N$ but may
  be a lot more desultory on the real numbers. For instance, consider
  the function $f(x) := x - \sin(\pi x)$. Its restriction to the
  natural numbers gives the sequence $1,2,3,\dots$ which is
  increasing. However, the function is not increasing throughout $\R$,
  because the derivative $f'(x) = 1 - \pi \cos(\pi x)$ takes both
  positive and negative values. An analogous statement holds for
  decreasing functions, and analogous examples work.
\item If $\lim_{x \to \infty} f(x)$ exists, then the limit of the
  restriction to natural numbers, i.e., $\lim_{n \to \infty} f(n)$
  exists and is equal to $\lim_{x \to \infty} f(x)$. However, the
  existence of the limit $\lim_{n \to \infty} f(n)$ for $n \in \N$
  does {\em not} imply the existence of the limit for all $x$. For
  instance, consider the function $f(x) := \sin (\pi x)$. This is a
  {\em constant} function with value $0$ when restricted to $\N$ and
  hence limits to $0$. On the other hand, the function on $\R$ is
  periodic and oscillatory and has no limit.
\end{enumerate}

\subsection{Continuous and discrete: relationship of the derivatives}

We now turn to the question of how the derivative of a continuous
function is related to the discrete derivative (i.e., forward
difference operator) of its restriction to $\N$ (or $\Z$). 

The forward difference operator is a coarse measure of the average
change over an interval of length $1$ (from $n$ to $n + 1$). The
derivative, on the other hand, is an {\em instantaneous} rate of
change at a given point. The forward difference operator is thus the
{\em average} value of the derivative over the interval from $n$ to $n
+ 1$. We can thus apply the mean value theorem, and conclude that, if
$f$ is continuous on $[n,n+1]$, and differentiable on $(n,n+1)$, then
$(\Delta f)(n)$ equals the value $f'(c)$ for some $c \in (n,n+1)$.

Similar to the way we made observations in the previous subsection, we
make some observations here:

\begin{enumerate}
\item Suppose $f$ has a fixed sense of concavity on $[n,n+1]$, i.e., $f'$
  is either increasing throughout the interval or decreasing throughout
  the interval. This forces that $(\Delta f)(n)$ lies between $f'(n)$
  and $f'(n + 1)$.
\item If the continuous function $f$ has a certain sense of concavity,
  then its discrete version also has the same sense of concavity. We
  say that a function on $\N$ is {\em concave up} if $\Delta^2 f(n) >
  0$ for all $n$, and is {\em concave down} if $\Delta^2 f(n) < 0$ for
  all $n$.
\item If $f$ is a differentiable function and $\lim_{x \to \infty}
  f'(x) = L$ for some finite $L$, then $\lim_{n \to \infty} (\Delta
  f)(n) = L$ as well. In particular, if $\lim_{x \to \infty} f'(x) = 0$,
  then $\lim_{n \to \infty} f(n + 1) - f(n) = 0$.
\end{enumerate}

In all these cases, {\em universal constraints} on the continuous
versions give, via some averaging procedures, corresponding
constraints on the discrete versions. It is usually harder to go form
universal constraints on the discrete version to corresponding
constraints on the continuous version, because there is too much
slack.

\subsection{Continuous and discrete: periodic functions}

Here are some easy facts:

\begin{enumerate}
\item If a continuous function is periodic with period a positive
  integer, then the corresponding discrete function is also periodic
  with period at most that positive integer. (The discrete function
  could repeat at even shorter intervals).
\item Given a discrete periodic function with period $k$, we can write
  it as a linear combination of a bunch of continuous periodic
  functions all arising from trigonometry. The details of this are
  beyond the current scope and have to do with Fourier analysis. An
  illustrative example is $k = 3$: any periodic sequence with period
  $3$ can be expressed as a linear combination of the functions
  $\sin(2 \pi n/3)$, $\cos(2\pi n/3)$ and the constant function
  $1$. The coefficients for the linear combination can be determined
  by solving a system of linear equations.
\end{enumerate}
\subsection{Continuous and discrete: integration}

[We may not get time to cover this in class right now, but will get
back to it later anyway.]
 
We have already seen this, albeit without a lot of thoughtful
reflection. Again, things are easiest to see when $f$ is a monotonic
function, though some of the observations carry over to functions of
bounded variation.

Suppose we are looking at a continuous function $f$ defined on
$[1,n]$. We can now take a partition of the interval $[1,n]$: $1 < 2 <
3 < \dots < n$. Let's assume that $f$ is increasing on $[1,n]$. Then,
the lower sum of $f$ for the partition is $f(1) + f(2) + \dots f(n -
1)$. The upper sum is $f(2) + f(3) + \dots + f(n)$. The integral is
between these, and we obtain:

$$f(1) + f(2) + \dots + f(n-1) < \int_1^n f(x) \, dx < f(2) + \dots + f(n)$$

On the other hand, if $f$ is decreasing, we get:

$$f(1) + f(2) + \dots + f(n-1) > \int_1^n f(x) \, dx > f(2) + \dots + f(n)$$

Now, suppse $f$ is a decreasing function, and suppose further that
$\int_1^\infty f(x) \, dx$ is finite. Then, taking the limit as $n \to
\infty$ in the above, we get:

$$f(1) + f(2) + \dots > \int_1^\infty f(x) \, dx > f(2) + f(3) + \dots $$

Rearranging, we obtain that the sum:

$$f(1) + f(2) + f(3) + \dots $$

is between $\int_1^{\infty} f(x) \, dx$ and $f(1) + \int_1^\infty f(x)
\, dx$.

{\em This procedure actually allows us to calculate sums using
integrals}. Specifically, in cases where an infinite integral is easy
to compute but an infinite sum is not, the infinite sum can be
computed approximately using the computation of the infinite
integral. A modification of this procedure, that we will look at
later, allows us to compute the infinite sum to a very high degree of
precision using a combination of integral calculations and finite
sums.
\end{document}