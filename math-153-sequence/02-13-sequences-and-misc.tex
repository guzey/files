\documentclass[10pt]{amsart}

%Packages in use
\usepackage{fullpage, hyperref, vipul, enumerate}

%Title details
\title{Take-home class quiz: due February 13: Sequences and miscellanea}
\author{Math 153, Section 55 (Vipul Naik)}
%List of new commands

\begin{document}
\maketitle

Your name (print clearly in capital letters): $\underline{\qquad\qquad\qquad\qquad\qquad\qquad\qquad\qquad\qquad\qquad}$

{\bf YOU ARE FREE TO DISCUSS ALL QUESTIONS, BUT PLEASE MAKE SURE TO
ONLY ENTER ANSWER CHOICES THAT YOU PERSONALLY ENDORSE}

\begin{enumerate}
\item Consider the sequence $a_n = 2a_{n-1} - \alpha$, with $a_1 =
  \beta$, for $\alpha, \beta$ real numbers. What can we say about this
  sequence for sure?

  \begin{enumerate}[(A)]
  \item $(a_n)$ is eventually increasing for all values of $\alpha,
    \beta$.
  \item $(a_n)$ is eventually decreasing for all values of $\alpha,
    \beta$.
  \item $(a_n)$ is eventually constant for all values of $\alpha,
    \beta$.
  \item $(a_n)$ is either increasing or decreasing, and which case
    occurs depends on the values of $\alpha$ and $\beta$.
  \item $(a_n)$ is increasing, decreasing, or constant, and which case
    occurs depends on the values of $\alpha$ and $\beta$.
  \end{enumerate}

  \vspace{0.1in}
  Your answer: $\underline{\qquad\qquad\qquad\qquad\qquad\qquad\qquad}$
  \vspace{0.15in}

\item {\em This is a generalization of the preceding question}. Suppose
  $f$ is a continuous increasing function on $\R$. Define a sequence
  recursively by $a_n = f(a_{n-1})$, with $a_1$ chosen
  separately. What can we say about this sequence for sure?
  \begin{enumerate}[(A)]
  \item $(a_n)$ is eventually increasing regardless of the choice of $a_1$.
  \item $(a_n)$ is eventually decreasing regardless of the choice of $a_1$.
  \item $(a_n)$ is eventually constant regardless of the choice of $a_1$.
  \item $(a_n)$ is either increasing or decreasing, and which case
    occurs depends on the value of $a_1$ and the nature of $f$.
  \item $(a_n)$ is increasing, decreasing, or constant, and which case
    occurs depends on the value of $a_1$ and the nature of $f$.
  \end{enumerate}

  \vspace{0.1in}
  Your answer: $\underline{\qquad\qquad\qquad\qquad\qquad\qquad\qquad}$
  \vspace{0.15in}

  For a function $f: \R \to \R$ and a particular element $a \in \R$,
  define $g: \N \to \R$ by $g(n) = f(f(\dots(f(a))\dots))$ with the
  $f$ occurring $n - 1$ times. Thus, $g(1) = a$, $g(2) = f(a)$, and so
  on. Choose the right expression for $g$ for each of these choices of $f$.

\item $f(x) := x + \pi$.
  \begin{enumerate}[(A)]
  \item $g(n) := a + n\pi$.
  \item $g(n) := a + n\pi - 1$.
  \item $g(n) := a + n(\pi - 1)$.
  \item $g(n) := a + \pi(n - 1)$.
  \item $g(n) := \pi + n(a - 1)$.
  \end{enumerate}

  \vspace{0.1in}
  Your answer: $\underline{\qquad\qquad\qquad\qquad\qquad\qquad\qquad}$
  \vspace{0.15in}

\item $f(x) := mx$, $m \ne 0$.
  \begin{enumerate}
  \item $g(n) := mna$.
  \item $g(n) := m^na$.
  \item $g(n) := n^ma$.
  \item $g(n) := m^{n-1}a$.
  \item $g(n) := n^{m-1}a$.
  \end{enumerate}

  \vspace{0.1in}
  Your answer: $\underline{\qquad\qquad\qquad\qquad\qquad\qquad\qquad}$
  \vspace{0.15in}

\item $f(x) := x^2$.
  \begin{enumerate}[(A)]
  \item $g(n) := a^{2^n} - 1$.
  \item $g(n) := a^{2^n - 1}$.
  \item $g(n) := a^{2^{n-1}}$.
  \item $g(n) := a^{2^{n^{-1}}}$.
  \item $g(n) := (a^{2^n})^{-1}$.
  \end{enumerate}

  \vspace{0.1in}
  Your answer: $\underline{\qquad\qquad\qquad\qquad\qquad\qquad\qquad}$
  \vspace{0.15in}

\item One of these sequences can {\em not} be obtained using the
  procedure described in the previous questions (i.e., iterated
  application of a function ). Identify this sequence. Only the first
  five terms of the sequence are presented:

  \begin{enumerate}[(A)]
  \item $1,2,3,3,3$
  \item $1,2,3,2,3$
  \item $1,2,3,2,1$
  \item $1,2,3,4,5$
  \item $1,2,3,4,3$
  \end{enumerate}

  \vspace{0.1in}
  Your answer: $\underline{\qquad\qquad\qquad\qquad\qquad\qquad\qquad}$
  \vspace{0.15in}

\item Suppose $f:\R \to \R$ is a function. Identify which of these
  definitions is {\em not} correct for $\lim_{x \to c} f(x) = L$,
  where $c$ and $L$ are both finite real numbers.
  \begin{enumerate}[(A)]
  \item For every $\epsilon > 0$, there exists $\delta > 0$ such that
    if $x \in (c - \delta, c + \delta) \setminus \{ c \}$, then $f(x)
    \in (L - \epsilon, L + \epsilon)$.
  \item For every $\epsilon_1 > 0$ and $\epsilon_2 > 0$, there exist
    $\delta_1 > 0$ and $\delta_2 > 0$ such that if $x \in (c -
    \delta_1,c+\delta_2)\setminus \{ c \}$, then $f(x) \in (L -
    \epsilon_1, L + \epsilon_2)$.
  \item For every $\epsilon_1 > 0$ and $\epsilon_2 > 0$, there exists
    $\delta > 0$ such that if $x \in (c - \delta, c + \delta)
    \setminus \{ c \}$, then $f(x) \in (L - \epsilon_1, L + \epsilon_2)$.
  \item For every $\epsilon > 0$, there exist $\delta_1 > 0$ and
    $\delta_2 > 0$ such that if $x \in (c - \delta_1, c + \delta_2)
    \setminus \{ c \}$, then $f(x) \in (L - \epsilon, L + \epsilon)$.
  \item None of these, i.e., all definitions are correct.
  \end{enumerate}

  \vspace{0.1in}
  Your answer: $\underline{\qquad\qquad\qquad\qquad\qquad\qquad\qquad}$
  \vspace{0.15in}

\item In the usual $\epsilon-\delta$ definition of limit for a given
  limit $\lim_{x \to c} f(x) = L$, if a given value $\delta > 0$ works
  for a given value $\epsilon > 0$, then which of the following is
  true?
  \begin{enumerate}[(A)]
  \item Every smaller positive value of $\delta$ works for the same
    $\epsilon$. Also, the given value of $\delta$ works for every
    smaller positive value of $\epsilon$.
  \item Every smaller positive value of $\delta$ works for the same
    $\epsilon$. Also, the given value of $\delta$ works for every
    larger value of $\epsilon$.
  \item Every larger value of $\delta$ works for the same
    $\epsilon$. Also, the given value of $\delta$ works for every
    smaller positive value of $\epsilon$.
  \item Every larger value of $\delta$ works for the same
    $\epsilon$. Also, the given value of $\delta$ works for every
    larger value of $\epsilon$.
  \item None of the above statements need always be true.
  \end{enumerate}

  \vspace{0.1in}
  Your answer: $\underline{\qquad\qquad\qquad\qquad\qquad\qquad\qquad}$
  \vspace{0.15in}

\item In the usual $\epsilon-\delta$ definition of limit, we find that
  the value $\delta = 0.2$ for $\epsilon = 0.7$ for a function $f$ at
  $0$, and the value $\delta = 0.5$ works for $\epsilon = 1.6$ for a
  function $g$ at $0$. What value of $\delta$ {\em definitely} works
  for $\epsilon = 2.3$ for the function $f + g$ at $0$?

  \begin{enumerate}[(A)]
  \item $0.2$
  \item $0.3$
  \item $0.5$
  \item $0.7$
  \item $0.9$
  \end{enumerate}

  \vspace{0.1in}
  Your answer: $\underline{\qquad\qquad\qquad\qquad\qquad\qquad\qquad}$
  \vspace{0.15in}

\item The sum of limits theorem states that $\lim_{x \to c} [f(x) +
  g(x)] = \lim_{x \to c} f(x) + \lim_{x \to c} g(x)$ {\em if} the
  right side is defined. One of the choices below gives an example
  where the left side of the equality is defined and finite but the right side
  makes no sense. Identify the correct choice.

  \begin{enumerate}[(A)]
  \item $f(x) := 1/x$, $g(x) := -1/(x + 1)$, $c = 0$.
  \item $f(x) := 1/x$, $g(x) := (x - 1)/x$, $c = 0$.
  \item $f(x) := \arcsin x$, $g(x) := \arccos x$, $c = 1/2$.
  \item $f(x) := 1/x$, $g(x) = x$, $c = 0$.
  \item $f(x) := \tan x$, $g(x) := \cot x$, $c = 0$.
  \end{enumerate}

  \vspace{0.1in}
  Your answer: $\underline{\qquad\qquad\qquad\qquad\qquad\qquad\qquad}$
  \vspace{0.15in}

\end{enumerate}

\end{document}
