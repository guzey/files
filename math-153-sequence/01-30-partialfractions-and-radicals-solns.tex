\documentclass[10pt]{amsart}

%Packages in use
\usepackage{fullpage, hyperref, vipul, enumerate}

%Title details
\title{Class quiz solutions: January 30: Partial fractions and radicals}
\author{Math 153, Section 55 (Vipul Naik)}
%List of new commands

\begin{document}
\maketitle

\section{Performance review}

$11$ people took this $8$-question quiz. The score distribution was as follows:

\begin{itemize}
\item Score of $3$: $2$ people.
\item Score of $4$: $4$ people.
\item Score of $5$: $4$ people.
\item Score of $6$: $1$ person.
\end{itemize}

The mean score was $4.36$.

Here is the performance summary by question:

\begin{enumerate}
\item Option (D): $10$ people.
\item Option (D): $7$ people.
\item Option (B): $1$ person.
\item Option (B): $8$ people.
\item Option (D): $2$ people.
\item Option (E): $4$ people. {\em This will come up again later in
  the course, so you might benefit from internalizing the answer right
  now.}
\item Option (D): $10$ people.
\item Option (D): $6$ people.
\end{enumerate}

\section{Solutions}

\begin{enumerate}
\item Which of these functions of $x$ is {\em not} elementarily
  integrable?

  \begin{enumerate}[(A)]
  \item $x\sqrt{1 + x^2}$
  \item $x^2\sqrt{1 + x^2}$
  \item $x(1 + x^2)^{1/3}$
  \item $x\sqrt{1 + x^3}$
  \item $x^2\sqrt{1 + x^3}$
  \end{enumerate}

  {\em Answer}: Option (D)

  {|em Explanation}: For options (A) and (C), the substitution $u = 1
  + x^2$ works fine. For option (E), the substitution $u = 1 + x^3$
  works fine. For option (B), we can solve the problem using a
  trigonometric substitution. This leaves option (D) (which,
  incidentally, requires the use of elliptic integrals).

  {\em Performance review}: $10$ out of $11$ got this correct. $1$
  chose (C).

  {\em Historical note (last year)}: $22$ out of $27$ people got this
  correct. $4$ people chose (B) and $1$ person chose (C).
\item For which of these functions of $x$ does the antiderivative
  necessarily involve {\em both} $\arctan$ {\em and} $\ln$?

  \begin{enumerate}[(A)]
  \item $1/(x + 1)$
  \item $1/(x^2 + 1)$
  \item $x/(x^2 + 1)$
  \item $x/(x^3 + 1)$
  \item $x^2/(x^3 + 1)$
  \end{enumerate}

  {\em Answer}: Option (D)

  {\em Explanation}: Option (A) integrates to $\ln|x|$, option (B)
  integrates to $\arctan x$, option (C) integrates to $(1/2) \ln(x^2 +
  1)$, and option (E) integrates to $(1/3) \ln|x^3 + 1|$. For option
  (D), we need to use partial fractions with denominators $x + 1$ and
  $x^2 - x + 1$, and we end up getting nonzero coefficients on terms
  that integrate to $\ln$ and to $\arctan$.

  {\em Performance review}: $7$ out of $11$ got it. $3$ chose (C), $1$
  chose (B).

  {\em Historical note (last year)}: $21$ out of $27$ people got this
  correct. $2$ people chose (E), and $1$ person each chose (A), (B),
  and (C). $1$ person left the question blank.

\item Consider the function $f(k) := \int_1^2 \frac{dx}{\sqrt{x^2 +
  k}}$. $f$ is defined for $k \in (-1,\infty)$. What can we say about
  the nature of $f$ within this interval?

  \begin{enumerate}[(A)]
  \item $f$ is increasing on the interval $(-1,\infty)$.
  \item $f$ is decreasing on the interval $(-1,\infty)$.
  \item $f$ is increasing on $(-1,0)$ and decreasing on $(0,\infty)$.
  \item $f$ is decreasing on $(-1,0)$ and increasing on $(0,\infty)$.
  \item $f$ is increasing on $(-1,0)$, decreasing on $(0,2)$, and
    increasing again on $(2,\infty)$.
  \end{enumerate}

  {\em Answer}: Option (B)

  {\em Explanation}: For any fixed value of $x \in [1,2]$, the
  integrand $1/\sqrt{x^2 + k}$ is a {\em decreasing} function of $k$
  for $k \in (-1,\infty)$. Hence, the value we get upon integrating it
  for $x \in [1,2]$ should also be a decreasing function of $k$.

  {\em Performance review}: $1$ out of $11$ got this. $4$ chose(D),
  $3$ chose (E), $2$ chose (C), $1$ chose (A).

  {\em Historical note (last year)}: $4$ out of $27$
  people got this correct. $11$ people chose (A), $7$ people chose
  (C), $3$ people chose (D), $1$ person chose (E), and $1$ person left
  the question blank.

  Mainly, people confused the roles of the dummy variable $x$ (which
  gets integrated away) and the variable $k$.

  {\em Action point}: This is qualitatively similar to Question 9 on
  the midterm and a corresponding question on the mock midterm. The
  solution approach is also the same. After reviewing this solution,
  you may wish to take another look at that question.

\item Suppose $F$ is a (not known) function defined on $\R \setminus
  \{ -1,0,1\}$, differentiable everywhere on its domain, such that
  $F'(x) = 1/(x^3 - x)$ everywhere on $\R \setminus \{-1,0,1\}$. For
  which of the following sets of points is it true that knowing the
  value of $F$ at these points {\bf uniquely} determines $F$?

  \begin{enumerate}[(A)]
  \item $\{ -\pi, -e, 1/e,1/\pi \}$
  \item $\{ -\pi/2, -\sqrt{3}/2, 11/17,\pi^2/6 \}$
  \item $\{ -\pi^3/7,-\pi^2/6,\sqrt{13},11/2 \}$
  \item Knowing $F$ at any of the above determines the value of $F$
    uniquely.
  \item None of the above works to uniquely determine the value of
    $F$.
  \end{enumerate}

  {\em Answer}: Option (B)

  {\em Explanation}: The domain of $F$ has four connected components:
  the open intervals $(-\infty,-1)$, $(-1,0)$, $(0,1)$, and
  $(1,\infty)$. We need to know the value of $F$ at one point in each
  of these intervals. By computing values, we see that the set of
  points in option (B) has the property that it contains one point in
  each of these intervals, and those in options (A) and (C) do not.

  {\em Performance review}: $8$ out of $11$ got this. $2$ chose (D)
  and $1$ chose (C).

  {\em Historical note (last year)}: $14$ out of $27$ people got this
  correct. $6$ people chose (D), $4$ people chose (C), and $3$ people
  chose (A).

  Many people spent time trying to determine the coefficients of the
  partial fraction decomposition. This is not relevant to what we need
  to do in the question.

  {\em Action point}: The idea here (that you need to know the value
  at one point in each connected component) is a crucial one that you
  should understand.

\item Consider a rational function $f(x) := p(x)/q(x)$ where $p$ and
  $q$ are nonzero polynomials and the degree of $p$ is strictly less
  than the degree of $q$. Suppose $q(x)$ is monic of degree $n$ and
  has $n$ distinct real roots $a_1,a_2,\dots,a_n$, so $q(x) =
  \prod_{i=1}^n (x - a_i)$. Then, we can write:

  $$f(x) = \frac{c_1}{x - a_1} + \frac{c_2}{x - a_2} + \dots + \frac{c_n}{x - a_n}$$

  for suitable constants $c_i \in \R$. What can we say about the sum
  $\sum_{i=1}^n c_i$?

  \begin{enumerate}[(A)]
  \item The sum is always $0$.
  \item The sum equals the leading coefficient of $p$.
  \item The sum is $0$ if $p$ has degree $n - 1$. If the degree of $p$
  is smaller, the sum equals the leading coefficient of $p$.
  \item The sum is $0$ if $p$ has degree smaller than $n - 1$. If $p$
    has degree equal to $n - 1$, the sum is the leading coefficient of
    $p$.
  \item The sum is $0$ if $p$ is a constant polynomial. Otherwise, it
    equals the leading coefficient of $p$.
  \end{enumerate}

  {\em Answer}: Option (D)

  {\em Explanation}: If we take a common denominator and simplify the
  right side, we see that the coefficient of $x^{n-1}$ on the
  numerator of the right side is $\sum_{i=1}^n c_i$. Equating
  coefficients, we obtain that the coefficient of $x^{n-1}$ on the
  left side is also $\sum_{i=1}^n c_i$. If $p$ has degree less than $n
  - 1$, the coefficient on the left side is $0$, so $\sum_{i=1}^n c_i
  = 0$. If $p$ has degree equal to $n - 1$, the coefficient on the
  left side is the leading coefficient of $p$.

  {\em Performance review}: $2$ out of $11$ got this. $8$ chose (E)
  (which is pretty close) and $1$ chose (B).

  {\em Historical note (last year)}: $12$ out of $27$ people got this
  correct. $7$ people chose (E), $3$ people each chose (B) and (C),
  $1$ person chose (A).

  When the denominator is quadratic, then options (D) and (E) are
  euqivalent. This is what might have led to many people choosing
  option (E).

\item {\em Hard right now, will become easier later}: Suppose $F$ is a
  continuously differentiable function whose domain contains
  $(a,\infty)$ for some $a \in \R$, and $F'(x)$ is a rational function
  $p(x)/q(x)$ on the domain of $F$. Further, suppose that $p$ and $q$
  are nonzero polynomials. Denote by $d_p$ the degree of $p$ and by
  $d_q$ the degree of $q$. Which of the following is a {\bf necessary
  and sufficient condition} to ensure that $\lim_{x \to \infty} F(x)$
  is finite?

  \begin{enumerate}[(A)]
  \item $d_p - d_q \ge 2$
  \item $d_p - d_q \ge 1$
  \item $d_p = d_q$
  \item $d_q - d_p \ge 1$
  \item $d_q - d_p \ge 2$
  \end{enumerate}

  {\em Answer}: Option (E)

  {\em Explanation}: This can be justified in terms of partial
  fractions. Th ecase where $q$ is a product of linear factors can be
  justified using the previous question. But that is not the most
  elegant justification. When we cover sequences and series, we will
  see some comparison tests that make it clear why this holds. The
  basic example you can keep in mind is that the antiderivative of
  $1/x^2$ is $-1/x$, which has a finite limit as $x \to \infty$.

  {\em Performance review}: $4$ out of $11$ got this. $5$ chose (D),
  $1$ each chose (A) and (C).

  {\em Historical note (last year)}: $3$ out of $27$ people got this
  correct. $10$ people chose (D), $7$ people chose (C), $6$ people
  chose (B), and $1$ person chose (A).

  Those who chose (D) had the right idea but failedto account for the
  extra margin that needs to be maintained because an integration is
  being performed.

  For the remaining questions, we build on the observation: For any
  nonconstant monic polynomial $q(x)$, there exists a finite
  collection of transcendental functions $f_1, f_2, \dots, f_r$ such
  that the antiderivative of any rational function $p(x)/q(x)$, on an
  open interval where it is defined and continuous, can be expressed
  as $g_0 + f_1g_1 + f_2g_2 + \dots + f_rg_r$ where $g_0, g_1, \dots,
  g_r$ are rational functions.

\item For the polynomial $q(x) = 1 + x^2$, what collection of $f_i$s
  works (all are written as functions of $x$)?

  \begin{enumerate}[(A)]
  \item $\arctan x$ and $\ln|x|$
  \item $\arctan x$ and $\arctan(1 + x^2)$
  \item $\ln|x|$ and $\ln(1 + x^2)$ 
  \item $\arctan x$ and $\ln(1 + x^2)$
  \item $\ln|x|$ and $\arctan(1 + x^2)$
  \end{enumerate}

  {\em Answer}: Option (D)

  {\em Explanation}: Follows from the standard partial fraction
  decomposition. $2x/(1 + x^2)$ gives the $\ln$ integration and $1/(1
  + x^2)$ gives the $\arctan$ integration.

  {\em Performance review}: $10$ out of $11$ got this. $1$ chose (E).

  {\em Historical note (last year)}: $15$ out of $27$ people got this
  correct. $7$ people chose (A), $2$ people chose (C), $1$ person each
  chose (A) and (E), and $1$ person left the question blank.
\item For the polynomial $q(x) := 1 + x^2 + x^4$, what is the size of
  the smallest collection of $f_i$s that works?

  \begin{enumerate}[(A)]
  \item $1$
  \item $2$
  \item $3$
  \item $4$
  \item $5$
  \end{enumerate}

  {\em Answer}: Option (D)

  {\em Explanation}: The denominator factors into $x^2 - x + 1$ and
  $x^2 + x + 1$. Each of these contributes one $\arctan$ possibility
  and one $\ln$ possibility. A total of $4$ possibilities is achieved.

  In general, if there are no repeated factors, the smallest number of
  pieces equals the degree of the polynomial.

  {\em Performance review}: $6$ out of $11$ got this. $2$ chose (B),
  $1$ each chose (A), (C), and (E).

  {\em Historical note (last year)}: $7$ out of $27$ people got this
  correct. $9$ people chose (C), $6$ people chose (B), $4$ people
  chose (A), and $1$ person left the question blank.
\end{enumerate}

\end{document}
