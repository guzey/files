\documentclass[10pt]{amsart}

%Packages in use
\usepackage{fullpage, hyperref, vipul, enumerate}

%Title details
\title{Take-home class quiz: due February 15: Interplay of continuous and discrete}
\author{Math 153, Section 55 (Vipul Naik)}
%List of new commands

\begin{document}
\maketitle

Your name (print clearly in capital letters): $\underline{\qquad\qquad\qquad\qquad\qquad\qquad\qquad\qquad\qquad\qquad}$

{\bf YOU ARE FREE TO DISCUSS ALL QUESTIONS, BUT PLEASE MAKE SURE TO
ONLY ENTER ANSWER CHOICES THAT YOU PERSONALLY ENDORSE}

\begin{enumerate}

\item Consider a function $f$ defined on all real numbers. Consider
  also the sequence $a_n = f(n)$ defined for $n$ a natural
  number. Which of the following is true?

  \begin{enumerate}[(A)]
  \item $\lim_{x \to \infty} f(x)$ is finite if and only if $\lim_{n
    \to \infty} a_n$ is finite, and if so, both limits are equal.
  \item $\lim_{x \to \infty} f(x)$ is finite if and only if $\lim_{n
    \to \infty} a_n$ is finite, but the limits need not be equal.
  \item If $\lim_{x \to \infty} f(x)$ is finite, then $\lim_{n \to
    \infty} a_n$ is finite, but the converse is not true. Moreover, if
    both limits are finite, they must be equal.
  \item If $\lim_{n \to \infty} a_n$ is finite, then $\lim_{x \to
    \infty} f(x)$ is finite, but the converse is not true. Moreover, if
    both limits are finite, they must be equal.
  \item It is possible for either of the limits $\lim_{x \to \infty}
    f(x)$ and $\lim_{n \to \infty} a_n$ to be finite, but for the
    other one not to be finite. Moreover, even if both limits exist,
    they need not be equal.
  \end{enumerate}

  \vspace{0.1in}
  Your answer: $\underline{\qquad\qquad\qquad\qquad\qquad\qquad\qquad}$
  \vspace{0.15in}

\item Consider a function $f:\R \to \R$. Restricting the
  domain of $f$ to the natural numbers, obtain a sequence whose
  $n^{th}$ member $a_n$ is defined as $f(n)$. Which of the following
  statements is {\bf false} about the relationship between $f$ and the
  sequence $(a_n)$?

  \begin{enumerate}[(A)]
  \item If $f$ is an increasing function, then $(a_n)$ form an
    increasing sequence.
  \item If $f$ is a decreasing function, then $(a_n)$ form a
    decreasing sequence.
  \item If $f$ is a bounded function, (i.e., its range is a bounded
    set) then $(a_n)$ form a bounded sequence.
  \item If $f$ is a periodic function, then $(a_n)$ form a periodic sequence.
  \item If $f$ has a limit at infinity, then $(a_n)$ is a convergent
    sequence.
  \end{enumerate}

  \vspace{0.1in}
  Your answer: $\underline{\qquad\qquad\qquad\qquad\qquad\qquad\qquad}$
  \vspace{0.15in}

\item We are given a sequence $a_1, a_2, \dots, a_n, \dots$ of real
  numbers. The goal is to find a {\em continuous} function $f$ on all
  of $\R$ such that $f(n) = a_n$ for all $n \in \N$. Which of the
  following is true?

  \begin{enumerate}[(A)]
  \item There is a unique choice of $f$ that works.
  \item There exist infinitely many different choices of $f$ that work.
  \item The number of possible choices of $f$ depends on the
    sequence. Depending on the sequence, the number of possible
    choices of $f$ may be zero, one, or infinite.
  \item The number of possible choices of $f$ depends on the
    sequence. Depending on the sequence, the number of possible
    choices of $f$ may be zero or one. It can never be infinite.
  \item The number of possible choices of $f$ depends on the
    sequence. Depending on the sequence, the number of possible
    choices of $f$ may be one or infinite. It can never be zero.
  \end{enumerate}

  \vspace{0.1in}
  Your answer: $\underline{\qquad\qquad\qquad\qquad\qquad\qquad\qquad}$
  \vspace{0.15in}

\item We are given a sequence $a_1, a_2, \dots, a_n, \dots$ of real
  numbers. The goal is to find an {\em infinitely differentiable}
  function $f$ on all of $\R$ such that $f(n) = a_n$ for all $n \in
  \N$. Which of the following is true?

  \begin{enumerate}[(A)]
  \item There is a unique choice of $f$ that works.
  \item There exist infinitely many different choices of $f$ that work.
  \item The number of possible choices of $f$ depends on the
    sequence. Depending on the sequence, the number of possible
    choices of $f$ may be zero, one, or infinite.
  \item The number of possible choices of $f$ depends on the
    sequence. Depending on the sequence, the number of possible
    choices of $f$ may be zero or one. It can never be infinite.
  \item The number of possible choices of $f$ depends on the
    sequence. Depending on the sequence, the number of possible
    choices of $f$ may be one or infinite. It can never be zero.
  \end{enumerate}

  \vspace{0.1in}
  Your answer: $\underline{\qquad\qquad\qquad\qquad\qquad\qquad\qquad}$
  \vspace{0.15in}
\item We are given a sequence $a_1, a_2, \dots, a_n, \dots$ of real
  numbers. The goal is to find a {\em polynomial}
  function $f$ on all of $\R$ such that $f(n) = a_n$ for all $n \in
  \N$. Which of the following is true?

  \begin{enumerate}[(A)]
  \item There is a unique choice of $f$ that works.
  \item There exist infinitely many different choices of $f$ that work.
  \item The number of possible choices of $f$ depends on the
    sequence. Depending on the sequence, the number of possible
    choices of $f$ may be zero, one, or infinite.
  \item The number of possible choices of $f$ depends on the
    sequence. Depending on the sequence, the number of possible
    choices of $f$ may be zero or one. It can never be infinite.
  \item The number of possible choices of $f$ depends on the
    sequence. Depending on the sequence, the number of possible
    choices of $f$ may be one or infinite. It can never be zero.
  \end{enumerate}

  \vspace{0.1in}
  Your answer: $\underline{\qquad\qquad\qquad\qquad\qquad\qquad\qquad}$
  \vspace{0.15in}

  For the remaining questions: For a function $f:\N \to \R$, define
  $\Delta f$ as the function $n \mapsto f(n+1) -f(n)$. Denote by
  $\Delta^k f$ the function obtained by applying $\Delta$ $k$ times to
  $f$.
\item  If $f(n) = n^2$, what is $(\Delta f)(n)$?

  \begin{enumerate}[(A)]
  \item $1$
  \item $n$
  \item $2n - 1$
  \item $2n$
  \item $2n + 1$
  \end{enumerate}

  \vspace{0.1in}
  Your answer: $\underline{\qquad\qquad\qquad\qquad\qquad\qquad\qquad}$
  \vspace{0.15in}

\item If $f$ is expressible as a polynomial function of degree $d >
  0$, what is the smallest $k$ for which $\Delta^k f$ is identically
  the zero function? {\em Hint: Think of the analogous question using
  continuous derivatives. Although $\Delta$ differs from the
  continuous derivative, much of the qualitative behavior is the same.}

  \begin{enumerate}[(A)]
  \item $d - 2$
  \item $d - 1$
  \item $d$
  \item $d + 1$
  \item $d + 2$
  \end{enumerate}

  \vspace{0.1in}
  Your answer: $\underline{\qquad\qquad\qquad\qquad\qquad\qquad\qquad}$
  \vspace{0.15in}

\item If $f$ is a function such that $\Delta f = af$ for some positive
  constant $a$, and $f(1)$ is positive, which of the following best
  describes the nature of growth of $f$? {\em Hint: Think of the
  analogous differential equation using continuous derivatives. The
  precise solution is different but the nature of the solution is
  similar.}

  \begin{enumerate}[(A)]
  \item $f$ grows like a sublinear function of $n$.
  \item $f$ grows like a linear function of $n$.
  \item $f$ grows like a superlinear but subexponential function of
    $n$.
  \item $f$ grows like an exponential function of $n$.
  \item $f$ grows like a superexponential function of $n$.
  \end{enumerate}

  \vspace{0.1in}
  Your answer: $\underline{\qquad\qquad\qquad\qquad\qquad\qquad\qquad}$
  \vspace{0.15in}

\end{enumerate}
\end{document}