\documentclass[10pt]{amsart}

%Packages in use
\usepackage{fullpage, hyperref, vipul, enumerate}

%Title details
\title{Take-home quiz solutions: due February 13: Sequences and miscellanea}
\author{Math 153, Section 55 (Vipul Naik)}
%List of new commands

\begin{document}
\maketitle

\section{Performance review}

$11$ people took this quiz. The score distribution was as follows:

\begin{itemize}
\item Score of $3$: $2$ people.
\item Score of $4$: $1$ person.
\item Score of $6$: $2$ people.
\item Score of $7$: $1$ person.
\item Score of $8$: $5$ people.
\end{itemize}

The mean score was $6.27$. The question wise answer choices and
performance were as follows:

\begin{enumerate}
\item Option (E): $7$ people.
\item Option (E): $7$ people.
\item Option (D): $9$ people.
\item Option (D): $9$ people.
\item Option (C): $11$ people.
\item Option (C): $0$ people. {\em Please review this solution!}
\item Option (E): $1$ person.
\item Option (B): $9$ people.
\item Option (A): $7$ people.
\item Option (B): $9$ people.
\end{enumerate}
\section{Solutions}

\begin{enumerate}
\item Consider the sequence $a_n = 2a_{n-1} - \alpha$, with $a_1 =
  \beta$, for $\alpha, \beta$ real numbers. What can we say about this
  sequence for sure?

  \begin{enumerate}[(A)]
  \item $(a_n)$ is eventually increasing for all values of $\alpha,
    \beta$.
  \item $(a_n)$ is eventually decreasing for all values of $\alpha,
    \beta$.
  \item $(a_n)$ is eventually constant for all values of $\alpha,
    \beta$.
  \item $(a_n)$ is either increasing or decreasing, and which case
    occurs depends on the values of $\alpha$ and $\beta$.
  \item $(a_n)$ is increasing, decreasing, or constant, and which case
    occurs depends on the values of $\alpha$ and $\beta$.
  \end{enumerate}

  {\em Answer}: Option (E)

  {\em Explanation}: See the answer explanation for the next
  question. Note that the {\em constant} case could arise when $\alpha
  = \beta = 1$.

  {\em Performance review}: $7$ out of $11$ got this correct. $2$
  chose (A), $2$ chose (D).

  {\em Historical note (last year)}: $10$ out of $26$ people got this
  correct. $13$ people chose (D), which is pretty close to correct,
  and $3$ people chose (A).
\item {\em This is a generalization of the preceding question}. Suppose
  $f$ is a continuous increasing function on $\R$. Define a sequence
  recursively by $a_n = f(a_{n-1})$, with $a_1$ chosen
  separately. What can we say about this sequence for sure?
  \begin{enumerate}[(A)]
  \item $(a_n)$ is eventually increasing regardless of the choice of $a_1$.
  \item $(a_n)$ is eventually decreasing regardless of the choice of $a_1$.
  \item $(a_n)$ is eventually constant regardless of the choice of $a_1$.
  \item $(a_n)$ is either increasing or decreasing, and which case
    occurs depends on the value of $a_1$ and the nature of $f$.
  \item $(a_n)$ is increasing, decreasing, or constant, and which case
    occurs depends on the value of $a_1$ and the nature of $f$.
  \end{enumerate}

  {\em Answer}: Option (E)

  {\em Explanation}: Since $f$ is increasing, if $f(a_1) < a_1$, then
  $f(f(a_1)) < f(a_1)$. we can inductively show that if $f(a_1) <
  a_1$, then $(a_n)$ is decreasing. If $f(a_1) = a_1$, then $(a_n)$ is
  constant. If $f(a_1) > a_1$, then $(a_n)$ is increasing. Any of these
  cases may occur (as we can see using specific example from the
  previous problem). So, $(a_n)$ is either increasing, decreasing or
  constant, but which case occurs depends on the value of $a_1$ and
  the nature of $f$.

  For a function $f: \R \to \R$ and a particular element $a \in \R$,
  define $g: \N \to \R$ by $g(n) = f(f(\dots(f(a))\dots))$ with the
  $f$ occurring $n - 1$ times. Thus, $g(1) = a$, $g(2) = f(a)$, and so
  on. Choose the right expression for $g$ for each of these choices of $f$.

  {\em Performance review}: $7$ out of $11$ got this correct. $2$
  chose (A), $1$ each chose (C) and (D).

 {\em Historical note (last year)}: $11$ ouot of $26$ people got this
  correct. $6$ people chose (D), $6$ people chose (A), $2$ people
  chose (B), and $1$ person left the question blank.
\item $f(x) := x + \pi$.
  \begin{enumerate}[(A)]
  \item $g(n) := a + n\pi$.
  \item $g(n) := a + n\pi - 1$.
  \item $g(n) := a + n(\pi - 1)$.
  \item $g(n) := a + \pi(n - 1)$.
  \item $g(n) := \pi + n(a - 1)$.
  \end{enumerate}

  {\em Answer}: Option (D)

  {\em Explanation}: Straightforward
  summation/induction/observation. If you aren't able to arrive at the
  expression, just plug in and check the values $n = 1$ and $n = 2$.

  {\em Performance review}: $9$ out of $11$ got this correct. $1$ each
  chose (B) and (C).

  {\em Historical note (last year)}: $20$ out of $26$ people got this
  correct. $3$ people chose (C), $2$ people chose (B), and $1$ person
  chose (A).

  {\em Action point}: This is the kind of question that everybody
  should be able to get correct in the future!

\item $f(x) := mx$, $m \ne 0$.
  \begin{enumerate}
  \item $g(n) := mna$.
  \item $g(n) := m^na$.
  \item $g(n) := n^ma$.
  \item $g(n) := m^{n-1}a$.
  \item $g(n) := n^{m-1}a$.
  \end{enumerate}

  {\em Answer}: Option (D)

  {\em Explanation}: Straightforward
  summation/induction/observation. If you aren't able to arrive at the
  expression, just plug in and check the values $n = 1$ and $n = 2$.

  {\em Performance review}: $9$ out of $11$ got this. $2$ chose (B).

  {\em Historical note (last year)}: $20$ out of $26$ people got this
  correct. $5$ people chose (B) and $1$ person chose (A).

  {\em Action point}: This is the kind of question that everybody
  should be able to get correct in the future!
\item $f(x) := x^2$.
  \begin{enumerate}[(A)]
  \item $g(n) := a^{2^n} - 1$.
  \item $g(n) := a^{2^n - 1}$.
  \item $g(n) := a^{2^{n-1}}$.
  \item $g(n) := a^{2^{n^{-1}}}$.
  \item $g(n) := (a^{2^n})^{-1}$.
  \end{enumerate}

  {\em Answer}: Option (C)

  {\em Explanation}: Each time we square, the exponent gets multiplied
  by $2$. Thus, the exponent itself is growing like $2^{n-1}$ (it
  starts out at $1$).

  {\em Performance review}: Everybody got this correct.

  {\em Historical note (last year)}: $19$ out of $26$ people got this
  correct. $5$ people chose (B), indicating a lack of care in keeping
  track of exponent towers. $1$ person each chose (A) and (E).

  {\em Action point}: This is the kind of question that everybody
  should be able to get correct in the future!

\item One of these sequences can {\em not} be obtained using the
  procedure described in the previous questions (i.e., iterated
  application of a function ). Identify this sequence. Only the first
  five terms of the sequence are presented:

  \begin{enumerate}[(A)]
  \item $1,2,3,3,3$
  \item $1,2,3,2,3$
  \item $1,2,3,2,1$
  \item $1,2,3,4,5$
  \item $1,2,3,4,3$
  \end{enumerate}

  {\em Answer}: Option (C).

  {\em Explanation}: For a sequence obtained by function iteration, it
  must be true that the successor of an element is uniquely determined
  by that element. For the sequence with first five terms $1,2,3,2,1$,
  we note that at one place in the sequence, $2$ is followed by $3$,
  but at another place, $2$ is followed by one. This is not possible,
  because $f(2)$ cannot be both $3$ and $1$.

  {\em Performance review}: Nobody got this correct! $10$ chose (A),
  $1$ chose (E).

  {\em Historical note (last year)}: $6$ out of $26$ people got this
  correct. $14$ people chose (A), $3$ people chose (B), $2$ chose (E),
  and $1$ chose (D).

\item Suppose $f:\R \to \R$ is a function. Identify which of these
  definitions is {\em not} correct for $\lim_{x \to c} f(x) = L$,
  where $c$ and $L$ are both finite real numbers.
  \begin{enumerate}[(A)]
  \item For every $\epsilon > 0$, there exists $\delta > 0$ such that
    if $x \in (c - \delta, c + \delta) \setminus \{ c \}$, then $f(x)
    \in (L - \epsilon, L + \epsilon)$.
  \item For every $\epsilon_1 > 0$ and $\epsilon_2 > 0$, there exist
    $\delta_1 > 0$ and $\delta_2 > 0$ such that if $x \in (c -
    \delta_1,c+\delta_2)\setminus \{ c \}$, then $f(x) \in (L -
    \epsilon_1, L + \epsilon_2)$.
  \item For every $\epsilon_1 > 0$ and $\epsilon_2 > 0$, there exists
    $\delta > 0$ such that if $x \in (c - \delta, c + \delta)
    \setminus \{ c \}$, then $f(x) \in (L - \epsilon_1, L + \epsilon_2)$.
  \item For every $\epsilon > 0$, there exist $\delta_1 > 0$ and
    $\delta_2 > 0$ such that if $x \in (c - \delta_1, c + \delta_2)
    \setminus \{ c \}$, then $f(x) \in (L - \epsilon, L + \epsilon)$.
  \item None of these, i.e., all definitions are correct.
  \end{enumerate}

  {\em Answer}: Option (E)

  {\em Explanation}: Although the usual $\epsilon-\delta$ definition
  uses centered intervals, i.e., intervals centered at the points $c$
  and $L$, this is not a necessary aspect of the definition. So,
  instead of taking centered intervals $(c - \delta, c + \delta)$ or
  $(L - \epsilon,L + \epsilon)$, we could consider open intervals that
  have different amounts on the left and on the right. Thus, all four
  definitions are correct.

  {\em Performance review}: $1$ out of $11$ got this correct. $4$
  chose (B), $3$ chose (A), $2$ chose (D), $1$ chose (C).

  {\em Historical note (last year)}: $6$ out of $26$ people got this
  correct. $9$ people chose (C), $4$ people each chose (A) and (D),
  $2$ chose (B), and $1$ left the question blank.

  {\em Action point}: Revisit this question at the end of the week,
  after we have covered related ideas in class.
\item In the usual $\epsilon-\delta$ definition of limit for a given
  limit $\lim_{x \to c} f(x) = L$, if a given value $\delta > 0$ works
  for a given value $\epsilon > 0$, then which of the following is
  true?
  \begin{enumerate}[(A)]
  \item Every smaller positive value of $\delta$ works for the same
    $\epsilon$. Also, the given value of $\delta$ works for every
    smaller positive value of $\epsilon$.
  \item Every smaller positive value of $\delta$ works for the same
    $\epsilon$. Also, the given value of $\delta$ works for every
    larger value of $\epsilon$.
  \item Every larger value of $\delta$ works for the same
    $\epsilon$. Also, the given value of $\delta$ works for every
    smaller positive value of $\epsilon$.
  \item Every larger value of $\delta$ works for the same
    $\epsilon$. Also, the given value of $\delta$ works for every
    larger value of $\epsilon$.
  \item None of the above statements need always be true.
  \end{enumerate}

  {\em Answer}: Option (B)

  {\em Explanation}: This can be understood in multiple ways. One is
  in terms of the prover-skeptic game. A particular choice of $\delta$
  that works for a specific $\epsilon$ also works for larger
  $\epsilon$s, because the function is already ``trapped'' in a
  smaller region. Further, smaller choices of $\delta$ also work
  because the skeptic has fewer values of $x$.

  Rigorous proofs are being skipped here, but you can review the
  formal definition of limit notes if this stuff confuses you.

  {\em Performance review}: $9$ out of $11$ got this correct. $2$
  chose (C).

  {\em Historical note (last year)}: $17$ out of $26$ people got this
  correct. $5$ people chose (A), $3$ chose (C), and $1$ chose (D).

\item In the usual $\epsilon-\delta$ definition of limit, we find that
  the value $\delta = 0.2$ for $\epsilon = 0.7$ for a function $f$ at
  $0$, and the value $\delta = 0.5$ works for $\epsilon = 1.6$ for a
  function $g$ at $0$. What value of $\delta$ {\em definitely} works
  for $\epsilon = 2.3$ for the function $f + g$ at $0$?

  \begin{enumerate}[(A)]
  \item $0.2$
  \item $0.3$
  \item $0.5$
  \item $0.7$
  \item $0.9$
  \end{enumerate}

  {\em Answer}: Option (A)

  {\em Explanation}: We choose the {\em smaller} of the $\delta$s to
  guarantee that {\em both} $f$ and $g$ are within their respective
  $\epsilon$-distances of the targets -- $0.7$ in the case of $f$ and
  $1.6$ in the case of $g$. Now, the triangle inequality guarantees
  that $f + g$ is within $2.3$ of its proposed limit.

  {\em Performance review}: $7$ out of $11$ got this correct. $2$
  chose (E), $1$ each chose (C) and (D).

  {\em Historical note (last year)}: $10$ out of $26$ people got this
  correct. $9$ people chose (D) -- {\em naive addition} -- which does
  not really make sense. $4$ chose (B), $2$ chose (E), and $1$ chose
  (C).
\item The sum of limits theorem states that $\lim_{x \to c} [f(x) +
  g(x)] = \lim_{x \to c} f(x) + \lim_{x \to c} g(x)$ {\em if} the
  right side is defined. One of the choices below gives an example
  where the left side of the equality is defined and finite but the right side
  makes no sense. Identify the correct choice.

  \begin{enumerate}[(A)]
  \item $f(x) := 1/x$, $g(x) := -1/(x + 1)$, $c = 0$.
  \item $f(x) := 1/x$, $g(x) := (x - 1)/x$, $c = 0$.
  \item $f(x) := \arcsin x$, $g(x) := \arccos x$, $c = 1/2$.
  \item $f(x) := 1/x$, $g(x) = x$, $c = 0$.
  \item $f(x) := \tan x$, $g(x) := \cot x$, $c = 0$.
  \end{enumerate}

  {\em Answer}: Option (B)

  {\em Explanation}: $f + g$ is the constant function $1$, so it has a
  limit. On the other hand, both $f$ and $g$ have one-sided limits of
  $\pm \infty$.

  For options (A), (D), and (E), one of the function $f$ and $g$ has a
  finite limit, and the other has an infinite or undefined limit, and
  the sum has an infinite or undefined limit. Option (C) is a case
  where $f$, $g$, and $f + g$ all have finite limits.

  {\em Performance review}: $9$ out of $11$ got this correct. $1$ each
  chose (D) and (E).

  {\em Historical note (last year)}: $13$ out of $26$ people got this
  correct. $5$ people chose (C), $3$ chose (E), $2$ chose (A), $1$
  chose (D), and $2$ left the question blank.


\end{enumerate}

\end{document}
