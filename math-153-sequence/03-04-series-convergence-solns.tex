\documentclass[10pt]{amsart}

%Packages in use
\usepackage{fullpage, hyperref, vipul, enumerate}

%Title details
\title{Class quiz solutions: March 4: Series convergence}
\author{Math 153, Section 55 (Vipul Naik)}
%List of new commands

\begin{document}
\maketitle

\section{Performance review}

$26$ people took this $8$-question quiz. The mean score was
$3.46$. The score distribution was as follows:

\begin{itemize}
\item Score of $1$: $4$ people.
\item Score of $2$: $3$ people.
\item Score of $3$: $6$ people.
\item Score of $4$: $5$ people.
\item Score of $5$: $6$ people.
\item score of $6$: $2$ people.
\end{itemize}

Here are the answers and performance summary by question. A (*) in
front of a question indicates that there was a single incorrect option
that was chosen by more people than the correct option for that
question:

\begin{enumerate}
\item Option (A): $14$ people.
\item (*) Option (E): $7$ people. {\em Master this!}
\item (*) Option (D): $8$ people. {\em Master this!}
\item Option (E): $14$ people.
\item Option (C): $14$ people.
\item Option (B): $16$ people. {\em Master this!}
\item Option (D): $10$ people.
\item Option (A): $7$ people.
\end{enumerate}

\section{Solutions}

\begin{enumerate}

\item Consider the function $f(x) := \sum_{n=1}^\infty
  \frac{x^n}{n(n + 2)}$ defined on the closed interval
  $[-1,1]$. What are the values of $f(1)$ and $f(-1)$?

  \begin{enumerate}[(A)]
  \item $f(1) = 3/4$ and $f(-1) = 1/4$
  \item $f(1) = 3/4$ and $f(-1) = -1/4$
  \item $f(1) = 3/4$ and $f(-1) = -3/4$
  \item $f(1) = 1/4$ and $f(-1) = 3/4$
  \item $f(1) = 1/4$ and $f(-1) = -1/4$
  \end{enumerate}

  {\em Answer}: Option (B)

  {\em Explanation}: $f(1) =
  \sum_{n=1}^\infty\frac{1}{n(n+2)}$. Rewrite each summand as:

  $$\frac{1}{2}\left[\frac{1}{n} - \frac{1}{n + 2}\right]$$

  The terms cancel (telescoping) and we are left with:

  $$\frac{1}{2}\left(1 + \frac{1}{2}\right)$$

  Simplifying, we get $3/4$.

  Similarly, for $f(-1)$, the summands are:

  $$\frac{(-1)^n}{2}\left[\frac{1}{n} - \frac{1}{n + 2}\right]$$

  Because the term repetition is every two steps, the $n^{th}$ summand
  and $(n+2)^{th}$ summand have the same outer sign, so cancellation
  still proceeds as before. We are left with:

  $$\frac{1}{2}\left[\frac{(-1)^1}{1} + \frac{(-1)^2}{2}\right]$$

  Simplifying, we get $-1/4$.

  {\em ``Alternative'' approach}: Even if you're unable to do these
  summations, you can estimate the sums quickly using the first few
  terms and rule out all the other possibilities. For $f(1)$, we have:

  $$\frac{1}{3} + \frac{1}{8} + \frac{1}{15} + \dots$$

  The sum of the first three terms is $21/40$, which is slightly more
  than $1/2$, so $3/4$ is the only viable option.

  For $f(-1)$, we have:

  $$-\frac{1}{3} + \frac{1}{8} - \frac{1}{15} + \dots$$

  The sum of the first three terms is $-11/30$, which bounds the
  alternating sum from {\em below}. On the other hand, the sum of the
  first two terms, $-5/24$, bounds the alternating sum from {\em
  above}. Among the given options, the only possibility is $-1/4$.

  {\em Performance review}: $16$ out of $26$ got this correct. $5$
  chose (C), $3$ chose (A), and $1$ each chose (D) and (E).

\item Given that we have the following: $\sum_{n=1}^\infty x^n/n =
  -\ln(1 - x)$ for all $-1 < x < 1$ and the series converges
  absolutely in the interval, what is an explicit expression for the
  summation $\sum_{n=1}^\infty x^n/(n(n+1))$ for $x \in (-1,1)
  \setminus \{ 0 \}$?

  \begin{enumerate}[(A)]
  \item $1 + \ln(1 - x)$
  \item $1 - \ln(1 - x)$
  \item $1 + \frac{(1 + x)\ln(1 - x)}{x}$
  \item $1 + \frac{(1 - x)\ln(1 - x)}{x}$
  \item $1 + \frac{(x - 1)\ln(1 - x)}{x}$
  \end{enumerate}

  {\em Answer}: Option (D)

  {\em Explanation}: We telescope and rewrite the summands as:

  $$x^n\left[\frac{1}{n} - \frac{1}{n + 1}\right]$$

  Due to absolute convergence, we can split the summation across the
  $-$ sign and get:

  $$\sum_{n=1}^\infty \frac{x^n}{n} - \sum_{n=1}^\infty \frac{x^n}{n + 1}$$

  The first sum is $-\ln(1 - x)$. We now calculate the second sum. If
  $x \ne 0$, we can multiply and divide by $x$ to obtain:

  $$\sum_{n=1}^\infty \frac{x^n}{n + 1} = \frac{1}{x} \sum_{n=1}^\infty \frac{x^{n +1}}{n + 1}$$

  Choose $m = n + 1$ to rewrite the right side as $\frac{1}{x}
  \sum_{m=2}^\infty \frac{x^m}{m}$. Note that the sum starts from $2$,
  so adding and subtracting the case $m =1$ gives:

  $$\frac{1}{x} \left[\frac{-x^1}{1} + \sum_{m=1}^\infty \frac{x^m}{m}\right]$$

  This becomes:

  $$\frac{1}{x}\left[-\ln(1 - x) - x \right]$$

  which becomes:

  $$\frac{-\ln(1 - x)}{x} - 1$$

  Plugging this back into the original, we get:

  $$-\ln(1 - x) - \left[\frac{-\ln(1 - x)}{x} - 1\right]$$

  This simplifies to $1 + \ln(1 - x)(-1 + (1/x))$ which simplifies to
  option (D).

  {\em Reality check}: There are two reality checks we can perform on
  the expression obtained: taking the limit as $x \to 1^-$, and $x \to
  0$. We first consider the limit as $x$ approaches $1$ from the left.

  In this case, the terms approach $1/(n(n+1))$, which is $(1/n) -
  (1/(n+1))$, which upon telescoping cancellation gives $1$.

  On the other hand, the limit:

  $$\lim_{x \to 1^-} 1 + \frac{(1 - x)\ln(1 - x)}{x}$$

  is also $1$.

  Let's now consider the case that $x$ approaches $0$.

  In this case, the summation approaches $0$ because all its terms
  approach zero.

  For the expression, we have:

  $$\lim_{x \to 0} 1 + \frac{(1 - x)\ln(1 - x)}{x}$$

  Take out the $1 + $ and we get:

  $$1 + \lim_{x \to 0} (1 - x) \lim_{x \to 0} \frac{\ln(1 - x)}{x}$$

  Simple stripping or LH rule gives that the remaining limit is $-1$,
  so the overall answer is $1 + (-1) = 0$, as desired.

  {\em Performance review}: $10$ out of $26$ got this correct. $6$
  chose (C), $4$ chose (A), $4$ chose (E), $2$ chose (B).

\item Given that we have the following: $\sum_{n=1}^\infty x^n/n =
  -\ln(1 - x)$ for all $-1 < x < 1$ and the series converges
  absolutely in the interval, what is an explicit expression for the
  summation $\sum_{n=1}^\infty x^n/(n(n+2))$ for $x \in (-1,1)
  \setminus \{ 0 \}$?

  \begin{enumerate}[(A)]
  \item $\frac{1}{4} + \frac{1}{2x} + \frac{(1 - x^2)\ln(1 - x)}{2x^2}$
  \item $\frac{1}{4} + \frac{1}{2x} + \frac{(x^2 - 1)\ln(1 - x)}{2x^2}$
  \item $\frac{1}{4} - \frac{1}{2x} + \frac{(x^2 - 1)\ln(1 - x)}{2x^2}$
  \item $\frac{1}{4} - \frac{1}{2x} + \frac{(1 - x^2)\ln(1 - x)}{2x^2}$
  \item $\frac{1}{4} + \frac{1}{2x}$
  \end{enumerate}

  {\em Answer}: Option (A)

  {\em Explanation}: This is similar to the previous question -- work
  it out yourself. It is character building.

  {\em Reality check}: We now have three interesting limit cases to
  check: $1^-$, $-1^+$, and $0$. We know from Question 6 that the
  limits for $1^-$ and $-1^+$ should be $3/4$ and $-1/4$
  respectively. Plugging in the values gives the same answer.

  This leaves the case $x = 0$. Here, the limit of the summation
  should be $0$, because all the terms tend to $0$. Let's see if this
  is indeed the case:

  $$\lim_{x \to 0} \left[\frac{1}{4} + \frac{1}{2x} + \frac{(1 - x^2)\ln(1 - x)}{2x^2}\right]$$

  We can take the $1/4$ out and take $2x^2$ as a common denominator on
  the rest:

  $$\frac{1}{4}  + \lim_{x \to 0} \frac{x + (1 - x^2)\ln(1 - x)}{2x^2}$$

  The limit is now a $\to 0/\to 0$ form, so we can use LH rule:

  $$\frac{1}{4} + \lim_{x \to 0} \frac{1 - 2x\ln(1 - x) - (1 - x^2)/(1 - x)}{4x}$$

  Simplifying:

  $$\frac{1}{4} + \lim_{x \to 0} \frac{1 - 2x\ln(1 - x) - (1 + x)}{4x}$$
  
  Simplify further:

  $$\frac{1}{4} + \lim_{x \to 0} \frac{x(-1 - 2\ln(1 - x))}{4x}$$

  Cancel the $x$ and evaluate. We get $-1/4$ for the limit, and adding
  to the outer $+1/4$, we get $0$, as expected.

  {\em Performance review}: $7$ out of $26$ got this correct. $6$ each
  chose (B), (C), (D), and $1$ person left the question blank.
\end{enumerate}

\end{document}
