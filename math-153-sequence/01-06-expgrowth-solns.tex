\documentclass[10pt]{amsart}

%Packages in use
\usepackage{fullpage, hyperref, vipul, enumerate}

%Title details
\title{Class quiz solutions: January 6: Exponential growth}
\author{Math 153, Section 55 (Vipul Naik)}
%List of new commands

\begin{document}
\maketitle

\section{Performance review}

$12$ people took this quiz. The score distribution was as follows:

\begin{itemize}
\item Score of $2$: $1$ person
\item Score of $4$: $2$ persons
\item Score of $5$: $3$ persons
\item Score of $7$: $6$ persons
\end{itemize}

The mean score was $5.58$.

Here is the question-wise performance (full solutions in the next
section):

\begin{enumerate}
\item Option (D): $10$ people
\item Option (B): $10$ people
\item Option (E): $7$ people
\item Option (E): $9$ people
\item Option (C): $12$ people
\item Option (E): $10$ people
\item Option (E): $9$ people
\end{enumerate}

\section{Solutions}

\begin{enumerate}
\item A species of unicellular micro-organisms doubles in number every
  one hour at room temperature and remains constant when placed in a
  refrigerator. Given that the initial number of micro-organisms in a
  dish is $N_0$, and the dish is kept at room temperature for $A$ hours
  and in a refrigerator for $B$ hours, what is the {\bf total number} of
  micro-organisms at {\bf the end}? 

  \begin{enumerate}[(A)]
  \item $N_0 \cdot 2^{A - B}$
  \item $N_0 \cdot 2^{A + B}$
  \item $N_0 \cdot 2^{AB}$
  \item $N_0 \cdot 2^A$
  \item $N_0 \cdot 2^B$
  \end{enumerate}

  {\em Answer}: Option (D)

  {\em Explanation}: The hours in the refrigerator play no role in
  changing the number of micro-organisms, so all that matters is the
  hours spent at room temperature. For every hour spent, the number
  doubles, so the number is $N_0 \cdot 2^A$.

  {\em Performance review}: $10$ out of $12$ got this correct. $1$
  chose (A), $1$ chose (B).

  {\em Historical note (last year)}: All $29$ students got this correct.

  {\em Historical note 2}: A version of this question (without option
  (C), so there were only four options) appeared in last year's 152
  final. $27$ out of $30$ students got it correct.


\item A radioactive substance has a half-life of $3$ years. {\bf
  Determine the integer} $n$ such that $90\%$ of the substance decays
  within somewhere between $n - (1/2)$ and $n + (1/2)$ years.

  \begin{enumerate}[(A)]
  \item $5$
  \item $10$
  \item $15$
  \item $20$
  \item $25$
  \end{enumerate}

  {\em Answer}: Option (B)

  {\em Explanation}: Since $90\%$ of the radioactive substance decays,
  the fraction left is $0.1$. Let $x$ be the time taken for this to
  happen. Then:

  $$\frac{\ln 0.5}{3} = \frac{\ln 0.1}{x}$$

  Simplifying, we obtain that:

  $$x = \frac{3 \ln 10}{\ln 2}$$

  Since $\ln 10 \approx 2.3$ and $\ln 2 \approx 0.69$, we see that $x
  \approx 10$. Thus, the closest integer to $x$ is $n = 10$.

  We can work this out even if we do not remember $\ln 10$ and $\ln
  2$. In order to do this, we need to recognize that:

  $$x = 3 \log_2(10)$$

  Now, $\log_2(10) > 3$ since $2^3 = 8 < 10$, so $3 \log_2(10) >
  9$. Also, $\log_2(10) < 4$ since $2^4 = 16 > 10$, so $3 \log_2(10) <
  12$. Thus, the closest integer should be somewhere betwee $9$ and
  $12$. For further estimation, we verify that $2^{1/3} \approx 1.26$,
  so $2^{10/3} \approx 8 (1.26)$ which is slightly greater than
  $10$. Thus, $\log_2(10)$ is just slightly less than $10/3$, so $3
  \log_2(10)$ is just slightly less than $10$, so $n = 10$ works.

  Another way of seeing this is that $2^{10} = 1024$ (a standard fact
  in computer storage measurements) which is close to $10^3$ (in fact,
  $1KB$ actually means $1024$ bytes, not $1000$ bytes). Thus,
  $2^{10/3} \approx 10$, which is what we need.

  {\em Performance review}: $10$ out of $12$ got this correct. $1$
  chose (A), $1$ chose (C).

  {\em Historical note (last year)}: $23$ out of $29$ people got this
  correct. $4$ people chose (C), $1$ person chose (A), and $1$ person
  left the question blank.
\item $A$, $B$, and $C$ are three species of unicellular
  micro-organisms. Under specified conditions, species $A$ doubles in
  number every $2$ hours, species $B$ triples in number every $3$
  hours, and species $C$ quadruples (i.e., becomes $4$ times) in
  number every $4$ hours. Assume that they start off in the same
  quantities at the beginning. What can we say about their relative
  rates of growth?
  \begin{enumerate}[(A)]
  \item They are all growing at the same rate.
  \item Species $A$ is growing fastest, species $C$ is growing
    slowest, and species $B$ is growing at an intermediate rate.
  \item Species $A$ is growing slowest, species $C$ is growing
    fastest, and species $B$ is growing at an intermediate rate.
  \item Species $A$ and $C$ are both growing at the same rate, which
    is faster than the rate at which species $B$ is growing.
  \item Species $A$ and $C$ are both growing at the same rate, which
    is slower than the rate at which species $B$ is growing.
  \end{enumerate}

  {\em Answer}: Option (E)

  {\em Explanation}: If a population multiplies by a factor of $q$ in
  a time period $t$, the rate of growth is $(\ln q)/t$. We thus need
  to compare $(\ln 2)/2$, $(\ln 3)/3$, and $(\ln 4)/4$.

  First, note that since $\ln 4 = 2\ln 2$, $(\ln 2)/2 = (\ln
  4)/4$. Thus $A$ and $C$ grow at the same rate. This makes sense:
  doubling every $2$ hours is the same as quadrupling every $4$ hours.

  It remains to compare the value against $(\ln 3)/3$. We know that
  $\ln 3$ is between $1.09$ and $1.1$, so $(\ln 3)/3$ is between
  $0.363$ and $0.367$. $\ln 2$ is between $0.69$ and $0.7$, so $(\ln
  2)/2$ is between $0.345$ and $0.35$. Thus, $(\ln 3)/3$ is the bigger number.

  The comparison can be carried out without needing to know the $\ln$
  values. For this, consider the question: what happens after $6$
  hours? Species $A$ has doubled thrice, so it has multiplied by a
  factor of $2^3 = 8$. Species $B$ has tripled twice, so it has
  multiplied by a factor of $3^2 = 9$. Thus, after six hours, species
  $B$ is more numerous than species $A$. Hence, it must have a higher
  growth rate.

  {\em Performance review}: $7$ out of $12$ got this correct. $4$
  chose (D), $1$ chose (C).

  {\em Historical note (last year)}: $22$ out of $29$ people got this
  correct. $3$ people chose (D), $2$ people chose (B), and $1$ person
  each chose (A) and (C).
\item A species of bacteria doubles in number every hour. It takes $9$
  hours for a given initial quantity of this species to fill up a
  petri dish volume. How many hours from the start did the species
  occupy half the petri dish volume (assume that the volume occupied is
  proportional to the quantity)?

  \begin{enumerate}[(A)]
  \item $1$ hour from the beginning
  \item $3$ hours from the beginning
  \item $4.5$ hours from the beginning
  \item $6$ hours from the beginning
  \item $8$ hours from the beginning
  \end{enumerate}

  {\em Answer}: Option (E)

  {\em Explanation}: Since the species doubles in number every hour,
  it must have doubled in the last hour. Thus, at the beginning of the
  last hour, it must have occupied half the petri dish volume. This is
  $8$ hours from the beginning, since $9 - 1 = 8$.

  {\em Performance review}: $9$ out of $12$ got this correct. $2$
  chose (B), $1$ chose (C).

  {\em Historical note (last year)}: $28$ out of $29$ people got this
  correct. $1$ person chose (C).

  {\em Note}: This is a standard trick/conceptual question. Variants
  of this are used to test people's ``intuitive'' understanding of
  exponential growth.

\item Suppose the populations in two countries $A$ and $B$ are growing
  exponentially at possibly different rates. Which of the following
  statements is {\bf false}?

  \begin{enumerate}[(A)]
  \item If the initial population of $A$ is more, and the exponential
    population growth rate of $A$ is greater, then the population of
    $A$ will always be greater than that of $B$.
  \item If the initial population of $A$ is more, and the exponential
    population growth rate of $B$ is greater, then the population of
    $B$ will eventually overtake the population of $A$.
  \item If the initial population of $A$ is more, and the exponential
    population growth rates of $A$ and $B$ are equal, then the
    populations of $A$ and $B$ will eventually become equal.
  \item All of the above.
  \item None of the above.
  \end{enumerate}

  {\em Answer}: Option (C)

  {\em Explanation}: With equal growth rates, the populations do not
  become equal -- rather, the original proportion of the populations
  remains preserved forever. For instance, if the original population
  of $A$ was $11$ times the original population of $B$, then at any
  later instant of time, it will continue to be $11$ times.

  {\em Performance review}: Everybody got this correct!

  {\em Historical note (last year)}: $24$ out of $29$ people got this
  correct. $3$ people chose (B), $1$ person chose (A), and $1$ person
  left the question blank.

\item It takes time $T$ for $1/5$ of a radioactive substance to
  decay. How much time does it take for $2/5$ of the same radioactive
  substance to decay? 

  \begin{enumerate}[(A)]
  \item Precisely $T/2$
  \item Between $T/2$ and $T$
  \item Between $T$ and $2T$
  \item Precisely $2T$
  \item Between $2T$ and $3T$
  \end{enumerate}

  {\em Answer}: Option (E)

  {\em Explanation}: In time $T$, the quantity gets multiplied by $4/5
  = 0.8$. In time $2T$, the quantity gets multiplied by $0.64$. In
  time $3T$, the quantity gets multiplied by $(0.8)^3 = 0.512$. Thus, it
  takes a between $2T$ and $3T$ to multiply by $0.6$, which means that
  $2/5$ has decayed.

  {\em Performance review}: $10$ out of $12$ got this correct. $1$
  chose (B) and $1$ chose (D).

  {\em Historical note (last year)}: $7$ out of $28$ people got this
  correct. $10$ people chose (C), $7$ people chose (B), and $4$ people
  chose (D).

\item The population in the island of Andrognesia as a function of
  time is believed to be an exponential function. On January 1, 1984,
  the population was measured to be $3 * 10^5$ with a measurement
  error of up to $10^5$ on either side, i.e., the population was
  measured to be between $2* 10^5$ and $4 * 10^5$. On January 1,
  1998, the population was measured to be $1.2 * 10^6$ with a
  measurememt error of up to $4 * 10^5$ on either side, i.e., the
  population was measured to be between $8 * 10^5$ and $1.6 *
  10^6$. If the population is an exponential function of time (i.e.,
  the increment in population per year is a fixed proportion of the
  population that year), what is the {\bf range of possible values} of
  the population measured on January 1, 2012?

  \begin{enumerate}[(A)]
  \item Between $3.2 * 10^6$ and $6.4 * 10^6$
  \item Between $3.2 * 10^6$ and $1.28 * 10^7$
  \item Between $1.6 * 10^6$ and $3.2 * 10^6$
  \item Between $1.6 * 10^6$ and $6.4 * 10^6$
  \item Between $1.6 * 10^6$ and $1.28 * 10^7$
  \end{enumerate}

  {\em Answer}: Option (E)

  {\em Explanation}: Note first that $2012 - 1998 = 1998 - 1984 = 14$.

  The key idea is that the lowest estimate occurs
  if the 1998 population was measured as low as possible {\em and} the
  rate of population growth estimated using the 1984 and 1998
  populations is as low as possible. The lowest possible rate of
  growth we can measure occurs if we choose the highest possible 1984
  value and the lowest possible 1998 value. Picking these, we obtain
  that the population estimate for 1984 is $4 * 10^5$ and the
  population estimate for 1998 is $8 * 10^5$. Since the
  multiplicative growth of the population depends on the time elapsed,
  the total population in 2012 will be the solution $x$ to:

  $$\frac{x}{8 * 10^5} = \frac{8 * 10^5}{4 * 10^5}$$

  which solves to $x = 1.6 * 10^6$.

  Similarly, the highest estimate will occur if we take the highest
  estimate possible for the 1998 population and the lowest estimate
  possibly for the 1984 population.

  {\em Performance review}: $9$ out of $12$ got this correct. $2$
  chose (C) and $1$ chose (D).

  {\em Historical note (last year)}: $4$ out of $29$ students got this
  correct. It is not clear by looking at the explanations given
  whether these people got it correct for the correct reasons. $22$
  people chose (A), which is the correct choice if you match the low
  estimates with each other and the high estimates with each
  other. $2$ people chose (B) (possibly trying to get (A), but made a
  calculation error?) and $1$ person left the question blank.

  {\em Action point}: This is a {\em real-life error}! Making such an
  error in real life can be very costly to you (and to those whose
  lives depend upon your decisions), so internalize this
  concept. Please!

  The key concept is that when one measurement depends on two others,
  the worst case scenario happens when the errors in the other two
  measurements are correlated in the {\em worst possible manner}.

  Here's another example of this kind of error. Suppose a test
  measures student knowledge of calculus. The test is used at the
  beginning of a calculus course and then again at the end of Calculus
  class. The test has a margin of error $\delta$ in measurement of
  current student knowledge. If we're using the test to measure {\em
  gain in student knowledge} by doing final knowledge minus initial
  knowledge, the maximum possible error for the gain measure is
  $2\delta$ -- the worst case occurs when the initial and final
  measurement errors are in opposite directions.

  In the real world, because we are working over probability
  distributions, the margins of error do not double; rather, under
  suitable nice assumption, the expected margin of error multiplies by
  a factor of $\sqrt{2}$. But it certainly {\em does increase}.

\end{enumerate}

\end{document}
