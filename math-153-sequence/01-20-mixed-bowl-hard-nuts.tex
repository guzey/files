\documentclass[10pt]{amsart}

%Packages in use
\usepackage{fullpage, hyperref, vipul, enumerate}

%Title details
\title{Take-home class quiz: due January 20: Mixed bowl of hard nuts}
\author{Math 153, Section 55 (Vipul Naik)}
%List of new commands

\begin{document}
\maketitle

Your name (print clearly in capital letters): $\underline{\qquad\qquad\qquad\qquad\qquad\qquad\qquad\qquad\qquad\qquad}$

{\bf YOU ARE FREE TO DISCUSS ALL QUESTIONS, BUT PLEASE MAKE SURE TO
ONLY ENTER ANSWER CHOICES THAT YOU PERSONALLY ENDORSE}

The questions in this quiz include material ostensibly covered in Math
152, and are an attempt to revive your understanding of these
topics. Most questions are repeats, with modifications, of questions
froms 152 quizzes.

\begin{enumerate}
\item Which of the following statements is {\bf always true}?

  \begin{enumerate}[(A)]
  \item The range of a continuous nonconstant function on an open
    bounded interval (i.e., an interval of the form $(a,b)$) is an
    open bounded interval (i.e., an interval of the form $(m,M)$).
  \item The range of a continuous nonconstant function on a closed
    bounded interval (i.e., an interval of the form $[a,b]$) is a
    closed bounded interval (i.e., an interval of the form $[m,M]$).
  \item The range of a continuous nonconstant function on an open
    interval that may be bounded or unbounded (i.e., an interval of
    the form $(a,b)$,$(a,\infty)$, $(-\infty,a)$, or
    $(-\infty,\infty)$), is also an open interval that may be bounded
    or unbounded.
  \item The range of a continuous nonconstant function on a closed
    interval that may be bounded or unbounded (i.e., an interval of
    the form $[a,b]$, $[a,\infty)$, $(-\infty,a]$, or
    $(-\infty,\infty)$) is also a closed interval that may be bounded
    or unbounded.
  \item None of the above.
  \end{enumerate}

  \vspace{0.05in}
  Your answer: $\underline{\qquad\qquad\qquad\qquad\qquad\qquad\qquad}$
  \vspace{0.05in}

\item For which of the following specifications is there {\bf no
  continuous function} satisfying the specifications?

  \begin{enumerate}[(A)]
  \item Domain $(0,1)$ and range $(0,1)$
  \item Domain $[0,1]$ and range $(0,1)$
  \item Domain $(0,1)$ and range $[0,1]$
  \item Domain $[0,1]$ and range $[0,1]$
  \item None of the above, i.e., we can get a continuous function for
    each of the specifications.
  \end{enumerate}

  \vspace{0.05in}
  Your answer: $\underline{\qquad\qquad\qquad\qquad\qquad\qquad\qquad}$
  \vspace{0.05in}

\item Suppose $f$ is a continuously differentiable function from the
  open interval $(0,1)$ to $\R$. Suppose, further, that there are
  exactly $14$ values of $c$ in $(0,1)$ for which $f(c) = 0$. What can
  we say is {\bf definitely true} about the number of values of $c$ in
  the open interval $(0,1)$ for which $f'(c) = 0$? 

  \begin{enumerate}[(A)]
  \item It is at least $13$ and at most $15$.
  \item It is at least $13$, but we cannot put any upper bound on it
    based on the given information.
  \item It is at most $15$, but we cannot put any lower bound (other
  than the meaningless bound of $0$) based on the given information.
  \item It is at most $13$.
  \item It is at least $15$.
  \end{enumerate}

  \vspace{0.05in}
  Your answer: $\underline{\qquad\qquad\qquad\qquad\qquad\qquad\qquad}$
  \vspace{0.05in}
  
\item Consider the function $f(x) := \lbrace\begin{array}{rl} x, & 0
  \le x \le 1/2 \\ x - (1/7), & 1/2 < x \le 1 \\\end{array}$. Define by
  $f^{[n]}$ the function obtained by iterating $f$ $n$ times, i.e.,
  the function $f \circ f \circ f \circ \dots \circ f$ where $f$
  occurs $n$ times. What is the smallest $n$ for which $f^{[n]} =
  f^{[n + 1]}$? 

  \begin{enumerate}[(A)]
  \item $1$
  \item $2$
  \item $3$
  \item $4$
  \item $5$
  \end{enumerate}

  
  \vspace{0.05in}
  Your answer: $\underline{\qquad\qquad\qquad\qquad\qquad\qquad\qquad}$
  \vspace{0.05in}

\item Suppose $f$ and $g$ are functions $(0,1)$ to $(0,1)$ that are
  both right continuous on $(0,1)$. Which of the following is {\em
  not} guaranteed to be right continuous on $(0,1)$? 

  \begin{enumerate}[(A)]
  \item $f + g$, i.e., the function $x \mapsto f(x) + g(x)$
  \item $f - g$, i.e., the function $x \mapsto f(x) - g(x)$
  \item $f \cdot g$, i.e., the function $x \mapsto f(x)g(x)$
  \item $f \circ g$, i.e., the function $x \mapsto f(g(x))$
  \item None of the above, i.e., they are all guaranteed to be right
    continuous functions
  \end{enumerate}

  \vspace{0.05in}
  Your answer: $\underline{\qquad\qquad\qquad\qquad\qquad\qquad\qquad}$
  \vspace{0.05in}

\item Suppose $f$ and $g$ are increasing functions from $\R$ to
  $\R$. Which of the following functions is {\em not} guaranteed to be
  an increasing function from $\R$ to $\R$? 

  \begin{enumerate}[(A)]

  \item $f + g$
  \item $f \cdot g$
  \item $f \circ g$
  \item All of the above, i.e., none of them is guaranteed to be increasing.
  \item None of the above, i.e., they are all guaranteed to be increasing.
  \end{enumerate}

  \vspace{0.05in}
  Your answer: $\underline{\qquad\qquad\qquad\qquad\qquad\qquad\qquad}$
  \vspace{0.05in}

\item Suppose $F$ and $G$ are two functions defined on $\R$ and $k$ is
  a natural number such that the $k^{th}$ derivatives of $F$ and $G$
  exist and are equal on all of $\R$. Then, $F - G$ must be a
  polynomial function. What is the {\bf maximum possible degree} of $F
  - G$?  (Note: Assume constant polynomials to have degree zero) 

  \begin{enumerate}[(A)]
  \item $k - 2$
  \item $k - 1$
  \item $k$
  \item $k + 1$
  \item There is no bound in terms of $k$.
  \end{enumerate}

  \vspace{0.05in}
  Your answer: $\underline{\qquad\qquad\qquad\qquad\qquad\qquad\qquad}$
  \vspace{0.05in}

\item Suppose $f$ is a continuous function on $\R$. Clearly, $f$ has
  antiderivatives on $\R$. For all but one of the following
  conditions, it is possible to guarantee, without any further
  information about $f$, that there exists an antiderivative $F$
  satisfying that condition. {\bf Identify the exceptional condition}
  (i.e., the condition that it may not always be possible to
  satisfy). 

  \begin{enumerate}[(A)]
  \item $F(1) = F(0)$.
  \item $F(1) + F(0) = 0$.
  \item $F(1) + F(0) = 1$.
  \item $F(1) = 2F(0)$.
  \item $F(1)F(0) = 0$.
  \end{enumerate}

  \vspace{0.05in}
  Your answer: $\underline{\qquad\qquad\qquad\qquad\qquad\qquad\qquad}$
  \vspace{0.05in}

\item Suppose $F$ is a function defined on $\R \setminus \{ 0 \}$ such
  that $F'(x) = -1/x^2$ for all $x \in \R \setminus \{ 0 \}$. Which of
  the following pieces of information is/are {\bf sufficient} to determine
  $F$ completely? 

  \begin{enumerate}[(A)]
  \item The value of $F$ at any two positive numbers.
  \item The value of $F$ at any two negative numbers.
  \item The value of $F$ at a positive number and a negative number.
  \item Any of the above pieces of information is sufficient, i.e., we
    need to know the value of $F$ at any two numbers.
  \item None of the above pieces of information is sufficient.
  \end{enumerate}

  \vspace{0.05in}
  Your answer: $\underline{\qquad\qquad\qquad\qquad\qquad\qquad\qquad}$
  \vspace{0.05in}

\item Suppose $F$ and $G$ are continuously differentiable functions on
  all of $\R$ (i.e., both $F'$ and $G'$ are continuous). Which of the
  following is {\bf not necessarily true}? 

  \begin{enumerate}[(A)]
  \item If $F'(x) = G'(x)$ for all integers $x$, then $F - G$ is a
    constant function when restricted to integers, i.e., it takes the
    same value at all integers.
  \item If $F'(x) = G'(x)$ for all numbers $x$ that are not integers,
    then $F - G$ is a constant function when restricted to the set of
    numbers $x$ that are not integers.
  \item If $F'(x) = G'(x)$ for all rational numbers $x$, then $F - G$
    is a constant function when restricted to the set of rational
    numbers.
  \item If $F'(x) = G'(x)$ for all irrational numbers $x$, then $F -
    G$ is a constant function when restricted to the set of irrational
    numbers.
  \item None of the above, i.e., they are all necessarily true.
  \end{enumerate}
  
  \vspace{0.05in}
  Your answer: $\underline{\qquad\qquad\qquad\qquad\qquad\qquad\qquad}$
  \vspace{0.05in}
  
\item Consider the four functions $\sin(\sin x)$, $\sin(\cos x)$,
  $\cos(\sin x)$, and $\cos(\cos x)$. Which of the following
  statements are true about their periodicity? 

  \begin{enumerate}[(A)]
  \item All four functions are periodic with a period of $\pi$.
  \item All four functions are periodic with a period of $2\pi$.
  \item $\cos(\sin x)$ and $\cos(\cos x)$ have a period of $\pi$,
    whereas $\sin(\sin x)$ and $\sin(\cos x)$ have a period of $2\pi$.
  \item $\sin(\sin x)$ and $\sin(\cos x)$ have a period of $\pi$,
    whereas $\cos(\sin x)$ and $\cos(\cos x)$ have a period of $2\pi$.
  \item $\sin(\sin x)$ has a period of $2\pi$, the other three
    functions have a period of $\pi$.
  \end{enumerate}
  
  \vspace{0.05in}
  Your answer: $\underline{\qquad\qquad\qquad\qquad\qquad\qquad\qquad}$
  \vspace{0.05in}

\item Suppose $f$ is a one-to-one function with domain a closed
  interval $[a,b]$ and range a closed interval $[c,d]$. Suppose $t$ is
  a point in $(a,b)$ such that $f$ has left hand derivative $l$ and
  right-hand derivative $r$ at $t$, with both $l$ and $r$
  nonzero. What is the left hand derivative and right hand derivative
  to $f^{-1}$ at $f(t)$? {\em Earlier score: $6/15$}

  \begin{enumerate}[(A)]
  \item The left hand derivative is $1/l$ and the right hand
    derivative is $1/r$.
  \item The left hand derivative is $-1/l$ and the right hand
    derivative is $-1/r$.
  \item The left hand derivative is $1/r$ and the right hand
    derivative is $1/l$.
  \item The left hand derivative is $-1/r$ and the right hand
    derivative is $-1/l$.
  \item The left hand derivative is $1/l$ and the right hand
    derivative is $1/r$ if $l > 0$, otherwise the left hand derivative
    is $1/r$ and the right hand derivative is $1/l$.
  \end{enumerate}

  \vspace{0.05in}
  Your answer: $\underline{\qquad\qquad\qquad\qquad\qquad\qquad\qquad}$
  \vspace{0.05in}

\item Which of these functions is one-to-one? 

  \begin{enumerate}[(A)]
  \item $f_1(x) := \lbrace \begin{array}{rl} x, & x \text{ rational} \\ x^2, & x \text{ irrational}\\\end{array}$ 
  \item $f_2(x) := \lbrace \begin{array}{rl} x, & x \text{ rational} \\ x^3, & x \text{ irrational}\\\end{array}$
  \item $f_3(x) := \lbrace\begin{array}{rl} x, & x \text{ rational} \\ 1/(x - 1), & x \text{ irrational}\\\end{array}$
  \item All of the above
  \item None of the above
  \end{enumerate}

  \vspace{0.05in}
  Your answer: $\underline{\qquad\qquad\qquad\qquad\qquad\qquad\qquad}$
  \vspace{0.05in}

\item Consider the following function $f:[0,1] \to [0,1]$ given by
  $f(x) := \lbrace\begin{array}{rl} \sin(\pi x/2), & 0 \le x \le 1/2 \\
  \sqrt{x}, & 1/2 < x \le 1\\\end{array}$. What is the correct
  expression for $(f^{-1})'(1/2)$? 
  \begin{enumerate}[(A)]
  \item It does not exist, since the two-sided derivatives of $f$ at
    $1/2$ do not match.
  \item $\sqrt{2}$
  \item $2\sqrt{2}/\pi$
  \item $4/\pi$
  \item $4/(\sqrt{3}\pi)$
  \end{enumerate}

  \vspace{0.05in}
  Your answer: $\underline{\qquad\qquad\qquad\qquad\qquad\qquad\qquad}$

\end{enumerate}

\end{document}
